\documentclass[12pt]{article}

% Packages
\usepackage[margin=1in]{geometry}
\usepackage{amsmath,amssymb,amsthm}
\usepackage{enumitem}
\usepackage{hyperref}
\usepackage{xcolor}
\usepackage{import}
\usepackage{xifthen}
\usepackage{pdfpages}
\usepackage{transparent}
\usepackage{listings}
\usepackage{tikz}
\usepackage{physics}
\usepackage{siunitx}
\usepackage{booktabs}
\usepackage{cancel}
  \usetikzlibrary{calc,patterns,arrows.meta,decorations.markings}


\DeclareMathOperator{\Log}{Log}
\DeclareMathOperator{\Arg}{Arg}

\lstset{
    breaklines=true,         % Enable line wrapping
    breakatwhitespace=false, % Wrap lines even if there's no whitespace
    basicstyle=\ttfamily,    % Use monospaced font
    frame=single,            % Add a frame around the code
    columns=fullflexible,    % Better handling of variable-width fonts
}

\newcommand{\incfig}[1]{%
    \def\svgwidth{\columnwidth}
    \import{./Figures/}{#1.pdf_tex}
}
\theoremstyle{definition} % This style uses normal (non-italicized) text
\newtheorem{solution}{Solution}
\newtheorem{proposition}{Proposition}
\newtheorem{problem}{Problem}
\newtheorem{lemma}{Lemma}
\newtheorem{theorem}{Theorem}
\newtheorem{remark}{Remark}
\newtheorem{note}{Note}
\newtheorem{definition}{Definition}
\newtheorem{example}{Example}
\newtheorem{corollary}{Corollary}
\theoremstyle{plain} % Restore the default style for other theorem environments
%

% Theorem-like environments
% Title information
\title{}
\author{Jerich Lee}
\date{\today}

\begin{document}

\maketitle
%---------------------------------------------------------------------------
%  Semicontinuity — definitions and concrete examples
%---------------------------------------------------------------------------

\begin{definition}[Lower/Upper Semicontinuous Functions]\label{def:lsc-usc}
  Let $(X,\tau)$ be a topological space and $f\colon X\to[-\infty,\infty]$.
  \begin{enumerate}
      \item $f$ is \emph{lower semicontinuous} (l.s.c.) if 
            \[
                \{x\in X : f(x)>\alpha\}\in\tau \quad
                \text{for every }\alpha\in\mathbb{R}.
            \]
      \item $f$ is \emph{upper semicontinuous} (u.s.c.) if 
            \[
                \{x\in X : f(x)<\alpha\}\in\tau \quad
                \text{for every }\alpha\in\mathbb{R}.
            \]
  \end{enumerate}
  A function is continuous iff it is both l.s.c.\ and u.s.c.
  \end{definition}
  
  \medskip
  \noindent\textbf{Concrete examples}
  
  \begin{enumerate}[]
  %-------------------------------------------------------------
  \item \textbf{Characteristic functions.}
        Let $X=\mathbb R$ with the usual topology.
        \begin{enumerate}[]
            \item $\displaystyle \chi_{(0,1)}(x)=
                     \begin{cases}
                         1 & 0<x<1,\\[2pt]
                         0 & \text{otherwise}
                     \end{cases}$
                  is \emph{lower} semicontinuous:
                  for $\alpha\in(0,1)$ the set $\{x:\chi_{(0,1)}(x)>\alpha\}=(0,1)$
                  is open.
            \item $\displaystyle \chi_{[0,1]}(x)=
                     \begin{cases}
                         1 & 0\le x\le1,\\[2pt]
                         0 & \text{otherwise}
                     \end{cases}$
                  is \emph{upper} semicontinuous:
                  for $\alpha\in(0,1)$ the set $\{x:\chi_{[0,1]}(x)<\alpha\}=\mathbb{R}\setminus[0,1]$
                  is open.
        \end{enumerate}
  
  %-------------------------------------------------------------
  \item \textbf{A non-trivial l.s.c.\ function.}
        Define on $\mathbb R$
        \[
            f(x)=
            \begin{cases}
                0, & x<0,\\
                x, & x\ge 0.
            \end{cases}
        \]
        Each super-level set $\{x:f(x)>\alpha\}$ is either $(\alpha,\infty)$
        (if $\alpha\ge0$) or $(-\infty,\infty)$ (if $\alpha<0$),
        hence open.  
        $f$ is l.s.c.\ but not u.s.c.\ at $x=0$.
  
  %-------------------------------------------------------------
  \item \textbf{Supremum of l.s.c.\ functions is l.s.c.}
        For $n\in\mathbb{N}$ let 
        \[
            g_n(x)=\max\!\bigl(0,\,x-1/n\bigr),\qquad x\in\mathbb R.
        \]
        Each $g_n$ is continuous $\Rightarrow$ l.s.c.  
        Their pointwise supremum is
        \[
            g(x)=\sup_{n\ge1} g_n(x)=\max(0,x),
        \]
        which is exactly the l.s.c.\ function in (2).  
  
  %-------------------------------------------------------------
  \item \textbf{Infimum of u.s.c.\ functions is u.s.c.}
        Let 
        \[
            h_n(x)=\min\!\bigl(1,\,x+1/n\bigr),\qquad x\in\mathbb R.
        \]
        Each $h_n$ is continuous $\Rightarrow$ u.s.c.  
        Their pointwise infimum
        \[
            h(x)=\inf_{n\ge1} h_n(x)=\min(1,x)
        \]
        is u.s.c.\ but fails to be l.s.c.\ at $x=1$.
  \end{enumerate}
  
  \medskip
  \noindent\textbf{Summary of properties (cf.~Definition~\ref{def:lsc-usc})}
  \begin{itemize}
      \item Characteristic functions of \emph{open} sets are l.s.c.;  
            of \emph{closed} sets are u.s.c.
      \item $\sup$ of any family of l.s.c.\ functions is l.s.c.  
            $\inf$ of any family of u.s.c.\ functions is u.s.c.
  \end{itemize}
  \pagebreak
  \paragraph{Why continuity $\;\Longleftrightarrow\;$ (u.s.c.\ \&\ l.s.c.)}

\begin{enumerate}
%-----------------------------------------------------------------
\item \textbf{Continuity $\;\Longrightarrow$ upper \& lower semicontinuity.}

      If $f\colon X\to\mathbb{R}$ is continuous, the inverse image of
      \emph{every} open set is open.  
      In particular, for any $\alpha\in\mathbb{R}$,
      \[
          \{x:f(x)>\alpha\}=f^{-1}((\alpha,\infty)), 
          \qquad
          \{x:f(x)<\alpha\}=f^{-1}((-\infty,\alpha))
      \]
      are open; hence $f$ is simultaneously
      lower semicontinuous (l.s.c.) and upper semicontinuous (u.s.c.).

%-----------------------------------------------------------------
\item \textbf{Upper \& lower semicontinuity $\;\Longrightarrow$ continuity.}

      Assume $f$ is both u.s.c.\ and l.s.c.
      The basic open sets of $\mathbb{R}$ are open intervals $(\alpha,\beta)$.
      Observe that
      \[
          f^{-1}\bigl((\alpha,\beta)\bigr)
          \;=\;
          \underbrace{\{x:f(x)>\alpha\}}_{\text{open by l.s.c.}}
          \;\cap\;
          \underbrace{\{x:f(x)<\beta\}}_{\text{open by u.s.c.}},
      \]
      which is an intersection of two open sets, hence open.
      Therefore the inverse image of \emph{every} basic open set is open,
      and by standard topology $f$ is continuous.
\end{enumerate}

\noindent
Consequently \(f\) is continuous \emph{iff} it is both upper and lower
semicontinuous.
\pagebreak
%------------------------------------------------------------------
%  Urysohn’s Lemma (locally compact Hausdorff version)
%  \;– detailed step–by–step proof in Rudin’s style
%------------------------------------------------------------------

\begin{theorem}[Urysohn]\label{thm:urysohn}
  Let $X$ be a locally compact Hausdorff space.
  If $K\subset V\subset X$ with $K$ \emph{compact} and $V$ \emph{open},
  then there exists a continuous function $f\in C_c(X)$ such~that
  \[
            K\;<\;f\;<\;V,
  \qquad\text{i.e.}\qquad
        \begin{cases}
            0\le f\le1,\\[2pt]
            f=1 \text{ on }K,\\[2pt]
            \operatorname{supp} f \subset V.
        \end{cases}
  \]
  \end{theorem}
  
  \begin{proof}[Step‐by‐step construction]
  
  \textbf{1.\;Choose a dense sequence of rationals.}
  Let $\{r_k\}_{k\ge1}$ be an enumeration of the rationals in $(0,1)$ and set
  $r_1:=0,\; r_2:=1$.
  Thus we have
  \[
     0=r_1<r_2<r_3<\dots<1.
  \]
  
  \medskip
  \textbf{2.\;First two open sets $V_0\subset V_1$.}
  Because $X$ is locally compact Hausdorff,
  theorem~2.7 (Rudin) gives \emph{nested} open sets
  \[
       K\subset V_1\subset \overline{V_1}\subset V_0\subset\overline{V_0}\subset V,
  \]
  with $\overline{V_0}$ compact.   \hfill(Equation~(2))
  
  \medskip
  \textbf{3.\;Inductive construction of open sets $V_k$ ($k\ge2$).}
  Assume $V_1,\dots,V_n$ have been chosen so that
  $r_i<r_j\;\Rightarrow\;\overline{V_{r_j}}\subset V_{r_i}$.
  Pick $r_i$ \emph{largest} below $r_{n+1}$ and $r_j$ \emph{smallest} above $r_{n+1}$.
  Local compactness again yields an open set
  \[
      \overline{V_{r_j}}\subset V_{n+1}\subset\overline{V_{n+1}}\subset V_{r_i}.
  \]
  Iterating produces a collection $\{V_r\}_{r\in\mathbb{Q}\cap[0,1]}$
  satisfying
  \[
     s>r\quad\Longrightarrow\quad \overline{V_s}\subset V_r.
     \tag{3}
  \]
  
  \medskip
  \textbf{4.\;Auxiliary semicontinuous functions.}
  For each rational $r\in[0,1]$ define \hfill(Equation (4))
  \[
     f_r(x)=
     \begin{cases}
         r,& x\in V_r,\\
         0,& x\notin V_r,
     \end{cases}
     \qquad
     g_r(x)=
     \begin{cases}
         1,& x\in\overline{V_r},\\
         r,& x\notin \overline{V_r}.
     \end{cases}
  \]
  \emph{Remarks after Definition 2.8:}
  $V_r$ open $\;\Rightarrow\;$ each $f_r$ is \underline{lower} semicontinuous;
  $\overline{V_r}$ closed $\;\Rightarrow\;$ each $g_r$ is \underline{upper}
  semicontinuous.
  
  \medskip
  \textbf{5.\;Take pointwise extremum.}\hfill(Equation (5))
  \[
       f(x)=\sup_{r} f_r(x),
       \qquad
       g(x)=\inf_{s} g_s(x).
  \]
  Hence $f$ is lower and $g$ is upper semicontinuous.  
  We always have $0\le f\le g\le1$.
  
  \medskip
  \textbf{6.\;Show $f=g$.}
  Fix $x\in X$.
  \begin{itemize}
  \item If $x\in K$, then $x\in V_r$ for \emph{every} $r$ (by $K\subset V_1$),
        so $f_r(x)=r$ and $f(x)=1$.  
        Likewise $x\in\overline{V_r}$, hence $g(x)=1$.
  \item If $x\notin K$, pick rationals $r<s$ close enough that
        $x\notin\overline{V_r}$ (possible by compactness and (3)).
        Then $g_s(x)=s$ while $f_r(x)=0$; property (3) ensures
        $f_r\le g_s$ everywhere, so $f(x)\le r<s\le g(x)$.  
        Taking $r\uparrow s$ forces $f(x)=g(x)$.
  \end{itemize}
  Hence $f\equiv g$ and is \emph{both} l.s.c.\ and u.s.c., therefore
  \underline{continuous}.
  
  \medskip
  \textbf{7.\;Verify the required properties.}
  \begin{itemize}
  \item $0\le f\le1$ by construction.
  \item $f=1$ on $K$ (step 6).
  \item If $x\notin\overline{V_0}$ then $x\notin V_{r}$ for \emph{all} $r$,
        hence $f(x)=0$.
        Thus $\operatorname{supp}f\subset\overline{V_0}\subset V$.
  \end{itemize}
  
  \medskip\noindent
  Therefore $f\in C_c(X)$ and $K<f<V$, completing the proof.
  \end{proof}
  \pagebreak
  %------------------------------------------------------------------
%  Clarifying the choice of the rationals  $\{r_k\}$  in Step 1
%------------------------------------------------------------------

The text you quoted meant only this:

* “pick a sequence of rationals that is \emph{dense} in $(0,1)$,”  
* “label two special values $r_1:=0$ and $r_2:=1$ for later use.”

The chain
\[
0=r_1 < r_2 < r_3 < r_4 < \cdots < 1
\]
is \emph{not} literally correct if $r_2=1$.  
A perfectly consistent (and explicit) way to set up the rationals is:

\[
r_1 := 0,\qquad
r_k := \frac{k-1}{k}\quad(k\ge 2).
\]

\begin{itemize}
\item Then
      \[
          0=r_1 < r_2=\tfrac12 < r_3=\tfrac23 < r_4=\tfrac34 < \dots < 1,
      \]
      so every $r_k$ is strictly between $0$ and $1$ except for $r_1$.
\item The sequence $\{r_k\}_{k\ge 1}$ is \emph{increasing} and
      converges to $1$, hence it is dense in $[0,1]$ and
      certainly dense in $(0,1)$.
\item If you really want the endpoint $1$ to have its own label,
      just declare an auxiliary symbol $r_\infty:=1$—it will never be
      used in the inductive construction, so no ordering conflicts arise.
\end{itemize}

This fixes the apparent “$1<1$’’ paradox while preserving the idea
behind Rudin’s proof: we need a countable, \emph{ordered} list of numbers
getting arbitrarily close to every point of $(0,1)$ so that the nested
open sets \(V_r\) can be chosen one‐by‐one.
\pagebreak
\paragraph{Why do we single out $r_1=0$ and $r_2=1$?}

In Urysohn’s construction we need \emph{at least two} rational numbers:

\begin{enumerate}
    \item  a \textbf{minimum} level, labelled $r_1$, at which the auxiliary functions will always vanish, and
    \item  a \textbf{maximum} level, labelled $r_2$, at which they will eventually attain the value $1$ on the compact set $K$.
\end{enumerate}

Choosing\;\;$r_1=0$ \emph{and}\; $r_2=1$ guarantees that the final function $f$ satisfies
\[
   0\;\le\;f\;\le\;1,
   \qquad
   f=1 \text{ on }K,
   \qquad
   \operatorname{supp}f\subset V,
\]
exactly the three inequalities $K<f<V$ required by Urysohn’s lemma.

After those two anchor points are fixed, \emph{any} countable dense
subset of $(0,1)$ can be appended as $r_3,r_4,\dots$ to drive the
inductive construction of the nested open sets $V_r$.  The extra
rationals are used only to refine the “step heights’’ between $0$ and
$1$; the endpoints $0$ and $1$ themselves are essential to set the lower
and upper bounds of the function.
\pagebreak
%-----------------------------------------------------------------
%  Why   $f_r$   is lower semicontinuous  and   $g_s$   is upper
%  semicontinuous — and why the same holds for their pointwise
%  supremum / infimum
%-----------------------------------------------------------------

Recall the auxiliary functions introduced in Urysohn’s proof:

\[
  f_r(x)=
  \begin{cases}
      r, & x\in V_r,\\[2pt]
      0, & x\notin V_r,
  \end{cases}
  \qquad
  g_s(x)=
  \begin{cases}
      1, & x\in\overline{V_s},\\[2pt]
      s, & x\notin\overline{V_s},
  \end{cases}
  \tag{4}
\]

where each \(V_r\) is an \emph{open} set and \(\overline{V_s}\) is its
\emph{compact} (hence closed) closure.  
We must check semicontinuity in the sense of Definition~2.8.

\paragraph{1.  Each \(f_r\) is \emph{lower} semicontinuous.}

For any real number \(\alpha\) the super‐level set is
\[
  \{x : f_r(x) > \alpha\} =
  \begin{cases}
      V_r,          & \alpha<r,\\[4pt]
      \varnothing,  & \alpha\ge r.
  \end{cases}
\]
Because \(V_r\) is \emph{open} and the empty set is also open, every
super‐level set is open.  
Hence \(f_r\) satisfies the defining property of
\textbf{lower semicontinuity (l.s.c.)}.

\paragraph{2.  Each \(g_s\) is \emph{upper} semicontinuous.}

For any \(\alpha\) the sub‐level set is
\[
  \{x : g_s(x) < \alpha\} =
  \begin{cases}
      X\setminus \overline{V_s}, & \alpha> s,\\[4pt]
      \varnothing,               & \alpha\le s.
  \end{cases}
\]
Since \(\overline{V_s}\) is \emph{closed}, its complement
\(X\setminus\overline{V_s}\) is open; the empty set is open as well.
Thus every sub‐level set is open, and \(g_s\) meets the definition of
\textbf{upper semicontinuity (u.s.c.)}.

\paragraph{3.  Extremal property of semicontinuity.}

A basic topological fact (mentioned in Rudin right after
Definition 2.8) is:

\[
  \sup_{r}(\text{l.s.c.\ functions})\;\text{ is l.s.c.},\qquad
  \inf_{s}(\text{u.s.c.\ functions})\;\text{ is u.s.c.}
\]

Intuitively, taking a pointwise supremum can only introduce new “jumps
\emph{down}’’ (allowed for l.s.c.) but never “jumps up’’
(forbidden for l.s.c.); the dual statement holds for infima of u.s.c.
functions.

Applying this to the definitions
\[
    f=\sup_{r} f_r, 
    \qquad
    g=\inf_{s} g_s 
    \tag{5}
\]
gives:

* \(f\) is lower semicontinuous because each \(f_r\) is,
  and a supremum of l.s.c. functions remains l.s.c.

* \(g\) is upper semicontinuous because each \(g_s\) is,
  and an infimum of u.s.c. functions remains u.s.c.

Hence the semicontinuity assertions in Rudin’s proof are justified.
\pagebreak
\paragraph{Meaning of $\alpha$ in the argument}

In the definitions of semicontinuity we test the behaviour of a function against \emph{all real thresholds}.  
Concretely:

* For \textbf{lower semicontinuity} we require every set
  \[
      \{x\in X : f(x)>\alpha\}
  \]
  to be open, \emph{for each real number $\alpha$}.  
  Here $\alpha$ is simply a placeholder for an arbitrary real constant that we compare $f(x)$ with.

* For \textbf{upper semicontinuity} we require every set
  \[
      \{x\in X : f(x)<\alpha\}
  \]
  to be open, again for \emph{every} real $\alpha$.

So in the step–by–step check we picked an \emph{arbitrary} real
$\alpha\in\mathbb{R}$ and computed the corresponding super-level (or
sub-level) set.  
Showing that this set is open for that arbitrary $\alpha$ proves it is
open for \emph{all} $\alpha$, hence establishes semicontinuity.

In short, $\alpha$ is just the variable threshold that appears in the
definition; no special numerical value is assigned to it.
\pagebreak

\paragraph{Clarifying the role of $\alpha$}

When we test lower semicontinuity we do **not** get to pick
a different threshold $\alpha$ for each individual point~$x$.
Instead, we fix one real number $\alpha\in\mathbb{R}$ \emph{once and
for all}, form the super-level set
\[
    U_\alpha \;:=\;\{x\in X : f_r(x)>\alpha\},
\]
and ask: \emph{is this whole set open?}

For the particular function
\[
    f_r(x)=
    \begin{cases}
         r, & x\in V_r,\\
         0, & x\notin V_r,
    \end{cases}
\]
the answer depends on the \emph{fixed} value of~$\alpha$:

\[
    U_\alpha =
    \begin{cases}
        V_r,          & \alpha<r, \\[6pt]
        \varnothing,  & \alpha\ge r.
    \end{cases}
\]

* If $\alpha<r$, then $f_r(x)>\alpha$ \textbf{only} on $V_r$, so
  $U_\alpha=V_r$, which is open because $V_r$ is open.
* If $\alpha\ge r$, the inequality fails everywhere, so
  $U_\alpha=\varnothing$, and the empty set is open by convention.

Because \emph{for every fixed $\alpha$} the corresponding
$U_\alpha$ is an open set, the definition of lower semicontinuity is
satisfied.

In other words, the definition requires openness of \emph{all} these
threshold sets taken one at a time; it does \emph{not} allow us to
choose $\alpha$ separately for each point $x$.
\pagebreak

%--------------------------------------------------------------------
%  A \textbf{concrete} choice of the auxiliary functions $f_r,\,g_s$
%  in $\mathbb R$ so you can see them explicitly  
%--------------------------------------------------------------------

Fix once and for all two sets
\[
    K=[0,\tfrac14], 
    \qquad 
    V=(-1,1)
\]
so that $K$ is compact and $K\subset V\subset\mathbb R$.
For every rational $r\in(0,1)$ define the open set
\[
    V_r := (-r,r).
\]
Then
\[
   s>r \;\Longrightarrow\; \overline{V_s}=[-s,s]\subset(-r,r)=V_r,
\]
exactly the nesting property required in Rudin’s proof.

---

a1.\;Pick a particular rational and write down \(f_r\)

Take \(r=\tfrac12\).  
With the above rule,
\[
   V_{1/2}=(-\tfrac12,\tfrac12).
\]
Now apply equation (4):
\[
    f_{1/2}(x)=
    \begin{cases}
        \tfrac12, & -\tfrac12 < x < \tfrac12,\\[4pt]
        0,        & \text{otherwise}.
    \end{cases}
\]

*Super-level sets.*  
For any fixed $\alpha\in\mathbb R$,
\[
    \bigl\{x: f_{1/2}(x)>\alpha\bigr\}=
    \begin{cases}
        (-\tfrac12,\tfrac12) &\text{if }\alpha<\tfrac12,\\
        \varnothing          &\text{if }\alpha\ge\tfrac12,
    \end{cases}
\]
and both cases are open sets.  
Hence \(f_{1/2}\) is **lower semicontinuous**.

---

2.\;Pick a particular rational and write down \(g_s\)

Choose \(s=\tfrac34\).  
Then $\overline{V_{3/4}}=[-\tfrac34,\tfrac34]$ and
\[
    g_{3/4}(x)=
    \begin{cases}
        1, & -\tfrac34 \le x \le \tfrac34,\\[4pt]
        \tfrac34, & \text{otherwise}.
    \end{cases}
\]

*Sub-level sets.*  
For any fixed $\alpha\in\mathbb R$,
\[
    \bigl\{x: g_{3/4}(x)<\alpha\bigr\}=
    \begin{cases}
        \mathbb R\setminus[-\tfrac34,\tfrac34] &\text{if }\alpha>\tfrac34,\\
        \varnothing                            &\text{if }\alpha\le\tfrac34,
    \end{cases}
\]
and both cases are open.  
Thus \(g_{3/4}\) is **upper semicontinuous**.

---

3.\;Pointwise extremum gives the final \(f\) and \(g\)

Define
\[
    f(x)=\sup_{r\in\mathbb Q\cap(0,1)} f_r(x),
    \qquad
    g(x)=\inf_{s\in\mathbb Q\cap(0,1)} g_s(x).
\]

*What do they look like?*

* **Inside** $(-\tfrac14,\tfrac14)$ (which contains $K$):  
  for every $r>\tfrac14$ we have $x\in V_r$, so
  $f_r(x)=r$; taking $\sup_r$ yields $f(x)=1$.
  Likewise $x\in[-s,s]$ for all $s>\tfrac14$, so $g_s(x)=1$, hence
  $g(x)=1$.

* **Outside** $(-1,1)$ (the complement of $\overline{V_{1}}$):  
  $x\notin V_r$ for every $r<1$, so $f_r(x)=0$ and $f(x)=0$.
  Also $x\notin[-s,s]$ for every $s<1$, so $g_s(x)=s$ and $g(x)=0$.

Thus
\[
    f(x)=g(x)=
    \begin{cases}
        1, & x\in(-\tfrac14,\tfrac14),\\
        0, & x\notin(-1,1).
    \end{cases}
\]
Between $x=\pm\tfrac14$ and $x=\pm1$ the value tapers off but always
lies in $[0,1]$.

---

4.\;Why semicontinuity survives the extremum

Because each $f_r$ is lower semicontinuous, their pointwise supremum $f$
remains lower semicontinuous; analogously, $g$ is upper semicontinuous.
Since we have just seen \(f=g\), the common function is actually
**continuous**, satisfies $f=1$ on $K$, vanishes outside~$V$, and lives
in $C_c(\mathbb R)$—exactly what Urysohn’s lemma promises.

---

\textbf{Take-away.}  
The tiny “plateau’’ at height $r$ inside $V_r$ and $0$ elsewhere
produces an l.s.c.\ block.  
The “roof’’ at height $1$ over $\overline{V_s}$ and level $s$ elsewhere
produces a u.s.c.\ block.
Piling up (\(\sup\)) all the blocks and shaving down (\(\inf\)) all the
roofs meet precisely at a continuous function separating $K$ from~$V$.
\end{document}
