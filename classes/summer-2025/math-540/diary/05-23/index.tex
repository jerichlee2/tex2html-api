\documentclass[12pt]{article}

% Packages
\usepackage[margin=1in]{geometry}
\usepackage{amsmath,amssymb,amsthm}
\usepackage{enumitem}
\usepackage{hyperref}
\usepackage{xcolor}
\usepackage{import}
\usepackage{xifthen}
\usepackage{pdfpages}
\usepackage{transparent}
\usepackage{listings}
\usepackage{tikz}
\usepackage{physics}
\usepackage{siunitx}
\usepackage{booktabs}
\usepackage{cancel}
  \usetikzlibrary{calc,patterns,arrows.meta,decorations.markings}


\DeclareMathOperator{\Log}{Log}
\DeclareMathOperator{\Arg}{Arg}

\lstset{
    breaklines=true,         % Enable line wrapping
    breakatwhitespace=false, % Wrap lines even if there's no whitespace
    basicstyle=\ttfamily,    % Use monospaced font
    frame=single,            % Add a frame around the code
    columns=fullflexible,    % Better handling of variable-width fonts
}

\newcommand{\incfig}[1]{%
    \def\svgwidth{\columnwidth}
    \import{./Figures/}{#1.pdf_tex}
}
\theoremstyle{definition} % This style uses normal (non-italicized) text
\newtheorem{solution}{Solution}
\newtheorem{proposition}{Proposition}
\newtheorem{problem}{Problem}
\newtheorem{lemma}{Lemma}
\newtheorem{theorem}{Theorem}
\newtheorem{remark}{Remark}
\newtheorem{note}{Note}
\newtheorem{definition}{Definition}
\newtheorem{example}{Example}
\newtheorem{corollary}{Corollary}
\theoremstyle{plain} % Restore the default style for other theorem environments
%

% Theorem-like environments
% Title information
\title{MATH 540: HW 1}
\author{Jerich Lee}
\date{\today}

\begin{document}

\maketitle
\begin{problem}
  Does there exist an infinite $\sigma$-algebra which has only countably many members?
  \end{problem}

  \begin{solution}
    
  \end{solution}
  \begin{problem}
    Prove an analogue of Theorem 1.8 for $n$ functions.
    \end{problem}

    \begin{solution}
      
    \end{solution}
    \begin{problem}
       Prove that if $f$ is a real function on a measurable space $X$ such that $\{x : f(x) \ge r\}$ is measurable for every rational $r$, then $f$ is measurable.
      \end{problem}

      \begin{solution}
        
      \end{solution}

\begin{problem}
  Let $\{a_n\}$ and $\{b_n\}$ be sequences in $[-\infty, \infty]$, and prove the following assertions:
  
  \begin{enumerate}
      \item[(a)] 
      \[
          \limsup_{n \to \infty} (-a_n) = -\liminf_{n \to \infty} a_n.
      \]
  
      \item[(b)] 
      \[
          \limsup_{n \to \infty} (a_n + b_n) \le \limsup_{n \to \infty} a_n + \limsup_{n \to \infty} b_n
      \]
      \textit{provided none of the sums is of the form $\infty - \infty$.}
  
      \item[(c)] \textit{If $a_n \le b_n$ for all $n$, then}
      \[
          \liminf_{n \to \infty} a_n \le \liminf_{n \to \infty} b_n.
      \]
  \end{enumerate}
  
  Show by an example that strict inequality can hold in (b).
  \end{problem}
\begin{solution}
  
\end{solution}
\begin{problem}
 \noindent
 \begin{enumerate}
  \item Suppose $f \colon X \to [-\infty, \infty]$ and $g \colon X \to [-\infty, \infty]$ are measurable. Prove that the sets
  \[
  \{x : f(x) < g(x)\}, \quad \{x : f(x) = g(x)\}
  \]
  are measurable.
  \item Prove that the set of points at which a sequence of measurable real-valued functions converges (to a finite limit) is measurable.
 \end{enumerate}
  
  \end{problem}
\begin{solution}
  
\end{solution}
\begin{problem}
  Let $X$ be an uncountable set, let $\mathcal{M}$ be the collection of all sets $E \subset X$ such that either $E$ or $E^c$ is at most countable, and define $\mu(E) = 0$ in the first case, $\mu(E) = 1$ in the second. Prove that $\mathcal{M}$ is a $\sigma$-algebra in $X$ and that $\mu$ is a measure on $\mathcal{M}$. Describe the corresponding measurable functions and their integrals.
  \end{problem}
\begin{solution}
  
\end{solution}
\begin{problem}
  Suppose $f_n \colon X \to [0, \infty]$ is measurable for $n = 1, 2, 3, \dots$, $f_1 \ge f_2 \ge f_3 \ge \cdots \ge 0$, $f_n(x) \to f(x)$ as $n \to \infty$, for every $x \in X$, and $f_1 \in L^1(\mu)$. Prove that then
  \[
  \lim_{n \to \infty} \int_X f_n\, d\mu = \int_X f\, d\mu
  \]
  and show that this conclusion does \emph{not} follow if the condition ``$f_1 \in L^1(\mu)$'' is omitted.
  \end{problem}
\begin{solution}
  
\end{solution}
\begin{problem}
  Put $f_n = \chi_E$ if $n$ is odd, $f_n = 1 - \chi_E$ if $n$ is even. What is the relevance of this example to Fatou's lemma?
  \end{problem}
\begin{solution}
  
\end{solution}
\begin{problem}
  Suppose $\mu$ is a positive measure on $X$, $f \colon X \to [0, \infty]$ is measurable, $\int_X f\, d\mu = c$, where $0 < c < \infty$, and $\alpha$ is a constant. Prove that
  \[
  \lim_{n \to \infty} \int_X n \log\left[1 + \left(\frac{f}{n}\right)^\alpha\right]\, d\mu =
  \begin{cases}
  \infty & \text{if } 0 < \alpha < 1, \\
  c      & \text{if } \alpha = 1, \\
  0      & \text{if } 1 < \alpha < \infty.
  \end{cases}
  \]
  
  \textit{Hint}: If $\alpha \ge 1$, the integrands are dominated by $\alpha f$. If $\alpha < 1$, Fatou’s lemma can be applied.
  \end{problem}
\begin{solution}
  
\end{solution}
\begin{problem}
  Suppose $\mu(X) < \infty$, $\{f_n\}$ is a sequence of bounded complex measurable functions on $X$, and $f_n \to f$ uniformly on $X$. Prove that
  \[
  \lim_{n \to \infty} \int_X f_n\, d\mu = \int_X f\, d\mu,
  \]
  and show that the hypothesis “$\mu(X) < \infty$” cannot be omitted.
  \end{problem}
\begin{solution}
  
\end{solution}
\begin{problem}
  Show that
  \[
  A = \bigcap_{n=1}^{\infty} \bigcup_{k=n}^{\infty} E_k
  \]
  in Theorem 1.41, and hence prove the theorem without any reference to integration.
  \end{problem}
\begin{solution}
  
\end{solution}
\begin{problem}
  Suppose $f \in L^1(\mu)$. Prove that to each $\epsilon > 0$ there exists a $\delta > 0$ such that $\int_E |f|\, d\mu < \epsilon$ whenever $\mu(E) < \delta$.
  \end{problem}
\begin{solution}
  
\end{solution}
\begin{problem}
  Show that proposition 1.24(c) is also true when $c = \infty$.
  \end{problem}
\begin{solution}
  
\end{solution}
\end{document}
