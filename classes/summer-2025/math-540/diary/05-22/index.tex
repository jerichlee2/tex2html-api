\documentclass[12pt]{article}

% Packages
\usepackage[margin=1in]{geometry}
\usepackage{amsmath,amssymb,amsthm}
\usepackage{enumitem}
\usepackage{hyperref}
\usepackage{xcolor}
\usepackage{import}
\usepackage{xifthen}
\usepackage{pdfpages}
\usepackage{transparent}
\usepackage{listings}
\usepackage{tikz}
\usepackage{physics}
\usepackage{siunitx}
\usepackage{booktabs}
\usepackage{cancel}
  \usetikzlibrary{calc,patterns,arrows.meta,decorations.markings}


\DeclareMathOperator{\Log}{Log}
\DeclareMathOperator{\Arg}{Arg}

\lstset{
    breaklines=true,         % Enable line wrapping
    breakatwhitespace=false, % Wrap lines even if there's no whitespace
    basicstyle=\ttfamily,    % Use monospaced font
    frame=single,            % Add a frame around the code
    columns=fullflexible,    % Better handling of variable-width fonts
}

\newcommand{\incfig}[1]{%
    \def\svgwidth{\columnwidth}
    \import{./Figures/}{#1.pdf_tex}
}
\theoremstyle{definition} % This style uses normal (non-italicized) text
\newtheorem{solution}{Solution}
\newtheorem{proposition}{Proposition}
\newtheorem{problem}{Problem}
\newtheorem{lemma}{Lemma}
\newtheorem{theorem}{Theorem}
\newtheorem{remark}{Remark}
\newtheorem{note}{Note}
\newtheorem{definition}{Definition}
\newtheorem{example}{Example}
\newtheorem{corollary}{Corollary}
\theoremstyle{plain} % Restore the default style for other theorem environments
%

% Theorem-like environments
% Title information
\title{}
\author{Jerich Lee}
\date{\today}

\begin{document}

\maketitle
%-------------------------------------------------------------
%  Detail for Step (ii) — complements belong to 𝓜*
%-------------------------------------------------------------
\begin{enumerate}[label=\textbf{(ii)}]
  \item \textbf{Goal.}  
        Show that if $E\in\mathcal{M}^{*}$ then $E^{c}\in\mathcal{M}^{*}$.

  \item \textbf{Data for $E$.}  
        Because $E\in\mathcal{M}^{*}$, we can choose
        \[
            A,\;B\in\mathcal{M},
            \quad
            A\subset E\subset B,
            \quad
            \mu\!\bigl(B\setminus A\bigr)=0.
        \]

  \item \textbf{Take complements.}  
        Since $\mathcal{M}$ is a $\sigma$–algebra, it is closed under complements, so
        \[
            A^{c},\;B^{c}\in\mathcal{M}.
        \]
        Reversing inclusions gives
        \[
            B^{c}\subset E^{c}\subset A^{c}.
        \]
        Hence $B^{c}$ will play the role of the \emph{inner} measurable set
        and $A^{c}$ the \emph{outer} one for $E^{c}$.

  \item \textbf{Difference of the outer and inner sets.}  
        Compute
        \[
            A^{c}\setminus B^{c}
            ~=~A^{c}\cap B
            ~=~B\setminus A.
        \]
        (First equality: by definition of set difference.
         Second equality: $x\in A^{c}$ and $x\in B$ precisely when $x\in B$ but $x\notin A$.)

        Therefore
        \[
            \mu\bigl(A^{c}\setminus B^{c}\bigr)
            ~=~\mu\bigl(B\setminus A\bigr)
            ~=~0.
        \]

  \item \textbf{Conclusion.}  
        We have found $B^{c},A^{c}\in\mathcal{M}$ such that
        \[
            B^{c}\subset E^{c}\subset A^{c},
            \qquad
            \mu\bigl(A^{c}\setminus B^{c}\bigr)=0,
        \]
        which is exactly the criterion for $E^{c}$ to lie in $\mathcal{M}^{*}$.

        Hence
        \[
            E\in\mathcal{M}^{*}\;\Longrightarrow\;E^{c}\in\mathcal{M}^{*}.
        \]
        This verifies the \emph{closure‐under‐complements} axiom for the enlarged $\sigma$–algebra $\mathcal{M}^{*}$.
\end{enumerate}
\pagebreak
%-----------------------------------------------------------------------
%  Detailed expansion of Step (iii):  countable unions and additivity
%-----------------------------------------------------------------------
\begin{enumerate}[label=\textbf{(iii)}]
  \item \textbf{Objective.}  
        Verify that $\mathcal{M}^{*}$ is closed under countable unions
        and that the extended measure $\mu$ is countably additive.

  \item \textbf{Setup.}  
        Let $\{E_i\}_{i=1}^{\infty}\subset\mathcal{M}^{*}$.  
        For each $i$ choose measurable sets 
        $$A_i,\;B_i\in\mathcal{M}\quad\text{with}\quad
              A_i\subset E_i\subset B_i,
              \qquad
              \mu(B_i\setminus A_i)=0.$$

  \item \textbf{Build unions of the ``inner'' and ``outer'' sets.}  
        Define
        \[
            A \;:=\;\bigcup_{i=1}^{\infty}A_i,
            \qquad
            B \;:=\;\bigcup_{i=1}^{\infty}B_i.
        \]
        Because $\mathcal{M}$ is a $\sigma$--algebra, both 
        $A$ and $B$ lie in $\mathcal{M}$.  
        Moreover
        \[
            A\;\subset\;\bigcup_{i=1}^{\infty}E_i\;\subset\;B .
        \]

  \item \textbf{The gap has measure zero.}  
        Compute
        \[
            B\setminus A
            ~=~\Bigl(\;\bigcup_{i=1}^{\infty}B_i\Bigr)
             \setminus
             \Bigl(\;\bigcup_{i=1}^{\infty}A_i\Bigr)
            ~=~\bigcup_{i=1}^{\infty}\bigl(B_i\setminus A_i\bigr).
        \]
        Each $B_i\setminus A_i$ is null, and the countable union of
        null sets is still null, hence
        \[
            \mu(B\setminus A)=0.
        \]
        Therefore
        \[
            E \;:=\;\bigcup_{i=1}^{\infty}E_i
            \quad\Longrightarrow\quad
            E\in\mathcal{M}^{*}.
        \]

  \item \textbf{Countable additivity.}  
        Assume now that the $\{E_i\}$ are pairwise \emph{disjoint}.  
        Then the corresponding $\{A_i\}$ are also disjoint
        (because $A_i\subset E_i$).  
        Using the definition $\mu(E_i)=\mu(A_i)$,
        \[
            \mu(E)
            =\mu(A)
            =\mu\!\Bigl(\bigsqcup_{i=1}^{\infty}A_i\Bigr)
            =\sum_{i=1}^{\infty}\mu(A_i)
            =\sum_{i=1}^{\infty}\mu(E_i),
        \]
        where $\bigsqcup$ denotes a disjoint union and we used the
        countable additivity of $\mu$ on $\mathcal{M}$ for the second equality.

        Hence $\mu$ is countably additive on $\mathcal{M}^{*}$.
\end{enumerate}
\pagebreak
%-----------------------------------------------------------------
%  Why  B - A = ⋃_{i}(B_i - A)  and  ⊆ ⋃_{i}(B_i - A_i)
%-----------------------------------------------------------------
\[
\begin{aligned}
B - A
&:= B\setminus A
      && \text{(definition of set difference)}\\[4pt]
&=   \Bigl(\bigcup_{i=1}^{\infty} B_i\Bigr)
      \setminus
      \Bigl(\bigcup_{j=1}^{\infty} A_j\Bigr)
      && \text{(because }B=\bigcup B_i,\; A=\bigcup A_j\text{)}\\[6pt]
&=   \bigcup_{i=1}^{\infty}\Bigl( B_i\setminus\bigl(\bigcup_{j=1}^{\infty} A_j\bigr) \Bigr)
      && \text{(set--difference distributes over unions: }
         (X\cup Y)\setminus Z=(X\setminus Z)\cup(Y\setminus Z))\\[6pt]
&=   \bigcup_{i=1}^{\infty}\bigl( B_i\setminus A \bigr)
      && \text{(since }A=\bigcup_j A_j\text{).}
\end{aligned}
\]

\[
\text{\bf Inclusion } \bigl(B - A\bigr)\subset
                     \bigcup_{i=1}^{\infty}\bigl(B_i\setminus A_i\bigr)
\]

For each fixed \(i\) we have \(A_i\subset A\).  
Therefore
\[
    B_i\setminus A
    \;\subset\;
    B_i\setminus A_i.
\]
Taking the union over all \(i\) preserves inclusion, giving
\[
    \bigcup_{i=1}^{\infty}\bigl(B_i\setminus A\bigr)
    \;\subset\;
    \bigcup_{i=1}^{\infty}\bigl(B_i\setminus A_i\bigr).
\]
Combining the equality and the inclusion yields
\[
    B - A
    \;=\;
    \bigcup_{i=1}^{\infty}(B_i - A)
    \;\subset\;
    \bigcup_{i=1}^{\infty}(B_i - A_i),
\]
which is precisely the line quoted in the text.
\pagebreak
%-------------------------------------------------------------------
%  Expanded explanation of the paragraph preceding Theorem 1.37
%-------------------------------------------------------------------
\begin{remark}[Enlarged notion of measurability \& its usefulness]
  \mbox{}\\[-0.8\baselineskip]
  \begin{enumerate}
      \item \textbf{Why enlarge the definition?}  
            Lebesgue integration ignores what happens on sets of measure~$0$
            (``null sets’’).  
            If two functions differ only on a null set they have the \emph{same} integral
            and are therefore indistinguishable from the point of view of integration.
            This suggests that our notion of ``measurable function’’ should not be
            overly sensitive to values taken on null sets.

      \item \textbf{The new convention.}  
            Let \(E\in\mathcal{M}\) with \(\mu(E^{c})=0\).  
            A function \(f:E\to\mathbb{R}\) (or \(\mathbb{C}\)) is declared 
            \emph{measurable on \(X\)} if
            \(
                f^{-1}(V)\cap E
            \) is measurable for every open set \(V\subset\mathbb{R}\).
            In practice one then \emph{extends} \(f\) to all of \(X\) by setting
            \(
                f(x)=0
            \)
            (or any convenient value) whenever \(x\in E^{c}\).
            The integral
            \(
                \displaystyle\int_{A}f\,d\mu
            \)
            for any \(A\in\mathcal{M}\) is unaffected by how we choose the
            values of \(f\) on \(E^{c}\) because \(\mu(A\cap E^{c})=0\).

      \item \textbf{Role of completeness.}  
            If the measure space \((X,\mathcal{M},\mu)\) is \emph{complete}
            (i.e.\ every subset of a null set is measurable),
            then \emph{any} arbitrary definition of \(f\) on \(E^{c}\)
            still yields a measurable function on \(X\).
            In a non-complete space one can first pass to the
            \(\mu\)\nobreakdash–completion \(\mathcal{M}^{*}\) (Theorem~1.36) and obtain
            completeness for free.

      \item \textbf{Independence of the integral.}  
            Because \(f\) is integrable only through its values on~\(E\),
            \[
                \int_{A}f\,d\mu
                \;=\;
                \int_{A\cap E}f\,d\mu,
                \qquad
                \forall\,A\in\mathcal{M},
            \]
            so the choice of extension on \(E^{c}\) is literally irrelevant
            to every Lebesgue integral.

      \item \textbf{Natural situations where this matters.}
            \begin{enumerate}[]
                \item \emph{Differentiability a.e.}  
                      A real-valued function \(f\) may possess a derivative 
                      \(f'\) only almost everywhere.  
                      Under suitable hypotheses one can still prove
                      \(f(x)=f(a)+\displaystyle\int_{a}^{x}f'(t)\,dt\)
                      for all \(x\) (Fundamental Theorem of Calculus, a.e. version).
                      The derivative is ignored on the null set
                      where it fails to exist.
                \item \emph{Almost-everywhere convergence.}  
                      If a sequence \(\{f_n\}\) of measurable functions on \(X\) 
                      converges to a function \(f\) \emph{almost everywhere},
                      the pointwise limit may disagree with the limiting values
                      on a null set.  
                      With the enlarged notion we simply define \(f\) on that null
                      set arbitrarily; \(f\) is automatically measurable, so
                      results like the Dominated Convergence Theorem apply
                      without having to ``trim’’ the domain.
            \end{enumerate}
  \end{enumerate}
\end{remark}
\pagebreak
%-----------------------------------------------------------------
%  Detailed proof of Theorem 1.39(a)
%-----------------------------------------------------------------
\begin{proof}[Proof of part (a)]
  Assume $f:X\to[0,\infty]$ is measurable, $E\in\mathcal{M}$, and
  \[
      \int_{E} f\,d\mu \;=\;0.
  \]
  We must show $f=0$ almost everywhere (a.e.) on $E$.

  \medskip\noindent
  \textbf{Step 1:  Construct level sets that isolate the ``positive’’ region.}  
  For each $n\in\mathbb{N}$ define
  \[
      A_n \;:=\;\bigl\{x\in E : f(x)>\tfrac1n \bigr\}.
  \]
  Because $f$ is measurable, each $A_n$ is measurable.

  \medskip\noindent
  \textbf{Step 2:  Relate the integral over $A_n$ to its measure.}  
  On the set $A_n$ we have $f(x)>\tfrac1n$, so
  \[
      f(x) \;\ge\;\frac1n\quad\text{for all }x\in A_n .
  \]
  Integrating this inequality over $A_n$ gives
  \[
      \int_{A_n} f\,d\mu
      \;\ge\;
      \int_{A_n} \frac1n \,d\mu
      \;=\;
      \frac1n\,\mu(A_n).
  \tag{$\ast$}
  \]

  \medskip\noindent
  \textbf{Step 3:  Compare with the integral over $E$.}  
  Since $A_n\subset E$ and $f\ge0$, we also have
  \[
      0 \;\le\;\int_{A_n} f\,d\mu
      \;\le\;
      \int_{E} f\,d\mu
      \;=\;0,
  \]
  hence
  \[
      \int_{A_n} f\,d\mu = 0.
  \]
  Combining this with $(\ast)$ forces
  \[
      \frac1n\,\mu(A_n)=0
      \quad\Longrightarrow\quad
      \mu(A_n)=0
      \qquad (n=1,2,\dots).
  \]

  \medskip\noindent
  \textbf{Step 4:  Cover the set where $f$ is strictly positive.}  
  Observe that
  \[
      \bigl\{x\in E : f(x)>0\bigr\}
      \;=\;
      \bigcup_{n=1}^{\infty} A_n ,
  \]
  because any $x$ with $f(x)>0$ satisfies $f(x)>\tfrac1n$ for some~$n$.

  \medskip\noindent
  \textbf{Step 5:  Conclude measure zero and the desired result.}  
  By countable sub-additivity,
  \[
      \mu\bigl(\{x\in E : f(x)>0\}\bigr)
      \;\le\;
      \sum_{n=1}^{\infty}\mu(A_n)
      \;=\;0.
  \]
  Therefore $f(x)=0$ for $\mu$-almost every $x\in E$, as claimed.
\end{proof}
\pagebreak
%------------------------------------------------------------------
%  Why $\displaystyle\int_E f\,d\mu = 0$ (with $f\ge 0$) implies 
%  $f=0$ \emph{almost} everywhere, not everywhere
%------------------------------------------------------------------

\begin{remark}
  The Lebesgue integral is insensitive to what happens on sets of
  measure\/ $0$.  
  Hence an equation such as
  \[
      \int_E f\,d\mu = 0, 
      \qquad f\ge 0,
  \]
  can control $f$ only on the $\mu$--\emph{significant} part of $E$.
  On a subset of $E$ that already has measure $0$, the integral
  provides no information; $f$ may take arbitrary (even positive)
  values there.  
  Therefore the strongest universal conclusion one can draw is
  \[
      f=0 \quad\text{$\mu$--almost everywhere on $E$.}
  \]
\end{remark}

\begin{example}[Why ``a.e.'' cannot be strengthened to ``everywhere'']
  Define
  \[
      f(x)=
      \begin{cases}
          1, & x\in \mathbb{Q}\cap[0,1],\\[4pt]
          0, & x\in[0,1]\setminus\mathbb{Q},
      \end{cases}
      \qquad
      E=[0,1], 
      \qquad
      \mu=\lambda\ \text{(Lebesgue measure).}
  \]
  Then
  \[
      \int_E f\,d\lambda
      =\lambda(\mathbb{Q}\cap[0,1]) = 0,
  \]
  but $f(x)=1$ for every rational $x\in[0,1]$.  
  Because the rationals have Lebesgue measure\/ $0$, we conclude
  $f=0$ \emph{a.e.} on $E$, yet $f$ certainly is not zero
  \emph{everywhere}.
\end{example}

\begin{center}
  \hrulefill
\end{center}

\noindent\textbf{Summary.}
\begin{enumerate}
  \item The proof of Theorem~1.39(a) shows that the set
        $\{x\in E:f(x)>0\}$ has measure\/ $0$.
  \item A statement about integrals cannot rule out exceptional
        behaviour on such null sets.
  \item Thus ``$f=0$ a.e.'' is optimal without extra regularity
        assumptions (e.g.\ continuity) on $f$.
\end{enumerate}
\pagebreak
%--------------------------------------------------------------------
%  Theorem 1.40  –  Step-by-step proof (with detailed commentary)
%--------------------------------------------------------------------
\begin{problem}
  \textbf{Theorem 1.40.}
  Suppose $\mu(X)<\infty$, $f\in L^{1}(\mu)$, $S\subset\mathbb{C}$ is
  closed, and
  \[
      A_{E}(f)
      \;:=\;
      \frac{1}{\mu(E)}\int_{E}f\,d\mu
      \;\in\;S
      \qquad
      \text{for every }E\in\mathcal{M}\text{ with }0<\mu(E)<\infty.
  \]
  Then $f(x)\in S$ for $\mu$-almost every $x\in X$.

  \begin{enumerate}
      %----------------------------------------------------------
      \item \textbf{Reduce to discs that miss $S$.}\label{step:discs}
            \begin{enumerate}
                \item $S^{c}$ is an \emph{open} subset of $\mathbb{C}$,
                      hence it can be written as a countable union of
                      closed discs  
                      \(
                          S^{c}=\bigcup_{k\ge1}\Delta_{k},
                      \)
                      where $\Delta_{k}$ has centre $\alpha_{k}$ and radius $r_{k}>0$.
                \item Proving $\mu\!\bigl(f^{-1}(S^{c})\bigr)=0$
                      is thus equivalent to proving
                      $\mu\!\bigl(f^{-1}(\Delta_{k})\bigr)=0$ for
                      \emph{each} $k\in\mathbb{N}$.
            \end{enumerate}

      %----------------------------------------------------------
      \item \textbf{Fix one disc and show its pre-image is null.}
            Pick a particular $k$ and set
            \[
                \Delta:=\Delta_{k}
                \quad(\text{centre }\alpha,\ \text{radius }r),
                \qquad
                E:=f^{-1}(\Delta)\in\mathcal{M}.
            \]
            We will prove $\mu(E)=0$.

            \begin{enumerate}
                \item \textit{Assume, toward contradiction, that $\mu(E)>0$.}
                \item By hypothesis, $A_{E}(f)\in S$.
                      Compute
                      \[
                          \bigl|A_{E}(f)-\alpha\bigr|
                          =\left|
                              \frac{1}{\mu(E)}
                              \int_{E}\bigl(f-\alpha\bigr)\,d\mu
                            \right|
                          \le
                          \frac{1}{\mu(E)}
                          \int_{E}\bigl|f-\alpha\bigr|\,d\mu
                          \le r ,
                      \]
                      because $|f(x)-\alpha|\le r$ for all $x\in E$ 
                      (that is, $f(E)\subset\Delta$).
                \item Consequently $A_{E}(f)\in\overline{\Delta}$, 
                      yet $\overline{\Delta}\subset S^{c}$ by construction,
                      contradicting $A_{E}(f)\in S$.
                \item Hence our assumption $\mu(E)>0$ is impossible;
                      therefore $\mu(E)=0$.
            \end{enumerate}

      %----------------------------------------------------------
      \item \textbf{Finish the argument.}
            Since each set $f^{-1}(\Delta_{k})$ has measure $0$ and
            \(
              f^{-1}(S^{c})
              =\displaystyle\bigcup_{k\ge1}f^{-1}(\Delta_{k}),
            \)
            countable sub-additivity gives
            \[
                \mu\bigl(f^{-1}(S^{c})\bigr)=0.
            \]
            Hence $f(x)\in S$ for $\mu$-almost every $x\in X$.
  \end{enumerate}
\end{problem}
\pagebreak
%--------------------------------------------------------------------
%  Detailed justification of the chain of inequalities
%--------------------------------------------------------------------
\[
\bigl|A_{E}(f)-\alpha\bigr|
      =\left|
          \frac{1}{\mu(E)}\int_{E}\bigl(f-\alpha\bigr)\,d\mu
        \right|
      \;\le\;
      \frac{1}{\mu(E)}\int_{E}\bigl|f-\alpha\bigr|\,d\mu
      \;\le\;
      r.
\]

\begin{enumerate}
    \item \textbf{Definition of the average}  
          By definition
          \[
              A_{E}(f)
              =\frac{1}{\mu(E)}\int_{E}f\,d\mu .
          \]
          Therefore
          \[
              A_{E}(f)-\alpha
              =\frac{1}{\mu(E)}\int_{E}\bigl(f-\alpha\bigr)\,d\mu .
          \]

    \item \textbf{Absolute value moves inside an integral.}  
          For any integrable complex-valued function $g$ we have
          \[
              \Bigl|\int_{E}g\,d\mu\Bigr|
              \;\le\;
              \int_{E}|g|\,d\mu ,
          \]
          which is just the triangle inequality in integral form
          (sometimes called Jensen’s inequality for the modulus).
          Apply this to $g=f-\alpha$:
          \[
              \Bigl| \int_{E}(f-\alpha)\,d\mu \Bigr|
              \;\le\;
              \int_{E}|f-\alpha|\,d\mu .
          \]
          Dividing by the positive constant $\mu(E)$ preserves the inequality:
          \[
              \left|
                \frac{1}{\mu(E)}\int_{E}(f-\alpha)\,d\mu
              \right|
              \;\le\;
              \frac{1}{\mu(E)}\int_{E}|f-\alpha|\,d\mu .
          \]

    \item \textbf{Bounding the integrand by \(r\).}  
          Because $E=f^{-1}(\Delta)$ and $\Delta$ is the closed disc
          \(\{z\in\mathbb{C}:|z-\alpha|\le r\}\),
          every $x\in E$ satisfies
          \(|f(x)-\alpha|\le r\).
          Hence
          \[
              \int_{E}|f-\alpha|\,d\mu
              \;\le\;
              \int_{E} r\,d\mu
              = r\,\mu(E).
          \]

    \item \textbf{Combine the estimates.}  
          Insert the bound from step~(3) into the right-hand side of
          step~(2) to obtain
          \[
              \bigl|A_{E}(f)-\alpha\bigr|
              \;\le\;
              \frac{1}{\mu(E)} \bigl(r\,\mu(E)\bigr)
              = r.
          \]
\end{enumerate}

This completes the detailed verification of the displayed inequality.
\pagebreak
\paragraph{Why the identity holds}

Start with the definition of the average (or “mean value’’) of $f$ on $E$:
\[
    A_E(f) \;=\; \frac{1}{\mu(E)}\int_E f\,d\mu .
\]

Subtracting the \emph{constant} $\alpha$ gives
\[
    A_E(f)-\alpha
    \;=\;
    \frac{1}{\mu(E)}\int_E f\,d\mu
    \;-\;
    \alpha .
\]

\begin{enumerate}
    \item \textbf{Rewrite the constant as an integral.}  
          Because $\alpha$ does not depend on $x$,
          \[
              \alpha
              \;=\;
              \frac{1}{\mu(E)}\underbrace{\bigl(\alpha\,\mu(E)\bigr)}_{\displaystyle=\int_E \alpha\,d\mu}
              \;=\;
              \frac{1}{\mu(E)}\int_E \alpha\,d\mu .
          \]

    \item \textbf{Use linearity of the integral.}  
          Now subtract the two integrals that share the same prefactor:
          \[
              \frac{1}{\mu(E)}
              \Bigl(\,\int_E f\,d\mu \;-\; \int_E \alpha\,d\mu \Bigr)
              \;=\;
              \frac{1}{\mu(E)}
              \int_E (f-\alpha)\,d\mu .
          \]
          The equality
          \[
              \int_E f\,d\mu - \int_E \alpha\,d\mu
              \;=\;
              \int_E\bigl(f-\alpha\bigr)\,d\mu
          \]
          is simply the additivity (linearity) of the Lebesgue integral.
\end{enumerate}

Combining these two observations yields
\[
    A_E(f)-\alpha
    \;=\;
    \frac{1}{\mu(E)}\int_E (f-\alpha)\,d\mu,
\]
which is exactly the identity used in the proof.
\pagebreak
%--------------------------------------------------------------------
%  Motivation for \emph{Lebesgue measure} via linear functionals
%--------------------------------------------------------------------
\begin{problem}
  \textbf{Why do we introduce Lebesgue measure and how does the
  linear functional $\displaystyle\Lambda f=\int_0^1 f(x)\,dx$
  guide its construction?}
  \begin{enumerate}
  %----------------------------------------------------------------
  \item \textbf{Integration as a linear (positive) functional.}
        \begin{enumerate}[label=(\alph*)]
            \item The space 
                  \(
                      C=C([0,1];\mathbb{C})
                  \)
                  of continuous functions on $[0,1]$ is a vector space.
            \item The usual Riemann integral
                  \(
                      \Lambda:C\to\mathbb{C},\;
                      \Lambda f=\int_0^1 f(x)\,dx
                  \)
                  is \emph{linear}:
                  $\Lambda(af+bg)=a\Lambda f+b\Lambda g$.
            \item It is also \emph{positive}:
                  for $f\ge0$ (pointwise) we have $\Lambda f\ge0$.
        \end{enumerate}
  %----------------------------------------------------------------
  \item \textbf{Recovering ``length’’ from $\Lambda$.}
        \begin{enumerate}[label=(\alph*)]
            \item Fix an interval $(a,b)\subset[0,1]$.
                  Consider the class
                  \[
                      \mathcal{F}_{(a,b)}
                      =\Bigl\{f\in C:0\le f\le1\text{ on }[0,1]
                             \text{ and }f(x)=0\text{ for }x\notin(a,b)\Bigr\}.
                  \]
            \item For any $f\in\mathcal{F}_{(a,b)}$ we have
                  $\Lambda f\le b-a$ because $f\le1$ and is supported
                  on $(a,b)$.
            \item By choosing ``tall\,/\,thin’’ bumps that approximate
                  the indicator $\chi_{(a,b)}$, we can make $\Lambda f$
                  arbitrarily close to $b-a$:
                  \[
                      \sup_{f\in\mathcal{F}_{(a,b)}}\Lambda f=b-a.
                  \]
            \item Hence \emph{interval length is encoded in $\Lambda$}.  
                  We want a set function $m$ on \emph{all} subsets of
                  $[0,1]$ that extends this idea.
        \end{enumerate}
  %----------------------------------------------------------------
  \item \textbf{From lengths of intervals to a measure on \boldmath$[0,1]$.}
        \begin{enumerate}[label=(\alph*)]
            \item Start with outer measure:
                  \[
                      m^\ast(E)=\inf\Bigl\{\;\sum_{k} (b_k-a_k):
                      E\subset\bigcup_k(a_k,b_k)\Bigr\}.
                  \]
                  The infimum is taken over \emph{all} open covers by
                  countable unions of intervals.
            \item Use Carathéodory’s \emph{criterion} to carve out the
                  $\sigma$--algebra of $m^\ast$–measurable sets; the
                  resulting restriction
                  \(
                      m=m^\ast|_{\mathcal{L}}
                  \)
                  is the \textbf{Lebesgue measure}.
            \item By construction $m((a,b))=b-a$ and
                  $m$ is complete, translation‐invariant, and countably
                  additive.
        \end{enumerate}
  %----------------------------------------------------------------
  \item \textbf{Why not stay with Riemann?}
        \begin{enumerate}[label=(\alph*)]
            \item Riemann integration fails for many limits:
                  if $f_n\to f$ pointwise but not uniformly, 
                  interchange of limit and integral breaks down.
            \item Sets of ``size’’ zero (e.g.\ Cantor set) can still
                  carry functions whose Riemann integral is undefined,
                  yet Lebesgue theory handles them effortlessly.
            \item In $L^p$ analysis, probability, Fourier transform
                  theory, etc., we \emph{must} integrate highly
                  discontinuous functions; Lebesgue’s extension
                  provides the required robustness.
        \end{enumerate}
  %----------------------------------------------------------------
  \item \textbf{Functional perspective summarised.}
        \begin{itemize}
            \item A positive linear functional on $C([0,1])$
                  ``remembers’’ the length of intervals via the
                  approximation argument in step~\textbf{2}.
            \item The Riesz–Markov representation theorem
                  (proved later) formalises this:
                  every positive linear functional on $C([0,1])$
                  arises from a \emph{unique} Borel measure $\mu$
                  such that
                  \[
                      \Lambda f=\int_{[0,1]} f\,d\mu
                      \quad (f\in C).
                  \]
            \item When $\Lambda$ is the usual Riemann integral,
                  the representing measure \emph{must} coincide with
                  Lebesgue measure.
            \item Thus Lebesgue measure is \emph{the canonical measure}
                  whose integrals extend and complete the classical
                  Riemann process \emph{while retaining linearity and
                  positivity}.
        \end{itemize}
  \end{enumerate}
\end{problem}
\end{document}
