\documentclass[12pt]{article}

% Packages
\usepackage[margin=1in]{geometry}
\usepackage{amsmath,amssymb,amsthm}
\usepackage{enumitem}
\usepackage{hyperref}
\usepackage{xcolor}
\usepackage{import}
\usepackage{xifthen}
\usepackage{pdfpages}
\usepackage{transparent}
\usepackage{listings}
\usepackage{tikz}
\usepackage{physics}
\usepackage{siunitx}
\usepackage{booktabs}
\usepackage{cancel}
  \usetikzlibrary{calc,patterns,arrows.meta,decorations.markings}


\DeclareMathOperator{\Log}{Log}
\DeclareMathOperator{\Arg}{Arg}

\lstset{
    breaklines=true,         % Enable line wrapping
    breakatwhitespace=false, % Wrap lines even if there's no whitespace
    basicstyle=\ttfamily,    % Use monospaced font
    frame=single,            % Add a frame around the code
    columns=fullflexible,    % Better handling of variable-width fonts
}

\newcommand{\incfig}[1]{%
    \def\svgwidth{\columnwidth}
    \import{./Figures/}{#1.pdf_tex}
}
\theoremstyle{definition} % This style uses normal (non-italicized) text
\newtheorem{solution}{Solution}
\newtheorem{proposition}{Proposition}
\newtheorem{problem}{Problem}
\newtheorem{lemma}{Lemma}
\newtheorem{theorem}{Theorem}
\newtheorem{remark}{Remark}
\newtheorem{note}{Note}
\newtheorem{definition}{Definition}
\newtheorem{example}{Example}
\newtheorem{corollary}{Corollary}
\theoremstyle{plain} % Restore the default style for other theorem environments
%

% Theorem-like environments
% Title information
\title{}
\author{Jerich Lee}
\date{\today}

\begin{document}

\maketitle
%--------------------------------------------------------------------
%  Why (8) together with  $\Lambda f\le \mu(V)<\mu(K)+\varepsilon$
%  yields (7)
%--------------------------------------------------------------------

\begin{align}
  \mu(K) \;\le\; \Lambda f \tag{8}
\end{align}

\noindent
Now fix an arbitrary $\varepsilon>0$.  
By Urysohn’s lemma there exists an open set $V\supset K$ with  
$\mu(V)<\mu(K)+\varepsilon$ and a continuous function  
$\,f$ such that
\[
  K \prec f \prec V .
\]
For this particular $f$ we therefore have  
\[
  \Lambda f \;\le\; \mu(V) \;<\; \mu(K)+\varepsilon .
\]
Hence
\[
  \inf\{\Lambda g : K \prec g\}\;\le\;\Lambda f
  \;<\;\mu(K)+\varepsilon .
\]
Because $\varepsilon>0$ is arbitrary, letting $\varepsilon\to 0$ gives  
\[
  \inf\{\Lambda g : K \prec g\}\;\le\;\mu(K).
\]
Together with (8), which states $\mu(K)\le\Lambda f$ for \emph{every}  
$K\prec f$, we obtain the reverse inequality
\[
  \mu(K)\;\le\;\inf\{\Lambda g : K \prec g\}.
\]
Consequently,
\[
  \boxed{\,
     \mu(K)=\inf\{\Lambda f : K \prec f\}
  \,},
\]
which is exactly equation (7).  \qed
\pagebreak
%--------------------------------------------------------------------
%  STEP III — Every open set $V$ with $\mu(V)<\infty$ satisfies (3)
%--------------------------------------------------------------------
%
%  Recall (3):   $\displaystyle 
%          \mu(E)=\sup\{\mu(K)\;:\;K\subset E,\;K\text{ compact}\}$.
%
%  Goal:  Show this for any \emph{open} $V$ with $\mu(V)<\infty$.
%--------------------------------------------------------------------

\paragraph{Proof.}
Fix an arbitrary real number $\alpha$ with $\alpha<\mu(V)$.  
Since $\mu(V)=\sup\{\Lambda f: f\prec V\}$ by definition of $\mu$,  
there exists a continuous function $f$ such that  
\[
      f\prec V
      \quad\text{and}\quad
      \alpha<\Lambda f,
\]
where $\Lambda f=\int_X f\,d\mu$ (or whatever functional $\Lambda$ denotes).

\medskip
Let $K:=\operatorname{supp}f$ be the (necessarily compact) support of $f$.  
Because $V$ is open and $f\prec V$, we have $K\subset V$.

\medskip
Now take any open set $W$ with $K\subset W$ (in particular we may take
$W=V$).  Since $f\prec W$, monotonicity of $\mu$ with respect to the
order $\prec$ yields
\[
      \Lambda f\;\le\;\mu(W).
\]
Choosing $W=K$ gives $\Lambda f\le\mu(K)$, whence
\[
      \alpha\;<\;\Lambda f\;\le\;\mu(K)\;\le\;
      \sup\{\mu(L):L\subset V,\;L\text{ compact}\}.
\]

\medskip
Because $\alpha<\mu(V)$ was \emph{arbitrary}, it follows that
\[
      \mu(V)\;\le\;\sup\{\mu(K):K\subset V,\;K\text{ compact}\}.
\]
The reverse inequality is immediate since every compact
$K\subset V$ is contained in $V$, so $\mu(K)\le\mu(V)$.
Consequently
\[
      \boxed{\;
         \mu(V)=\sup\{\mu(K):K\subset V,\;K\text{ compact}\}
      \;},
\]
which is precisely condition~(3).  Hence every open
$V$ with $\mu(V)<\infty$ lies in $\mathfrak{M}_F$.
\qed
\pagebreak
%---------------------------------------------------------------
%  Why do we introduce the number $\alpha$?
%---------------------------------------------------------------

The rôle of the auxiliary number $\alpha$ is purely \emph{approximation}:

\begin{itemize}
  \item We know \(\mu(V)\) is the \emph{supremum} of all numbers
        \(\Lambda f\) with \(f\prec V\), but in general that supremum
        need not be \emph{attained} by any single \(f\).
  \item By picking an \emph{arbitrary} real number
        \(\alpha<\mu(V)\) we set a target “just below’’ \(\mu(V)\).
        The definition of a supremum then guarantees
        the existence of some \(f\prec V\) with
        \[
             \alpha<\Lambda f\;(\le \mu(V)).
        \]
        Thus we manufacture a concrete function whose integral comes as
        close to \(\mu(V)\) as we like.
  \item The support \(K=\operatorname{supp}f\) of that \(f\) is compact
        and satisfies
        \[
             \alpha<\Lambda f\le\mu(K)\le\mu(V).
        \]
        Because \(\alpha\) was \emph{arbitrary}, this shows that for
        every neighbourhood below \(\mu(V)\) we can find a compact
        subset \(K\subset V\) whose measure falls inside that
        neighbourhood.  In symbols,
        \[
             \text{for every } \alpha<\mu(V)\text{ there is a compact }
             K\subset V\text{ with }\alpha<\mu(K).
        \]
        Precisely this property forces
        \[
             \mu(V)\le\sup\{\mu(K):K\subset V,\;K\text{ compact}\},
        \]
        completing the verification of condition~(3) for \(V\).
\end{itemize}

\noindent
\textbf{In short:}  
\(\alpha\) plays the same rôle that an \(\varepsilon\) does in
``$\sup$–$\inf$’’ arguments—it lets us push arbitrarily close to
\(\mu(V)\) even when \(\mu(V)\) is merely a limit value rather than an
attained one.
\pagebreak
%--------------------------------------------------------------------
%  Detailed walk-through of the chain of inequalities on p.\,47
%--------------------------------------------------------------------
\begin{align}
  \Lambda f
    &= \sum_{i=1}^{n}\Lambda\!\bigl(h_i f\bigr)
        &&\text{(linearity of $\Lambda$ and $\sum_{i=1}^{n}h_i=1$ on $K$)}\\[6pt]
    &\le \sum_{i=1}^{n}(y_i+\epsilon)\,\Lambda h_i
        &&\text{since $h_i f\le(y_i+\epsilon)h_i$ everywhere}\\[6pt]
    &= \sum_{i=1}^{n}\bigl(|a|+y_i+\epsilon\bigr)\Lambda h_i
          -|a|\sum_{i=1}^{n}\Lambda h_i
        &&\text{add--subtract the constant $|a|$}\\[6pt]
    &\le \sum_{i=1}^{n}\bigl(|a|+y_i+\epsilon\bigr)
          \bigl[\mu(E_i)+\epsilon/n\bigr]-|a|\mu(K)
        &&\text{(19): $\Lambda h_i\le\mu(E_i)+\epsilon/n$ and  
                     $\sum_{i=1}^{n}\Lambda h_i=\mu(K)$}\\[6pt]
    &= \sum_{i=1}^{n}(y_i-\epsilon)\,\mu(E_i)
          +2\epsilon\mu(K)
          +\frac{\epsilon}{n}\sum_{i=1}^{n}\bigl(|a|+y_i+\epsilon\bigr)
        &&\text{expand and gather $\epsilon$-terms}\\[6pt]
    &\le \int_{X}f\,d\mu
          +\epsilon\bigl[2\mu(K)+|a|+b+\epsilon\bigr]
        &&\text{$y_i-\epsilon<f$ on $E_i$ and $y_i\le b$}
  \end{align}
  
  \begin{enumerate}
    \item \textbf{Line 1.}  Write $f=\sum_{i=1}^{n}h_i f$ on $K$; apply the linear functional $\Lambda$.
    \item \textbf{Line 2.}  On $V_i$ we constructed $f(x)<y_i+\epsilon$, hence $h_i f\le(y_i+\epsilon)h_i$; monotonicity of $\Lambda$ preserves the inequality.
    \item \textbf{Line 3.}  Insert and remove the same constant $|a|$ (the left-endpoint of $[a,b]$, the range of $f$) to prepare for the next estimate.
    \item \textbf{Line 4.}  From (19) we have $\Lambda h_i\le\mu(V_i)\le\mu(E_i)+\epsilon/n$.  
          Because $\sum h_i=1$ on $K$, $\sum_{i=1}^{n}\Lambda h_i=\Lambda1=\mu(K)$.
    \item \textbf{Line 5.}  Distribute the bracket, collect like terms,
          and isolate factors of $\epsilon$.
    \item \textbf{Line 6.}  On each $E_i$ one has $y_i-\epsilon<f$, so $\sum_{i=1}^{n}(y_i-\epsilon)\mu(E_i)\le\int_{X}f\,d\mu$.  
          The remaining sum is bounded by $\epsilon(|a|+b+\epsilon)$ because $y_i\le b$.
  \end{enumerate}
  
  Since $\epsilon>0$ was arbitrary, letting $\epsilon\to0$ yields
  \[
    \Lambda f\;\le\;\int_{X}f\,d\mu,
  \]
  establishing inequality (16) and completing the proof.
  \pagebreak
  %--------------------------------------------------------------------
%  Zooming in on \textbf{Line 5} and \textbf{Line 6}
%--------------------------------------------------------------------
Recall the expression we reached at the end of Line 4:
\[
  S
  \;=\;
  \sum_{i=1}^{n}\bigl(|a|+y_i+\epsilon\bigr)\bigl[\mu(E_i)+\tfrac{\epsilon}{n}\bigr]
  \;-\;|a|\,\mu(K).
\]
Here $\mu(K)=\sum_{i=1}^{n}\mu(E_i)$ because the $E_i$ are pairwise
disjoint and cover $K$.

%--------------------------------------------------------------------
\paragraph{Line 5.  Algebraic clean-up.}
First expand the bracket and separate the terms:
\begin{align*}
S
&=
  \sum_{i=1}^{n}\bigl(|a|+y_i+\epsilon\bigr)\mu(E_i)
  \;+\;
  \frac{\epsilon}{n}\sum_{i=1}^{n}\bigl(|a|+y_i+\epsilon\bigr)
  \;-\;
  |a|\,\mu(K).
\end{align*}

\noindent
Split the first sum into a ``$|a|$-part’’ and a ``$(y_i+\epsilon)$-part’’,
then use $\sum_{i=1}^{n}\mu(E_i)=\mu(K)$ to cancel the $|a|$-terms:
\begin{align*}
S
&=
  |a|\,\mu(K)+\sum_{i=1}^{n}(y_i+\epsilon)\mu(E_i)
  \;+\;
  \frac{\epsilon}{n}\sum_{i=1}^{n}\bigl(|a|+y_i+\epsilon\bigr)
  \;-\;
  |a|\,\mu(K)
\\[4pt]
&=
  \sum_{i=1}^{n}(y_i+\epsilon)\mu(E_i)
  \;+\;
  \frac{\epsilon}{n}\sum_{i=1}^{n}\bigl(|a|+y_i+\epsilon\bigr).
\end{align*}

\noindent
Finally, write
\(
  (y_i+\epsilon)\mu(E_i)
  =(y_i-\epsilon)\mu(E_i) + 2\epsilon\mu(E_i)
\)
and sum over $i$:
\[
  S
  =
  \sum_{i=1}^{n}(y_i-\epsilon)\mu(E_i)
  \;+\;
  2\epsilon\mu(K)
  \;+\;
  \frac{\epsilon}{n}\sum_{i=1}^{n}\bigl(|a|+y_i+\epsilon\bigr).
\]
All $\epsilon$-dependence is now explicit—that is the whole point of
Line 5.

%--------------------------------------------------------------------
\paragraph{Line 6.  Estimating the two pieces.}

\begin{enumerate}
\item\emph{The main (non-$\epsilon$) term.}
      On $E_i$ we have
      \(
        y_{i-1}<f(x)<y_i
      \)
      by construction, hence
      \(
        y_i-\epsilon<f(x)
      \)
      whenever $\epsilon>0$ is small enough.  Therefore
      \[
          \sum_{i=1}^{n}(y_i-\epsilon)\mu(E_i)
          \;\le\;
          \int_{X}f\,d\mu.
      \]

\item\emph{The $\epsilon$-terms.}
      Because $y_i\le b$ for every $i$ (the $y_i$ live in the range
      $[a,b]$ of $f$),
      \[
          \frac{\epsilon}{n}\sum_{i=1}^{n}\bigl(|a|+y_i+\epsilon\bigr)
          \;\le\;
          \epsilon\bigl(|a|+b+\epsilon\bigr).
      \]
      Adding the $2\epsilon\mu(K)$ coming from the previous line gives
      the total $\epsilon$-contribution
      \[
         \epsilon\bigl[2\mu(K)+|a|+b+\epsilon\bigr].
      \]
\end{enumerate}

Putting the two pieces together we arrive at
\[
   \Lambda f
   \;\le\;
   \int_{X}f\,d\mu
   \;+\;
   \epsilon\bigl[2\mu(K)+|a|+b+\epsilon\bigr].
\]
Since $\epsilon>0$ was arbitrary, letting $\epsilon\to0$ yields the
desired inequality
\(
  \Lambda f\le\int_{X}f\,d\mu,
\)
completing the proof.
\end{document}
