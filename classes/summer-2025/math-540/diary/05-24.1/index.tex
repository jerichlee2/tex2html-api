\documentclass[12pt]{article}

% Packages
\usepackage[margin=1in]{geometry}
\usepackage{amsmath,amssymb,amsthm}
\usepackage{enumitem}
\usepackage{hyperref}
\usepackage{xcolor}
\usepackage{import}
\usepackage{xifthen}
\usepackage{pdfpages}
\usepackage{transparent}
\usepackage{listings}
\usepackage{tikz}
\usepackage{physics}
\usepackage{siunitx}
\usepackage{booktabs}
\usepackage{cancel}
  \usetikzlibrary{calc,patterns,arrows.meta,decorations.markings}


\DeclareMathOperator{\Log}{Log}
\DeclareMathOperator{\Arg}{Arg}

\lstset{
    breaklines=true,         % Enable line wrapping
    breakatwhitespace=false, % Wrap lines even if there's no whitespace
    basicstyle=\ttfamily,    % Use monospaced font
    frame=single,            % Add a frame around the code
    columns=fullflexible,    % Better handling of variable-width fonts
}

\newcommand{\incfig}[1]{%
    \def\svgwidth{\columnwidth}
    \import{./Figures/}{#1.pdf_tex}
}
\theoremstyle{definition} % This style uses normal (non-italicized) text
\newtheorem{solution}{Solution}
\newtheorem{proposition}{Proposition}
\newtheorem{problem}{Problem}
\newtheorem{lemma}{Lemma}
\newtheorem{theorem}{Theorem}
\newtheorem{remark}{Remark}
\newtheorem{note}{Note}
\newtheorem{definition}{Definition}
\newtheorem{example}{Example}
\newtheorem{corollary}{Corollary}
\theoremstyle{plain} % Restore the default style for other theorem environments
%

% Theorem-like environments
% Title information
\title{}
\author{Jerich Lee}
\date{\today}

\begin{document}

\maketitle
\begin{solution}
  Recall that a real-valued function $h:X\to\mathbb{R}$ is  
  \emph{lower semicontinuous (l.s.c.)} iff, for every $\alpha\in\mathbb{R}$,
  \begin{align*}
    \{x\in X : h(x)>\alpha\}
  \end{align*}
  is open (equivalently, $\displaystyle\liminf_{x\to x_0}h(x)\ge h(x_0)$ for all $x_0\in X$).  
  Dually, $h$ is \emph{upper semicontinuous (u.s.c.)} iff the set  
  $\{x\in X : h(x)<\alpha\}$ is open for every $\alpha$.
  
  \begin{enumerate}
      \item \textbf{Each $f_r$ is l.s.c.}\;
            For a rational $r\in[0,1]$,
            \begin{align*}
                f_r(x)=
                \begin{cases}
                    r & \text{if } x\in V_r,\\[2pt]
                    0 & \text{otherwise.}
                \end{cases}
            \end{align*}
            Let $\alpha\in\mathbb{R}$.
            \begin{itemize}
                \item If $\alpha\ge r$, then $\{f_r>\alpha\}=\varnothing$, which is open.
                \item If $0\le\alpha<r$, then
                      \begin{align*}
                          \{f_r>\alpha\}=V_r,
                      \end{align*}
                      and $V_r$ is open by construction.
                \item If $\alpha<0$, then $\{f_r>\alpha\}=X$, which is open.
            \end{itemize}
            Hence $f_r$ is lower semicontinuous.
  
      \item \textbf{$f=\sup_{r} f_r$ is l.s.c.}\;
            By Remark 2.8(c) in Rudin, the pointwise supremum of any family of l.s.c.\ functions is again l.s.c.; therefore $f$ is lower semicontinuous.
  
      \item \textbf{Each $g_s$ is u.s.c.}\;
            For a rational $s\in[0,1]$,
            \begin{align*}
                g_s(x)=
                \begin{cases}
                    1 & \text{if } x\in\overline{V_s},\\[2pt]
                    s & \text{otherwise.}
                \end{cases}
            \end{align*}
            Let $\alpha\in\mathbb{R}$.
            \begin{itemize}
                \item If $\alpha\le s$, then $\{g_s<\alpha\}=\varnothing$, which is open.
                \item If $s<\alpha\le 1$, then
                      \begin{align*}
                          \{g_s<\alpha\}=X\setminus\overline{V_s},
                      \end{align*}
                      the complement of a closed set, hence open.
                \item If $\alpha>1$, then $\{g_s<\alpha\}=X$, which is open.
            \end{itemize}
            Thus every lower level set of $g_s$ is open, so $g_s$ is upper semicontinuous.
  
      \item \textbf{$g=\inf_{s} g_s$ is u.s.c.}\;
            By Remark 2.8(d) in Rudin, the pointwise infimum of u.s.c.\ functions is again u.s.c.; therefore $g$ is upper semicontinuous.
  \end{enumerate}
  \end{solution}
  \pagebreak
  \begin{solution}
    The proof begins by fixing a \emph{sequence of rational numbers}
    \[
      r_{1},\; r_{2},\; r_{3},\; r_{4},\;\dotsc
      \qquad\text{with}\qquad
      r_{1}=0,\; r_{2}=1,
    \]
    and where
    \[
      r_{3},\,r_{4},\,r_{5},\dotsc
    \]
    is \emph{an enumeration of all rationals in $(0,1)$}.  
    Here is what that means, precisely:
    
    \begin{enumerate}
        \item The set $\mathbb{Q}\cap(0,1)$ is \emph{countable}, so there exists a bijection
              \[
                 \mathbb{N}\;\longrightarrow\;\mathbb{Q}\cap(0,1),\qquad 
                 n\;\longmapsto\; q_{n}.
              \]
              In other words, we can list the rationals in $(0,1)$ as a sequence
              $q_{1},q_{2},q_{3},\dotsc$ with no repetition.
    
        \item We simply relabel that list by setting
              \[
                r_{3}=q_{1},\; r_{4}=q_{2},\; r_{5}=q_{3},\;\dotsc,
              \]
              while retaining $r_{1}=0$ and $r_{2}=1$.
    
        \item Thus the full sequence is
              \[
                0,\; 1,\; r_{3},\; r_{4},\; r_{5},\;\dotsc
                \quad=\quad
                0,\; 1,\; q_{1},\; q_{2},\; q_{3},\;\dotsc,
              \]
              which lists \emph{every} rational number in the closed interval
              $[0,1]$ exactly once, starting with the two ``boundary'' values
              $0$ and~$1$.
    \end{enumerate}
    
    \medskip
    \noindent
    \textbf{Concrete example of an enumeration.}  
    One common way to enumerate $\mathbb{Q}\cap(0,1)$ is to order
    the rationals by increasing denominator, then by increasing numerator, and
    skip any fraction that is not in lowest terms:
    \[
      \frac{1}{2},\;
      \frac{1}{3},\;
      \frac{2}{3},\;
      \frac{1}{4},\;
      \frac{3}{4},\;
      \frac{1}{5},\;
      \frac{2}{5},\;
      \frac{3}{5},\;
      \frac{4}{5},\;\dotsc
    \]
    Assigning these in order to $r_{3},r_{4},r_{5},\dotsc$ gives a perfectly
    valid sequence $r_{n}$ for the proof.
    
    \end{solution}
    \pagebreak
    \begin{example}[A fully–worked concrete model in $\mathbb{R}$]
      Let  
      \[
         X=\mathbb{R}, \qquad 
         K=[{-}\tfrac12,\tfrac12], \qquad 
         V=(-3,3).
      \]
      
      \medskip
      \noindent
      \textbf{Step 1.  Enumerate the rationals in $[0,1]$.}  
      Take
      \[
        r_{1}=0,\; r_{2}=1,\; r_{3}=\tfrac12,\; r_{4}=\tfrac13,\; r_{5}=\tfrac23,\;\dotsc
      \]
      (any bijective listing of $\mathbb{Q}\cap(0,1)$ after $r_{1},r_{2}$ will do).
      
      \medskip
      \noindent
      \textbf{Step 2.  Define the open sets.}  
      For every rational $r\in[0,1]$ put
      \[
         V_{r}:=\bigl(-2+r,\;2-r\bigr)\subset\mathbb{R}.
      \]
      
      \medskip
      \noindent
      \textbf{Step 3.  Verify the required properties.}
      \begin{enumerate}
          \item \emph{$K\subset V_{1}$.}  
                Indeed, $V_{1}=(-1,1)$, and $[-\tfrac12,\tfrac12]\subset(-1,1)$.
      
          \item \emph{$\overline{V}_{0}\subset V$.}  
                We have $V_{0}=(-2,2)$, so $\overline{V}_{0}=[-2,2]\subset(-3,3)=V$.
      
          \item \emph{Each $\overline{V}_{r}$ is compact.}  
                Every $\overline{V}_{r}=[-2+r,\,2-r]$ is closed and bounded in $\mathbb{R}$.
      
          \item \emph{Nested‐closure condition:  
                if $s>r$ then $\overline{V}_{s}\subset V_{r}$.}  
                When $s>r$,
                \[
                     -2+r \;<\; -2+s 
                     \quad\text{and}\quad 
                     2-s \;<\; 2-r,
                \]
                hence
                \[
                  \overline{V}_{s}
                    =[-2+s,\,2-s]
                    \subset(-2+r,\,2-r)
                    =V_{r}.
                \]
      \end{enumerate}
      
      \medskip
      \noindent
      \textbf{Concrete first few sets ($n\ge 2$).}
      \[
      \begin{aligned}
        V_{r_{1}} &=V_{0}=(-2,2),\\
        V_{r_{2}} &=V_{1}=(-1,1),\\[2pt]
        V_{r_{3}} &=V_{1/2}=(-1.5,1.5),\\
        V_{r_{4}} &=V_{1/3}=(-1.\overline{6},\,1.\overline{6}),\\
        V_{r_{5}} &=V_{2/3}=(-1.\overline{3},\,1.\overline{3}),\ \text{etc.}
      \end{aligned}
      \]
      
      This sequence $\{V_{r}\}$ illustrates exactly the situation Rudin constructs:
      \[
         K\subset V_{1}\subset\overline{V}_{1}\subset V_{0}\subset\overline{V}_{0}\subset V,
         \qquad
         s>r\;\Longrightarrow\; \overline{V}_{s}\subset V_{r}.
      \]
      \end{example}
      \pagebreak
      \begin{remark}
        The adjective \emph{lower} in “lower–semicontinuous’’ refers to the fact
        that the function is “continuous from \emph{below}’’: it cannot jump
        \emph{downward} at a point, whereas it \emph{may} jump upward.
        
        \bigskip
        \textbf{Equivalent characterisations}
        
        For a real-valued function $h:X\to\mathbb{R}$ the following are
        equivalent.
        
        \begin{enumerate}
            \item For every $x_{0}\in X$
                  \[
                     h(x_{0})
                     \;\le\;
                     \liminf_{x\to x_{0}}h(x)
                     \;=\;
                     \lim_{\varepsilon\downarrow 0}\;
                     \bigl(\inf_{d(x,x_{0})<\varepsilon}h(x)\bigr).
                  \tag{1}
                  \]
        
            \item For every $\alpha\in\mathbb{R}$ the
                  \emph{super-level set}
                  \[
                      \{x\in X : h(x)>\alpha\}
                  \]
                  is open (equivalently,
                  $\{x\in X : h(x)\ge\alpha\}$ is closed).              
        
            \item The \emph{epigraph}
                  \[
                    \operatorname{epi}(h)
                    :=
                    \{(x,t)\in X\times\mathbb{R}: t\ge h(x)\}
                  \]
                  is a closed subset of $X\times\mathbb{R}$.            
        \end{enumerate}
        
        Any of these three statements defines \textbf{lower semicontinuity}.
        Statement (2) is the one used in Rudin, hence the appearance of the set
        $\{h>\alpha\}$.
        
        \bigskip
        \textbf{Why does (2) capture “continuity from below’’?}
        
        Fix $x_{0}$ and let $\alpha:=h(x_{0})-\varepsilon$ for some
        $\varepsilon>0$.  
        Because $\{h>\alpha\}$ is open and contains $x_{0}$, there is a small
        neighbourhood $U$ of $x_{0}$ such that $h(x)>\alpha$ for every
        $x\in U$.  In particular,
        \[
           h(x) > h(x_{0})-\varepsilon
           \quad\text{for all }x\text{ near }x_{0}.
        \]
        Since $\varepsilon$ is arbitrary, the values of $h$ near $x_{0}$ can be
        forced to lie \emph{arbitrarily close to or above} $h(x_{0})$, but they
        are \emph{not} allowed to dip significantly below $h(x_{0})$.  Thus the
        function is protected against downward jumps, matching the inequality in
        (1).
        
        \bigskip
        \textbf{Dual notion}
        
        If we reverse the inequalities we obtain \emph{upper semicontinuity}:
        $h$ is u.s.c.\ iff $\{h<\alpha\}$ is open for every $\alpha$, i.e.\ the
        function cannot jump \emph{upwards}.  Hence the terms “lower’’ and
        “upper’’ indicate which side of the value $h(x_{0})$ is
        \emph{semicontinuously} controlled.
        \end{remark}
        \pagebreak
        \begin{remark}[Why $\operatorname{supp}f\subset\overline{V_{0}}$]
          Recall the definitions used in the proof.
          
          \begin{itemize}
              \item For every rational $r\in(0,1]$ an open set $V_{r}$ is chosen so that  
                    $s>r\;\Longrightarrow\;\overline{V}_{s}\subset V_{r}$ and, in particular,
                    \[
                        V_{r}\subset V_{0}\quad\text{for all }r>0.
                    \]
          
              \item For each such $r$ define
                    \[
                       f_{r}(x)=
                       \begin{cases}
                          r & \text{if }x\in V_{r},\\[4pt]
                          0 & \text{otherwise.}
                       \end{cases}
                    \]
              \item Set $f(x)=\displaystyle\sup_{r>0}f_{r}(x)$.
          \end{itemize}
          
          \medskip
          \noindent
          \textbf{Step 1.  Locate the points where $f$ can be non–zero.}
          
          Because every $f_{r}$ is identically $0$ outside $V_{r}$, we have
          \[
             f(x)\neq 0
             \;\;\Longrightarrow\;\;
             f_{r}(x)\neq 0\text{ for some }r
             \;\;\Longrightarrow\;\;
             x\in V_{r}\subset V_{0}.
          \]
          Thus
          \[
             \bigl\{x\in X:f(x)\neq 0\bigr\}\subset V_{0}.
          \]
          
          \medskip
          \noindent
          \textbf{Step 2.  Take the closure.}
          
          The \emph{support} of $f$ is
          \[
             \operatorname{supp}f
             :=\overline{\{x:f(x)\neq 0\}}.
          \]
          Since the set inside the closure is already contained in $V_{0}$, its
          closure must lie inside $\overline{V_{0}}$:
          \[
             \operatorname{supp}f
             \subset \overline{V_{0}}.
          \]
          
          \bigskip
          \noindent
          Hence it is “clear’’—in the sense of an immediate consequence of the
          definitions—that $f$ vanishes outside $V_{0}$ and therefore
          $\operatorname{supp}f\subset\overline{V_{0}}$.
          \end{remark}
\end{document}
