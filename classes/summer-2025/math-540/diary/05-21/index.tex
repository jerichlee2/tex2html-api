\documentclass[12pt]{article}

% Packages
\usepackage[margin=1in]{geometry}
\usepackage{amsmath,amssymb,amsthm}
\usepackage{enumitem}
\usepackage{hyperref}
\usepackage{xcolor}
\usepackage{import}
\usepackage{xifthen}
\usepackage{pdfpages}
\usepackage{transparent}
\usepackage{listings}
\usepackage{tikz}
\usepackage{physics}
\usepackage{siunitx}
\usepackage{booktabs}
\usepackage{cancel}
  \usetikzlibrary{calc,patterns,arrows.meta,decorations.markings}


\DeclareMathOperator{\Log}{Log}
\DeclareMathOperator{\Arg}{Arg}

\lstset{
    breaklines=true,         % Enable line wrapping
    breakatwhitespace=false, % Wrap lines even if there's no whitespace
    basicstyle=\ttfamily,    % Use monospaced font
    frame=single,            % Add a frame around the code
    columns=fullflexible,    % Better handling of variable-width fonts
}

\newcommand{\incfig}[1]{%
    \def\svgwidth{\columnwidth}
    \import{./Figures/}{#1.pdf_tex}
}
\theoremstyle{definition} % This style uses normal (non-italicized) text
\newtheorem{solution}{Solution}
\newtheorem{proposition}{Proposition}
\newtheorem{problem}{Problem}
\newtheorem{lemma}{Lemma}
\newtheorem{theorem}{Theorem}
\newtheorem{remark}{Remark}
\newtheorem{note}{Note}
\newtheorem{definition}{Definition}
\newtheorem{example}{Example}
\newtheorem{corollary}{Corollary}
\theoremstyle{plain} % Restore the default style for other theorem environments
%

% Theorem-like environments
% Title information
\title{}
\author{Jerich Lee}
\date{\today}

\begin{document}

\maketitle
% Concrete example of a simple (non-negative) function expressed with characteristic functions

\begin{enumerate}
  \item Let 
  \[
      A_1=[0,1), 
      \qquad 
      A_2=[1,2)
  \]
  be measurable subsets of $\mathbb{R}$ (endowed with the Lebesgue $\sigma$-algebra).

  \item Define the function $s:\mathbb{R}\to[0,\infty)$ by
  \begin{align}
      s(x)=
      \begin{cases}
          3 & \text{if } x\in A_1,\\[6pt]
          1 & \text{if } x\in A_2,\\[6pt]
          0 & \text{otherwise.}
      \end{cases}
  \end{align}

  \item Using characteristic functions $\chi_{A_i}$ we can write
  \begin{align}
      s(x)=3\,\chi_{A_1}(x)+1\,\chi_{A_2}(x),
  \end{align}
  which is precisely of the form $\displaystyle s=\sum_{i=1}^{2}\alpha_i\chi_{A_i}$ with
  $\alpha_1=3$, $\alpha_2=1$.
\end{enumerate}
% Concrete illustration of Definition 1.23
\pagebreak
\begin{enumerate}
  \item \textbf{Measure space.}  
        Take the Lebesgue measure space $(X,\mathcal{M},\mu)
        =(\mathbb{R},\mathcal{B}(\mathbb{R}),\lambda)$,  
        where $\lambda$ denotes Lebesgue measure.

  \item \textbf{A simple function.}  
        Define
        \[
            s(x)=3\,\chi_{[0,1)}(x)+1\,\chi_{[1,2)}(x),
        \]
        so the distinct values are $\alpha_1=3$ on
        $A_1=[0,1)$ and $\alpha_2=1$ on $A_2=[1,2)$.

  \item \textbf{Set of integration.}  
        Fix the measurable set
        \[
            E=[0.5,1.5]\subset X.
        \]

  \item \textbf{Integral of the simple function over $E$.}  
        By Definition 1.23,
        \begin{align}
            \int_E s\,d\mu
            &=\sum_{i=1}^{2} \alpha_i\,\mu(A_i\cap E) \\[4pt]
            &=3\,\mu\bigl([0,1)\cap[0.5,1.5]\bigr)
              +1\,\mu\bigl([1,2)\cap[0.5,1.5]\bigr) \\[4pt]
            &=3\,\mu([0.5,1)) + 1\,\mu([1,1.5]) \\[4pt]
            &=3\,(1-0.5) + 1\,(1.5-1) \\[4pt]
            &=3\cdot 0.5 + 1\cdot 0.5 = 1.5 + 0.5 = 2.
        \end{align}

        Hence \(\displaystyle \int_E s\,d\mu = 2\).
\end{enumerate}

% ------------------------------------------------------------------
% Optional: showing how Definition (3) works for a non-simple f
% ------------------------------------------------------------------
\medskip
\noindent
\textbf{Bonus: approximating an ordinary (non-simple) positive function.}

Let
\[
  f(x)=
  \begin{cases}
      x, & 0\le x\le 1,\\
      0, & \text{otherwise}.
  \end{cases}
\]
For each $n\in\mathbb{N}$ define the simple function
\[
  s_n(x)=\sum_{k=1}^{n}
          \frac{k-1}{n}\;
          \chi_{\,[(k-1)/n,\,k/n)}(x).
\]
Then \(0\le s_n\le f\) and
\[
  \int_{\mathbb{R}} s_n\,d\mu
    =\sum_{k=1}^{n}\frac{k-1}{n}\cdot\frac{1}{n}
    =\frac{n-1}{2n}.
\]
As \(n\to\infty\) we have \(\frac{n-1}{2n}\to\frac12\), so by (3)
\[
  \int_{\mathbb{R}} f\,d\mu
  =\sup_{\,0\le s\le f}\int_{\mathbb{R}} s\,d\mu
  =\frac12,
\]
which matches the classical result \(\displaystyle\int_0^1 x\,dx=\tfrac12\).
\pagebreak
%------------------------------------------------------------
%  Why every (positive) measure satisfies  $\mu(\varnothing)=0$
%------------------------------------------------------------

\begin{problem}
  Prove that if $\mu$ is a positive measure on a $\sigma$--algebra $\mathcal{M}$, then 
  $
      \mu(\varnothing)=0.
  $
  \begin{enumerate}
      \item[\textbf{Step 1.}] 
            Choose any set $A\in\mathcal{M}$ with a \emph{finite} measure, i.e.\ $0\le \mu(A)<\infty$.  
            This is always possible because, by definition of a measure, $\mu$ takes only non–negative values and 
            there is at least one measurable set with finite measure (for instance, the empty union of countably
            many measurable sets of finite measure, or a bounded interval under Lebesgue measure).

      \item[\textbf{Step 2.}] 
            Construct a sequence of pairwise–disjoint measurable sets
            \begin{align}
                A_1 &= A, &
                A_2 &= A_3 = A_4 = \dots = \varnothing .
            \end{align}
            Notice that the sets $\{A_k\}_{k\ge1}$ are disjoint and
            their union is exactly $A$:
            \begin{align}
                \bigcup_{k=1}^{\infty} A_k 
                = A_1 \cup \bigl(\bigcup_{k\ge2}\varnothing\bigr)
                = A.
            \end{align}

      \item[\textbf{Step 3.}] 
            Apply the \emph{countable additivity} axiom of a measure (cf.\ Theorem~1.18(1)):
            \begin{align}
                \mu(A)
                &= \mu\!\Bigl(\bigcup_{k=1}^{\infty} A_k\Bigr)
                = \sum_{k=1}^{\infty} \mu(A_k) \\
                &= \mu(A_1) + \sum_{k=2}^{\infty} \mu(\varnothing) 
                = \mu(A) + \underbrace{\mu(\varnothing)+\mu(\varnothing)+\dots}_{\text{countably many terms}} .
            \end{align}

      \item[\textbf{Step 4.}] 
            Isolate the infinite tail:
            \begin{align}
                0 
                = \mu(A) - \mu(A) 
                = \sum_{k=2}^{\infty} \mu(\varnothing)
                = \mu(\varnothing)\,\sum_{k=2}^{\infty} 1 .
            \end{align}
            The geometric series $\sum_{k=2}^{\infty} 1$ diverges to $+\infty$.  
            Because the left–hand side is $0$ (a finite number) and the measure $\mu$ is \emph{non–negative},
            the only way this equality can hold is if
            \begin{align}
                \mu(\varnothing)=0.
            \end{align}
  \end{enumerate}
  Hence every positive measure assigns measure $0$ to the empty set.
\end{problem}
\pagebreak
\begin{problem}
  \textbf{Relationship between the motivational paragraph and Theorem 1.36}

  \begin{enumerate}

      \item \textbf{Informal motivation}  
            \begin{enumerate}
                \item Sets with measure $0$ are ``negligible'' for integration.
                \item Intuitively, \emph{every} subset of a negligible set should also be negligible.
                \item If $N\in\mathcal{M}$ with $\mu(N)=0$ and $E\subset N$ but $E\notin\mathcal{M}$, we would like to declare $\mu(E)=0$ anyway.
                \item Question: After adding all such subsets, do we still have a $\sigma$--algebra, and can $\mu$ be extended to those new sets while remaining a measure?
            \end{enumerate}

      \item \textbf{Formal realisation (Theorem 1.36)}  
            Let $(X,\mathcal{M},\mu)$ be a measure space and define
            \begin{align}
                \mathcal{M}^{*}
                &=
                \bigl\{E\subset X \;\big|\; \exists\,A,B\in\mathcal{M}\text{ such that }A\subset E\subset B
                     \text{ and }\mu(B\setminus A)=0\bigr\}, \\[6pt]
                \mu(E) &:= \mu(A)\quad\text{for any such }A .
            \end{align}
            Then:
            \begin{enumerate}
                \item $\mathcal{M}^{*}$ is a $\sigma$--algebra containing $\mathcal{M}$.
                \item $\mu$ extends to a measure on $\mathcal{M}^{*}$ via the above definition.
                \item In particular, if $E\subset N$ with $\mu(N)=0$, then choosing $A=\varnothing$ and $B=N$ gives $E\in\mathcal{M}^{*}$ and $\mu(E)=0$, fulfilling the original intuition.
            \end{enumerate}

      \item \textbf{How the pieces fit}  

% --- Place this snippet where you want the comparison table to appear ---
% Requires: \usepackage{booktabs,adjustbox}  (add to your preamble).

\begin{center}
  \begin{tabular}{@{}p{.45\textwidth} p{.45\textwidth}@{}}
  \toprule
  \textbf{Motivational paragraph} & \textbf{Theorem 1.36} \\ \midrule
  Every subset of a null set should be null. 
  & $E\subset N,\; \mu(N)=0 \;\Longrightarrow\; E\in\mathcal{M}^{*},\; \mu(E)=0$ \\[4pt]
  Will the extension still be a measure on a $\sigma$--algebra? 
  & $\mathcal{M}^{*}$ is a $\sigma$--algebra and $\mu$ extends to it \\ \bottomrule
  \end{tabular}
  \end{center}

            Thus, Theorem 1.36 is the rigorous theorem that executes the informal plan described in the preceding paragraph; it is commonly called the \emph{completion of a measure space}.
  \end{enumerate}
\end{problem}
\end{document}
