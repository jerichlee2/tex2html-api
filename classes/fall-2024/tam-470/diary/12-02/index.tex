\documentclass[12pt]{article}

% Packages
\usepackage[margin=1in]{geometry}
\usepackage{amsmath,amssymb,amsthm}
\usepackage{enumitem}
\usepackage{hyperref}
\usepackage{xcolor}
\usepackage{import}
\usepackage{xifthen}
\usepackage{pdfpages}
\usepackage{transparent}
\usepackage{listings}


\lstset{
    breaklines=true,         % Enable line wrapping
    breakatwhitespace=false, % Wrap lines even if there's no whitespace
    basicstyle=\ttfamily,    % Use monospaced font
    frame=single,            % Add a frame around the code
    columns=fullflexible,    % Better handling of variable-width fonts
}

\newcommand{\incfig}[1]{%
    \def\svgwidth{\columnwidth}
    \import{./Figures/}{#1.pdf_tex}
}
\theoremstyle{definition} % This style uses normal (non-italicized) text
\newtheorem{solution}{Solution}
\newtheorem*{proposition}{Proposition}
\newtheorem{problem}{Problem}
\newtheorem{lemma}{Lemma}
\theoremstyle{plain} % Restore the default style for other theorem environments
%

% Theorem-like environments
% Title information
\title{Spectral Methods}
\author{Jerich Lee}
\date{2024-12-02}

\begin{document}

\maketitle
\section{What?}
\begin{enumerate}
    \item high degree of accuracy
    \item combo of sines + cosines
    \item complex functions
    \item solving odes + pdes
    \item fluid dynamics + turbulence modeling :‑O
\end{enumerate}
\subsection{Fourier Series (Continuous)}
\begin{align}
    f(x)= \sum_{k=-\infty}^{\infty} \hat{f}_{k}e^{ikx}  \\[10pt] 
    \text{where } e^{ikx} = \cos kx + i\sin kx 
\end{align}
$k$ is the wave number such that $k\in \mathbb{\MakeUppercase{n}} $
The derivative:
\begin{align}
    f^{^\prime}(x)= \sum_{k=-\infty }^{\infty }ik \hat{f_k}e^{ikx} 
\end{align}
\subsection{Discrete Fourier Series}
$N-1$ points vs. $N$ points bc the end point is just $f_0$! bc of periodicity 
\begin{align}
    f_j = \sum_{k=1}{-\frac{N}{2}}^{\frac{N}{2}} \hat{f_k} e^{ikx_j}, \quad j=0,1,2, \ldots , N-1 
\end{align}
\textbf{N is even}
\subsection{Discrete Orthogonality Property}
\begin{align}
    \sum_{j=0}^{N-1} e^{ikx_j}e^{-ik^{^\prime}x_j}= 
    \begin{cases}
        N, &\text{ if } k=k^\prime +mN, m=0,\pm 1, \pm 2, \ldots   ;\\
        0, &\text{ if } else .
    \end{cases}
\end{align}
\subsection{Discrete Fourier Transform}
\begin{align}
    \hat{f_k} \frac{1}{N}\sum_{i=0}^{N-1} f_j e^{-\left( ikx \right) }
\end{align}
\end{document}
