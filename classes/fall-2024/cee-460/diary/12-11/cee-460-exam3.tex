\documentclass[12pt]{article}
\usepackage{amsmath}
\usepackage{amssymb}
\usepackage{geometry}
\geometry{margin=1in}

\begin{document}

\section*{Problem 1 (50 points):}
The moment-unbraced length curve for a beam section is shown in Figure 1.

\begin{enumerate}
    \item Assuming the bending coefficient \(c_b = 1.0\), it is required to:
    \begin{enumerate}
        \item Determine:
        \begin{enumerate}
            \item[i)] The plastic moment \(\phi M_p\),
            \item[ii)] The maximum unbraced length at which lateral torsional buckling does not control \(L_p\),
            \item[iii)] The unbraced length marking the transition between elastic and inelastic buckling \(L_r\).
        \end{enumerate}
        \item Find the moment capacity of the cross-section if:
        \begin{enumerate}
            \item \(L_u = 12'\),
            \item \(L_u = 24'\),
            \item \(L_u = 42'\).
        \end{enumerate}
        State which flexural limit state is governing in each case.
        \item Determine the maximum allowable value for the unbraced length that the beam compression flange may have if the beam is to resist a factored moment of 600 ft-kips.
    \end{enumerate}
    
    Following a careful computation, the designer was able to determine a new value for the bending coefficient: \(c_b = 1.5\).
    \begin{enumerate}
        \setcounter{enumii}{3}
        \item Draw the modified moment-unbraced length curve corresponding to this new value of the bending coefficient. Find the new values of \(L_p\) and \(L_r\).
        \item Using the modified curve from (d), find the moment capacity of the cross-section if:
        \begin{enumerate}
            \item \(L_u = 12'\),
            \item \(L_u = 24'\),
            \item \(L_u = 42'\).
        \end{enumerate}
        Briefly comment on the results comparing them to your findings from part (b).
    \end{enumerate}
\end{enumerate}

\end{document}