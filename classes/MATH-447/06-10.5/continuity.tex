\documentclass[12pt]{article}

% Packages
\usepackage[margin=1in]{geometry}
\usepackage{amsmath,amssymb,amsthm}
\usepackage{enumitem}
\usepackage{hyperref}
\usepackage{xcolor}

% Define the solution environment with normal text
\theoremstyle{definition} % This style uses normal (non-italicized) text
\newtheorem{solution}{Solution}
\newtheorem*{proposition}{Proposition}
\newtheorem{problem}{Problem}
\newtheorem{lemma}{Lemma}
\theoremstyle{plain} % Restore the default style for other theorem environments
%


% Title information
\title{Continuity}
\author{Jerich Lee}
\date{\today}

\begin{document}

\maketitle
\begin{problem}[18.7]
  Prove $xe^{x} =2$ for some $x$ in $(0,1)$.  
\end{problem}
\begin{solution}
  \begin{proof}
    Our goal is to show that $f=x$ and $g=e^{x}$ are both continuous, then invoke Theorem 17.4 part iii): Let $f$ and $g$ be real-valued functions that are continuous at $x_0$ in $\mathbb{R} $. Then: 
    \begin{align}
     fg \text{ is continuous at } x_0 
    \end{align}
    To show that $f=x$ is continuous, we can show the following:
    \begin{align}
     \left\vert x_n -x\right\vert <\varepsilon
    \end{align}
    Choosing $\delta =\epsilon $, we get:
    \begin{align}
     \left\vert x_{n} -x \right\vert <\delta \implies \left\vert f(x_n)-f(x) \right\vert <\epsilon
    \end{align} 
    We wish to prove that the function $ f(x) = e^x $ is continuous.
  
    Let $ \epsilon > 0 $. We need to show that for every $ \epsilon > 0 $, there exists $ \delta > 0 $ such that if $ |x - x_0| < \delta $, then $ |e^x - e^{x_0}| < \epsilon $.
    
    Starting with the expression:
    \begin{align}
    |e^x - e^{x_0}| &= e^{x_0} |e^{x - x_0} - 1|
    \end{align}
    Now, we use the elementary inequality:
    \begin{align}
    e^y &\geq 1 + y
    \end{align}
    For $ y = -y $, we get:
    \begin{align}
    e^{-y} &\geq 1 - y
    \end{align}
    which implies:
    \begin{align}
    \frac{1}{1 - y} &\geq e^y \quad \text{for } y < 1
    \end{align}
    Hence:
    \begin{align}
    |e^y - 1| &\leq \max \left\{ |y|, \left| \frac{y}{1 - y} \right| \right\} \quad \text{for } y < 1
    \end{align}
    
    Thus, taking $ y = x - x_0 $, we have:
    \begin{align}
    |e^y - 1| &\leq \max \left\{ |y|, \left| \frac{y}{1 - y} \right| \right\}
    \end{align}
    
    Now, choose $ \delta $ small enough such that $ |y| = |x - x_0| < \delta $ satisfies:
    \begin{align}
    \max \left\{ |y|, \left| \frac{y}{1 - y} \right| \right\} &< e^{-x_0} \epsilon
    \end{align}
    and $ |y| < 1 $.
    
    Therefore, we can conclude that $ e^x $ is continuous at $ x_0 $.
    \vspace{.5cm} 
    Invoking Theorem 17.4 iii), we show that $xe^{x}$ is a continuous function. To show that there exists $x$ such that $h=xe^{x}=2$, we can show that pick two points in $(0,1)$ : $x_1=0.01$, and $x_2=0.99$, and substitute these into $h$, getting $h(x_1)=0.01$ and $f(x_2)=2.66$. $h(x_1)<2<h(x_2)$, and because $h$ is continuous on $(0,1)$, we can invoke the IVT, which proves that there exists $x$ such that $xe^{x}=2$.  
  
  \end{proof}
\end{solution}
\begin{problem}[21.2]
    Consider $f:S\to S^{*}$ where $S,d$ and $S^{*}, d^{*}$ are metric spaces. Show that $f$ is continuous at $s_{0}\in S $ if and only if for every open set $U$ in $S^{*} $ containing $f(s_{0} )$, there is an open set $V$ in $S$ containing $s_{0}$ such that $f(V) \subseteq U$.
\end{problem}
\begin{solution}
\begin{proof}
  $\implies:$ The goal is to show that every point in $V$ has a neighborhood, i.e., is open. Because $U$ is open, we know that there exists $r=\varepsilon$ for each point $y\in U$. Because $f$ is continuous, we also know that there is a corresponding $\delta>0$ such that $s\in B_{\delta}(x)=V$, so this implies that there exists an open ball $V$ around each point $s$ in $S$. Because $d(s,s_0)<\delta \implies d(f(s),f(s))<\varepsilon$, and $\delta = V$, $\varepsilon \subseteq U$, as there may be other points not in $\delta$ that map into $U$, this implies $f(V)\subseteq U$. 
  \vspace{.5cm}

  $\impliedby:$ We know that $V$ is open for every $U$ in $S^{*}$. Choose $\varepsilon \in U \ s.t. \ \varepsilon >0$. Then choose $\delta \in V \ s.t. \ \delta >0$. By implication, we know that $\delta$ is open. Then, we can state that $d(s, s_0)<\delta \implies d(f(s),f(s_0))<\varepsilon$, which is the definition of continuity at a point.    
\end{proof}
\end{solution}
\begin{problem}[21.3]
    Let $(S,d)$ be a metric space and choose $s_{0} \in S$. Show $f(s)=d(s,s_{0})$ defines a uniformly continuous real-valued function $f$ on $S$.
\end{problem}
\begin{solution}
\begin{proof}
  We to show the definition of continuity:
  \begin{align}
    \forall \epsilon > 0, \exists \delta > 0 \text{ such that } d_Y(f(p), f(q)) &< \epsilon \\
    \forall p, q \in X \text{ for which } d_X(p, q) &< \delta. 
    \end{align}
    We know that:
    \begin{align}
      f(s_0) = d(s_0, s_0) = 0 \\[10pt] 
      f(p)= d(p, s_0) \\[10pt] 
      f(q)= d(q, s_0) \\[10pt] 
      d_Y(f(p),f(q))\leq d(f(p)) + d(f(q))\\[10pt] 
    \end{align}
We want to show using the triangle inequality:
\begin{align} 
      d(f(p),f(q))\leq d(f(p),f(s_0))+d(f(q),f(s_0))< \varepsilon
\end{align}
From Eqn 13, 14, and 15, we can say: 
\begin{align}
  d(f(p),f(q))\leq d(p,s_0)+d(q,s_0).
\end{align}
Choosing $p,q$ such that $d(p,s_0)<\frac{\varepsilon}{2}$ and $d(q,s_0)<\varepsilon$ , we can pick $\delta =\varepsilon$ and use Eqn 14 and 15 to show:
\begin{align}
  d(f(p),f(q))\leq d(f(p))+d(f(q))\\[10pt] 
  \leq d(p,s_0)+d(q,s_0)<\varepsilon
\end{align} 
\end{proof}
\end{solution}
\begin{problem}[21.4]
    Consider $f:S\to \mathbb{R} $ where $(S,d)$ is a metric space. Show the following are equivalent:
    \begin{enumerate}
        \item $f$ is continuous;
        \item $f^{-1}((a,b)) $ is open in $S$ for all $a<b$;
        \item $f^{-1}((a,b))$ is open in $S$ for all rational $a<b$.
    \end{enumerate}
\end{problem}
\begin{solution}
  \begin{proof}  
    $1\implies 2$: We know that any open interval $(a,b)$ in $R^{1}$ is complete, i.e., every subsequence converges to a limit point contained in $\mathbb{R}$. Therefore, $(a,b)$ is an open set in $\mathbb{R}$. We know from Problem 2 that if $f$ is continuous, then every open set in the range corresponds to an open set in the domain. $(a,b)$ is open, so $f^{-1}(a,b)$ is also open.    
    \end{proof}
\begin{proof}  
$1\implies 3$: By $1\implies 2$, we know that if $a,b\in \mathbb{Q} $, then $(a,b)$ is open by the denseness of the rationals, i.e., between every real there exists a rational.  
\end{proof}
\begin{proof}  
$3\implies 1$: We achieve this by taking the converse of $1\implies 3$, which exists by the bijection of $1\implies 2$.
\end{proof}

\begin{proof}  
$2\implies 1$: This exists by the bijection of $1\implies 2$.  
\end{proof}
\begin{proof}  
$3\implies 2$: This is always true by the denseness of the rationals.
\end{proof}
\begin{proof}  
$2\implies 3$: This is always true, as $\mathbb{Q} \subset \mathbb{R}$ 
\end{proof}
\end{solution}
\begin{problem}[21.5]
  Let $E$ be a noncompact subset of $\mathbb{R}^{k}$.
  \begin{enumerate}
    \item Show there is an unbounded continuous real-valued function on $E$. \emph{Hint}: Either $E$ is unbounded or else its closure $E^{-}$ contains $\mathbf{x}_{0} \notin E$. In the latter case, use $\frac{1}{g}$ where $g(\mathbf{x}) =d(\mathbf{x},\mathbf{x}_{0})$.
    \item Show there is a bounded continuous real-valued function on $E$ that does not assume its maximum on $E$.
  \end{enumerate}
\end{problem}
\begin{solution}
 \begin{enumerate}
  \item \begin{enumerate}
    \item Suppose that $E$ is bounded, and $x_0$ is not a point of $E$. To show that there exists a continuous unbounded real-valued function on $E$, consider the function:
    \begin{align}
     f(x)=\frac{1}{x-x_0}
    \end{align} 
    This function is continuous on $E$.
    \item Suppose that $E$ is unbounded. Then, $f(x)=x$ is unbounded and is a continuous real-valued function on $E$. 
  \end{enumerate}
  \item Suppose that $E$ is bounded. Then the following:
  \begin{align}
    g(x)=\frac{1}{1+(x-x_0)^{2}}
  \end{align}
$g(x)$ is bounded, as $0<g(x)<1$ for all $x$. $g(x)$ has no maximal element, as $x_0$ is not a member of $E$. 
 \end{enumerate} 
\end{solution}
\begin{problem}[21.10(d)]
    Explain why there are no continuous functions mapping $[0,1]$ onto $(0,1)$ or $\mathbb{R}$.
\end{problem}
\begin{solution}
  We know that $[0,1]$ is compact according to the according to the Heine-Borel Theorem, as it is closed and bounded. According to Theorem 21.4, if $E$ is compact, then $f(E)$ is compact. $(0,1)$ and $\mathbb{R}$ are not compact, so this contradicts Theorem 21.4..
\end{solution}
\begin{problem}
  Is it true that any bounded continuous function on $\mathbb{R}$ is uniformly continuous? 
\end{problem}
\begin{solution}
  No. Choose $f=\frac{1}{x-x_0}$ with $x\in (0,x_0)$. Uniform continuity states that $\forall \varepsilon, \exists \delta \ s.t. \ d(p,q)<\delta \implies d(f(p),f(q))$ for all $p, q$. Choose an arbitrary $\varepsilon$. Then, there exists a corresponding $\delta$ such that $d(x,x_0)<\delta\implies d(f(x),f(x_0))<\varepsilon$. But as we take the same $\delta$ about $x$ closer and closer to $x_0$, $d(f(x),f(x_n))$ will grow larger and larger, eventually exceeding our chosen $\varepsilon$. Therefore, there exists no constant $\delta$ that satisfies our chosen $\varepsilon$. Therefore, our bounded, continuous function $f$ is not uniformly continuous.    
\end{solution}
\end{document}
