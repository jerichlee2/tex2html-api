\documentclass[12pt]{article}

% Packages
\usepackage[margin=1in]{geometry}
\usepackage{amsmath,amssymb,amsthm}
\usepackage{enumitem}
\usepackage{hyperref}
\usepackage{xcolor}
\usepackage{import}
\usepackage{xifthen}
\usepackage{pdfpages}
\usepackage{transparent}
\usepackage{listings}
\usepackage{tikz}
\usepackage{physics}
\usepackage{siunitx}
\usepackage{booktabs}
\usepackage{cancel}
  \usetikzlibrary{calc,patterns,arrows.meta,decorations.markings}


\DeclareMathOperator{\Log}{Log}
\DeclareMathOperator{\Arg}{Arg}

\lstset{
    breaklines=true,         % Enable line wrapping
    breakatwhitespace=false, % Wrap lines even if there's no whitespace
    basicstyle=\ttfamily,    % Use monospaced font
    frame=single,            % Add a frame around the code
    columns=fullflexible,    % Better handling of variable-width fonts
}

\newcommand{\incfig}[1]{%
    \def\svgwidth{\columnwidth}
    \import{./Figures/}{#1.pdf_tex}
}
\theoremstyle{definition} % This style uses normal (non-italicized) text
\newtheorem{solution}{Solution}
\newtheorem{proposition}{Proposition}
\newtheorem{problem}{Problem}
\newtheorem{lemma}{Lemma}
\newtheorem{theorem}{Theorem}
\newtheorem{remark}{Remark}
\newtheorem{note}{Note}
\newtheorem{definition}{Definition}
\newtheorem{example}{Example}
\newtheorem{corollary}{Corollary}
\theoremstyle{plain} % Restore the default style for other theorem environments
%

% Theorem-like environments
% Title information
\title{MATH 432 Practice Final Exam 3}
\author{Jerich Lee}
\date{\today}

\begin{document}

\maketitle
%%%%%%%%%%%%%%%%%%%%%%%%%%%%%%%%%%%%%%%%%%%%%%%%%%%%%%%%%%%%%%%%%%%%%%%%%%%%
%                MATH 432 – Practice Final Examination (C)                %
%                (Eight problems; give complete proofs.)                  %
%%%%%%%%%%%%%%%%%%%%%%%%%%%%%%%%%%%%%%%%%%%%%%%%%%%%%%%%%%%%%%%%%%%%%%%%%%%%


\begin{problem}[All partial orders on a four-element set]
  Let $X=\{1,2,3,4\}$.
  \begin{enumerate}[label=(\alph*)]
    \item Describe \emph{all} distinct partial orders that can be placed
          on $X$ (up to isomorphism).  
    \item How many non-isomorphic posets of size~$4$ are there?
          Justify your count.
  \end{enumerate}
\end{problem}

\begin{problem}[Compositions preserving injectivity]
  Let $f:A\to B$ and $g:B\to C$ be functions.
  \begin{enumerate}[label=(\alph*)]
    \item Show that if both $f$ and $g$ are one-to-one
          (injective), then $g\circ f$ is injective.
    \item Show that if $g\circ f$ is injective, then $f$ must be
          injective.
    \item Prove: if $f$ is \emph{onto} (surjective) and
          $g\circ f$ is injective, then $g$ is injective.
    \item Give an explicit example in which $g\circ f$ is injective
          while $g$ is \emph{not} injective.
  \end{enumerate}
\end{problem}

\begin{problem}[Images and pre-images under a function]
  Let $f:A\to B$ be any function.
  \begin{enumerate}[label=(\alph*)]
    \item For $B_{1},B_{2}\subset B$ prove that
          \[
              f^{-1}(B_{1}\cup B_{2})
              \;=\;
              f^{-1}(B_{1})\cup f^{-1}(B_{2}).
          \]
    \item For $T\subset B$ prove that
          \[
              f^{-1}(T^{\prime})
              \;=\;
              \bigl(f^{-1}(T)\bigr)^{\prime},
          \]
          where $T^{\prime}$ denotes the complement of $T$ in~$B$.
    \item Prove that for any $A_{1},A_{2}\subset A$,
          \[
              f(A_{1}\cup A_{2})
              \;=\;
              f(A_{1})\cup f(A_{2}).
          \]
    \item Construct counterexamples to show that, in general,
          \[
              f(A_{1}^{\prime})\subset (f(A_{1}))^{\prime}
              \quad\text{and}\quad
              f(A_{1}^{\prime})\supset (f(A_{1}))^{\prime}
          \]
          need \emph{not} hold.
  \end{enumerate}
\end{problem}

\begin{problem}[A power-cardinality identity]
  Let $e$ be an infinite cardinal and let
  $d$ be a cardinal satisfying $2\le d\le 2^{e}$.
  Prove that
  \[
      d^{\,e}=2^{e}.
  \]
  (Avoid citing Theorems 13–16 from the lecture notes;
  construct explicit injections or bijections instead.)
\end{problem}

\begin{problem}[Ordinal numbers and the axiom of choice]
  Suppose $A$ and $B$ are non-empty sets and
  $f:A\twoheadrightarrow B$ is a surjective map.
  Using the axiom of choice, prove that
  the \emph{ordinal number} of $B$ does not exceed that of $A$; i.e.
  \[
      \operatorname{o}(B)\le\operatorname{o}(A).
  \]
\end{problem}

\begin{problem}[Distance to a set is continuous]
  Let $(M,D)$ be a metric space and let $A\subset M$ be non-empty.
  Define, for each $x\in M$,
  \[
      f(x)
      \;:=\;
      D(x,A)
      \;:=\;
      \inf_{a\in A}D(x,a).
  \]
  Prove that the function $f:M\to\mathbb{R}$ is continuous.
\end{problem}

\begin{problem}[Separability passed to subspaces]
  Prove that every subspace of a separable metric space is separable.
  (Recall that a metric space is \emph{separable} if it contains a
  countable dense subset.)
\end{problem}

\begin{problem}[Infinite discrete subset in an infinite metric space]
  Let $(M,D)$ be an infinite metric space.
  Show that $M$ contains an \emph{infinite discrete subset};
  that is, there exists $A\subset M$ such that
  $A$ is infinite and each point of $A$ is isolated within~$A$.
  \emph{Hint:}  Separate the cases where $M$ is already discrete and
  where it has a limit point.
\end{problem}
\end{document}
