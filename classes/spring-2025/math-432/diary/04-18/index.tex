\documentclass[12pt]{article}

% Packages
\usepackage[margin=1in]{geometry}
\usepackage{amsmath,amssymb,amsthm}
\usepackage{enumitem}
\usepackage{hyperref}
\usepackage{xcolor}
\usepackage{import}
\usepackage{xifthen}
\usepackage{pdfpages}
\usepackage{transparent}
\usepackage{listings}
\usepackage{tikz}
  \usetikzlibrary{calc,patterns,arrows.meta,decorations.markings}


\DeclareMathOperator{\Log}{Log}
\DeclareMathOperator{\Arg}{Arg}

\lstset{
    breaklines=true,         % Enable line wrapping
    breakatwhitespace=false, % Wrap lines even if there's no whitespace
    basicstyle=\ttfamily,    % Use monospaced font
    frame=single,            % Add a frame around the code
    columns=fullflexible,    % Better handling of variable-width fonts
}

\newcommand{\incfig}[1]{%
    \def\svgwidth{\columnwidth}
    \import{./Figures/}{#1.pdf_tex}
}
\theoremstyle{definition} % This style uses normal (non-italicized) text
\newtheorem{solution}{Solution}
\newtheorem{proposition}{Proposition}
\newtheorem{problem}{Problem}
\newtheorem{lemma}{Lemma}
\newtheorem{theorem}{Theorem}
\newtheorem{remark}{Remark}
\newtheorem{note}{Note}
\newtheorem{definition}{Definition}
\newtheorem{example}{Example}
\newtheorem{corollary}{Corollary}
\theoremstyle{plain} % Restore the default style for other theorem environments
%

% Theorem-like environments
% Title information
\title{MATH 432: HW 9}
\author{Jerich Lee}
\date{\today}

\begin{document}

\maketitle
\begin{problem}[]
  
\end{problem}
\begin{solution}
  \begin{theorem}
    Let $(M,D)$ be a metric space and let $A\subseteq M$.  
    Define
    \[
    f:M\longrightarrow\mathbb{R},\qquad 
    f(x)\;:=\;D(x,A)\;:=\;\inf_{a\in A}D(x,a)\quad(x\in M).
    \]
    Then $f$ is continuous (in fact, $1$–Lipschitz).
    \end{theorem}
    
    \begin{proof}
    We prove the stronger statement that for all $x,y\in M$,
    \[
    \lvert f(x)-f(y)\rvert\;\le D(x,y).
    \tag{$\ast$}
    \]
    
    \begin{enumerate}
        \item Fix $x,y\in M$ and let $a\in A$ be arbitrary.  
              By the triangle inequality,
              \[
                  D(x,a)\;\le\;D(x,y)+D(y,a).
              \]
        \item Taking the infimum over $a\in A$ on the right–hand side yields
              \[
                  D(x,A)\;\le\;D(x,y)+D(y,A),
              \]
              i.e.\ $f(x)\le D(x,y)+f(y)$.
        \item Reversing the roles of $x$ and $y$ gives
              \[
                  f(y)\;\le\;D(x,y)+f(x).
              \]
        \item Combining the two inequalities gives $(\ast)$:
              \[
                  \lvert f(x)-f(y)\rvert
                  \;=\;\max\{f(x)-f(y),\,f(y)-f(x)\}
                  \;\le\;D(x,y).
              \]
    \end{enumerate}
    
    \smallskip
    \noindent\textbf{Continuity.}
    Let $(x_n)_{n\in\mathbb{N}}\subseteq M$ with $x_n\to x\in M$.  
    Since $D(x_n,x)\to 0$, inequality $(\ast)$ gives
    \[
    \lvert f(x_n)-f(x)\rvert\;\le\;D(x_n,x)\;\xrightarrow[n\to\infty]{}\;0,
    \]
    so $f(x_n)\to f(x)$.  
    Hence $f$ is continuous at every point of $M$.
    \end{proof} 
\end{solution}
\begin{problem}[]
  
\end{problem}
\begin{solution}
  \begin{theorem}
    Let $(M,D)$ be a metric space, let $A\subseteq M$ be closed, and let $y\in M\setminus A$.  
    There exists a continuous function $f:M\to\mathbb{R}$ such that
    \[
    f\bigl\lvert_{\,A}\equiv 0
    \quad\text{while}\quad
    f(y)\neq 0.
    \]
    \end{theorem}
    
    \begin{proof}
    Define
    \[
    f:M\longrightarrow\mathbb{R},
    \qquad
    f(x)\;:=\;D(x,A)\;=\;\inf_{a\in A}D(x,a)\quad(x\in M).
    \]
    
    \begin{enumerate}
        \item[\textbf{1.}] \textbf{Continuity of $f$.}  
              For any $x,z\in M$ and every $a\in A$,
              \[
                    D(x,a)\;\le\;D(x,z)+D(z,a)
              \]
              by the triangle inequality.  
              Taking $\inf_{a\in A}$ on both sides yields
              \[
                    f(x)\;\le\;D(x,z)+f(z).
              \]
              Interchanging $x$ and $z$ and combining the two inequalities we obtain
              \[
                    |f(x)-f(z)|\;\le\;D(x,z)
              \]
              for all $x,z\in M$.  
              Hence $f$ is $1$–Lipschitz and therefore continuous on $M$.
    
        \item[\textbf{2.}] \textbf{Vanishing on $A$.}  
              If $a\in A$ then $D(a,A)=0$ (take $a$ itself in the infimum), so $f(a)=0$.  
              Thus $f\bigl\lvert_{\,A}\equiv 0$.
    
        \item[\textbf{3.}] \textbf{Non‑vanishing at $y$.}  
              Because $A$ is closed and $y\notin A$, the distance
              \[
                    d\;:=\;D(y,A)\;=\;\inf_{a\in A}D(y,a)
              \]
              satisfies $d>0$.  
              Consequently $f(y)=d\neq 0$.
    
    \end{enumerate}
    
    \noindent
    Therefore $f$ satisfies all required conditions.  \qedhere
    \end{proof} 
\end{solution}
\begin{problem}[]
  
\end{problem}
\begin{solution}
  \begin{theorem}
    Let $(X,d_X)$ and $(Y,d_Y)$ be metric spaces and let $f:X\to Y$.
    \begin{enumerate}[]
        \item If $X=\displaystyle\bigcup_{i\in I}U_i$ with each $U_i$ open in $X$ and
              $f\!\restriction_{U_i}$ continuous for every $i\in I$, then $f$ is continuous on $X$.
        \item If $n\in\mathbb{N}$ and $X=\displaystyle\bigcup_{i=1}^{n}F_i$ with each
              $F_i$ closed in $X$ and $f\!\restriction_{F_i}$ continuous for $i=1,\dots,n$,
              then $f$ is continuous on $X$.
        \item Statement \textup{(b)} need not hold for an \emph{infinite} collection of closed sets.
    \end{enumerate}
    \end{theorem}
    
    \begin{proof}
    Throughout we use the \emph{closed‐set characterisation} of continuity:  
    $f$ is continuous on $X$ iff $f^{-1}(C)$ is closed in $X$ for every closed $C\subseteq Y$.
    
    \smallskip
    \noindent\textbf{(a) Open cover.}
    Let $C\subseteq Y$ be closed.  
    For each $i\in I$ the restriction $f\!\restriction_{U_i}$ is continuous, hence
    \[
    f^{-1}(C)\cap U_i
       \;=\;(f\!\restriction_{U_i})^{-1}(C)
    \]
    is closed in $U_i$.  
    Because $U_i$ is open in $X$, any subset that is closed in $U_i$ is in fact closed in $X$.  
    Consequently $f^{-1}(C)=\bigcup_{i\in I}\bigl(f^{-1}(C)\cap U_i\bigr)$ is a union of closed
    subsets of $X$—hence closed.  By the closed‐set criterion, $f$ is continuous on $X$.
    
    \smallskip
    \noindent\textbf{(b) Finite closed cover.}
    Let $C\subseteq Y$ be closed.  For each $i=1,\dots,n$,
    continuity of $f\!\restriction_{F_i}$ gives
    \[
    F_i\cap f^{-1}(C)=(f\!\restriction_{F_i})^{-1}(C)
    \]
    closed in $F_i$.  
    Because $F_i$ itself is closed in $X$, the set $F_i\cap f^{-1}(C)$ is the intersection of two closed subsets of $X$, hence closed in $X$.  
    Since a \emph{finite} union of closed sets is closed, 
    \[
    f^{-1}(C)=\bigcup_{i=1}^{n}\bigl(F_i\cap f^{-1}(C)\bigr)
    \]
    is closed in $X$, proving that $f$ is continuous on $X$.
    
    \smallskip
    \noindent\textbf{(c) Failure for infinitely many closed sets.}
    The argument in (b) hinges on the union being finite.
    If we allow infinitely many closed sets, the conclusion can fail.
    
    \emph{Counterexample.}  
    Let $X=\mathbb{R}$, $Y=\mathbb{R}$, and for each $x\in\mathbb{R}$ set
    \[
    F_x:=\{x\}\quad(\text{a singleton, hence closed}).
    \]
    Clearly $\displaystyle X=\bigcup_{x\in\mathbb{R}}F_x$ is an \emph{infinite} closed cover of $X$.
    Define
    \[
    f:\mathbb{R}\longrightarrow\mathbb{R},\qquad
    f(t):=
    \begin{cases}
    0,& t\le 0,\\[2pt]
    1,& t>0.
    \end{cases}
    \]
    For each closed set $F_x=\{x\}$ the restriction $f\!\restriction_{F_x}$ is a constant map
    and therefore continuous.
    Yet $f$ is \emph{not} continuous on $\mathbb{R}$ (it jumps at $t=0$).
    Thus the finite–closed–cover result does not extend to an infinite family of closed sets.
    \end{proof} 
\end{solution}
\begin{problem}[]
  
\end{problem}
\begin{solution}
  \begin{theorem}
    Every metric space is homeomorphic to one whose diameter is finite (indeed, $\le 1$).
    \end{theorem}
    
    \begin{proof}[Construction and verification]
    Let $(X,d)$ be an arbitrary metric space.  
    Define a new function
    \[
    \varrho:X\times X\longrightarrow[0,1),\qquad
    \varrho(x,y)\;:=\;g\!\bigl(d(x,y)\bigr)
    \quad\text{where}\quad
    g(t)\;:=\;\frac{t}{1+t}\quad(t\ge 0).
    \]
    
    \smallskip
    \noindent\textbf{Step 1: $\varrho$ is a metric.}
    \begin{enumerate}
        \item \emph{Non‑negativity and identity of indiscernibles} are immediate from those of $d$ and the fact that $g(0)=0$ and $g$ is strictly increasing.
        \item \emph{Symmetry} holds because $d$ is symmetric and $g$ depends only on the value of $d(x,y)$.
        \item \emph{Triangle inequality.}  
              For $a,b\ge 0$ observe that
              \[
                  g(a)+g(b)-g(a)g(b)
                  \;=\;\frac{a}{1+a}+\frac{b}{1+b}-\frac{ab}{(1+a)(1+b)}
                  \;=\;\frac{a+b}{1+a+b}.
              \]
              Since $g(a+b)=\dfrac{a+b}{1+a+b}$, we have
              \[
                  g(a+b)\;=\;g(a)+g(b)-g(a)g(b)\;\le\;g(a)+g(b).
              \]
              Now let $x,y,z\in X$.  
              Setting $a=d(x,y)$ and $b=d(y,z)$ and using $d(x,z)\le a+b$,
              \[
                    \varrho(x,z)
                    =g\!\bigl(d(x,z)\bigr)
                    \le g(a+b)
                    \le g(a)+g(b)
                    =\varrho(x,y)+\varrho(y,z).
              \]
    \end{enumerate}
    Hence $\varrho$ is a metric on $X$.
    
    \smallskip
    \noindent\textbf{Step 2: $\varrho$ has finite diameter.}
    Because $g(t)<1$ for every $t\ge 0$, we have
    \[
    \operatorname{diam}(X,\varrho)\;=\;\sup_{x,y\in X}\varrho(x,y)\;\le\;1.
    \]
    
    \smallskip
    \noindent\textbf{Step 3: Topologies coincide.}
    The function $g:[0,\infty)\to[0,1)$ is a homeomorphism with inverse
    $g^{-1}(s)=\dfrac{s}{1-s}$ ($0\le s<1$).  
    For $r>0$ and $x\in X$,
    \[
    B_{\varrho}(x,r)=\bigl\{y\in X : \varrho(x,y)<r\bigr\}
                   =\bigl\{y\in X : d(x,y)<g^{-1}(r)\bigr\}
                   =B_{d}\!\bigl(x,g^{-1}(r)\bigr).
    \]
    Thus the open balls (and hence the open sets) are the same for $d$ and $\varrho$.
    Therefore the identity map
    \[
    \text{id}_X:(X,d)\longrightarrow (X,\varrho),\qquad x\mapsto x,
    \]
    is a homeomorphism.
    
    \smallskip
    \noindent\textbf{Conclusion.}
    The metric $\varrho$ endows $X$ with the same topology as $d$ while ensuring
    $\operatorname{diam}(X,\varrho)\le 1$.  
    Hence every metric space is homeomorphic to a metric space of finite diameter, as claimed.
    \end{proof} 
\end{solution}
\begin{problem}[]
  
\end{problem}
\begin{solution}
  \begin{theorem}
    Let $(M,D)$ be a metric space and let $X\subseteq M$ be dense.
    Assume that \emph{every} Cauchy sequence contained in $X$ converges to a point of $M$.
    Then $M$ is complete.
    \end{theorem}
    
    \begin{proof}
    Let $(x_n)_{n\in\mathbb{N}}$ be an arbitrary Cauchy sequence in $M$.
    We must show that $(x_n)$ converges in $M$.
    
    \medskip
    \noindent\textbf{Step 1: Approximate each $x_n$ by a point in $X$.}
    
    Because $X$ is dense in $M$, for every $n\in\mathbb{N}$ we can choose
    \[
    y_n\in X
    \quad\text{such that}\quad
    D(x_n,y_n)<\frac1n.
    \]
    
    \medskip
    \noindent\textbf{Step 2: $(y_n)$ is a Cauchy sequence in $X$.}
    
    Given $\varepsilon>0$, choose $N\in\mathbb{N}$ such that
    $D(x_m,x_k)<\varepsilon/3$ whenever $m,k\ge N$ (possible because $(x_n)$ is Cauchy).
    For $m,k\ge N$ we have
    \[
    D(y_m,y_k)
    \;\le\;
    D(y_m,x_m)+D(x_m,x_k)+D(x_k,y_k)
    \;<\;
    \frac1m+\varepsilon/3+\frac1k
    \;\le\;
    \varepsilon,
    \]
    so $(y_n)$ is Cauchy.
    
    \medskip
    \noindent\textbf{Step 3: $(y_n)$ converges in $M$.}
    
    By hypothesis, every Cauchy sequence in $X$ converges in $M$.
    Hence there exists $y\in M$ with
    \[
    y_n\longrightarrow y.
    \]
    
    \medskip
    \noindent\textbf{Step 4: $(x_n)$ converges to the same limit $y$.}
    
    For $n\in\mathbb{N}$,
    \[
    D(x_n,y)
    \;\le\;
    D(x_n,y_n)+D(y_n,y)
    \;<\;
    \frac1n + D(y_n,y).
    \]
    Taking $n\to\infty$ the first term $\tfrac1n\to 0$,
    and the second term tends to $0$ because $y_n\to y$.
    Thus $D(x_n,y)\to 0$, i.e.\ $x_n\to y$ in $M$.
    
    \medskip
    \noindent\textbf{Conclusion.}
    Every Cauchy sequence in $M$ converges in $M$, so $(M,D)$ is complete.
    \end{proof} 
\end{solution}
\end{document}
