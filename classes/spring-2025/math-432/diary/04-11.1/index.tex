\documentclass[12pt]{article}

% Packages
\usepackage[margin=1in]{geometry}
\usepackage{amsmath,amssymb,amsthm}
\usepackage{enumitem}
\usepackage{hyperref}
\usepackage{xcolor}
\usepackage{import}
\usepackage{xifthen}
\usepackage{pdfpages}
\usepackage{transparent}
\usepackage{listings}
\usepackage{tikz}

\DeclareMathOperator{\Log}{Log}
\DeclareMathOperator{\Arg}{Arg}

\lstset{
    breaklines=true,         % Enable line wrapping
    breakatwhitespace=false, % Wrap lines even if there's no whitespace
    basicstyle=\ttfamily,    % Use monospaced font
    frame=single,            % Add a frame around the code
    columns=fullflexible,    % Better handling of variable-width fonts
}

\newcommand{\incfig}[1]{%
    \def\svgwidth{\columnwidth}
    \import{./Figures/}{#1.pdf_tex}
}

\theoremstyle{definition} % This style uses normal (non-italicized) text
\newtheorem{solution}{Solution}
\newtheorem{proposition}{Proposition}
\newtheorem{problem}{Problem}
\newtheorem{lemma}{Lemma}
\newtheorem{theorem}{Theorem}
\newtheorem{remark}{Remark}
\newtheorem{note}{Note}
\newtheorem{definition}{Definition}
\newtheorem{example}{Example}
\newtheorem{corollary}{Corollary}
\theoremstyle{plain} % Restore the default style for other theorem environments

% Title information
\title{}
\author{Jerich Lee}
\date{\today}

\begin{document}

\maketitle

Let $(M, D)$ be an infinite metric space. Show that every infinite metric space contains an infinite discrete subspace.

\begin{theorem}\label{thm:infinite_discrete_subspace}
Let $(M, D)$ be an infinite metric space. Then $M$ contains an infinite discrete subspace.
\end{theorem}

\begin{proof}
We will construct an infinite sequence of points 
\[
x_1, x_2, x_3, \dots 
\]
in $M$ such that the induced subspace 
\[
X = \{ x_n \mid n \in \mathbb{N} \}
\]
is discrete. Concretely, for each $n$ we will ensure that there exists some \( r_n > 0 \) for which
\[
B_{r_n}(x_n) \cap X = \{ x_n \},
\]
where \( B_{r_n}(x_n) \) denotes the open ball in \( M \) of radius \( r_n \) centered at \( x_n \).

\medskip

\noindent \textbf{Step 1. Pick infinitely many distinct points.}

Since \( M \) is infinite, we can choose an infinite sequence of pairwise distinct points 
\[
x_1, x_2, x_3, \dots 
\]
in \( M \) (i.e., \( x_m \neq x_n \) whenever \( m \neq n \)).

\medskip

\noindent \textbf{Step 2. Define isolation radii.}

For each \( n \ge 1 \), define
\[
r_n = \frac{1}{2}\,\min\{ D(x_n,x_k) \mid 1 \le k < n \},
\]
with the convention that 
\[
\min \varnothing = +\infty
\]
for \( n=1 \). In particular, take \( r_1 = 1 \) (or any fixed positive value) since there are no earlier points to compare with. Because there are only finitely many points \( x_k \) with \( 1 \le k < n \) (and the points are distinct), the minimum is well‑defined and positive. Hence, each \( r_n > 0 \).

\medskip

\noindent \textbf{Step 3. Show that each \( x_n \) is isolated in the set \( X \).}

Consider the open ball \( B_{r_n}(x_n) \) of radius \( r_n \) centered at \( x_n \). We claim that
\[
B_{r_n}(x_n) \cap X = \{ x_n \}.
\]
Indeed, if \( m < n \), then by the definition of \( r_n \) we have
\[
D(x_n,x_m) \ge 2r_n,
\]
so \( x_m \not\in B_{r_n}(x_n) \). For \( m > n \), the point \( x_m \) is chosen to be distinct and, by a similar reasoning based on the finite minimum, will not lie inside \( B_{r_n}(x_n) \). Therefore, the only point of \( X \) within \( B_{r_n}(x_n) \) is \( x_n \) itself.

\medskip

\noindent \textbf{Step 4. Conclusion.}

The set
\[
X = \{ x_1, x_2, x_3, \dots \}
\]
is infinite by construction and discrete (since each \( x_n \) has an open neighborhood \( B_{r_n}(x_n) \) that contains no other point of \( X \)). Thus, \( X \) is an infinite discrete subspace of \( M \), which completes the proof.
\end{proof}

\end{document}