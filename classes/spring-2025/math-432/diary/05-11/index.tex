\documentclass[12pt]{article}

% Packages
\usepackage[margin=1in]{geometry}
\usepackage{amsmath,amssymb,amsthm}
\usepackage{enumitem}
\usepackage{hyperref}
\usepackage{xcolor}
\usepackage{import}
\usepackage{xifthen}
\usepackage{pdfpages}
\usepackage{transparent}
\usepackage{listings}
\usepackage{tikz}
\usepackage{physics}
\usepackage{siunitx}
\usepackage{booktabs}
\usepackage{cancel}
  \usetikzlibrary{calc,patterns,arrows.meta,decorations.markings}


\DeclareMathOperator{\Log}{Log}
\DeclareMathOperator{\Arg}{Arg}

\lstset{
    breaklines=true,         % Enable line wrapping
    breakatwhitespace=false, % Wrap lines even if there's no whitespace
    basicstyle=\ttfamily,    % Use monospaced font
    frame=single,            % Add a frame around the code
    columns=fullflexible,    % Better handling of variable-width fonts
}

\newcommand{\incfig}[1]{%
    \def\svgwidth{\columnwidth}
    \import{./Figures/}{#1.pdf_tex}
}
\theoremstyle{definition} % This style uses normal (non-italicized) text
\newtheorem{solution}{Solution}
\newtheorem{proposition}{Proposition}
\newtheorem{problem}{Problem}
\newtheorem{lemma}{Lemma}
\newtheorem{theorem}{Theorem}
\newtheorem{remark}{Remark}
\newtheorem{note}{Note}
\newtheorem{definition}{Definition}
\newtheorem{example}{Example}
\newtheorem{corollary}{Corollary}
\theoremstyle{plain} % Restore the default style for other theorem environments
%

% Theorem-like environments
% Title information
\title{MATH 432 Final Practice Exam 1}
\author{Jerich Lee}
\date{\today}

\begin{document}

\maketitle
%%%%%%%%%%%%%%%%%%%%%%%%%%%%%%%%%%%%%%%%%%%%%%%%%%%%%%%%%%%%%%%%%%%%%%%%%%%%
%                MATH 432 – Practice Final Examination (8 Problems)
%                (All work must be fully justified.)                      %
%%%%%%%%%%%%%%%%%%%%%%%%%%%%%%%%%%%%%%%%%%%%%%%%%%%%%%%%%%%%%%%%%%%%%%%%%%%%



\begin{problem}[Set algebra and symmetric difference]
  For any sets $A,B,C$ define
  \[
      A\setminus B \;:=\; \{\,x\in A : x\notin B\,\},
      \qquad
      A\triangle B \;:=\;(A\setminus B)\cup(B\setminus A).
  \]
  Prove each of the following.
  \begin{enumerate}[label=(\alph*)]
    \item $A\triangle B=(A\cup B)\setminus(A\cap B)$.
    \item $A\cap(B\triangle C)=(A\cap B)\triangle(A\cap C)$.
    \item $A\triangle C\;\subset\;(A\triangle B)\cup(B\triangle C)$.
    \item $C\setminus(A\cup B)=(C\setminus A)\cap(C\setminus B)$.
    \item \emph{(Associativity)}\;
          $(A\triangle B)\triangle C
          =A\triangle(B\triangle C)$.
  \end{enumerate}
\end{problem}

\begin{problem}[Posets, lattices, and completeness]
  Let $(L,\le)$ be a lattice.  Assume that \emph{every chain} in $L$
  has both a least upper bound and a greatest lower bound.
  \begin{enumerate}[label=(\alph*)]
    \item Show that \emph{every} subset of $L$ has a least upper bound.
    \item Show that \emph{every} subset of $L$ has a greatest lower bound.
    \item Conclude that $L$ is a \textbf{complete lattice}.
  \end{enumerate}
\end{problem}

\begin{problem}[Composition of functions]
  Let $f:A\to B$ and $g:B\to C$ be functions.
  \begin{enumerate}[label=(\alph*)]
    \item Prove: if both $f$ and $g$ are onto, then $g\circ f$ is onto.
    \item Prove: if $g\circ f$ is onto, then $g$ is onto.
    \item Give an explicit example where $g\circ f$ is onto but $f$ is
          \emph{not} onto.
  \end{enumerate}
\end{problem}

\begin{problem}[Countability]
  \begin{enumerate}[label=(\alph*)]
    \item Prove that the countable product
          $\mathbb{N}\times\mathbb{N}\times\cdots\times\mathbb{N}$
          (finite length) is countable.  Deduce that any finite product
          of countable sets is countable.
    \item Let $A$ be a countably infinite set and let
          $X:=\{\,F\subset A : F\text{ is finite}\,\}$.
          Show that $X$ is countable.
  \end{enumerate}
\end{problem}

\begin{problem}[Cardinal arithmetic \& bijections]
  \begin{enumerate}[label=(\alph*)]
    \item Exhibit an explicit bijection showing
          $(0,1)\times(0,1)\times(0,1)\;\cong\;(0,1)$
          and hence prove that $|\,\mathbb{R}^{3}\,|=|\mathbb{R}|$.
    \item Prove, without using Theorems 13–14 from the notes, that
          $c+c=c$, where $c$ is the cardinality of~$\mathbb{R}$.
  \end{enumerate}
\end{problem}

\begin{problem}[Topology of metric spaces]
  Let $(M,D)$ be a metric space and fix $x\in M$ together with
  $0<r<s$.
  \begin{enumerate}[label=(\alph*)]
    \item Prove that the \emph{punctured annulus}
          $A(x,r,s):=\{\,y\in M\mid r<D(x,y)<s\,\}$ is open.
    \item Prove that the \emph{sphere}
          $S(x,r):=\{\,y\in M\mid D(x,y)=r\,\}$ is closed.
    \item Show that $\overline{A}(x,r,s)
          :=\{\,y\in M\mid r\le D(x,y)\le s\,\}$ is closed.
  \end{enumerate}
\end{problem}

\begin{problem}[Uniform continuity]
  Determine which of the following maps
  $f:\mathbb{R}\to\mathbb{R}$ are \emph{uniformly} continuous.
  Prove your answers.
  \begin{enumerate}[label=(\roman*)]
    \item $f(x)=x^{2}$\quad($x\in\mathbb{R}$).
    \item $f(x)=|x|$\quad($x\in\mathbb{R}$).
    \item $f(x)=\dfrac{1}{1+x^{2}}$\quad($x\in\mathbb{R}$).
  \end{enumerate}
\end{problem}

\begin{problem}[A basis via Zorn’s Lemma]
  Let $V$ be a (possibly infinite-dimensional) real vector space.
  Recall that a subset $B\subset V$ is a \emph{basis} if
  $B$ is linearly independent and
  every $v\in V$ can be written as a finite linear combination of
  vectors from $B$.
  Use Zorn’s Lemma to prove that \emph{every} vector space possesses a
  basis.
\end{problem}
\end{document}
