\documentclass[12pt]{article}

% Packages
\usepackage[margin=1in]{geometry}
\usepackage{amsmath,amssymb,amsthm}
\usepackage{enumitem}
\usepackage{hyperref}
\usepackage{xcolor}
\usepackage{import}
\usepackage{xifthen}
\usepackage{pdfpages}
\usepackage{transparent}
\usepackage{listings}
\usepackage{tikz}
  \usetikzlibrary{calc,patterns,arrows.meta,decorations.markings}


\DeclareMathOperator{\Log}{Log}
\DeclareMathOperator{\Arg}{Arg}
\DeclareMathOperator{\diam}{diam}

\lstset{
    breaklines=true,         % Enable line wrapping
    breakatwhitespace=false, % Wrap lines even if there's no whitespace
    basicstyle=\ttfamily,    % Use monospaced font
    frame=single,            % Add a frame around the code
    columns=fullflexible,    % Better handling of variable-width fonts
}

\newcommand{\incfig}[1]{%
    \def\svgwidth{\columnwidth}
    \import{./Figures/}{#1.pdf_tex}
}
\theoremstyle{definition} % This style uses normal (non-italicized) text
\newtheorem{solution}{Solution}
\newtheorem{proposition}{Proposition}
\newtheorem{problem}{Problem}
\newtheorem{lemma}{Lemma}
\newtheorem{theorem}{Theorem}
\newtheorem{remark}{Remark}
\newtheorem{note}{Note}
\newtheorem{definition}{Definition}
\newtheorem{example}{Example}
\newtheorem{corollary}{Corollary}
\theoremstyle{plain} % Restore the default style for other theorem environments
%

% Theorem-like environments
% Title information
\title{}
\author{Jerich Lee}
\date{\today}

\begin{document}

\maketitle
%------------------------------------------------
% Completeness – Cauchy sequences
%------------------------------------------------
\section*{5.1 \quad Completeness}

\begin{definition}[Cauchy sequence]
    Let $(M,D)$ be a metric space.  
    A sequence $\{x_i\}_{i=1}^{\infty}\subseteq M$ is called \emph{Cauchy} if
    \[
        \forall\,\varepsilon>0\;\exists\,N\in\mathbb N\ 
        \text{such that}\ 
        D(x_i,x_j)<\varepsilon
        \quad\text{for all }i,j\ge N.
    \]
    Intuitively, the points $x_i$ eventually become arbitrarily close to
    one another, regardless of a specific limit point.
\end{definition}

\begin{theorem}
    Any convergent sequence in a metric space is a Cauchy sequence.
\end{theorem}

\begin{proof}
    Let $\{x_i\}$ be a sequence in $(M,D)$ and suppose $x_i\to y\in M$.
    We verify the Cauchy condition step by step.

    \begin{enumerate}
        \item[\textbf{1.}] \textbf{Fix an accuracy threshold.}\\
              Take an arbitrary $\varepsilon>0$.

        \item[\textbf{2.}] \textbf{Exploit convergence.}\\
              Because $x_i\to y$, there exists
              $N\in\mathbb N$ such that
              \[
                  D(x_i,y)<\frac{\varepsilon}{2}
                  \qquad\text{whenever }i\ge N.
              \]

        \item[\textbf{3.}] \textbf{Estimate pairwise distances.}\\
              For any indices $i,j\ge N$,
              \begin{align}
                  D(x_i,x_j)
                  &\le D(x_i,y)+D(y,x_j)
                      &&\text{(triangle inequality)}\\[4pt]
                  &<\frac{\varepsilon}{2}+\frac{\varepsilon}{2}
                      =\varepsilon.
              \end{align}

        \item[\textbf{4.}] \textbf{Conclusion.}\\
              The existence of such an $N$ for the chosen $\varepsilon$
              shows that the sequence $\{x_i\}$ is Cauchy.
    \end{enumerate}
\end{proof}

\medskip
\noindent
\textbf{Remark.}\; The converse need not hold unless the ambient space is
\emph{complete}; that is, a Cauchy sequence may fail to converge in an
incomplete metric space.

\begin{theorem}\label{thm:cauchy_subseq}
  Let $\{x_i\}_{i=1}^{\infty}$ be a \emph{Cauchy} sequence in a metric space $(M,D)$.  
  If some subsequence $\{x_{i_k}\}_{k=1}^{\infty}$ converges to a point $y\in M$,  
  then the \textbf{entire} sequence $\{x_i\}$ converges to the same limit~$y$.
\end{theorem}

\begin{proof}
  We show that every $\varepsilon>0$ eventually forces \emph{all} terms of the
  sequence to lie within $\varepsilon$ of $y$.

  \smallskip
  \textbf{Step 1 (Cauchy control).}
  Because $\{x_i\}$ is Cauchy, for the positive number $\varepsilon/2$
  there exists an index $N_1\in\mathbb N$ such that
  \[
      D(x_i,x_j)<\frac{\varepsilon}{2}
      \qquad\text{for all }i,j\ge N_1.
  \tag{$\ast$}
  \]

  \smallskip
  \textbf{Step 2 (Subsequence convergence).}
  The subsequence $\{x_{i_k}\}$ converges to $y$, 
  so there exists $N_2\in\mathbb N$ with
  \[
      k\ge N_2
      \quad\Longrightarrow\quad
      D(x_{i_k},y)<\frac{\varepsilon}{2}.
  \tag{$\ast\!\ast$}
  \]

  \smallskip
  \textbf{Step 3 (Choose a suitable index).}
  Let $N:=\max\{\,N_1,i_{N_2}\,\}$.  
  Then $N\ge N_1$ and the subsequence term $x_{i_{N_2}}$ satisfies $i_{N_2}\le N$.

  \smallskip
  \textbf{Step 4 (Estimate distance to the limit).}
  For any $i\ge N$ pick the subsequence index $k=N_2$; by construction $i_{N_2}\ge N_1$
  and $i_{N_2}\ge N$.  
  Apply the triangle inequality using $(\ast)$ and $(\ast\!\ast)$:
  \[
      D(x_i,y)
      \;\le\;
      D\bigl(x_i,x_{i_{N_2}}\bigr)+D\bigl(x_{i_{N_2}},y\bigr)
      \;<\;
      \frac{\varepsilon}{2}+\frac{\varepsilon}{2}
      \;=\;
      \varepsilon.
  \]

  \smallskip
  \textbf{Step 5 (Conclusion).}
  Since $\varepsilon>0$ was arbitrary, we have shown that
  $D(x_i,y)\to 0$ as $i\to\infty$; hence $x_i\to y$ and the whole
  sequence converges to $y$.
\end{proof}
\begin{theorem}[Cauchy sequences are bounded]\label{thm:cauchy_bounded}
  Let $(x_n)_{n\ge 1}$ be a Cauchy sequence in a metric space $(M,D)$.
  Then the set
  \[
      S:=\{x_n\mid n\in\mathbb N\}
  \]
  has \emph{finite diameter}; i.e.\ there exists $R<\infty$ with
  $D(x_i,x_j)\le R$ for all $i,j\in\mathbb N$.
\end{theorem}

\begin{proof}
  We give a step–by–step argument.

  \begin{enumerate}
      \item[\textbf{1.}] \textbf{Choose an accuracy threshold.}  
            Fix an arbitrary $\varepsilon>0$ (e.g.\ $\varepsilon=1$ will
            already establish boundedness, but any positive choice works).

      \item[\textbf{2.}] \textbf{Invoke the Cauchy property.}  
            Because the sequence is Cauchy, there exists
            $N\in\mathbb N$ such that
            \[
                D(x_i,x_j)<\varepsilon
                \qquad\text{for all }i,j\ge N.
            \tag{$\ast$}
            \]

      \item[\textbf{3.}] \textbf{Separate the sequence into two parts.}
            \begin{itemize}
               \item The \emph{tail}
                     $\{x_n\}_{n\ge N}$ is contained in the ball
                     $B\bigl(x_N,\varepsilon\bigr)$ by $(\ast)$,
                     so any two points in the tail are at most
                     $\varepsilon$ apart.
               \item The \emph{head}
                     $\{x_1,\dots,x_{N-1}\}$ is a \emph{finite} set and
                     therefore has a finite diameter; denote
                     \[
                         R_0
                         :=\max_{1\le i,j\le N-1}D(x_i,x_j)
                         <\infty.
                     \]
            \end{itemize}

      \item[\textbf{4.}] \textbf{Estimate distances between head and tail.}  
            For $1\le i\le N-1$ and $j\ge N$ we use the triangle inequality:
            \[
                D(x_i,x_j)
                \le D(x_i,x_N)+D(x_N,x_j)
                < D(x_i,x_N)+\varepsilon.
            \]
            Set
            \[
                R_1:=\max_{1\le i\le N-1}D(x_i,x_N)<\infty.
            \]
            \paragraph{4.  Estimate distances between the \emph{head} and the \emph{tail}.}
            Let \(1\le i\le N-1\) (so \(x_i\) lies in the finite head) and \(j\ge N\) (so \(x_j\) lies in the Cauchy–controlled tail).  
            Since \(\{x_n\}\) is Cauchy, the choice of \(N\) in \eqref{eq:Cauchy} gives  
            \[
            D(x_N,x_j)<\varepsilon . \tag{\(*\)}
            \]
            Apply the triangle inequality in the form \(D(a,c)\le D(a,b)+D(b,c)\) with  
            \(a=x_i,\; b=x_N,\; c=x_j\):
            \[
            D(x_i,x_j)\;\le\;D(x_i,x_N)+D(x_N,x_j)
                         \;<\;D(x_i,x_N)+\varepsilon ,
            \]
            where the strict inequality uses \((*)\).
            
            \medskip
            Because the set \(\{x_1,\dots,x_{N-1}\}\) is \emph{finite},
            \[
            R_1\;:=\;\max_{1\le i\le N-1} D(x_i,x_N)\;<\infty .
            \]
            Thus every “head–tail” pair satisfies
            \[
            D(x_i,x_j)\;<\;R_1+\varepsilon
            \quad\text{for all }1\le i\le N-1,\; j\ge N.
            \]
      \item[\textbf{5.}] \textbf{Collect the bounds.}  
            Define
            \[
                R:=\max\{R_0,\,R_1+\varepsilon,\,\varepsilon\}.
            \]
            Then every pair of indices $i,j\in\mathbb N$ satisfies
            $D(x_i,x_j)\le R$.  Hence $\operatorname{diam}(S)\le R<\infty$.

      \item[\textbf{6.}] \textbf{Conclusion.}  
            The sequence’s range $S$ is contained in the closed ball
            $\overline{B}(x_N,R)$, showing that every Cauchy sequence is
            bounded (indeed, has finite diameter).
            \qedhere
  \end{enumerate}
\end{proof}
%------------------------------------------------
% Monotone subsequence in a chain (Theorem 46)
%------------------------------------------------
\begin{definition}
  A \emph{chain} is a totally ordered set; i.e.\ for any two elements
  $a,b$ we have either $a<b$ or $b<a$.
\end{definition}
\begin{theorem}[Monotone subsequence]\label{thm:monotone}
  Every (infinite) real sequence $(s_n)_{n\ge1}$ contains an infinite monotone
  subsequence.
  \end{theorem}
  
  \begin{proof}
  Call the \(n\)-th term \emph{dominant} if it is greater than every term
  that follows it:
  \[
     s_m < s_n \quad\text{for all } m>n. \tag{1}
  \]
  
  \textbf{Case 1.  Infinitely many dominant terms.}
  Let \(\bigl(s_{n_k}\bigr)_{k\ge1}\) be the subsequence of all dominant
  terms.  Then by~(1)
  \[
     s_{n_1} > s_{n_2} > s_{n_3} > \cdots,
  \]
  so \(\bigl(s_{n_k}\bigr)\) is strictly decreasing.
  
  \medskip
  \textbf{Case 2.  Only finitely many dominant terms.}
  Choose \(n_1\) beyond the last dominant index.  Because
  there are no dominant terms beyond \(n_1\), for every \(N\ge n_1\) we can find
  an index \(m>N\) with
  \[
     s_m \ge s_N. \tag{2}
  \]
  Apply~(2) with \(N=n_1\) to select \(n_2>n_1\) such that
  \(s_{n_2}\ge s_{n_1}\).  Suppose \(n_1,\dots,n_{k-1}\) have been chosen with
  \[
     n_1 < n_2 < \cdots < n_{k-1} \qquad\text{and}\qquad
     s_{n_1} \le s_{n_2} \le \cdots \le s_{n_{k-1}}. \tag{3}
  \]
  Applying~(2) with \(N=n_{k-1}\) yields \(n_k>n_{k-1}\) such that
  \(s_{n_k}\ge s_{n_{k-1}}\).  The relations~(3) then hold with \(k\) in place
  of \(k-1\).  Proceeding inductively we obtain an infinite subsequence
  \[
     s_{n_1} \le s_{n_2} \le s_{n_3} \le \cdots,
  \]
  which is non‑decreasing.
  
  \medskip
  In either case, the original sequence contains a monotone subsequence,
  completing the proof.
  \end{proof}

\begin{theorem}[Monotone subsequence]\label{thm:monotone_subseq}
  Let $x_1,x_2,x_3,\dots$ be a sequence of \emph{distinct} elements of a
  chain $(C,<)$.  Then the sequence contains
  \emph{either} a monotone increasing subsequence
  \[
      x_{i_1}<x_{i_2}<x_{i_3}<\dots,
      \qquad i_1<i_2<i_3<\dots,
  \]
  \emph{or} a monotone decreasing subsequence
  \[
      x_{j_1}>x_{j_2}>x_{j_3}>\dots,
      \qquad j_1<j_2<j_3<\dots.
  \]
\end{theorem}

\begin{proof}
  We argue by contradiction.

  \smallskip
  \textbf{Step 1 – Assume no increasing subsequence exists.}
  Suppose that \emph{no} subsequence of $\{x_n\}$ is strictly increasing.  
  We shall construct a strictly decreasing one.

  \smallskip
  \textbf{Step 2 – ``Hunt‐down’’ procedure and the notion of dominance.}
  Fix an index $i$.  
  Because the sequence is \emph{not} eventually increasing, there exists
  $j>i$ with $x_j<x_i$.  
  Continue: since $\{x_{j+1},x_{j+2},\dots\}$ is likewise not increasing,
  pick $k>j$ with $x_k<x_j$, and so on.
  This selection process must terminate in finitely many steps because the
  set of indices is finite up to any bound.

  The last chosen term, say $x_m$, is \emph{larger} than all terms that
  follow it; otherwise we could extend the chain and contradict its
  maximality.  Call such an $x_m$ a \emph{dominant element}:
  \[
      x_m\text{ is dominant } \iff 
      x_m>x_\ell\text{ for every }\ell>m.
  \]
  %---  Detailed justification of the stopping condition  ---%
\medskip\noindent
\emph{Why the procedure stops with a dominant element.}
During the ``hunt–down’’ construction we have already produced indices
\[
      i_{1}\;<\;i_{2}\;<\;\dots\;<\;i_{r}=m
      \qquad\text{such that}\qquad
      x_{i_{k+1}} \;<\; x_{i_{k}}
      \quad(1\le k<r).
\]
We halt the process the first time we fail to locate an index
\(j>m\) with \(x_{j}<x_{m}\).
Assume, \emph{toward a contradiction}, that such an index actually exists;
pick one and call it \(\ell\!>\!m\) with \(x_{\ell}<x_{m}\).
Appending \(\ell\) to the list yields a strictly longer decreasing chain
\[
      x_{i_{1}}
      \;>\;
      x_{i_{2}}
      \;>\;\dots\;>\;
      x_{m}
      \;>\;
      x_{\ell},
\]
which contradicts our choice of \(m\) as the last (i.e.\ maximal) element
selected by the procedure.
Therefore no such \(\ell\) can exist, and we must have
\[
      x_{m} \;>\; x_{\ell} 
      \qquad\text{for every } \ell>m,
\]
so \(x_{m}\) is \emph{dominant}.
\medskip

  \smallskip
  \textbf{Step 3 – Existence of infinitely many dominant elements.}
  The argument above shows:  
  \emph{after} any index $i$ one can find a dominant element.  
  Repeating inductively, we obtain an infinite sequence of indices
  \[
      j_1<j_2<j_3<\dots
  \]
  such that $x_{j_s}$ is dominant for each $s$.

  \smallskip
  \textbf{Step 4 – Dominant elements form a decreasing subsequence.}
  By definition of dominance we have
  \(
      x_{j_1}>x_{j_2}>x_{j_3}>\dots
  \);
  thus $\{x_{j_s}\}_{s\ge1}$ is a strictly monotone decreasing
  subsequence.

  \smallskip
  \textbf{Step 5 – Dichotomy.}
  If the original assumption (no increasing subsequence) fails, then a
  strictly increasing subsequence already exists.
  Hence in \emph{every} case the sequence contains a monotone subsequence,
  completing the proof.
\end{proof}
% ------------------------------------------------------------
%  FULLY EXPANDED HUNT–DOWN EXAMPLE (to accompany Theorem 5)
% ------------------------------------------------------------
\begin{example}[Expanded illustration of Steps 1–5]\label{ex:long}
  We work with the sequence
  \[
        x_1=5,\;x_2=-1,\;x_3=4,\;x_4=-2,\;x_5=3,\;x_6=-3,\;
        x_7=2,\;x_8=-4,\;x_9=1,\;x_{10}=-5,\dots
  \]
  generated by
  \(x_{2k-1}=6-k,\;x_{2k}=-k\).
  
  \bigskip\noindent
  \textbf{Step 1 (temporary hypothesis).}
  \emph{Assume that no increasing subsequence exists.}
  We shall see this forces the appearance of a decreasing one.
  
  \bigskip\noindent
  \textbf{Step 2 (hunt–down search inside a finite window).}
  Fix an upper bound \(N=12\) so that only the finitely many indices
  \(\{1,2,\dots,12\}\) are in play.
  Start at the left end and repeatedly look for a strictly smaller
  value to the right until either we fall off the window (forced stop) or no
  smaller value exists (dominant element found).
  
  \begin{center}
  \begin{tabular}{|c|c|c|c|}
  \hline
  \textbf{round} & \(\boldsymbol{i_r}\) chosen & \(\boldsymbol{x_{i_r}}\) &  
  next \(j>i_r\) with \(x_j<x_{i_r}\)?\\
  \hline
  1 & \(1\) & \(5\)  & \(\boxed{2}\) (because \(-1<5\))\\
  2 & \(2\) & \(-1\) & \(\boxed{4}\) (because \(-2<-1\))\\
  3 & \(4\) & \(-2\) & \(\boxed{6}\) (because \(-3<-2\))\\
  4 & \(6\) & \(-3\) & \(\boxed{8}\) (because \(-4<-3\))\\
  5 & \(8\) & \(-4\) & \(\boxed{10}\) (because \(-5<-4\))\\
  6 & \(10\)& \(-5\)& none in \(\{11,12\}\) \(\Rightarrow\) \emph{stop}\\
  \hline
  \end{tabular}
  \end{center}
  
  \noindent
  Hence the finite search terminates at \(m=10\)
  with “last chosen’’ value \(x_{10}=-5\).
  By construction,
  \(x_{10}>x_\ell\) for all \(\ell>10\) with \(\ell\le12\);
  for that window \(x_{10}\) is therefore \emph{dominant}.
  
  \bigskip\noindent
  \textbf{Step 3 (restart on the next tail).}
  Move the left–hand edge of the window just past \(10\),
  i.e.\ begin anew at index \(11\) and run the same search
  on \(\{11,12,\dots,22\}\).
  A short calculation shows the procedure stops at \(m=11\)
  because \(x_{11}=0\) exceeds every later \(x_\ell\)
  up to index \(22\).
  Repeating forever we obtain the \emph{infinite chain of dominants}
  
  \[
     j_1=1,\;j_2=3,\;j_3=5,\;j_4=7,\dots,
     \qquad
     x_{j_s}=6-s\quad(s\ge1).
  \]
  
  \bigskip\noindent
  \textbf{Step 4 (dominants \(\Rightarrow\) decreasing subsequence).}
  Because each \(x_{j_s}\) is larger than \emph{every} subsequent term,
  in particular
  \(
        x_{j_1}>x_{j_2}>x_{j_3}>\dots
  \),
  so the subsequence
  \(
        x_{j_s}=5,4,3,2,1,\dots
  \)
  is strictly decreasing.
  
  \bigskip\noindent
  \textbf{Step 5 (dichotomy concluded).}
  Our “no increasing subsequence’’ hypothesis has produced a
  \emph{decreasing} subsequence, while its negation
  would give an increasing one immediately.
  Thus, in all cases, the original non‑monotone sequence possesses a
  monotone subsequence, confirming Theorem 5. \qed
  \end{example}
\begin{theorem}[Completeness of the real line]\label{thm:R_complete}
  The metric space $(\mathbb R,|\!-\!|)$ is \emph{complete};\,
  i.e.\ every Cauchy sequence of real numbers converges to a real number.
\end{theorem}

\begin{proof}
  Let $\{x_n\}_{n\ge 1}$ be an arbitrary Cauchy sequence in $\mathbb R$.
  We proceed in a sequence of logical reductions that culminate in the
  desired convergence.

  %------------------------------------------------------------
  \paragraph{\textbf{Step 1.}  Extract a convergent subsequence.}
  \begin{enumerate}[label=\arabic*.]
      \item If some value occurs \emph{infinitely} often in the sequence,
            then the constant subsequence consisting of these repetitions
            is trivially convergent; go directly to Step 4.
      \item Otherwise every real number appears only \emph{finitely}
            many times.  Hence we can discard repeats and obtain a
            subsequence of \emph{distinct} terms
            \(
                x_{n_1},x_{n_2},x_{n_3},\dots
            \)
            with $n_1<n_2<n_3<\cdots$.
  \end{enumerate}

  %------------------------------------------------------------
  \paragraph{\textbf{Step 2.}  Produce a monotone subsequence.}
  By Theorem 46 (monotone subsequence lemma) applied to the chain
  $(\mathbb R,<)$, the subsequence of distinct terms itself contains
  a further subsequence that is either \emph{monotone increasing}
  or \emph{monotone decreasing}.  Denote it by
  \(
      x_{m_1},x_{m_2},x_{m_3},\dots
  \).

  %------------------------------------------------------------
  \paragraph{\textbf{Step 3.}  Show the monotone subsequence is bounded.}
  The original sequence $\{x_n\}$ is Cauchy and therefore bounded by
  Theorem 45.  Every subsequence of a bounded sequence is bounded, so
  $\{x_{m_k}\}$ is a bounded monotone sequence of real numbers.

  %------------------------------------------------------------
  \paragraph{\textbf{Step 4.}  Convergence of the subsequence.}
  A classical fact from real analysis states that every bounded monotone
  sequence of real numbers converges.  Thus there exists
  $y\in\mathbb R$ such that
  \[
      x_{m_k}\;\longrightarrow\; y\quad(k\to\infty).
  \]

  %------------------------------------------------------------
  \paragraph{\textbf{Step 5.}  Conclude convergence of the whole sequence.}
  The original sequence $\{x_n\}$ is Cauchy and possesses a convergent
  subsequence $\{x_{m_k}\}\to y$.  By Theorem 44
  (``Cauchy $+$ convergent subsequence $\Rightarrow$ convergence’’)
  the \emph{entire} sequence $\{x_n\}$ converges to the same limit~$y$.

  \medskip
  \noindent
  Since $\{x_n\}$ was an arbitrary Cauchy sequence in $\mathbb R$, we
  have proved that every Cauchy sequence of real numbers converges in
  $\mathbb R$.  Therefore $(\mathbb R,|\!-\!|)$ is complete.
\end{proof}
\begin{theorem}\label{thm:closed_subset_complete}
  Let $(M,D)$ be a \emph{complete} metric space and let
  $A\subseteq M$ be \emph{closed}.  
  Then $(A,D|_{A\times A})$ is complete; that is, every Cauchy sequence
  contained in $A$ converges to a point of $A$.
\end{theorem}

\begin{proof}[Step‑by‑step proof]
  \textbf{1.  Start with an arbitrary Cauchy sequence in $A$.}

  Let $\{a_n\}_{n\ge 1}\subseteq A$ be Cauchy with respect to $D$.

  \medskip
  \textbf{2.  View the same sequence inside $M$.}

  Because $A\subseteq M$, the very same sequence
  $\{a_n\}$ is also a Cauchy sequence in the ambient space $M$.

  \medskip
  \textbf{3.  Invoke completeness of $M$ to obtain a limit in $M$.}

  Since $M$ is complete, there exists a point $b\in M$ such that
  \[
      a_n \;\longrightarrow\; b
      \qquad(n\to\infty).
  \]

  \medskip
  \textbf{4.  Use closedness of $A$ to trap the limit inside $A$.}

  Closedness means: if a sequence of points in $A$ converges in $M$,
  then its limit lies in $A$.  
  Therefore $b\in A$.

  \medskip
  \textbf{5.  Conclude completeness of the subspace.}

  We have shown that an \emph{arbitrary} Cauchy sequence in $A$ actually
  converges to a point of $A$.  Hence $(A,D|_{A\times A})$ is complete.
\end{proof}

\begin{remark}[Converse statement]
  Conversely, if $A\subseteq M$ is complete \emph{as a metric space with the
  restricted metric}, then $A$ must be closed in $M$:
  if $x_n\in A$ converges to $x\in M$, the sequence is Cauchy in $A$,
  so by completeness of $A$ it converges to a point of $A$; uniqueness of
  limits forces $x\in A$.  Thus $\overline{A}=A$.
\end{remark}
\begin{theorem}\label{thm:complete_impl_closed}
  Let $(M,D)$ be a metric space and let $A\subseteq M$.  
  If $(A,D|_{A\times A})$ is \emph{complete} (with the induced metric),
  then $A$ is \emph{closed} in $M$.
\end{theorem}

\begin{proof}[Step‑by‑step proof]
  We must show that \(\overline{A}=A\).  
  Equivalently, if a sequence in \(A\) converges in \(M\), its limit
  actually lies in \(A\).

  \smallskip
  \textbf{1.  Take an arbitrary convergent sequence from \(A\).}  
  Let \(\{a_i\}_{i\ge1}\subseteq A\) and suppose
  \[
      a_i \;\longrightarrow\; b
      \quad\text{in } (M,D)
      \quad\text{for some }b\in M.
  \]

  \smallskip
  \textbf{2.  Use convergence to deduce the sequence is Cauchy.}  
  Every convergent sequence in a metric space is Cauchy
  (Theorem \ref{thm:cauchy_subseq} or standard first‑year analysis),
  so \(\{a_i\}\) is Cauchy in \(M\) and hence also Cauchy when regarded
  inside the subspace \(A\).

  \smallskip
  \textbf{3.  Invoke completeness of \(A\).}  
  Because \(A\) is complete, the Cauchy sequence \(\{a_i\}\)
  converges (with respect to \(D\)) to some limit point \(a\in A\).

  \smallskip
  \textbf{4.  Identify the limit.}  
  Limits in a metric space are \emph{unique}; but \(\{a_i\}\) already
  converges to \(b\) in \(M\).
  Therefore \(a=b\).

  \smallskip
  \textbf{5.  Conclude closedness.}  
  We have shown: whenever a sequence of points of \(A\) converges in \(M\),
  its limit lies in \(A\).  
  This is precisely the definition of \(A\) being \emph{closed} in \(M\).
\end{proof}
\begin{theorem}\label{thm:complete_impl_closed}
  Let $(M,D)$ be a metric space and let $A\subseteq M$.  
  If $(A,D|_{A\times A})$ is \emph{complete} (with the induced metric),
  then $A$ is \emph{closed} in $M$.
\end{theorem}

\begin{proof}[Step‑by‑step proof]
  We must show that \(\overline{A}=A\).  
  Equivalently, if a sequence in \(A\) converges in \(M\), its limit
  actually lies in \(A\).

  \smallskip
  \textbf{1.  Take an arbitrary convergent sequence from \(A\).}  
  Let \(\{a_i\}_{i\ge1}\subseteq A\) and suppose
  \[
      a_i \;\longrightarrow\; b
      \quad\text{in } (M,D)
      \quad\text{for some }b\in M.
  \]

  \smallskip
  \textbf{2.  Use convergence to deduce the sequence is Cauchy.}  
  Every convergent sequence in a metric space is Cauchy
  (Theorem \ref{thm:cauchy_subseq} or standard first‑year analysis),
  so \(\{a_i\}\) is Cauchy in \(M\) and hence also Cauchy when regarded
  inside the subspace \(A\).

  \smallskip
  \textbf{3.  Invoke completeness of \(A\).}  
  Because \(A\) is complete, the Cauchy sequence \(\{a_i\}\)
  converges (with respect to \(D\)) to some limit point \(a\in A\).

  \smallskip
  \textbf{4.  Identify the limit.}  
  Limits in a metric space are \emph{unique}; but \(\{a_i\}\) already
  converges to \(b\) in \(M\).
  Therefore \(a=b\).

  \smallskip
  \textbf{5.  Conclude closedness.}  
  We have shown: whenever a sequence of points of \(A\) converges in \(M\),
  its limit lies in \(A\).  
  This is precisely the definition of \(A\) being \emph{closed} in \(M\).
\end{proof}
%------------------------------------------------
% Theorem 50  –  Characterisation of completeness
%------------------------------------------------
\begin{theorem}\label{thm:nested_diam}
  For a metric space $(M,D)$ the following statements are equivalent:
  \begin{enumerate}[label=\textup{(\alph*)}]
      \item \textbf{Completeness:} every Cauchy sequence in $M$ converges to a
            point of $M$;
      \item \textbf{Nested–set property:} for \emph{every} descending chain of
            non‑empty closed subsets
            \[
                A_1 \supseteq A_2 \supseteq A_3 \supseteq \dots
                \qquad\text{with}\qquad
                \diam(A_n):=\sup_{x,y\in A_n}D(x,y)\;\longrightarrow\;0,
            \]
            the intersection $\displaystyle\bigcap_{n=1}^{\infty}A_n$ is
            non‑empty.
  \end{enumerate}
\end{theorem}

\begin{proof}
  We prove \((a)\!\implies\!(b)\) and \((b)\!\implies\!(a)\).

  %------------------------------------------------------------
  \paragraph{\textbf{(a) $\boldsymbol{\Longrightarrow}$ (b).}}
  Assume $M$ is complete and let
  \(
      A_1 \supseteq A_2 \supseteq \dots
  \)
  satisfy the hypotheses of~(b).

  \begin{enumerate}[label=\arabic*.]
      \item \textbf{Pick one representative from each $A_n$.}  
            For every $n$ choose $a_n\in A_n$.

      \item \textbf{Show that $\{a_n\}$ is Cauchy.}  
            Given $\varepsilon>0$, pick
            $N$ so large that $\diam(A_N)<\varepsilon$.  
            Because $A_{N}\supseteq A_{N+1}\supseteq\dots$, we have
            $a_i,a_j\in A_N$ for all $i,j\ge N$, hence
            \(
                D(a_i,a_j)\le\diam(A_N)<\varepsilon.
            \)

      \item \textbf{Invoke completeness of $M$.}  
            The sequence $\{a_n\}$ is Cauchy, so by completeness it
            converges to some $b\in M$.

      \item \textbf{Locate $b$ in every $A_k$.}  
            Fix $k$.  Since $a_n\in A_k$ for all $n\ge k$ and $A_k$ is
            closed, the limit $b$ must lie in $A_k$.

      \item \textbf{Conclusion.}  
            Thus $b\in\bigcap_{n\ge1}A_n$, proving (b).
  \end{enumerate}

  %------------------------------------------------------------
  \paragraph{\textbf{(b) $\boldsymbol{\Longrightarrow}$ (a).}}
  Assume (b) holds and let $\{x_n\}$ be a Cauchy sequence in $M$.

  \begin{enumerate}[label=\arabic*.]
      \item \textbf{Create nested closed balls.}  
            For each $n$ put
            \(
                B_n:=\overline{B}\!\left(x_n,\;
                \displaystyle\sup_{k\ge n}D(x_n,x_k)\right).
            \)
            Because $\{x_n\}$ is Cauchy, the radii shrink to $0$, whence
            \(
              \diam(B_n)=\diam\bigl(\{x_n,x_{n+1},\dots\}\bigr)\longrightarrow 0.
            \)
            Moreover $B_{n+1}\subseteq B_n$.

      \item \textbf{Apply the nested‐set property.}  
            By (b) the intersection
            \(
              \bigcap_{n\ge1}B_n
            \)
            is non‑empty; choose $y$ in it.

      \item \textbf{Prove $x_n\to y$.}  
            For any $\varepsilon>0$ pick $N$ so large that
            \(\diam(B_N)<\varepsilon\).  
            Since $x_n,y\in B_N$ for $n\ge N$, we have
            \(D(x_n,y)\le\diam(B_N)<\varepsilon\).
            Hence $x_n\to y\in M$.

      \item \textbf{Conclusion.}  
            Every Cauchy sequence in $M$ converges; $M$ is complete.
  \end{enumerate}
\end{proof}

%------------------------------------------------
% Remarks
%------------------------------------------------
\begin{remark}
  \begin{enumerate}[label=\textup{\arabic*.}, leftmargin=2.8em]
      \item If $\diam(A_n)\to 0$, the intersection $\bigcap_n A_n$ contains
            at most one point.
      \item Dropping the requirement $\diam(A_n)\to 0$ yields a stronger
            property—\emph{compactness}—which is treated in Section~5.3.
  \end{enumerate}
\end{remark}
%------------------------------------------------
% Uniform continuity: definition and step‑by‑step example
%------------------------------------------------
\begin{definition}[Uniform continuity]
  Let $(X,D_X)$ and $(Y,D_Y)$ be metric spaces and  
  $f:X\to Y$ a function.  
  The map $f$ is \textbf{uniformly continuous} if
  \[
      \forall\,\varepsilon>0\;\exists\,\delta>0
      \quad\text{such that}\quad
      D_X(x,x')<\delta \;\Longrightarrow\;
      D_Y\bigl(f(x),f(x')\bigr)<\varepsilon
      \;\;\text{for \emph{all}}\,x,x'\in X.
  \]
  The key point is that $\delta$ depends \emph{only} on~$\varepsilon$,
  not on the particular pair $(x,x')$; hence the same $\delta$ works
  \emph{globally} across $X$.
\end{definition}

\bigskip
\noindent
%------------------------------------------------------------
%  Step 4 – f is NOT uniformly continuous  (expanded version)
%------------------------------------------------------------
\paragraph{Step 4 — $f$ is \emph{not} uniformly continuous.}

Fix\footnote{Any $\,\varepsilon>0$ would work; taking $\varepsilon=1$ makes
the arithmetic look clean.} $\varepsilon = 1$ and assume, for
contradiction, that a number $\delta>0$ exists such that
\[
   x,y\in X,\;d_X(x,y)<\delta
   \;\Longrightarrow\;
   |f(x)-f(y)|<\varepsilon.
   \tag{$\star$}
\]

\medskip
\noindent\textbf{(i)  Choose two \emph{very} close points of $X$.}
Because $\delta>0$, pick an \emph{even} integer $n$ so large that
\[
   \frac1n < \frac{\delta}{2}.  \tag{a}
\]
Now let $m:=n+1$ (which is \emph{odd}).  Then
\[
   d_X\!\Bigl(\tfrac1n,\tfrac1m\Bigr)
      =\Bigl|\frac1n-\frac1m\Bigr|
      =\frac{1}{n(n+1)}
      <\frac1n
      <\frac{\delta}{2}
      <\delta. \tag{b}
\]
Hence the two points $\tfrac1n,\tfrac1m\in X$ satisfy the
$\delta$‑proximity required in $(\star)$.

\medskip
\noindent\textbf{(ii)  But their $f$‑values stay $1$ apart.}
Because $n$ is even and $m$ is odd,
\[
   f\!\Bigl(\tfrac1n\Bigr)=0,
   \qquad
   f\!\Bigl(\tfrac1m\Bigr)=1,
   \quad\Longrightarrow\quad
   \Bigl|\,f\!\bigl(\tfrac1n\bigr)-f\!\bigl(\tfrac1m\bigr)\Bigr|=1=\varepsilon.
\]

\medskip
\noindent\textbf{(iii)  Contradiction.}  
The distance condition \((\star)\) is met by~(b), yet the
$f$‑values violate the desired bound $|f(x)-f(y)|<\varepsilon$.  
Therefore no such $\delta$ can exist, and $f$ fails to be uniformly
continuous on $X$.
\qed
\begin{theorem}\label{thm:uniform_cauchy}
  Let $(X,D_X)$ and $(Y,D_Y)$ be metric spaces and
  $f:X\to Y$ be \emph{uniformly continuous}.
  If $\{x_i\}_{i\ge 1}\subseteq X$ is a Cauchy sequence,  
  then $\{f(x_i)\}_{i\ge 1}\subseteq Y$ is also Cauchy.
\end{theorem}

\begin{proof}[Step‑by‑step proof]
  \textbf{1.  Recall the two hypotheses.}
  \begin{enumerate}[label=\alph*., wide, labelwidth=!, labelindent=0pt]
      \item \emph{Uniform continuity of $f$:}\\[2pt]
            $\displaystyle
            \forall\,\varepsilon>0\;\exists\,\delta>0
            \text{ such that }
            D_X(x,x')<\delta
            \;\Longrightarrow\;
            D_Y\!\bigl(f(x),f(x')\bigr)<\varepsilon
            \text{ for \emph{all} }x,x'\in X.$
      \item \emph{$\{x_i\}$ is Cauchy in $X$:}\\[2pt]
            $\displaystyle
            \forall\,\delta>0\;\exists\,N\in\mathbb N
            \text{ such that }
            D_X(x_i,x_j)<\delta
            \text{ whenever }i,j\ge N.$
  \end{enumerate}

  \medskip
  \textbf{2.  Fix an arbitrary accuracy level in the target space.}\\
  Let $\varepsilon>0$ be given.  
  Use uniform continuity to obtain a corresponding $\delta>0$
  (as in hypothesis a).

  \medskip
  \textbf{3.  Exploit the Cauchy property of the original sequence.}\\
  Because $\{x_i\}$ is Cauchy, choose $N\in\mathbb N$ such that  
  \[
      D_X(x_i,x_j)<\delta
      \qquad\text{for all }i,j\ge N.
  \]

  \medskip
  \textbf{4.  Transfer the estimate through $f$.}\\
  For any $i,j\ge N$ the above inequality triggers the
  uniform‑continuity implication, yielding
  \[
      D_Y\!\bigl(f(x_i),f(x_j)\bigr)<\varepsilon.
  \]

  \medskip
  \textbf{5.  Conclude that $\{f(x_i)\}$ is Cauchy.}\\
  We have shown that for the \emph{arbitrary} $\varepsilon>0$
  there exists $N$ (the same for the whole sequence) such that
  $D_Y\!\bigl(f(x_i),f(x_j)\bigr)<\varepsilon$ whenever $i,j\ge N$.
  This is precisely the Cauchy criterion in $(Y,D_Y)$.

  \medskip
  \textbf{6.  Statement proven.}\\
  Therefore a uniformly continuous map carries Cauchy sequences
  in the domain to Cauchy sequences in the codomain.
\end{proof}
%------------------------------------------------
% Extension of a uniformly continuous map
%------------------------------------------------
\begin{theorem}\label{thm:uc_extension}
  Let $A$ be a \emph{dense} subset of a metric space $(B,D_B)$ and let
  $f:A\to C$ be a \textbf{uniformly continuous} map into a \emph{complete}
  metric space $(C,D_C)$.
  \begin{enumerate}[label=\textup{(\alph*)}]
    \item There exists a \emph{unique} extension
          $\tilde f : B \to C$ that agrees with $f$ on $A$ and is
          \textbf{continuous}.
    \item The extension $\tilde f$ is \textbf{uniformly continuous}.
    \item If $f$ is an \emph{isometry} on $A$
      (i.e., $D_C\bigl(f(a_1),f(a_2)\bigr)=D_B(a_1,a_2)$ for all
      $a_1,a_2\in A$), then $\tilde f$ is an \textbf{isometry}
      on all of $B$.
\end{enumerate}
\end{theorem}

\begin{proof}
  We construct $\tilde f$ point‑by‑point and then verify the properties.

  %------------------------------------------------------------
  \paragraph{\textbf{Step 1 – Definition of $\tilde f$.}}
  Fix $b\in B$.  Because $A$ is dense, choose a sequence
  $\{a_i\}_{i\ge1}\subseteq A$ with $a_i\to b$ in $B$.
  By Theorem 51, $\{f(a_i)\}$ is a Cauchy sequence in $C$;
  completeness of $C$ gives a limit $c\in C$.
  \[
      \boxed{\;\tilde f(b):=\displaystyle\lim_{i\to\infty}f(a_i)\;}
  \]

  %------------------------------------------------------------
  \paragraph{\textbf{Step 2 – Well‑definedness (independence of sequence).}}
  Let $\{\alpha_i\}\subseteq A$ be another sequence with $\alpha_i\to b$.
  Interlace the two sequences to form
  \(
      a_1,\alpha_1,a_2,\alpha_2,\dots
  \),
  which still converges to $b$.  
  Applying $f$ yields a Cauchy sequence whose subsequences
  $\{f(a_i)\}$ and $\{f(\alpha_i)\}$ share the same limit, say $c$.
  Thus $\tilde f(b)$ does not depend on the particular approximating
  sequence.

  %------------------------------------------------------------
  \paragraph{\textbf{Step 3 – Extension property.}}
  If $b\in A$ we may take the constant sequence $a_i=b$,
  giving $\tilde f(b)=f(b)$.  Hence $\tilde f$ truly extends $f$.

  %------------------------------------------------------------
  \paragraph{\textbf{Step 4 – Uniform continuity of $\tilde f$.}}
  Take $\varepsilon>0$ and choose $\delta>0$ given by uniform continuity
  of $f$.  
  Suppose $b,\beta\in B$ satisfy $D_B(b,\beta)<\delta$.
  Select sequences $a_i,\;\alpha_i\in A$ with
  $a_i\to b$ and $\alpha_i\to\beta$.
  For large $i$ we have simultaneously
  \[
      D_B(a_i,\alpha_i)\;<\;\delta,
  \]
  hence
  \(
      D_C\!\bigl(f(a_i),f(\alpha_i)\bigr)<\varepsilon
  \)
  by uniform continuity of $f$.
  Passing to the limit and using continuity of $D_C$,
  \[
      D_C\!\bigl(\tilde f(b),\tilde f(\beta)\bigr)\le\varepsilon.
  \]
  The same $\delta$ works for \emph{all} pairs, proving uniform
  continuity; continuity is automatic.

  %------------------------------------------------------------
  \paragraph{\textbf{Step 5 – Uniqueness.}}
  If $g:B\to C$ is any other continuous extension of $f$,
  then for $b\in B$ we have $g(b)=\lim f(a_i)=\tilde f(b)$
  by continuity of $g$ along $a_i\to b$.  Hence $g=\tilde f$.

  %------------------------------------------------------------
  \paragraph{\textbf{Step 6 – Preservation of isometry (part (c)).}}
  Assume $f$ is an isometry on $A$ and take $b,\beta\in B$.
  Pick sequences $a_i\to b$, $\alpha_i\to\beta$ from $A$.
  Using the isometry property on $A$ and continuity of $D_B,D_C$,
  \[
      \boxed{\;
      D_B(b,\beta)=\lim_{i\to\infty}D_B(a_i,\alpha_i)
                 =\lim_{i\to\infty}D_C\!\bigl(f(a_i),f(\alpha_i)\bigr)
                 =D_C\!\bigl(\tilde f(b),\tilde f(\beta)\bigr)\; }.
  \]
  Thus $\tilde f$ is an isometry on $B$.
\end{proof}
%----------------------------------------
% Completion of a metric space – uniqueness
%----------------------------------------

\begin{definition}[Completion]
  Let $(X,D_X)$ and $(Y,D_Y)$ be metric spaces.  
  We say that \emph{$Y$ is a completion of $X$} if
  \[
      \text{$Y$ is complete \quad and \quad $X$ is a dense subspace of $Y$.}
  \]
\end{definition}

\begin{theorem}[Uniqueness of completion]\label{thm:completion_unique}
  Let $Y$ and $Z$ be two completions of the same metric space $X$.
  Then there exists an \textbf{isometry}
  \[
      \Phi:Y\;\longrightarrow\;Z
  \]
  such that $\Phi$ \emph{restricts to the identity} on $X$.
  Consequently, completions are unique up to a canonical isometry.
\end{theorem}

\begin{proof}[Step‑by‑step]
  \textbf{1.  Apply Theorem \ref{thm:uc_extension}.}  
  Consider the identity map
  \[
      f:X\longrightarrow Z,\qquad f(x)=x .
  \]
  It is uniformly continuous (trivially, with $\delta=\varepsilon$).  
  By Theorem \ref{thm:uc_extension}\,(a)–(b), $f$ extends uniquely to a
  \emph{uniformly continuous} map
  \[
      \widetilde f:Y\longrightarrow Z
  \]
  that coincides with $\mathrm{id}_X$ on the dense subset $X$.

  \medskip
  \textbf{2.  $\widetilde f$ is an isometry on $Y$.}  
  Because $f$ is the identity (hence an isometry) on $X$,
  the “furthermore’’ clause of Theorem \ref{thm:uc_extension}\,(c)
  implies
  \[
      D_Z\bigl(\widetilde f(y_1),\widetilde f(y_2)\bigr)
      =D_Y(y_1,y_2)
      \qquad\text{for all }y_1,y_2\in Y .
  \]

  \medskip
  \textbf{3.  Surjectivity of $\widetilde f$.}  
  Fix $z\in Z$.  
  Because $X$ is dense in $Z$, we can choose a sequence
  $(x_i)\subseteq X$ with $x_i\to z$ in $Z$.
  Completeness of $Y$ ensures that $(x_i)$—viewed inside $Y$—converges
  to some $y\in Y$.
  By continuity of $\widetilde f$,
  \[
      \widetilde f(y)=\lim_{i\to\infty}\widetilde f(x_i)
                    =\lim_{i\to\infty}x_i
                    =z .
  \]
  Hence $\widetilde f$ is onto.

  \medskip
  \textbf{4.  Identification of $\Phi$.}  
  Set $\Phi:=\widetilde f$.  
  Steps 2 and 3 show that $\Phi$ is a surjective isometry
  extending $\mathrm{id}_X$.
\end{proof}

\medskip
\noindent
Because any two completions of a metric space are canonically isometric, we
may unambiguously refer to \emph{the} completion of $X$.
%------------------------------------------------
% Uniform convergence and continuity
%------------------------------------------------

\begin{definition}[Uniform convergence]
  Let $(X,D_X)$ and $(Y,D_Y)$ be metric spaces.
  A sequence of functions $f_i:X\to Y$ \emph{converges uniformly} to a
  function $f:X\to Y$ (notation $f_i\!\xrightarrow{\;u\;} f$) if
  \[
      \forall\,\varepsilon>0\;\exists\,N\in\mathbb N
      \quad\text{such that}\quad
      D_Y\!\bigl(f_i(x),f(x)\bigr)<\varepsilon
      \quad\text{for all }x\in X\text{ and all }i\ge N.
  \]
  Equivalently,
  $\displaystyle\sup_{x\in X}D_Y\!\bigl(f_i(x),f(x)\bigr)\longrightarrow 0$
  as $i\to\infty$.
\end{definition}

\begin{theorem}\label{thm:uniform_limit_continuous}
  Let $f_i,f:X\to Y$ be functions between metric spaces with
  $f_i\!\xrightarrow{\;u\;} f$ and each $f_i$ \emph{continuous}.
  Then the limit function $f$ is \textbf{continuous}.
\end{theorem}

\begin{proof}[Step‑by‑step]
  \textbf{1.  Fix a candidate point of continuity.}  
  Let $x_0\in X$ be arbitrary; we will show $f$ is continuous at $x_0$.

  \medskip
  \textbf{2.  Choose an accuracy threshold.}  
  Given $\varepsilon>0$,
  uniform convergence provides an index $i$ such that
  \[
      D_Y\!\bigl(f_i(x),f(x)\bigr)<\frac{\varepsilon}{3}
      \qquad\text{for \emph{all} }x\in X.
  \]

  \medskip
  \textbf{3.  Use continuity of the chosen $f_i$.}  
  Because $f_i$ is continuous at $x_0$, there exists $\delta>0$ with
  \[
      D_X(x,x_0)<\delta
      \;\Longrightarrow\;
      D_Y\!\bigl(f_i(x),f_i(x_0)\bigr)<\frac{\varepsilon}{3}.
  \]

  \medskip
  \textbf{4.  Estimate $D_Y\bigl(f(x),f(x_0)\bigr)$.}  
  Whenever $D_X(x,x_0)<\delta$,
  \begin{align*}
      D_Y\!\bigl(f(x),f(x_0)\bigr)
      &\le D_Y\!\bigl(f(x),f_i(x)\bigr)
           +D_Y\!\bigl(f_i(x),f_i(x_0)\bigr)
           +D_Y\!\bigl(f_i(x_0),f(x_0)\bigr)\\[4pt]
      &<\frac{\varepsilon}{3}+\frac{\varepsilon}{3}+\frac{\varepsilon}{3}
      =\varepsilon.
  \end{align*}

  \medskip
  \textbf{5.  Conclude continuity.}  
  We have found $\delta>0$ such that
  $D_X(x,x_0)<\delta \Rightarrow D_Y\bigl(f(x),f(x_0)\bigr)<\varepsilon$.
  Since $\varepsilon>0$ was arbitrary, $f$ is continuous at $x_0$.
  As $x_0$ was arbitrary, $f$ is continuous on all of $X$.
\end{proof}

%------------------------------------------------
% Metric on the space of bounded continuous maps
%------------------------------------------------
\paragraph{Remark.}
For metric spaces $(X,D_X)$ and $(Y,D_Y)$ define
\[
  C(X,Y):=\Bigl\{\,g:X\to Y\;\Big|\; g\ \text{continuous and }
  \diam\bigl(g(X)\bigr)<\infty\Bigr\}.
\]
Equipping $C(X,Y)$ with the metric
\[
  D_{\!\infty}(f,g):=\sup_{x\in X}D_Y\!\bigl(f(x),g(x)\bigr)
\]
turns it into a metric space in which convergence is precisely uniform
convergence.
%------------------------------------------------
% Theorem 55 – Completeness of the function space $C(X,Y)$
%------------------------------------------------
\begin{theorem}\label{thm:CXY_complete}
  Let $(X,D_X)$ be \emph{any} metric space and $(Y,D_Y)$ a
  \emph{complete} metric space.
  Denote by
  \[
      C(X,Y)\;=\;\Bigl\{\,f:X\to Y \ \Bigm|\ 
          f \text{ is continuous and }
          \operatorname{diam}f(X)<\infty\Bigr\}
  \]
  the set of bounded continuous maps,
  equipped with the \emph{sup metric}
  \[
      D_\infty(f,g)\;:=\;\sup_{x\in X}D_Y\!\bigl(f(x),g(x)\bigr)
      \qquad (f,g\in C(X,Y)).
  \]
  Then $\bigl(C(X,Y),D_\infty\bigr)$ is a \textbf{complete} metric space.
\end{theorem}

\begin{proof}[Step‑by‑step]
  \textbf{1.  Start with a Cauchy sequence of functions.}  
  Let $\{f_i\}_{i\ge1}\subseteq C(X,Y)$ be Cauchy with respect to
  $D_\infty$; i.e.\
  \[
      \forall\,\varepsilon>0\;\exists\,N\
      \text{such that }D_\infty(f_i,f_j)<\varepsilon
      \text{ whenever }i,j\ge N. \tag{$\star$}
  \]

  \textbf{2.  Fix $x\in X$ and look pointwise.}  
  For each $x$ the sequence $\{f_i(x)\}$ in $Y$ satisfies
  \[
      D_Y\!\bigl(f_i(x),f_j(x)\bigr)\;\le\;D_\infty(f_i,f_j)
      \xrightarrow[i,j\to\infty]{}\;0,
  \]
  so $\{f_i(x)\}$ is Cauchy in $Y$; completeness of $Y$ provides a limit:
  \[
      f(x):=\lim_{i\to\infty}f_i(x)\quad\in Y. \tag{$\dagger$}
  \]

  \textbf{3.  Show that convergence is \emph{uniform}.}  
  Given $\varepsilon>0$, choose $N$ from ($\star$) with $\varepsilon/2$
  in place of $\varepsilon$.  For $i\ge N$ and \emph{all} $x\in X$,
  \[
      D_Y\!\bigl(f_i(x),f(x)\bigr)
          =\lim_{j\to\infty}D_Y\!\bigl(f_i(x),f_j(x)\bigr)
          \le\frac{\varepsilon}{2}.
  \]
  Hence $D_\infty(f_i,f)\le\varepsilon/2$ for $i\ge N$, so
  $f_i\xrightarrow{\;u\;}f$.

  \textbf{4.  Continuity of the limit map.}  
  Uniform limit of continuous functions is continuous
  (Theorem \ref{thm:uniform_limit_continuous}); thus $f\in C(X,Y)$.

  \textbf{5.  Boundedness of $f$.}  
  From uniform convergence we have
  \(
      D_\infty(f_i,f)<1
  \)
  for some $i$, which implies
  \(
      \operatorname{diam}f(X)\le
      \operatorname{diam}f_i(X)+2<\infty.
  \)

  \textbf{6.  Completion achieved.}  
  We have produced $f\in C(X,Y)$ with $D_\infty(f_i,f)\to0$.
  Therefore every Cauchy sequence converges; $\bigl(C(X,Y),D_\infty\bigr)$
  is complete.
\end{proof}

%------------------------------------------------
% Theorem 56 – Isometric embedding of $X$ into $C(X,\mathbb R)$
%------------------------------------------------
\begin{theorem}\label{thm:Kuratowski_embedding}
  Let $(X,D_X)$ be a metric space and fix a point $a\in X$.
  Define, for each $u\in X$, the function
  \[
      f_u:X\longrightarrow\mathbb R,
      \qquad
      f_u(x):=D_X(u,x)-D_X(a,x).
  \]
  Then the mapping
  \[
      \iota:X\longrightarrow C\bigl(X,\mathbb R\bigr),
      \qquad
      \iota(u)=f_u,
  \]
  is an \textbf{isometry}; i.e.\
  \(
      D_\infty\!\bigl(f_u,f_v\bigr)=D_X(u,v)
      \text{ for all }u,v\in X.
  \)
\end{theorem}
%-----------------------------------------------------------
%  Why Theorem 16 (Kuratowski / Fréchet embedding) matters
%-----------------------------------------------------------
\begin{center}
  \textbf{Significance of Theorem~16}
  \end{center}
  
  \smallskip
  \noindent
  Let
  \[
     \iota:X\longrightarrow C(X,\mathbb R),\qquad
     \iota(u)=f_u,\qquad
     f_u(x)=D_X(u,x)-D_X(a,x),
  \]
  be the isometric embedding furnished by Theorem~16, where
  \(C(X,\mathbb R)\) is endowed with the sup–metric
  \(D_\infty(\varphi,\psi)=\sup_{x\in X}|\varphi(x)-\psi(x)|\).
  Its importance can be summarised as follows:
  
  \begin{enumerate}
    \item \textbf{Concrete linear model.}
          Every metric space becomes a \emph{literal} subset of the Banach
          space \((C(X,\mathbb R),\|\cdot\|_\infty)\), with all distances
          preserved.
          This is the classical \emph{Kuratowski} (or \emph{Fréchet})
          embedding.
  
    \item \textbf{Immediate access to completions.}
          Because \(C(X,\mathbb R)\) is complete, the closure
          \(\overline{\iota(X)}\) automatically supplies a completion of
          \(X\); uniqueness of completions follows by identifying any two
          via the canonical isometry.
  
    \item \textbf{Functional–analytic toolkit.}
          Inside a Banach space one can form convex combinations, use
          Hahn–Banach separation, apply duality, etc.
          The embedding lets such linear methods attack problems that were
          originally stated purely in metric terms.
  
    \item \textbf{Universal property for \(1\)-Lipschitz maps.}
          The space \(\ell^\infty(X)=C(X,\mathbb R)\) is
          \emph{injective}: every \(1\)-Lipschitz map
          \(g:X\to Y\) extends to a \(1\)-Lipschitz linear map
          on \(\overline{\iota(X)}\).
          This categorical viewpoint is central in modern metric geometry.
  
    \item \textbf{Foundation for Gromov–Hausdorff distance.}
          To measure how \emph{close} two metric spaces are, embed both
          isometrically into a common \(\ell^\infty\)–type space via
          \(\iota\) and compute the Hausdorff distance of their images;
          the theorem guarantees such embeddings always exist.
  
    \item \textbf{Role of the base point \(a\).}
          Subtracting \(D_X(a,x)\) merely renders each \(f_u\) \emph{bounded}:
          \(|f_u(x)|\le D_X(u,a)\) by the triangle inequality.
          Any other choice of \(a\) yields an equally good embedding,
          differing only by an additive constant function.
  \end{enumerate}
  
  \medskip
  \noindent
  \emph{Key takeaway.}
  Theorem~16 shows that \textbf{every metric space already hides inside a
  complete normed space in a distance‐preserving way}, unlocking powerful
  linear techniques for metric problems.
\begin{proof}
  \textbf{1.  Continuity and boundedness of $f_u$.}  
  Each $f_u$ is continuous (difference of continuous functions) and,
  by the triangle inequality,
  \(
      |f_u(x)|\le D_X(u,a)
  \)
  for all $x$, so $f_u\in C(X,\mathbb R)$.

  \textbf{2.  Compute the sup distance.}  
  For $u,v\in X$,
  \[
      D_\infty\!\bigl(f_u,f_v\bigr)
      =\sup_{x\in X}\bigl|D_X(u,x)-D_X(v,x)\bigr|
      \overset{(\ast)}{=}D_X(u,v),
  \]
  where $(\ast)$ follows because  
  $\bigl|D_X(u,x)-D_X(v,x)\bigr|\le D_X(u,v)$ (triangle inequality),
  and equality is attained at $x=v$ (or $x=u$).

  \textbf{3.  Conclusion.}  
  Thus $\iota$ preserves distances and embeds $X$ isometrically into the
  complete space $C(X,\mathbb R)$.
\end{proof}

\medskip
Combining Theorems \ref{thm:CXY_complete},
\ref{thm:Kuratowski_embedding},
and the fact that closed subsets of complete spaces are complete
(Theorem 48), we obtain the classical construction of the
\emph{completion} of an arbitrary metric space by closing the embedded
copy $\iota(X)$ inside $C(X,\mathbb R)$.
%------------------------------------------------
% Theorem 57 – Every metric space admits a completion
%------------------------------------------------
\begin{theorem}[Existence of a completion]\label{thm:completion_exist}
  For every metric space $(X,D)$ there exists a \emph{complete}
  metric space $(\widehat{X},\widehat{D})$ that contains an
  isometric copy of $X$ as a \textbf{dense} subset.
  Such a pair $(\widehat{X},\widehat{D})$ is called a
  \emph{completion} of $X$.
\end{theorem}

\begin{proof}[Step‑by‑step construction (Kuratowski embedding)]
  We build $\widehat{X}$ explicitly and then verify the required
  properties.

  %------------------------------------------------------------
  \paragraph{\textbf{Step 1 – Embed $X$ into a known complete space.}}
  Fix a reference point $a\in X$.
  For each $u\in X$ define
  \[
      f_u:X\longrightarrow\mathbb R,
      \qquad
      f_u(x):=D(u,x)-D(a,x).
  \]
  \begin{enumerate}[label=\alph*.]
      \item \textbf{Continuity.}  
            Each $f_u$ is continuous (difference of continuous maps).
      \item \textbf{Boundedness.}  
            $\displaystyle|f_u(x)|
            \le D(u,a)$ by the triangle inequality, so
            $f_u$ is \emph{bounded}.
  \end{enumerate}
  Hence $f_u$ lies in
  \[
      C(X,\mathbb R):=\Bigl\{g:X\to\mathbb R\;\Bigm|\;
          g\text{ continuous and } \operatorname{diam}g(X)<\infty\Bigr\},
  \]
  which we equip with the sup‑metric
  $
      D_\infty(g,h):=\sup_{x\in X}|g(x)-h(x)|.
  $

  \smallskip
  \emph{Isometry.}  
  The map
  \[
      \iota:X\to C(X,\mathbb R),\qquad
      \iota(u)=f_u,
  \]
  is an isometry:
  \[
      D_\infty\!\bigl(f_u,f_v\bigr)
      =\sup_{x\in X}\bigl|D(u,x)-D(v,x)\bigr|
      =D(u,v)\quad(u,v\in X).
  \]

  %------------------------------------------------------------
  \paragraph{\textbf{Step 2 – Take the closure inside a complete space.}}
  By Theorem 55 the space
  $\bigl(C(X,\mathbb R),D_\infty\bigr)$ is complete.
  Define
  \[
      \widehat{X}:=\overline{\iota(X)}
      \subseteq C(X,\mathbb R),
      \qquad
      \widehat{D}:=D_\infty\big|_{\widehat{X}\times\widehat{X}}.
  \]
  Because $\widehat{X}$ is a \emph{closed} subset of a complete space,
  Theorem 48 implies that $(\widehat{X},\widehat{D})$ is complete.

  %------------------------------------------------------------
  \paragraph{\textbf{Step 3 – Density of $X$ in $\widehat{X}$.}}
  By construction $\iota(X)$ is dense in its own closure
  $\widehat{X}$, so $\iota(X)$ is dense in $\widehat{X}$.

  %------------------------------------------------------------
  \paragraph{\textbf{Step 4 – Identify $X$ with its image.}}
  Since $\iota$ is an isometry, we may regard $X$ as a
  subspace of $\widehat{X}$ via $\iota$ without altering distances.
  With this identification, $X$ is dense in $\widehat{X}$ and
  $\widehat{X}$ is complete.

  %------------------------------------------------------------
  \paragraph{\textbf{Step 5 – Conclusion.}}
  The pair $(\widehat{X},\widehat{D})$ constructed above is a
  completion of $(X,D)$, establishing the theorem.
\end{proof}
%------------------------------------------------
% 5.2  \;Separability
%------------------------------------------------

\section*{5.2 \quad Separability}

\begin{definition}[Separable space]
    A metric space $(M,D)$ is called \textbf{separable} if it
    possesses a \emph{countable dense} subset;
    i.e.\ there exists $\{x_1,x_2,\dots\}\subseteq M$ such that
    $\overline{\{x_n\}}=M$.
\end{definition}

\medskip
\noindent
{\small
\emph{Examples.}  $\mathbb R$ with the usual metric is separable
(countable dense set $\mathbb Q$).  
More generally, $\mathbb R^{n}$ is separable via points with rational
coordinates.  
By contrast, an \emph{uncountable discrete} space is not separable
because its only dense subset is the entire uncountable space itself.
}

\bigskip
\begin{definition}[Open base]
    A collection $\mathcal U$ of open subsets of a metric space $M$
    is an \textbf{open base} (or simply a \emph{basis})
    if every open set in $M$ can be written as a union of members of
    $\mathcal U$.
\end{definition}

\medskip
\noindent
The family of \emph{all} open balls forms a base, but one can often find
a smaller, more economical base.  
The theorem below shows how separability provides a \emph{countable} base.

\bigskip
\begin{theorem}[Countable base from a dense set]\label{thm:countable_base}
    Let $\{x_i\}_{i=1}^{\infty}$ be a \emph{dense} subset of a metric space
    $(M,D)$.  
    Denote by $\mathcal U$ the collection of all open balls
    \[
        \mathcal U
        :=\Bigl\{\,B\!\bigl(x_i,r\bigr)\ \Bigm|\ 
            i\in\mathbb N,\ r\in\mathbb Q_{>0}\Bigr\}.
    \]
    Then $\mathcal U$ is a (countable) open base for $M$.
\end{theorem}

\begin{proof}[Step‑by‑step]
    \textbf{1.  Countability of $\mathcal U$.}\;
    The index set is $\mathbb N\times\mathbb Q_{>0}$, which is countable.

    \medskip
    \textbf{2.  Goal.}\;
    Show every open set $V\subseteq M$ equals
    $\displaystyle\bigcup\mathcal U_V$ for some subfamily
    $\mathcal U_V\subseteq\mathcal U$.

    \medskip
    \textbf{3.  Fix $y\in V$.}\;
    Because $V$ is open, there exists an open ball
    $B\bigl(y,r\bigr)\subseteq V$ with radius $r>0$.

    \medskip
    \textbf{4.  Choose a rational radius.}\;
    Select $q\in\mathbb Q_{>0}$ with $0<q<r$.

    \medskip
    \textbf{5.  Use density of $\{x_i\}$.}\;
    Since the set $\{x_i\}$ is dense, pick an index $j$ so close to $y$
    that $D\!\bigl(x_j,y\bigr)<q/2$.

    \medskip
    \textbf{6.  Construct a basis ball inside $V$.}\;
    Consider $B:=B\!\bigl(x_j,q\bigr)$.  
    For our chosen $y$ we have
    \[
        D\!\bigl(y,x_j\bigr)<\frac{q}{2}<\frac{q}{2}+0<q,
    \]
    so $y\in B$.  
    Moreover, for any $z\in B$,
    \[
        D(y,z)\le D\!\bigl(y,x_j\bigr) + D\!\bigl(x_j,z\bigr)
                 < \frac{q}{2} + q
                 < r,
    \]
    hence $z\in B(y,r)\subseteq V$.  
    Thus $B\subseteq V$.
    % Why do we have to check every \(z\in B(x_j,q)\) when proving 
% \(B(x_j,q)\subseteq B(y,r)\subseteq V\)?

\begin{enumerate}
  \item \textbf{Objective.}  
        To establish the set‐inclusion  
        \[
            B(x_j,q)\subseteq B(y,r)\subseteq V,
        \]
        we must verify that \emph{every} element of \(B(x_j,q)\) also belongs
        to \(B(y,r)\).  Set containment is an “all–points’’ property, so a
        single‐point check is insufficient.

  \item \textbf{Take an arbitrary point.}  
        Let \(z\in B(x_j,q)\); by definition of an open ball,
        \[
            D(x_j,z)<q.
        \]

  \item \textbf{Invoke the triangle inequality.}  
        From the construction we have \(D(y,x_j)<\dfrac{q}{2}\).  
        Hence
        \[
            D(y,z)\;\le\;D(y,x_j)+D(x_j,z)
                    \;<\;\frac{q}{2}+q
                    \;=\;\frac{3q}{2}.
        \]

  \item \textbf{Use the size of \(q\).}  
        We chose a rational radius \(q\) so small that \(0<q<\tfrac{2}{3}r\)
        (in particular \(q<r\)).  Therefore
        \[
            D(y,z)\;<\;\frac{3q}{2}\;<\;r,
        \]
        which means \(z\in B(y,r)\).

  \item \textbf{Conclude inclusion.}  
        Because the point \(z\) was \emph{arbitrary} in \(B(x_j,q)\), the
        calculation shows
        \[
            B(x_j,q)\subseteq B(y,r)\subseteq V.
        \]
        Thus \(B(x_j,q)\subseteq V\), completing the proof that the
        countable family \(\mathcal{U}\) is indeed a base for \(M\).
\end{enumerate}

    \medskip
    \textbf{7.  Cover $V$ by balls from $\mathcal U$.}\;
    As $y$ was arbitrary in $V$, each point of $V$ lies in some ball
    from $\mathcal U$ that is itself contained in $V$.  
    Therefore
    \[
        V=\bigcup\Bigl\{\,B\in\mathcal U \mid B\subseteq V\Bigr\},
    \]
    proving that $\mathcal U$ is an open base.
\end{proof}

%------------------------------------------------
% Theorem 59 – Equivalences among separability, countable base, and Lindelöf
%------------------------------------------------
\begin{theorem}\label{thm:sep_base_lindelof}
  For a metric space $(M,D)$ the following properties are equivalent:
  \begin{enumerate}[label=\textup{(\alph*)}]
      \item $M$ is \emph{separable} (admits a countable dense set);
      \item $M$ possesses a \emph{countable open base};
      \item Every open cover of $M$ contains a \emph{countable subcover}
            (the Lindelöf property).
  \end{enumerate}
\end{theorem}

\begin{proof}
  We show $(a)\!\Rightarrow\!(b)\!\Rightarrow\!(c)\!\Rightarrow\!(a)$.

  %------------------------------------------------------------
  \paragraph{\textbf{(a) $\;\Longrightarrow\;$ (b).}}
  Assume $M$ has a countable dense subset $\{x_1,x_2,\dots\}$.
  Form the collection
  \[
      \mathcal U
      :=\Bigl\{\,B\!\bigl(x_i,r\bigr)\ \Bigm|\ 
          i\in\mathbb N,\ r\in\mathbb Q_{>0}\Bigr\}.
  \]
  \begin{itemize}
      \item $\mathcal U$ is \emph{countable} because it is indexed by
            $\mathbb N\times\mathbb Q_{>0}$.
      \item Given an arbitrary open set $V\subseteq M$ and $y\in V$,
            pick $r>0$ with $B(y,r)\subseteq V$; choose rational
            $q\in(0,r)$.  
            Density supplies $x_j$ with $D(y,x_j)<q/2$.
            The ball $B(x_j,q)\in\mathcal U$
            contains $y$ and lies in $V$.
            Hence $V=\bigcup\mathcal U_V$ for a subfamily
            $\mathcal U_V\subseteq\mathcal U$,
            showing $\mathcal U$ is an open base.
  \end{itemize}

  %------------------------------------------------------------
  \paragraph{\textbf{(b) $\;\Longrightarrow\;$ (c).}}
  Suppose $\mathcal U=\{U_i\}_{i\in\mathbb N}$ is a countable base for $M$
  and let $\mathcal V=\{V_j\}_{j\in J}$ be an \emph{arbitrary} open cover.
  For each $U_i$ that is contained in \emph{some} $V_j$,
  choose one such $V_j$ and keep it; discard the rest.
  The survivors form a countable family
  $\{V_{j(i)}\}_{i\in I}\subseteq\mathcal V$.
  To see they still cover $M$, fix $x\in M$:
  there is $V_k\in\mathcal V$ with $x\in V_k$,
  and some $U_m\in\mathcal U$ with $x\in U_m\subseteq V_k$.
  By construction $V_{j(m)}$ was retained and contains $U_m$, hence $x$.
  Thus $\{V_{j(i)}\}$ is a countable subcover of $\mathcal V$.

  %------------------------------------------------------------
  \paragraph{\textbf{(c) $\;\Longrightarrow\;$ (a).}}
  Assume every open cover of $M$ admits a countable subcover.
  For each $n\in\mathbb N$ cover $M$ with the collection of open balls
  $\{B(x,1/n)\}_{x\in M}$.
  Extract a \emph{countable} subcover
  $\{B(x_{n,k},1/n)\}_{k\in\mathbb N}$.
  Let
  \[
      T_n:=\{x_{n,k}\mid k\in\mathbb N\}
      \quad\text{and}\quad
      D:=\bigcup_{n=1}^{\infty}T_n.
  \]
  \begin{itemize}
      \item $D$ is \emph{countable} (countable union of countable sets).
      \item For any $y\in M$ and $\varepsilon>0$ choose $n$ with $1/n<\varepsilon$.
            Some ball $B(x_{n,k},1/n)$ contains $y$, so
            $D(y,x_{n,k})<\varepsilon$.  
            Hence $D$ is \emph{dense} in $M$.
  \end{itemize}
  Therefore $M$ is separable.
  %  What does the symbol \(x_{n,k}\) stand for in the proof of \((c)\Rightarrow(a)\)?

\begin{enumerate}
  \item[\textbf{Step 1.}] 
      \textbf{Fix a radius.}  
      For each positive integer \(n\in\mathbb N\) we look at the open‐cover
      of the metric space \((M,D)\) consisting of \emph{all} balls of
      radius \(1/n\):
      \[
          \mathcal V_n \;=\;\bigl\{\,B(x,1/n)\mid x\in M\,\bigr\}.
      \]
      Observe that \(\mathcal V_n\) may be uncountable, because there could
      be uncountably many centres \(x\).

  \item[\textbf{Step 2.}] 
      \textbf{Apply the Lindelöf (countable‐subcover) property.}  
      Hypothesis (c) says \emph{every} open cover has a countable subcover.
      Therefore, for our particular cover \(\mathcal V_n\) there exists a
      countable family of balls
      \[
          \bigl\{\,B(x_{n,k},1/n)\mid k\in\mathbb N\,\bigr\}\;\subseteq\;\mathcal V_n
      \]
      that still covers \(M\).

  \item[\textbf{Step 3.}] 
      \textbf{Meaning of the notation \(x_{n,k}\).}  
      \begin{itemize}
          \item The first index \(n\) records the \emph{radius} \(1/n\).  
                For each fixed \(n\) we are working with balls of that
                radius.
          \item The second index \(k\) simply enumerates (1,2,3,\dots) the
                balls in the countable subcover obtained for that \(n\).
                Concretely,
                \[
                    B(x_{n,1},1/n),\;B(x_{n,2},1/n),\;B(x_{n,3},1/n),\;\ldots
                \]
                are the first, second, third, \emph{etc.} balls whose
                centres lie somewhere in \(M\).
          \item There is \emph{no} canonical choice for these centres; we
                just know \emph{some} such countable subcover exists, and we
                label its centres \(x_{n,1},x_{n,2},\dots\).
      \end{itemize}

  \item[\textbf{Step 4.}] 
      \textbf{Collecting the centres.}  
      For each fixed \(n\) define
      \[
          T_n \;:=\;\bigl\{\,x_{n,k}\mid k\in\mathbb N\,\bigr\},
      \]
      the (countable) set of all centres appearing in the \(n\)-th
      subcover.  Then set
      \[
          D \;:=\;\bigcup_{n=1}^\infty T_n.
      \]
      Because \(D\) is a countable union of countable sets, it is itself
      countable.

  \item[\textbf{Step 5.}] 
      \textbf{Why \(D\) is dense.}  
      Given any point \(y\in M\) and any \(\varepsilon>0\), choose
      \(n\in\mathbb N\) with \(1/n<\varepsilon\).  Since
      \(\{B(x_{n,k},1/n)\}_{k\in\mathbb N}\) covers \(M\),
      there is some \(k\) such that \(y\in B(x_{n,k},1/n)\); equivalently
      \(D(y,x_{n,k})<1/n<\varepsilon\).  Thus every open ball contains at
      least one element of \(D\), proving \(D\) is dense in \(M\).

  \item[\textbf{Conclusion.}]  
      The symbols \(x_{n,k}\) therefore denote the centres of the kth ball
      in a countable subcover of the radius‑\(1/n\) open‐ball cover of \(M\).
      Their union over all \(n,k\) furnishes the desired countable dense
      subset that witnesses separability.
\end{enumerate}

  \medskip\noindent
  Since we have completed the cycle of implications,
  the three statements are equivalent.
\end{proof}
%------------------------------------------------
% Cardinality of separable spaces and condensation points
%------------------------------------------------

%------------------  Theorem 60  ------------------
\begin{theorem}\label{thm:card_separable}
  Let $(M,D)$ be a \emph{separable} metric space.
  Then the cardinality of $M$ is \emph{at most} $\frak c$
  (the cardinality of the continuum~$\mathbb R$).
\end{theorem}

\begin{proof}[Step-by-step]
  \textbf{1.  Pick a countable dense set.}\;
  Because $M$ is separable there exists a dense subset
  $\{x_1,x_2,\dots\}$ of cardinality~$\aleph_0$.

  \medskip
  \textbf{2.  Encode points of $M$ by convergent sequences.}\;
  For each $y\in M$ choose a sequence of indices
  $(i_1,i_2,i_3,\dots)\in\mathbb N^{\mathbb N}$ such that
  $x_{i_k}\to y$.
  Thus we obtain a \emph{surjective} map
  \[
      \Phi:\mathbb N^{\mathbb N}\twoheadrightarrow M,
      \qquad
      \Phi\bigl((i_k)_{k}\bigr):=\lim_{k\to\infty}x_{i_k}.
  \]
%  Why does the index sequence \((i_1,i_2,i_3,\dots)\) lie in \(\mathbb N^{\mathbb N}\)?

\begin{enumerate}
  \item \textbf{Definition of \(\mathbb N^{\mathbb N}\).}  
        The notation 
        \[
             \mathbb N^{\mathbb N}
        \]
        denotes the set of all functions
        \(\sigma:\mathbb N\longrightarrow\mathbb N\).  
        Equivalently, it is the set of all infinite sequences 
        \((\sigma(1),\sigma(2),\sigma(3),\dots)\) of natural numbers.  
        One often writes
        \[
            \mathbb N^{\mathbb N}
            \;=\;
            \bigl\{(a_1,a_2,a_3,\dots)\mid a_k\in\mathbb N\bigr\}.
        \]

  \item \textbf{Our chosen sequence.}  
        For each \(y\in M\) we pick indices
        \[
            i_1,i_2,i_3,\dots \;\in\; \mathbb N
        \]
        such that \(x_{i_k}\to y\).  
        Thus we have a function
        \[
            \begin{aligned}
                \sigma:\mathbb N &\longrightarrow \mathbb N,\\
                k &\longmapsto i_k,
            \end{aligned}
        \]
        or, in sequence notation, \(\sigma=(i_1,i_2,i_3,\dots)\).

  \item \textbf{Verification of membership.}  
        Because every coordinate \(i_k\) is a natural number,
        \(\sigma\) satisfies the defining property of an element of
        \(\mathbb N^{\mathbb N}\).  
        Hence
        \[
            (i_1,i_2,i_3,\dots)\;\in\;\mathbb N^{\mathbb N}.
        \]

  \item \textbf{Why this matters in the proof.}  
        Once we know that \(\sigma\in\mathbb N^{\mathbb N}\), the mapping
        \[
            \Phi:\mathbb N^{\mathbb N}\twoheadrightarrow M,
            \qquad
            \Phi(\sigma)=\lim_{k\to\infty}x_{\sigma(k)}
        \]
        is well‑defined and surjective.  
        Counting the possible \(\sigma\)’s (there are \(\mathfrak c\) of
        them) gives \(|M|\le\mathfrak c\).
\end{enumerate}
  \medskip
  \textbf{3.  Count the possible sequences.}\;
  The set $\mathbb N^{\mathbb N}$ of all sequences of natural numbers has
  cardinality $\aleph_0^{\aleph_0}=\frak c$.

  \medskip
  \textbf{4.  Conclude the inequality.}\;
  Since $|M|\le|\mathbb N^{\mathbb N}|=\frak c$, the result follows.
\end{proof}

%------------------  Definition ------------------
\begin{definition}[Condensation point]
  A point $x\in M$ is called a \textbf{condensation point} of $M$ if
  \emph{every} neighbourhood of $x$ contains \emph{uncountably} many
  points of~$M$.
\end{definition}

%------------------  Theorem 61  ------------------
\begin{theorem}\label{thm:condensation}
  Let $M$ be an \emph{uncountable} separable metric space.
  Then \emph{all but countably many} points of $M$ are
  condensation points.
\end{theorem}


\begin{proof}[Step-by-step]
  
  \textbf{Step 1 -- Existence of some condensation point.}

  Assume, to obtain a contradiction, that no point of \(M\) is a
  condensation point.  Then each \(x\in M\) possesses a countable open
  neighborhood \(N_x\) (i.e.\ \(\lvert N_x\rvert\) is countable).
  The family \(\{N_x\}_{x\in M}\) is an open cover of \(M\).
  By Theorem 59(c) (countable subcover property of separable spaces),
  there exists a countable subcover \(\{N_{x_k}\}_{k\in\mathbb N}\).
  The union of these countable sets is countable, contradicting the
  assumption that \(M\) itself is uncountable.  Hence at least one
  condensation point exists.
  %  Why is the union of countably many countable sets itself countable?
%
%  Context (Step 1 of Theorem 21):  we have a countable subcover
%  $\{N_{x_k}\}_{k\in\mathbb N}$ of $M$ where each $N_{x_k}$ is a
%  \emph{countable} set.  We must show
%  \[
%        \bigcup_{k=1}^{\infty} N_{x_k}
%        \quad\text{is countable.}
%  \]
%  Here is a standard argument.

\begin{enumerate}
  \item[\textbf{1.}] %
      \textbf{Fix enumerations.}  
      Because every $N_{x_k}$ is countable, we may list its elements as
      \[
          N_{x_k}=\{\,y_{k,1},y_{k,2},y_{k,3},\dots\}
          \qquad(k=1,2,\dots).
      \]
      Thus every point in the union has a \emph{pair} of indices
      $(k,m)\in\mathbb N\times\mathbb N$ with
      $y_{k,m}\in N_{x_k}$.

  \item[\textbf{2.}] %
      \textbf{Enumerate $\mathbb N\times\mathbb N$.}  
      The Cartesian product $\mathbb N\times\mathbb N$ is countable; an
      explicit bijection is given by the classical “Cantor pairing’’
      function
      \[
          \pi(k,m)=\frac{(k+m-2)(k+m-1)}{2}+k,
      \]
      or, more concretely, by listing the pairs in antidiagonals:
      \[
          (1,1),\;(1,2),(2,1),\;(1,3),(2,2),(3,1),\;\ldots
      \]
      Hence there exists a bijection
      \(
          f:\mathbb N\longrightarrow\mathbb N\times\mathbb N.
      \)

  \item[\textbf{3.}] %
      \textbf{Transfer the enumeration to the union.}  
      Define a map
      \[
          g:\mathbb N\;\longrightarrow\;\bigcup_{k=1}^{\infty}N_{x_k},
          \qquad
          g(n)=y_{\,f(n)_1,\,f(n)_2},
      \]
      where $f(n)=(k,m)$ and $f(n)_1=k,\;f(n)_2=m$ extract the two
      coordinates.  By construction, each element of the union is hit at
      least once, so $g$ is surjective.

  \item[\textbf{4.}] %
      \textbf{Conclude countability.}  
      A surjective map from $\mathbb N$ onto a set proves that set is
      \emph{at most} countable.  Therefore
      \[
          \bigcup_{k=1}^{\infty} N_{x_k}
          \quad\text{is countable.}
      \]
      (If desired, duplicates in the enumeration can be removed to obtain
      an explicit bijection rather than merely a surjection.)
\end{enumerate}

\noindent
Hence the union of countably many countable sets is countable, which is the
fact used in Step 1 of the proof of Theorem 21.  \qedhere

  \medskip
  \textbf{Step 2 -- Show the exceptional set is countable.}

  Let
  \[
    Z := \{\text{condensation points of }M\}, 
    \quad 
    E := M\setminus Z.
  \]
  Suppose \(E\) were uncountable.  Since \(E\subseteq M\) and \(M\) is
  separable, \(E\) is also separable; let \(\{e_n\}\) be a countable
  dense subset of \(E\).  For each \(n\), choose an open ball
  \(B(e_n,r_n)\) whose intersection with \(E\) is countable (possible
  because \(e_n\notin Z\)).  The countable family
  \(\mathcal B := \{B(e_n,r_n)\}\) covers \(E\).  Adding the single
  open set \(M\setminus E\) gives an open cover
  \(\mathcal B\cup\{M\setminus E\}\) of \(M\).  Again by Theorem 59(c)
  there is a countable subcover; removing \(M\setminus E\) if necessary
  still leaves a countable cover of \(E\) by sets each meeting \(E\) in
  only countably many points.  Thus \(E\) is a countable union of
  countable sets and hence countable—a contradiction.
  %  Why does a subset \(E\subseteq M\) inherit separability from \(M\)?

\begin{proof}[Proof that \(E\) is separable if \(M\) is separable]
  Let \((M,D)\) be a separable metric space.  
  By definition, there exists a \emph{countable} set
  \[
      D_0=\{d_1,d_2,d_3,\dots\}\subseteq M
  \]
  that is \emph{dense} in \(M\); i.e.\ every non‑empty open set of \(M\)
  meets \(D_0\).
  
  \medskip
  Define 
  \[
      D_E\;:=\;D_0\cap E.
  \]
  \begin{enumerate}
      \item[\textbf{(i)}] \textbf{Countability.}  
            Because \(D_0\) is countable, any subset of it—such as
            \(D_E\)—is also countable (possibly finite or even empty if
            \(E=\varnothing\)).  Thus \(D_E\) satisfies the “countable’’ part
            of separability.
            
      \item[\textbf{(ii)}] \textbf{Density in \(E\).}  
            Consider \(E\) with the \emph{subspace metric} inherited from
            \(M\).  Let \(U\subseteq E\) be a non‑empty open set in this
            subspace topology.  By definition of the subspace topology, there
            exists an open set \(V\subseteq M\) such that
            \[
                U \;=\; V\cap E.
            \]
            Since \(D_0\) is dense in \(M\), we have \(D_0\cap V\neq\varnothing\).
            Consequently
            \[
                D_E\cap U
                \;=\;(D_0\cap E)\cap(V\cap E)
                \;=\;D_0\cap V\cap E
                \;=\;D_0\cap V
                \;\;\neq\;\varnothing.
            \]
            Hence every non‑empty open subset of \(E\) meets \(D_E\), so
            \(D_E\) is dense in \(E\).
  \end{enumerate}
  
  Because \(D_E\) is \emph{countable} and \emph{dense} in \(E\), the subspace
  \((E,D|_{E\times E})\) is separable.  \qedhere
  \end{proof}
  %  Let us make the definition of \(E\) explicit.
%  Suppose \((M,D)\) is a separable metric space and let
%  \[
%      E\subseteq M
%  \]
%  be any (non‑empty) subset endowed with the subspace metric
%  \(D|_{E\times E}\).
%  We show that \(E\) is separable.

\begin{proof}[Proof that \(E\) is separable]
  Because \(M\) is separable, there exists a countable dense set
  \[
      D_0=\{d_1,d_2,d_3,\dots\}\subseteq M.
  \]
  
  \medskip
  \noindent\textbf{Step 1 – Construct a candidate dense set for \(E\).}
  Define
  \[
      D_E \;:=\; D_0\cap E.
  \]
  
  \begin{enumerate}
      \item[\textbf{(i)}] \emph{Countability.}
            Since \(D_0\) is countable, any subset of it—particularly
            \(D_E\)—is also countable.
  
      \item[\textbf{(ii)}] \emph{Density in \(E\).}
            Let \(U\subseteq E\) be a non‑empty open set in the
            subspace topology.  By definition, there is an open set
            \(V\subseteq M\) such that \(U = V\cap E\).
            Because \(D_0\) is dense in \(M\), we have \(D_0\cap V\neq\varnothing\).
            Hence
            \[
                D_E\cap U
                \;=\;(D_0\cap E)\cap(V\cap E)
                \;=\;D_0\cap V\cap E
                \;=\;D_0\cap V
                \;\neq\;\varnothing.
            \]
            Thus every non‑empty open subset of \(E\) meets \(D_E\), so
            \(D_E\) is dense in \(E\).
  \end{enumerate}
  
  \noindent
  Because \(D_E\) is countable and dense in \(E\), the metric space
  \((E,D|_{E\times E})\) is separable.  \qedhere
  \end{proof}
  %  Why does ``\(D_E\cap U\neq\varnothing\) for every non‑empty open
%  \(U\subseteq E\)’’ imply that \(D_E\) is dense in \(E\)?

\paragraph{1.  Two equivalent definitions of density.}
For a subset \(A\) of a topological space \(X\) the following are
equivalent:

\begin{enumerate}
    \item[(a)] \(\overline{A}=X\) (the \emph{closure} of \(A\) equals \(X\));
    \item[(b)] \textit{every} non‑empty open subset \(U\subseteq X\) meets
               \(A\), i.e.\ \(U\cap A\neq\varnothing\).
\end{enumerate}

\noindent
\emph{Proof of the equivalence.}
\begin{itemize}
    \item[(a)\(\Rightarrow\)(b)]  
          If \(\overline{A}=X\) and \(U\subseteq X\) is open and non‑empty,
          then \(U\cap\overline{A}=U\neq\varnothing\).  
          Since \(U\cap\overline{A}= U\cap A\) (because \(A\subseteq\overline{A}\)),
          we get \(U\cap A\neq\varnothing\).
    \item[(b)\(\Rightarrow\)(a)]  
          Suppose every non‑empty open \(U\) meets \(A\).
          If \(x\in X\setminus\overline{A}\), the set
          \(U:=X\setminus\overline{A}\) is an open neighbourhood of \(x\)
          disjoint from \(A\), contradicting (b).  Hence no such \(x\) exists
          and \(\overline{A}=X\).
\end{itemize}

\paragraph{2.  Apply the criterion to \(D_E\subseteq E\).}
In the argument you are reading, we showed:

\[
    \forall\,\text{non‑empty open }U\subseteq E,\quad D_E\cap U\neq\varnothing.
\]

Because \(E\) is being considered with its own (subspace) topology, this
statement is exactly condition (b) for the space \(X=E\) and the subset
\(A=D_E\).  By the equivalence above it follows that

\[
    \overline{D_E}^{\,E}=E,
\]

i.e.\ \(D_E\) is \emph{dense in \(E\)}.

\paragraph{3.  Conclusion.}
Therefore the fact that \(D_E\) intersects every non‑empty open subset of
\(E\) is sufficient—indeed, equivalent—to saying that \(D_E\) is dense in
\(E\).  No additional work is needed.  \qed

  \medskip
  \textbf{Step 3 -- Conclusion.}

  We have shown \(M\setminus Z\) is countable, i.e.\ all but countably
  many points of \(M\) are condensation points.
\end{proof}
\begin{theorem}\label{thm:card_c}
  Every \emph{uncountable, complete, separable} metric space $M$
  has cardinality $\mathfrak c$ (the cardinality of the continuum).
\end{theorem}

\begin{proof}
  We already know from Theorem~60 that
  $\lvert M\rvert\le\mathfrak c$ for any separable space,  
  so it remains to prove the \emph{lower} bound
  $\lvert M\rvert\ge\mathfrak c$.

  \medskip
  \noindent\textbf{Step 1.  Existence of disjoint uncountable balls of
  arbitrarily small radius.}

  \begin{enumerate}[label=\arabic*., leftmargin=2.8em]
      \item By Theorem~61 an uncountable separable space contains
            at least two \emph{distinct condensation points}
            $p_1,p_2\in M$.
      \item Choose $\rho>0$ with $0<\rho<\tfrac12D(p_1,p_2)$ and put
            \[
                B_i:=\overline{B}(p_i,\rho)\subseteq M,\qquad i=1,2 .
            \]
            The balls are disjoint (by choice of~$\rho$) and
            \emph{uncountable} because each contains its centre
            $p_i$ together with uncountably many other points of $M$.
      \item Since $\rho$ can be taken as small as desired,
            $M$ contains two disjoint uncountable closed balls
            of \emph{arbitrarily small radius}.
  \end{enumerate}

  \medskip
  \noindent\textbf{Step 2.  Recursive construction of a binary tree of balls.}

  \begin{enumerate}[label=\arabic*., leftmargin=2.8em]
      \item \emph{Level 1:}  
            Find disjoint uncountable closed balls
            $A_1,A_2$ with radius $\le 1$.
      \item \emph{Induction step:}  
            Given a ball $A$ of radius $\le 2^{-k}$ that is
            uncountable, complete, and separable
            (closed subspaces inherit these properties),
            apply Step 1 \emph{inside $A$} to obtain two
            disjoint uncountable closed sub‑balls of radius $\le 2^{-(k+1)}$.
      \item Continuing inductively we obtain, for each finite sequence
            $\sigma\in\{1,2\}^k$, a closed ball
            $A_\sigma$ of radius $\le 2^{-k}$ such that
            \[
                A_{\sigma 1},\;A_{\sigma 2}\subseteq A_\sigma,
                \quad
                A_{\sigma 1}\cap A_{\sigma 2}=\varnothing .
            \]
  \end{enumerate}

  \medskip
  \noindent\textbf{Step 3.  Associate a unique point to each infinite
  binary sequence.}

  For an \emph{infinite} sequence
  $\omega=\omega_1\omega_2\omega_3\cdots\in\{1,2\}^{\mathbb N}$
  consider the nested chain
  \[
      A_{\omega_1}\supseteq
      A_{\omega_1\omega_2}\supseteq
      A_{\omega_1\omega_2\omega_3}\supseteq\dots
  \]
  whose radii tend to $0$.  
  By Theorem 50 (nested‑set criterion in complete spaces) the
  intersection
  $\bigcap_{k\ge1}A_{\omega_1\cdots\omega_k}$ is a \emph{single point};
  denote it by $x_\omega\in M$.

  \medskip
  \noindent\textbf{Step 4.  Distinct sequences give distinct points.}

  If $\omega\ne\omega'$, let $k$ be the first index with
  $\omega_k\ne\omega'_k$.  
  Then $x_\omega\in A_{\omega_1\cdots\omega_k}$ and
  $x_{\omega'}\in A_{\omega_1\cdots\omega'_k}$, two \emph{disjoint}
  balls; hence $x_\omega\ne x_{\omega'}$.

  \medskip
  \noindent\textbf{Step 5.  Cardinality count.}

  The set $\{1,2\}^{\mathbb N}$ of infinite binary sequences has
  cardinality $2^{\aleph_0}=\mathfrak c$, and we have injected it into
  $M$ via $\omega\mapsto x_\omega$.  
  Therefore $\lvert M\rvert\ge\mathfrak c$.

  \medskip
  \noindent\textbf{Step 6.  Combine bounds.}

  Together with $\lvert M\rvert\le\mathfrak c$
  (Theorem~\ref{thm:card_separable}), we conclude
  $\lvert M\rvert=\mathfrak c$.
\end{proof}


%------------------------------------------------
% 5.3  \;Compactness – Sequential Lemmas
%------------------------------------------------
\section*{5.3 \quad Compactness}
%-------------  Theorem 63  -------------
\begin{theorem}\label{thm:limit_subseq}
  Let $\{x_i\}_{i\ge 1}$ be a sequence of \emph{distinct} points in a
  metric space $M$ and set $A=\{x_i\mid i\ge1\}$.
  If $x\in\overline{A}\setminus A$, then $\{x_i\}$ possesses a
  subsequence converging to $x$.
\end{theorem}

\begin{proof}
  Every neighbourhood of $x$ intersects $A$ infinitely often.
  Construct the desired subsequence inductively:

  \smallskip
  \noindent
  \emph{Step $n$.}  
  Having chosen indices $i_1<\dots<i_{n-1}$, pick $i_n>i_{n-1}$
  such that
  \[
      x_{i_n}\in B\!\bigl(x,\tfrac1n\bigr)\setminus\{x_{i_1},\dots,x_{i_{n-1}}\}.
  \]
  This is possible because $B\!\bigl(x,\tfrac1n\bigr)$ contains
  infinitely many $x_i$’s.

  \smallskip
  The sequence $(x_{i_n})$ satisfies
  $D(x_{i_n},x)<\tfrac1n\to 0$, hence $x_{i_n}\to x$.
\end{proof}

%-------------  Theorem 64  -------------
\begin{theorem}\label{thm:seq_limit_pt}
  For a metric space $M$ the following are equivalent:
  \begin{enumerate}[label=\textup{(\alph*)}]
      \item Every sequence in $M$ has a convergent subsequence.
      \item Every infinite subset of $M$ has a limit point.
  \end{enumerate}
\end{theorem}

\begin{proof}
  \textbf{(a)$\!\implies\!$(b).}\;
  Given an infinite set $A\subseteq M$, list its elements as a sequence.
  By~(a) some subsequence converges to a point $x\in M$, making $x$ a
  limit point of $A$.

  \medskip
  \textbf{(b)$\!\implies\!$(a).}\;
  Let $\{x_i\}$ be any sequence; without loss of generality assume the
  $x_i$’s are distinct.  
  The set $A=\{x_i\}$ is infinite and, by~(b), possesses a limit
  point $x$.  
  If $x\notin A$, Theorem \ref{thm:limit_subseq} provides a subsequence
  $x_{i_n}\to x$.  
  If $x=x_j$ for some $j$, delete $x_j$ and apply the same argument to
  the remaining infinite set $A\setminus\{x_j\}$.
\end{proof}
%------------------------------------------------
% 5.3  \;Compactness – Basic facts
%------------------------------------------------

\begin{definition}[Compact metric space]\label{def:compact_metric}
  A metric space $(M,D)$ is \textbf{compact} if it satisfies
  either (hence both) of the equivalent conditions in
  Theorem \ref{thm:seq_compact} below; concretely,
  every sequence has a convergent subsequence, or equivalently,
  every infinite subset has a limit point.
\end{definition}

%-------------  Sequential forms of compactness  -------------
\begin{theorem}[Sequential compactness]\label{thm:seq_compact}
  For a metric space $M$ the following are equivalent:
  \begin{enumerate}[label=\textup{(\alph*)}]
      \item Every sequence in $M$ possesses a \emph{convergent subsequence};
      \item Every infinite subset of $M$ possesses a \emph{limit point}.
  \end{enumerate}
\end{theorem}

\begin{proof}
  $(a)\!\Rightarrow\!(b)$ is immediate:
  given an infinite set, enumerate it as a sequence and use~(a).
  For $(b)\!\Rightarrow\!(a)$, enumerate the given sequence by distinct
  points, obtain a limit point $x$ via~(b), and apply
  Theorem 63 to extract a convergent subsequence.
\end{proof}

%-------------  Theorem 65  -------------
\begin{theorem}\label{thm:closed_interval_compact}
  Any closed interval $[a,b]\subseteq\mathbb R$ is compact.
\end{theorem}

\begin{proof}[Sketch]
  Modify the proof of Theorem 47 (completeness of $\mathbb R$)
  by observing that every bounded sequence in $\mathbb R$ has a
  convergent subsequence whose limit lies in~$[a,b]$.
\end{proof}

%-------------  Theorem 66  -------------
\begin{theorem}\label{thm:closed_subset_compact}
  A \emph{closed} subset of a compact metric space is compact.
\end{theorem}

\begin{proof}
  Let $A\subseteq M$ with $M$ compact and $A$ closed.
  For any sequence $\{a_i\}\subseteq A$,
  $\{a_i\}$ has a convergent subsequence in $M$,
  say $a_{i_k}\to x\in M$.
  Closedness of $A$ implies $x\in A$,
  hence $A$ is compact by Theorem \ref{thm:seq_compact}.
\end{proof}

%-------------  Theorem 67  -------------
\begin{theorem}\label{thm:compact_complete}
  Every compact metric space is complete.
\end{theorem}

\begin{proof}
  Let $\{x_i\}$ be Cauchy in $M$ compact.
  A subsequence converges to some $y\in M$; by Theorem 44
  (Cauchy $+$ convergent subsequence $\Rightarrow$ convergence),
  the whole sequence converges to $y$.
\end{proof}

\begin{remark}
  As a corollary, compact subsets of metric spaces are \emph{closed}.
\end{remark}

%-------------  Theorem 68  -------------
\begin{theorem}\label{thm:finite_diameter}
  A compact metric space $M$ has finite diameter.
  Moreover, there exist points $x,y\in M$ such that
  $D(x,y)=\operatorname{diam}M$.
\end{theorem}

\begin{proof}
  Let $t:=\sup\{D(u,v)\mid u,v\in M\}$ (possibly $\infty$).
  Choose sequences $\{x_i\},\{y_i\}\subseteq M$
  with $D(x_i,y_i)\to t$.
  Compactness gives convergent subsequences
  $x_i\to x$, $y_i\to y$.
  Continuity of $D$ (Theorem 37) yields
  $D(x,y)=t$, so $t<\infty$ and equals the diameter.
\end{proof}

%------------------------------------------------
% Consequence in $\mathbb R$
%------------------------------------------------
\begin{corollary}[Heine–Borel for $\mathbb R$]\label{cor:HB}
  A subset of the real line is compact \emph{iff} it is
  \textbf{closed} and \textbf{bounded}.
\end{corollary}
%------------------------------------------------
% 5.3  \;Compactness – Fundamental Results
%------------------------------------------------

%-------------  Theorem 69  -------------
\begin{theorem}[Extreme–value theorem]\label{thm:EVT}
  Let $M$ be a compact metric space and
  $f:M\to\mathbb R$ a continuous map.
  Then $f$ is \emph{bounded} on $M$ and attains both its
  maximum and minimum values; i.e.\ there exist
  $x_{\min},x_{\max}\in M$ such that
  \[
      f(x_{\min})=\inf_{x\in M}f(x),
      \qquad
      f(x_{\max})=\sup_{x\in M}f(x).
  \]
\end{theorem}

\begin{proof}[Sketch]
  By compactness $\{f(x)\mid x\in M\}\subset\mathbb R$
  is compact (continuous image of a compact set) and hence closed and
  bounded; closedness guarantees the extrema are attained.
\end{proof}

%-------------  Theorem 70  -------------
\begin{theorem}[Compact $\;\Rightarrow\;$ separable]\label{thm:compact_sep}
  Every compact metric space $(M,D)$ is separable.
\end{theorem}

\begin{proof}[Step‑by‑step]
  \textbf{1.  Construct maximal $1/n$–separated sets.}  
  For each $n\in\mathbb N$ choose $T_n\subseteq M$ \emph{maximal} with
  the property
  \(
      D(x,y)\ge 1/n\ \text{for }x\ne y\text{ in }T_n.
  \)
  Compactness prevents $T_n$ from being infinite; otherwise one could
  extract a convergent subsequence whose limit violates the separation
  property.  Thus $T_n$ is finite for each $n$.
  %-----------------------------------------------------------------
%  Theorem 31  (compact ⇒ separable)  –  Step 1 in more detail
%-----------------------------------------------------------------
\paragraph{Goal of Step 1.}
For each positive integer \(n\) we want a finite set
\(T_n\subseteq M\) such that
\[
    D(x,y)\;\ge\;\frac1n
    \quad\text{whenever }x,y\in T_n,\;x\ne y,
    \tag{1}
\]
and which is \emph{maximal} with respect to this separation property
(no additional point of \(M\) can be adjoined without breaking (1)).
Such a set is called \textbf{\(1/n\)-separated and maximal}.

\bigskip
\begin{enumerate}
%---------------------------------------------
\item[\textbf{(a)}] \textbf{Existence of a \(1/n\)-separated set.}
      Pick any point \(p\in M\); the singleton \(\{p\}\) already satisfies
      (1).  Thus non‑empty \(1/n\)-separated sets exist.

%---------------------------------------------
\item[\textbf{(b)}] \textbf{Zorn’s lemma (or a greedy construction) gives a \emph{maximal} one.}

      \begin{itemize}
          \item \emph{Partial order.}
                Order the family
                \(\mathcal S:=\{S\subseteq M\mid S \text{ is }1/n\text{-separated}\}\)
                by inclusion.
          \item \emph{Chains have upper bounds.}
                If \(\{S_\alpha\}_{\alpha\in\Lambda}\)
                is a chain in \(\mathcal S\),
                its union \(S=\bigcup_{\alpha}S_\alpha\)
                is still \(1/n\)-separated (any conflicting pair would
                already live in some \(S_\alpha\)).
          \item \emph{Apply Zorn.}
                Hence every chain has an upper bound,
                so Zorn’s lemma furnishes a maximal element
                \(T_n\in\mathcal S\).
      \end{itemize}

      \emph{Greedy alternative:}
      Start with any point, then keep adding new points that are
      at least \(1/n\) away from all previously chosen ones.
      Stop when no further point can be added; the result is maximal
      by construction.

%---------------------------------------------
\item[\textbf{(c)}] \textbf{Compactness forces \(T_n\) to be finite.}

      Suppose, for contradiction, that \(T_n\) were infinite.
      Because \(M\) is compact,
      \(T_n\) has a convergent subsequence \(\{x_{k}\}_{k\ge1}\)
      with limit \(x_\infty\in M\).

      \medskip\noindent
      Since \(x_{k}\to x_\infty\), there exists some index \(k_0\)
      with
      \[
          D\bigl(x_{k_0},x_\infty\bigr)\;<\;\frac1n.
      \]
      The limit \(x_\infty\) itself cannot belong to \(T_n\),
      for otherwise it would violate (1) together with \(x_{k_0}\).
      But if \(x_\infty\notin T_n\) we could adjoin it to \(T_n\),
      contradicting maximality.
      Hence \(T_n\) must be finite.

%---------------------------------------------
\item[\textbf{(d)}] \textbf{Summary.}
      For every \(n\in\mathbb N\) we have produced a \emph{finite}
      maximal \(1/n\)-separated set \(T_n\subseteq M\).
      These sets will be used in Step 2 to construct a \emph{countable}
      dense subset
      \[
           D\;=\;\bigcup_{n\ge1}T_n,
      \]
      proving that \(M\) is separable.
\end{enumerate}

  \textbf{2.  Density of the union.}  
  Any $x\in M$ lies within $1/n$ of some point of $T_n$
  (otherwise $T_n\cup\{x\}$ would contradict maximality).  
  Hence
  \[
      D:=\bigcup_{n\ge1}T_n
  \]
  is \emph{countable} and \emph{dense} in $M$.
\end{proof}
%-----------------------------------------------------------------
%  Why adding the limit point \(x_\infty\notin T_n\) would defeat
%  the assumption that \(T_n\) is \emph{maximal \(1/n\)-separated}.
%-----------------------------------------------------------------


%-----------------------------------------------------------------
%  Theorem.  Every compact metric space is separable.
%-----------------------------------------------------------------
\begin{theorem}
  Let \((M,D)\) be a compact metric space.  
  Then \(M\) contains a countable dense subset; i.e.\ \(M\) is separable.
  \end{theorem}
  
  \begin{proof}[Step‑by‑step proof]
  \textbf{Step 1 – Build \(\tfrac1n\)-separated, maximal sets.}\par
  For each \(n\in\mathbb N\) choose a subset \(T_n\subseteq M\) that is
  \emph{maximal} with the property
  \[
      (\forall x\neq y\in T_n)\qquad D(x,y)\;\ge\;\frac1n.
      \tag{$\ast_n$}
  \]
  Such a set exists by Zorn’s Lemma or by a greedy “add points until you
  cannot’’ construction.
  
  \medskip
  \textbf{Step 2 – Each \(T_n\) is finite (compactness).}\par
  Assume, for contradiction, that some \(T_n\) were infinite.
  Compactness of \(M\) gives a convergent subsequence
  \(\{x_k\}_{k\ge1}\subseteq T_n\) with limit \(x_\infty\in M\).
  Choose \(k_0\) so large that \(D(x_{k_0},x_\infty)<\tfrac1n\).
  If \(x_\infty\in T_n\) this violates \((\ast_n)\) with \(x_{k_0}\);  
  if \(x_\infty\notin T_n\) we could adjoin it and still respect
  \((\ast_n)\), contradicting maximality.  
  Hence \(T_n\) must be \emph{finite}.
  
  \medskip
  \textbf{Step 3 – \(T_n\) is a \(\tfrac1n\)-net (coverage).}\par
  Take any point \(x\in M\).  
  If \(D(x,t)\ge\tfrac1n\) for every \(t\in T_n\), then \(T_n\cup\{x\}\)
  would still satisfy \((\ast_n)\), again contradicting maximality.
  Therefore
  \[
      (\forall x\in M)\quad
      \exists\,t\in T_n\;:\;D(x,t)\;<\;\frac1n.
      \tag{$\dagger_n$}
  \]
  
  \medskip
  \textbf{Step 4 – Countability of the union.}\par
  Because each \(T_n\) is finite and \(n\) ranges over \(\mathbb N\),
  their union
  \[
      D\;:=\;\bigcup_{n=1}^\infty T_n
  \]
  is a \emph{countable} (finite–by–countable) union of finite sets.
  
  \medskip
  \textbf{Step 5 – Density of the union.}\par
  Fix \(x\in M\) and \(\varepsilon>0\).  
  Choose \(n\) so large that \(\tfrac1n<\varepsilon\).
  By \((\dagger_n)\) there is \(t\in T_n\) with \(D(x,t)<\tfrac1n<\varepsilon\);  
  thus every open ball around \(x\) meets \(D\).  
  Hence \(D\) is \emph{dense} in \(M\).
  
  \medskip
  \textbf{Conclusion.}\;
  The set \(D\) is both countable and dense, so \(M\) is separable.
  \end{proof}

%-------------  Theorem 71  -------------
\begin{theorem}[Nested closed–set criterion]\label{thm:nested_closed_compact}
  For a metric space $M$ the following are equivalent:
  \begin{enumerate}[label=\textup{(\alph*)}]
      \item $M$ is \textbf{compact};
      \item Every descending sequence of non‑empty closed sets
            \(
                F_1\supseteq F_2\supseteq F_3\supseteq\dots
            \)
            has non‑empty intersection
            $\displaystyle\bigcap_{i=1}^{\infty}F_i\ne\varnothing$.
  \end{enumerate}
\end{theorem}

\begin{proof}
  \textbf{(a)$\!\Rightarrow\!$(b).}\;
  Given descending $F_i\ne\varnothing$, pick $x_i\in F_i$.
  Compactness yields a subsequence $x_{i_k}\to y\in M$.
  For any $m$, $x_{i_k}\in F_m$ once $i_k\ge m$, so $y\in F_m$.  
  Hence $y$ lies in the intersection.

  \medskip
  \textbf{(b)$\!\Rightarrow\!$(a).}\;
  Contrapositively, assume $M$ is \emph{not} compact.
  Then there exists a sequence $\{x_i\}$ with no convergent subsequence.
  (Theorem \ref{thm:seq_compact}.)
  Define
  \(
      F_i:=\{x_i,x_{i+1},x_{i+2},\dots\}.
  \)
  Each $F_i$ is closed (Theorem 63) and non‑empty,
  the $F_i$’s are nested, but their intersection is empty,
  contradicting (b).
\end{proof}

%-------------  Theorem 72  -------------
\begin{theorem}[Heine–Borel characterisation]\label{thm:HB_metric}
  For a metric space $M$ the following are equivalent:
  \begin{enumerate}[label=\textup{(\alph*)}]
      \item $M$ is \textbf{compact};
      \item Every open cover of $M$ admits a \emph{finite} subcover;
      \item For every family $\{F_i\}$ of closed subsets of $M$,
            if every finite subfamily has non‑empty intersection then
            $\displaystyle\bigcap_i F_i\ne\varnothing$.
  \end{enumerate}
\end{theorem}

\begin{proof}
  \textbf{(b)$\Leftrightarrow$(c)} by taking complements (Exercise 10
  of §5.2).

  \medskip
  \textbf{(a)$\!\Rightarrow\!$(b).}\;
  Compactness $\Rightarrow$ separability (Theorem \ref{thm:compact_sep})
  and Theorem 59 (countable base) reduce any open cover to a
  \emph{countable} one $\{U_k\}_{k\ge1}$.
  Suppose no finite subcollection covers $M$.
  Let $F_n:=M\setminus\bigl(U_1\cup\dots\cup U_n\bigr)$;
  then $F_1\supseteq F_2\supseteq\dots\ne\varnothing$,
  contradicting Theorem \ref{thm:nested_closed_compact}.

  \medskip
  \textbf{(c)$\!\Rightarrow\!$(a).}\;
  Assume (c) and let $\{x_i\}$ be a sequence in $M$.
  As in the proof of Theorem \ref{thm:nested_closed_compact},
  set
  \(
      F_i:=\{x_i,x_{i+1},\dots\}.
  \)
  Finite intersections are non‑empty, so by (c)
  $\bigcap_i F_i\ne\varnothing$,
  hence $\{x_i\}$ has a cluster point and therefore a convergent
  subsequence (Theorem \ref{thm:limit_subseq}).
  Sequential compactness implies compactness.
\end{proof}

\medskip\noindent
\textbf{Uniform continuity.}\;
Every continuous map on a compact metric space is \emph{uniformly}
continuous, since the usual $\varepsilon$–$\delta$ constants can be
chosen uniformly via compactness (continuous image of a compact set is
compact and real‐valued functions attain extrema).
%------------------------------------------------
% Theorem 73 – Continuous maps on compact spaces are uniformly continuous
%------------------------------------------------
\begin{theorem}\label{thm:compact_uc}
  Let $(X,D_X)$ be a \textbf{compact} metric space and  
  $(Y,D_Y)$ an arbitrary metric space.
  If $f:X\to Y$ is \emph{continuous}, then $f$ is
  \textbf{uniformly continuous}; i.e.
  \[
      \forall\,\varepsilon>0\;\exists\,\delta>0
      \quad\text{s.t.}\quad
      D_X(x,x')<\delta\;\Longrightarrow\;
      D_Y\!\bigl(f(x),f(x')\bigr)<\varepsilon
      \quad\text{for \emph{all }}x,x'\in X.
  \]
\end{theorem}

\begin{proof}[Step‑by‑step (contrapositive argument)]
  \textbf{Step 1.  Negate uniform continuity and set up sequences.}  

  Assume, for contradiction, that $f$ is \emph{not} uniformly
  continuous.
  Then there exists $\varepsilon_0>0$ such that
  no matter how small $\delta>0$ is chosen, we can find
  $x,x'\in X$ with
  \[
      D_X(x,x')<\delta
      \quad\text{but}\quad
      D_Y\!\bigl(f(x),f(x')\bigr)\ge\varepsilon_0.
  \]
  Select a sequence $\delta_n\downarrow0$ (e.g.\ $\delta_n=1/n$) and,
  for each $n$, pick points $x_n,y_n\in X$ satisfying
  \[
      D_X(x_n,y_n)<\delta_n, \qquad
      D_Y\!\bigl(f(x_n),f(y_n)\bigr)\ge\varepsilon_0. \tag{1}
  \]

  \medskip
  \textbf{Step 2.  Extract convergent subsequences using compactness.}

  Because $X$ is compact, the sequence $\{x_n\}$ has a convergent
  subsequence; relabel if necessary to assume
  \[
      x_n\longrightarrow x\in X. \tag{2}
  \]
  Condition $D_X(x_n,y_n)<\delta_n\to0$ implies
  \[
      y_n\longrightarrow x \quad\text{as well}. \tag{3}
  \]

  \medskip
  \textbf{Step 3.  Pass to the limit via continuity of $f$.}

  Continuity of $f$ at $x$ yields
  \[
      f(x_n)\;\longrightarrow\;f(x),
      \qquad
      f(y_n)\;\longrightarrow\;f(x). \tag{4}
  \]

  \medskip
  \textbf{Step 4.  Reach a contradiction.}

  From (4) we have
  \(
      D_Y\!\bigl(f(x_n),f(y_n)\bigr)\to 0.
  \)
  This contradicts inequality (1),
  which asserts
  \(
      D_Y\!\bigl(f(x_n),f(y_n)\bigr)\ge\varepsilon_0>0
  \)
  for \emph{every} $n$.

  \medskip
  \textbf{Step 5.  Conclude uniform continuity.}

  The contradiction shows our assumption was false; hence
  such an $\varepsilon_0$ cannot exist, and $f$ is uniformly continuous.
\end{proof}
%------------------------------------------------
% Theorem 74 – Total boundedness criteria
%------------------------------------------------
\begin{theorem}\label{thm:totally_bounded}
  For a metric space $M$ the following five statements are equivalent:
  \begin{enumerate}[label=\textup{(\alph*)}]
      \item[\textup{(a)}] The completion $M^{\!*}$ of $M$ is \textbf{compact}.
      \item[\textup{(b)}] \emph{Every} sequence in $M$ possesses a \textbf{Cauchy subsequence}.
      \item[\textup{(c)}] For every $\varepsilon>0$ there exists a \emph{finite} set
                         $T_\varepsilon\subseteq M$ such that
                         \[
                             \forall\,x\in M\;\exists\,t\in T_\varepsilon:\;
                             D(x,t)<\varepsilon .
                         \]
      \item[\textup{(d)}] For every $\varepsilon>0$ the space $M$ can be written as the
                         union of a \emph{finite} family of subsets, each having
                         \emph{diameter} $<\varepsilon$.
      \item[\textup{(e)}] For every $\varepsilon>0$ and every \emph{infinite} subset
                         $A\subseteq M$ there exists an \emph{infinite}
                         subset $B\subseteq A$ with $\operatorname{diam}B<\varepsilon$.
  \end{enumerate}
  A metric space satisfying any (hence all) of these conditions is often
  called \textbf{totally bounded}.
\end{theorem}

\begin{proof}[Logical cycle of implications]
  We establish
  \[
      (a)\Longleftrightarrow(b)\quad\text{and}\quad
      (b)\;\Longrightarrow\;(c)\;\Longrightarrow\;(d)\;\Longrightarrow\;(e)\;\Longrightarrow\;(b).
  \]

  %------------------------------------------------------------
  \paragraph{\textbf{$(a)\Longrightarrow(b)$}.}
      Let $\{x_i\}\subseteq M$ be a sequence.
      As $M$ embeds isometrically and densely in the \emph{compact}
      space $M^{\!*}$, $\{x_i\}$ has a convergent subsequence in
      $M^{\!*}$; Theorem 43 shows this subsequence is Cauchy in $M$.

  %------------------------------------------------------------
  \paragraph{\textbf{$(b)\Longrightarrow(a)$}.}
      Take a sequence $\{y_i\}\subseteq M^{\!*}$.
      For each $i$ choose $x_i\in M$ with $D(x_i,y_i)<\tfrac1i$.
      By~(b) the sequence $\{x_i\}$ has a Cauchy subsequence,
      hence converges in $M^{\!*}$; so does $\{y_i\}$.
      Thus every sequence in $M^{\!*}$ converges—$M^{\!*}$ is compact.

  %------------------------------------------------------------
  \paragraph{\textbf{$(b)\Longrightarrow(c)$}.}
      Fix $\varepsilon>0$ and build $T_\varepsilon$ greedily:
      choose $x_1\in M$, then $x_2$ at distance $\ge\varepsilon$ from
      $x_1$, etc.
      If this process never stops we obtain an infinite sequence with
      mutual distances $\ge\varepsilon$—hence \emph{no} Cauchy
      subsequence, contradicting~(b).
      Therefore the process terminates in a \emph{finite} set $T_\varepsilon$
      that $\varepsilon$‑spans $M$.

  %------------------------------------------------------------
  \paragraph{\textbf{$(c)\Longrightarrow(d)$}.}
      Given $\varepsilon>0$, pick the finite set
      $T_\varepsilon=\{u_1,\dots,u_n\}$ from~(c) and observe that
      \(
          M=\bigcup_{k=1}^{n}B(u_k,\varepsilon)
      \)
      is a finite cover by sets of diameter $<2\varepsilon$.
      Re‑scale $\varepsilon$ to obtain statement (d).

  %------------------------------------------------------------
  \paragraph{\textbf{$(d)\Longrightarrow(e)$}.}
      Let $\varepsilon>0$ and $A\subseteq M$ be infinite.
      Cover $M$ by finitely many sets $B_1,\dots,B_m$ of
      diameter $<\varepsilon$ (by (d)).
      The pigeon‑hole principle yields $B_j$ with $A\cap B_j$ infinite
      and $\operatorname{diam}(A\cap B_j)<\varepsilon$.

  %------------------------------------------------------------
  \paragraph{\textbf{$(e)\Longrightarrow(b)$}.}
      Let $\{x_i\}$ be a sequence in $M$ (w.l.o.g.\ with distinct
      points).  
      Apply~(e) with $\varepsilon_1=1$ to obtain an infinite
      subsequence whose mutual distances are $<1$; denote it
      $(x_{1n})$.
      Repeat with $\varepsilon_2=\tfrac12$, etc.\ (Cantor’s diagonal
      process):
      \[
          \begin{array}{cccc}
             x_{11}, & x_{12}, & x_{13}, & \dots  \\
             x_{21}, & x_{22}, & x_{23}, & \dots  \\
             \vdots  &          &        &
          \end{array}
      \]
      The diagonal $x_{11},x_{22},x_{33},\dots$ is a Cauchy
      subsequence of $\{x_i\}$, establishing (b).
\end{proof}

\medskip\noindent
The terminology \emph{totally bounded} (or sometimes \emph{pre‑compact})
is standard for spaces satisfying the equivalent conditions of
Theorem \ref{thm:totally_bounded}.

\begin{definition}
  Let $\{U_i\}_{i\in I}$ be an open cover of a metric space $(M,D)$.  
  A \emph{Lebesgue number} for this cover is a number $\varepsilon>0$ such that for
  \emph{every} subset $A\subset M$ with 
  \[
  \operatorname{diam}(A)\;=\;\sup\{D(a,b):a,b\in A\}\;<\varepsilon,
  \]
  there exists an index $i$ for which $A\subset U_i$.
  \end{definition}
  
  \begin{theorem}[Lebesgue Number Lemma]\label{thm:lebesgue_number}
  Any open cover of a \emph{compact} metric space has a Lebesgue number.
  \end{theorem}
  
  \begin{proof}
  Let $\{U_i\}_{i\in I}$ be an open cover of a compact metric space $M$.
  Assume \emph{towards a contradiction} that the cover admits no Lebesgue number.
  Then for every $n\in\mathbb{N}$ we can find a set $A_n\subset M$ with
  $\operatorname{diam}(A_n)<\frac1n$ such that $A_n$ is \emph{not} contained
  in any single $U_i$.
  
  Choose $x_n\in A_n$ for each $n$.
  Because $M$ is compact, the sequence $(x_n)$
  has a convergent subsequence, which we relabel as $(x_n)$, with
  $x_n\to x\in M$.
  
  Since the $U_i$ cover $M$, pick $U_j$ with $x\in U_j$.
  Because $U_j$ is open, there exists $r>0$ such that the open ball
  \[
  S_r(x)=\{y\in M:\;D(y,x)<r\}
  \]
  is contained in $U_j$.
  
  Convergence implies $D(x_n,x)<r$ for all sufficiently large $n$.
  Fix such an $n$ and note that
  $\operatorname{diam}(A_n)<\frac1n<r$.  For any $y\in A_n$ we have
  \[
  D(x,y)\le D(x,x_n)+D(x_n,y)<r+r=2r,
  \]
  so $A_n\subset S_{2r}(x)\subset U_j$.
  This contradicts the choice of $A_n$, which by construction was not
  contained in any $U_i$.
  Hence the assumption is impossible, and a Lebesgue number must exist.
  \end{proof}
\end{document}
