\documentclass[12pt]{article}

% Packages
\usepackage[margin=1in]{geometry}
\usepackage{amsmath,amssymb,amsthm}
\usepackage{enumitem}
\usepackage{hyperref}
\usepackage{xcolor}
\usepackage{import}
\usepackage{xifthen}
\usepackage{pdfpages}
\usepackage{transparent}
\usepackage{listings}


\lstset{
    breaklines=true,         % Enable line wrapping
    breakatwhitespace=false, % Wrap lines even if there's no whitespace
    basicstyle=\ttfamily,    % Use monospaced font
    frame=single,            % Add a frame around the code
    columns=fullflexible,    % Better handling of variable-width fonts
}

\newcommand{\incfig}[1]{%
    \def\svgwidth{\columnwidth}
    \import{./Figures/}{#1.pdf_tex}
}
\theoremstyle{definition} % This style uses normal (non-italicized) text
\newtheorem{solution}{Solution}
\newtheorem{proposition}{Proposition}
\newtheorem{problem}{Problem}
\newtheorem{lemma}{Lemma}
\newtheorem{theorem}{Theorem}
\newtheorem{remark}{Remark}
\newtheorem{note}{Note}
\newtheorem{definition}{Definition}
\newtheorem{example}{Example}
\theoremstyle{plain} % Restore the default style for other theorem environments
%

% Theorem-like environments
% Title information
\title{MATH 432: HW 6}
\author{Jerich Lee}
\date{\today}

\begin{document}

\maketitle
\begin{problem}[]
    
\end{problem}
\begin{solution}
   

\noindent
\textbf{Step 1.} We claim that $\aleph_1 \le 2^{\aleph_0}$. 
Indeed, $2^{\aleph_0}$ is the cardinality of the continuum (the set of real numbers), 
which is uncountable, so it is at least $\aleph_1$, the first uncountable cardinal. 
Hence,
\[
\aleph_1 \;\le\; 2^{\aleph_0}.
\]

\medskip
\noindent
\textbf{Step 2.} Raise both sides of the inequality to the power $\aleph_0$:
\[
\aleph_1^{\aleph_0} \;\le\; \bigl(2^{\aleph_0}\bigr)^{\aleph_0}.
\]
Using the laws of exponents for cardinals, we have
\[
\bigl(2^{\aleph_0}\bigr)^{\aleph_0}
\;=\;
2^{\aleph_0 \cdot \aleph_0}
\;=\;
2^{\aleph_0}.
\]
Therefore,
\[
\aleph_1^{\aleph_0} \;\le\; 2^{\aleph_0}.
\]

\medskip
\noindent
\textbf{Step 3.} We also show $\aleph_1^{\aleph_0} \ge 2^{\aleph_0}$. 
Observe that $\aleph_1 \ge \aleph_0$, so by monotonicity of exponentiation
on infinite cardinals,
\[
\aleph_1^{\aleph_0} \;\ge\; \aleph_0^{\aleph_0}.
\]
But it is a standard fact that $\aleph_0^{\aleph_0} = 2^{\aleph_0}$. 
(For instance, $\aleph_0^{\aleph_0}$ is the cardinality of all infinite
sequences of natural numbers, which has the same cardinality 
as the reals.)

\medskip
\noindent
\textbf{Conclusion.} Combining the two inequalities
\[
2^{\aleph_0} \;\le\; \aleph_1^{\aleph_0} \;\le\; 2^{\aleph_0},
\]
we deduce
\[
\aleph_1^{\aleph_0} \;=\; 2^{\aleph_0}.
\]
\qed 
\end{solution}
\begin{problem}[]
    
\end{problem}
\begin{solution}

    \noindent \textbf{Problem Statement.} 
    Given two ordinals $\alpha$ and $\beta$, form disjoint well-ordered sets 
    $A$ (of order type $\alpha$) and $B$ (of order type $\beta$), 
    and define the order on $A \cup B$ by declaring that 
    every element of $B$ exceeds every element of $A$. 
    Then $A \cup B$ is well-ordered, and we let $\alpha + \beta$ 
    denote its order type. We want to address:
    
    \begin{enumerate}
    \item[(a)] Is $\alpha + \beta$ always equal to $\beta + \alpha$?
    \item[(b)] Prove that if $\alpha + \beta = \alpha + \gamma$, then $\beta = \gamma$.
    \item[(c)] Give an example where $\beta + \alpha = \gamma + \alpha$ fails to imply $\beta = \gamma$.
    \end{enumerate}
    
    \subsection*{(a) Non-commutativity of Ordinal Addition}
    
    Ordinal addition is \emph{not} commutative in general.  
    A standard counterexample is to take $\alpha = 1$ and $\beta = \omega$ 
    (the first infinite ordinal).  We compute:
    \[
    1 + \omega = \omega \quad \text{but} \quad \omega + 1 \neq \omega.
    \]
    Indeed, $\omega + 1$ is $\omega$ followed by one extra element, strictly larger 
    than all natural numbers, so its order type is different from $\omega$. 
    Hence $\alpha + \beta$ may differ from $\beta + \alpha$.
    
    \subsection*{(b) Left-Cancellation: 
    \texorpdfstring{$\alpha + \beta = \alpha + \gamma \implies \beta = \gamma$}{Left-Cancellation}}
    
    To see why $\alpha + \beta = \alpha + \gamma$ forces $\beta = \gamma$, 
    we outline a common argument:
    
    \begin{enumerate}
    \item[(1)] \textbf{Disjoint representatives.}
    Let $A$ represent $\alpha$, $B$ represent $\beta$, and $C$ represent $\gamma$, 
    with $A, B, C$ pairwise disjoint. Then:
    \[
    \alpha + \beta \quad \leftrightarrow \quad A \cup B \quad (\text{all of }B\text{ placed above all of }A),
    \]
    \[
    \alpha + \gamma \quad \leftrightarrow \quad A \cup C \quad (\text{all of }C\text{ placed above all of }A).
    \]
    \item[(2)] \textbf{Order-isomorphism.}
    If $\alpha + \beta = \alpha + \gamma$, there must be an 
    order-isomorphism 
    \[
    \varphi\colon A \cup B \;\longrightarrow\; A \cup C
    \]
    between these two well-ordered sets.
    
    \item[(3)] \textbf{Initial segment is fixed.}
    Because $A$ is an \emph{initial} segment in both $A \cup B$ and $A \cup C$ 
    (no elements of $B$ or $C$ can appear below any element of $A$), 
    the map $\varphi$ must send $A$ \emph{onto} $A$ itself. 
    (If it did not, the image of an initial segment of length $\alpha$ could not match the 
    corresponding initial segment in the codomain.)
    
    \item[(4)] \textbf{Remainders match.}
    Since $\varphi$ takes $A$ onto $A$, the restriction of $\varphi$ 
    to the ``tail segments'' $B$ and $C$ defines an order-isomorphism $B \cong C$. 
    Hence $\beta = \gamma$.
    \end{enumerate}
    
    Thus the ``left-cancellation law'' for ordinals holds:
    \[
    \alpha + \beta \;=\; \alpha + \gamma \quad \Longrightarrow\quad \beta = \gamma.
    \]
    
    \subsection*{(c) Failure of Right-Cancellation}
    
    However, \emph{right-cancellation} often fails. That is,
    \[
    \beta + \alpha \;=\; \gamma + \alpha \quad \not\Longrightarrow\quad \beta = \gamma.
    \]
    A classic example: let $\alpha = \omega$, $\beta = 1$, and $\gamma = 2$.  Then
    \[
    1 + \omega = \omega 
    \quad \text{and} \quad
    2 + \omega = \omega,
    \]
    so $1 + \omega = 2 + \omega = \omega$.  Yet obviously $1 \neq 2$. 
    Therefore $\beta + \alpha = \gamma + \alpha$ need \emph{not} imply $\beta = \gamma$.
    
    \bigskip
    
    \noindent
    \textbf{Answer Summary.}
    \begin{itemize}
    \item[(a)] \emph{No.} Ordinal addition is not commutative, e.g.\ $1 + \omega = \omega$ but $\omega + 1 \neq \omega$.
    \item[(b)] \emph{Yes.} If $\alpha + \beta = \alpha + \gamma$, then $\beta = \gamma$ (the left-cancellation law).
    \item[(c)] Right-cancellation need not hold.  For example, $1 + \omega = 2 + \omega = \omega$ yet $1 \neq 2$.
    \end{itemize} 
\end{solution}
\begin{problem}[]
    
\end{problem}
\begin{solution}

    \noindent
    \textbf{Problem Statement.}
    Let $A$ and $B$ be sets, and suppose there is a surjective function 
    $f\colon A \to B$.  Use the Axiom of Choice to show that 
    \[
    Card(B) \; \le \; Card(A).
    \]
    (Here $Card(X)$ denotes the cardinality of $X$, 
    though your text may write it as $o(X)$.)
    
    \bigskip
    \noindent
    \textbf{Step 1.  Surjectivity and fiber sets.}  
    Since $f$ is onto $B$, for each $b \in B$, the \emph{fiber} 
    \[
    f^{-1}(\{b\}) \;=\; \{\,a \in A : f(a) = b\}
    \]
    is a nonempty subset of $A$.
    
    \medskip
    \noindent
    \textbf{Step 2.  Applying the Axiom of Choice.}  
    By the Axiom of Choice, we can choose one element $a_b$ from each nonempty fiber 
    $f^{-1}(\{b\})$.  Concretely, we obtain a function 
    \[
    g\colon B \,\longrightarrow\, A
    \]
    defined by 
    \[
    g(b) \;=\; a_b,\quad\text{where}\quad f(a_b) = b.
    \]
    Such a choice of $a_b$ for every $b \in B$ requires the Axiom of Choice 
    if $B$ is infinite and the fibers have no natural ``canonical'' choice.
    
    \medskip
    \noindent
    \textbf{Step 3.  Verifying injectivity.}  
    Claim: $g$ is injective.  Indeed, suppose $b_1,b_2 \in B$ with $b_1 \neq b_2$. 
    Because $f(a_{b_1}) = b_1$ and $f(a_{b_2}) = b_2$, we must have 
    $a_{b_1} \neq a_{b_2}$.  Hence $g(b_1) \neq g(b_2)$, so $g$ is injective.
    
    \medskip
    \noindent
    \textbf{Step 4.  Conclusion on cardinalities.}  
    Since $g$ is an injection from $B$ into $A$, 
    by definition of cardinal comparison we conclude 
    \[
    Card(B) \;\le\; Card(A).
    \]
    Thus the existence of a surjective map $f\colon A \to B$ 
    implies $Card(B) \le Card(A)$, and the Axiom of Choice 
    was used to construct the needed injection $g$.
     
\end{solution}
\begin{problem}[]
    
\end{problem}
\begin{solution}
    \noindent
    \textbf{Problem Statement.} 
    Let $D$ be an infinite set of cardinality $d$. Define 
    \[
    F \;=\;\{\,A \subset D : A\text{ is finite}\}.
    \]
    Prove that $Card(F) = d$.
    
    \bigskip
    \noindent
    \textbf{Step 1: Decompose $F$ by finite sizes.}
    Any finite subset $A$ of $D$ has some finite cardinality $n \in \omega$. 
    Hence 
    \[
    F \;=\; \bigcup_{n=0}^{\infty} F_n,
    \quad\text{where}\quad
    F_n \;=\; \{\,A \subset D : |A| = n\}.
    \]
    Thus $F$ is a countable union of the sets $F_0, F_1, F_2, \dots$, 
    where $F_n$ is the collection of all $n$-element subsets of $D$.  
    
    \medskip
    \noindent
    \textbf{Step 2: Cardinality of each $F_n$.}
    Since $D$ is infinite of cardinality $d$, one shows (using standard cardinal arithmetic) 
    that for every fixed integer $n \ge 1$, we have
    \[
    Card(F_n) \;=\; \binom{d}{n} \;=\; d.
    \]
    (The intuitive reason: from an infinite set of size $d$, there are ``$d^n$'' ways 
    to choose an $n$-tuple of distinct elements, which still has cardinal $d$, and thus 
    the set of $n$-element subsets has cardinal $d$ as well.)
    
    \medskip
    \noindent
    \textbf{Step 3: A countable union of cardinal $d$.}
    We now write 
    \[
    F \;=\; F_0 \,\cup\, F_1 \,\cup\, F_2 \,\cup\, \cdots.
    \]
    Clearly $Card(F_0) = 1$ (there is only the empty subset), 
    and $Card(F_n) = d$ for $n \ge 1$.  
    Thus $F$ is a countable union of sets, each of cardinal at most $d$, 
    and at least one of them ($F_1$, say) has cardinal $d$.  
    Since $d$ is infinite and a countable union of sets of size at most $d$ still has size at most $d$, 
    it follows that
    \[
    Card(F)
    \;=\;
    \bigl|\bigcup_{n=0}^{\infty} F_n\bigr|
    \;=\; d.
    \]
    
    \medskip
    \noindent
    \textbf{Conclusion.}
    Hence the set of all finite subsets of an infinite set $D$ of cardinality $d$ itself has cardinality $d$:
    \[
    Card(F) = d.
    \] 
\end{solution}
\end{document}
