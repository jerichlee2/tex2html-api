\documentclass[12pt]{article}

% Packages
\usepackage[margin=1in]{geometry}
\usepackage{amsmath,amssymb,amsthm}
\usepackage{enumitem}
\usepackage{hyperref}
\usepackage{xcolor}
\usepackage{import}
\usepackage{xifthen}
\usepackage{pdfpages}
\usepackage{transparent}
\usepackage{listings}
\usepackage{tikz}
\usepackage{physics}
\usepackage{siunitx}
\usepackage{cancel}
\usepackage{booktabs}

  \usetikzlibrary{calc,patterns,arrows.meta,decorations.markings}


\DeclareMathOperator{\Log}{Log}
\DeclareMathOperator{\Arg}{Arg}

\lstset{
    breaklines=true,         % Enable line wrapping
    breakatwhitespace=false, % Wrap lines even if there's no whitespace
    basicstyle=\ttfamily,    % Use monospaced font
    frame=single,            % Add a frame around the code
    columns=fullflexible,    % Better handling of variable-width fonts
}

\newcommand{\incfig}[1]{%
    \def\svgwidth{\columnwidth}
    \import{./Figures/}{#1.pdf_tex}
}
\theoremstyle{definition} % This style uses normal (non-italicized) text
\newtheorem{solution}{Solution}
\newtheorem{proposition}{Proposition}
\newtheorem{problem}{Problem}
\newtheorem{lemma}{Lemma}
\newtheorem{theorem}{Theorem}
\newtheorem{remark}{Remark}
\newtheorem{note}{Note}
\newtheorem{definition}{Definition}
\newtheorem{example}{Example}
\newtheorem{corollary}{Corollary}
\theoremstyle{plain} % Restore the default style for other theorem environments
%

% Theorem-like environments
% Title information
\title{}
\author{Jerich Lee}
\date{\today}

\begin{document}

\maketitle
\section*{Measure Theory in Finite Element Analysis}

\subsection*{0.  Why ``measure theory’’ at all?}
At its heart, FEM approximates \emph{integral} identities that describe a
boundary–value problem (BVP).  
To make those integrals precise—especially on irregular domains and for
functions that may be only piecewise smooth—we need the full power of modern
integration theory:
\[
	\int_\Omega f(x)\,dx 
	\quad\text{means Lebesgue integration over a measurable set }
	\Omega\subset\mathbb{R}^d .
\]
Measure theory guarantees that these integrals exist, behave well under
limits, and interact correctly with differentiation (via the divergence,
Green, and trace theorems in their measure‑theoretic formulations).

\subsection*{1.  Weak formulations live in $L^p$ and Sobolev spaces}
\begin{itemize}
	\item \textbf{Lebesgue spaces.}
		The canonical energy or \emph{variational} norm for a PDE is almost
		always an $L^p(\Omega)$ norm, defined through the Lebesgue integral
		\(
			\|u\|_{L^p} = \bigl(\int_\Omega |u|^p\,dx\bigr)^{\!1/p}.
		\)
	\item \textbf{Sobolev spaces $W^{k,p}(\Omega)$.}
		Weak derivatives are defined via integration by parts; their very
		definition relies on Lebesgue integration and the ability to move
		derivatives onto test functions.  The FEM solution space
		$V_h\subset H^1(\Omega)=W^{1,2}(\Omega)$ is measurable,
		complete, and \emph{reflexive} because of measure theory.
\end{itemize}

\subsection*{2.  Existence and uniqueness (Lax–Milgram, Céa)}
Proofs hinge on:
\[
	a(u,v)=\int_\Omega A\nabla u\!\cdot\!\nabla v\,dx ,
	\quad
	\ell(v)=\int_\Omega fv\,dx ,
\]
with $a(\,\cdot\!,\cdot\,)$ coercive and continuous on $H^1(\Omega)$.
All three ingredients—integrals, norms, and dual pairings—are rigorously
measure‑theoretic.

\subsection*{3.  Defining the actual finite‑element space}
\begin{itemize}
	\item Each element \(K\) is a \emph{measurable} subset.
	\item Shape functions $\varphi_i$ live in $L^p(K)$ and inherit their
	      approximation properties (e.g.\ Bramble–Hilbert lemma) from
	      measure‑based norms.
\end{itemize}

\subsection*{4.  Numerical quadrature (Gauss, Dunavant, \dots)}
When we replace an exact integral by a quadrature rule,
\[
	\int_K g(x)\,dx \;\approx\; \sum_{q=1}^Q w_q\,g(x_q),
\]
error estimates ($\mathcal{O}(h^{m})$) are derived by comparing two \emph{measures}:
the exact Lebesgue measure $dx$ and the atomic measure
$\mu_Q=\sum w_q\delta_{x_q}$.

\subsection*{5.  Convergence and error analysis}
Key theorems such as
\[
	\|u - u_h\|_{H^1(\Omega)}
	\;\le\; C\,\inf_{v_h\in V_h}\|u-v_h\|_{H^1(\Omega)}
\]
rely on:
\begin{enumerate}
	\item \emph{Projection operators} $P_h:H^1\to V_h$ that are bounded in
	      $L^p$ and $H^1$ norms (both Lebesgue‑defined);
	\item \emph{Compactness} (Rellich–Kondrachov) and \emph{interpolation}
	      inequalities—both purely measure‑theoretic statements.
\end{enumerate}

\subsection*{6.  Irregular data and domains}
\begin{itemize}
	\item Loads in $H^{-1}$ or even in the space of bounded measures
	      $\mathcal{M}(\Omega)$ require integrals of the form
	      $\langle f, v\rangle = \int_\Omega v\,d\mu$, again a
	      measure‑theoretic pairing.
	\item For crack tips or re‑entrant corners, $u\notin H^2$; Lebesgue
	      theory is indispensable for estimating singular
	      behaviour and designing graded meshes.
\end{itemize}

\subsection*{7.  Beyond the deterministic setting}
Stochastic FEM introduces random parameters
\(
	\omega\mapsto \kappa(\omega,x)
\)
on a probability space $(\Omega_\mathrm{prob},\mathcal{F},\mathbb{P})$,
merging \emph{probability measures} with the Lebesgue measure on the
physical domain—an explicit product‑measure construction.

\subsection*{8.  Summary}
Measure theory is not a decorative abstraction; it
\begin{enumerate}
	\item legitimises every integral in the weak form,
	\item defines the function spaces where solutions live,
	\item provides the tools (dual spaces, projections, compactness) for
	      proving FEM convergence,
	\item underpins quadrature error bounds,
	\item and extends naturally to discontinuous, singular, or random data.
\end{enumerate}
Without it, the celebrated accuracy and reliability of the finite‑element
method would rest on heuristic ground.
\section*{Why Finite‐Element Analysis Still Matters in the Age of Machine Learning}

\begin{enumerate}
  \item \textbf{First‑principles insight \& interpretability.}\\
        FEA solves the governing \emph{partial differential equations}
        (balance of linear momentum, heat conduction, mass transport)
        subject to well‑understood boundary conditions.  
        The resulting stress, strain, and temperature fields have a
        one‑to‑one physical meaning, enabling root‑cause analysis
        and mechanism discovery—something ``black‑box’’ regressors
        cannot guarantee.

  \item \textbf{Reliable extrapolation to novel alloys and load cases.}\\
        ML excels at \emph{interpolation} inside its training envelope.
        When a new alloy composition, heat‑treatment, or loading path
        pushes the system into an unseen regime, physics‑based models
        (with material parameters calibrated once) maintain predictive
        power, whereas purely data‑driven surrogates often fail silently.

  \item \textbf{Sparse– or high‑cost–data regimes.}\\
        Obtaining comprehensive fatigue‑life data for a single
        Ti-6Al-4V batch can take months and $\mathcal{O}(10^5)$ in
        test coupons.
        FEA lets a researcher explore hundreds of \emph{virtual}
        geometries and loading histories from that same sparse dataset,
        amortising the cost.

  \item \textbf{Design iteration and gradient information.}\\
        Topology optimisation, shape sensitivity, and adjoint
        gradient‑based design all rely on the variational structure of
        the PDE.  
        FEA produces the consistent derivatives that drive these
        algorithms; generic ML models do not provide such physics‑aware
        sensitivities out‑of‑the‑box.

  \item \textbf{Verification, validation, and certification.}\\
        Regulatory bodies (FAA, EASA, DoD) already publish
        verification‐and‑validation guidelines for FEA of
        safety‑critical metallic components.
        No analogous, universally accepted framework yet exists for
        certifying ML predictions, especially those that cannot supply
        conservative bounds on failure risk.

  \item \textbf{Error control and adaptivity.}\\
        FEA offers \emph{a posteriori} error estimators that drive
        mesh-refinement algorithms until a target tolerance is met.
        Such guaranteed error bars are essential when a 1 % stress
        mis‑prediction can halve high‑cycle fatigue life.

  \item \textbf{Micro‐to‑macro scale bridging.}\\
        Crystal‑plasticity finite‑element models (CPFEM) explicitly
        resolve grain orientations and slip‑system kinetics, linking
        microstructural features to macroscopic anisotropy and fatigue
        crack initiation—physics that ML has difficulty inferring
        without enormous labelled microstructure datasets.

  \item \textbf{Synergy: FEA \emph{feeds} and \emph{accelerates} ML.}\\
        \begin{itemize}
          \item \emph{Data augmentation}: high‑fidelity FEA generates
                synthetic training data for ML regressors or classifiers.
          \item \emph{Model reduction}: ML surrogates (e.g.\ neural ODEs,
                graph networks) replace repeated FEA solves in
                real‑time digital twins, but are trained on an FEA
                backbone.
          \item \emph{Physics‑informed ML}: hybrid approaches embed the
                weak form of the PDE as a soft constraint in the loss
                function, marrying both worlds rather than superseding
                one with the other.
        \end{itemize}

  \item \textbf{Community trust and knowledge transfer.}\\
        Decades of metallurgical research—constitutive models, damage
        criteria, S–N curves—are encoded in FEA user subroutines
        (\texttt{UMAT}, \texttt{VUMAT}, \texttt{UEL}).
        Abandoning that trove would discard hard‑won domain knowledge
        that ML alone has yet to capture fully.
\end{enumerate}

\paragraph{Bottom line.}
Machine learning is transforming \emph{how} we run FEA (faster solvers,
surrogate models, automated parameter identification), but not \emph{why} we
need it.  
For structural metals—where safety factors are tight, microstructures
matter, and regulatory scrutiny is intense—physics‑grounded finite‑element
models remain the non‑negotiable foundation on which data‑driven
accelerators are being built.

\section*{Functional Analysis in Finite Element Analysis}

\subsection*{0.  Why “functional analysis’’?}
FEM is a \emph{Galerkin} projection of an operator equation
\(
	\mathcal{A}u = f
\)
posed on an infinite‑dimensional space $V$.  
Functional analysis provides the language (Hilbert/Banach spaces),
the tools (duality, projections, spectrum), and the theorems
(existence, stability, convergence) that make this projection rigorous.

\subsection*{1.  Casting the PDE as an operator equation}
\[
	\mathcal{A}:V \longrightarrow V', 
	\qquad 
	\text{find }u\in V\text{ such that } \langle \mathcal{A}u,\,v\rangle = \langle f,\,v\rangle \;\;\forall v\in V ,
\]
where $V$ is typically $H^1(\Omega)$ or $H(\mathrm{div};\Omega)$ and
$V'$ its dual.  
Everything here—weak derivatives, dual pairings—belongs to functional
analysis.

\subsection*{2.  Existence \& uniqueness: Lax–Milgram and Babuška–Brezzi}
\begin{itemize}
  \item \textbf{Hilbert spaces and bounded bilinear forms.}\\
        For symmetric problems, $a(u,v)$ is a bounded, coercive
        bilinear form on a Hilbert space; the Lax–Milgram theorem then
        guarantees a unique solution.
  \item \textbf{Inf–sup (BB) conditions.}\\
        Mixed formulations (e.g.\ Stokes, incompressible elasticity)
        rely on the Babuška–Brezzi inf–sup condition—again a purely
        functional‑analytic stability criterion.
\end{itemize}

\subsection*{3.  The Galerkin method as an \emph{orthogonal projection}}
Given a finite‑dimensional subspace $V_h\subset V$,
the Galerkin solution $u_h$ satisfies
\[
	u_h = P_h u,
	\qquad
	a(u-u_h,\,v_h)=0\quad\forall v_h\in V_h,
\]
where $P_h$ is the $a$‑orthogonal projector onto $V_h$.  
Properties of projections in Hilbert spaces (idempotence, stability)
drive the entire error analysis.

\subsection*{4.  Céa’s lemma (quasi‑optimality)}
\[
	\|u-u_h\|_V \;\le\; \frac{C}{\alpha}\;
	\inf_{v_h\in V_h}\|u-v_h\|_V ,
\]
with $\alpha$ the coercivity constant and $C$ the continuity constant of
the bilinear form.  
Céa’s lemma is nothing but the estimate for a bounded projector on a
normed space.

\subsection*{5.  Approximation theory \& interpolation operators}
\begin{itemize}
  \item \emph{Scott–Zhang}, \emph{Clément}, or canonical nodal
        interpolants $\Pi_h:V\to V_h$ are bounded linear operators;
        their approximation properties stem from functional‑analytic
        inequalities such as the Bramble–Hilbert lemma.
  \item Sobolev embeddings, Poincaré and trace inequalities—all
        functional‑analytic—supply the constants in those estimates.
\end{itemize}

\subsection*{6.  Eigenvalue problems \& spectral theory}
For
\(
	\mathcal{A}u = \lambda Mu
\)
(FEM stiffness $A$ and mass $M$ matrices) the continuous problem is
framed via compact, self‑adjoint operators.  
The min–max (Courant–Fischer) principle gives ordering and
convergence of discrete eigenpairs.

\subsection*{7.  Non‑linear operators and monotonicity}
When $\mathcal{A}$ is monotone or pseudo‑monotone (e.g.\ $p$‑Laplacian,
viscoplasticity), Browder–Minty and Leray–Schauder theorems guarantee
solvability.  
FEM inherits these results through discrete analogues defined on $V_h$.

\subsection*{8.  Duality, goal‑oriented error control}
A posteriori estimators such as the \emph{dual‑weighted residual} rely
on solving an \emph{adjoint} problem in $V'$, exploiting the reflexive
duality provided by functional analysis to translate residual norms
into observable error bounds.

\subsection*{9.  Compactness \& convergence of adaptive meshes}
Adaptive algorithms often need
\(
	V_h \to V \text{ in } H^1, \quad h\!\to\!0 ,
\)
which follows from Rellich–Kondrachov compactness and Banach–Alaoglu
weak convergence, ensuring that limit points of mesh‑refinement
sequences are indeed solutions.

\subsection*{10.  Summary}
Functional analysis supplies:
\begin{enumerate}
  \item the \emph{spaces} ($H^1$, $H(\mathrm{curl})$, $\dots$) and their duals,
  \item the \emph{operators} and their spectral or monotone properties,
  \item the \emph{projections} underpinning Galerkin orthogonality,
  \item the \emph{inequalities} driving approximation and stability,
  \item the \emph{convergence theorems}—for both linear and non‑linear,
        steady and eigenvalue problems.
\end{enumerate}
Strip these pieces away and FEM would be a collection of heuristics
without mathematical guarantees; with them, it is a provably convergent,
systematically improvable framework for solving PDEs in structural
mechanics and beyond.
\section*{Differential Geometry in Elasticity Theory}

\subsection*{0.  Why “differential geometry’’ at all?}
Large‑deformation elasticity describes a body that moves from a
\emph{reference configuration} $\mathcal{B}_0$ to a
\emph{current configuration} $\mathcal{B}_t$ via a smooth embedding
\[
	\boldsymbol{\varphi}:\mathcal{B}_0 \longrightarrow \mathbb{R}^3 .
\]
Treating $\mathcal{B}_0$ and $\mathcal{B}_t$ as 3‑dimensional
\emph{Riemannian manifolds} turns every kinematic and energetic quantity
into a coordinate‑free geometric object.

\subsection*{1.  Tangent map and deformation gradient}
\[
	F = T\boldsymbol{\varphi} : T\mathcal{B}_0 \longrightarrow T\mathcal{B}_t ,
	\qquad 
	F^i_{\;A} = \frac{\partial \varphi^i}{\partial X^A} .
\]
Here $T\mathcal{B}_0$ is the tangent bundle of the material manifold.
Interpreting $F$ as a \emph{bundle map} clarifies how it pushes forward
material line elements and how its pull‑back acts on covectors.

\subsection*{2.  Metric pull‑back and Green–Lagrange strain}
\[
	C := \varphi^{\!*}g_t = F^\top g_t F ,
	\qquad
	E := \tfrac12\!\left(C - g_0\right) ,
\]
where $g_0$ and $g_t$ are the metric tensors on
$\mathcal{B}_0$ and $\mathcal{B}_t$ respectively.
Strain is therefore \textit{intrinsically} the change in metric induced
by the deformation—not merely a matrix of numbers in a chosen chart.

\subsection*{3.  Volume forms and Jacobian}
Let $\mathrm{d}V_0$ and $\mathrm{d}v_t$ be the natural volume
elements associated with $g_0$ and $g_t$.
Then
\[
	J = \frac{\mathrm{d}v_t}{\mathrm{d}V_0} = \det F
\]
arises from the pull‑back of the volume form
$\varphi^{\!*}(\mathrm{d}v_t) = J\,\mathrm{d}V_0$.
Conservation of mass is simply the statement that $J\rho = \rho_0$.

\subsection*{4.  Covariant derivatives and balance laws}
\begin{itemize}
  \item \textbf{Material derivative:} 
        $\displaystyle \frac{\mathrm{D}}{\mathrm{D}t} = 
           \partial_t + v^i\nabla_i$,
        with $\nabla$ the Levi‑Civita connection of $g_t$.
  \item \textbf{Divergence theorem on manifolds:} 
        $\displaystyle \int_{\mathcal{B}_t}\!\!\nabla\!\cdot\!\sigma\,\mathrm{d}v_t
        = \int_{\partial\mathcal{B}_t}\!\!\sigma n\,\mathrm{d}a_t$,
        valid because $\nabla$ is metric‑compatible.
\end{itemize}
Using intrinsic divergence removes any dependence on Cartesian
coordinates and extends immediately to shells or rods embedded in
$\mathbb{R}^3$.

\subsection*{5.  Curvature and incompatibility}
\[
	R_{ABC}^{\;\;\;\;\;D} = 0 \quad\Longleftrightarrow\quad
	\text{strain field $E_{AB}$ is \emph{compatible}.}
\]
Compatibility conditions for strain are geometric statements about the
vanishing of the Riemann curvature of the induced connexion on
$\mathcal{B}_0$.  
In plates and shells, Gaussian curvature dictates whether a given
in‑plane strain can be realised without out‑of‑plane bending
(Gauss–Codazzi equations).

\subsection*{6.  Stress as a covector‑valued 2‑form}
The Cauchy stress can be written
\[
	\boldsymbol{\sigma}^\flat := \sigma_{ij}\,e^i\!\otimes\!e^j
	\quad\Longrightarrow\quad
	\mathfrak{t} = \star\,\boldsymbol{\sigma}^\flat
\]
where $\star$ is the Hodge star.  
Balance of linear momentum becomes
$\mathrm{d}\mathfrak{t} + \boldsymbol{b}\,\mathrm{d}v_t = 0$,
i.e.\ a single exterior‑calculus equation.

\subsection*{7.  Surface and shell theories}
Mid‑surfaces are 2‑manifolds $(\mathcal{S},\bar{g})$ whose
\emph{second fundamental form} $b$ encodes curvature.  
Koiter–Naghdi shell models express energy in terms of the change of
$(\bar{g},b)$—purely differential‑geometric invariants—under
deformation.

\subsection*{8.  Material symmetry and Lie groups}
Crystal rotations constitute the Lie group $\mathrm{SO}(3)$ acting on
the material manifold.  
Using group theory one writes constitutive laws as
\(
	W(F) = \widehat{W}(C) \quad\text{with}\quad
	\widehat{W}(Q^\top C Q)=\widehat{W}(C)\,\forall Q\in\mathrm{SO}(3),
\)
achieving frame‑indifference by construction.

\subsection*{9.  Geodesic distance in defect mechanics}
Dislocations and disclinations are modelled by torsion and curvature
of an \emph{affine connection}.  
The closure failure of Burgers circuits is measured by parallel
transport—another quintessential differential‑geometric concept.

\subsection*{10.  Summary}
Differential geometry supplies:
\begin{enumerate}
  \item Coordinate‑free definitions of strain, stress, and volume change;
  \item Intrinsic balance laws via covariant derivatives and Stokes’ theorem;
  \item Compatibility and curvature criteria tying strain to geometry;
  \item Natural frameworks for shells, rods, and crystal defects;
  \item Group‑theoretic tools ensuring material frame‑indifference.
\end{enumerate}
Without this geometric backbone, non‑linear elasticity would devolve
into cumbersome tensor calculus tied to specific coordinate charts;
with it, the theory becomes compact, elegant, and readily extensible to
complex bodies, surfaces, and micro‑structured materials.
\section*{What Does It Mean to \emph{Do} Differential Geometry Numerically?}

\subsection*{0.  Big picture}
A geometric object (manifold, vector field, connection, curvature,
$\dots$) is defined analytically by smooth maps and limits.
\emph{Numerical differential geometry} replaces those infinite‑precision
objects with \textbf{finite, computable data structures}—meshes,
simplicial complexes, sparse matrices—and algorithms whose outputs
converge to the smooth theory as the discretisation is refined.

\subsection*{1.  Discretising the manifold}
\begin{itemize}
  \item \textbf{Meshes:} triangles (2‑D), tetrahedra (3‑D), hexahedra,
        or higher‑order Bézier/NURBS patches (isogeometric analysis).
  \item \textbf{Point clouds:} unstructured samples augmented by
        neighbourhood graphs (for intrinsic methods on scanned data).
\end{itemize}
The mesh is a piecewise‑linear (or higher‑order) \emph{chart atlas}
providing local coordinates for computation.

\subsection*{2.  Discrete differential forms}
Via the \emph{de Rham complex}
\(
	0 \!\to\! \Omega^0 \!\xrightarrow{d}\! \Omega^1 \!\xrightarrow{d}\! \dots,
\)
one replaces smooth $k$‑forms by:
\[
	\underbrace{\text{vertex values}}_{\text{0‑forms}},\;
	\underbrace{\text{edge integrals}}_{\text{1‑forms}},\;
	\underbrace{\text{face fluxes}}_{\text{2‑forms}},\;\dots
\]
together with a \emph{coboundary} operator—an incidence matrix that
satisfies $d^2=0$ exactly on the computer.

\subsection*{3.  Exterior calculus on a computer}
Operations such as 
\(
	\mathrm{d},\;\star,\;\wedge,\;\delta=\pm\!\star\!d\star
\)
are encoded as sparse matrices acting on the discrete forms above.
This is the essence of \textbf{Discrete Exterior Calculus (DEC)} and
\textbf{Finite‑Element Exterior Calculus (FEEC)}.

\subsection*{4.  Metric information}
Geometry enters through:
\[
	\text{Mass matrix }M_k
	\;=\;\bigl[\!\int_{\text{cell}} \varphi_i \,\varphi_j\,\mathrm{d}A
	\bigr]
	\quad\text{or}\quad
	\bigl[\!\int_{\text{cell}} \alpha_i \!\wedge\! \star\alpha_j\bigr],
\]
built from local pull‑backs of the smooth metric.
These matrices weight inner products, Laplacians, and energies.

\subsection*{5.  Numerical curvature}
\begin{itemize}
  \item \emph{Gaussian curvature} via angle deficits (Regge calculus).
  \item \emph{Mean curvature} via cotangent Laplacian or surface FEM.
  \item \emph{Riemann curvature} tensors approximated by parallel
        transport around discrete loops (lattice gauge theory).
\end{itemize}

\subsection*{6.  Geodesics and parallel transport}
Algorithms such as:
\[
	\text{Dijkstra / fast marching (intrinsic distance)}, \qquad
	\text{geodesic shooting (variational, FEM)}, \qquad
	\text{log–exp maps (manifold optimisation)}.
\]
Parallel transport is realised via discrete connections: rotation angles
on edges or $SO(3)$ matrices on tetrahedral faces.

\subsection*{7.  Geometric PDEs}
Heat flow, mean‑curvature flow, Hodge–Laplace eigenproblems, and elastic
shell equations are solved with FEM/DEC/FEEC on the discrete manifold,
preserving identities like the discrete Leibniz rule and $d^2\!=\!0$.

\subsection*{8.  Structure preservation}
Good discretisations are \emph{mimetic}: they conserve topological
invariants (cohomology), symmetries (Noether), or energies exactly at
the discrete level, ensuring physical fidelity and numerical stability.

\subsection*{9.  Convergence + error control}
As mesh size $h\!\to\!0$ (or element order $p\!\to\!\infty$),
discrete operators and solutions converge to their smooth counterparts,
with rates proven by \textbf{approximation theory on manifolds}
(FEEC, isogeometric analysis) or empirically observed via mesh studies.

\subsection*{10.  Summary}
“Doing differential geometry numerically’’ means:
\begin{enumerate}
  \item Representing manifolds and tensor fields with finite data;
  \item Implementing discrete analogues of $\mathrm{d}$, $\star$, $\nabla$ that
        respect algebraic identities exactly;
  \item Solving geometric PDEs and optimisation problems on those
        discrete structures while controlling errors;
  \item Preserving geometric/topological invariants to maintain physical
        realism and stability.
\end{enumerate}
The result is a toolbox that brings the elegance of smooth geometry to
real‑world computation, from curvature‑driven surface design to general
relativity, robotics, and computer graphics.

\section*{FEM \& Exterior Calculus on a Computer:  A Unified View}

\subsection*{0.\  Same DNA, different dialects}
Both frameworks discretise \emph{weak forms} of PDEs:
\[
  \text{continuous problem:}\quad
  \int_\Omega \!\langle \mathcal{A}u , v \rangle \, \mathrm{d}x
  \;=\; \int_\Omega \!\langle f , v\rangle \, \mathrm{d}x ,
  \qquad \forall v\in V,
\]
but they package the ingredients differently:  
classical FEM speaks in \emph{scalars, vectors, tensors};  
exterior calculus speaks in \emph{$k$‑forms, $d$, $\star$, $\wedge$}.  
On a computer, the two dialects can be translated back and forth.

\subsection*{1.\  Whitney forms = lowest‑order finite elements}
For a simplicial mesh,
\[
  \text{\emph{Whitney~0‑forms}} \;\;\Longleftrightarrow\;\; P^1
  \text{ (linear Lagrange shape functions)},\\
  \text{\emph{Whitney~1‑forms}} \;\;\Longleftrightarrow\;\;
     \text{Nédélec edge elements},\\
  \text{\emph{Whitney~2‑forms}} \;\;\Longleftrightarrow\;\;
     \text{Raviart–Thomas face elements}.
\]
Thus a large subset of everyday FEM bases are \emph{already}
discrete differential forms.

\subsection*{2.\  Hilbert complexes and FEEC}
\begin{itemize}
  \item \textbf{Hilbert complex:}\;
        \( H\Lambda^0 \xrightarrow{d} H\Lambda^1 \xrightarrow{d}
        \cdots \) is a sequence of Sobolev spaces of differential forms.
  \item \textbf{Finite‑Element Exterior Calculus (FEEC)} chooses
        compatible subspaces
        \( V_h^k \subset H\Lambda^k \) such that
        \( d(V_h^k) \subset V_h^{k+1} \).
  \item \textbf{Pay‑off:}\; commuting diagrams $d\Pi_h = \Pi_h d$
        give stability and optimal error estimates \emph{automatically},
        reproducing Céa's lemma within each de-Rham block.
\end{itemize}

\subsection*{3.\  DEC: purely topological, metric added later}
\[
  C^k(\mathcal{T}) \xrightarrow{\;\delta\;} C^{k+1}(\mathcal{T}),
  \qquad \delta^2 = 0 \text{ exactly at the matrix level.}
\]
DEC fixes the \emph{coboundary} operator first (incidence matrices),
then introduces metric data via diagonal Hodge‑star matrices
(cotangent weights, Voronoi areas).  
It is the \emph{finite‑difference} analogue to FEEC’s finite‑element
approach.

\subsection*{4.\  Variational crimes made safe}
Classical FEM sometimes breaks algebraic identities
(e.g.\ $\nabla\!\times\nabla = 0$) on coarse meshes.  
DEC/FEEC preserve $d^2=0$, Poincaré duality, and Green–Stokes theorems
\emph{by construction}, giving:
\[
  \text{exact charge conservation in Maxwell,}\quad
  \text{exact mass conservation in geophysical flow,}\;\ldots
\]

\subsection*{5.\  What DEC/FEEC add to FEM}
\begin{enumerate}
  \item Coordinate‑free assembly of stiffness and mass matrices;
  \item Guaranteed compatibility for mixed problems
        (e.g.\ Stokes, elasticity with weak symmetry);
  \item Built‑in topological invariants (cohomology classes) that act
        as error checks for mesh defects.
\end{enumerate}

\subsection*{6.\  What FEM adds to DEC/FEEC}
\begin{enumerate}
  \item High‑order polynomial spaces ($p$‑version, $hp$ adaptive);
  \item Efficient quadrature and sparse linear‑algebra technology;
  \item Mature a~posteriori error estimators and mesh generators,
        crucial for large‑scale science/engineering simulations.
\end{enumerate}

\subsection*{7.\  Hybrid algorithms in practice}
\begin{itemize}
  \item \textbf{Electromagnetics:}\; Nédélec/Whitney elements with
        DEC‑style Hodge stars ensure lossless energy transfer in RF
        cavities.
  \item \textbf{Elastic shells:}\; FEEC provides $H^2$‑conforming
        spaces on curved surfaces; FEM handles geometric non‑linearity.
  \item \textbf{Fluids on the sphere:}\; DEC guarantees
        divergence‑free velocity; FEM gives high‑order accuracy for
        climate models.
\end{itemize}

\subsection*{8.\  Summary}
\begin{align}
  &\textbf{FEM} = \text{high-order polynomial approximation +
                    variational machinery},\\
  &\textbf{Exterior calculus on a computer} = \text{topology‑preserving
     algebra for }d,\star,\wedge.\\[4pt]
  &\textbf{Together:}\\
  &\phantom{=}\;\;
     \boxed{\text{Structure‑preserving, high‑accuracy,
              mesh‑flexible solvers for geometric PDEs.}}
\end{align}
\section*{Graduate–Level Mathematics Courses That Pay Dividends in Prof.\ Sehitoglu’s Lab}

\noindent
The High Temperature Materials Lab tackles \emph{fatigue\,/\,plasticity\,/\,phase transformation} problems that cut across
\[
  \text{continuum mechanics} \;+\;
  \text{atomistic simulations} \;+\;
  \text{data‐driven predictions}.
\]
A mathematics M.S.\ student can contribute most effectively by mastering the
courses below, grouped by theme and ordered (roughly) from \emph{foundational}
to \emph{lab‐specific impact}.%
\footnote{Where a title seems “applied,” check your math department;
many now cross‐list these courses with engineering.}

%-----------------------------------------------------------------------
\subsection*{1.\  Functional and Real Analysis (core toolkit)}
\begin{itemize}
  \item \textbf{Measure–theoretic Real Analysis}
        \\Solid footing for Lebesgue integration, Sobolev spaces, and weak
        solutions of PDEs used in fatigue and phase‐field models.
  \item \textbf{Functional Analysis}
        \\Banach/Hilbert space theory, Lax–Milgram, compactness—all
        underpin finite‑element convergence proofs and homogenisation
        theorems that appear in strain–gradient plasticity.
\end{itemize}

%-----------------------------------------------------------------------
\subsection*{2.\  Partial Differential Equations (linear $\Rightarrow$ nonlinear)}
\begin{itemize}
  \item \textbf{Elliptic \& Parabolic PDEs I\,/\,II}
        \\Existence, regularity, and maximum principles for
        diffusion‐type and elasticity equations.
  \item \textbf{Nonlinear PDEs}
        \\Critical for martensitic phase‐transformation models
        (e.g.\ Allen–Cahn, Cahn–Hilliard) and crystal plasticity
        formulations with rate dependence.
\end{itemize}

%-----------------------------------------------------------------------
\subsection*{3.\  Calculus of Variations \& Nonlinear Elasticity}
\begin{itemize}
  \item Direct methods, quasiconvexity, $\Gamma$‐convergence
        \\Provide the language for energy‐minimising microstructures,
        twin‐boundary motion, and fatigue crack‐tip fields.
\end{itemize}

%-----------------------------------------------------------------------
\subsection*{4.\  Numerical Analysis \& Scientific Computing}
\begin{itemize}
  \item \textbf{Finite Element Methods (FEM)}
        \\Mandatory for implementing crystal‐plasticity UMAT/VUMATs and
        phase‐field fatigue simulations in \textsc{abaqus}, \textsc{moose}, etc.
  \item \textbf{Numerical Linear Algebra}
        \\Sparse factorizations, Krylov solvers, pre‑conditioning for the
        very large FE systems that arise in 3‑D crack‐growth studies.
  \item \textbf{Computational PDEs / Adaptive Meshes}
        \\Vital when coupling atomistic length scales (nanometres) to
        macro fatigue lives (centimetres).
\end{itemize}

%-----------------------------------------------------------------------
\subsection*{5.\  Differential Geometry \& Tensor Calculus (for finite strains)}
\begin{itemize}
  \item Frame‐indifferent kinematics, covariant derivatives, pull‑back of
        metrics—precisely the language used in finite‐deformation
        plasticity and shape‑memory constitutive models.
\end{itemize}

%-----------------------------------------------------------------------
\subsection*{6.\  Probability, Stochastic Processes \& Uncertainty Quantification}
\begin{itemize}
  \item Fatigue life scatter is inherently statistical; crack‐nucleation
        models often couple Weibull statistics with deterministic FEM
        fields.  Bayesian UQ courses help quantify model/form parameter
        uncertainty.
\end{itemize}

%-----------------------------------------------------------------------
\subsection*{7.\  Dynamical Systems \& Asymptotic/Multiscale Analysis}
\begin{itemize}
  \item Bifurcations in cyclic loading, time‐scale separation between
        dislocation glide and fatigue crack advance, and
        homogenisation of heterogeneous alloys all draw on these topics.
\end{itemize}

%-----------------------------------------------------------------------
\subsection*{8.\  Optimisation \& Inverse Problems}
\begin{itemize}
  \item Gradient‐based calibration of crystal‐plasticity parameters,
        phase‐field mobility tensors, or DFT‐informed interatomic
        potentials relies on PDE‑constrained optimisation.
\end{itemize}

%-----------------------------------------------------------------------
\subsection*{9.\  Data-Driven and Machine-Learning Methods for Scientific Computing}
\begin{itemize}
  \item Physics‑informed neural networks (PINNs) and reduced‐order
        models accelerate FE simulations and ab‑initio workflows
        currently explored in the lab; a modern ML elective slots here.
\end{itemize}

%-----------------------------------------------------------------------
\subsection*{10.\  Cross‑listed Engineering Electives (optional but synergistic)}
\begin{itemize}
  \item \textbf{Continuum Mechanics / Elasticity / Plasticity}
  \item \textbf{Crystal Plasticity \& Phase‐Field Modeling}
\end{itemize}
Even one such course bridges abstract math to the specific material
behaviours Prof.\ Sehitoglu studies.

%-----------------------------------------------------------------------
\subsection*{Putting it together}
Aim for a balanced transcript:
\[
  \underbrace{\text{(1–3) rigorous analysis}}_{\text{proof‐level depth}}
  \;+\;
  \underbrace{\text{(4,7) numerical \& multiscale}}_{\text{computability}}
  \;+\;
  \underbrace{\text{(5,6,9) mechanics \& data}}_{\text{lab relevance}}.
\]
This blend positions you to
\begin{enumerate}
  \item formulate new fatigue / phase‐transformation PDEs,
  \item prove well‐posedness or derive reduced models,
  \item implement efficient FEM or ML‑accelerated solvers,
  \item quantify uncertainty—all skills that directly enhance the
        High Temperature Materials Lab’s mission.
\end{enumerate}
\section*{Should You Interlace Mathematics \& TAM Coursework?}

\subsection*{0.\  Guiding principle}
\[
  \textbf{Synergy} \;=\; 
    \bigl[\text{rigorous math foundation}\bigr]
    + \bigl[\text{mechanics depth}\bigr]
    + \bigl[\text{efficient time--to–degree}\bigr].
\]
Interlacing the \emph{right} mathematics electives with your TAM
sequence tends to \emph{increase} synergy and \emph{decrease} total
time spent re–learning prerequisites.  
The caveat: do not overload any single semester with all‑theory courses
\emph{and} the most concept‑heavy TAM offerings.

%--------------------------------------------------------------------
\subsection*{1.\  Identify the “mathematical spine’’ already inside TAM}

\begin{itemize}
  \item \textbf{TAM 541–542} (Mathematical Methods I–II)  
        \quad$\Rightarrow$ complex analysis, special functions, Greens
        functions, integral transforms.
  \item \textbf{TAM 549} (Asymptotic Methods)  
        \quad$\Rightarrow$ matched‐layer expansions, multiple scales.
  \item \textbf{TAM 574} (Advanced FEM)  
        \quad$\Rightarrow$ weak formulations, Sobolev norms.
\end{itemize}

\noindent
These already cover \emph{applied} complex analysis, perturbation, and
finite‑element theory.  
Pure‑math electives should therefore \emph{complement} (not duplicate)
these topics.

%--------------------------------------------------------------------
\subsection*{2.\  High‑impact math electives to weave in}

\vspace{-1em}
\begin{center}
\begin{tabular}{@{}llc@{}}
\toprule
\textbf{Math Course} & \textbf{Why it pairs well} & \textbf{Insert before$^\dagger$} \\ 
\midrule
Functional Analysis & Lax–Milgram, spectral theory                          & TAM 574, 551 \\
Measure‑theoretic Real Analysis & Lebesgue/Sobolev tools for PDE proofs      & TAM 549, 559 \\
Numerical Linear Algebra & Krylov, pre‑conditioning for large FE systems     & TAM 574, 570 \\
Partial Differential Equations I–II & Existence/regularity for elasticity    & TAM 551–552, 554 \\
Calculus of Variations & Energy principles, fracture, phase–field           & TAM 555, 561 \\
Probability & Stochastic fatigue, random media                               & TAM 557 \\
Optimization / Inverse Problems & Parameter ID, design                      & TAM 574, 598 (topics) \\
Differential Geometry & Finite‑strain kinematics, shell models               & TAM 545, 554 \\
\bottomrule
\end{tabular}
\end{center}

\begingroup\footnotesize
\noindent
$^\dagger$“Insert before’’ means \emph{take these math courses one
semester earlier} than the paired TAM class so the theory is fresh when
you need it.
\endgroup

%--------------------------------------------------------------------
\subsection*{3.\  A balanced 4‑semester template}

\begin{center}
\begin{tabular}{@{}cll@{}}
\toprule
\textbf{Sem} & \textbf{Core TAM} & \textbf{Math / Complement} \\
\midrule
1 & TAM 541 (Math Meth I), TAM 551 (Solid I)    & Real Analysis, Num.\ Linear Alg. \\
2 & TAM 542 (Math Meth II), TAM 552 (Solid II)  & Functional Analysis, PDE I \\
3 & TAM 574 (Adv.\ FEM), TAM 555 (Fracture)     & PDE II, Calc.\ of Variations \\
4 & TAM 545 (Cont.\ Mech.), TAM 559 (Atomistic) & Prob./UQ or Diff.\ Geometry \\
\bottomrule
\end{tabular}
\end{center}

\noindent
*Swap Fluid‑ or Wave‑track courses as needed; the logic—\emph{math
just‐ahead‐of‐application}—remains.*

%--------------------------------------------------------------------
\subsection*{4.\  When \emph{not} to interlace}
\begin{enumerate}
  \item Heavy experimental semester (e.g.\ TAM 537) where lab time is
        the bottleneck.
  \item Qualifying‑exam periods where mastering TAM content trumps
        additional theory.
  \item If math and TAM offerings clash in the timetable \emph{and}
        postponing the math does not delay research progress.
\end{enumerate}

%--------------------------------------------------------------------
\subsection*{5.\  Take‑away checklist}
\begin{enumerate}
  \item \textbf{Map each TAM course to its math prerequisite.}
        If it already \emph{teaches} the math, do not duplicate.
  \item \textbf{Front‑load} foundational pure‑math electives so later
        continuum mechanics feels “plug‑and‑play.’’
  \item \textbf{Pair} computational math (numerical linear algebra,
        scientific computing) directly with TAM 574/570 semesters.
  \item \textbf{Leave slack.}  One rigorous proof course + one
        concept‑heavy TAM class per term is usually the upper safe limit.
\end{enumerate}

\paragraph{Bottom line.}
Strategic interlacing beats strict separation: you reinforce theoretical
concepts just before applying them, shorten the learning curve for
advanced mechanics topics, and demonstrate an integrative skill set that
Prof.\ Sehitoglu’s lab—and the broader solid‑mechanics community—values
highly.

\section*{Which \textsc{Math} Courses Best Position You for
Prof.\ Sehitoglu’s  Fatigue\;/\,Plasticity Lab?}

\noindent
The lab’s core research themes
\[
  \underbrace{\text{non‑linear continuum mechanics}}_{\text{finite strain,
       crystal plasticity}}
  \;+\;
  \underbrace{\text{multiscale\,/\,atomistic modeling}}_{\text{ab‑initio,
       dislocation–twin interactions}}
  \;+\;
  \underbrace{\text{computational fatigue}}_{\text{phase‑field,
       FEM, UQ}}
\]
draw far more on \emph{analysis, geometry, numerics, probability} than on
\emph{abstract algebra, algebraic geometry, or set theory}.  
Below is a filtered “menu’’ of graduate mathematics that will
\emph{directly} accelerate your PhD trajectory, followed by a practical
2‑year (4‑semester) timetable that dovetails with finishing the TAM M.S.

%---------------------------------------------------------------------
\subsection*{1.\  High‑leverage courses (strongly recommended)}

\begin{tabular}{@{}p{2.8cm}p{9.6cm}@{}}
\toprule
\textbf{Course} & \textbf{Why it matters for fatigue/plasticity} \\
\midrule
\textbf{MATH 540 Real Analysis} &
Sobolev/Lebesgue machinery that underlies weak forms, energy methods,
and homogenisation. \\[0.2em]
\textbf{MATH 541 Functional Analysis} &
Lax–Milgram, compactness, spectral theory $\;\Rightarrow\;$ existence,
stability, and convergence proofs for FEM and phase‑field models. \\[0.2em]
\textbf{MATH 552 Numerical Methods for PDEs} &
Sparse solvers, error analysis, Krylov/AMG pre‑conditioning—
exactly what TAM 574 (\emph{Adv.\ FEM}) builds on. \\[0.2em]
\textbf{MATH 553 \& 554 PDE I–II} &
Linear operators, distribution theory, Fourier–Sobolev tools used in
elasticity and viscoelasticity (TAM 529). \\[0.2em]
\textbf{MATH 555 Non‑linear PDE} &
Variational inequalities, finite‑time blow‑up, and regularity—central
for phase‑field fracture, cyclic‑plasticity models. \\[0.2em]
\textbf{MATH 558 Applied Math Methods} &
Matched‑asymptotics, multiscale homogenisation, stochastic averaging—
connects atomistic length scales to macro fatigue life. \\[0.2em]
\textbf{MATH 518 Differentiable Manifolds I} &
Coordinate‑free strain, covariant derivatives, and pull‑backs used in
TAM 545 and finite‑strain crystal plasticity UMATs. \\[0.2em]
\textbf{MATH 522 Lie Groups I} &
SO(3) symmetry, lattice point groups, slip‑system rotations—minimal but
powerful group theory for crystallography. \\[0.2em]
\textbf{MATH 561 Probability I} &
Weibull life scatter, stochastic crack‑nucleation, uncertainty
quantification for fatigue design. \\[0.2em]
\textbf{MATH 589 Convex Analysis \& Optim.} &
PDE‑constrained optimisation for parameter ID \& microstructure‑aware
topology optimisation. \\
\bottomrule
\end{tabular}

%---------------------------------------------------------------------
\subsection*{2.\  Nice‑to‑have electives (if time/interest permits)}

\begin{itemize}
  \item \textbf{MATH 546 Hilbert Spaces} – spectral decompositions for
        elastodynamics (TAM 514) and wave motion (TAM 518).
  \item \textbf{MATH 520 Symplectic Geometry} – Hamiltonian structure of
        finite‑strain dynamics; useful but not essential.
  \item \textbf{MATH 562 Stochastic Processes} – random‑load spectra,
        Markov crack‑growth models.
\end{itemize}

\noindent
Courses \emph{least} connected to Sehitoglu‑style mechanics  
(Abstract/Commutative Algebra, Algebraic Geometry, Topology II, Logic,
Set Theory, most of Graph/Combinatorics) are intellectually rich but
unlikely to translate into lab impact.

%---------------------------------------------------------------------
\subsection*{3.\  Two‑year (4‑semester) math M.S.\ template}

\begin{center}
\begin{tabular}{@{}cll@{}}
\toprule
\textbf{Sem} & \textbf{Core Math} & \textbf{Elective/Bridge to TAM} \\
\midrule
1 & MATH 540 (Real An.) & MATH 553 (PDE I) \\
2 & MATH 541 (Func.\ An.) & MATH 552 (Num.\ PDE) \\
3 & MATH 554 (PDE II) & MATH 518 (Manifolds I) or 522 (Lie Gp) \\
4 & MATH 555 (Non‑lin.\ PDE) & MATH 558 (Applied Math) or 561 (Prob.)\\
\bottomrule
\end{tabular}
\end{center}

\noindent
*Overlap strategy*:  
Take MATH 540 during your \emph{final} TAM semester; the rest of the
sequence then lines up naturally with thesis research.

%---------------------------------------------------------------------
\subsection*{4.\  Research synergy tips}

\begin{itemize}
  \item \textbf{Integrate early}—use class projects in MATH 552/554/555
        to prototype phase‑field or FEM codes you can later port into
        \textsc{abaqus} \texttt{UEL}/\texttt{UMAT} frameworks.
  \item \textbf{Leverage Lie groups}—a small independent study on
        crystal symmetry links directly to slip‑system activation rules
        in Sehitoglu’s shape‑memory work.
  \item \textbf{Probability ties to experiments}—combine MATH 561 with
        TAM 557 (\emph{Random Media}) to model microstructure‑induced
        scatter in fatigue crack‑growth rates.
\end{itemize}

%---------------------------------------------------------------------
\subsection*{5.\  Bottom line}

Focus your mathematics M.S.\ on \textit{analysis, PDEs, numerics,
differential geometry, and probability}.  
These subjects are the proven workhorses behind cutting‑edge research in
fatigue, crystal plasticity, and multiscale materials modeling—the very
problems Prof.\ Sehitoglu’s lab will expect you to tackle during your
PhD.
%%--------------------------------------------------------------------
%%  Paste this into any LaTeX document.
%%  Requires:  \usepackage{amsmath,amssymb,booktabs}
%%--------------------------------------------------------------------

\section*{Integrated 7--Semester Plan\\
(M.Eng.\ Mechanical Engineering \textit{+\;} M.S.\ Mathematics)}

\subsection*{1.\ Three--Semester \textbf{M.Eng.\ Mechanical Engineering}
(Fall 2025 → Fall 2026)}

\begin{center}
\begin{tabular}{@{}ccp{8cm}p{4.9cm}@{}}
\toprule
\textbf{Term} & \textbf{Credits} & \textbf{Engineering Core \& Electives}
& \textbf{Math (banked for M.S.)} \\ \midrule
Fall 2025 & 12 &
\textbf{TAM 551 Solid Mechanics I} (4)\\[-2pt]
& & \textbf{TAM 541 Mathematical Methods I} (4) &
\textbf{MATH 540 Real Analysis} (4) \\[6pt]
Spring 2026 & 12 &
\textbf{TAM 552 Solid Mechanics II} (4)\\[-2pt]
& & \textbf{TAM 542 Mathematical Methods II} (4) &
\textbf{MATH 541 Functional Analysis} (4) \\[6pt]
Fall 2026 & 12 &
\textbf{TAM 531 Inviscid Flow} (4)\\[-2pt]
& & Professional–Development Elective$^\dagger$ (4) &
\textbf{MATH 518 Differentiable Manifolds I} (4) \\ \bottomrule
\end{tabular}
\end{center}

\noindent
\emph{M.Eng.\ credit summary:}\;
4xx h = 12,\; 5xx h = 20,\; prof‑dev = 4 \quad$\Rightarrow$\quad
\textbf{Total = 32 h}

%--------------------------------------------------------------------
\subsection*{2.\ Four--Semester \textbf{M.S.\ Mathematics}
(Fall 2026 → Spring 2028)}

\begin{center}
\begin{tabular}{@{}ccp{7.5cm}p{5.4cm}@{}}
\toprule
\textbf{Term} & \textbf{Credits} & \textbf{Core Analysis / Geometry / PDE}
& \textbf{Depth \& Application Electives} \\ \midrule
\multicolumn{4}{@{}l}{\textit{Carry‑over from Fall 2026:}
MATH 518 (4 h) already earned} \\[4pt]
Fall 2026 & +4 &
\textbf{MATH 552 Numerical Methods for PDEs} (4) & --- \\[6pt]
Spring 2027 & 12 &
\textbf{MATH 553 PDE I} (4)\\[-2pt]
& & \textbf{MATH 554 PDE II} (4) &
\textbf{MATH 561 Probability I} (4) \\[6pt]
Fall 2027 & 12 &
\textbf{MATH 555 Non‑linear PDE} (4) &
\textbf{MATH 522 Lie Groups I} (4)\\[-2pt]
& & & MATH 546 Hilbert Spaces (4) \\[6pt]
Spring 2028 & 12 &
\textbf{MATH 558 Applied Math Methods} (4) &
\textbf{MATH 589 Convex Analysis \& Optimization} (4)\\[-2pt]
& & & \textit{Optional:} MATH 562 Stochastic Processes (0–4) \\ \bottomrule
\end{tabular}
\end{center}

\noindent
\emph{Math M.S.\ credit summary:}\;
Analysis/PDE 20 h,\; Geometry/Group 8 h,\; Numerics/Prob/Optim 8 h
\quad$\Rightarrow$\quad
\textbf{Total = 36 h}  
(4 h cushion for seminar, reading, or thesis research if desired)

%--------------------------------------------------------------------
\subsection*{3.\ Milestone Calendar}

\begin{center}
\begin{tabular}{@{}ccc@{}}
\toprule
\textbf{2025--26 AY} & \textbf{2026--27 AY} & \textbf{2027--28 AY} \\ \midrule
\textbf{F 25:} M.Eng.\ Yr 1 start & \textbf{F 26:} Finish M.Eng.; begin Math M.S. &
\textbf{F 27:} Math M.S.\ Yr 2 (Non‑lin.\ PDE) \\[2pt]
\textbf{S 26:} Solid II + Func.\ An. &
\textbf{S 27:} Math M.S.\ Yr 1 spring (PDE pair) &
\textbf{S 28:} Math M.S.\ finish / thesis defense \\ \bottomrule
\end{tabular}
\end{center}

\bigskip
$^\dagger$ Examples: \textit{ENGR 572 Engineering Law},
\textit{ENGR 573 Tech Entrepreneurship}, or an approved internship/co‑op.

\medskip\noindent
\textbf{Target PhD start date:} August 2028

Subject: Prospective M.Eng. student interested in summer fatigue research

Dear Professor Sehitoglu,

My name is <Name>; I have applied for Fall 2025 entry to the
M.Eng. in Mechanical Engineering (decision expected late May).
My goal is to pursue fatigue and phase‑transformation research
and ultimately a PhD in your High‑Temperature Materials Lab.

• Background – B.S. ME (UIUC ’25, GPA 3.8).  Senior capstone: finite‑difference
  thermal modeling of steel compressor blades (TAM 470).  
• Skills – ABAQUS UMAT scripting, Python + NumPy for post‑processing,
  and introductory DFT (VASP) from a materials informatics project.

I will be in Champaign this summer (June – August) and can commit
10–15 h/week to a volunteer or hourly project—e.g., calibrating
a phase‑field fatigue model or helping a PhD student with dislocation
image‑correlation analysis.  If no slots are available, any advice on
datasets or reading you recommend would still be invaluable.

I have attached a two‑page résumé; a brief project portfolio is
here (<GitHub link>).  Would it be possible to discuss potential
involvement at your convenience?

Thank you for your time.

Best regards,

<Name>  
<UIUC NetID>  |  <Phone>
Subject: Summer research help for fatigue / thermo‑mechanical projects

Dear Professor Sehitoglu,

I will graduate this May with a B.S. in Engineering Mechanics (minor: Mathematics) and have applied for Fall 2025 entry to the M.Eng.–ME program; admission decisions arrive later this month.  My long‑term aim is a PhD focused on thermo‑mechanical fatigue and phase transformation in structural metals, and your High Temperature Materials Lab is an ideal fit.

• Senior capstone (TAM 470): finite‑difference model of 2‑D transient heat conduction from a stationary laser on steel; implemented Newton–Raphson iteration and von Neumann stability checks.  
• Internships:  
  – Caterpillar Inc.—Python automation of Fatigue Equivalent Load Analysis; full Rainflow implementation and documentation.  
  – Littelfuse Inc.—CNC manufacturing workflow → software tool to map fuse‑body geometries.  
• Tools: ABAQUS user subroutines (UMAT basics), Python/NumPy, GitHub portfolio (linked below).

I will be on campus 3 June – 9 August and can commit 10–15 h/week—volunteer or hourly—to assist with, for example, phase‑field fatigue calibration or slip/twin post‑processing.  If no positions are available, I would still value any recommended readings or datasets to study in preparation for graduate work.

Résumé (pdf) and brief code samples are attached.  Thank you for considering my request; I would welcome the chance to discuss how I might contribute this summer.

Best regards,

<Your Name>  
UIUC B.S. Engineering Mechanics '25  
\end{document}
