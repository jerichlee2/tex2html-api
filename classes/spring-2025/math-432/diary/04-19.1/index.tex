\documentclass[12pt]{article}

% Packages
\usepackage[margin=1in]{geometry}
\usepackage{amsmath,amssymb,amsthm}
\usepackage{enumitem}
\usepackage{hyperref}
\usepackage{xcolor}
\usepackage{import}
\usepackage{xifthen}
\usepackage{pdfpages}
\usepackage{transparent}
\usepackage{listings}
\usepackage{tikz}
  \usetikzlibrary{calc,patterns,arrows.meta,decorations.markings}


\DeclareMathOperator{\Log}{Log}
\DeclareMathOperator{\Arg}{Arg}

\lstset{
    breaklines=true,         % Enable line wrapping
    breakatwhitespace=false, % Wrap lines even if there's no whitespace
    basicstyle=\ttfamily,    % Use monospaced font
    frame=single,            % Add a frame around the code
    columns=fullflexible,    % Better handling of variable-width fonts
}

\newcommand{\incfig}[1]{%
    \def\svgwidth{\columnwidth}
    \import{./Figures/}{#1.pdf_tex}
}
\theoremstyle{definition} % This style uses normal (non-italicized) text
\newtheorem{solution}{Solution}
\newtheorem{proposition}{Proposition}
\newtheorem{problem}{Problem}
\newtheorem{lemma}{Lemma}
\newtheorem{theorem}{Theorem}
\newtheorem{remark}{Remark}
\newtheorem{note}{Note}
\newtheorem{definition}{Definition}
\newtheorem{example}{Example}
\newtheorem{corollary}{Corollary}
\theoremstyle{plain} % Restore the default style for other theorem environments
%

% Theorem-like environments
% Title information
\title{MATH 432: HW 9}
\author{Jerich Lee}
\date{\today}

\begin{document}

\maketitle
\begin{problem}
  Let $(M,D)$ be a metric space and let $A\subseteq M$.  
  Define the \emph{distance–to–$A$ function}
  \[
      f:M\longrightarrow\mathbb R,\qquad 
      f(x):=D(x,A):=\inf_{a\in A}D(x,a).
  \]
  Show that $f$ is continuous on $M$.
\end{problem}

\begin{proof}
  We break the argument into two parts.

  \begin{enumerate}
      \item \textbf{Lipschitz estimate.}  
            For arbitrary $x,y\in M$ we claim
            \begin{align}
                |f(x)-f(y)|\le D(x,y). \tag{1}
            \end{align}

            \begin{enumerate}
                \item Fix $x,y\in M$ and an element $a\in A$.  
                      By the triangle inequality,
                      \[
                          D(x,a)\le D(x,y)+D(y,a).
                      \]
                      Taking the infimum over $a\in A$ on both sides yields
                      \[
                          f(x)=\inf_{a\in A}D(x,a)\le D(x,y)+\inf_{a\in A}D(y,a)=D(x,y)+f(y),
                      \]
                      i.e.\ $f(x)-f(y)\le D(x,y)$.
                \item Interchanging the roles of $x$ and $y$ gives
                      $f(y)-f(x)\le D(x,y)$.
                \item Combining the two inequalities we obtain (1).
            \end{enumerate}

      \item \textbf{Continuity of $f$.}  
            Fix $x\in M$ and $\varepsilon>0$.  
            Choose $\delta:=\varepsilon$.  
            If $y\in M$ satisfies $D(x,y)<\delta$, then by (1)
            \[
                |f(x)-f(y)|\le D(x,y)<\delta=\varepsilon.
            \]
            Hence $f$ is continuous at $x$.  
            Because $x$ was arbitrary, $f$ is continuous on all of $M$.
  \end{enumerate}
\end{proof}

\begin{problem}
  Let $(M,D)$ be a metric space, let $A\subseteq M$ be \emph{closed},
  and choose a point $y\in M\setminus A$.  
  Show that there exists a \textbf{continuous}
  function $f:M\to\Bbb R$ such that
  \[
      f\!\restriction_{A}=0
      \quad\text{but}\quad 
      f(y)\neq 0 .
  \]
\end{problem}
%------------------------------------------------
% Distance–to–A map: concrete illustrations
%------------------------------------------------
\begin{example}
  Let $(M,D)$ be a metric space and $A\subseteq M$.  
  Define the \emph{distance–to–$A$} map
  \begin{align}
      f_A:M\longrightarrow\mathbb R_{\ge 0},\qquad 
      f_A(x)=D(x,A)=\inf_{a\in A} D(x,a).
  \end{align}
  Below are several explicit instances.

  \begin{enumerate}
      \item \textbf{Single point in $\mathbb R$.}\\
            $M=\mathbb R$, $A=\{0\}$.  Then
            \begin{align}
                f_A(x)=|x|.
            \end{align}

      \item \textbf{Closed interval in $\mathbb R$.}\\
            $M=\mathbb R$, $A=[a,b]$ with $a<b$.  Then
            \begin{align}
                f_A(x)=
                \begin{cases}
                    0,     & a\le x\le b,\\[4pt]
                    a-x,   & x<a,\\[4pt]
                    x-b,   & x>b.
                \end{cases}
            \end{align}

      \item \textbf{Integers inside $\mathbb R$.}\\
            $M=\mathbb R$, $A=\mathbb Z$.  Then
            \begin{align}
                f_A(x)=\min_{n\in\mathbb Z}|x-n|
            \end{align}
            is the distance from $x$ to the nearest integer.

      \item \textbf{Line in the plane.}\\
            $M=\mathbb R^{2}$ with the Euclidean metric,  
            $A=\{(x,0)\mid x\in\mathbb R\}$ (the $x$‑axis).  Then
            \begin{align}
                f_A(x,y)=|y|.
            \end{align}

      \item \textbf{Circle in the plane.}\\
            $M=\mathbb R^{2}$,  
            $A=\{(x,y)\mid x^{2}+y^{2}=r^{2}\}$ (radius $r>0$ centred at the origin).  
            Then
            \begin{align}
                f_A(x,y)=\Bigl|\sqrt{x^{2}+y^{2}}-r\Bigr|.
            \end{align}

      \item \textbf{Discrete metric space.}\\
            Let $D(x,y)=\begin{cases}0,&x=y,\\ 1,&x\neq y.\end{cases}$ on $M$.  
            For any $A\subseteq M$,
            \begin{align}
                f_A(x)=
                \begin{cases}
                    0, & x\in A,\\[4pt]
                    1, & x\notin A.
                \end{cases}
            \end{align}

      \item \textbf{Closed ball in $\mathbb R^{n}$.}\\
            $M=\mathbb R^{n}$,   
            $A=\overline{B}(c,R)=\{x\in\mathbb R^{n}\mid \|x-c\|\le R\}$.  Then
            \begin{align}
                f_A(x)=\max\{\|x-c\|-R,\,0\}.
            \end{align}
  \end{enumerate}
\end{example}
\begin{proof}
  \textbf{Step 1 – Define the candidate function.}  
  For every $x\in M$ set
  \[
      f(x):=D(x,A):=\inf_{a\in A}D(x,a).
  \]
  In the previous problem we proved that the map
  $x\mapsto D(x,A)$ is \emph{1‑Lipschitz}, hence continuous on $M$.

  \medskip
  \textbf{Step 2 – Verify that $f$ vanishes on $A$.}  
  If $x\in A$, then the infimum in the definition of $f(x)$ is attained at
  $a=x$, so
  \[
      f(x)=D(x,x)=0 .
  \]
  Thus $f\!\restriction_{A}=0$.

  \medskip
  \textbf{Step 3 – Show that $f(y)>0$.}  
  Because $A$ is closed and $y\notin A$, the point $y$ has positive
  distance from $A$:
  \[
      d:=D(y,A)=\inf_{a\in A}D(y,a)\;>\;0.
  \]
  Indeed, if $d$ were $0$ there would exist a sequence $(a_n)\subseteq A$
  with $D(y,a_n)<\tfrac1n$, whence $a_n\to y$.  
  Closedness of $A$ would then force $y\in A$, contradicting $y\notin A$.
  Consequently $f(y)=d\neq 0$.

  \medskip
  \textbf{Step 4 – Conclusion.}  
  The function $f$ is continuous on $M$, satisfies $f|_{A}=0$, and takes
  the non‑zero value $f(y)=d$ at the prescribed point $y$.  This completes
  the proof.
\end{proof}
\begin{problem}
  Let $X$ and $Y$ be metric spaces and $f:X\to Y$ be a map.

  \begin{enumerate}
      \item[(a)]  Suppose $X=\bigcup_{i\in I}U_i$ where each $U_i\subseteq X$ is
      \emph{open} and the restriction $f\!\restriction_{U_i}$ is continuous for
      every $i\in I$.  
      Show that $f$ is continuous on $X$.

      \item[(b)]  Let $n\in\Bbb N$ and suppose
      $
          X=\bigcup_{i=1}^{n}F_i
      $
      where each $F_i\subseteq X$ is \emph{closed} and 
      $f\!\restriction_{F_i}$ is continuous for $i=1,\dots,n$.
      Prove that $f$ is continuous on $X$.

      \item[(c)]  Decide whether the conclusion of part~(b) remains true if
      the finite family $\{F_i\}_{i=1}^{n}$ is replaced by an \emph{infinite}
      family of closed sets.  Justify your answer.
  \end{enumerate}
\end{problem}

\begin{proof}
  Throughout, recall the \emph{open–set characterisation} of continuity:
  $f$ is continuous on $X$ iff $f^{-1}(V)$ is open in $X$ for every open
  $V\subseteq Y$.

  \medskip
  \noindent\textbf{(a)  Union of open sets.}
  Let $V\subseteq Y$ be open.  Then
  \[
      f^{-1}(V)
      =\bigcup_{i\in I}\!\bigl(f^{-1}(V)\cap U_i\bigr)
      =\bigcup_{i\in I}\!\bigl(f\!\restriction_{U_i}\bigr)^{-1}(V).
  \]
  For each $i$, $f\!\restriction_{U_i}$ is continuous, so
  $\bigl(f\!\restriction_{U_i}\bigr)^{-1}(V)$ is open in $U_i$, and thus
  open in $X$ because $U_i$ itself is open.
  The union of \emph{any} collection of open subsets of $X$ is open, hence
  $f^{-1}(V)$ is open.  Therefore $f$ is continuous on~$X$.

  \medskip
  \noindent\textbf{(b)  Finite union of closed sets.}
  Let $G\subseteq Y$ be open.  Write
  \[
      f^{-1}(G)
      =\bigcup_{i=1}^{n}\!\bigl(f^{-1}(G)\cap F_i\bigr)
      =\bigcup_{i=1}^{n}\!\bigl(f\!\restriction_{F_i}\bigr)^{-1}(G).
  \]
  Because each $f\!\restriction_{F_i}$ is continuous,
  $\bigl(f\!\restriction_{F_i}\bigr)^{-1}(G)$ is open in $F_i$; i.e.\ its
  complement in $F_i$ is closed in $F_i$ and hence closed in $X$ (subspace
  closed $\;\Rightarrow\;$ closed in the ambient space).
  Consequently each \(
      \bigl(f\!\restriction_{F_i}\bigr)^{-1}(G)
  \) is closed in $X$.
  The union of \emph{finitely many} closed subsets of a metric
  space is closed, so $f^{-1}(G)$ is closed in $X$, i.e.\ open in $X$.
  Therefore $f$ is continuous on $X$.

  \medskip
  \noindent\textbf{(c)  Infinite union of closed sets – \emph{not} sufficient.}
  The conclusion of~(b) \emph{fails} for countably or arbitrarily many
  closed sets.

  \begin{example}
      Let $X=Y=\Bbb R$ with the usual metric, and set
      \[
          F_m:=\{m\},\qquad m\in\Bbb Z.
      \]
      Each $F_m$ is closed in $\Bbb R$.  
      Define
      \[
          f(x):=
          \begin{cases}
              1 & \text{if }x\in\Bbb Z,\\[4pt]
              0 & \text{otherwise.}
          \end{cases}
      \]
      \emph{Observation.}  
      For every $m\in\Bbb Z$ the restriction $f\!\restriction_{F_m}$ is
      trivially continuous (its domain is a singleton), yet $f$ is
      \emph{discontinuous} at every integer point.  Indeed,
      $\Bbb Z=\bigcup_{m\in\Bbb Z}F_m$ is an \emph{infinite} union of
      closed sets on which $f$ is continuous, but $f$ fails to be
      continuous on $X$ as a whole.
  \end{example}

  The key obstruction is that an arbitrary union of closed
  sets need not be closed; the proof technique in~(b) breaks down when
  the family is infinite because the pre–image $f^{-1}(G)$ becomes a union
  of infinitely many closed sets, which need \emph{not} be closed.
\end{proof}
\begin{theorem}[Finite intersection of open sets is open]    % 2.24 (c)
  Let $(X,d)$ be a metric space and let $G_{1},\dots,G_{n}\subseteq X$
  be \emph{open}.  Then their intersection
  \[
      H:=\bigcap_{i=1}^{n}G_{i}
  \]
  is also an open subset of $X$.
\end{theorem}

\begin{proof}
  The idea is to show that every point of $H$ possesses a radius
  neighbourhood which remains inside \emph{all} the sets $G_{i}$ at once.

  \medskip
  \noindent\textbf{Step 1 – Pick an arbitrary point of the intersection.}\\
  Fix $x\in H$.  
  By definition of $H$ this means $x\in G_{i}$ for \emph{each}
  index $i\in\{1,\dots,n\}$.

  \medskip
  \noindent\textbf{Step 2 – Exploit the openness of every $G_{i}$.}\\
  Because each $G_{i}$ is open, there exists a positive
  radius $r_{i}>0$ such that the open ball
  \[
      B(x,r_{i})=\{\,y\in X\mid d(x,y)<r_{i}\,\}
  \]
  is contained in $G_{i}$:
  \[
      B(x,r_{i})\subseteq G_{i}\qquad(i=1,\dots,n).
  \]

  \medskip
  \noindent\textbf{Step 3 – Choose a \emph{uniform} radius.}\\
  Define
  \[
      r:=\min\{\,r_{1},r_{2},\dots,r_{n}\,\}\;>\;0.
  \]
  (The minimum exists and is strictly positive because we are taking the
  minimum of finitely many positive numbers.)

  \medskip
  \noindent\textbf{Step 4 – Verify inclusion for the common ball.}\\
  For any point $y\in B(x,r)$ we have $d(x,y)<r\le r_{i}$ for
  \emph{every} $i$, hence
  \[
      y\in B(x,r_{i})\subseteq G_{i}\quad\text{for all }i.
  \]
  Consequently $y\in\bigcap_{i=1}^{n}G_{i}=H$; that is,
  \[
      B(x,r)\subseteq H.
  \]

  \medskip
  \noindent\textbf{Step 5 – Conclude openness of $H$.}\\
  We have shown that for the arbitrary point $x\in H$ there exists a
  neighbourhood \(B(x,r)\) fully contained in $H$.  
  Therefore every point of $H$ is an interior point of $H$, which by
  definition means that \(H\) is open in \(X\).
\end{proof}

\begin{problem}
  Show that every metric space $(M,D)$ is \textbf{homeomorphic} to a metric
  space of \emph{finite diameter} (e.g.\ $\le 1$).
\end{problem}

\begin{proof}
  We proceed in five steps.

  \begin{enumerate}
      \item \textbf{Compress the non–negative reals into a bounded
            interval.}

            Define 
            \[
                f:[0,\infty)\longrightarrow[0,1), 
                \qquad 
                f(t)=\frac{t}{1+t}.
            \]

            \begin{enumerate}
                \item $f(0)=0$ and $\displaystyle\lim_{t\to\infty}f(t)=1$, so
                      $f$ maps \([0,\infty)\) onto \([0,1)\).
                \item $f'(t)=\dfrac{1}{(1+t)^2}>0$ for all $t\ge0$
                      $\;\Rightarrow\;$ \emph{strictly increasing}.
                \item $f''(t)=-\dfrac{2}{(1+t)^3}<0$
                      $\;\Rightarrow\;$ \emph{concave}.
            \end{enumerate}

      \item \textbf{Define a new metric of diameter $<1$.}

            Put
            \[
                \widetilde D(x,y):=f\!\bigl(D(x,y)\bigr),
                \qquad
                x,y\in M.
            \]
            Since $D(x,y)\ge0$, we have $0\le\widetilde D(x,y)<1$; hence
            $(M,\widetilde D)$ has \emph{diameter} at most~$1$.

      \item \textbf{$\widetilde D$ is indeed a metric.}

            Let $x,y,z\in M$ be arbitrary.

            \begin{enumerate}
                \item[(i)] \emph{Positivity \& identity of indiscernibles:}\\
                      Because $f(0)=0$ and $f(t)>0$ for $t>0$, we get
                      $\widetilde D(x,y)=0\iff D(x,y)=0\iff x=y$.
                \item[(ii)] \emph{Symmetry:}  
                      $D(x,y)=D(y,x)\ \Rightarrow\ 
                      \widetilde D(x,y)=\widetilde D(y,x)$.
                \item[(iii)] \emph{Triangle inequality:}\\
                      By the triangle inequality for $D$,
                      $D(x,z)\le D(x,y)+D(y,z)$.
                      Because $f$ is increasing,
                      \[
                          \widetilde D(x,z)=f\!\bigl(D(x,z)\bigr)
                          \le f\!\bigl(D(x,y)+D(y,z)\bigr).
                      \]
                      It therefore suffices to prove the \textbf{sub–additivity}
                      \[
                          f(a+b)\le f(a)+f(b),
                          \qquad a,b\ge0. \tag{$\ast$}
                      \]
                      A direct calculation yields
                      \[
                          f(a)+f(b)-f(a+b)
                          =\frac{a}{1+a}+\frac{b}{1+b}-\frac{a+b}{1+a+b}
                          =\frac{ab}{(1+a)(1+b)(1+a+b)}\;\ge\;0,
                      \]
                      verifying $(\ast)$.  Hence
                      $\widetilde D(x,z)\le
                      \widetilde D(x,y)+\widetilde D(y,z)$.
            \end{enumerate}

      \item \textbf{The identity map is a bi–continuous bijection.}

            The underlying sets of $(M,D)$ and $(M,\widetilde D)$ coincide, so
            the identity map $\operatorname{id}_M$ is bijective.
            To see it is continuous $(M,D)\to(M,\widetilde D)$, fix $x\in M$
            and $\varepsilon>0$; choose $\delta:=\dfrac{\varepsilon}{1-\varepsilon}$.
            If $D(x,y)<\delta$, then
            \[
                \widetilde D(x,y)
                =\frac{D(x,y)}{1+D(x,y)}
                <\frac{\delta}{1+\delta}=\varepsilon.
            \]
            Thus $\operatorname{id}_M$ is continuous.
            Since $f$ is strictly increasing, the same argument with
            $f^{-1}(s)=\dfrac{s}{1-s}$ shows continuity of the inverse
            identity map $(M,\widetilde D)\to(M,D)$.

      \item \textbf{Conclusion.}

            The identity map furnishes a \emph{homeomorphism}
            \[
                (M,D)\;\longleftrightarrow\;(M,\widetilde D),
            \]
            and the latter space has finite (in fact $\le1$) diameter.
            Therefore \emph{every} metric space is homeomorphic to one of
            finite diameter.
  \end{enumerate}
\end{proof}
\begin{problem}
  Let $X$ be a \emph{dense} subset of a metric space $(M,D)$.  
  Assume that \textbf{every} Cauchy sequence whose terms lie in $X$
  converges to a point of $M$.  
  Prove that $M$ is complete; that is, every Cauchy sequence in $M$
  converges in $M$.
\end{problem}

\begin{proof}
  Let $(y_n)_{n\ge1}$ be an \emph{arbitrary} Cauchy sequence in $M$.  
  We construct an auxiliary sequence in $X$ that “shadows’’ $(y_n)$ and
  then show that both sequences have the same limit.

  \begin{enumerate}
      \item[\textbf{Step 1.}] \textbf{Approximate each $y_n$ by a point in $X$.}  
            Density of $X$ guarantees, for every $n\in\Bbb N$,  
            the existence of $x_n\in X$ with
            \[
                D(x_n,y_n)<\frac1n. \tag{1}
            \]

      \item[\textbf{Step 2.}] \textbf{$(x_n)$ is Cauchy in $X$.}  
            Fix $\varepsilon>0$.  
            Because $(y_n)$ is Cauchy, choose $N_0\in\Bbb N$ such that
            \[
                D(y_m,y_n)<\frac{\varepsilon}{2}
                \quad\text{for all }m,n\ge N_0. \tag{2}
            \]
            Pick $N\in\Bbb N$ satisfying
            \(
                N\ge N_0\ \text{and}\ N\ge\dfrac{2}{\varepsilon}. 
            \)
            For $m,n\ge N$ we combine (1) and (2) with the triangle
            inequality:
            \[
                D(x_m,x_n)
                \le D(x_m,y_m)+D(y_m,y_n)+D(y_n,x_n)
                <\frac1m+\frac{\varepsilon}{2}+\frac1n
                \le\frac{\varepsilon}{2}+\frac{\varepsilon}{2}
                =\varepsilon.
            \]
            Hence $(x_n)$ is a Cauchy sequence in $X$.

      \item[\textbf{Step 3.}] \textbf{Convergence of $(x_n)$.}  
            By hypothesis, every Cauchy sequence in $X$ converges in $M$;
            therefore there exists $y\in M$ with
            \[
                \lim_{n\to\infty}x_n=y. \tag{4}
            \]

      \item[\textbf{Step 4.}] \textbf{Show that $y_n\to y$.}  
            Let $\varepsilon>0$ be given.  
            Choose $N_1\in\Bbb N$ so large that
            \(
                D(x_n,y)<\dfrac{\varepsilon}{2}
            \)
            for all $n\ge N_1$ (possible by (4)).
            Pick $N_2\in\Bbb N$ with $N_2\ge\max\{N_1,\,2/\varepsilon\}$.
            For $n\ge N_2$ we have, by (1),
            \[
                D(y_n,y)
                \le D(y_n,x_n)+D(x_n,y)
                <\frac1n+\frac{\varepsilon}{2}
                \le\frac{\varepsilon}{2}+\frac{\varepsilon}{2}
                =\varepsilon.
            \]
            Since $\varepsilon$ was arbitrary, $y_n\to y$.

      \item[\textbf{Step 5.}] \textbf{Completion of the proof.}  
            The original Cauchy sequence $(y_n)$ converges in $M$ to $y$.
            Because $(y_n)$ was arbitrary, \emph{every} Cauchy sequence in
            $M$ converges; thus $(M,D)$ is complete.
  \end{enumerate}
\end{proof}
\end{document}
