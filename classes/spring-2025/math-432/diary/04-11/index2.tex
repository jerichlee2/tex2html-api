\documentclass[12pt]{article}

% Packages
\usepackage[margin=1in]{geometry}
\usepackage{amsmath,amssymb,amsthm}
\usepackage{enumitem}
\usepackage{hyperref}
\usepackage{xcolor}
\usepackage{import}
\usepackage{xifthen}
\usepackage{pdfpages}
\usepackage{transparent}
\usepackage{listings}
\usepackage{tikz}

\DeclareMathOperator{\Log}{Log}
\DeclareMathOperator{\Arg}{Arg}

\lstset{
    breaklines=true,         % Enable line wrapping
    breakatwhitespace=false, % Wrap lines even if there's no whitespace
    basicstyle=\ttfamily,    % Use monospaced font
    frame=single,            % Add a frame around the code
    columns=fullflexible,    % Better handling of variable-width fonts
}

\newcommand{\incfig}[1]{%
    \def\svgwidth{\columnwidth}
    \import{./Figures/}{#1.pdf_tex}
}
\theoremstyle{definition} % This style uses normal (non-italicized) text
\newtheorem{solution}{Solution}
\newtheorem{proposition}{Proposition}
\newtheorem{problem}{Problem}
\newtheorem{lemma}{Lemma}
\newtheorem{theorem}{Theorem}
\newtheorem{remark}{Remark}
\newtheorem{note}{Note}
\newtheorem{definition}{Definition}
\newtheorem{example}{Example}
\newtheorem{corollary}{Corollary}
\theoremstyle{plain} % Restore the default style for other theorem environments
%

% Theorem-like environments
% Title information
\title{}
\author{Jerich Lee}
\date{\today}

\begin{document}

\maketitle
\begin{problem}
    \begin{align}
        \text{Let } d,e_1,e_2 \text{ be cardinals.  Prove that } 
        d^{\,e_1+e_2}=d^{\,e_1}\,d^{\,e_2}.
    \end{align}

    \begin{enumerate}
        \item\textbf{Choose representatives of the cardinals.}\\
              Pick pairwise disjoint sets $D,E_1,E_2$ such that 
              $|D|=d,\;|E_1|=e_1,$ and $|E_2|=e_2$.

        \item\textbf{Translate the cardinal equation into a statement about sets of functions.}\\
              For any sets $X$ and $Y$ write $H(X,Y)$ for the set of all functions $X\to Y$.  
              Set
              \begin{align}
                  A&=H(E_1\cup E_2,D),\\
                  B&=H(E_1,D)\times H(E_2,D).
              \end{align}
              Then 
              \begin{align}
                  |A|&=d^{\,e_1+e_2}, &
                  |B|&=d^{\,e_1}\,d^{\,e_2}.
              \end{align}
              Hence it suffices to construct a bijection $F:A\to B$.

        \item\textbf{Define the forward map $F$.}\\
              Given $f\in A$ (so $f:E_1\cup E_2\to D$), define its restrictions
              \begin{align}
                  f_1&=f|_{E_1}:E_1\to D, &
                  f_2&=f|_{E_2}:E_2\to D,
              \end{align}
              and set 
              \begin{align}
                  F(f)=(f_1,f_2)\in B.
              \end{align}

        \item\textbf{Define the inverse map $G$.}\\
              For $(g_1,g_2)\in B$ (so $g_1:E_1\to D$ and $g_2:E_2\to D$), define
              \begin{align}
                  G(g_1,g_2)(a)=
                  \begin{cases}
                      g_1(a), & a\in E_1,\\[4pt]
                      g_2(a), & a\in E_2.
                  \end{cases}
              \end{align}
              Because $E_1\cap E_2=\varnothing$, the definition is unambiguous and
              $G(g_1,g_2)\in A$.

        \item\textbf{Verify that $F$ and $G$ are inverse to each other.}\\
              \begin{align}
                  (F\circ G)(g_1,g_2)
                  &=F\!\bigl(G(g_1,g_2)\bigr)
                    =(g_1,g_2)
                    =\operatorname{id}_B(g_1,g_2),\\
                  (G\circ F)(f)
                  &=G\!\bigl(F(f)\bigr)
                    =f
                    =\operatorname{id}_A(f),
              \end{align}
              for all $(g_1,g_2)\in B$ and all $f\in A$.

        \item\textbf{Conclude.}\\
              Since $F$ is a bijection between $A$ and $B$, their cardinalities are equal:
              \begin{align}
                  d^{\,e_1+e_2}=|A|=|B|=d^{\,e_1}\,d^{\,e_2}.
              \end{align}
              \qedhere
    \end{enumerate}
\end{problem}

\begin{problem}
    \begin{align}
        \text{Suppose } e \text{ is an infinite cardinal and } 2\le d\le 2^{\,e}.
        \text{ Prove that } d^{\,e}=2^{\,e}.
    \end{align}

    %--- parameters (as requested) --------------------------------------------
    \begin{align}
        &e \colon \text{infinite cardinal},\\
        &d \colon \text{cardinal with } 2\le d\le 2^{\,e}.
    \end{align}

    \begin{enumerate}
        \item \textbf{Lower bound.}\\
              Because $2\le d$ and exponentiation by a fixed infinite
              cardinal is \emph{monotone}, we have
              \begin{align}
                  2^{\,e}\;\le\;d^{\,e}.
              \end{align}

        \item \textbf{Upper bound.}\\
              From $d\le 2^{\,e}$ and the same monotonicity property,
              \begin{align}
                  d^{\,e}\;\le\;(2^{\,e})^{\,e}.
              \end{align}

        \item \textbf{Simplify the right‐hand side.}\\
              Cardinal arithmetic gives
              \begin{align}
                  (2^{\,e})^{\,e}=2^{\,e\cdot e}.
              \end{align}
              For an \emph{infinite} cardinal $e$ we have $e\cdot e=e$
              (there is a bijection between $e\times e$ and $e$), so
              \begin{align}
                  (2^{\,e})^{\,e}=2^{\,e}.
              \end{align}

        \item \textbf{Combine the inequalities.}\\
              Putting the previous steps together,
              \begin{align}
                  2^{\,e}\;\le\;d^{\,e}\;\le\;2^{\,e},
              \end{align}
              whence
              \begin{align}
                  d^{\,e}=2^{\,e}.
              \end{align}
              \qedhere
    \end{enumerate}
\end{problem}

\begin{problem}
    \begin{align}
        \text{Let $D$ be an infinite set with $|D|=d$.  Prove that $D$ possesses $2^{d}$ distinct subsets of cardinality $d$.}
    \end{align}

    %----------------------- parameters ---------------------------------------
    \begin{align}
        D &: \text{infinite set},\\
        d &= |D|.
    \end{align}

    \begin{enumerate}
        \item \textbf{Split $D$ into two large pieces.}\\
              For any infinite cardinal $d$ we have $d+d=d$.  
              Hence we can choose disjoint subsets
              \begin{align}
                  D_1,\;D_2\subseteq D
              \end{align}
              such that
              \begin{align}
                  |D_1|=|D_2|=d
                  \quad\text{and}\quad
                  D_1\cup D_2=D.
              \end{align}

        \item \textbf{Inject $P(D_1)$ into the family of $d$–element subsets of $D$.}\\
              Define
              \begin{align}
                  F:P(D_1)\longrightarrow P(D),
                  \qquad
                  F(B)=B\cup D_2.
              \end{align}

        \item \textbf{$F$ is injective.}\\
              If $B_1\neq B_2$ then $B_1\cup D_2\neq B_2\cup D_2$ because 
              $D_2$ is common to both unions while $B_1$ and $B_2$ differ inside $D_1$.

        \item \textbf{Each image has size $d$.}\\
              For any $B\subseteq D_1$
              \begin{align}
                  |F(B)|
                  &=|B\cup D_2|
                    =|B|+|D_2|
                    =d,
              \end{align}
              since $|B|\le d$ and $|D_2|=d$.  
              Thus
              \begin{align}
                  \operatorname{Im}F\subseteq 
                  Y=\{A\subseteq D:\,|A|=d\}.
              \end{align}

        \item \textbf{Cardinality bounds for $Y$.}\\
              Injectivity yields
              \begin{align}
                  2^{d}=|P(D_1)|\le |Y|.
              \end{align}
              Trivially $Y\subseteq P(D)$, so $|Y|\le 2^{d}$.

        \item \textbf{Conclusion.}\\
              Combining the inequalities,
              \begin{align}
                  2^{d}\le |Y|\le 2^{d}\;\Longrightarrow\;|Y|=2^{d}.
              \end{align}
              Hence $D$ has exactly $2^{d}$ subsets whose cardinality is $d$.
              \qedhere
    \end{enumerate}
\end{problem}
%--- clarification of the notation used in Step 2 ----------------------------
\begin{enumerate}
    \setcounter{enumi}{1}  % so it shows up as item 2.
    \item \textbf{Inject $P(D_1)$ into the family of $d$–element subsets of $D$.}

          \medskip\noindent
          \emph{What is $B$?}\;
          By definition $P(D_1)$ is the power set of $D_1$, so its elements
          are \emph{all} subsets of $D_1$.  We use the letter
          $B$ to denote a \emph{generic} element of $P(D_1)$:
          \[
              B\;\subseteq\;D_1.
          \]

          \medskip\noindent
          \emph{What is the map $F$?}\;
          \[
              F : P(D_1)\longrightarrow P(D),
              \qquad
              F(B)=B\cup D_2.
          \]
          \begin{itemize}
              \item \textbf{Domain:} every subset $B$ of $D_1$.
              \item \textbf{Codomain:} the power set $P(D)$.
              \item \textbf{Rule:} adjoin the fixed block $D_2$ to $B$.
          \end{itemize}

          Because $D_2$ is contained in \emph{every} image, two different
          subsets $B_1\neq B_2$ of $D_1$ yield different unions:
          \[
              F(B_1)=B_1\cup D_2\;\neq\;B_2\cup D_2=F(B_2),
          \]
          so $F$ is injective.

          Moreover, for each $B\subseteq D_1$ we have
          \[
              |F(B)|=|B\cup D_2|=|B|+|D_2|=d,
          \]
          using $|B|\le d$, $|D_2|=d$, and $d+d=d$ (valid for infinite
          cardinals).  Hence the image of $F$ lies inside the set
          \(
              Y=\{A\subseteq D:|A|=d\},
          \)
          and the injection establishes the lower bound
          $2^{d}=|P(D_1)|\le |Y|$ used later in the proof.
\end{enumerate}
%---------------------------------------------------------------------------
% Continuation / elaboration of Steps 4 and 5
\begin{enumerate}
    \setcounter{enumi}{3}  % start at 4.
    \item \textbf{Each image has size $d$.}\\
          For any $B\subseteq D_{1}$ we have
          \begin{align}
              |F(B)| &= |B\cup D_{2}| \\
                     &= |B| + |D_{2}| \tag{$B,D_{2}$ disjoint}.
          \end{align}
          Now $|B|\le d$ and $|D_{2}|=d$, and for infinite cardinals
          $d+d=d$ and $|B|+d=d$, so $|F(B)|=d$.  Hence
          \[
              \operatorname{Im}F \subseteq
              Y=\{A\subseteq D : |A|=d\}.
          \]

    \item \textbf{Cardinality bounds for $Y$.}\\
          Since $F$ is injective,
          \[
              |P(D_{1})| \le |\operatorname{Im}F| \le |Y|.
          \]
          Because $|D_{1}|=d$ we have $|P(D_{1})|=2^{d}$, giving the lower
          bound $2^{d}\le |Y|$.  Trivially $Y\subseteq P(D)$, so
          $|Y|\le 2^{d}$.  Combining,
          \[
              2^{d}\le |Y|\le 2^{d}\quad\Longrightarrow\quad |Y|=2^{d}.
          \]
          \qedhere
\end{enumerate}

%-------------------------------------------------------------------------------
% “Big‑picture” summary of the proof that there are $2^{d}$ subsets of $D$
% having size $d$, together with an explicit description of the set $Y$.
%-------------------------------------------------------------------------------
\begin{problem}
    \begin{align}
        \text{Let $D$ be an infinite set with } |D|=d
        \quad\Longrightarrow\quad
        \bigl|\{A\subseteq D:|A|=d\}\bigr| = 2^{d}.
    \end{align}

    \begin{enumerate}
        \item \textbf{Duplicate the set inside itself.}\\
              Because $d$ is an \emph{infinite} cardinal, 
              \(
                  d+d=d.
              \)
              Hence we can split $D$ into two disjoint blocks
              \(
                  D_{1},D_{2}\subseteq D
              \)
              with
              \(
                  D_{1}\cap D_{2}=\varnothing,
                  \;
                  D_{1}\cup D_{2}=D,
                  \;
                  |D_{1}|=|D_{2}|=d.
              \)

        \item \textbf{Encode subsets of $D_{1}$ by adding the wrapper $D_{2}$.}\\
              Define
              \[
                  F : P(D_{1}) \longrightarrow P(D),
                  \qquad
                  F(B)=B\cup D_{2}.
              \]
              The fixed “wrapper’’ $D_{2}$ ensures each image still has
              size $d$, while the variable part $B$ carries the information.

        \item \textbf{Injectivity gives a lower bound.}\\
              If $B_{1}\neq B_{2}$ then
              \(
                  F(B_{1})\neq F(B_{2}),
              \)
              so $F$ is injective.  
              Therefore
              \(
                  |P(D_{1})| \le |\operatorname{Im}F|.
              \)
              Since $|D_{1}|=d$, we have
              \(
                  |P(D_{1})| = 2^{d},
              \)
              giving the \emph{lower} bound
              \(
                  2^{d} \le |\operatorname{Im}F|.
              \)

        \item \textbf{Each image has size $d$.}\\
              For any $B\subseteq D_{1}$,
              \[
                  |F(B)|
                  =|B\cup D_{2}|
                  =|B|+|D_{2}|
                  =d,
              \]
              because $|B|\le d$, $|D_{2}|=d$, and $d+d=d$ for infinite
              cardinals.

        \item \textbf{Define the target family $Y$ and squeeze its size.}\\
              Let
              \[
                  Y=\{A\subseteq D : |A|=d\}.
              \]
              Step 4 shows
              \(
                  \operatorname{Im}F\subseteq Y,
              \)
              so
              \(
                  2^{d}\le |Y|.
              \)
              Trivially
              \(
                  Y\subseteq P(D),
              \)
              and $|P(D)|=2^{d}$, whence
              \(
                  |Y|\le 2^{d}.
              \)
              Combining the bounds forces
              \(
                  |Y|=2^{d}.
              \)

        \item \textbf{Conceptual takeaway.}\\
              The proof exploits the “self‑copying’’ property of infinite
              cardinals ($d+d=d$) to inject a full power set into the family
              of $d$‑element subsets, then traps that family between identical
              cardinal bounds.  In effect, \emph{“most’’} of the $2^{d}$
              subsets of an infinite set are already as large as the set
              itself.
              \qedhere
    \end{enumerate}
\end{problem}

%------------------------------------------------------------------------------
% Concrete illustrations of the statement
%   “An infinite set $D$ with $|D|=d$ has $2^{d}$ distinct subsets
%    whose own cardinality is $d$.”
%------------------------------------------------------------------------------
\begin{problem}
    \begin{align}
        \text{Give explicit examples for several familiar infinite sets $D$.}
    \end{align}

    \begin{enumerate}
    %------------------------------------------------------------------------
    \item \textbf{Countably infinite set: $\;D=\Bbb N$ (the natural numbers).}

          \begin{enumerate}
              \item Split $\Bbb N$ into two disjoint, countably infinite
                    blocks:
                    \[
                        D_{1}=2\Bbb N=\{0,2,4,\dots\},
                        \qquad
                        D_{2}=2\Bbb N+1=\{1,3,5,\dots\}.
                    \]
              \item Map every subset $B\subseteq D_{1}$ to
                    \[
                        F(B)=B\cup D_{2}.
                    \]
              \item Each image $F(B)$ is an \emph{infinite} subset of $\Bbb N$
                    (because it contains the whole block $D_{2}$), and
                    distinct $B$’s give distinct images.  Hence there are
                    \[
                        2^{\aleph_{0}}
                    \]
                    different countably–infinite subsets of $\Bbb N$.
          \end{enumerate}

    %------------------------------------------------------------------------
    \item \textbf{Uncountable set of size continuum: $\;D=\Bbb R$.}

          \begin{enumerate}
              \item Let
                    \(
                        D_{1}=(-\infty,0),\;
                        D_{2}=[0,\infty).
                    \)
                    Both have cardinality $\frak c=2^{\aleph_{0}}$.
              \item For every subset $B\subseteq D_{1}$ put
                    \(
                        F(B)=B\cup D_{2}.
                    \)
              \item Distinct $B$’s give distinct images, each of which still
                    has size $\frak c$.  Thus there are $2^{\frak c}$ many
                    subsets of $\Bbb R$ whose size is \emph{also} $\frak c$.
          \end{enumerate}

    %------------------------------------------------------------------------
    \item \textbf{Vector‑space flavour: $\;D=\Bbb Q$ (the rationals).}

          \begin{enumerate}
              \item Write $\Bbb Q=\Bbb Z\cup\bigl(\Bbb Q\setminus\Bbb Z\bigr)$.
                    Both pieces are countably infinite.
              \item For each subset $B\subseteq\Bbb Z$ define
                    \(
                        F(B)=B\cup\bigl(\Bbb Q\setminus\Bbb Z\bigr).
                    \)
              \item The map $B\mapsto F(B)$ is injective and every image is
                    countably infinite, giving again $2^{\aleph_{0}}$ many
                    countably–infinite subsets of $\Bbb Q$.
          \end{enumerate}

    %------------------------------------------------------------------------
    \item \textbf{General pattern (summary).}

          For \emph{any} infinite set $D$ we can choose disjoint
          $D_{1},D_{2}\subseteq D$ with
          $|D_{1}|=|D_{2}|=|D|=d$ and set
          \[
              F:P(D_{1})\to P(D),\qquad F(B)=B\cup D_{2}.
          \]
          This injects $P(D_{1})$ (size $2^{d}$) into
          \(
              Y=\{A\subseteq D:|A|=d\},
          \)
          and since $Y\subseteq P(D)$, we trap $|Y|$ between two identical
          bounds:
          \(
              2^{d}\le |Y|\le 2^{d}\;\Longrightarrow\;|Y|=2^{d}.
          \)
          \qedhere
    \end{enumerate}
\end{problem}

\begin{problem}
    \begin{align}
        \text{Let $C$ be a chain (totally ordered set) in which every \emph{countable}
        subset is well‑ordered.  Prove that $C$ itself is well‑ordered.}
    \end{align}

    \begin{enumerate}
        \item \textbf{Recall the definition.}\\
              A chain $C$ is \emph{well‑ordered} when every non‑empty
              subset of $C$ possesses a least element.

        \item \textbf{Argue by contradiction.}\\
              Assume, toward a contradiction, that $C$ is \emph{not}
              well‑ordered.

        \item \textbf{Extract a subset without a least element.}\\
              Under this assumption there exists a non‑empty subset
              $S\subseteq C$ that has no least element.

        \item \textbf{Build an infinite descending sequence inside $S$.}\\
              Choose $c_{0}\in S$.  
              Because $S$ lacks a least element, there is
              $c_{1}\in S$ with $c_{1}<c_{0}$.
              Repeating inductively, we obtain
              \[
                  c_{0}>c_{1}>c_{2}>\dotsb
              \]
              with every $c_{n}\in S$.

        \item \textbf{Form a countable subset that is \emph{not} well‑ordered.}\\
              Let
              \[
                  B=\{c_{n}:n\in\Bbb N\}.
              \]
              $B$ is countable, yet it contains the infinite
              descending sequence above, so $B$ has no least element and is
              therefore \emph{not} well‑ordered.

        \item \textbf{Contradiction.}\\
              The existence of such a set $B$ contradicts the hypothesis that
              \emph{every} countable subset of $C$ is well‑ordered.

        \item \textbf{Conclude.}\\
              Hence our assumption was false and $C$ must be well‑ordered.
              \qedhere
    \end{enumerate}
\end{problem}

\begin{example}
    Consider the chain $\bigl((0,1),<\bigr)$, the open interval of real numbers
    between $0$ and $1$ equipped with the usual order.

    \begin{enumerate}
        \item It is a \emph{chain} (total order) because for any two real
              numbers $x,y\in(0,1)$ we have exactly one of $x<y$, $x=y$, or
              $y<x$.

        \item It is \textbf{not well‑ordered}.  
              The subset itself $(0,1)$ has no least element:
              for every $x\in(0,1)$ the number $x/2$ is also in $(0,1)$ and
              satisfies $x/2<x$.  Hence no element can serve as a minimum.

        \item An explicit infinite descending sequence is
              \[
                  1>\tfrac12>\tfrac14>\tfrac18>\dotsb,
              \]
              showing again that the order is not well‑founded.
    \end{enumerate}
\end{example}

\begin{example}
    Take the set of natural numbers $\bigl(\Bbb N,<\bigr)$ with the usual order.

    \begin{enumerate}
        \item It is a \emph{chain} (total order) because for any $m,n\in\Bbb N$
              exactly one of $m<n$, $m=n$, or $n<m$ holds.

        \item It is \textbf{well‑ordered}.  
              Every non‑empty subset $S\subseteq\Bbb N$ has a least element,
              namely $\min S$; this follows from the standard well‑ordering
              principle for the natural numbers.

        \item Consequently there can be no infinite strictly descending
              sequence in $\Bbb N$, and the usual induction arguments apply
              without modification.
    \end{enumerate}
\end{example}

\begin{problem}
    \begin{align}
        \text{(a) Let $C$ be a \emph{well‑ordered} set and let }
        f:C\longrightarrow C
        \text{ be an \emph{injective}, order‑preserving map}\\
        (\forall a,b\in C)\,[\,a\le b \;\Longrightarrow\; f(a)\le f(b)\,].
        \quad\text{Show that } a\le f(a)\text{ for every }a\in C.\\[6pt]
        \text{(b) Deduce Theorem~20: a well‑ordered set is \emph{never}
        order‑isomorphic to one of its proper initial segments.}
    \end{align}

    \begin{enumerate}
    %---------------------------------------------------------------------
    \item \textbf{Part (a): prove $a\le f(a)$.}

          \begin{enumerate}
              \item \emph{Assume the contrary.}  
                    Suppose there exists $a_{0}\in C$ with
                    $a_{0}>f(a_{0})$.

              \item \emph{Iterate the map.}  
                    Define inductively
                    $a_{n+1}=f(a_{n})\;(n\in\Bbb N)$.  
                    Because $f$ is order‑preserving,
                    \[
                        a_{0}>a_{1}>a_{2}>\dotsb
                    \]
                    is a strictly descending sequence in $C$.

              \item \emph{Contradiction with well‑ordering.}  
                    A well‑ordered set cannot contain an infinite
                    descending chain.  
                    Hence our assumption was impossible, and we must have
                    \[
                        a\le f(a)\quad\text{for all }a\in C.\qedhere
                    \]
          \end{enumerate}

    %---------------------------------------------------------------------
    \item \textbf{Part (b): deduce Theorem~20.}

          Let $A$ be a well-ordered set and fix $a\in A$.
          Denote its \emph{initial segment}
          \(
              S(a)=\{x\in A : x<a\}\subsetneq A.
          \)
          Suppose, toward a contradiction, that $A$ is order-isomorphic
          to $S(a)$.  
          Then there exists an order-isomorphism
          \[
              f:A\longrightarrow S(a)\subseteq A.
          \]
          In particular $f$ is injective and order-preserving, so Part (a)
          applies:
          \[
              a\le f(a).
          \]
          But $f(a)\in S(a)$ implies $f(a)<a$, a contradiction.
          Therefore no such order‑isomorphism can exist, proving
          Theorem~20.
    \end{enumerate}
\end{problem}

\section*{Order-Isomorphism}

Let \((A, \le_A)\) and \((B, \le_B)\) be two ordered sets (i.e., sets equipped with a total order).

\medskip

\textbf{Definition:} An \emph{order-isomorphism} is a function 
\[
    f : A \longrightarrow B
\]
that satisfies the following properties:
\begin{enumerate}
    \item \textbf{Bijection:} \(f\) is a one-to-one and onto function.
    \item \textbf{Order Preservation:} For all \(x,y \in A\), 
          \[
             x \le_A y \quad\Longleftrightarrow\quad f(x) \le_B f(y).
          \]
\end{enumerate}
If such an \(f\) exists, we say that the ordered sets \(A\) and \(B\) are \emph{order-isomorphic} and write:
\[
    A \cong B.
\]

\medskip

\textbf{Why is this important?}\\[4pt]
Order-isomorphisms are significant because they show that two ordered sets have the same \emph{order type}; that is, they have identical structural properties concerning order. Many properties (e.g., being well-ordered) are preserved under order-isomorphism.

\bigskip

\textbf{Example:} The set of natural numbers \(\Bbb{N}\) with the usual order is order-isomorphic to the set of non-negative even numbers, \(2\Bbb{N}\), via the mapping:
\[
    f(n)=2n.
\]
One can easily verify that \(f\) is bijective and for any \(m, n\in \Bbb{N}\),
\[
    m \le n \quad\Longleftrightarrow\quad 2m \le 2n,
\]
hence \(f\) preserves the order.

\begin{problem}
    \begin{align}
        &\text{Let } A \text{ be a chain (totally ordered set).  Assume }  
          B,C\subseteq A \text{ with } A=B\cup C, \\[-2pt]
        &\text{and suppose that both } (B,\le) \text{ and } (C,\le)
          \text{ are well‑ordered in the order inherited from } A.\\[4pt]
        &\text{Show that } A \text{ itself is well‑ordered.}
    \end{align}

    \begin{enumerate}
        \item \textbf{Recall the goal.}\\
              A chain \(A\) is well‑ordered iff it contains \emph{no}
              infinite strictly descending sequence.  We will prove the
              contrapositive:  
              \[
                 \text{If } A \text{ is \emph{not} well‑ordered, then at
                 least one of } B,C \text{ is not well‑ordered.}
              \]

        \item \textbf{Assume \(A\) is \emph{not} well‑ordered.}\\
              Then there exists an infinite descending sequence
              \(
                  a_{1}>a_{2}>a_{3}>\dotsb
              \)
              in \(A\).  
              Let
              \[
                  X=\{a_{1},a_{2},a_{3},\dotsc\}\subseteq A.
              \]

        \item \textbf{Pigeonhole step.}\\
              Because \(A=B\cup C\), each \(a_{n}\) lies in either \(B\)
              or \(C\).  
              The set \(X\) is infinite, so at least one of the intersections
              \(X\cap B\) or \(X\cap C\) must be infinite.  
              Without loss of generality, assume
              \(
                  X\cap B
              \)
              is infinite.

        \item \textbf{Contradiction inside \(B\).}\\
              The elements of \(X\cap B\) inherit the same strict
              inequalities from the sequence \((a_{n})\), so they form an
              infinite strictly descending sequence \emph{inside} \(B\).  
              This contradicts the assumption that \(B\) is well‑ordered.

        \item \textbf{Conclude.}\\
              The assumption that \(A\) is not well‑ordered led to a
              contradiction.  
              Therefore \(A\) \emph{must} be well‑ordered.
              \qedhere
    \end{enumerate}
\end{problem}
\begin{problem}
    \begin{align}
        \text{Prove that } \aleph_{1}^{\aleph_{0}} \;=\; 2^{\aleph_{0}}
        \;=\; \mathfrak c,
    \end{align}
    where $\mathfrak c$ denotes the cardinality of the continuum.
\end{problem}

\begin{enumerate}
%-----------------------------------------------------------------------------
\item \textbf{Lower bound: $\displaystyle 2^{\aleph_{0}}\le
        \aleph_{1}^{\aleph_{0}}$.}

      Because $2\le\aleph_{1}$ and cardinal–exponentiation is monotone in
      the \emph{base} (for any fixed exponent $\lambda\neq0$),
      \[
          2^{\aleph_{0}}
          \;\le\;
          \aleph_{1}^{\aleph_{0}}.
      \]
      But $2^{\aleph_{0}}$ is, by definition, the continuum~$\mathfrak c$,
      so we already have
      \[
          \mathfrak c\;\le\;\aleph_{1}^{\aleph_{0}}.
      \]

%-----------------------------------------------------------------------------
\item \textbf{Upper bound: $\displaystyle \aleph_{1}^{\aleph_{0}}\le
        \mathfrak c$.}

      \begin{enumerate}
          \item \emph{First note:}  
                $\aleph_{1}\le \mathfrak c$.  
                Indeed, $\aleph_{1}$ is the least uncountable
                cardinal, while the set of real numbers is uncountable, so
                its cardinality $\mathfrak c$ must dominate~$\aleph_{1}$.

          \item \emph{Apply monotonicity again.}  
                With the same monotonicity property,
                \[
                    \aleph_{1}^{\aleph_{0}}
                    \;\le\;
                    \mathfrak c^{\aleph_{0}}.
                \]

          \item \emph{Compute $\mathfrak c^{\aleph_{0}}$.}  
                Since $\mathfrak c = 2^{\aleph_{0}}$,
                \[
                    \mathfrak c^{\aleph_{0}}
                    \;=\;
                    (2^{\aleph_{0}})^{\aleph_{0}}
                    \;=\;
                    2^{\aleph_{0}\cdot\aleph_{0}}
                    \;=\;
                    2^{\aleph_{0}}
                    \;=\;
                    \mathfrak c,
                \]
                because $\aleph_{0}\cdot\aleph_{0}=\aleph_{0}$.
                Hence
                \[
                    \aleph_{1}^{\aleph_{0}}
                    \;\le\;
                    \mathfrak c.
                \]
      \end{enumerate}

%-----------------------------------------------------------------------------
\item \textbf{Squeeze to equality.}

      Combining the two inequalities,
      \[
          \mathfrak c
          \;\le\;
          \aleph_{1}^{\aleph_{0}}
          \;\le\;
          \mathfrak c,
      \qquad\Longrightarrow\qquad
          \aleph_{1}^{\aleph_{0}}=\mathfrak c.
      \]
      Because $\mathfrak c=2^{\aleph_{0}}$, we obtain
      \[
          \boxed{\;
            \aleph_{1}^{\aleph_{0}} = 2^{\aleph_{0}}\;}.
      \]
      \qedhere
\end{enumerate}

\begin{problem}
    \textbf{Ordinal addition.}\;
    Let $\alpha$ and $\beta$ be ordinals.  
    Pick disjoint well–ordered sets $A,B$ with
    $\operatorname{ord}(A)=\alpha$ and $\operatorname{ord}(B)=\beta$,
    and give $A\cup B$ the \emph{lexicographic} order in which every
    element of $B$ is declared \emph{larger} than every element of $A$.
    Because $A\cup B$ is well‑ordered, its order type is an ordinal
    denoted $\alpha+\beta$.

    \begin{enumerate}
    %---------------------------------------------------------------------
    \item \textbf{Is $\alpha+\beta=\beta+\alpha$ always true?}

          \emph{No.}  
          Take
          \(
             \alpha=\omega=\operatorname{ord}(\Bbb N),\;
             \beta=1.
          \)
          Then
          \[
              \alpha+\beta
              =\omega+1,
              \qquad
              \beta+\alpha
              =1+\omega
              =\omega,
          \]
          and $\omega+1\neq\omega$.  Hence ordinal addition is
          \emph{not} commutative.

    %---------------------------------------------------------------------
    \item \textbf{If $\alpha+\beta=\alpha+\gamma$, prove $\beta=\gamma$.}

          \smallskip
          \emph{Setup.}\;
          Choose pairwise disjoint well‑ordered sets
          $A,B,C$ with
          \[
              \operatorname{ord}(A)=\alpha,\qquad
              \operatorname{ord}(B)=\beta,\qquad
              \operatorname{ord}(C)=\gamma,
          \]
          and order $A\cup B$ and $A\cup C$ as in the definition of
          ordinal addition.  
          Let
          \(
              f:A\cup B\to A\cup C
          \)
          be an order‑isomorphism (it exists because the two unions have
          the same order type $\alpha+\beta=\alpha+\gamma$).

          \smallskip
          \emph{Notation.}\;
          Let
          \[
              b=\min B,\qquad
              c=\min C .
          \]

          \smallskip
          \emph{Case analysis.}

          \begin{description}
              \item[Case 1:] $f(b)=a\in A$.  
                  Because $f$ preserves order,
                  \(
                      f(A)=S(a)=\{x\in A:x<a\},
                  \)
                  making $A$ order-isomorphic to a \emph{proper} initial
                  segment of itself—impossible for a well-ordered set.
              \item[Case 2:] $f^{-1}(c)=a\in A$.  
                  Symmetric to Case 1 and equally impossible.
              \item[Case 3:] $f(b)\in C$ and $f^{-1}(c)\in B$.  
                  Then $f(b)\ge c$ and $f^{-1}(c)\ge b$.
                  Applying $f$ again gives
                  \(
                      f(f^{-1}(c))\ge f(b),
                  \)
                  hence $c\ge f(b)$.  
                  Together with $f(b)\ge c$ we obtain $f(b)=c$.
                  Now
                  \(
                      f(B)\subseteq C,\;
                      f^{-1}(C)\subseteq B
                  \)
                  and $f$ is bijective, so actually $f(B)=C$ and
                  $f|_{B}:B\to C$ is an order-isomorphism.  
                  Therefore
                  \(
                      \beta=\operatorname{ord}(B)=\operatorname{ord}(C)=\gamma.
                  \)
          \end{description}

          All possibilities except Case 3 are impossible; Case 3 forces
          $\beta=\gamma$.  Hence
          \[
              \boxed{\;\alpha+\beta=\alpha+\gamma\;\Longrightarrow\;
              \beta=\gamma\;}.
          \]

    %---------------------------------------------------------------------
    \item \textbf{Show that $\beta+\alpha=\gamma+\alpha$ need \emph{not}
          imply $\beta=\gamma$.}

          Take
          \[
              \alpha=\omega,\qquad
              \beta=1,\qquad
              \gamma=2 .
          \]
          Then
          \[
              \beta+\alpha
              =1+\omega
              =\omega
              =2+\omega
              =\gamma+\alpha,
          \]
          yet $1\neq2$.  Thus the cancellation law in part (b) fails
          when the common term $\alpha$ is on the \emph{right}.
          \qedhere
    \end{enumerate}
\end{problem}
\section*{What does \texorpdfstring{$\operatorname{ord}(A)$}{ord(A)} mean?}

Let $(A,\le_A)$ be a \textbf{well‑ordered set}—that is, $\le_A$ is a total order and every non‑empty subset of $A$ has a least element.

\begin{definition}
The \emph{order type} (or \emph{ordinal}) of $A$ is the unique ordinal
\[
    \operatorname{ord}(A)
\]
such that $(A,\le_A)$ is \textbf{order‑isomorphic} to that ordinal when the ordinal is viewed with its natural well‑ordering.  Concretely, there exists a bijection
\[
    f:A \longrightarrow \operatorname{ord}(A)
\]
satisfying
\(
    x\le_A y \;\Longleftrightarrow\; f(x)\le f(y)
\)
for all $x,y\in A$.
\end{definition}

\subsection*{Key points}
\begin{itemize}
    \item Every well‑ordered set is \emph{classified} (up to isomorphism) by a \emph{unique} ordinal; this ordinal is what $\operatorname{ord}(A)$ denotes.
    \item If $A$ and $B$ are well‑ordered, then
          \[
              \operatorname{ord}(A)=\operatorname{ord}(B)
              \quad\Longleftrightarrow\quad
              A \text{ and } B \text{ are order‑isomorphic.}
          \]
    \item Familiar examples:
          \begin{align*}
              \operatorname{ord}(\mathbb N) &= \omega,\\
              \operatorname{ord}(\{0,1,\dots,n-1\}) &= n\quad (n\in\mathbb N),\\
              \operatorname{ord}(\omega\!+\!1) &= \omega+1 \text{ (itself an ordinal)}.
          \end{align*}
    \item In ordinal arithmetic we often write expressions like
          $\alpha+\beta$ or $\alpha\cdot\beta$, where $\alpha,\beta$ are
          ordinals; when we build those sums or products from concrete
          well‑ordered sets, we label the sets by their order types using
          $\operatorname{ord}(\,\cdot\,)$.
\end{itemize}

\bigskip
\textbf{Summary:} \;
$\boxed{\operatorname{ord}(A)=\text{ ``the ordinal that has the same well‑ordering as }A\text{''}}$
\subsubsection*{Clarifying Case 1}

Recall the set‑up:

\begin{itemize}
    \item $A,B,C$ are pairwise disjoint well‑ordered sets with
          \(\operatorname{ord}(A)=\alpha,\;
            \operatorname{ord}(B)=\beta,\;
            \operatorname{ord}(C)=\gamma.\)
    \item Both $A\cup B$ and $A\cup C$ are ordered so that every element
          of $B$ (respectively $C$) is \emph{greater} than every element of
          $A$.
    \item An order‑isomorphism
          \[
              f : A\cup B \longrightarrow A\cup C
          \]
          exists because
          \(\alpha+\beta=\alpha+\gamma.\)
    \item $b=\min B$ and $c=\min C$.
\end{itemize}

\paragraph{Case 1: \(f(b)=a\in A\).}

\begin{enumerate}
    \item Since \(f\) is an order-isomorphism, it preserves and reflects the order.  
          Because every element of $A$ is \emph{smaller} than $b$ in
          $A\cup B$, their images under $f$ must be \emph{smaller} than
          \(f(b)=a\) in $A\cup C$.  
          Hence
          \[
              f(A)\;\subseteq\;S(a)
              :=\{x\in A : x<a\},
          \]
          the initial segment of $A$ determined by $a$.

    \item Conversely, for any \(x<a\) (so \(x\in S(a)\)) we have
          \(x<f(b)\) in $A\cup C$.  
          Applying \(f^{-1}\), which is also order‑preserving,
          gives \(f^{-1}(x)<b\) in $A\cup B$.  
          But every element that is $<b$ in $A\cup B$ lies in $A$.
          Therefore \(f^{-1}(x)\in A\), and thus \(x=f\bigl(f^{-1}(x)\bigr)
          \in f(A)\).
          This shows
          \[
              S(a)\;\subseteq\;f(A).
          \]

    \item Combining the two inclusions,
          \[
              f(A)=S(a).
          \]
          So \(A\) is order-isomorphic to the \emph{proper} initial
          segment \(S(a)\subsetneq A\).

    \item \textbf{Why is that impossible?}  
          A fundamental property of well-ordered sets (proved earlier as
          Theorem 20) states that no well-ordered set can be order-isomorphic
          to one of its \emph{proper} initial segments.  
          Intuitively, if such an isomorphism existed, you could shift the
          whole order "downward" and create an infinite descending chain,
          contradicting well-foundedness.

    \item Hence our assumption \(f(b)=a\in A\) leads to a contradiction,
          so Case 1 cannot occur.
\end{enumerate}

\begin{problem}
    Let $f:A\twoheadrightarrow B$ be a surjective function between two sets.
    Use the Axiom of Choice to show that
    \[
        \operatorname{o}(B)\;\le\;\operatorname{o}(A),
    \]
    i.e.\ there exists an \emph{injective} map $g:B\hookrightarrow A$.
\end{problem}

\begin{enumerate}
%-----------------------------------------------------------------------------
\item \textbf{Form a family of non–empty sets.}

      For each $b\in B$ the fibre
      \[
          A_{b}\;:=\;f^{-1}(\{b\})
          =\{\,a\in A : f(a)=b\,\}
      \]
      is non–empty because $f$ is surjective.  
      Collect these fibres into a family
      \(
          \mathcal{F}=\{A_{b}\}_{b\in B}.
      \)

%-----------------------------------------------------------------------------
\item \textbf{Invoke the Axiom of Choice.}

      The Axiom of Choice asserts that for any family of non–empty sets
      there exists a \emph{choice function}.  
      Hence we may select a function
      \[
          s:\mathcal{F}\longrightarrow A
          \qquad\text{with}\qquad
          s(A_{b})\in A_{b}\quad(\forall\,b\in B).
      \]
      Equivalently, define
      \[
          g:B\longrightarrow A,
          \qquad
          g(b):=s\bigl(A_{b}\bigr)\in A_{b}.
      \]

%-----------------------------------------------------------------------------
\item \textbf{$g$ is injective.}

      Suppose $g(b_{1})=g(b_{2})$ for $b_{1},b_{2}\in B$.  
      Then
      \[
          f\!\bigl(g(b_{1})\bigr)
          =b_{1},
          \qquad
          f\!\bigl(g(b_{2})\bigr)
          =b_{2},
      \]
      but the left‑hand sides are equal, so $b_{1}=b_{2}$.  
      Therefore $g$ is one‑to‑one.

%-----------------------------------------------------------------------------
\item \textbf{Conclude the cardinal inequality.}

      An injection $g:B\hookrightarrow A$ witnesses that
      \(
          |B|\le |A|.
      \)
      In the notation of the problem,
      \[
          \boxed{\;
            \operatorname{o}(B)\;\le\;\operatorname{o}(A)
          \;} .
      \]
      \qedhere
\end{enumerate}
\paragraph{3.\; $g$ is injective.}

Assume $g(b_{1})=g(b_{2})$ for some $b_{1},b_{2}\in B$.  
Write $a:=g(b_{1})=g(b_{2})\in A$.  
By construction of $g$ we know
\[
    a \in A_{b_{1}}
    \quad\text{and}\quad
    a \in A_{b_{2}},
\]
where $A_{b}=f^{-1}(\{b\})$ is the fibre over $b$.  
But the fibres are pairwise disjoint, so an element of $A$ can belong to
\emph{only one} of them.  Hence
\[
    b_{1}=b_{2}.
\]
Therefore distinct elements of $B$ have distinct images under $g$; that
is, $g$ is one‑to‑one (injective).

\begin{problem}
    Let $D$ be an infinite set with cardinality $|D|=d$.  
    Let
    \[
        F=\{A\subseteq D : A\text{ is finite}\}.
    \]
    Show that $|F|=d$.
\end{problem}

\begin{enumerate}
%---------------------------------------------------------------------------
\item \textbf{Build a surjection onto $F$.}

      For each $n\ge 1$ let $D^{n}=D\times\dotsm\times D$ ($n$ factors).  
      Define
      \[
          f:\;\bigsqcup_{n=1}^{\infty} D^{\,n}\longrightarrow F,
          \qquad
          f\bigl((a_{1},\dots,a_{n})\bigr)=\{a_{1},\dots,a_{n}\}.
      \]
      Every finite subset of $D$ can be listed as an $n$‑tuple for some
      $n$, so $f$ is \emph{surjective}.

%---------------------------------------------------------------------------
\item \textbf{Cardinality of the domain.}

      Because $d$ is infinite,
      \[
          |D^{\,n}|=d^{\,n}=d\qquad(n\ge 1).
      \]
      The domain of $f$ is a \emph{countable} disjoint union of sets of
      size $d$, hence
      \[
          \Bigl|\bigsqcup_{n=1}^{\infty} D^{\,n}\Bigr|
          \;=\;
          \aleph_{0}\cdot d
          \;=\;
          d.
      \]
      (Multiplying an infinite cardinal by $\aleph_{0}$ leaves it
      unchanged.)

%---------------------------------------------------------------------------
\item \textbf{Upper bound for $|F|$.}

      Since $f$ is onto,
      \[
          |F|\;\le\;\Bigl|\bigsqcup_{n=1}^{\infty} D^{\,n}\Bigr|
          \;=\;
          d.
      \]

%---------------------------------------------------------------------------
\item \textbf{Lower bound for $|F|$.}

      Define
      \[
          g:D\longrightarrow F,
          \qquad
          g(a)=\{a\}.
      \]
      The map $g$ is injective, so $|D|\le |F|$, i.e.\ $d\le |F|$.

%---------------------------------------------------------------------------
\item \textbf{Conclude.}

      Combining the bounds
      \[
          d\;\le\;|F|\;\le\;d
          \quad\Longrightarrow\quad
          |F|=d.
          \qedhere
      \]
\end{enumerate}
\section*{The Disjoint Union Symbol}

The symbol 
\[
   \bigsqcup
\]
is called the \emph{disjoint union} (or \emph{coproduct}) symbol. It typically means that one is taking the union of several sets in such a way that each set is understood (or made) to be pairwise disjoint from the others, even if they originally were not.

\subsection*{Example of usage}
\[
  \bigsqcup_{n=1}^{\infty} D^{n}
\]
means ``the disjoint union of the sets $D^{n}$ (the Cartesian product of $D$ with itself $n$ times), for $n=1,2,3,\dots$''. Each $D^{n}$ is treated as if it had no overlap with $D^{m}$ for $m\neq n$, by using distinct tagging or indexing for each component.

\subsection*{Why a Disjoint Union?}
In many arguments, especially when constructing a surjective map onto a family of finite subsets, one wants to ensure that elements coming from $D^n$ are kept distinct from those in $D^m$ when $m\ne n$. The disjoint union $\bigsqcup$ accomplishes this formally.

\subsection*{LaTeX Command}
To produce the symbol in LaTeX, type:
\[
  \bigsqcup
\]
There is also a smaller inline version: 
\[
  \sqcup.
\]
\section*{Understanding the Notation \texorpdfstring{$D^n$}{D^n}}

In our context, if \(D\) is an infinite set and \(n\) is a positive integer, then 
\[
    D^n
\]
denotes the \emph{n‑fold Cartesian product} of \(D\) with itself.  
That is,
\[
    D^n = \underbrace{D \times D \times \cdots \times D}_{n\text{ factors}},
\]
which is the set of all \(n\)-tuples
\[
    (a_{1}, a_{2}, \dots, a_{n})
\]
with each \(a_{i}\in D\).

\bigskip

\textbf{Important Points:}
\begin{itemize}
    \item Although \(D\) is infinite, \(D^n\) involves only a \emph{finite} number of factors.
    \item Consequently, if \(D\) is an infinite set, one typically has 
          \[
              |D^n| = |D|,
          \]
          because the cardinality does not change when taking a finite product of an infinite set.
    \item In contrast, an \emph{infinite Cartesian product} would be written as
          \[
              \prod_{i\in I} D_i,
          \]
          where \(I\) is an infinite index set (and, usually, each \(D_i\) is a copy of \(D\)).
\end{itemize}

\bigskip
\textbf{Summary:}  
\(D^n\) represents the finite Cartesian product of \(D\) with itself \(n\) times, not the infinite Cartesian product.

\section*{Why the Infinity Symbol?}

When you see
\[
   \bigsqcup_{n=1}^{\infty} D^n,
\]
the ``\(\infty\)'' refers to the fact that we are \emph{indexing} over
\emph{all positive integers} \(n\).  Specifically, we are taking the
\emph{disjoint union} of
\[
   D^1, \quad D^2, \quad D^3, \quad \dots
\]
\emph{finitely many factors at a time}, but for \emph{every} natural number \(n\).

\bigskip

\textbf{Key Points:}
\begin{itemize}
\item Each individual set \(D^n\) is a \emph{finite Cartesian product} of \(D\) with itself \(n\) times.
\item Because \(n\) runs from \(1\) to \(\infty\), we have \emph{infinitely many} such finite products.
\item We then form the \emph{disjoint union} of all these sets.  That is not the same as an infinite product 
      \(\prod_{i=1}^\infty D\). 
      Instead, we have a countable family of finite products, all combined with no overlap (except for having the same underlying set \(D\)).
\item The ``\(\infty\)'' is \emph{not} telling you to take an ``infinite Cartesian product''; 
      it’s simply saying ``do this for all \(n=1,2,3,\dots\).”
\end{itemize}

\bigskip
\textbf{Summary:}
\[
   \boxed{
     \text{The symbol ``\(\infty\)'' indicates a countably infinite index over which we take a disjoint union 
           of finite products }D^n.
   }
\]
\begin{problem}
    Prove that every chain \(C\) in a partially ordered set \((L,\le)\)
    is contained in a \emph{maximal} chain.
\end{problem}

\begin{proof}
    \textbf{Step 1:  Build a poset of chains that extend \(C\).}

    Let
    \[
        \mathcal L
        \;=\;
        \bigl\{\,A\subseteq L \;:\; C\subseteq A
              \text{ and } A \text{ is a chain in }(L,\le)\bigr\}.
    \]
    Order \(\mathcal L\) by set‑inclusion:
    \[
        A\preceq B \;\Longleftrightarrow\; A\subseteq B.
    \]
    Clearly \(\preceq\) is a partial order on \(\mathcal L\).

    \bigskip
    \textbf{Step 2:  Show that every chain of chains has an upper bound.}

    Let \(\mathcal C=\{A_{\alpha}\}_{\alpha\in I}\subseteq\mathcal L\)
    be a \(\preceq\)-chain (that is, the \(A_{\alpha}\) themselves are
    totally ordered by inclusion).
    Define
    \[
        C_{0}\;=\;\bigcup_{\alpha\in I} A_{\alpha}.
    \]
    We claim \(C_{0}\in\mathcal L\).

    \begin{itemize}
        \item \(C\subseteq C_{0}\) because each \(A_{\alpha}\) contains \(C\).
        \item To see \(C_{0}\) is a chain, take any \(a,b\in C_{0}\).
              Then \(a\in A_{\alpha}\) and \(b\in A_{\beta}\) for some
              indices \(\alpha,\beta\).
              Because \(\mathcal C\) is a chain under inclusion, either
              \(A_{\alpha}\subseteq A_{\beta}\) or
              \(A_{\beta}\subseteq A_{\alpha}\).
              Without loss of generality, assume
              \(A_{\alpha}\subseteq A_{\beta}\); then
              \(a,b\in A_{\beta}\), and since \(A_{\beta}\) is a chain in
              \(L\) we have \(a\le b\) or \(b\le a\).
              Hence \(C_{0}\) is a chain.
    \end{itemize}
    Thus \(C_{0}\in\mathcal L\) and \(A_{\alpha}\subseteq C_{0}\)
    for every \(\alpha\in I\); i.e.\ \(C_{0}\) is an upper bound of
    \(\mathcal C\) in \((\mathcal L,\preceq)\).

    \bigskip
    \textbf{Step 3:  Apply Zorn's Lemma.}

    Every \(\preceq\)-chain in \(\mathcal L\) has an upper bound, so by
    Zorn’s Lemma there exists a \(\preceq\)-maximal element
    \(C^{\ast}\in\mathcal L\).
    \begin{lemma}[Zorn's Lemma]
        Let $P$ be a partially ordered set (poset) in which every totally ordered subset 
        (i.e., every chain) has an upper bound in $P$. Then $P$ contains at least 
        one maximal element.
        \end{lemma}

    \bigskip
    \textbf{Step 4:  Verify maximality in \(L\).}

    By construction \(C^{\ast}\) is a chain containing \(C\).
    Suppose there were an element \(x\in L\setminus C^{\ast}\) such that
    \(C^{\ast}\cup\{x\}\) is still a chain.
    Then \(C^{\ast}\cup\{x\}\in\mathcal L\) and strictly contains
    \(C^{\ast}\), contradicting maximality in \(\mathcal L\).
    Therefore no such \(x\) exists, and \(C^{\ast}\) is a
    \emph{maximal} chain of \(L\) that contains \(C\).

    \bigskip
    \textbf{Conclusion.}\;
    Every chain in a partially ordered set can be extended to a maximal
    chain.
\end{proof}
\section*{Extending a Function Using the Axiom of Choice}

Let \(f:X\to X\) be a function. We wish to prove, using the Axiom of Choice, that there exists a function \(g:X\to X\) such that
\[
  f \circ g \circ f \;=\; f.
\]

\textbf{Sketch of the Proof:}
\begin{enumerate}
    \item Factor \(f\) through its image:
          Define \(Y=f(X)\) so that \(f\) may be written as 
          \[
              f:X \xrightarrow{\;f\;} Y \hookrightarrow X.
          \]
    \item Use the Axiom of Choice:
          For each \(y\in Y\) the fibre 
          \[
              f^{-1}(\{y\})
          \]
          is non‑empty. The Axiom of Choice guarantees the existence of a choice function 
          \[
              s:Y\to X,\quad s(y)\in f^{-1}(\{y\}) \quad (\forall\, y\in Y),
          \]
          so that
          \[
              f\circ s = \operatorname{id}_Y.
          \]
    \item \textbf{Extend \(s\) to a function \(g:X\to X\).}  
          We define \(g\) by considering two cases. It is convenient to express this using a \texttt{cases} environment. To avoid the "Missing \$ inserted" error, we wrap the whole construction in math mode:
          \[
          g(x)=
          \begin{cases}
             s(x), & \text{if } x\in Y, \\[3mm]
             x_0, & \text{if } x\in X\setminus Y,
          \end{cases}
          \]
          where \(x_{0}\in X\setminus Y\) is a fixed element chosen arbitrarily (if \(Y\neq X\); if \(Y=X\), then simply take \(g(x)=s(x)\) for all \(x\in X\)).
    \item \textbf{Check the identity:}  
          For any \(x\in X\), note that \(f(x)\in Y\). Hence,
          \[
              (f\circ g\circ f)(x) = f\Bigl(g\bigl(f(x)\bigr)\Bigr)
              = f\bigl(s(f(x))\bigr)
              = (f\circ s)(f(x))
              = \operatorname{id}_Y\bigl(f(x)\bigr)
              = f(x).
          \]
\end{enumerate}

Thus, we have constructed a function \(g:X\to X\) with \(f\circ g\circ f = f\).

\bigskip

\begin{problem}
    Let $M$ be a non‑empty set and let $D:M\times M\to\mathbb R$ satisfy
    \begin{enumerate}
    \item[(a)] $D(a,a)=0$ for every $a\in M$;
    \item[(b)] $D(a,b)\ne0$ whenever $a\ne b$;
    \item[(c)] $D(a,b)+D(b,c)\ge D(a,c)$ for all $a,b,c\in M$.
    \end{enumerate}
    Prove that $D$ is a metric on $M$.
    \end{problem}
    
    \begin{proof}
    To show that $D$ is a metric we must verify, for all $a,b,c\in M$,
    \[
    \text{(i) } D(a,b)\ge0,\quad
    \text{(ii) } D(a,b)=0\iff a=b,\quad
    \text{(iii) } D(a,b)=D(b,a),\quad
    \text{(iv) } D(a,c)\le D(a,b)+D(b,c).
    \]
    
    \smallskip
    \noindent\textbf{Triangle inequality (iv).}  
    This is exactly assumption~(c).
    
    \smallskip
    \noindent\textbf{Symmetry (iii).}  
    Set $c=b$ in~(c):
    \[
      D(a,b)+D(b,b)\;\ge\;D(a,b).
    \]
    Because $D(b,b)=0$ by~(a), we actually get
    \[
      D(a,b)\;\ge\;D(b,a).
    \]
    Exchanging $a$ and $b$ gives the reverse inequality, hence
    $D(a,b)=D(b,a)$.
    
    \smallskip
    \noindent\textbf{Non‑negativity (i).}  
    Put $c=a$ in~(c):
    \[
      D(a,b)+D(b,a)\;\ge\;D(a,a)=0.
    \]
    The left side is a sum of two equal terms (by symmetry), so
    $2D(a,b)\ge0$, i.e.\ $D(a,b)\ge0$.
    
    \smallskip
    \noindent\textbf{Identity of indiscernibles (ii).}  
    If $a\ne b$, assumption~(b) says $D(a,b)\ne0$, and by the
    non‑negativity just proved this means $D(a,b)>0$.  
    Conversely, $D(a,a)=0$ by~(a).  
    Hence $D(a,b)=0\iff a=b$.
    
    \smallskip
    \noindent
    Having verified (i)–(iv), we conclude that $D$ satisfies all metric
    axioms; therefore $D$ is a metric on $M$.
    \end{proof}
    \begin{theorem}
        Let $M = \mathbb{R}$ with the distance function $D(a,b) = |a - b|$ for all $a,b \in \mathbb{R}$. 
        Fix a point $x_0 \in \mathbb{R}$.  Show that there exist exactly two isometries $f : \mathbb{R}\to \mathbb{R}$ 
        such that $f(x_0) = x_0$, namely
        \[
        f(x) = x \quad \text{and} \quad f(x) = 2x_0 - x.
        \]
        \end{theorem}
        
        \begin{proof}[Proof (Step by Step)]
        \textbf{Step 1: Reduction to the case $x_0=0$.}
        
        Since we are looking for isometries that fix $x_0$, we can reduce to the case $x_0=0$ by a simple translation.  
        Define 
        \[
        g(x) = x - x_0 
        \quad \text{and hence} \quad 
        g^{-1}(x) = x + x_0.
        \]
        Both $g$ and $g^{-1}$ are isometries (they just translate by $x_0$).  
        If $f : \mathbb{R} \to \mathbb{R}$ is an isometry that fixes $x_0$, then the map 
        \[
        h = g \circ f \circ g^{-1}
        \]
        is an isometry that fixes $0$, because 
        \[
        f(x_0) = x_0
        \;\;\Longleftrightarrow\;\;
        h(0) = g(f(g^{-1}(0))) = g(f(x_0)) = g(x_0) = 0.
        \]
        Moreover, $f_1 = f_2$ if and only if $h_1 = h_2$ for the corresponding $h_i$.  
        Thus it suffices to classify all isometries $h$ fixing $0$.  Translating back will give the classification of the original $f$ fixing $x_0$.
        
        \medskip
        
        \textbf{Step 2: Classifying isometries $f : \mathbb{R} \to \mathbb{R}$ that fix $0$.}
        
        Let $f : \mathbb{R} \to \mathbb{R}$ be an isometry with $f(0)=0$.  
        First, observe that
        \[
        D\bigl(f(1), f(0)\bigr) = D(f(1),0) = D(1,0) = 1.
        \]
        Hence $|f(1)| = 1$, so $f(1)$ must be either $1$ or $-1$.
        
        \medskip
        
        \textbf{Case 1: $f(1) = 1$.}
        
        Pick any $x \in \mathbb{R}$.  Set $y = f(x)$.  Since $f$ preserves distances from $0$ and from $1$, we have
        \[
        |y| \;=\; D(f(x), 0) \;=\; D(x,0) \;=\; |x|,
        \]
        and
        \[
        |y - 1| \;=\; D(f(x), f(1)) \;=\; D(x,1) \;=\; |x - 1|.
        \]
        From $|y| = |x|$, we know $y$ is either $x$ or $-x$.  Then from 
        \[
        |y - 1| = |x - 1|,
        \]
        we see that if $y = -x$, then 
        \[
        |-x - 1| = |x - 1|.
        \]
        Squaring, $(x+1)^2 = (x-1)^2 \implies x^2 + 2x + 1 = x^2 - 2x + 1 \implies 2x = -2x \implies x=0.$  
        But for $x=0$, we already have $y=0$.  In all other cases $x \neq 0$, the only possibility consistent with the above distances is $y=x$.  Hence $f(x)=x$ for all $x$.  
        
        Thus, if $f(1)=1$, we conclude $f(x) = x$ for all $x$.  
        
        \medskip
        
        \textbf{Case 2: $f(1) = -1$.}
        
        Define 
        \[
        \varphi(x) = -\,f(x).
        \]
        Clearly, $\varphi(0) = -f(0)=0$.  Also 
        \[
        \varphi(1) = -f(1)= -(-1)=1.
        \]
        Thus $\varphi$ is another isometry (it is the composition of $f$ with a reflection $x \mapsto -x$) that fixes $0$ and sends $1$ to $1$.  By the argument in Case~1, such an isometry must be the identity map.  Hence $\varphi(x)=x$ for all $x$, which implies 
        \[
        -\,f(x) = x 
        \quad\Longrightarrow\quad 
        f(x) = -\,x.
        \]
        Therefore, if $f(1)=-1$, we conclude $f(x) = -x$ for all $x$.  
        
        \medskip
        
        \textbf{Step 3: Conclusion.}
        
        From the two cases, any isometry fixing $0$ must be either $f(x)=x$ or $f(x)=-x$.  
        Recalling the translation argument in Step~1, if we want an isometry $F$ that fixes a general $x_0$, it must come from 
        \[
        f(x) = \pm x
        \]
        by 
        \[
        F(x) = g^{-1}\bigl(f(g(x))\bigr) = x_0 \pm (x - x_0).
        \]
        Hence the only two possibilities are
        \[
        F(x) = x 
        \quad\text{or}\quad 
        F(x) = 2x_0 - x.
        \]
        These are exactly two isometries that fix $x_0$.  This completes the proof.
        \end{proof}
        \begin{theorem}
            Let $(M,D)$ be a metric space, let $x$ be a point in $M$, and let $r,s$ be real numbers with
            \[
            0 < r < s.
            \]
            Define the set 
            \[
            A(x,r,s) \;:=\; \bigl\{\,y \in M : r < D(x,y) < s\,\bigr\}.
            \]
            Show that $A(x,r,s)$ is open in $M$.
            \end{theorem}
            
            \begin{proof}[Proof (Step by Step)]
            \textbf{Step 1: Fix a point $y$ in $A(x,r,s)$.}
            
            By the definition of $A(x,r,s)$, we have
            \[
            r < D(x,y) < s.
            \]
            Therefore, both of the quantities
            \[
            D(x,y) - r
            \quad \text{and} \quad
            s - D(x,y)
            \]
            are strictly positive.
            
            \medskip
            
            \textbf{Step 2: Define a suitable radius for an open ball around $y$.}
            
            Let
            \[
            \delta \;=\; \min\bigl(D(x,y) - r, \; s - D(x,y)\bigr).
            \]
            Since both $D(x,y) - r > 0$ and $s - D(x,y) > 0$, it follows that $\delta > 0.$
            
            \medskip
            
            \textbf{Step 3: Check that all points $z$ within $B_\delta(y)$ remain in $A(x,r,s)$.}
            
            Consider any point $z \in M$ such that $D(y,z) < \delta.$  
            We want to show that 
            \[
            r < D(x,z) < s.
            \]
            Using the triangle inequality, we have
            \[
            D(x,z) \;\le\; D(x,y) + D(y,z)
            \;\;<\;\; D(x,y) + \delta.
            \]
            Since $\delta \le s - D(x,y)$, we get
            \[
            D(x,y) + \delta \;\le\; D(x,y) + \bigl(s - D(x,y)\bigr) \;=\; s.
            \]
            Hence,
            \[
            D(x,z) < s.
            \]
            On the other hand, using the reverse triangle inequality or the fact that $D(x,z) \ge |D(x,y) - D(y,z)|$, we get
            \[
            D(x,z) \;\ge\; D(x,y) - D(y,z)
            \;\;>\;\; D(x,y) - \delta.
            \]
            Since $\delta \le D(x,y) - r$, it follows
            \[
            D(x,y) - \delta \;\ge\; D(x,y) - \bigl(D(x,y) - r\bigr) \;=\; r.
            \]
            Thus,
            \[
            D(x,z) > r.
            \]
            
            Putting these together gives
            \[
            r < D(x,z) < s.
            \]
            Therefore, $z \in A(x,r,s)$.
            
            \medskip
            
            \textbf{Step 4: Conclude that $A(x,r,s)$ is open.}
            
            Because every point $y \in A(x,r,s)$ has some $\delta>0$ such that the open ball $B_\delta(y)$ is contained in $A(x,r,s)$, it follows by definition that $A(x,r,s)$ is an open set in $M$.
            
            \end{proof}
            \begin{theorem}
                Let $(M,D)$ be an infinite metric space. Show that there exists an open set $U \subset M$ such that both $U$ and $M \setminus U$ are infinite.
                \end{theorem}
                
                \begin{proof}[Proof (Step by Step)]
                \textbf{Step 1. Define $E$, the set of isolated points of $M$.}
                
                Recall that a point $x \in M$ is \emph{isolated} if there exists some $r>0$ such that $B_r(x) = \{x\}$.  
                Let $E$ be the set of all isolated points in $M$, i.e.
                \[
                E = \{\, x \in M : x \text{ is isolated}\}.
                \]
                
                \medskip
                
                \textbf{Step 2. Case 1: $E$ is infinite.}
                
                If $E$ is infinite, we can choose a subset $U \subset E$ such that $U$ is infinite and $E \setminus U$ is also infinite.  
                Because every point of $E$ is isolated, each $x \in E$ has some open ball $B_{r_x}(x)$ containing no other points of $M$.  In particular, each $x \in E$ is an interior point of $E$.  
                Hence if $U \subset E$, each $x \in U$ is also an interior point of $U$, making $U$ open in $M$.  
                Since $U$ and $E \setminus U$ are both infinite, we also see that $M\setminus U$ (which contains at least $E \setminus U$) must be infinite.  
                Thus in this case, $U$ is an open set in $M$ with $U$ infinite and $M\setminus U$ infinite, as desired.
                
                \medskip
                
                \textbf{Step 3. Case 2: $E$ is finite.}
                
                If $E$ is finite, there must exist a point $x \in M$ that is \emph{not} isolated (otherwise, $E$ would be all of $M$, contradicting the infinitude of $M$ when $E$ is finite).
                
                \medskip
                
                \textbf{Step 4. Construct an open ball with infinitely many points.}
                
                Since $x$ is not isolated, for every $r>0$, the ball $B_r(x)$ contains infinitely many points of $M$.  
                In particular, pick any $r>0$.  The set $B_r(x)$ is infinite.  
                We claim there is another point $y \in M$ with $D(x,y) = 2r$ and $y \notin B_r(x)$.  
                Because $M$ is infinite, we can choose $y$ so that $D(x,y) > r$; by scaling $r$ if needed, we ensure $D(x,y) = 2r$.  
                
                \medskip
                
                \textbf{Step 5. Two disjoint infinite open balls.}
                
                Observe that $B_r(x)$ and $B_r(y)$ are disjoint: if $z$ were in both, then by the triangle inequality,
                \[
                D(x,y) \; \le \; D(x,z) + D(z,y) < r + r = 2r,
                \]
                which contradicts $D(x,y) = 2r$.  
                Hence each of $B_r(x)$ and $B_r(y)$ is infinite and they are disjoint subsets of $M$.
                
                \medskip
                
                \textbf{Step 6. Conclude with the required open set $U = B_r(x)$.}
                
                Take $U := B_r(x)$.  Then $U$ is open in $M$, and since $U$ is infinite, it remains only to check that $M \setminus U$ is also infinite.  But $B_r(y)\subset M \setminus U$, and $B_r(y)$ is infinite.  
                Hence $M \setminus U$ is infinite as well.
                
                \medskip
                
                \textbf{Step 7. Final conclusion.}
                
                In both cases ($E$ infinite or $E$ finite), we have produced an open set $U$ such that both $U$ and $M \setminus U$ are infinite.  This completes the proof.
                \end{proof}
                \begin{theorem}
                    Let $A$ and $B$ be subsets of a metric space $(M,D)$.  Then
                    \begin{itemize}
                    \item[(a)] If $A \subset B$, then $\overline{A} \subset \overline{B}$.
                    \item[(b)] $\overline{A \cup B} = \overline{A} \;\cup\; \overline{B}$.
                    \item[(c)] Does $\overline{A \cap B} \;=\; \overline{A}\,\cap\,\overline{B}$ always hold?
                    \end{itemize}
                    \end{theorem}
                    
                    \begin{proof}[Proof (Step by Step)]
                    \textbf{Part (a).} Assume $A \subset B$.  
                    
                    \begin{itemize}
                    \item Since $B \subset \overline{B}$ (the closure of $B$), it follows from $A \subset B$ that $A \subset \overline{B}$.  
                    \item The closure $\overline{A}$ is the \emph{smallest} closed set containing $A$.  Because $\overline{B}$ is itself a closed set containing $A$, we conclude $\overline{A} \subset \overline{B}$.  
                    \end{itemize}
                    
                    Hence $A \subset B \implies \overline{A} \subset \overline{B}$.
                    
                    \medskip
                    
                    \textbf{Part (b).} We claim that
                    \[
                    \overline{A \cup B} \;=\; \overline{A}\,\cup\,\overline{B}.
                    \]
                    
                    \begin{itemize}
                    \item \emph{First, we show} $\overline{A \cup B} \;\subset\; \overline{A}\,\cup\,\overline{B}$.  
                    
                    Since $A \cup B \subset \overline{A}\,\cup\,\overline{B}$ and $\overline{A}\,\cup\,\overline{B}$ is closed (union of two closed sets is closed), it follows that the closure of $A \cup B$ (the smallest closed set containing $A \cup B$) must be contained in any closed set that contains $A \cup B$.  In particular,
                    \[
                    \overline{A \cup B} \;\subset\; \overline{A}\,\cup\,\overline{B}.
                    \]
                    
                    \item \emph{Next, we show} $\overline{A}\,\cup\,\overline{B} \;\subset\; \overline{A \cup B}$.  
                    
                    Let $F$ be any closed set containing $A \cup B$.  Then clearly $F \supset A$ and $F \supset B$.  
                    Since $F$ is closed and contains $A$, we have $F \supset \overline{A}$.  Similarly, $F \supset \overline{B}$.  
                    Hence $F \supset \overline{A}\,\cup\,\overline{B}$.  
                    
                    Because $\overline{A \cup B}$ is defined as the \emph{intersection} of all closed sets containing $A \cup B$, it must be that
                    \[
                    \overline{A \cup B}
                    \;\supset\;
                    \overline{A}\,\cup\,\overline{B}.
                    \]
                    \end{itemize}
                    
                    Combining both inclusions gives
                    \[
                    \overline{A \cup B} \;=\; \overline{A}\,\cup\,\overline{B}.
                    \]
                    
                    \medskip
                    
                    \textbf{Part (c).} Does
                    \[
                    \overline{A \cap B} \;=\; \overline{A}\,\cap\,\overline{B}
                    \]
                    always hold? The answer is \emph{no} in general.
                    
                    \medskip
                    
                    \textbf{Counterexample:}
                    
                    Take $M = \mathbb{R}$ with the usual metric.  Let 
                    \[
                    A = \mathbb{Q} \quad\text{(the set of rationals)}, 
                    \qquad
                    B = \mathbb{R} \setminus \mathbb{Q} \quad\text{(the set of irrationals)}.
                    \]
                    Then 
                    \[
                    A \cap B = \varnothing,
                    \]
                    so 
                    \[
                    \overline{A \cap B} = \overline{\varnothing} = \varnothing.
                    \]
                    On the other hand,
                    \[
                    \overline{A} = \overline{\mathbb{Q}} = \mathbb{R}
                    \quad\text{and}\quad
                    \overline{B} = \overline{\mathbb{R} \setminus \mathbb{Q}} = \mathbb{R}.
                    \]
                    Hence 
                    \[
                    \overline{A} \;\cap\; \overline{B}
                    = \mathbb{R} \cap \mathbb{R}
                    = \mathbb{R}.
                    \]
                    Clearly, $\varnothing \neq \mathbb{R}$.  
                    Thus $\overline{A \cap B} \neq \overline{A}\,\cap\,\overline{B}$ in this example.
                    
                    \end{proof}
                    \begin{theorem}
                        Let $(M,D)$ be a metric space, $x \in M$, and $r,s$ real numbers with $0 < r < s$. Then:
                        \begin{itemize}
                        \item[(a)] The closed ball $\overline{B}_r(x) = \{\,y \in M : D(y,x) \le r\}$ is closed in $M$.
                        \item[(b)] The sphere $\{\,y \in M : D(y,x) = r\}$ is closed in $M$.
                        \item[(c)] The closed annulus
                        \[
                        A(x,r,s) := \{\,y \in M : r \,\le\, D(y,x) \,\le\, s\}
                        \]
                        is closed in $M$.
                        \end{itemize}
                        \end{theorem}
                        
                        \begin{proof}[Proof (Step by Step)]
                        
                        \textbf{(a) The closed ball $\overline{B}_r(x)$ is closed.}
                        
                        Define
                        \[
                        \overline{B}_r(x) \;=\; \{\, y \in M : D(y,x) \,\le\, r\}.
                        \]
                        We show that its complement $M \setminus \overline{B}_r(x)$ is open.  
                        
                        \begin{itemize}
                        \item Take any $y \in M \setminus \overline{B}_r(x)$.  
                           By definition, $D(y,x) > r$.  
                           Set $\delta := D(y,x) - r$, which is strictly positive.  
                        \item Then for any $z \in M$ with $D(y,z) < \delta$, the triangle inequality yields
                        \[
                        D(z,x) \;\ge\; D(y,x) - D(y,z) \;>\; D(y,x) - \delta \;=\; r.
                        \]
                        Hence $z \notin \overline{B}_r(x)$, i.e.\ $z \in M \setminus \overline{B}_r(x)$.  
                        \item This shows that the open ball $B_\delta(y)$ around $y$ lies entirely in $M \setminus \overline{B}_r(x)$.  
                           Consequently, $M \setminus \overline{B}_r(x)$ is open.  
                        \end{itemize}
                        
                        Since the complement of $\overline{B}_r(x)$ is open, we conclude that $\overline{B}_r(x)$ is closed.
                        
                        \medskip
                        
                        \textbf{(b) The sphere $\{y : D(y,x) = r\}$ is closed.}
                        
                        Let 
                        \[
                        S_r(x) := \{\,y \in M : D(y,x) = r\}.
                        \]
                        We can write this sphere as 
                        \[
                        S_r(x) \;=\; \overline{B}_r(x) \;\cap\; \bigl(M \setminus B_r(x)\bigr),
                        \]
                        where $B_r(x)$ is the (open) ball of radius $r$.  
                        
                        \begin{itemize}
                        \item $\overline{B}_r(x)$ is closed by part (a).  
                        \item $M \setminus B_r(x)$ is also closed, because $B_r(x)$ is open.  
                        \end{itemize}
                        
                        Hence the intersection of two closed sets, $\overline{B}_r(x)\,\cap\,[M \setminus B_r(x)]$, is itself closed.  
                        Therefore, the set $\{y : D(y,x) = r\}$ is closed in $M$.
                        
                        \medskip
                        
                        \textbf{(c) The closed annulus $A(x,r,s)$ is closed.}
                        
                        Define
                        \[
                        A(x,r,s) \;=\; \{\,y \in M : r \,\le\, D(y,x) \,\le\, s\}.
                        \]
                        Note that we may write
                        \[
                        A(x,r,s)
                        \;=\;
                        \overline{B}_s(x) 
                        \;\cap\; 
                        \bigl(M \setminus B_r(x)\bigr).
                        \]
                        In other words, it is the intersection of the closed ball of radius $s$ (which is closed by part (a)) and the complement of the open ball of radius $r$ (which is again a closed set).  
                        
                        Since the intersection of two closed sets is closed, $A(x,r,s)$ is closed in $M$.  
                        \end{proof}
                        \begin{theorem}
                            Let $(M,D)$ be a metric space, and let $A \subset M$. Denote by $\overline{A}$ the closure of $A$. 
                            Show that 
                            \[
                            \mathrm{diam}(A) \;=\; \mathrm{diam}\bigl(\overline{A}\bigr).
                            \]
                            \end{theorem}
                            
                            \begin{proof}[Proof (Step by Step)]
                            Recall that the \emph{diameter} of a set $A$ is defined by
                            \[
                            \mathrm{diam}(A) 
                            \;=\; 
                            \sup\{\,D(x,y) : x,y \in A\}.
                            \]
                            
                            \medskip
                            
                            \textbf{Step 1: Show $\mathrm{diam}(A) \,\le\, \mathrm{diam}(\overline{A})$.}
                            
                            Since $A \subset \overline{A}$, any pair of points $x,y \in A$ also lies in $\overline{A}$. Therefore,
                            \[
                            \sup_{x,y \in A} D(x,y)
                            \;\le\;
                            \sup_{x,y \in \overline{A}} D(x,y).
                            \]
                            Hence
                            \[
                            \mathrm{diam}(A) 
                            \;\le\; 
                            \mathrm{diam}(\overline{A}).
                            \]
                            
                            \medskip
                            
                            \textbf{Step 2: Show $\mathrm{diam}(\overline{A}) \,\le\, \mathrm{diam}(A)$.}
                            
                            Take any $x,y \in \overline{A}$. By definition of closure, there exist sequences $\{x_n\}\subset A$ and $\{y_n\}\subset A$ such that $x_n \to x$ and $y_n \to y$ as $n \to \infty$.  
                            
                            \medskip
                            
                            \textit{Goal:} We want to show $D(x,y) \le \mathrm{diam}(A)$.  
                            
                            For any $\varepsilon > 0$, we can choose $N$ such that for all $n \ge N$,
                            \[
                            D(x_n,x) < \varepsilon
                            \quad\text{and}\quad
                            D(y_n,y) < \varepsilon.
                            \]
                            Then for $n \ge N$, by the triangle inequality:
                            \[
                            D(x,y)
                            \;\le\; 
                            D(x,x_n) + D(x_n,y_n) + D(y_n,y).
                            \]
                            Since $x_n, y_n \in A$, we know $D(x_n,y_n) \le \mathrm{diam}(A)$.  
                            Thus
                            \[
                            D(x,y)
                            \;\le\;
                            \underbrace{D(x,x_n)}_{<\,\varepsilon}
                            \;+\;
                            \underbrace{D(x_n,y_n)}_{\le\,\mathrm{diam}(A)}
                            \;+\;
                            \underbrace{D(y_n,y)}_{<\,\varepsilon}
                            \;\le\;
                            \mathrm{diam}(A) \;+\; 2\,\varepsilon.
                            \]
                            Because $\varepsilon>0$ was arbitrary, we deduce
                            \[
                            D(x,y) \;\le\; \mathrm{diam}(A).
                            \]
                            Taking the supremum over all $x,y\in \overline{A}$ then gives
                            \[
                            \mathrm{diam}(\overline{A})
                            \;\le\;
                            \mathrm{diam}(A).
                            \]
                            
                            \medskip
                            
                            \textbf{Step 3: Combine the two inequalities.}
                            
                            From Step~1 and Step~2, we conclude
                            \[
                            \mathrm{diam}(A) 
                            \;\le\; 
                            \mathrm{diam}(\overline{A})
                            \quad\text{and}\quad
                            \mathrm{diam}(\overline{A})
                            \;\le\;
                            \mathrm{diam}(A).
                            \]
                            Hence 
                            \[
                            \mathrm{diam}(A) \;=\; \mathrm{diam}(\overline{A}).
                            \]
                            This completes the proof.
                            \end{proof}

                            \begin{theorem}
                                \label{thm:equivalences}
                                The following statements are equivalent (over ZF set theory):
                                \begin{itemize}
                                  \item[\textnormal{(1)}] \textbf{Well‐Ordering Theorem (WOT):} Every set can be well‐ordered.
                                  \item[\textnormal{(2)}] \textbf{Axiom of Choice (AC):} For every family $\{X_i : i \in I\}$ of nonempty sets,
                                  there exists a choice function $f$ picking one element $f(i) \in X_i$ for each $i \in I$.
                                  \item[\textnormal{(3)}] \textbf{Zorn’s Lemma (ZL):} Every partially ordered set in which every chain
                                  (totally ordered subset) has an upper bound contains at least one maximal element.
                                \end{itemize}
                                \end{theorem}
                                
                                \begin{proof}
                                We will show (1) $\implies$ (2) $\implies$ (3) $\implies$ (1).
                                
                                \medskip
                                
                                \noindent
                                \textbf{(1) \;WOT $\implies$ AC.}
                                
                                Suppose every set can be well‐ordered. Let $\{X_i : i \in I\}$ be a family of nonempty sets. 
                                Define
                                \[
                                X \;=\; \bigcup_{i \in I} X_i.
                                \]
                                By hypothesis, there exists a well‐order $<$ on the entire set $X$. Since each $X_i$ is nonempty, pick its least element with respect to the well‐order $<$. Formally, for each $i \in I$:
                                \[
                                f(i) := \min_{<}(X_i).
                                \]
                                This defines a function $f: I \to X$ such that $f(i) \in X_i$ for all $i$. Thus $f$ is a \emph{choice function} for the family $\{X_i\}$. Hence the Axiom of Choice holds.
                                
                                \medskip
                                
                                \noindent
                                \textbf{(2) \;AC $\implies$ ZL.}
                                
                                Assume the Axiom of Choice. Let $(P,\le)$ be a partially ordered set in which every chain has an upper bound in $P$. We want to prove that $P$ has a maximal element.
                                
                                We use a standard “chain‐extension” argument. Consider the set 
                                \[
                                \mathcal{C} = \{\, C \subset P : C \text{ is a chain in }(P,\le) \}.
                                \]
                                We partially order $\mathcal{C}$ by inclusion $\subseteq$. We aim to show $\mathcal{C}$ has a maximal element under inclusion; that will correspond to a maximal chain in $P$. Once we have a maximal chain in $P$, any upper bound of that chain must lie in $P$; by a standard argument, such a chain will lead to or contain a maximal element of $P$ itself.
                                
                                \textit{Step:} Show that every chain (by inclusion) in $\mathcal{C}$ has an upper bound.  
                                If $\mathcal{T} \subset \mathcal{C}$ is totally ordered by inclusion, then let 
                                \[
                                C^* = \bigcup_{C \in \mathcal{T}} C.
                                \]
                                Clearly, $C^*$ is still a chain in $P$: any two elements in $C^*$ come from some $C \in \mathcal{T}$, and there they are comparable. Thus $C^* \in \mathcal{C}$, and $C^*$ is an upper bound of $\mathcal{T}$ (in the sense of inclusion).
                                
                                By the Axiom of Choice (in a form often called Kuratowski's Lemma or “maximal chain argument”), this implies $\mathcal{C}$ has a maximal element under inclusion, say $M \subset P$. But $M$ is a chain in $P$ which is maximal with respect to inclusion, i.e.\ a \emph{maximal chain}. A standard argument then shows $M$ yields (or directly contains) a maximal element of $P$:
                                
                                - If no point in $M$ is maximal in $P$, we could always extend $M$ by adding an element bigger than some element in $M$, contradicting the maximality of $M$ as a chain. 
                                
                                Therefore, $P$ contains a maximal element. This proves Zorn’s Lemma.
                                
                                \medskip
                                
                                \noindent
                                \textbf{(3) \;ZL $\implies$ WOT.}
                                
                                Assume Zorn’s Lemma. Let $X$ be any set. Our goal is to show that $X$ can be well‐ordered.
                                
                                Consider the set 
                                \[
                                \mathcal{W} = \{\,(Y, \prec)\colon Y \subseteq X,\; \prec \text{ is a well‐order on }Y\}.
                                \]
                                We say \((Y,\prec) \subseteq (Z,\prec')\) if $Y \subseteq Z$ and $\prec$ is the restriction of $\prec'$ to $Y$. In simpler terms, we are partially ordering these well‐ordered subsets of $X$ by “extension”: one is less than another if it is literally the same ordering on a smaller set.
                                
                                By Zorn’s Lemma, it suffices to show that every chain of well‐ordered subsets in $\mathcal{W}$ has an upper bound in $\mathcal{W}$. If $\{\,(Y_i,\prec_i)\}_{i\in I}$ is a chain in $\mathcal{W}$, define 
                                \[
                                Y^* = \bigcup_{i\in I} Y_i.
                                \]
                                We can define an order $\prec^*$ on $Y^*$ naturally, by saying that $y \prec^* z$ if and only if $y \prec_i z$ in one (and hence any) of the well‐ordered sets $(Y_i,\prec_i)$ containing both $y$ and $z$. Because the $Y_i$ form a chain under inclusion, this is well‐defined. One can check $\prec^*$ is a well‐ordering of $Y^*$.
                                
                                Hence $(\,Y^*,\,\prec^*)$ is an upper bound of our chain in $\mathcal{W}$ (by inclusion). By Zorn’s Lemma, there is a maximal element $(M,\prec_M)$ in $\mathcal{W}$. 
                                
                                To finish, we argue that $M = X$. If $M \neq X$, pick some $x_0 \in X\setminus M$. We can extend $\prec_M$ to a new well‐ordering on $M \cup \{x_0\}$ by placing $x_0$ above all elements of $M$ or, if you wish, integrate $x_0$ in any well‐ordered way you like (e.g.\ treat $x_0$ as a “new greatest” element, or insert it in the appropriate place if $X$ is not set‐like). This strictly extends $(M,\prec_M)$, contradicting its maximality. 
                                
                                Thus $M = X$ and $\prec_M$ well‐orders the entire set $X$. Therefore $X$ can be well‐ordered.
                                
                                \medskip
                                
                                Combining all three directions gives the equivalence: 
                                \[
                                \text{Well-Ordering Theorem} 
                                \;\iff\;
                                \text{Axiom of Choice} 
                                \;\iff\;
                                \text{Zorn's Lemma}.
                                \]
                                \end{proof}
\end{document}

