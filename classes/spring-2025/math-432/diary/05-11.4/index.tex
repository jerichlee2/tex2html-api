\documentclass[12pt]{article}

% Packages
\usepackage[margin=1in]{geometry}
\usepackage{amsmath,amssymb,amsthm}
\usepackage{enumitem}
\usepackage{hyperref}
\usepackage{xcolor}
\usepackage{import}
\usepackage{xifthen}
\usepackage{pdfpages}
\usepackage{transparent}
\usepackage{listings}
\usepackage{tikz}
\usepackage{physics}
\usepackage{siunitx}
\usepackage{booktabs}
\usepackage{cancel}
  \usetikzlibrary{calc,patterns,arrows.meta,decorations.markings}


\DeclareMathOperator{\Log}{Log}
\DeclareMathOperator{\Arg}{Arg}

\lstset{
    breaklines=true,         % Enable line wrapping
    breakatwhitespace=false, % Wrap lines even if there's no whitespace
    basicstyle=\ttfamily,    % Use monospaced font
    frame=single,            % Add a frame around the code
    columns=fullflexible,    % Better handling of variable-width fonts
}

\newcommand{\incfig}[1]{%
    \def\svgwidth{\columnwidth}
    \import{./Figures/}{#1.pdf_tex}
}
\theoremstyle{definition} % This style uses normal (non-italicized) text
\newtheorem{solution}{Solution}
\newtheorem{proposition}{Proposition}
\newtheorem{problem}{Problem}
\newtheorem{lemma}{Lemma}
\newtheorem{theorem}{Theorem}
\newtheorem{remark}{Remark}
\newtheorem{note}{Note}
\newtheorem{definition}{Definition}
\newtheorem{example}{Example}
\newtheorem{corollary}{Corollary}
\theoremstyle{plain} % Restore the default style for other theorem environments
%

% Theorem-like environments
% Title information
\title{MATH 432 Practice Final Exam 1 Solution}
\author{Jerich Lee}
\date{\today}

\begin{document}

\maketitle
%-------------------------------------------------------------------
%  Problem 1 – Set algebra with the symmetric difference
%-------------------------------------------------------------------
\paragraph{Notation.}
For sets \(A,B\) write
\[
   A\setminus B=\{x\in A:x\notin B\},
   \qquad
   A\triangle B=(A\setminus B)\cup(B\setminus A).
\]

\bigskip
\begin{enumerate}[label=\textbf{(\alph*)}]
%-----------------------------------------------------------------
\item \(\boxed{\,A\triangle B=(A\cup B)\setminus(A\cap B)\,}\)

      \[
      \begin{aligned}
      (A\cup B)\setminus(A\cap B)
           &=\bigl[(A\cup B)\cap(A\cap B)^{\mathrm c}\bigr]          \\[2pt]
           &=\bigl[(A\cap B^{\mathrm c})\cup(B\cap A^{\mathrm c})\bigr] \\[2pt]
           &=(A\setminus B)\cup(B\setminus A)
           =A\triangle B .
      \end{aligned}
      \]

%-----------------------------------------------------------------
\item \(\boxed{\,A\cap(B\triangle C)=(A\cap B)\triangle(A\cap C)\,}\)

      \[
      \begin{aligned}
      A\cap(B\triangle C)
         &=A\cap\bigl[(B\setminus C)\cup(C\setminus B)\bigr]\\
         &=(A\cap B\cap C^{\mathrm c})\cup(A\cap C\cap B^{\mathrm c})\\
         &=(A\cap B)\setminus(A\cap C)\;\cup\;
           (A\cap C)\setminus(A\cap B)\\
         &=(A\cap B)\triangle(A\cap C).
      \end{aligned}
      \]

%-----------------------------------------------------------------
\item \(\boxed{\,A\triangle C\subset (A\triangle B)\cup(B\triangle C)\,}\)

      Let \(x\in A\triangle C\).
      Then \(x\in A\setminus C\) or \(x\in C\setminus A\).

      *If \(x\in A\setminus C\)* and \(x\in B\) then  
      \(x\in B\setminus C\subset B\triangle C\);  
      if \(x\notin B\) then \(x\in A\setminus B\subset A\triangle B\).

      *If \(x\in C\setminus A\)* and \(x\in B\) then  
      \(x\in B\setminus A\subset A\triangle B\);  
      if \(x\notin B\) then \(x\in C\setminus B\subset B\triangle C\).

      In every case \(x\in(A\triangle B)\cup(B\triangle C)\).

%-----------------------------------------------------------------
\item \(\boxed{\,C\setminus(A\cup B)=(C\setminus A)\cap(C\setminus B)\,}\)

      This is one of De~Morgan’s laws:
      \[
         (A\cup B)^{\mathrm c}=A^{\mathrm c}\cap B^{\mathrm c}
         \quad\Longrightarrow\quad
         C\cap(A\cup B)^{\mathrm c}
           =(C\cap A^{\mathrm c})\cap(C\cap B^{\mathrm c}).
      \]

%-----------------------------------------------------------------
\item \textbf{(Associativity)}  
      \(\boxed{\, (A\triangle B)\triangle C = A\triangle(B\triangle C)\,}\)

      Work with characteristic functions
      \(\chi_S\colon X\to\{0,1\}\) (\(x\mapsto1\) if \(x\in S\)).
      Then
      \[
         \chi_{A\triangle B}
           =\chi_A+\chi_B\pmod2,\qquad
         \chi_{(A\triangle B)\triangle C}
           =\chi_A+\chi_B+\chi_C\pmod2,
      \]
      and similarly for the right–hand side; addition mod \(2\) is
      associative, hence so is \(\triangle\).
\end{enumerate}
\begin{problem}
  \textbf{Problem 2 (Posets, lattices, and completeness).}
  Let \((L,\le)\) be a lattice.
  Assume that \emph{every chain} \(C\subset L\) possesses both a least
  upper bound \(\sup C\) and a greatest lower bound \(\inf C\).
  
  \begin{enumerate}[label=\textbf{(\alph*)}]
  \item Show that \emph{every} subset \(S\subset L\) has a least upper
        bound.
  \item Show that \emph{every} subset \(S\subset L\) has a greatest lower
        bound.
  \item Conclude that \(L\) is a \emph{complete lattice}.
  \end{enumerate}
  \end{problem}
  
  \begin{solution}
  Throughout a \emph{chain} means a totally ordered subset of \(L\).
  
  %-----------------------------------------------------------------
  \subsection*{(a)  Existence of \(\sup S\) for an arbitrary subset
                \(S\subset L\)}
  
  Let \(S\subset L\) be arbitrary and set
  \[
     \mathcal U
        \;=\;
     \{\,u\in L \mid s\le u\text{ for all }s\in S\}
  \]
  (the set of \emph{upper bounds} of \(S\)).
  Order \(\mathcal U\) by the restriction of \(\le\).
  
  \smallskip
  \textit{Claim.}  Every chain \(\mathcal C\subset\mathcal U\) has an
  \emph{upper bound} in \(\mathcal U\).
  
  Indeed, \(\mathcal C\) is also a chain in \(L\); by the hypothesis it
  has a least upper bound \(c=\sup \mathcal C\in L\).
  Each \(s\in S\) is \(\le\) every element of \(\mathcal C\), hence
  \(s\le c\).
  Thus \(c\) is an upper bound of \(S\), i.e.\ \(c\in\mathcal U\), and
  certainly \(c\ge\mathcal C\).
  
  \smallskip
  By \textbf{Zorn’s lemma}, \(\mathcal U\) contains a
  \emph{minimal} element \(u^{\ast}\).
  For any upper bound \(u\in\mathcal U\) we have \(u^{\ast}\le u\)
  (otherwise \(\{u,u^{\ast}\}\) would contradict minimality), so
  \(u^{\ast}\) is the \emph{least} upper bound of \(S\):
  \[
     \boxed{\;
        \sup S \;=\; u^{\ast}\in L.
     \;}
  \]
  
  %-----------------------------------------------------------------
  \subsection*{(b)  Existence of \(\inf S\)}
  
  Apply the same argument to the set of \emph{lower} bounds
  \(
     \mathcal L=\{\,\ell\in L \mid \ell\le s\text{ for all }s\in S\},
  \)
  ordered by \(\le\).
  Chains in \(\mathcal L\) have lower bounds by the
  assumption on chains, so Zorn’s lemma yields a
  \emph{maximal} element \(\ell^{\ast}\), which is the greatest lower
  bound of \(S\):
  \[
     \boxed{\;
        \inf S \;=\; \ell^{\ast}\in L.
     \;}
  \]
  
  %-----------------------------------------------------------------
  \subsection*{(c)  Completeness of \(L\)}
  
  Parts (a) and (b) show that \emph{every} subset \(S\subset L\) has both
  a supremum and an infimum in \(L\).  
  Therefore \(L\) is a \textbf{complete lattice}.
  \end{solution}
  %--------------------------------------------------------------------
%  Problem 3 – Composition of functions and surjectivity
%--------------------------------------------------------------------
\paragraph{Problem.}
Let \(f:A\to B\) and \(g:B\to C\) be functions.
\begin{enumerate}[label=\textbf{(\alph*)}]
   \item Show that if both \(f\) and \(g\) are onto (surjective), then
         the composition \(g\circ f\) is onto.
   \item Show that if \(g\circ f\) is onto, then \(g\) is onto.
   \item Give an explicit example in which \(g\circ f\) is onto but
         \(f\) is \emph{not} onto.
\end{enumerate}

\bigskip
\paragraph{Solution.}

\begin{enumerate}[label=\textbf{(\alph*)}]
%-----------------------------------------------------------------
\item \textit{Surjectivity of a composition.}

   Assume \(f\) and \(g\) are surjective.
   Let \(c\in C\).  
   Since \(g\) is onto, there exists \(b\in B\) with \(g(b)=c\).
   Because \(f\) is onto, there exists \(a\in A\) with \(f(a)=b\).
   Then
   \[
      (g\circ f)(a)=g\bigl(f(a)\bigr)=g(b)=c.
   \]
   Hence every \(c\in C\) is attained by \(g\circ f\); the composition
   is surjective.

%-----------------------------------------------------------------
\item \textit{Surjectivity of \(g\) from surjectivity of the
   composition.}

   Suppose \(g\circ f\) is onto and let \(c\in C\).
   By surjectivity of \(g\circ f\) there exists \(a\in A\) such that
   \((g\circ f)(a)=c\).
   Set \(b=f(a)\in B\); then \(g(b)=c\).
   Thus every \(c\in C\) lies in the image of \(g\), so \(g\) is
   surjective.

%-----------------------------------------------------------------
\item \textit{Example where \(g\circ f\) is onto but \(f\) is not.}

   Take finite sets
   \[
      A=\{\,1\,\},\qquad
      B=\{\,1,2\,\},\qquad
      C=\{\,\star\,\}.
   \]
   Define
   \[
      f:A\longrightarrow B,\quad f(1)=1
      \quad(\text{not onto, since }2\notin\operatorname{im}f),
   \]
   \[
      g:B\longrightarrow C,\quad g(1)=\star,\; g(2)=\star
      \quad(\text{onto}).
   \]
   Then
   \[
      (g\circ f)(1)=g\bigl(f(1)\bigr)=g(1)=\star,
   \]
   so \(g\circ f:A\to C\) is onto even though \(f\) is not.

\end{enumerate}
%--------------------------------------------------------------------
%  Problem 4 — Countability
%--------------------------------------------------------------------
\paragraph{Problem 4.}
\begin{enumerate}[label=\textbf{(\alph*)}]
\item Prove that the finite Cartesian product
      \(\Bbb N\times\Bbb N\times\cdots\times\Bbb N\)
      (\(k\) copies, \(k<\infty\)) is countable.
      Deduce that any finite product of countable sets is countable.

\item Let \(A\) be a countably infinite set and put
      \[
         X=\bigl\{\,F\subset A : F\text{ is finite}\bigr\}.
      \]
      Show that \(X\) is countable.
\end{enumerate}

\bigskip
\paragraph{Solution.}

\begin{enumerate}[label=\textbf{(\alph*)}]
%-----------------------------------------------------------------
\item \textit{Finite powers of \(\Bbb N\) are countable.}

We prove by induction on \(k\ge1\) that
\(\Bbb N^{k}:=\underbrace{\Bbb N\times\dots\times\Bbb N}_{k\text{ times}}\)
is countable.

\emph{Base \(k=1\).}  \(\Bbb N\) is countable.

\emph{Induction step.}  Assume \(\Bbb N^{k}\) is countable.
Since \(\Bbb N\) is countable, the product \(\Bbb N^{k+1}
          =\Bbb N^{k}\times\Bbb N\) is a countable union of countable
sets:
\[
   \Bbb N^{k+1}
      =\bigcup_{n\in\Bbb N}\!\bigl(\Bbb N^{k}\times\{n\}\bigr),
      \qquad
      \bigl|\Bbb N^{k}\times\{n\}\bigr|
         =|\Bbb N^{k}|\;(<\infty).
\]
A countable union of countable sets is countable, completing the
induction.

\smallskip
\textit{Finite products of countable sets.}

Let \(B_{1},\dots,B_{k}\) be countable.
Choose enumerations \(B_{j}=\{b_{j,0},b_{j,1},\dots\}\).
Define
\[
   f: \Bbb N^{k}\longrightarrow B_{1}\times\dots\times B_{k},
   \qquad
   f(n_{1},\dots,n_{k})
     =(b_{1,n_{1}},\dots,b_{k,n_{k}}).
\]
The map \(f\) is surjective, so the product of the \(B_j\) is imaged
by a countable set; hence it is countable.

%-----------------------------------------------------------------
\item \textit{The set of finite subsets of a countable set is countable.}

Since \(A\) is countably infinite, list it as
\(A=\{a_{0},a_{1},a_{2},\dots\}\).

For each \(m\ge0\) let
\(
   X_{m}=\bigl\{\,F\subset A : |F|=m\bigr\}
\)
(the $m$–element subsets of \(A\)).
Then
\(
   X=\bigcup_{m=0}^{\infty} X_{m}.
\)

\emph{Claim.} Each \(X_{m}\) is countable.  
Indeed, identify \(F=\{a_{n_{1}},\dots,a_{n_{m}}\}\) with the strictly
increasing \(m\)-tuple \((n_{1},\dots,n_{m})\in\Bbb N^{m}\);
conversely every such tuple gives an \(m\)-element set.
Thus \(X_{m}\) is in bijection with a subset of \(\Bbb N^{m}\),
hence countable by part (a).

Finally, \(X\) is a \emph{countable union} of the countable sets
\(X_{m}\), so \(X\) itself is countable.

\[
   \boxed{\;X\text{ is countable.}\;}
\]

\end{enumerate}
\begin{solution}
  \textbf{Problem 5. Cardinal arithmetic and explicit bijections}
  
  \medskip
  \textbf{(a)  A bijection \(\displaystyle (0,1)\times(0,1)\times(0,1)\;\longrightarrow\;(0,1)\).}
  
  \begin{enumerate}
      \item[\emph{Step 1:  Fix a unique binary expansion.}]
            Every number \(x\in(0,1)\) has a binary expansion
            \[
                x \;=\; 0.x_{1}x_{2}x_{3}\dots_{\!2},
                \qquad x_{k}\in\{0,1\}.
            \]
            When \(x\) has \emph{two} binary expansions (the terminating one
            ending in an infinite string of \(1\)’s), always choose the
            non‑terminating version (ending in infinitely many \(0\)’s).
            This makes the expansion unique.
  
      \item[\emph{Step 2:  Define the map.}]
            For a triple \(\bigl(x,y,z\bigr)\in(0,1)^{3}\) write
            \[
                x=0.x_1x_2x_3\dots_{\!2},\quad
                y=0.y_1y_2y_3\dots_{\!2},\quad
                z=0.z_1z_2z_3\dots_{\!2}.
            \]
            Put the digits together \emph{three at a time}:
            \[
                f(x,y,z)
                \;=\;
                0.\,x_1y_1z_1\,x_2y_2z_2\,x_3y_3z_3\dots_{\!2}
                \;\in\;(0,1).
            \]
  
      \item[\emph{Step 3:  \(f\) is injective.}]
            If two distinct triples differ in at least one coordinate, the
            very first binary digit where they differ appears at position
            \(3k-2,\,3k-1\), or \(3k\) of \(f(x,y,z)\).  Hence
            \(f(x,y,z)\neq f(x',y',z')\).
  
      \item[\emph{Step 4:  \(f\) is surjective.}]
            Given \(w\in(0,1)\) with unique binary expansion
            \( w=0.d_{1}d_{2}d_{3}d_{4}d_{5}\dots_{\!2}\),
            regroup the digits in blocks of three:
            \[
                x=0.d_{1}d_{4}d_{7}\dots_{\!2},\quad
                y=0.d_{2}d_{5}d_{8}\dots_{\!2},\quad
                z=0.d_{3}d_{6}d_{9}\dots_{\!2}.
            \]
            Then \(f(x,y,z)=w\).  Hence \(f\) is bijective.
  
      \item[\emph{Step 5:  Cardinality consequence.}]
            Because \(\bigl|(0,1)\bigr|=\bigl|\mathbb{R}\bigr|\) via the
            standard bijection \(t\mapsto\frac{1}{\pi}\arctan t+\tfrac12\),
            the bijection above yields
            \[
                \bigl|\mathbb{R}^{3}\bigr|
                =\bigl|(0,1)^{3}\bigr|
                =\bigl|(0,1)\bigr|
                =\bigl|\mathbb{R}\bigr|.
            \]
            (The same proof, interleaving \(n\) binary strings, shows
            \(|\mathbb{R}^{n}|=|\mathbb{R}|\) for every finite \(n\ge 1\).)
  \end{enumerate}
  
  \bigskip
  \textbf{(b)  Showing \(c+c=c\) without Theorems 13-14.}
  
  Let \(c=\bigl|\mathbb{R}\bigr|\).  Adding \(c+c\) means taking the
  disjoint union of two copies of \(\mathbb{R}\);
  equivalently consider the product \(\mathbb{R}\times\{0,1\}\).
  
  \begin{enumerate}
      \item[\emph{Step 1:  Shrink two copies of \((0,1)\) inside \((0,1)\).}]
            Define
            \[
                \varphi:(0,1)\times\{0,1\}\;\longrightarrow\;(0,1),\qquad
                \varphi(x,i)=
                \begin{cases}
                     \dfrac{x}{2},              & i=0,\\[6pt]
                     \dfrac{x+1}{2},            & i=1.
                \end{cases}
            \]
            The images are the disjoint intervals \((0,\tfrac12)\) and
            \((\tfrac12,1)\), so \(\varphi\) is a bijection.
  
      \item[\emph{Step 2:  Transport the bijection to \(\mathbb{R}\).}]
            Use the earlier bijection
            \[
                \psi:\mathbb{R}\;\longrightarrow\;(0,1),\qquad
                \psi(t)=\frac{1}{\pi}\arctan t+\frac12,
            \qquad
                \psi^{-1}(u)=\tan\!\bigl(\pi(u-\tfrac12)\bigr).
            \]
            Define
            \[
                F:\mathbb{R}\times\{0,1\}\;\longrightarrow\;\mathbb{R},
                \qquad
                F(t,i)=\psi^{-1}\!\Bigl(\,\varphi\bigl(\psi(t),i\bigr)\Bigr).
            \]
            Because \(\psi\) and \(\varphi\) are bijections, so is \(F\).
  
      \item[\emph{Step 3:  Conclude the cardinal arithmetic.}]
            The domain of \(F\) has cardinality
            \(\bigl|\mathbb{R}\times\{0,1\}\bigr|=c\cdot 2=c+c\).
            Its codomain has cardinality \(c\).
            Hence \(c+c=c\).
  \end{enumerate}
  \end{solution}
  
%--------------------------------------------------------------------
%  Problem 6  –  Metric–space topology
%--------------------------------------------------------------------
Let \((M,D)\) be a metric space, fix a point \(x\in M\) and real numbers
\(0<r<s\).

\[
\begin{aligned}
A(x,r,s) &:= \{\,y\in M\mid r<D(x,y)<s\}\quad &\text{(punctured annulus)},\\
S(x,r)   &:= \{\,y\in M\mid D(x,y)=r\}\quad &\text{(sphere)},\\
\overline A(x,r,s) &:= \{\,y\in M\mid r\le D(x,y)\le s\}
                           &\text{(closed annulus).}
\end{aligned}
\]

\begin{enumerate}[label=\textbf{(\alph*)}]
%-----------------------------------------------------------------
\item \emph{\(A(x,r,s)\) is open.}

   Let \(y\in A(x,r,s)\); write \(d=D(x,y)\), so \(r<d<s\).
   Set
   \[
      \varepsilon=\frac12\min\bigl\{\,d-r,\;s-d\,\bigr\}>0.
   \]
   For any \(z\) with \(D(y,z)<\varepsilon\) the triangle inequality
   gives
   \[
      |D(x,z)-d|\le D(x,z)+D(x,y)-d<\varepsilon,
      \qquad\Longrightarrow\qquad
      r< D(x,z)< s .
   \]
   Hence the open ball \(B(y,\varepsilon)\subset A(x,r,s)\), so each
   point of \(A(x,r,s)\) is interior and the set is open.

%-----------------------------------------------------------------
\item \emph{\(S(x,r)\) is closed.}

   The map
   \[
      f:M\longrightarrow\Bbb R,\qquad f(y)=D(x,y),
   \]
   is continuous (triangle inequality).
   The sphere is the pre-image of the closed set \(\{r\}\subset\Bbb R\):
   \[
      S(x,r)=f^{-1}\bigl(\{r\}\bigr).
   \]
   A continuous pre-image of a closed set is closed, so \(S(x,r)\) is
   closed in \(M\).

%-----------------------------------------------------------------
\item \emph{\(\overline A(x,r,s)\) is closed.}

   With the same map \(f\) as above,
   \[
      \overline A(x,r,s)=f^{-1}\bigl([\,r,s\,]\bigr).
   \]
   The interval \([r,s]\) is closed in \(\Bbb R\); therefore its
   pre-image under the continuous map \(f\) is closed in \(M\).

\end{enumerate}
%--------------------------------------------------------------------
%  Problem 7 – Uniform continuity on \(\Bbb R\)
%--------------------------------------------------------------------
Determine which of the following maps \(f:\Bbb R\to\Bbb R\) are
uniformly continuous.  Prove each claim.

\[
\begin{array}{lll}
\text{(i)} & f(x)=x^{2}, \\[4pt]
\text{(ii)}& f(x)=|x|,   \\[4pt]
\text{(iii)}& f(x)=\dfrac{1}{1+x^{2}} .
\end{array}
\]

\bigskip
\hrule
\bigskip

\textbf{Answer.}
\[
   \boxed{\;
      \text{\(x^{2}\) is \emph{not} uniformly continuous, while
      \(|x|\) and \(\dfrac1{1+x^{2}}\) \textbf{are} uniformly
      continuous.}
   \;}
\]

\bigskip
\textbf{Proofs.}

\begin{enumerate}[label=\textbf{(\roman*)}]

%-----------------------------------------------------------------
\item \emph{\(f(x)=x^{2}\) is not uniformly continuous on \(\Bbb R\).}

   Take \(\varepsilon=1\).
   For any \(\delta>0\) choose
   \[
      x = \frac1\delta+\frac{\delta}{2}, \qquad
      t = \frac1\delta .
   \]
   Then \(|x-t|=\dfrac{\delta}{2}<\delta\) but
   \[
      |x^{2}-t^{2}| = (x+t)(x-t)
                    =\left(\frac{2}{\delta}+\frac{\delta}{2}\right)
                     \frac{\delta}{2}
                    > 1=\varepsilon .
   \]
   Hence the definition of uniform continuity fails; \(x^{2}\) is not
   uniformly continuous on \(\Bbb R\).

%-----------------------------------------------------------------
\item \emph{\(f(x)=|x|\) is uniformly continuous on \(\Bbb R\).}

   For all \(x,t\in\Bbb R\),
   \[
      \bigl||x|-|t|\bigr|\;\le\;|x-t| \quad\text{(triangle inequality).}
   \]
   Given \(\varepsilon>0\) choose \(\delta=\varepsilon\).
   Then \(|x-t|<\delta\Rightarrow |\,|x|-|t|\,|<\varepsilon\).
   Hence \(|x|\) is uniformly continuous.

%-----------------------------------------------------------------
\item \emph{\(\displaystyle f(x)=\dfrac{1}{1+x^{2}}\) is uniformly
        continuous on \(\Bbb R\).}

   For \(x,t\in\Bbb R\),
   \[
      |f(x)-f(t)|
        =\left|\frac{1}{1+x^{2}}-\frac{1}{1+t^{2}}\right|
        =\frac{|x^{2}-t^{2}|}{(1+x^{2})(1+t^{2})}
        =\frac{|x+t|\,|x-t|}{(1+x^{2})(1+t^{2})}.
   \]
   Because \(|x+t|\le |x|+|t|\) and the denominators are \(\ge1\),
   \[
      |f(x)-f(t)| \;\le\; (|x|+|t|)\,|x-t|
                    \;\le\; 2\,|x-t|.
   \]
   Given \(\varepsilon>0\) pick \(\delta=\dfrac{\varepsilon}{2}\);
   then \(|x-t|<\delta\Rightarrow |f(x)-f(t)|<\varepsilon\).
   Thus \(f\) is uniformly continuous on \(\Bbb R\).

\end{enumerate}
%---------------------------------------------------------------
%  Every vector space has a basis  (Zorn–Lemma proof)
%---------------------------------------------------------------
\begin{theorem}[Basis theorem]
   Every (possibly infinite–dimensional) real vector space \(V\) possesses
   a basis, i.e.\ a linearly independent subset \(B\subset V\) that spans
   the whole space.
   \end{theorem}
   
   \begin{proof}[Proof via Zorn’s lemma, step–by–step]
   \textbf{Step 1.  Set–up a partially ordered set.}
   
   Let
   \[
      \mathcal L
         \;=\;
      \bigl\{\,L\subset V \,\bigm|\; L\text{ is linearly independent}\bigr\},
   \]
   ordered by inclusion \(\subset\).
   Clearly \((\mathcal L,\subset)\) is a partially ordered set.
   
   \medskip
   \textbf{Step 2.  Chains admit upper bounds.}
   
   Let \(\mathcal C\subset\mathcal L\) be a \emph{chain}
   (totaly ordered by inclusion).
   Define
   \[
      L^{\ast}:=\bigcup_{L\in\mathcal C} L .
   \]
   \emph{Claim: \(L^{\ast}\) is linearly independent.}
   
   Take a finite subset \(\{v_{1},\dots,v_{n}\}\subset L^{\ast}\).
   Because \(\mathcal C\) is totally ordered, there exists
   \(L_{0}\in\mathcal C\) containing all the \(v_{i}\).
   Since \(L_{0}\) is linearly independent, any relation
   \(\sum_{i} \alpha_{i}v_{i}=0\) forces \(\alpha_{i}=0\);
   hence \(L^{\ast}\) is linearly independent and belongs to \(\mathcal L\).
   
   Thus \(L^{\ast}\) is an \emph{upper bound} of the chain \(\mathcal C\).
   
   \medskip
   \textbf{Step 3.  Apply Zorn’s lemma.}
   
   Because every chain in \(\mathcal L\) has an upper bound in
   \(\mathcal L\), Zorn’s lemma yields a
   \emph{maximal} linearly independent subset \(B\subset V\)
   (\(B\in\mathcal L\)).
   
   \medskip
   \textbf{Step 4.  Show \(B\) spans \(V\).}
   
   Assume, to obtain a contradiction, that
   \(v_{0}\in V\setminus\mathrm{span}(B)\).
   Then \(B\cup\{v_{0}\}\) is linearly independent
   (otherwise \(v_{0}\) would be a linear combination of elements of \(B\)).
   This strictly contains \(B\), contradicting the maximality of \(B\).
   Therefore no such \(v_{0}\) exists and \(\mathrm{span}(B)=V\).
   
   \medskip
   \textbf{Step 5.  Conclusion.}
   
   The set \(B\) is linearly independent (by construction) and spans \(V\),
   so \(B\) is a \emph{basis} of \(V\).
   \end{proof}

   
\end{document}

