\documentclass[12pt]{article}

% Packages
\usepackage[margin=1in]{geometry}
\usepackage{amsmath,amssymb,amsthm}
\usepackage{enumitem}
\usepackage{hyperref}
\usepackage{xcolor}
\usepackage{import}
\usepackage{xifthen}
\usepackage{pdfpages}
\usepackage{transparent}
\usepackage{listings}
\usepackage{tikz}
\usepackage{physics}
\usepackage{siunitx}
\usepackage{booktabs}
\usepackage{cancel}
  \usetikzlibrary{calc,patterns,arrows.meta,decorations.markings}


\DeclareMathOperator{\Log}{Log}
\DeclareMathOperator{\Arg}{Arg}

\lstset{
    breaklines=true,         % Enable line wrapping
    breakatwhitespace=false, % Wrap lines even if there's no whitespace
    basicstyle=\ttfamily,    % Use monospaced font
    frame=single,            % Add a frame around the code
    columns=fullflexible,    % Better handling of variable-width fonts
}

\newcommand{\incfig}[1]{%
    \def\svgwidth{\columnwidth}
    \import{./Figures/}{#1.pdf_tex}
}
\theoremstyle{definition} % This style uses normal (non-italicized) text
\newtheorem{solution}{Solution}
\newtheorem{proposition}{Proposition}
\newtheorem{problem}{Problem}
\newtheorem{lemma}{Lemma}
\newtheorem{theorem}{Theorem}
\newtheorem{remark}{Remark}
\newtheorem{note}{Note}
\newtheorem{definition}{Definition}
\newtheorem{example}{Example}
\newtheorem{corollary}{Corollary}
\theoremstyle{plain} % Restore the default style for other theorem environments
%

% Theorem-like environments
% Title information
\title{MATH 432 Practice Final Exam 2 Solution}
\author{Jerich Lee}
\date{\today}

\begin{document}

\maketitle
\begin{solution}
  \textbf{Problem 1. Chains in power sets}
  
  Let \(A\) be a set whose power set \(\mathcal P(A)\), ordered by
  inclusion \(\subseteq\), is a \emph{chain}.  Recall that a chain means:
  for every \(X,Y\subseteq A\) we have either \(X\subseteq Y\) or
  \(Y\subseteq X\).
  
  \medskip
  \textbf{Claim.}  The chain condition holds \emph{iff}
  \(\lvert A\rvert\le 1\).
  
  \begin{proof}[Proof of the claim]
  \emph{(Necessity).}
  Assume, to obtain a contradiction, that \(\lvert A\rvert\ge 2\).
  Pick two distinct elements \(a,b\in A\).
  Consider the singleton subsets
  \[
     \{a\}\quad\text{and}\quad\{b\}.
  \]
  Neither is contained in the other, i.e.
  \(\{a\}\nsubseteq\{b\}\) and \(\{b\}\nsubseteq\{a\}\).
  Thus \(\mathcal P(A)\) fails to be a chain.  Hence a chain power set
  implies \(\lvert A\rvert\le 1\).
  
  \smallskip
  \emph{(Sufficiency).}
  \begin{itemize}
      \item If \(A=\varnothing\) then \(\mathcal P(A)=\{\varnothing\}\)
            has a single element and is trivially a chain.
      \item If \(A=\{a\}\) then
            \(\mathcal P(A)=\{\varnothing,\{a\}\}\) and the two subsets
            satisfy \(\varnothing\subseteq\{a\}\); again a chain.
  \end{itemize}
  Therefore whenever \(\lvert A\rvert\le 1\), the power set is a chain.
  
  \smallskip
  Combining both directions gives the desired equivalence.
  \end{proof}
  
  \medskip
  \textbf{Conclusion.}
  The only sets whose power sets are chains (under inclusion) are
  \[
     A=\varnothing
     \quad\text{or}\quad
     A=\{a\}\ \text{for some element }a.
  \]
  \(\square\)
  \end{solution}
  %-------------------------------------------------------------------
%  Problem 2 – Equivalence classes induced by a preorder
%-------------------------------------------------------------------
\begin{solution}
  
Let \(A\) be a set and \(R\subset A\times A\) a \emph{pre-order}
(reflexive and transitive).  Define
\[
   \widetilde R
      \;=\;
   \bigl\{\, (a,b)\in A\times A
            \;\bigm|\;
            (a,b)\in R\ \text{and}\ (b,a)\in R
       \bigr\}.
\]

\bigskip
\begin{enumerate}[label=\textbf{(\alph*)}]
%-----------------------------------------------------------------
\item \textit{\(\widetilde R\) is an equivalence relation.}

\begin{itemize}
   \item \emph{Reflexive:}  
         For every \(a\in A\) we have \((a,a)\in R\)
         (reflexivity of \(R\)); hence \((a,a)\in\widetilde R\).

   \item \emph{Symmetric:}  
         If \((a,b)\in\widetilde R\) then \((a,b)\in R\) and \((b,a)\in R\).
         Interchanging \(a,b\) shows \((b,a)\in\widetilde R\).

   \item \emph{Transitive:}  
         Suppose \((a,b)\in\widetilde R\) and \((b,c)\in\widetilde R\).
         Then
         \((a,b),\,(b,a),\,(b,c),\,(c,b)\in R\).
         By transitivity of \(R\):
         \[
            (a,c)\in R \quad\text{and}\quad (c,a)\in R ,
         \]
         so \((a,c)\in\widetilde R\).
\end{itemize}
Thus \(\widetilde R\) is reflexive, symmetric and transitive; i.e.\ an
\emph{equivalence relation} on \(A\).

%-----------------------------------------------------------------
\item \textit{The quotient set \(A/{\sim}\) is partially ordered.}

For \(a\in A\) write
\[
   S_a=\{\,x\in A\mid(a,x)\in\widetilde R\}
\]
for its \(\widetilde R\)-equivalence class, and define
\[
   S_a\;\le\;S_b
   \quad\Longleftrightarrow\quad
   (a,b)\in R .
\]
We must check that this relation is

\smallskip\emph{(i) well-defined.}
If \(a'\in S_a\) and \(b'\in S_b\) (i.e.\ \((a,a')\in\widetilde R\) and
\((b,b')\in\widetilde R\)), then
\[
   (a',b')\in R:
   \quad
   (a',a),(a,b),(b,b')\in R
   \;\Longrightarrow\;
   (a',b')\in R\ \text{(transitivity).}
\]
Hence \(S_a\le S_b\) does not depend on the chosen representatives.

\smallskip\emph{(ii) reflexive.}
\((a,a)\in R\) for every \(a\), so \(S_a\le S_a\).

\smallskip\emph{(iii) antisymmetric.}
If \(S_a\le S_b\) and \(S_b\le S_a\) then \((a,b),(b,a)\in R\);
thus \((a,b)\in\widetilde R\) and \(S_a=S_b\).

\smallskip\emph{(iv) transitive.}
If \(S_a\le S_b\) and \(S_b\le S_c\) then
\((a,b),(b,c)\in R\); transitivity of \(R\) gives
\((a,c)\in R\), hence \(S_a\le S_c\).

\smallskip
Therefore the relation \(\le\) on the set of classes
\(\{S_a : a\in A\}\) satisfies reflexivity, antisymmetry, and
transitivity; it is a \emph{partial order}.
\end{enumerate}
\end{solution}

\begin{solution}
  \textbf{Problem 3.  Linear extensions of finite partial orders}
  
  Let \((L,\le)\) be a finite partially ordered set (poset).  
  We prove that \(L\) admits a \emph{total} (linear) order \(\preceq\) that
  \emph{extends} \(\le\); that is,
  \[
     \forall\,x,y\in L,\qquad x\le y\;\Longrightarrow\;x\preceq y.
  \]
  
  \medskip
  \textbf{Proof by induction on \(\lvert L\rvert\).}
  
  \smallskip
  \emph{Base case \(\lvert L\rvert=1\).}
  With a single element, the unique binary relation is already a total
  order extending \(\le\).
  
  \smallskip
  \emph{Inductive step.}
  Assume every poset with at most \(n\) elements has a linear extension.  
  Let \((L,\le)\) have \(n+1\) elements.
  
  \begin{enumerate}
      \item[\textbf{1.}] \emph{Choose a minimal element.}
            Because \(L\) is finite, it contains at least one
            \(\le\)-minimal element.  Fix such an element and call it
            \(m\).
  
      \item[\textbf{2.}] \emph{Apply the induction hypothesis to the
            remainder.}
            Set \(L' := L\setminus\{m\}\); then \(\lvert L'\rvert = n\).
            The restriction of \(\le\) to \(L'\) is a poset, so by the
            inductive hypothesis there exists a total order
            \(\preceq'\) on \(L'\) extending \(\le\).
  
      \item[\textbf{3.}] \emph{Adjoin \(m\) at the bottom.}
            Define a relation \(\preceq\) on \(L\) by
            \[
               m \preceq x\quad
               \text{for all }x\in L,
               \qquad\text{and}\qquad
               x\preceq y\iff x \preceq' y
               \ \text{for }x,y\in L'.
            \]
  
      \item[\textbf{4.}] \emph{Verify that \(\preceq\) is a linear
            extension of \(\le\).}
            \begin{itemize}
                \item \emph{Totality:}  
                      For any \(x,y\in L\) there are two cases.
                      If one of them is \(m\), then \(m\preceq x,y\).
                      Otherwise both lie in \(L'\) and are comparable by
                      \(\preceq'\).
                \item \emph{Transitivity and antisymmetry} follow from the
                      corresponding properties of \(\preceq'\).
                \item \emph{Extension property:}
                      If \(x\le y\) and \(x=m\) then \(m\preceq y\) by
                      definition.  If \(x\ne m\) then \(x,y\in L'\) and
                      \(x\preceq' y\), hence \(x\preceq y\).
            \end{itemize}
            Thus \(\preceq\) is a linear order on \(L\) extending \(\le\).
  \end{enumerate}
  
  By induction, every finite poset admits such a linear extension.
  
  \medskip
  \textbf{Alternative proof via Zorn’s Lemma (sketch).}
  Let \(\mathcal{F}\) be the set of partial orders on \(L\) that include
  \(\le\), partially ordered by inclusion of relations.
  Every chain \(\mathcal{C}\subseteq\mathcal{F}\) has an upper bound
  given by the union of its relations.  By Zorn’s Lemma,
  \(\mathcal{F}\) possesses a maximal element \(\preceq_{\max}\).
  If \(\preceq_{\max}\) were \emph{not} total, pick incomparable
  \(x,y\).  Adding either \(x\preceq_{\max} y\) or
  \(y\preceq_{\max} x\) enlarges the relation but keeps it a partial
  order extending \(\le\), contradicting maximality.  Hence
  \(\preceq_{\max}\) is total, furnishing the desired extension.
  \end{solution}
  \begin{solution}
    \textbf{Problem 4.  A lattice in which every chain is bounded above
    has a \emph{unique} maximal element}
    
    Let \(L\) be a lattice (i.e.\ every pair \(x,y\in L\) has both a join
    \(x\vee y\) and a meet \(x\wedge y\)).
    Assume that \emph{every} chain \(C\subseteq L\) possesses an upper
    bound in \(L\).
    
    \medskip
    \textbf{Step 1:  Existence of (at least one) maximal element.}
    
    Because \((L,\le)\) itself is a partially ordered set and every chain
    has an upper bound, the hypotheses of Zorn’s Lemma are satisfied.
    Hence \(L\) contains a maximal element; denote one such element by
    \(m\).
    
    \medskip
    \textbf{Step 2:  Uniqueness of the maximal element.}
    
    Suppose, for contradiction, that \(m_1\) and \(m_2\) are \emph{two}
    distinct maximal elements of \(L\).
    Since \(L\) is a lattice, their join \(m_1\vee m_2\) is defined
    and satisfies
    \[
          m_1 \;\le\; m_1\vee m_2,
          \qquad
          m_2 \;\le\; m_1\vee m_2.
    \]
    By maximality, nothing can strictly exceed either \(m_1\) or \(m_2\),
    so the above inequalities force
    \[
          m_1\vee m_2 = m_1
          \quad\text{and}\quad
          m_1\vee m_2 = m_2.
    \]
    Consequently \(m_1 = m_2\), contradicting the assumption that the two
    maximal elements were distinct.
    
    \medskip
    \textbf{Conclusion.}
    There exists precisely one maximal element of \(L\).
    \(\square\)
    \end{solution}
%-------------------------------------------------------------------
%  Problem 5 —  Removing a finite or countable subset
%-------------------------------------------------------------------

\begin{solution}
  
Let \(A\) be an infinite set and \(B\subset A\).  
Write \(C:=A\setminus B\).

\begin{enumerate}[label=\textbf{(\alph*)}]
%-----------------------------------------------------------------
\item \textit{\(B\) finite.  Construct an explicit bijection
      \(\varphi:A\to C\).}

      Enumerate the elements of \(B\) as
      \(B=\{b_{1},\dots,b_{m}\}\) \(\;(m<\infty)\).
      Because \(A\) is infinite, we may choose \emph{distinct}
      elements \(c_{1},\dots,c_{m}\in C\).

      Define
      \[
         \varphi(a)=
         \begin{cases}
            c_{k}, &\text{if }a=b_{k}\ (1\le k\le m),\\[4pt]
            a,     &\text{if }a\in C.
         \end{cases}
      \]

      \emph{Injective.}
      If \(\varphi(a_{1})=\varphi(a_{2})\) there are two cases:
      \begin{itemize}
         \item both \(a_{1},a_{2}\in C\) \(\Rightarrow\) \(a_{1}=a_{2}\);
         \item say \(a_{1}=b_{k}\).  Then
               \(\varphi(a_{1})=c_{k}\).
               If \(\varphi(a_{2})=c_{k}\) either \(a_{2}=b_{k}\)
               (so \(a_{1}=a_{2}\)) or \(a_{2}=c_{k}\in C\),
               contradicting distinctness of the \(c_{j}\).
      \end{itemize}

      \emph{Surjective.}
      Every \(c\in C\) is either one of the chosen \(c_{k}\) (hit by
      \(b_{k}\)) or belongs to \(C\setminus\{c_{1},\dots,c_{m}\}\) and
      is fixed by \(\varphi\).
      Thus \(\varphi\) is bijective, so \(|A|=|C|\).

%-----------------------------------------------------------------
\item \textit{\(B\) countable, \(A\) uncountable.
      Show \(|A|=|C|\).}

      Enumerate \(B=\{b_{1},b_{2},\dots\}\).
      Since \(A\) is uncountable and \(B\) is countable,
      \(C=A\setminus B\) is still uncountable; hence
      we can pick a sequence of \emph{distinct} elements
      \(\{c_{1},c_{2},\dots\}\subset C\).

      Define
      \[
         \psi:A\longrightarrow C,
         \qquad
         \psi(a)=
         \begin{cases}
            c_{k}, & a=b_{k}\;(k\ge1),\\[4pt]
            a,     & a\in C.
         \end{cases}
      \]

      The proof that \(\psi\) is injective and surjective is identical
      to the finite case, using distinctness of the \(c_{k}\).
      Therefore \(\psi\) is a bijection, and
      \[
         \boxed{|A|=|C|}.
      \]

\end{enumerate}
\end{solution}

\begin{solution}
  \textbf{Problem 6.  Exponent law for cardinals: \(d^{\,e_1+e_2}=d^{\,e_1}\cdot d^{\,e_2}\)}
  
  \medskip
  \textbf{Setup.}
  Fix pairwise disjoint sets
  \[
     D,\;E_1,\;E_2
     \quad\text{with}\quad
     |D|=d,\;\;|E_1|=e_1,\;\;|E_2|=e_2.
  \]
  (If the original \(E_1,E_2\) are \emph{not} disjoint, replace them by
  \(E_1\times\{1\}\) and \(E_2\times\{2\}\).)
  
  For any set \(E\) write
  \[
     D^{E}\;=\;\{\,f:E\to D\,\},
  \]
  so \(\bigl|D^{E}\bigr| = d^{\,|E|}\).
  
  \medskip
  \textbf{Define the bijection.}
  Consider the function
  \[
     \Phi:
     D^{\,E_1\cup E_2}\;\longrightarrow\;
     D^{\,E_1}\times D^{\,E_2},
     \qquad
     \Phi(f) \;=\; (\,f|_{E_1},\,f|_{E_2}\,),
  \]
  where \(f|_{E_i}\) is the restriction of \(f\) to \(E_i\).
  
  \begin{enumerate}
      \item \emph{Injectivity.}
            If \(\Phi(f)=\Phi(g)\) then \(f|_{E_i}=g|_{E_i}\)
            for \(i=1,2\).
            Since every element of \(E_1\cup E_2\) lies in exactly one
            \(E_i\), it follows that \(f=g\).
      \item \emph{Surjectivity.}
            Given a pair \(\bigl(f_1,f_2\bigr)\in
            D^{\,E_1}\times D^{\,E_2}\), define
            \[
                h:E_1\cup E_2 \longrightarrow D,
                \qquad
                h(x)=
                \begin{cases}
                    f_1(x), & x\in E_1,\\[4pt]
                    f_2(x), & x\in E_2.
                \end{cases}
            \]
            Then \(\Phi(h)=(f_1,f_2)\).
  \end{enumerate}
  Hence \(\Phi\) is a bijection.
  
  \medskip
  \textbf{Cardinality identity.}
  Taking cardinalities yields
  \[
      d^{\,e_1+e_2}
      \;=\;
      \bigl|D^{\,E_1\cup E_2}\bigr|
      \;=\;
      \bigl|D^{\,E_1}\times D^{\,E_2}\bigr|
      \;=\;
      d^{\,e_1}\cdot d^{\,e_2}.
  \]
  \(\square\)
  \end{solution}
%--------------------------------------------------------------------
%  Problem 7 – Large and small open sets in an infinite metric space
%--------------------------------------------------------------------

\begin{solution}
Let \((M,D)\) be an \emph{infinite} metric space.  
Show that there exists an open set \(U\subset M\) such that both
\(U\) and \(M\setminus U\) are infinite.

\bigskip

We distinguish two mutually exclusive cases, following the hint.

\medskip
\textbf{Case 1. \(M\) has an isolated point.}

Pick an isolated point \(p\in M\); hence
\(\{p\}=B(p,\varepsilon_{0})\) for some \(\varepsilon_{0}>0\).
Because \(M\) is infinite we can choose an infinite sequence of
\emph{distinct} points
\(
    \{q_{1},q_{2},\dots\}\subset M\setminus\{p\}.
\)
For each \(n\) set
\(
   \varepsilon_{n}:=\frac12D(q_{n},\{p,q_{1},\dots,q_{n-1}\})>0
\)
so that the open balls
\(B(q_{n},\varepsilon_{n})\) are pairwise disjoint and avoid \(p\).

Define
\[
   U:=\bigcup_{n\text{ even}} B(q_{n},\varepsilon_{n}).
\]
Then \(U\) is open and contains infinitely many \(q_{2k}\); its
complement contains \(p\) and the infinitely many points \(q_{2k-1}\):
\[
   |U|=\infty,\qquad |M\setminus U|=\infty.
\]

\medskip
\textbf{Case 2. \(M\) has no isolated points (every point is a
limit point).}

Fix any \(x_{0}\in M\).
Because \(x_{0}\) is a limit point, every ball \(B(x_{0},r)\) contains
infinitely many points of \(M\).
Pick radii \(r_{1}>r_{2}>\dots\to0\) such that the annuli
\[
   A_{n}:=B(x_{0},r_{n})\setminus\overline{B}(x_{0},r_{n+1})
\]
are all non–empty; choose a point \(x_{n}\in A_{n}\) for each \(n\).
Now set
\[
   U := \bigcup_{n\text{ even}} B\!\bigl(x_{n},\tfrac{r_{n}-r_{n+1}}{3}\bigr).
\]
The chosen radii make these balls pairwise disjoint, hence \(U\) is open
and infinite.
The balls with odd indices lie in \(M\setminus U\), so the complement is
also infinite.

\medskip
\textbf{Conclusion.}
In either case we have produced an open set \(U\subset M\) with
\[
   |U|=\infty
   \quad\text{and}\quad
   |M\setminus U|=\infty,
\]
as required.
\qed
\end{solution}

\begin{solution}
  \textbf{Problem 8.  Countable well–ordering implies well–ordering}
  
  Let \(C\) be a chain (i.e.\ a linearly ordered set) with the property:
  
  \begin{quote}
  \(\bigl(\star\bigr)\) \emph{Every countable subset of \(C\) is well‑ordered
  by the inherited order.}
  \end{quote}
  
  We prove that \(C\) itself is well‑ordered.
  
  \medskip
  \textbf{Recall.}
  A linear order is \emph{well‑ordered} iff every non‑empty subset has a
  least element.  Equivalently, a linear order fails to be well‑ordered
  iff it contains a strictly \emph{descending} infinite sequence
  \[
     x_0 > x_1 > x_2 > \dotsb
  \]
  (thus the subset \(\{x_n:n\in\mathbb N\}\) has no least element).
  
  \medskip
  \textbf{Contrapositive strategy.}
  Assume, for contradiction, that \(C\) is \emph{not} well‑ordered.
  Then there exists a non‑empty subset \(S\subseteq C\) that has no least
  element.  We show this leads to a countable subset of \(C\) that is
  not well‑ordered, contradicting \((\star)\).
  
  \smallskip
  \emph{Constructing a descending sequence.}
  Because \(S\) lacks a least element we can pick \(x_0\in S\).
  Inductively, having chosen \(x_n\in S\) (with \(n\ge 0\)), the set
  \(\{\,y\in S: y<x_n\}\) is non‑empty (otherwise \(x_n\) would be the
  least element of \(S\)).  Choose
  \[
     x_{n+1}\in S
     \quad\text{with}\quad
     x_{n+1}<x_n.
  \]
  By recursion on \(n\in\mathbb N\) (formally justified by the Axiom of
  Dependent Choice, though only countably many choices are needed) we
  obtain an infinite strictly descending sequence
  \[
     x_0 > x_1 > x_2 > \dotsb
  \]
  inside \(S\).
  
  \smallskip
  \emph{A bad countable subset.}
  Set \(X=\{x_n:n\in\mathbb N\}\).
  The set \(X\) is countable yet has no least element, because for any
  \(x_n\in X\) we have \(x_{n+1}<x_n\).
  Thus \(X\) is \emph{not} well‑ordered, contradicting property
  \((\star)\).
  
  \medskip
  \textbf{Conclusion.}
  Our assumption that \(C\) is not well‑ordered is untenable; hence
  \(C\) must be well‑ordered.
  \(\square\)
  \end{solution}
  
\end{document}
