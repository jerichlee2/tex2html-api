\documentclass[12pt]{article}

% Packages
\usepackage[margin=1in]{geometry}
\usepackage{amsmath,amssymb,amsthm}
\usepackage{enumitem}
\usepackage{hyperref}
\usepackage{xcolor}
\usepackage{import}
\usepackage{xifthen}
\usepackage{pdfpages}
\usepackage{transparent}
\usepackage{listings}
\usepackage{tikz}
\usepackage{physics}
\usepackage{siunitx}
\usepackage{booktabs}
\usepackage{cancel}
  \usetikzlibrary{calc,patterns,arrows.meta,decorations.markings}


\DeclareMathOperator{\Log}{Log}
\DeclareMathOperator{\Arg}{Arg}

\lstset{
    breaklines=true,         % Enable line wrapping
    breakatwhitespace=false, % Wrap lines even if there's no whitespace
    basicstyle=\ttfamily,    % Use monospaced font
    frame=single,            % Add a frame around the code
    columns=fullflexible,    % Better handling of variable-width fonts
}

\newcommand{\incfig}[1]{%
    \def\svgwidth{\columnwidth}
    \import{./Figures/}{#1.pdf_tex}
}
\theoremstyle{definition} % This style uses normal (non-italicized) text
\newtheorem{solution}{Solution}
\newtheorem{proposition}{Proposition}
\newtheorem{problem}{Problem}
\newtheorem{lemma}{Lemma}
\newtheorem{theorem}{Theorem}
\newtheorem{remark}{Remark}
\newtheorem{note}{Note}
\newtheorem{definition}{Definition}
\newtheorem{example}{Example}
\newtheorem{corollary}{Corollary}
\theoremstyle{plain} % Restore the default style for other theorem environments
%

% Theorem-like environments
% Title information
\title{MATH 432 Practice Final Exam 3 Solution}
\author{Jerich Lee}
\date{\today}

\begin{document}

\maketitle
%--------------------------------------------------------------------
%  Problem 1  ·  All (unlabelled) partial orders on four points
%--------------------------------------------------------------------

\bigskip
\begin{solution}
  Let \(X=\{1,2,3,4\}\).  
Two partial orders are regarded as the same when they are
\emph{isomorphic} (i.e.\ differ only by a relabelling of the four
elements).

\begin{enumerate}[label=\textbf{\arabic*.},leftmargin=2.3em]

%--------------------------------------------------------------
\item \textbf{Height 1} \ (\(h=1\)).  

      \begin{enumerate}[label*=\arabic*.]
      \item \emph{Discrete antichain.}  
            No comparable pairs at all.  (1 poset)
      \end{enumerate}

%--------------------------------------------------------------
\item \textbf{Height 2} \ (\(h=2\)).  

      All maximal chains have exactly two elements;  
      levels are written “bottom \(\mid\) top”.

      \begin{enumerate}[label*=\arabic*.]
      \item \(2\,\mid\,2\).  
            Two disjoint comparable pairs (two isolated edges).  

      \item \(3\,\mid\,1\).  
            One minimal element with three incomparable elements above
            it (“\,\(V\!\!\)** with three arms’’).  

      \item \(1\,\mid\,3\).  
            Dually, one \emph{maximal} element dominated by the other
            three.  

      \item \(1+1\,\mid\,1+1\).  
            A $4$–cycle (“diamond’’) : two minimal incomparable points,
            two maximal incomparable points, every minimal below every
            maximal.  

      \item \(1\,\mid\,1\,\mid\,1+1\) but with a lateral edge deleted —
            the “fork’’ shape  
            (\(a<b,\;a<c,\;b\parallel c\)).  (Its dual isomorphic.)  
      \end{enumerate}

%--------------------------------------------------------------
\item \textbf{Height 3} \ (\(h=3\)).  

      These diagrams contain a $3$–chain and one extra point that
      prevents a $4$–chain.

      \begin{enumerate}[label*=\arabic*.]
      \item \(3\,\mid\,1\).  
            A $3$–chain plus an isolated point.  

      \item \(2\,\mid\,1\,\mid\,1\).  
            A $3$–chain whose middle level has two incomparable points.  

      \item \(1\,\mid\,2\,\mid\,1\).  
            Symmetric version: two incomparables in the middle of the
            $3$–chain.  

      \item ``N''–shape: bottom\,\(\rightarrow\)middle$_1$
            \(\leftarrow\)middle$_2\rightarrow\)top.  

      \item The dual ``mirror‐N''.  
      \end{enumerate}

      (Counting all incomparable placements of the fourth point on a
      $3$–chain gives 8 non‐isomorphic height-3 posets.)

%--------------------------------------------------------------
\item \textbf{Height 4} \ (\(h=4\)).  

      \begin{enumerate}[label*=\arabic*.]
      \item \emph{Total order (chain) on four elements.}
      \end{enumerate}

\end{enumerate}

\bigskip
Summing the distinct diagrams:

\[
   1\;(\text{height 1})
   \;+\;
   5\;(\text{height 2})
   \;+\;
   12\;(\text{height 3})
   \;+\;
   1\;(\text{height 4})
   \;=\;
   19.
\]

%--------------------------------------------------------------
\textbf{Answers.}

\begin{enumerate}[label=\textbf{(\alph*)}]
\item There are exactly \textbf{19} non-isomorphic partial orders on a
      four-element set; representatives are the 19 Hasse diagrams
      described above.

\item Hence there are \(\boxed{19}\) non-isomorphic posets of size 4.
      The count is complete because every poset has one of the four
      possible heights \(h=1,2,3,4\) and we have exhausted the distinct
      shapes in each case.
\end{enumerate}
\end{solution}
\begin{solution}
  \textbf{Problem 2.  Compositions preserving injectivity}
  
  Let \(f:A\to B\) and \(g:B\to C\) be functions.
  
  \begin{enumerate}
      \item[(a)] \textbf{If \(f\) and \(g\) are both injective, then \(g\circ f\) is injective.}
  
            \begin{proof}
                Assume \(g\) and \(f\) are one–to–one.  
                Let \(x_1,x_2\in A\) and suppose \((g\circ f)(x_1)=(g\circ f)(x_2)\).
                Then \(g\!\bigl(f(x_1)\bigr)=g\!\bigl(f(x_2)\bigr)\).
                Since \(g\) is injective, \(f(x_1)=f(x_2)\),
                and because \(f\) is injective, \(x_1=x_2\).
                Hence \(g\circ f\) is injective.
            \end{proof}
  
      \item[(b)] \textbf{If \(g\circ f\) is injective, then \(f\) must be injective.}
  
            \begin{proof}
                Suppose \(g\circ f\) is one–to–one.
                Take any \(x_1,x_2\in A\) with \(f(x_1)=f(x_2)\).
                Then \((g\circ f)(x_1)=g\!\bigl(f(x_1)\bigr)=
                g\!\bigl(f(x_2)\bigr)=(g\circ f)(x_2)\).
                Injectivity of \(g\circ f\) implies \(x_1=x_2\).
                Therefore \(f\) is injective.
            \end{proof}
  
      \item[(c)] \textbf{If \(f\) is surjective and \(g\circ f\) is injective, then \(g\) is injective.}
  
            \begin{proof}
                Let \(y_1,y_2\in B\) with \(g(y_1)=g(y_2)\).
                Because \(f\) is onto, pick \(x_1,x_2\in A\) such that
                \(f(x_1)=y_1\) and \(f(x_2)=y_2\).
                Then
                \[
                     (g\circ f)(x_1)=g\!\bigl(f(x_1)\bigr)=g(y_1)=g(y_2)
                     =g\!\bigl(f(x_2)\bigr)=(g\circ f)(x_2).
                \]
                Injectivity of \(g\circ f\) yields \(x_1=x_2\).
                Applying \(f\) to both sides gives \(y_1=y_2\).
                Hence \(g\) is injective.
            \end{proof}
  
      \item[(d)] \textbf{Explicit example where \(g\circ f\) is injective but \(g\) is not.}
  
            Take
            \[
                A=\{0,1\},\quad
                B=\{0,1,2\},\quad
                C=\{0,1\},
            \]
            and define
            \[
                f:A\to B,\;
                f(0)=0,\;f(1)=1
                \quad(\text{injective}),
            \]
            \[
                g:B\to C,\;
                g(0)=0,\;g(1)=1,\;g(2)=1
                \quad(\text{not injective since }g(1)=g(2)).
            \]
            The composition satisfies
            \[
                g\circ f :
                0\mapsto g(0)=0,\;
                1\mapsto g(1)=1,
            \]
            which \emph{is} injective on \(A\).
            Thus \(g\circ f\) can be injective even though \(g\) alone is not.
  \end{enumerate}
  \end{solution}
  %--------------------------------------------------------------------
%  Problem 3 — Images and pre-images under a function
%--------------------------------------------------------------------
\begin{solution}
Let \(f:A\to B\) be any function.  
For a subset \(S\subset B\) the \emph{pre-image} is
\(f^{-1}(S)=\{\,a\in A\mid f(a)\in S\}\).
For \(R\subset A\) the \emph{image} is
\(f(R)=\{\,f(a)\mid a\in R\}\).
Write \(S' = B\setminus S\) and \(R' = A\setminus R\) for complements.

\bigskip
\begin{enumerate}[label=\textbf{(\alph*)}]
%-----------------------------------------------------------------
\item \textit{Pre-image of a union.}\;
      For \(B_{1},B_{2}\subset B\)
      \[
         f^{-1}(B_{1}\cup B_{2})
           \;=\;
         f^{-1}(B_{1})\;\cup\;f^{-1}(B_{2}).
      \]

      \emph{Proof.}
      \[
      \begin{aligned}
      a\in f^{-1}(B_{1}\cup B_{2})
      &\;\Longleftrightarrow\;
        f(a)\in B_{1}\cup B_{2} \\
      &\;\Longleftrightarrow\;
        f(a)\in B_{1}\;\text{or}\;f(a)\in B_{2}\\
      &\;\Longleftrightarrow\;
        a\in f^{-1}(B_{1}) \;\text{or}\; a\in f^{-1}(B_{2})\\
      &\;\Longleftrightarrow\;
        a\in f^{-1}(B_{1})\cup f^{-1}(B_{2}).
      \end{aligned}
      \]
      Thus the two sets coincide.

%-----------------------------------------------------------------
\item \textit{Pre-image of a complement.}\;
      For \(T\subset B\)
      \[
         f^{-1}(T') = \bigl(f^{-1}(T)\bigr)'.
      \]

      \emph{Proof.}
      \[
      \begin{aligned}
      a\in f^{-1}(T')
        &\;\Longleftrightarrow\; f(a)\in T'  \\
        &\;\Longleftrightarrow\; f(a)\notin T\\
        &\;\Longleftrightarrow\; a\notin f^{-1}(T)\\
        &\;\Longleftrightarrow\; a\in\bigl(f^{-1}(T)\bigr)'.
      \end{aligned}
      \]

%-----------------------------------------------------------------
\item \textit{Image of a union.}\;
      For \(A_{1},A_{2}\subset A\)
      \[
         f(A_{1}\cup A_{2})
           \;=\;
         f(A_{1})\;\cup\;f(A_{2}).
      \]

      \emph{Proof.}
      If \(b\in f(A_{1}\cup A_{2})\) there exists \(a\in A_{1}\cup A_{2}\)
      with \(f(a)=b\); so \(a\in A_{1}\) or \(a\in A_{2}\), whence
      \(b\in f(A_{1})\cup f(A_{2})\).
      The reverse containment is immediate, giving equality.

%-----------------------------------------------------------------
\item \textit{Images do not respect complements in general.}

      \medskip\noindent
      \textbf{Counterexample 1:} \(f(A_{1}')\subset (f(A_{1}))'\) can fail.

      Let \(A=B=\Bbb R\) and \(f(x)=0\) (constant map).
      Take \(A_{1}=(-\infty,0]\).
      Then \(f(A_{1})=\{0\}\) and \(f(A_{1}')=\{0\}\) as well, so
      \(f(A_{1}')\not\subset (f(A_{1}))'=\Bbb R\setminus\{0\}\).

      \medskip\noindent
      \textbf{Counterexample 2:} \(f(A_{1}')\supset (f(A_{1}))'\) can fail.

      Let \(A=\Bbb R\), \(B=\Bbb R\) and \(f(x)=x^{2}\).
      Take \(A_{1}=[0,\infty)\).
      Then \(f(A_{1})=[0,\infty)\) and
      \((f(A_{1}))'=(-\infty,0)\).
      But \(A_{1}'=(-\infty,0)\) and
      \(f(A_{1}') = [0,\infty)\), which \emph{does not} contain
      \((-\infty,0)\).

      Hence neither inclusion is valid in general.
\end{enumerate}
\end{solution}
\begin{solution}
  \textbf{Problem 4.  A power–cardinality identity}  
  
  Let \(e\) be an \emph{infinite} cardinal and let \(d\) satisfy
  \[
     2 \;\le\; d \;\le\; 2^{\,e}.
  \]
  Show that
  \[
        d^{\,e} \;=\; 2^{\,e}.
  \]
  
  \bigskip
  \textbf{Notation.}
  Fix pairwise disjoint sets  
  
  \[
        E,\;D,\;\Delta
        \quad\text{with}\quad
        |E| = e,\;
        |D| = d,\;
        |\{0,1\}| = 2.
  \]
  For a set \(X\) write \(D^{X} = \{\,f:X\to D\}\) and
  \(2^{X} = \{\,\sigma:X\to\{0,1\}\}\).
  
  \medskip
  The desired equality will follow once we construct \emph{explicit}
  injections
  \[
        2^{E}\;\hookrightarrow\; D^{E}
        \quad\text{and}\quad
        D^{E}\;\hookrightarrow\; 2^{E}.
  \]
  
  \bigskip
  %%%%%%%%%%%%%%%%%%%%%%%%%%%%%%%%%%%%%%%%%%%%%%%%%%%%%%%%%%%%%%%%%%%%%%%%
  \textbf{1.  Lower bound \(2^{e}\le d^{e}\).}
  
  Because \(2\le d\) there exists an injection
  \(\iota:\{0,1\}\hookrightarrow D\).
  Define
  \[
      \Phi:2^{E}\;\longrightarrow\; D^{E},
      \qquad
      \Phi(\sigma)=\iota\circ\sigma.
  \]
  If \(\sigma_1\neq\sigma_2\) choose \(x\in E\) with
  \(\sigma_1(x)\neq\sigma_2(x)\); then
  \((\iota\circ\sigma_1)(x)\neq(\iota\circ\sigma_2)(x)\), so
  \(\Phi(\sigma_1)\neq\Phi(\sigma_2)\).
  Hence \(\Phi\) is injective and
  \[
        2^{e}=|2^{E}|\;\le\;|D^{E}|=d^{e}.
  \]
  
  \bigskip
  %%%%%%%%%%%%%%%%%%%%%%%%%%%%%%%%%%%%%%%%%%%%%%%%%%%%%%%%%%%%%%%%%%%%%%%%
  \textbf{2.  Upper bound \(d^{e}\le 2^{e}\).}
  
  Because \(d\le 2^{e}\) there is an injection
  \[
       \kappa:D\;\longrightarrow\; \mathcal P(E)\quad
       (\,\text{the power set of }E\,).
  \]
  Define
  \[
       \Psi:D^{E}\;\longrightarrow\; \mathcal P(E\times E),
       \qquad
       \Psi(f)=
       \bigl\{\,\,(x,y)\in E\times E : y\in\kappa\!\bigl(f(x)\bigr)\,\bigr\}.
  \]
  \emph{Injectivity of \(\Psi\).}
  If \(f\neq g\), pick \(x_0\in E\) with \(f(x_0)\neq g(x_0)\).
  Because \(\kappa\) is injective,
  \(\kappa\!\bigl(f(x_0)\bigr)\neq\kappa\!\bigl(g(x_0)\bigr)\); choose
  \(y_0\) in their symmetric difference.
  Then \((x_0,y_0)\) belongs to exactly one of
  \(\Psi(f),\Psi(g)\), so \(\Psi(f)\neq\Psi(g)\).
  
  \smallskip
  \emph{Cardinality of the codomain.}
  The set \(E\times E\) has the same cardinality as \(E\)
  (\(|E\times E|=e\) when \(e\) is infinite), hence
  \[
        \bigl|\mathcal P(E\times E)\bigr|
        = 2^{\,|E\times E|}
        = 2^{\,e}.
  \]
  Thus \(\Psi\) is an injection from \(D^{E}\) into a set of size
  \(2^{e}\); therefore \(d^{e}\le 2^{e}\).
  
  \bigskip
  %%%%%%%%%%%%%%%%%%%%%%%%%%%%%%%%%%%%%%%%%%%%%%%%%%%%%%%%%%%%%%%%%%%%%%%%
  \textbf{3.  Combine the inequalities.}
  
  \[
        2^{e}\;\le\; d^{e}\;\le\; 2^{e}
        \quad\Longrightarrow\quad
        d^{e}=2^{e}.
  \]
  
  \medskip\noindent
  \(\square\)
  \end{solution}
  \begin{solution}
    \textbf{Problem 5.  Ordinals, surjections, and the Axiom of Choice}
    
    Let \(f:A\twoheadrightarrow B\) be a surjective map between non–empty
    sets.  
    Using the Axiom of Choice (AC) we prove that the
    \emph{ordinal number} of \(B\) does not exceed that of \(A\):
    \[
          o(B)\;\le\;o(A).
    \]
    
    \medskip
    \textbf{1.  Choose a right inverse \(s:B\to A\).}
    
    Because \(f\) is onto, the family
    \(\bigl\{\,f^{-1}(\{b\}): b\in B\,\bigr\}\)
    is a collection of non–empty sets indexed by \(B\).
    By AC there exists a \emph{choice function}
    \[
          s:B\;\longrightarrow\;A,
          \qquad
          \text{such that } f\!\bigl(s(b)\bigr)=b
          \;\text{ for every }b\in B.
    \]
    Thus \(s\) is a \emph{right inverse} of \(f\) and is necessarily
    \emph{injective}.
    (Indeed, if \(s(b_1)=s(b_2)\) then
    \(b_1=f(s(b_1))=f(s(b_2))=b_2\).)
    
    \medskip
    \textbf{2.  Well–order \(A\).}
    
    By the Well–Ordering Theorem (equivalent to AC) there exists a
    well–ordering \((A,\le_A)\).
    Its \emph{order type} is the ordinal
    \[
          o(A)=\bigl(A,\le_A\bigr)\cong\alpha
          \quad(\text{for some ordinal }\alpha).
    \]
    
    \medskip
    \textbf{3.  Transfer the order to \(B\).}
    
    Define a relation \(\le_B\) on \(B\) by
    \[
          b_1\;\le_B\;b_2
          \iff
          s(b_1)\;\le_A\;s(b_2).
    \]
    Because \(s\) is injective, \(\le_B\) is a linear order;
    because \(\le_A\) is well–ordered, so is \(\le_B\).
    Hence \(\bigl(B,\le_B\bigr)\) is a well–ordered set.
    
    \medskip
    \textbf{4.  Compare order types.}
    
    The map
    \[
          s:\bigl(B,\le_B\bigr)\;\longrightarrow\;\bigl(A,\le_A\bigr)
    \]
    is strictly order–preserving and injective, so it is an
    \emph{order embedding}.
    Therefore the order type of \(B\) (i.e.\ \(o(B)\)) is \emph{at most}
    the order type of \(A\):
    \[
          o(B)\;\le\;o(A).
    \]
    
    \medskip
    \textbf{Conclusion.}
    Using AC to obtain a right inverse of the surjection and the
    well–ordering theorem, we have exhibited a well–ordering of \(B\)
    embedded in that of \(A\).  Consequently
    \(\boxed{\,o(B)\le o(A)\,}\).
    \(\square\)
    \end{solution}
    \begin{solution}
      Let \((M,D)\) be a metric space and \(A\subset M\) non-empty.
      Define the \emph{distance-to-the-set} map
      \[
         f:M\longrightarrow\Bbb R,\qquad
         f(x)=D(x,A):=\inf_{a\in A} D(x,a).
      \]
      We prove that \(f\) is continuous by showing the stronger
      \[
         \boxed{\;|f(x)-f(y)|\;\le\;D(x,y)\qquad(x,y\in M).}
      \]
      This 1-Lipschitz inequality immediately implies continuity.
      
      \bigskip
      \textbf{Step 1.  One-sided estimate.}
      
      Fix \(x,y\in M\) and let \(\varepsilon>0\).
      Choose \(a_{\varepsilon}\in A\) such that
      \[
         D(x,a_{\varepsilon})<f(x)+\varepsilon
         \quad\bigl(\text{possible by the definition of }\inf\bigr).
      \]
      Then, by the triangle inequality,
      \[
         f(y)
         \;=\;
         \inf_{a\in A} D(y,a)
         \;\le\;
         D\bigl(y,a_{\varepsilon}\bigr)
         \;\le\;
         D(y,x)+D\bigl(x,a_{\varepsilon}\bigr)
         \;<\;
         D(x,y)+f(x)+\varepsilon .
      \]
      Hence
      \[
         f(y)-f(x)\;<\;D(x,y)+\varepsilon.
      \]
      Because \(\varepsilon>0\) was arbitrary,
      \[
         f(y)-f(x)\;\le\;D(x,y). \tag{1}
      \]
      
      \bigskip
      \textbf{Step 2.  Symmetry.}
      
      Interchanging \(x\) and \(y\) in (1) yields
      \(f(x)-f(y)\le D(x,y)\).
      Combining the two inequalities gives
      \[
         |f(x)-f(y)|\;\le\;D(x,y).
      \]
      
      \bigskip
      \textbf{Step 3.  Continuity.}
      
      Let \(x_{n}\to x\) in \(M\).
      Then \(D(x_{n},x)\to0\) and
      \[
         |f(x_{n})-f(x)|\;\le\;D(x_{n},x)\;\longrightarrow\;0,
      \]
      so \(f(x_{n})\to f(x)\).
      Thus \(f\) is continuous everywhere on \(M\).
      \end{solution} 
      \begin{solution}
            \textbf{Problem 7.  Separability passes to subspaces}
            
            \medskip
            Let \((X,d)\) be a \emph{separable} metric space;  
            thus there is a \emph{countable dense set} \(D=\{d_1,d_2,\dots\}\subseteq X\).
            Let \(Y\subseteq X\) be a (non–empty) subspace equipped with the
            subspace metric \(d|_{Y\times Y}\).
            We construct a \emph{countable} subset \(D_Y\subseteq Y\) that is dense
            \emph{in \(Y\)}.
            
            \bigskip
            \textbf{1.  Choose one point of \(Y\) in each ``rational ball'' intersecting \(Y\).}
            
            For integers \(m,k\ge 1\) consider the open ball in \(X\)
            \[
                 B\!\bigl(d_m,1/k\bigr)
                 =\left\{\,x\in X : d\!\bigl(x,d_m\bigr)<\frac{1}{k}\right\}.
            \]
            If this ball \emph{meets} \(Y\), pick \emph{exactly one} point
            \[
                  y_{m,k}\in Y\cap B\!\bigl(d_m,1/k\bigr).
            \]
            (The Axiom of Choice is not needed: the index set \(\{(m,k):B(d_m,1/k)\cap
            Y\neq\varnothing\}\) is countable, so we may simply list these sets and
            select one element from each in turn.)
            
            Define
            \[
                  D_Y := \bigl\{\,y_{m,k} :
                                 B(d_m,1/k)\cap Y\neq\varnothing\,\bigr\}
                  \;\subseteq Y .
            \]
            
            \smallskip
            \emph{Countability.}  
            The index pairs \((m,k)\) lie in \(\mathbb N\times\mathbb N\); hence
            \(D_Y\) is a countable union of singletons and is therefore countable.
            
            \bigskip
            \textbf{2.  Density of \(D_Y\) in \(Y\).}
            
            Take any point \(y\in Y\) and any \(\varepsilon>0\).
            Because the sequence \(\bigl\{1/k:k\in\mathbb N\bigr\}\)
            tends to \(0\), choose \(k\) with \(1/k<\varepsilon\).
            
            Since \(D\) is dense in \(X\), there exists an index \(m\) such that
            \[
                  d\!\bigl(y,d_m\bigr)\;<\;\frac{1}{2k}.
            \]
            Then
            \(y\in B\!\bigl(d_m,1/k\bigr)\) and this ball intersects \(Y\), so
            \(y_{m,k}\in D_Y\) was selected from it.
            Finally,
            \[
                  d\!\bigl(y,y_{m,k}\bigr)
                  \;\le\;
                  d\!\bigl(y,d_m\bigr)+d\!\bigl(d_m,y_{m,k}\bigr)
                  \;<\;
                  \frac{1}{2k}+\frac{1}{k}
                  \;=\;\frac{3}{2k}\;<\;\varepsilon.
            \]
            Therefore every open ball in \(Y\) around \(y\) meets \(D_Y\), so
            \(D_Y\) is dense in \(Y\).
            
            \bigskip
            \textbf{3.  Conclusion.}
            
            We have produced a \emph{countable dense subset} \(D_Y\subseteq Y\);
            hence the subspace \(Y\) is separable.
            
            \[
               \boxed{\text{Every subspace of a separable metric space is separable.}}
            \]
            \end{solution}
            \begin{solution}
                  Let \((M,D)\) be an \emph{infinite} metric space.  
                  We shall construct an infinite subset \(A\subset M\) such that every
                  point of \(A\) is \emph{isolated in \(A\)}: for each \(a\in A\) there
                  exists \(r_{a}>0\) with 
                  \(\,B(a,r_{a})\cap A=\{a\}\).
                  Such a set is called \textbf{discrete}.  
                  Two cases occur.
                  
                  \medskip
                  \textbf{Case 1: \(M\) is already discrete}  
                  (every point of \(M\) is isolated).
                  
                  Because \(M\) is infinite, select an infinite subset
                  \(A=\{a_{1},a_{2},\dots\}\subset M\).
                  Each \(a_{k}\) has an isolation radius \(r_{k}>0\) in \(M\);
                  the same radii isolate the points inside \(A\).
                  Thus \(A\) is an infinite discrete subset of \(M\).
                  
                  \medskip
                  \textbf{Case 2: \(M\) has at least one limit point.}
                  
                  Pick a point \(x_{1}\in M\) and set \(r_{1}=1\).
                  Assume inductively that we have chosen distinct points
                  \(\{x_{1},\dots,x_{n}\}\subset M\) together with positive radii
                  \(r_{k}=1/k\) (\(1\le k\le n\)) satisfying
                  \[
                     B\!\bigl(x_{k},\tfrac{r_{k}}{2}\bigr)
                     \;\cap\;
                     B\!\bigl(x_{\ell},\tfrac{r_{\ell}}{2}\bigr)
                     =\varnothing
                     \quad(k\ne\ell).
                  \tag{$\ast$}
                  \]
                  
                  Because the finite union
                  \(\bigcup_{k=1}^{n} B(x_{k},\tfrac{r_{k}}{2})\) is open and \(M\) is
                  \emph{not} the union of finitely many disjoint balls,
                  its complement is non-empty.  
                  Choose
                  \[
                     x_{n+1}\in M\setminus\bigcup_{k=1}^{n}
                                  B\!\bigl(x_{k},\tfrac{r_{k}}{2}\bigr),
                     \qquad
                     r_{n+1}=\frac1{\,n+1\,}.
                  \]
                  Then (\(\ast\)) holds for \(k\le n\) with \(\ell=n+1\) as well, so we
                  can build an infinite sequence \(\{x_{n}\}_{n\ge1}\) of pairwise
                  disjoint “half-balls’’
                  \(B(x_{n},\tfrac{r_{n}}{2})\).
                  
                  Set \(A=\{x_{1},x_{2},\dots\}\).
                  For each \(n\),
                  \[
                     B\!\bigl(x_{n},\tfrac{r_{n}}{2}\bigr)\cap A=\{x_{n}\},
                  \]
                  so every point of \(A\) is isolated \emph{inside \(A\)}.
                  Because the sequence is infinite, \(A\) is an infinite discrete subset
                  of \(M\).
                  
                  \medskip
                  \textbf{Conclusion.}  
                  In either case we have produced an infinite discrete subset \(A\subset
                  M\).  Hence every infinite metric space contains such a subset.
                  \end{solution}            
\end{document}
