\documentclass[12pt]{article}

% Packages
\usepackage[margin=1in]{geometry}
\usepackage{amsmath,amssymb,amsthm}
\usepackage{enumitem}
\usepackage{hyperref}
\usepackage{xcolor}
\usepackage{import}
\usepackage{xifthen}
\usepackage{pdfpages}
\usepackage{transparent}
\usepackage{listings}
\DeclareMathOperator{\Log}{Log}
\DeclareMathOperator{\Arg}{Arg}

\lstset{
    breaklines=true,         % Enable line wrapping
    breakatwhitespace=false, % Wrap lines even if there's no whitespace
    basicstyle=\ttfamily,    % Use monospaced font
    frame=single,            % Add a frame around the code
    columns=fullflexible,    % Better handling of variable-width fonts
}

\newcommand{\incfig}[1]{%
    \def\svgwidth{\columnwidth}
    \import{./Figures/}{#1.pdf_tex}
}
\theoremstyle{definition} % This style uses normal (non-italicized) text
\newtheorem{solution}{Solution}
\newtheorem{proposition}{Proposition}
\newtheorem{problem}{Problem}
\newtheorem{lemma}{Lemma}
\newtheorem{theorem}{Theorem}
\newtheorem{remark}{Remark}
\newtheorem{note}{Note}
\newtheorem{definition}{Definition}
\newtheorem{example}{Example}
\newtheorem{corollary}{Corollary}
\theoremstyle{plain} % Restore the default style for other theorem environments
%

% Theorem-like environments
% Title information
\title{HW 7: MATH 432}
\author{Jerich Lee}
\date{\today}

\begin{document}

\maketitle
\begin{problem}[1]
Let $(P, \leq)$ be a partially ordered set, and let $C \subseteq P$ be a chain (i.e.\ a totally ordered subset). We want to prove that there exists a \emph{maximal chain} $M$ in $P$ such that $C \subseteq M$. Recall that a chain $M$ is called \emph{maximal} if there is no strictly larger chain that properly contains it.
\end{problem}
\begin{solution}
    


\medskip

\textbf{Proof (via Zorn's Lemma).}

\begin{enumerate}
    \item[\textbf{Step 1.}] \textbf{Define the collection of all chains containing $C$.}\\
    Let 
    \[
        S = \{\,D \subseteq P : D \text{ is a chain and } C \subseteq D\,\}.
    \]
    In other words, $S$ is the set of all chains in $P$ that contain the given chain $C$. We will show that every chain in $S$ (under set inclusion) has an upper bound in $S$, and then apply Zorn's Lemma to conclude the existence of a maximal element in $S$.

    \item[\textbf{Step 2.}] \textbf{Partially order $S$ by inclusion.}\\
    We consider the partial order $\subseteq$ on $S$. That is, for any $D_1, D_2 \in S$, we say $D_1 \leq D_2$ if and only if $D_1 \subseteq D_2$. 

    \item[\textbf{Step 3.}] \textbf{Show that every chain in $(S, \subseteq)$ has an upper bound in $S$.}\\
    Let $\{D_i\}_{i \in I}$ be a chain in $S$ with respect to set inclusion. This means that for any $i, j \in I$, either $D_i \subseteq D_j$ or $D_j \subseteq D_i$. We claim that
    \[
        D^* := \bigcup_{i \in I} D_i
    \]
    is an upper bound for this family in $S$. We need to check two things:
    \begin{enumerate}
        \item $D^*$ is indeed a chain in $P$.\\
              Since each $D_i$ is a chain and the family $\{D_i\}_{i \in I}$ is totally ordered by inclusion, any two elements $x,y \in D^*$ must come from some $D_i$ or $D_j$ with $D_i \subseteq D_j$ or $D_j \subseteq D_i$. Hence $x$ and $y$ lie in at least one common chain $D_k$, and so they are comparable in $D_k$. Thus $D^*$ is a chain.
        \item $C \subseteq D^*$.\\
              Each $D_i$ in our family contains $C$. Therefore $C$ is contained in the union of all $D_i$, i.e.\ $C \subseteq D^*$. 
    \end{enumerate}
    Consequently, $D^*$ belongs to $S$ (it is a chain containing $C$) and $D^*$ is clearly an upper bound for the chain $\{D_i\}_{i \in I}$ in $(S, \subseteq)$.

    \item[\textbf{Step 4.}] \textbf{Apply Zorn's Lemma.}\\
    By Zorn's Lemma (which applies because every chain in $S$ has an upper bound in $S$), the poset $(S, \subseteq)$ has a maximal element. Denote this maximal element by $M \in S$.

    \item[\textbf{Step 5.}] \textbf{Conclude that $M$ is a maximal chain containing $C$.}\\
    By definition of $S$, we have $C \subseteq M$, and $M$ is a chain in $P$. The maximality of $M$ in $S$ means that there is no strictly larger chain in $S$ containing $M$. In other words, there is no chain $M' \supsetneq M$ with $C \subseteq M'$. Hence $M$ is a \emph{maximal chain} in $P$ and it contains $C$, as required.
\end{enumerate}
\end{solution}
\begin{problem}[2]
Let $X$ be a set, and let $f: X \to X$ be any function. We want to show that there exists a function $g: X \to X$ such that

\end{problem}
\begin{solution}

\[
   f \circ g \circ f \;=\; f,
\]
i.e.\ for every $x \in X$,
\[
   f\bigl(g\bigl(f(x)\bigr)\bigr) \;=\; f(x).
\]

\medskip

\textbf{Idea of the proof.} 
We want $f(g(y)) = y$ for every $y$ in the image of $f$. If we can define $g(y)$ so that $f(g(y)) = y$ whenever $y$ is in the range of $f$, then $f(g(f(x))) = f(x)$ automatically follows for all $x \in X$. 

\medskip

\textbf{Step 1. Identify the image of $f$.} 
Define
\[
   Y \;:=\; \{\, f(x) : x \in X \} \;\subseteq\; X.
\]
This is the \emph{range} (or image) of $f$.

\medskip

\textbf{Step 2. Use the Axiom of Choice to pick a preimage for each element of $Y$.} 
For each $y \in Y$, the set $f^{-1}(\{y\}) = \{\, x \in X : f(x) = y \}$ is nonempty by definition of $Y$. By the Axiom of Choice, there is a function 
\[
   h : Y \to X
\]
such that for each $y \in Y$, we have $f(h(y)) = y$. (This means $h(y)$ is a chosen element from the preimage of $y$ under $f$.)

\medskip

\textbf{Step 3. Define $g$ on $X$.} 
We now define a function $g : X \to X$ by specifying its values in two cases:
\begin{itemize}
    \item If $y \in Y$ (i.e.\ $y$ is in the image of $f$), set
    \[
       g(y) := h(y).
    \]
    \item If $y \notin Y$, we can define $g(y)$ arbitrarily (for example, pick any fixed element $x_0 \in X$ and set $g(y) = x_0$). 
\end{itemize}
This defines $g$ on all of $X$.

\medskip

\textbf{Step 4. Check that $f \circ g \circ f = f$.} 
Take any $x \in X$. Let $y = f(x)$. Then $y \in Y$ by definition. By our construction of $g$,
\[
   g(y) = g\bigl(f(x)\bigr) = h\bigl(f(x)\bigr).
\]
Hence,
\[
   f\bigl(g(f(x))\bigr) \;=\; f\bigl(h(f(x))\bigr).
\]
But by the defining property of $h$, we have $f(h(z)) = z$ for every $z \in Y$. Here, $z = f(x)$. Thus
\[
   f(h(f(x))) = f(x).
\]
Therefore,
\[
   f\bigl(g(f(x))\bigr) = f(x),
\]
which shows exactly that
\[
   (f \circ g \circ f)(x) = f(x)
\]
for all $x \in X$. Consequently,
\[
   f \circ g \circ f \;=\; f.
\]

\medskip

\textbf{Conclusion.} We have constructed a function $g: X \to X$ such that $f \circ g \circ f = f$, as required. The Axiom of Choice is used critically in choosing, for each $y$ in the image of $f$, a single $x$ such that $f(x) = y$.

\end{solution}
\begin{problem}[3]
    
\end{problem}
\begin{solution}

\textbf{Given:} 
\begin{itemize}
    \item[(a)] $D(a,a) = 0$ for all $a \in M$.
    \item[(b)] For any $a,b \in M$ with $a \neq b$, we have $D(a,b) \neq 0$.
    \item[(c)] For any $a,b,c \in M$, $D(a,b) + D(b,c) \geq D(c,a)$.
\end{itemize}

We wish to show that $D$ satisfies the axioms of a metric on $M$. Recall that to be a metric, $D$ must satisfy, for all $a,b,c \in M$:

\begin{enumerate}
    \item $D(a,b) \geq 0$.
    \item $D(a,b) = 0 \iff a = b$.
    \item $D(a,b) = D(b,a)$ (symmetry).
    \item $D(a,b) + D(b,c) \geq D(a,c)$ (triangle inequality).
\end{enumerate}

\medskip

\noindent \textbf{Step 1: Positivity and the condition $D(a,a)=0$.}

\begin{itemize}
    \item From (a), we know that for every $a \in M$, $D(a,a) = 0$.
    \item From (b), if $a \neq b$, then $D(a,b) \neq 0$. 
\end{itemize}
It remains to check that $D(a,b)$ is nonnegative. To see this, we use (c) in a special case:
\[
    D(a,b) + D(b,a) \;\ge\; D(a,a).
\]
But $D(a,a) = 0$ by (a). Hence $D(a,b) + D(b,a) \ge 0$. 
\emph{(We will soon show $D(a,b) = D(b,a)$, which then implies $D(a,b) \ge 0$.)}

\medskip

\noindent \textbf{Step 2: Symmetry, $D(a,b) = D(b,a)$.}

We claim $D$ is symmetric, i.e.\ $D(a,b) = D(b,a)$ for all $a,b \in M$. 

\begin{itemize}
    \item \emph{First inequality:} Set $b = a$ in (c):
    \[
        D(a,a) + D(a,c) \;\ge\; D(c,a).
    \]
    Since $D(a,a) = 0$, it follows that $D(a,c) \ge D(c,a)$.

    \item \emph{Second inequality:} Rename the variables in (c) by swapping the roles of $a$ and $c$. In other words, replace $(a,b,c)$ by $(c,b,a)$:
    \[
        D(c,b) + D(b,a) \;\ge\; D(a,c).
    \]
    Now set $b = c$ in this rewritten form:
    \[
        D(c,c) + D(c,a) \;\ge\; D(a,c).
    \]
    Since $D(c,c) = 0$, we get $D(c,a) \ge D(a,c)$.

\end{itemize}

Combining $D(a,c) \ge D(c,a)$ and $D(c,a) \ge D(a,c)$ gives 
\[
    D(a,c) \;=\; D(c,a).
\]
Hence $D$ is symmetric.

\medskip

\noindent \textbf{Step 3: Nonnegativity and uniqueness of zero distance.}

Since $D(a,b) = D(b,a)$, the inequality $D(a,b) + D(b,a) \ge 0$ becomes
\[
   2\,D(a,b) \;\ge\; 0 \quad\Longrightarrow\quad D(a,b) \;\ge\; 0.
\]
Moreover, if $a \neq b$, then by (b) we know $D(a,b) \neq 0$. Hence $D(a,b) > 0$ whenever $a \neq b$, and $D(a,a) = 0$.

Thus the usual metric conditions
\[
   D(a,b) \ge 0
   \quad\text{and}\quad
   D(a,b) = 0 \;\;\Leftrightarrow\;\; a = b
\]
are satisfied.

\medskip

\noindent \textbf{Step 4: Triangle inequality in the usual form.}

We need to show $D(a,c) \le D(a,b) + D(b,c)$ for all $a,b,c \in M$. 

Given (c), we have $D(a,b) + D(b,c) \ge D(c,a)$. By symmetry (just proved), $D(c,a) = D(a,c)$. Hence
\[
   D(a,b) + D(b,c) \;\ge\; D(a,c),
\]
which is precisely the usual triangle inequality.

\medskip

\noindent \textbf{Conclusion.} All four metric axioms are verified:

\begin{enumerate}
    \item $D(a,b) \ge 0$ and $D(a,b) = 0 \iff a = b$,
    \item $D(a,b) = D(b,a)$,
    \item $D(a,b) + D(b,c) \ge D(a,c)$.
\end{enumerate}

Therefore, $D$ is indeed a metric on $M$.

\end{solution}
\begin{problem}[4]
    
\end{problem}
\begin{solution}
    \section*{Problem Statement}
Let $M = \mathbb{R}$ with the usual metric $D(a,b) = |a - b|$. Fix some point $\alpha_0 \in \mathbb{R}$. Show that there are \emph{exactly two} isometries $f: \mathbb{R} \to \mathbb{R}$ (i.e.\ distance-preserving maps) such that $f(\alpha_0) = \alpha_0$.

\medskip

\textbf{Key idea:} Any distance-preserving map $f$ on the real line must be either of the form $x \mapsto x + c$ or $x \mapsto -x + c$, i.e.\ a translation or a reflection (possibly combined with a translation). The condition $f(\alpha_0) = \alpha_0$ forces $c$ to be chosen so that $\alpha_0$ is fixed, which leaves exactly two possibilities: the identity map or the reflection around $\alpha_0$.

\section*{Step-by-step Solution}

\subsection*{Step 1. Isometries of $\mathbb{R}$ have the form $x \mapsto ax + b$ with $|a| = 1$.}

First, recall that an isometry $f : \mathbb{R} \to \mathbb{R}$ must satisfy
\[
   |f(x) - f(y)| \;=\; |x - y|
   \quad\text{for all } x,y \in \mathbb{R}.
\]
One can show (either by an elementary argument or using more advanced classification theorems) that every such $f$ on the real line is of the form
\[
   f(x) \;=\; ax + b,
   \quad
   \text{where } a = \pm 1 \text{ and } b \in \mathbb{R}.
\]
The condition $|a|=1$ (that is, $a = \pm 1$) is necessary for the map to preserve distances. Specifically:
\begin{itemize}
    \item If $a = 1$, then $f(x) = x + b$ is a \emph{translation} by $b$.
    \item If $a = -1$, then $f(x) = -x + b$ is a \emph{reflection} (possibly combined with a translation).
\end{itemize}

\subsection*{Step 2. Impose the condition $f(\alpha_0) = \alpha_0$.}

We are given that $f$ fixes a specific point $\alpha_0$. Hence,
\[
   f(\alpha_0) = \alpha_0.
\]
In the form $f(x) = ax + b$, this means
\[
   a\,\alpha_0 + b = \alpha_0.
\]
Solving for $b$, we get
\[
   b = \alpha_0 - a\,\alpha_0 \;=\; \alpha_0(1 - a).
\]
Since $a$ can be either $+1$ or $-1$, we have two cases:

\begin{itemize}
    \item \textbf{Case 1:} $a = 1$. Then $b = \alpha_0(1 - 1) = 0$.\\
          Thus $f(x) = x + 0 = x$. This is the \emph{identity map}, which clearly fixes $\alpha_0$.

    \item \textbf{Case 2:} $a = -1$. Then $b = \alpha_0(1 - (-1)) = \alpha_0(2) = 2\alpha_0$.\\
          Thus $f(x) = -x + 2\alpha_0 = 2\alpha_0 - x$. This is the \emph{reflection} around the point $\alpha_0$ (i.e.\ it sends $\alpha_0$ to itself and flips the real line across that point).
\end{itemize}

\subsection*{Step 3. Conclusion: Exactly two isometries fix $\alpha_0$.}

From the above two cases, the only possibilities for an isometry on $\mathbb{R}$ that keeps $\alpha_0$ fixed are:
\[
   f(x) = x
   \quad\text{or}\quad
   f(x) = 2\alpha_0 - x.
\]
These are indeed two different maps (the identity and the reflection), and no other isometry can fix $\alpha_0$. Hence there are \emph{exactly two} such isometries.

\bigskip
\noindent
\textbf{Alternative Elementary Argument (Without Explicit Form):}

One can also argue directly from the metric condition:
\begin{enumerate}
    \item Since $f$ is an isometry, for any $x \in \mathbb{R}$,
          \[
             |f(x) - f(\alpha_0)| \;=\; |x - \alpha_0|.
          \]
          But $f(\alpha_0) = \alpha_0$, so $|f(x) - \alpha_0| = |x - \alpha_0|$.
    \item Thus $f(x)$ is one of the two points at distance $|x - \alpha_0|$ from $\alpha_0$, namely $x$ or $2\alpha_0 - x$.
    \item To preserve distances \emph{between all points}, $f$ cannot ``mix'' these assignments for different $x$. Either $f$ always sends $x$ to $x$ (the identity) or it always sends $x$ to $2\alpha_0 - x$ (the reflection).
\end{enumerate}
This again yields exactly two isometries fixing $\alpha_0$.

\end{solution}
\end{document}
