\documentclass[12pt]{article}

% Packages
\usepackage[margin=1in]{geometry}
\usepackage{amsmath,amssymb,amsthm}
\usepackage{enumitem}
\usepackage{hyperref}
\usepackage{xcolor}
\usepackage{import}
\usepackage{xifthen}
\usepackage{pdfpages}
\usepackage{transparent}
\usepackage{listings}
\usepackage{tikz}
\usepackage{physics}
\usepackage{siunitx}
\usepackage{booktabs}
\usepackage{cancel}
  \usetikzlibrary{calc,patterns,arrows.meta,decorations.markings}


\DeclareMathOperator{\Log}{Log}
\DeclareMathOperator{\Arg}{Arg}
\DeclareMathOperator{\card}{Card}

\lstset{
    breaklines=true,         % Enable line wrapping
    breakatwhitespace=false, % Wrap lines even if there's no whitespace
    basicstyle=\ttfamily,    % Use monospaced font
    frame=single,            % Add a frame around the code
    columns=fullflexible,    % Better handling of variable-width fonts
}

\newcommand{\incfig}[1]{%
    \def\svgwidth{\columnwidth}
    \import{./Figures/}{#1.pdf_tex}
}
\theoremstyle{definition} % This style uses normal (non-italicized) text
\newtheorem{solution}{Solution}
\newtheorem{proposition}{Proposition}
\newtheorem{problem}{Problem}
\newtheorem{lemma}{Lemma}
\newtheorem{theorem}{Theorem}
\newtheorem{remark}{Remark}
\newtheorem{note}{Note}
\newtheorem{definition}{Definition}
\newtheorem{example}{Example}
\newtheorem{corollary}{Corollary}
\theoremstyle{plain} % Restore the default style for other theorem environments
%

% Theorem-like environments
% Title information
\title{}
\author{Jerich Lee}
\date{\today}

\begin{document}

\maketitle
\begin{theorem}[Cantor–Schröder–Bernstein]
  \label{thm:CSB}
  Let $A$ and $B$ be sets.
  If there exist \emph{injective} maps
  \[
     f : A \longrightarrow B
     \quad\text{and}\quad
     g : B \longrightarrow A,
  \]
  then there exists a \emph{bijection}
  \[
     h : A \longrightarrow B .
  \]
  Equivalently, the existence of injections both ways implies that
  $A$ and $B$ have the same cardinality.
  \end{theorem}
  
  \begin{proof}[Idea of the proof (sketch)]
  Partition each set into two parts—one that will be matched by
  $f$ and one by $g$—using the following construction.
  
  \smallskip
  1.  Define a subset of $A$ by transfinite recursion:
      \[
         A_0 := A\setminus g(B),
         \qquad
         A_{n+1} := g\!\bigl(f(A_n)\bigr)\quad(n\ge 0),
         \qquad
         A_\infty := \bigcup_{n\ge 0} A_n .
      \]
      Elements of $A_\infty$ never re–enter the image of $g$ after finitely
      many alternations of $f$ and $g$.
  
  \smallskip
  2.  Set
      \[
          h(a)=
          \begin{cases}
            f(a), & a\in A_\infty,\\[6pt]
            g^{-1}(a), & a\notin A_\infty.
          \end{cases}
      \]
      One proves that $h$ is well‑defined, injective, and surjective,
      hence a bijection.
  
  \smallskip
  Detailed proofs can be presented via graph decompositions,
  equivalence classes generated by $f$ and $g$, or using
  order‑theoretic fixed‑point arguments; all yield the same bijection.
  \end{proof}
  \begin{definition}[Totally unordered subset (antichain)]
    Let $(P,\le)$ be a partially ordered set.
    A subset $A\subseteq P$ is called \textbf{totally unordered}
    (also: \emph{totally incomparable} or an \emph{antichain})
    if
    \[
       \forall\,x,y\in A,\;
       x\neq y
       \;\Longrightarrow\;
       \bigl(x\not\le y \;\text{ and }\; y\not\le x\bigr).
    \]
    In words, \emph{no two distinct elements of $A$ are comparable
    under the order $\le$}.
    \end{definition}
    
    \begin{remark}
    \begin{itemize}
      \item A totally unordered set is the opposite extreme of a
            \emph{totally ordered set}, where every pair of elements
            \emph{is} comparable.
      \item In many texts, “totally unordered subset” is synonymous with
            \textbf{antichain}.
            The entire poset $(P,\le)$ is itself totally unordered precisely
            when $\le$ is the identity relation ($x\le y\!\iff\!x=y$).
      \item Typical examples:
            \begin{enumerate}
              \item In $(\mathcal P(\mathbb N),\subseteq)$,
                    the family of singletons
                    $\{\{n\}\mid n\in\mathbb N\}$ is an antichain,
                    since no singleton contains another.
              \item In $(\mathbb R,\le)$, every two-element set
                    $\{a,b\}$ with $a\neq b$ is \emph{not} totally unordered,
                    because $\le$ is a total order.
            \end{enumerate}
    \end{itemize}
    \end{remark}
    \begin{theorem}[Cardinal exponentiation via function sets]
      \label{thm:card-exp}
      Let $d$ and $e$ be cardinals and choose \emph{any} sets
      \[
        D,\;E \quad\text{with}\quad |D| = d,\; |E| = e.
      \]
      Denote by $D^{E}$ the set of all functions $f\colon E\to D$.
      Then
      \[
        d^{\,e} \;:=\; |D^{E}|
      \]
      is well‑defined; that is, the right–hand cardinal depends only on
      $d$ and~$e$, not on the particular representatives $D$ and $E$.
      \end{theorem}
      
      \begin{proof}
      There are two points to check.
      
      \medskip
      \noindent\textbf{1.  The definition makes sense for \emph{some}
         choice of $D$ and $E$.}
      
      Given any sets of the indicated sizes, the collection
      $D^{E}=\{f\mid f:E\to D\}$ is a well‑defined set
      (by the axioms of set theory), so its cardinality $|D^{E}|$ exists.
      
      \medskip
      \noindent\textbf{2.  The value $|D^{E}|$ is \emph{independent}
         of the chosen representatives.}
      
      Suppose $D',E'$ are other sets with $|D'|=d$ and $|E'|=e$.
      Choose bijections
      \[
        \alpha\colon D\xrightarrow{\;\sim\;} D',
        \qquad
        \beta\colon E'\xrightarrow{\;\sim\;} E.
      \]
      
      \smallskip
      \emph{Define a map between the two function sets:}
      \[
        \Phi\colon D^{E}\longrightarrow (D')^{E'},
        \qquad
        \Phi(f)
            :=\;
            \alpha\circ f\circ \beta
            \;=\;
            \bigl(e'\mapsto \alpha\bigl(f(\beta(e'))\bigr)\bigr).
      \]
      
      \smallskip
      \begin{itemize}
        \item \emph{Injectivity.}
              If $\Phi(f_1)=\Phi(f_2)$, then
              $\alpha\circ f_1\circ\beta=\alpha\circ f_2\circ\beta$.
              Apply $\alpha^{-1}$ on the left and $\beta^{-1}$ on the right
              to conclude $f_1=f_2$.
        \item \emph{Surjectivity.}
              Given any $g\in (D')^{E'}$, set
              $f:=\alpha^{-1}\circ g\circ \beta^{-1}\in D^{E}$.
              Then $\Phi(f)=g$.
      \end{itemize}
      
      Thus $\Phi$ is a bijection,
      so $|D^{E}| = |(D')^{E'}|$.
      Consequently the cardinal we denote by $d^{\,e}$ is
      \emph{independent of the chosen realisations} of $d$ and~$e$,
      and the definition is sound.
      \end{proof}
      
      \paragraph{Remark.}
      When $d$ and $e$ are finite natural numbers,
      $|D^{E}|$ counts the usual number of functions
      $e\mapsto d$, agreeing with elementary arithmetic
      ($d^{\,e}$ distinct functions).
      The theorem shows that this interpretation extends coherently
      to infinite cardinals.
      %------------------------------------------------------------
% Intuitive explanation of cardinal exponentiation d^e = |D^E|
%------------------------------------------------------------
\begin{center}
  \textbf{Why \(\displaystyle d^{\,e}\) counts the functions \(E\to D\)}
  \end{center}
  
  Let \(D\) and \(E\) be sets with \(|D|=d\) and \(|E|=e\).  
  Cardinal exponentiation is defined by
  \[
    d^{\,e}\;:=\;\bigl|\,D^{E}\bigr|,
    \qquad D^{E}=\{f:E\to D\}.
  \]
  
  \subsection*{1.  Label the inputs}
  Because \(|E|=e\) is finite in the intuitive argument, we may index
  \[
    E=\{1,2,\dots,e\}.
  \]
  
  \subsection*{2.  Encode functions as tuples}
  A function \(f:E\to D\) is completely determined by the ordered list of
  its values:
  \[
    f \ \longleftrightarrow\ (\,f(1),\,f(2),\,\dots,\,f(e)\,)\ \in D^{e}.
  \]
  Hence counting such functions is the same as counting ordered \(e\)-tuples
  with entries in \(D\).
  
  \subsection*{3.  Count choices coordinate--wise}
  \[
    \underbrace{d\times d\times\cdots\times d}_{\text{$e$ factors}}
    = d^{\,e}.
  \]
  Each coordinate has \(d\) independent possibilities, so the
  multiplication principle yields \(d^{\,e}\) functions.
  
  \subsection*{4.  Concrete example \((d=2,\;e=3)\)}
  Take \(D=\{0,1\}\) and \(E=\{1,2,3\}\).
  The eight functions correspond to the binary strings of length~3:
  
  \[
  \begin{array}{c|c|c|c}
  f(1) & f(2) & f(3) & \text{binary string}\\ \hline
  0 & 0 & 0 & 000\\
  0 & 0 & 1 & 001\\
  0 & 1 & 0 & 010\\
  0 & 1 & 1 & 011\\
  1 & 0 & 0 & 100\\
  1 & 0 & 1 & 101\\
  1 & 1 & 0 & 110\\
  1 & 1 & 1 & 111
  \end{array}
  \]
  
  Indeed, \(2^{3}=8\) functions.
  
  \subsection*{5.  Extension to arbitrary cardinals}
  For infinite \(d\) or \(e\) an explicit listing is impossible,
  but the same idea persists:
  “choosing an element of \(D\) for every element of \(E\)’’
  \emph{is} the set \(D^{E}\).
  The definition \(d^{\,e}=|D^{E}|\) therefore generalises the
  finite counting argument to all cardinals.
  \begin{theorem}
    \label{thm:continuum}
    Let $\frak c$ denote the cardinality of the set of real numbers
    $\Bbb R.$
    Then
    \[
      \frak c \;=\; 2^{\aleph_0},
    \]
    i.e.\ the continuum is the same size as the power set of
    $\Bbb N$ (or, equivalently, the set of all infinite binary
    sequences).
    \end{theorem}
    
    \begin{proof}
    Write
    \[
      \mathcal B := \{\,0,1\}^{\Bbb N}
         =\bigl\{\,(\varepsilon_1,\varepsilon_2,\dots):
                  \varepsilon_k\in\{0,1\}\bigr\}.
    \]
    By definition $|\mathcal B| = 2^{\aleph_0}$, because each element of
    $\Bbb N$ (there are $\aleph_0$ of them) independently chooses one of
    two values.
    
    \vspace{4pt}
    \textbf{1.  Inject $\mathcal B$ into $(0,1)$.}
    For
    $\boldsymbol\varepsilon=(\varepsilon_1,\varepsilon_2,\dots)\in\mathcal B$
    define
    \[
       \beta(\boldsymbol\varepsilon)
       := 0.\varepsilon_1\varepsilon_2\varepsilon_3\dots_{\!(2)}
       = \sum_{k=1}^{\infty}\frac{\varepsilon_k}{2^{\,k}},
    \]
    the binary‑expansion map.
    To avoid ambiguity, forbid the representation that terminates in an
    infinite tail of $1$’s (e.g.\ use $0.1000\ldots$ instead of
    $0.0111\ldots$).
    Under this convention $\beta$ is \emph{injective}, so
    $2^{\aleph_0}=|\mathcal B|\le |(0,1)|$.
    
    \vspace{4pt}
    \textbf{2.  Inject $(0,1)$ into $\mathcal B$.}
    Every real $x\in(0,1)$ admits a binary expansion
    $x=0.\varepsilon_1\varepsilon_2\varepsilon_3\dots_{\!(2)}$.
    Choose the unique expansion that does \emph{not} terminate in an
    infinite string of $1$’s, and set
    \[
      \alpha(x) := (\varepsilon_1,\varepsilon_2,\dots)\in\mathcal B .
    \]
    The map $\alpha$ is injective, hence
    $|(0,1)|\le |\mathcal B| = 2^{\aleph_0}$.
    
    \vspace{4pt}
    \textbf{3.  Conclude via Cantor–Schröder–Bernstein.}
    We have injections
    \[
        \mathcal B \xrightarrow{\;\beta\;} (0,1)
        \quad\text{and}\quad
        (0,1)\xrightarrow{\;\alpha\;}\mathcal B,
    \]
    so the Cantor–Schröder–Bernstein theorem
    gives a \emph{bijection} between $\mathcal B$ and $(0,1)$.
    Therefore
    \[
       |(0,1)| \;=\; |\mathcal B|
       \;=\; 2^{\aleph_0}.
    \]
    
    \vspace{4pt}
    \textbf{4.  Extend to all of $\Bbb R$.}
    The interval $(0,1)$ is in bijection with $\Bbb R$
    (e.g.\ via the tangent or logistic maps), so
    $|\Bbb R|=|(0,1)|=2^{\aleph_0}$.
    Thus $\frak c = 2^{\aleph_0}$.
    \end{proof}
    
    \begin{remark}
    Intuitively, specifying a real number in $(0,1)$ is equivalent to
    choosing, \emph{once for each natural number $k$}, whether the
    $k$‑th binary digit is $0$ or $1$.
    Those $2$ independent choices made $\aleph_0$ times explain the
    exponentiation notation \(2^{\aleph_0}\).
    \end{remark}
    \paragraph{Clarifying the notation in Step 1}

\begin{itemize}
  \item \(\boldsymbol{L\times L}\).  
        The \emph{Cartesian product} of \(L\) with itself:  
        \[
          L\times L \;=\;
          \bigl\{\, (x,y)\; :\; x\in L,\; y\in L \bigr\}.
        \]
        Each element \((x,y)\) is an \emph{ordered pair}.

  \item \(\boldsymbol{R\subseteq L\times L}\).  
        A \emph{binary relation} on \(L\) is simply a subset of
        \(L\times L\).
        Writing \((x,y)\in R\) means “\(x\) is related to \(y\) by \(R\).”

  \item \textbf{Partial order.}  
        A binary relation \(R\subseteq L\times L\) is a
        \emph{partial order} iff it satisfies
        \begin{align*}
           &\text{(Reflexive)}      &\forall x\in L\!: &\ (x,x)\in R,\\
           &\text{(Antisymmetric)}  &\forall x,y\in L\!: &\
                                     (x,y),(y,x)\in R\implies x=y,\\
           &\text{(Transitive)}     &\forall x,y,z\in L\!: &\
                                     (x,y),(y,z)\in R\implies (x,z)\in R.
        \end{align*}

  \item \(\boldsymbol{\le\subseteq R}\).  
        The symbol \(\le\) denotes the \emph{given} partial order on \(L\).
        Saying \(\le\subseteq R\) means
        \[
          (x,y)\in \le\ \Longrightarrow\ (x,y)\in R
          \qquad(\text{for all }x,y\in L),
        \]
        i.e.\ every comparison already allowed by \(\le\)
        is also allowed by \(R\).
        In this situation we say that \(R\) \emph{extends} the order \(\le\).

  \item \(\displaystyle
          \mathcal P \;=\;
          \bigl\{\,R\subseteq L\times L \mid \le\subseteq R
                 \text{ and } R\text{ is a partial order}\bigr\}.
        \)
        This is \emph{set‑builder notation}:
        \(\mathcal P\) collects \emph{all} binary relations \(R\)
        on \(L\) that
        \begin{enumerate}[label=\textbullet, leftmargin=1.5em]
          \item contain the original order (\(\le\subseteq R\)), and
          \item are themselves partial orders.
        \end{enumerate}
        Elements of \(\mathcal P\) are precisely those orderings of \(L\)
        that keep every comparison the original \(\le\) already had,
        possibly adding more.
\end{itemize}

\noindent
In summary, \(\mathcal P\) is the \emph{poset of all “larger” partial orders}
on \(L\) that include the given order~\(\le\).
This set becomes the arena for Zorn’s lemma in the proof:  
we search within \(\mathcal P\) for a \emph{maximal} relation,
which will turn out to be a total (linear) order extending~\(\le\).
\begin{theorem}
  \label{thm:many‑big‑subsets}
  Let \(D\) be an \emph{infinite} set and write \(\card{D}=d\).
  Then the family
  \[
    \bigl\{\,S\subseteq D : \card{S}=d\bigr\}
  \]
  has cardinality \(2^{d}\).
  \end{theorem}
  
  \begin{proof}
  We will exhibit \(2^{d}\) \emph{pairwise distinct} subsets of \(D\),
  each of cardinality \(d\).
  That will show the desired family has \emph{at least} \(2^{d}\) elements;
  but it can have no more than \(2^{d}\) elements, because
  \(\abs{\mathcal P(D)}=2^{d}\).
  Hence the cardinality is exactly \(2^{d}\).
  
  \bigskip
  \textbf{Step 1.\@ Split \(D\) into two disjoint pieces of size \(d\).}
  
  For every infinite cardinal \(d\) we have \(d+d=d\); informally,
  putting two copies of an infinite set side by side does not increase
  its size.  Concretely, there are disjoint subsets
  \(D_{1},D_{2}\subseteq D\) such that
  \[
    \card{D_{1}} = d, 
    \quad
    \card{D_{2}} = d,
    \quad
    D_{1}\cup D_{2}=D,
    \quad
    D_{1}\cap D_{2}=\varnothing.
  \]
  
  \bigskip
  \textbf{Step 2.\@ Use the power set of \(D_{1}\).}
  
  Consider the power set
  \(
    \mathcal P(D_{1})=\{\,B : B\subseteq D_{1}\}.
  \)
  Because \(\card{D_{1}}=d\), we have
  \[
    \abs{\mathcal P(D_{1})}=2^{d}.
  \]
  
  \bigskip
  \textbf{Step 3.\@ Map every \(B\subseteq D_{1}\) to a “large’’ subset of \(D\).}
  
  Define a map
  \[
    \phi : \mathcal P(D_{1}) \longrightarrow \mathcal P(D),
    \qquad
    \phi(B):=B\cup D_{2}.
  \]
  
  \emph{Injectivity of \(\phi\).}
  If \(B_{1}\neq B_{2}\) then, because \(D_{2}\) is fixed and
  disjoint from \(D_{1}\), we have
  \(\phi(B_{1})=B_{1}\cup D_{2}\neq B_{2}\cup D_{2}=\phi(B_{2})\).
  Thus \(\phi\) is injective.
  
  \bigskip
  \textbf{Step 4.\@ The image of \(\phi\) consists of \(2^{d}\) subsets,
  each of size \(d\).}
  
  Because \(\phi\) is injective,
  \[
    \bigl|\phi\!\bigl[\mathcal P(D_{1})\bigr]\bigr|
    \;=\;
    \abs{\mathcal P(D_{1})}
    \;=\;
    2^{d}.
  \]
  
  Fix \(B\subseteq D_{1}\).
  Since \(B\cap D_{2}=\varnothing\) and
  \(\card{D_{2}}=d\ge\card{B}\),
  basic cardinal arithmetic for disjoint unions gives
  \[
    \card{\phi(B)}
    \;=\;
    \card{B\cup D_{2}}
    \;=\;
    \max\!\bigl\{\card{B},\card{D_{2}}\bigr\}
    \;=\;
    d.
  \]
  
  Hence every set in the image of \(\phi\) has cardinality \(d\).
  
  \bigskip
  \textbf{Step 5.\@ Conclude.}
  
  We have produced \(2^{d}\) pairwise distinct subsets of \(D\),
  each of size \(d\).
  Since \(\mathcal P(D)\) itself has \(2^{d}\) elements, no family of
  subsets of \(D\) can be larger than \(2^{d}\).
  Therefore the collection of subsets of \(D\) having size \(d\)
  also has cardinality \(2^{d}\).
  \end{proof}
  %------------------------------------------------------------
% Continuum Hypothesis (CH) --- statement and basic facts
%------------------------------------------------------------
\begin{definition}[Cardinalities involved]
  \begin{itemize}
    \item \(\displaystyle\aleph_{0}\) is the cardinality of the set
          of natural numbers \(\Bbb N\) (the least infinite cardinal).
    \item \(\displaystyle\frak c := |\Bbb R| = 2^{\aleph_{0}}\) is the
          \emph{continuum}––the cardinality of the real line
          (equivalently, of the power set \(\mathcal P(\Bbb N)\)).
  \end{itemize}
  \end{definition}
  
  \begin{theorem}[Continuum Hypothesis (CH)]
  \label{thm:CH}
  There is no set whose cardinality lies \emph{strictly} between
  \(\aleph_{0}\) and \(\frak c\); formally,
  \[
    \forall\,X
    \quad
    \bigl(
        \aleph_{0}<|X|<\frak c
    \bigr)
    \;\;\Longrightarrow\;\;
    \text{(impossible)} .
  \]
  Equivalently, CH asserts
  \[
    \frak c \;=\; \aleph_{1},
  \]
  where \(\aleph_{1}\) is the \emph{next} infinite cardinal after
  \(\aleph_{0}\).
  \end{theorem}
  
  \begin{remark}[Historical context]
  \begin{enumerate}
    \item Georg Cantor posed CH in 1878 while founding set theory.
    \item \textbf{Independence from ZFC.}
          \begin{itemize}
            \item \emph{Consistency of CH.}  
                  Gödel (1940) showed that if the usual axioms of
                  set theory (ZFC) are consistent, then so is
                  ZFC + CH.
            \item \emph{Consistency of $\lnot$CH.}  
                  Cohen (1963) introduced \emph{forcing}
                  and proved that ZFC + $\lnot$CH is also consistent
                  relative to ZFC.
          \end{itemize}
          Hence CH can neither be proved nor refuted using the
          standard axioms of set theory.
    \item \textbf{Generalised Continuum Hypothesis (GCH).}  
          GCH states \(2^{\aleph_{\alpha}}=\aleph_{\alpha+1}\) for every
          ordinal~\(\alpha\).
          Like CH, GCH is independent of ZFC.
  \end{enumerate}
  \end{remark}
\end{document}
