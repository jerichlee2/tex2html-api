\documentclass[12pt]{article}

% Packages
\usepackage[margin=1in]{geometry}
\usepackage{amsmath,amssymb,amsthm}
\usepackage{enumitem}
\usepackage{hyperref}
\usepackage{xcolor}
\usepackage{import}
\usepackage{xifthen}
\usepackage{pdfpages}
\usepackage{transparent}
\usepackage{listings}
\usepackage{tikz}
\usepackage{physics}
\usepackage{siunitx}
\usepackage{booktabs}
\usepackage{cancel}
  \usetikzlibrary{calc,patterns,arrows.meta,decorations.markings}


\DeclareMathOperator{\Log}{Log}
\DeclareMathOperator{\Arg}{Arg}

\lstset{
    breaklines=true,         % Enable line wrapping
    breakatwhitespace=false, % Wrap lines even if there's no whitespace
    basicstyle=\ttfamily,    % Use monospaced font
    frame=single,            % Add a frame around the code
    columns=fullflexible,    % Better handling of variable-width fonts
}

\newcommand{\incfig}[1]{%
    \def\svgwidth{\columnwidth}
    \import{./Figures/}{#1.pdf_tex}
}
\theoremstyle{definition} % This style uses normal (non-italicized) text
\newtheorem{solution}{Solution}
\newtheorem{proposition}{Proposition}
\newtheorem{problem}{Problem}
\newtheorem{lemma}{Lemma}
\newtheorem{theorem}{Theorem}
\newtheorem{remark}{Remark}
\newtheorem{note}{Note}
\newtheorem{definition}{Definition}
\newtheorem{example}{Example}
\newtheorem{corollary}{Corollary}
\theoremstyle{plain} % Restore the default style for other theorem environments
%

% Theorem-like environments
% Title information
\title{MATH 432 HW 10}
\author{Jerich Lee}
\date{\today}

\begin{document}

\maketitle
\begin{proof}
  Assume every countable closed subset of a metric space \(M\) is complete.
  We show that \(M\) itself is complete, i.e.\ every Cauchy sequence in
  \(M\) converges.
  
  \medskip
  Let \((x_n)\) be a Cauchy sequence in \(M\).
  Set
  \[
     A=\{x_n : n\in\Bbb N\}\subset M .
  \]
  The set \(A\) is \emph{countable} by construction.
  We first check that \(A\) is \emph{closed} in~\(M\).
  
  \smallskip
  \textit{Claim.} \(A\) is closed.  
  Indeed, suppose \(x\in\overline{A}\setminus A\).
  Then there exists a subsequence \((x_{n_k})\) with \(x_{n_k}\to x\).
  Because \((x_n)\) is Cauchy, it has at most one limit;
  convergence of the subsequence forces the whole sequence to converge to
  the same \(x\in M\), contradicting \(x\notin A\).
  Hence \(\overline{A}=A\).
  
  \smallskip
  Since \(A\) is \emph{countable and closed}, the hypothesis tells us that
  \(A\) is a \emph{complete} subspace of \(M\).
  But \((x_n)\subset A\) is Cauchy, so it must converge to a point
  \(x\in A\subset M\).
  
  \medskip
  Therefore every Cauchy sequence in \(M\) converges; thus \(M\) is
  complete.
  \end{proof}
  %--------------------------------------------------------------------
%  2.  Uniform continuity of three functions  (complete solution)
%--------------------------------------------------------------------
\begin{enumerate}
  \item[\textbf{2.}]  Which of the following functions \(f:\Bbb R\to\Bbb R\)
                      are \emph{uniformly} continuous?  Justify your answer.
        \begin{enumerate}[]
        \item \(f(x)=x^{2}\) for all \(x\in\Bbb R\);
        \item \(f(x)=|x|\) for all \(x\in\Bbb R\);
        \item \(f(x)=\dfrac{1}{1+x^{2}}\) for all \(x\in\Bbb R\).
        \end{enumerate}
  
  \smallskip
  \textbf{Solution.}
  \begin{enumerate}[]
  %----------------------------------------------------------
  \item \emph{\(f(x)=x^{2}\) is \textbf{not} uniformly continuous.}
  
        Take \(\varepsilon=1\).
        For any \(\delta>0\) choose
        \[
           x=\frac1\delta+\frac{\delta}{2},
           \qquad
           t=\frac1\delta .
        \]
        Then \( |x-t|=\dfrac{\delta}{2}<\delta \) but
        \[
           |f(x)-f(t)|
              =|x^{2}-t^{2}|
              =(x+t)(x-t)
              =\left(\frac{2}{\delta}+\frac{\delta}{2}\right)\frac{\delta}{2}
              >1=\varepsilon .
        \]
        Hence the \(\delta\)-condition fails; \(x^{2}\) is not uniformly
        continuous on \(\Bbb R\).
  
  %----------------------------------------------------------
  \item \emph{\(f(x)=|x|\) \textbf{is} uniformly continuous.}
  
        For all \(x,t\in\Bbb R\) we have the Lipschitz estimate
        \[
           \bigl||x|-|t|\bigr|\le |x-t|.
        \]
        Given \(\varepsilon>0\) choose \(\delta=\varepsilon\); then
        \( |x-t|<\delta \) implies
        \( |\,|x|-|t|\,|<\varepsilon\).
        Thus \(|x|\) is uniformly continuous on \(\Bbb R\).
  
  %----------------------------------------------------------
  \item \emph{\(f(x)=\dfrac1{1+x^{2}}\) \textbf{is} uniformly continuous.}
  
        For \(x,t\in\Bbb R\)
        \[
           |f(x)-f(t)|
             =\left|\frac{1}{1+x^{2}}-\frac{1}{1+t^{2}}\right|
             =\frac{|x^{2}-t^{2}|}{(1+x^{2})(1+t^{2})}
             =\frac{|x+t|\,|x-t|}{(1+x^{2})(1+t^{2})}.
        \]
        Because \(|x+t|\le |x|+|t|\) and each denominator is \(\ge1\),
        \[
           |f(x)-f(t)|
             \le (|x|+|t|)\,|x-t|
             \le 2\,|x-t| .
        \]
        Given \(\varepsilon>0\) choose \(\delta=\dfrac{\varepsilon}{2}\);
        then \( |x-t|<\delta\Rightarrow |f(x)-f(t)|<\varepsilon\).
        Hence \(f(x)=\dfrac1{1+x^{2}}\) is uniformly continuous on
        \(\Bbb R\).
  \end{enumerate}
  \end{enumerate}
  %--------------------------------------------------------------------
%  Problem 3  •  Induced metric on an open subset of a complete space
%--------------------------------------------------------------------
\paragraph{Problem.}
Let \((M,D)\) be a \emph{complete} metric space, let \(X\subset M\) be
open, and set \(A=M\setminus X\).
Define for \(x,y\in X\)
\[
   \widetilde D(x,y)\;:=\;
   D(x,y)+\Bigl|\frac1{D(x,A)}-\frac1{D(y,A)}\Bigr|.
\]
Show that
\begin{enumerate}[label=(\alph*)]
   \item \(\widetilde D\) is a metric on \(X\);
   \item \((X,\widetilde D)\) is complete;
   \item \((X,D)\) and \((X,\widetilde D)\) are homeomorphic.
\end{enumerate}

\bigskip
\paragraph{Solution.}
\begin{enumerate}[label=(\alph*)]

%--------------------------------------------------------------
\item \emph{\(\widetilde D\) is a metric.}

Positivity, symmetry, and the fact that \(\widetilde D(x,y)=0\)
implies \(x=y\) are immediate from the definition.
We check the triangle inequality.

For \(x,y,z\in X\) the triangle inequality for \(D\) gives
\[
   D(x,y)\le D(x,z)+D(z,y). \tag{1}
\]
Applying the triangle inequality in \(\Bbb R\) to
\(D(\cdot,A)^{-1}\) yields
\[
   \Bigl|
     \frac1{D(x,A)}-\frac1{D(y,A)}
   \Bigr|
   \le
   \Bigl|
     \frac1{D(x,A)}-\frac1{D(z,A)}
   \Bigr|
   +
   \Bigl|
     \frac1{D(y,A)}-\frac1{D(z,A)}
   \Bigr|. \tag{2}
\]
Adding (1) and (2) proves
\(
   \widetilde D(x,y)\le \widetilde D(x,z)+\widetilde D(z,y).
\)
Hence \(\widetilde D\) is a metric.

%--------------------------------------------------------------
\item \emph{\((X,\widetilde D)\) is complete.}

Let \((x_n)\) be a Cauchy sequence in \((X,\widetilde D)\).
Because \(D\le\widetilde D\), the sequence is Cauchy in \((M,D)\) and
thus converges to some limit \(x\in M\) (completeness of \(M\)).

\smallskip
\textit{Claim.} \(x\in X\).
Otherwise \(x\in A\) and \(D(x_n,x)\to0\) implies \(D(x_n,A)\to0\).
But
\[
   \Bigl|
     \frac1{D(x_m,A)}-\frac1{D(x_n,A)}
   \Bigr|
   \le \widetilde D(x_m,x_n)\longrightarrow 0
   \quad(m,n\to\infty),
\]
so \(\bigl(1/D(x_n,A)\bigr)\) is Cauchy in \(\Bbb R\).
This contradicts \(D(x_n,A)\to0\).
Therefore \(x\in X\).

\smallskip
Finally, continuity of \(D(\,\cdot\!,A)\) gives
\(1/D(x_n,A)\to 1/D(x,A)\).
Hence
\[
   \widetilde D(x_n,x)
      =D(x_n,x)+
       \Bigl|
         \frac1{D(x_n,A)}-\frac1{D(x,A)}
       \Bigr|
      \longrightarrow 0,
\]
so \(x_n\to x\) in \((X,\widetilde D)\).
Thus \((X,\widetilde D)\) is complete.

%--------------------------------------------------------------
\item \emph{\((X,D)\) and \((X,\widetilde D)\) are homeomorphic.}

The identity map
\[
   \mathrm{id}:(X,D)\;\longrightarrow\;(X,\widetilde D)
\]
is continuous because \(D\le\widetilde D\).
Conversely, if \(x_n\to x\) in \((X,\widetilde D)\) then
\(\widetilde D(x_n,x)\to0\) forces \(D(x_n,x)\to0\);
hence the inverse identity map is also continuous.
Therefore the two metric spaces are homeomorphic.

\end{enumerate}
%--------------------------------------------------------------------
%  Any subspace of a separable metric space is separable
%--------------------------------------------------------------------
\begin{theorem}
  If \((M,d)\) is a separable metric space and \(A\subset M\), then
  \(A\) is separable (with the subspace topology).
  \end{theorem}
  
  \begin{proof}
  Since \(M\) is separable, it has a \emph{countable base}  
  \(\mathcal U=\{U_n : n\in\Bbb N\}\) of open subsets of \(M\).
  
  Define
  \[
     \mathcal V \;=\;
     \bigl\{\, U_n\cap A \;\bigm|\; n\in\Bbb N \bigr\}.
  \]
  Because each \(U_n\) is open in \(M\) and \(A\) carries the subspace
  topology, every \(U_n\cap A\) is open in \(A\); thus
  \(\mathcal V\subset\mathcal T_A\).
  
  \medskip
  \textit{Claim.} \(\mathcal V\) is a base for the topology on \(A\).
  
  Let \(x\in A\) and let \(V\subset A\) be an open neighbourhood of \(x\).
  Then there exists an open set \(W\subset M\) such that
  \(V=W\cap A\).
  Because \(\mathcal U\) is a base of \(M\),
  there exists \(n\) with \(x\in U_n\subset W\).
  Consequently
  \[
     x\in U_n\cap A \subset W\cap A = V,
  \]
  so every open set of \(A\) contains some member of \(\mathcal V\)
  containing \(x\).
  Hence \(\mathcal V\) is indeed a base for \(A\).
  
  \medskip
  Since \(\mathcal V\) is countable, \(A\) possesses a countable base and
  is therefore \emph{separable}.
  \end{proof}
\end{document}
