\documentclass[12pt]{article}

% Packages
\usepackage[margin=1in]{geometry}
\usepackage{amsmath,amssymb,amsthm}
\usepackage{enumitem}
\usepackage{hyperref}
\usepackage{xcolor}
\usepackage{import}
\usepackage{xifthen}
\usepackage{pdfpages}
\usepackage{transparent}
\usepackage{listings}
\usepackage{tikz}
  \usetikzlibrary{calc,patterns,arrows.meta,decorations.markings}


\DeclareMathOperator{\Log}{Log}
\DeclareMathOperator{\Arg}{Arg}

\lstset{
    breaklines=true,         % Enable line wrapping
    breakatwhitespace=false, % Wrap lines even if there's no whitespace
    basicstyle=\ttfamily,    % Use monospaced font
    frame=single,            % Add a frame around the code
    columns=fullflexible,    % Better handling of variable-width fonts
}

\newcommand{\incfig}[1]{%
    \def\svgwidth{\columnwidth}
    \import{./Figures/}{#1.pdf_tex}
}
\theoremstyle{definition} % This style uses normal (non-italicized) text
\newtheorem{solution}{Solution}
\newtheorem{proposition}{Proposition}
\newtheorem{problem}{Problem}
\newtheorem{lemma}{Lemma}
\newtheorem{theorem}{Theorem}
\newtheorem{remark}{Remark}
\newtheorem{note}{Note}
\newtheorem{definition}{Definition}
\newtheorem{example}{Example}
\newtheorem{corollary}{Corollary}
\theoremstyle{plain} % Restore the default style for other theorem environments
%

% Theorem-like environments
% Title information
\title{}
\author{Jerich Lee}
\date{\today}

\begin{document}

\maketitle
%---------------------------------------------------------------
% Copy–paste LaTeX code: step‑by‑step force estimate for a 105 mph fastball
%---------------------------------------------------------------

\begin{enumerate}[label=\textbf{Step \arabic*:}, leftmargin=2.2em]

  \item \textbf{Specify the known quantities.}
  \[
    \begin{aligned}
      m &= 0.145\;\text{kg} &&\text{(mass of an MLB baseball)}\\
      v_{\text{mph}} &= 105\;\text{mph} &&\text{(speed out of the pitcher’s hand)}
    \end{aligned}
  \]

  \item \textbf{Convert the speed to SI units.}
  \[
    1\;\text{mph}=0.44704\;\text{m\,s}^{-1}\quad\Longrightarrow\quad
    v = v_{\text{mph}}\times 0.44704
      = 105\,(0.44704)\;\text{m\,s}^{-1}
      \approx 46.94\;\text{m\,s}^{-1}.
  \]

  \item \textbf{Compute the kinetic energy when the ball reaches the catcher.}
  \[
    KE = \tfrac12 m v^{2}
        = \frac{1}{2}(0.145)\,(46.94)^{2}
        \approx 1.60\times 10^{2}\;\text{J}.
  \]

  \item \textbf{Energy–work method (average force from stopping distance).}

        Assume the catcher brings the ball to rest over a distance
        \(d = 0.10\text{–}0.15\;\text{m}\) inside the mitt.

        \[
          F_{\text{avg}}
            = \frac{\text{work}}{d}
            = \frac{KE}{d}
            = \frac{1.60\times 10^{2}\;\text{J}}{0.10\text{–}0.15\;\text{m}}
            \approx 1.1\text{–}1.6\;\text{kN}.
        \]

  \item \textbf{Impulse–momentum method (average force from stopping time).}

        High‑speed video typically shows a stopping time of
        \(t = 0.005\text{–}0.008\;\text{s}\).

        \[
          F_{\text{avg}}
            = \frac{\Delta p}{\Delta t}
            = \frac{m\,v}{t}
            = \frac{0.145\,(46.94)}{0.005\text{–}0.008}
            \approx 0.85\text{–}1.36\;\text{kN}.
        \]

        The two independent methods are consistent to within
        experimental uncertainty.

  \item \textbf{Convert the result to pounds‑force (optional).}
        \[
          1\;\text{kN} = 224.8\;\text{lbf}
          \quad\Longrightarrow\quad
          F_{\text{avg}}\approx
          (1.0\text{–}1.6)\times 224.8
          \approx 2.3\text{–}3.6\times 10^{2}\;\text{lbf}.
        \]

  \item \textbf{Comment on peak force.}

        Because the deceleration is not perfectly uniform,
        instantaneous peak forces can reach $\sim2\;\text{kN}$
        ($\approx450\;\text{lbf}$) before the glove padding
        spreads and dissipates the load.

\end{enumerate}
\end{document}
