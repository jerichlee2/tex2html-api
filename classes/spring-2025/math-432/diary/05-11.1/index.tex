\documentclass[12pt]{article}

% Packages
\usepackage[margin=1in]{geometry}
\usepackage{amsmath,amssymb,amsthm}
\usepackage{enumitem}
\usepackage{hyperref}
\usepackage{xcolor}
\usepackage{import}
\usepackage{xifthen}
\usepackage{pdfpages}
\usepackage{transparent}
\usepackage{listings}
\usepackage{tikz}
\usepackage{physics}
\usepackage{siunitx}
\usepackage{booktabs}
\usepackage{cancel}
  \usetikzlibrary{calc,patterns,arrows.meta,decorations.markings}


\DeclareMathOperator{\Log}{Log}
\DeclareMathOperator{\Arg}{Arg}

\lstset{
    breaklines=true,         % Enable line wrapping
    breakatwhitespace=false, % Wrap lines even if there's no whitespace
    basicstyle=\ttfamily,    % Use monospaced font
    frame=single,            % Add a frame around the code
    columns=fullflexible,    % Better handling of variable-width fonts
}

\newcommand{\incfig}[1]{%
    \def\svgwidth{\columnwidth}
    \import{./Figures/}{#1.pdf_tex}
}
\theoremstyle{definition} % This style uses normal (non-italicized) text
\newtheorem{solution}{Solution}
\newtheorem{proposition}{Proposition}
\newtheorem{problem}{Problem}
\newtheorem{lemma}{Lemma}
\newtheorem{theorem}{Theorem}
\newtheorem{remark}{Remark}
\newtheorem{note}{Note}
\newtheorem{definition}{Definition}
\newtheorem{example}{Example}
\newtheorem{corollary}{Corollary}
\theoremstyle{plain} % Restore the default style for other theorem environments
%

% Theorem-like environments
% Title information
\title{MATH 432 Practice Final Exam 2}
\author{Jerich Lee}
\date{\today}

\begin{document}

\maketitle
%%%%%%%%%%%%%%%%%%%%%%%%%%%%%%%%%%%%%%%%%%%%%%%%%%%%%%%%%%%%%%%%%%%%%%%%%%%%
%                MATH 432 – Practice Final Examination (B)                %
%                (Eight problems; show all work.)                         %
%%%%%%%%%%%%%%%%%%%%%%%%%%%%%%%%%%%%%%%%%%%%%%%%%%%%%%%%%%%%%%%%%%%%%%%%%%%%


\begin{problem}[Chains in power sets]
  Let $A$ be a set such that its power set $\mathcal{P}(A)$,
  ordered by $\subseteq$, is a \emph{chain} (i.e.\ any two subsets are
  comparable).
  Determine all possible sets $A$ and justify your answer.
\end{problem}

\begin{problem}[Equivalence classes induced by a preorder]
  Let $R\subset A\times A$ be a reflexive and transitive relation.
  Define
  \[
      \widetilde{R}\;:=\;\bigl\{(a,b)\in A\times A: (a,b)\in R
      \text{ and }(b,a)\in R\bigr\}.
  \]
  \begin{enumerate}[label=(\alph*)]
    \item Prove that $\widetilde{R}$ is an equivalence relation.
    \item For $a\in A$ let $S_a$ denote its
          $\widetilde{R}$–equivalence class, and for
          $a,b\in A$ set $S_a\le S_b$ whenever $(a,b)\in R$.
          Show that this defines a partial order on the set of
          equivalence classes.
  \end{enumerate}
\end{problem}

\begin{problem}[Linear extensions of partial orders]
  Let $(L,\le)$ be a finite partially ordered set.
  Show that there exists a \emph{total} order $\preccurlyeq$ on $L$
  that extends~$\le$, i.e.\ $x\le y$ implies $x\preccurlyeq y$.
  \emph{Hint:}  Proceed by induction on~$|L|$ or invoke Zorn’s Lemma
  for the family of partial orders extending~$\le$.
\end{problem}

\begin{problem}[Maximal elements from chain conditions]
  Let $L$ be a lattice in which \emph{every} chain has an upper bound.
  Prove that $L$ has a \emph{unique} maximal element.
\end{problem}

\begin{problem}[Removing a finite or countable subset]
  \begin{enumerate}[label=(\alph*)]
    \item Let $A$ be an infinite set and $B\subset A$ finite.
          Put $C:=A\setminus B$.  Construct an explicit bijection
          between $A$ and $C$.
    \item Let $A$ be uncountable and $B\subset A$ countable, with
          $C:=A\setminus B$.  Show that $A$ and $C$ have the same
          cardinality.
  \end{enumerate}
\end{problem}

\begin{problem}[Exponent laws for cardinals]
  Let $d,e_{1},e_{2}$ be cardinal numbers.
  Prove, without appealing to Theorems 13–16 in the notes, that
  \[
      d^{\,e_{1}+e_{2}}\;=\;
      d^{\,e_{1}}\cdot d^{\,e_{2}}.
  \]
  \emph{Suggestion:}  Exhibit explicit bijections between the sets
  involved.
\end{problem}

\begin{problem}[Large and small open sets in an infinite metric space]
  Let $(M,D)$ be an infinite metric space.
  Show that there exists an open set $U\subset M$ such that both
  $U$ and $M\setminus U$ are infinite.
  (\emph{Hint:}  Consider whether $M$ contains an isolated point or a
  limit point.)
\end{problem}

\begin{problem}[Countable well-ordering implies well-ordering]
  Let $C$ be a chain (i.e.\ a linearly ordered set) with the property
  that \emph{every countable subset of $C$ is well-ordered}.
  Prove that $C$ itself is well-ordered.
\end{problem}
\end{document}
