\documentclass[12pt]{article}

% Packages
\usepackage[margin=1in]{geometry}
\usepackage{amsmath,amssymb,amsthm}
\usepackage{enumitem}
\usepackage{hyperref}
\usepackage{xcolor}
\usepackage{import}
\usepackage{xifthen}
\usepackage{pdfpages}
\usepackage{transparent}
\usepackage{listings}
\usepackage{tikz}
\usepackage{physics}
\usepackage{siunitx}
\usepackage{booktabs}
\usepackage{cancel}
  \usetikzlibrary{calc,patterns,arrows.meta,decorations.markings}


\DeclareMathOperator{\Log}{Log}
\DeclareMathOperator{\Arg}{Arg}

\lstset{
    breaklines=true,         % Enable line wrapping
    breakatwhitespace=false, % Wrap lines even if there's no whitespace
    basicstyle=\ttfamily,    % Use monospaced font
    frame=single,            % Add a frame around the code
    columns=fullflexible,    % Better handling of variable-width fonts
}

\newcommand{\incfig}[1]{%
    \def\svgwidth{\columnwidth}
    \import{./Figures/}{#1.pdf_tex}
}
\theoremstyle{definition} % This style uses normal (non-italicized) text
\newtheorem{solution}{Solution}
\newtheorem{proposition}{Proposition}
\newtheorem{problem}{Problem}
\newtheorem{lemma}{Lemma}
\newtheorem{theorem}{Theorem}
\newtheorem{remark}{Remark}
\newtheorem{note}{Note}
\newtheorem{definition}{Definition}
\newtheorem{example}{Example}
\newtheorem{corollary}{Corollary}
\theoremstyle{plain} % Restore the default style for other theorem environments
%

% Theorem-like environments
% Title information
\title{MATH-432 Midterm 2}
\author{Jerich Lee}
\date{\today}

\begin{document}

\maketitle
\  \begin{problem}
    Let $a,b,d$ be \emph{infinite} cardinals.
    
    \begin{enumerate}[label=\textup{(\alph*)}]
      \item Show that $d^{\,d}=2^{\,d}$.
      \item Show that, if $d=2^{\,a}\ge 2^{\,b}$, then $d^{\,b}=d$.
    \end{enumerate}
    \end{problem}
    
    \begin{solution}
    Throughout we use three standard facts of cardinal arithmetic for \emph{infinite}
    cardinals $\kappa,\lambda\ge\aleph_0$:
    \begin{align}
      \text{(F1)}\quad & \kappa\le\lambda \;\Longrightarrow\;
                         2^{\kappa}\le 2^{\lambda}, \\[2pt]
      \text{(F2)}\quad &
          (\kappa^{\lambda})^{\mu}= \kappa^{\lambda\cdot\mu}, \\[2pt]
      \text{(F3)}\quad &
          \lambda\cdot\lambda=\lambda.
    \end{align}
    
    \medskip
    \noindent\textbf{(a)  Prove $d^{\,d}=2^{\,d}$.}
    
    \begin{enumerate}
      \item[] \emph{Lower bound.}  
            Since $2\le d$, monotonicity of the base gives
            \(
              2^{\,d}\le d^{\,d}.
            \)
    
      \item[] \emph{Upper bound.}  
            Because $d\le 2^{\,d}$ (any infinite cardinal is
            bounded above by its power set),
            \[
                d^{\,d}\le(2^{\,d})^{d}
                       = 2^{\,d\cdot d}
                       = 2^{\,d} \quad\text{by (F2) and (F3).}
            \]
    
      \item[] \emph{Equality.}  
            From Steps~1 and~2 we have the double inequality
            \(
               2^{\,d}\le d^{\,d}\le 2^{\,d},
            \)
            hence $d^{\,d}=2^{\,d}$.
    \end{enumerate}
    
    \bigskip
    \noindent\textbf{(b)  Assume $d=2^{\,a}\ge 2^{\,b}$ and show $d^{\,b}=d$.}
    
    \begin{enumerate}
      \item[] \emph{Lower bound.}
            Because $b\ge 1$ and $d\ge 2$,
            \(
              d^{\,b}\ge d.
            \)
    
      \item[] \emph{Upper bound.}
            The hypothesis $2^{\,a}\ge 2^{\,b}$ and fact (F1) allow us to write
            \[
               d
               = 2^{\,a}
               = (2^{\,a})^{a}
               = 2^{\,a\cdot a}
               \;\overset{\text{(F3)}}{=}\;
                 2^{\,a}
               \ge (2^{\,b})^{a}.
            \]
            Apply (F2) twice to the right--most term:
            \[
               (2^{\,b})^{a}
               =2^{\,b\cdot a}
               =(2^{\,a})^{b}
               =d^{\,b}.
            \]
            Hence $d\ge d^{\,b}$.
    
      \item[] \emph{Equality.}
            Combining Steps~1 and~2 yields
            \(
              d\le d^{\,b}\le d,
            \)
            so $d^{\,b}=d$.
    \end{enumerate}
    \end{solution}

    \begin{solution}
      \begin{theorem}[Hausdorff’s Maximal Chain Theorem]
        \label{thm:max-chain}
        Let $(L,\le)$ be a partially ordered set and let $C\subseteq L$ be a
        (non‑empty) chain.  
        Then there exists a \emph{maximal chain}
        $C^{\ast}\subseteq L$ with $C\subseteq C^{\ast}$.
        \end{theorem}
        
        \begin{proof}
        We proceed with a Zorn–lemma argument, spelled out in detail.
        
        \bigskip
        \textbf{Step 1 (Poset of chains extending $C$).}
        Define
        \[
          \mathcal L
          \;:=\;
          \bigl\{\,A\subseteq L \mid C\subseteq A
                  \text{ and $A$ is a chain in }(L,\le)\bigr\}.
        \]
        Partially order $\mathcal L$ by inclusion:
        for $A,B\in\mathcal L$ write
        $A\preceq B \iff A\subseteq B$.
        Then $(\mathcal L,\preceq)$ is itself a poset.
        
        \bigskip
        \textbf{Step 2 (Chains in $\mathcal L$ have upper bounds in
        $\mathcal L$).}
        Let
        \(
          \mathcal C=\{A_\alpha : \alpha\in I\}\subseteq\mathcal L
        \)
        be a chain with respect to $\preceq$ (i.e.\ $\forall\alpha,\beta$,
        either $A_\alpha\subseteq A_\beta$ or $A_\beta\subseteq A_\alpha$).
        Define
        \[
          A_0 \;:=\; \bigcup_{\alpha\in I} A_\alpha.
        \]
        
        \emph{Claim: $A_0\in\mathcal L$.}
        
        \begin{itemize}
          \item $C\subseteq A_0$ by construction since $C\subseteq A_\alpha$
                for every $\alpha$.
          \item \emph{Chain property.}  
                Take arbitrary $a,b\in A_0$.
                Choose $\alpha,\beta$ with $a\in A_\alpha$ and $b\in A_\beta$.
                Because $\mathcal C$ is a chain in $(\mathcal L,\preceq)$,
                assume without loss of generality $A_\alpha\subseteq A_\beta$.
                Then $a,b\in A_\beta$, and $A_\beta$ is totally ordered,
                so $a\le b$ or $b\le a$ in $L$.
        \end{itemize}
        Thus $A_0$ is a chain extending $C$, i.e.\ $A_0\in\mathcal L$,
        and clearly $A_\alpha\preceq A_0$ for all $\alpha$.
        Hence $A_0$ is an \emph{upper bound} of the chain $\mathcal C$
        in $(\mathcal L,\preceq)$.
        
        \bigskip
        \textbf{Step 3 (Apply Zorn’s lemma).}
        Every chain in $\mathcal L$ has an upper bound in $\mathcal L$
        (Step~2); therefore, by Zorn’s lemma, the poset
        $(\mathcal L,\preceq)$ possesses a maximal element.
        Denote this maximal element by $C^{\ast}$.
        
        \bigskip
        \textbf{Step 4 (Interpretation of maximality).}
        By definition, $C^{\ast}\in\mathcal L$ is a chain with $C\subseteq C^{\ast}$.
        Its maximality means that if
        $D\in\mathcal L$ satisfies $C^{\ast}\subseteq D$,
        then necessarily $D=C^{\ast}$;
        equivalently, $C^{\ast}$ is a \emph{maximal} chain in $L$
        that contains $C$.
        
        \bigskip
        \textbf{Step 5 (Conclusion).}
        Such a chain $C^{\ast}$ is the desired maximal extension of $C$,
        completing the proof.
        \end{proof}
    \end{solution}

    \begin{solution}
      \begin{theorem}
        Let $(A,\le)$ be a \emph{chain} (i.e.\ a totally ordered set).
        Fix $n\in\Bbb N$ and suppose
        \[
           A \;=\; \bigcup_{i=1}^{n} A_i,
           \qquad
           A_i\subseteq A,
           \qquad
           (A_i,\le)\text{ well‑ordered for each }i.
        \]
        Then $(A,\le)$ is itself \emph{well‑ordered}.
        \end{theorem}
        
        \begin{proof}
        Recall that a chain is well‑ordered iff it contains
        \emph{no infinite descending sequence};
        we argue by contradiction using this equivalent formulation.
        
        \medskip
        \textbf{Step 1 (Assume $A$ is \emph{not} well‑ordered).}
        Then there exists an infinite descending sequence in $A$:
        \[
           a_1 \;>\; a_2 \;>\; a_3 \;>\;\cdots .
        \]
        Define the subset
        \[
           X \;:=\; \{a_1,a_2,a_3,\dots\}\subseteq A .
        \]
        $X$ is infinite by construction.
        
        \medskip
        \textbf{Step 2 (Cover $X$ by the $A_i$’s).}
        Because $A=\bigcup_{i=1}^{n}A_i$, we have
        \[
           X
           \;=\;
           \bigcup_{i=1}^{n}\!\bigl(X\cap A_i\bigr).
        \]
        
        \medskip
        \textbf{Step 3 (Pigeonhole principle on a finite union).}
        A finite union of sets is infinite
        only if at least one constituent set is infinite.
        Hence there is an index $k\in\{1,\dots,n\}$ such that
        \[
           X\cap A_k\quad\text{is infinite.}
        \]
        
        \medskip
        \textbf{Step 4 (Infinite descending sequence lives in $A_k$).}
        Every element of $X\cap A_k$ belongs to the original
        descending sequence $(a_j)$, and the order on $A_k$
        is the restriction of $\le$ on $A$.
        Therefore $(X\cap A_k,\le)$ contains an infinite
        strictly descending sequence.
        
        \medskip
        \textbf{Step 5 (Contradiction).}
        $A_k$ was assumed to be well‑ordered, which forbids
        infinite descending sequences.
        Step~4 contradicts this,
        so our initial assumption (that $A$ is not well‑ordered)
        must be false.
        
        \medskip
        \textbf{Step 6 (Conclusion).}
        Thus $(A,\le)$ contains no infinite descending sequence,
        hence is well‑ordered.
        \end{proof}      
    \end{solution}

    \begin{solution}
      \begin{theorem}
        Every infinite metric space $(M,d)$ contains an \emph{infinite
        discrete subspace} $A\subseteq M$; that is, for every $x\in A$
        there exists $r_x>0$ such that the open ball $B(x,r_x)\cap A=\{x\}$.
        \end{theorem}
        
        \begin{proof}
        We treat two mutually exclusive cases.
        
        \medskip
        \textbf{Case 1: $M$ is discrete.}
        
        Because every singleton $\{x\}$ is open in $M$, \emph{any} infinite
        subset $A\subseteq M$ is automatically discrete.
        
        \medskip
        \textbf{Case 2: $M$ is \emph{not} discrete (so $M$ has a limit point).}
        
        \begin{enumerate}
          \item Choose a limit (accumulation) point $x\in M$.
                By definition, every open ball centred at $x$ contains
                infinitely many points of $M$.
        
          \item Construct inductively a sequence of distinct points
                $x_1,x_2,\dots$ in $M\setminus\{x\}$ satisfying
                \[
                  d(x_n,x)<\frac1n
                  \quad\text{and}\quad
                  x_n\neq x_m \text{ for } n\neq m.
                \]
                This is possible because each ball
                $B\!\bigl(x,\tfrac1n\bigr)$ contains infinitely many points
                different from~$x$.
        
          \item Set
                \[
                  A:=\{x_n : n\in\mathbb N\}.
                \]
                The set $A$ is infinite by construction.
          \item Show that every point of $A$ is isolated:
        
                \begin{itemize}
                  \item If $x\notin A$, then for each $x_n$,
                        the ball
                        $B\!\bigl(x_n,\tfrac12 d(x_n,x)\bigr)$
                        meets $A$ only at $x_n$.
                  \item If $x\in A$ (say $x=x_m$ for some $m$), suppose
                        contrary to claim that $x$ is not isolated in $A$.
                        Then some subsequence $(x_{n_k})$ with
                        $n_k\neq m$ converges to $x_m$, hence also to $x$,
                        forcing $x_m=x_{n_k}$ for large $k$, a contradiction.
                        Therefore $x$ \emph{is} isolated in~$A$.
                \end{itemize}
        
                Thus $A$ is an infinite discrete subspace.
        \end{enumerate}
        
        \medskip
        In either case we obtain the desired infinite discrete subspace~$A$.
        \end{proof}      
    \end{solution}
\end{document}
