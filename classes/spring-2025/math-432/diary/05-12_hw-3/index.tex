\documentclass[12pt]{article}

% Packages
\usepackage[margin=1in]{geometry}
\usepackage{amsmath,amssymb,amsthm}
\usepackage{enumitem}
\usepackage{hyperref}
\usepackage{xcolor}
\usepackage{import}
\usepackage{xifthen}
\usepackage{pdfpages}
\usepackage{transparent}
\usepackage{listings}
\usepackage{tikz}
\usepackage{physics}
\usepackage{siunitx}
\usepackage{booktabs}
\usepackage{cancel}
  \usetikzlibrary{calc,patterns,arrows.meta,decorations.markings}


\DeclareMathOperator{\Log}{Log}
\DeclareMathOperator{\Arg}{Arg}

\lstset{
    breaklines=true,         % Enable line wrapping
    breakatwhitespace=false, % Wrap lines even if there's no whitespace
    basicstyle=\ttfamily,    % Use monospaced font
    frame=single,            % Add a frame around the code
    columns=fullflexible,    % Better handling of variable-width fonts
}

\newcommand{\incfig}[1]{%
    \def\svgwidth{\columnwidth}
    \import{./Figures/}{#1.pdf_tex}
}
\theoremstyle{definition} % This style uses normal (non-italicized) text
\newtheorem{solution}{Solution}
\newtheorem{proposition}{Proposition}
\newtheorem{problem}{Problem}
\newtheorem{lemma}{Lemma}
\newtheorem{theorem}{Theorem}
\newtheorem{remark}{Remark}
\newtheorem{note}{Note}
\newtheorem{definition}{Definition}
\newtheorem{example}{Example}
\newtheorem{corollary}{Corollary}
\theoremstyle{plain} % Restore the default style for other theorem environments
%

% Theorem-like environments
% Title information
\title{MATH-432 HW 3}
\author{Jerich Lee}
\date{\today}

\begin{document}

\maketitle
\begin{solution}
  \textbf{Background facts.}
  \begin{enumerate}[label=\arabic*.]
      \item[(F1)] A set is \emph{countable} if it is finite or can be put in
            bijection with $\mathbb{N}$.
      \item[(F2)] \emph{(Countable union of countable sets is countable).}  
            If $\{E_m:m\in\mathbb{N}\}$ is a family of countable sets, then
            $\displaystyle\bigcup_{m\in\mathbb{N}}E_m$ is countable.
            (One proof lists the elements of each $E_m$ in a table and reads the
            table diagonally; another exhibits an explicit injection into
            $\mathbb{N}$.)
  \end{enumerate}
  
  Throughout, the symbol ``$\sqcup$’’ denotes a \emph{disjoint} union.
  
  \bigskip
  \textbf{(a)  Product of two countable sets.}
  
  \smallskip
  \emph{Step 1:  Reduce to the infinite case.}
  If either $A$ or $B$ is finite, then $A\times B$ is a finite union of
  countable (indeed finite) sets, hence countable.
  So assume both $A$ and $B$ are \emph{countably infinite}.
  
  \smallskip
  \emph{Step 2:  Fix enumerations.}
  Choose bijections
  \[
      A\;\xrightarrow{\;\ \alpha\;\ }\;\mathbb{N},
      \qquad
      B\;\xrightarrow{\;\ \beta\;\ }\;\mathbb{N};
      \quad
      \alpha^{-1}(n)=:a_n,\;
      \beta^{-1}(i)=:b_i
      \quad(n,i\in\mathbb{N}).
  \]
  
  \smallskip
  \emph{Step 3:  Decompose $A\times B$ into countable slices.}
  For each $n\in\mathbb{N}$ set
  \[
      C_n\;:=\;\{(a_n,b_i): i\in\mathbb{N}\}\subseteq A\times B.
  \]
  Because $B$ is countable, each $C_n$ is countable.
  
  \smallskip
  \emph{Step 4:  Reassemble.}
  Every pair $(a,b)\in A\times B$ equals
  $(a_n,b_i)$ for the unique $n:=\alpha(a)$ and $i:=\beta(b)$,
  hence belongs to $C_n$.
  Thus
  \[
     A\times B \;=\; \bigcup_{n\in\mathbb{N}} C_n.
  \]
  By fact (F2) the countable union of the countable sets $C_n$
  is countable.
  Therefore $A\times B$ is countable.
  
  \bigskip
  \textbf{(b)  Finite Cartesian products of countable sets.}
  
  \smallskip
  Let $n\in\mathbb{N}$ and suppose
  $A_1,A_2,\dots,A_n$ are countable.
  We proceed by induction on $n$.
  
  \smallskip
  \emph{Base case $n=1$:} $A_1$ is countable by hypothesis.
  
  \smallskip
  \emph{Inductive step.}
  Assume the product
  \(
     P_{k}:=A_1\times\cdots\times A_k
  \)
  is countable for some $k\ge1$.
  Because $A_{k+1}$ is countable, part~(a) implies
  \[
     P_{k+1}=P_k\times A_{k+1}
  \]
  is countable.
  Hence by induction $A_1\times\cdots\times A_n$ is countable for every
  positive integer $n$.
  
  \bigskip
  \textbf{Conclusion.}
  \begin{enumerate}[label=\textbullet]
    \item The Cartesian product of two countable sets is countable.
    \item Any finite Cartesian product of countable sets is countable.
  \end{enumerate}
  \end{solution}
  \begin{solution}
    Let \(A\) be a countable set and let  
    \[
       X \;:=\;\{\,F\subseteq A : F\ \text{is finite}\,\}
    \]
    denote the collection of \emph{all} finite subsets of \(A\).
    We prove that \(X\) itself is countable.
    
    \bigskip
    \textbf{Step 1.  Decompose \(X\) by cardinality.}
    
    For each integer \(n\ge 0\) define
    \[
       X_{n}\;:=\;\{\,F\subseteq A : |F|=n\,\},
    \qquad
       \text{so that}\quad
       X \;=\;\bigcup_{n=0}^{\infty} X_{n}.
    \]
    
    \bigskip
    \textbf{Step 2.  Show that each \(X_{n}\) is countable.}
    
    Because \(A\) is countable, there exists a bijective enumeration
    \(A=\{a_{1},a_{2},\dots\}\).
    
    \smallskip
    \emph{(2a) Countability of \(A^{n}\).}  
    For fixed \(n\in\mathbb{N}\), the \(n\)-fold Cartesian product
    \(A^{n}=A\times\cdots\times A\) is countable
    (by repeated application of the fact that the product of two countable
    sets is countable).
    
    \smallskip
    \emph{(2b) A subset of \(A^{n}\) with distinct coordinates.}  
    Set
    \[
       A^{(n)}
         :=\{\, (x_{1},\dots,x_{n})\in A^{n} : x_{1},\dots,x_{n}
               \text{ are pairwise distinct}\,\}.
    \]
    Clearly \(A^{(n)}\subseteq A^{n}\); hence \(A^{(n)}\) is countable.
    
    \smallskip
    \emph{(2c) Surjection onto \(X_{n}\).}  
    Define
    \[
       \pi : A^{(n)} \longrightarrow X_{n},
       \qquad
       \pi(x_{1},\dots,x_{n}) \;=\; \{x_{1},\dots,x_{n}\}.
    \]
    The map \(\pi\) is \emph{onto} because every \(n\)-element subset
    \(F=\{y_{1},\dots,y_{n}\}\) of \(A\) arises as
    \(\pi(y_{1},\dots,y_{n})\).
    Since \(A^{(n)}\) is countable and \(\pi\) is surjective,
    its image \(X_{n}\) is countable.
    
    \bigskip
    \textbf{Step 3.  \(X\) is a countable union of countable sets.}
    
    We have expressed \(X\) as  
    \(X=\bigcup_{n=0}^{\infty} X_{n}\)
    with each \(X_{n}\) countable.
    A countable union of countable sets is countable,
    so \(X\) itself is countable.
    
    \bigskip
    \textbf{Conclusion.}
    The set of all finite subsets of a countable set is countable:
    \[
       \boxed{\;
          A\ \text{countable} \;\Longrightarrow\;
          X=\{F\subseteq A : F \text{ finite}\}\ \text{countable}.
       \;}
    \]
    \end{solution}
  \begin{solution}

    \begin{theorem}
      Let $A$ be an infinite set and let $B\subseteq A$ be finite.
      Put $C:=A\setminus B$.
      There exists a bijection $f\colon A\to C$.
      \end{theorem}
      
      \begin{proof}
      We proceed in detailed steps.
      
      \textbf{Step 1 (A countably infinite subset of $C$).}
      Because $C$ is infinite, it contains a countably infinite subset.
      Fix such a subset and denote it by
      \[
        D=\{d_0,d_1,d_2,\dots\}\subseteq C.
      \]
      
      \textbf{Step 2 ($B\cup D$ is countably infinite).}
      The union of a finite set $B$ with a countably infinite set $D$ is
      countably infinite, so we may enumerate
      \[
        B\cup D=\{e_0,e_1,e_2,\dots\}.
      \]
      
      \textbf{Step 3 (A bijection $g\colon B\cup D\to D$).}
      Define
      \[
        g(e_k)=d_k\qquad(k\ge 0).
      \]
      Since every element of $B\cup D$ appears exactly once as some $e_k$
      and every element of $D$ appears exactly once as $d_k$,
      $g$ is a bijection.
      
      \textbf{Step 4 (Define $f\colon A\to C$).}
      \[
        f(a)=
        \begin{cases}
          a,   &\text{if }a\notin B\cup D,\\[6pt]
          g(a),&\text{if }a\in B\cup D.
        \end{cases}
      \]
      
      \textbf{Step 5 ($f$ is injective).}
      Let $a_1,a_2\in A$ and assume $f(a_1)=f(a_2)$.
      \begin{itemize}
        \item If $a_1,a_2\notin B\cup D$, then $f(a_i)=a_i$, so $a_1=a_2$.
        \item If $a_1,a_2\in B\cup D$, then $f(a_i)=g(a_i)$ and the injectivity of $g$
              implies $a_1=a_2$.
        \item If exactly one of $a_1,a_2$ lies in $B\cup D$, then
              $f(a_1)\in D$ while $f(a_2)\notin D$, a contradiction.
      \end{itemize}
      Thus $f$ is injective.
      
      \textbf{Step 6 ($f$ is surjective onto $C$).}
      Let $c\in C$.
      \begin{itemize}
        \item If $c\in D$, surjectivity of $g$ yields $a\in B\cup D$ with $g(a)=c$,
              hence $f(a)=c$.
        \item If $c\in C\setminus D$, then $c\notin B\cup D$ and $f(c)=c$.
      \end{itemize}
      Therefore every $c\in C$ is hit by $f$.
      
      \textbf{Step 7 (Conclusion).}
      $f$ is both injective and surjective, hence a bijection $A\to C$.
      \end{proof}
    \end{solution}
    \begin{theorem}
      Let $A$ be an \emph{uncountable} set and let $B\subseteq A$ be
      \emph{countable}.  
      Put $C:=A\setminus B$.
      There exists a bijection $f\colon A\longrightarrow C$.
      \end{theorem}
      
      \begin{proof}
      We give a detailed construction.
      
      \medskip
      \textbf{Step 1 (Why $C$ is uncountable).}
      If $C$ were countable, then the union
      \(
      A=C\cup B
      \)
      of two countable sets would be countable,
      contradicting the hypothesis that $A$ is uncountable.
      Hence $C$ is uncountable and therefore infinite.
      
      \medskip
      \textbf{Step 2 (Choose a countably infinite subset $D\subseteq C$).}
      Because $C$ is infinite, it contains a countably infinite subset.
      Fix one and write it as
      \[
        D=\{d_0,d_1,d_2,\dots\}\subseteq C .
      \]
      
      \medskip
      \textbf{Step 3 ($B\cup D$ is countable).}
      The union of two countable sets is countable,
      so we may enumerate
      \[
        B\cup D = \{e_0,e_1,e_2,\dots\}.
      \]
      
      \medskip
      \textbf{Step 4 (A bijection $g\colon B\cup D\to D$).}
      Define
      \[
        g(e_k)=d_k\qquad(k\ge 0).
      \]
      Because every element of $B\cup D$ appears exactly once as some $e_k$
      and every $d_k$ appears exactly once in the codomain,
      $g$ is a bijection.
      
      \medskip
      \textbf{Step 5 (Define the desired map $f\colon A\to C$).}
      \[
        f(a)=
        \begin{cases}
          a,   &\text{if }a\notin B\cup D,\\[6pt]
          g(a),&\text{if }a\in B\cup D.
        \end{cases}
      \]
      
      \medskip
      \textbf{Step 6 ($f$ is injective).}
      Let $a_1,a_2\in A$ and assume $f(a_1)=f(a_2)$.
      
      \begin{itemize}
        \item \emph{Both outside $B\cup D$.}
              Then $f(a_i)=a_i$, so $a_1=a_2$.
        \item \emph{Both inside $B\cup D$.}
              Then $f(a_i)=g(a_i)$ and $g$ is injective,
              hence $a_1=a_2$.
        \item \emph{Exactly one inside $B\cup D$.}
              Say $a_1\in B\cup D$ while $a_2\notin B\cup D$.
              Then $f(a_1)=g(a_1)\in D$, but $f(a_2)=a_2\notin D$,
              a contradiction.
      \end{itemize}
      Thus $f$ is injective.
      
      \medskip
      \textbf{Step 7 ($f$ is surjective onto $C$).}
      Take any $c\in C$.
      
      \begin{itemize}
        \item If $c\in D$, surjectivity of $g$ yields an $a\in B\cup D$
              with $g(a)=c$, so $f(a)=c$.
        \item If $c\in C\setminus D$, then $c\notin B\cup D$
              and $f(c)=c$.
      \end{itemize}
      Hence every element of $C$ is attained by $f$.
      
      \medskip
      \textbf{Step 8 (Conclusion).}
      $f$ is both injective and surjective, i.e.\ a bijection $A\to C$.
      \end{proof}

      \begin{solution}
        \begin{theorem}
          The open unit cube
          \[
            (0,1)^3 \;=\; (0,1)\times(0,1)\times(0,1)\subset\mathbb R^{3}
          \]
          has the same cardinality as the open unit interval $(0,1)\subset\mathbb R$.
          \end{theorem}
          
          \begin{proof}
          We construct injections in both directions and then invoke the
          Cantor--Schröder--Bernstein theorem.
          
          \medskip
          \textbf{Step 1 (Injection $(0,1)\hookrightarrow(0,1)^3$).}
          Define
          \[
            f:(0,1)\longrightarrow(0,1)^3,
            \qquad
            f(a):=\Bigl(a,\;\tfrac12,\;\tfrac12\Bigr).
          \]
          If $f(a_1)=f(a_2)$ then $a_1=a_2$, so $f$ is injective.
          
          \medskip
          \textbf{Step 2 (Fixing unique decimal expansions).}
          Every real $t\in(0,1)$ has a decimal expansion  
          $t=0.t_1t_2t_3\ldots$ with digits $t_k\in\{0,\dots,9\}$.
          To make this expansion \emph{unique}, we \emph{forbid}
          the representation that terminates with an infinite string of $9$’s
          (e.g.\ we use $0.5000\ldots$ instead of $0.4999\ldots$).
          
          \medskip
          \textbf{Step 3 (Injection $(0,1)^3\hookrightarrow(0,1)$).}
          Given $(x,y,z)\in(0,1)^3$, write their unique decimal expansions
          \[
            x=0.x_1x_2x_3\ldots,\qquad
            y=0.y_1y_2y_3\ldots,\qquad
            z=0.z_1z_2z_3\ldots .
          \]
          Interleave the digits to form
          \[
            h(x,y,z)\;:=\;0.x_1y_1z_1x_2y_2z_2x_3y_3z_3\ldots\;\in(0,1).
          \]
          Distinct triples $(x,y,z)$ produce distinct digit strings,
          so $h$ is injective.
          
          \medskip
          \textbf{Step 4 (Cardinality comparison).}
          We now have injections
          \[
            (0,1)\xrightarrow{\;f\;}(0,1)^3
            \qquad\text{and}\qquad
            (0,1)^3\xrightarrow{\;h\;}(0,1).
          \]
          By the Cantor--Schröder--Bernstein theorem there exists a \emph{bijection}
          between $(0,1)$ and $(0,1)^3$.
          Hence the two sets have the same cardinality.
          \end{proof}        
      \end{solution}
\end{document}
