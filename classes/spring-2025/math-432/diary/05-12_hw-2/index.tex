\documentclass[12pt]{article}

% Packages
\usepackage[margin=1in]{geometry}
\usepackage{amsmath,amssymb,amsthm}
\usepackage{enumitem}
\usepackage{hyperref}
\usepackage{xcolor}
\usepackage{import}
\usepackage{xifthen}
\usepackage{pdfpages}
\usepackage{transparent}
\usepackage{listings}
\usepackage{tikz}
\usepackage{physics}
\usepackage{siunitx}
\usepackage{booktabs}
\usepackage{cancel}
  \usetikzlibrary{calc,patterns,arrows.meta,decorations.markings}


\DeclareMathOperator{\Log}{Log}
\DeclareMathOperator{\Arg}{Arg}

\lstset{
    breaklines=true,         % Enable line wrapping
    breakatwhitespace=false, % Wrap lines even if there's no whitespace
    basicstyle=\ttfamily,    % Use monospaced font
    frame=single,            % Add a frame around the code
    columns=fullflexible,    % Better handling of variable-width fonts
}

\newcommand{\incfig}[1]{%
    \def\svgwidth{\columnwidth}
    \import{./Figures/}{#1.pdf_tex}
}
\theoremstyle{definition} % This style uses normal (non-italicized) text
\newtheorem{solution}{Solution}
\newtheorem{proposition}{Proposition}
\newtheorem{problem}{Problem}
\newtheorem{lemma}{Lemma}
\newtheorem{theorem}{Theorem}
\newtheorem{remark}{Remark}
\newtheorem{note}{Note}
\newtheorem{definition}{Definition}
\newtheorem{example}{Example}
\newtheorem{corollary}{Corollary}
\theoremstyle{plain} % Restore the default style for other theorem environments
%

% Theorem-like environments
% Title information
\title{MATH-432 HW 2}
\author{Jerich Lee}
\date{\today}

\begin{document}

\maketitle
\begin{problem}
  Let \(f:A\to B\) and \(g:B\to C\) be functions.
  
  \begin{enumerate}[label=(\alph*)]
      \item If \(f\) and \(g\) are \emph{one–to–one} (injective), prove that
            the composition \(g\circ f\) is one–to–one.
      \item If \(g\circ f\) is one–to–one, prove that \(f\) is one–to–one.
      \item If \(f\) is \emph{onto} (surjective) and \(g\circ f\) is one–to–one,
            prove that \(g\) is one–to–one.
      \item Give an explicit example in which \(g\circ f\) is one–to–one
            while \(g\) itself is \emph{not} one–to–one.
  \end{enumerate}
  \end{problem}
  
  \begin{solution}
  Remember that a map \(h:X\to Y\) is \emph{one–to–one} (injective) iff
  \[
  h(x_{1})=h(x_{2})\;\Longrightarrow\;x_{1}=x_{2}\qquad(x_{1},x_{2}\in X),
  \]
  and it is \emph{onto} (surjective) iff for every \(y\in Y\) there exists
  \(x\in X\) with \(h(x)=y\).
  
  \begin{enumerate}[label=(\alph*)]
  %--------------------------------------------------------------
  \item \textbf{\(f\) and \(g\) injective \(\Longrightarrow\) \(g\circ f\) injective.}
  
      Let \(a_{1},a_{2}\in A\) and assume
      \((g\circ f)(a_{1})=(g\circ f)(a_{2})\).
      Then \(g\bigl(f(a_{1})\bigr)=g\bigl(f(a_{2})\bigr)\).
      Since \(g\) is injective, we deduce \(f(a_{1})=f(a_{2})\);
      injectivity of \(f\) now yields \(a_{1}=a_{2}\).
      Hence \(g\circ f\) is injective.
  
  %--------------------------------------------------------------
  \item \textbf{\(g\circ f\) injective \(\Longrightarrow\) \(f\) injective.}
  
      Suppose, toward a contradiction, that \(f\) is \emph{not} injective.
      Then there exist \(a_{1},a_{2}\in A\) with
      \(a_{1}\ne a_{2}\) yet \(f(a_{1})=f(a_{2})\).
      Applying \(g\) we get
      \[
         (g\circ f)(a_{1})=g\bigl(f(a_{1})\bigr)
                         =g\bigl(f(a_{2})\bigr)
                         =(g\circ f)(a_{2}),
      \]
      contradicting injectivity of \(g\circ f\).
      Therefore \(f\) must be injective.
  
  %--------------------------------------------------------------
  \item \textbf{\(f\) surjective and \(g\circ f\) injective \(\Longrightarrow\) \(g\) injective.}
  
      Take any \(b_{1},b_{2}\in B\) with \(g(b_{1})=g(b_{2})\).
      Surjectivity of \(f\) provides \(a_{1},a_{2}\in A\) such that
      \(f(a_{1})=b_{1}\) and \(f(a_{2})=b_{2}\).
      Then
      \[
           (g\circ f)(a_{1}) = g(b_{1}) = g(b_{2}) = (g\circ f)(a_{2}),
      \]
      so injectivity of \(g\circ f\) forces \(a_{1}=a_{2}\),
      which in turn implies \(b_{1}=b_{2}\).
      Hence \(g\) is injective.
  
  %--------------------------------------------------------------
  \item \textbf{An example with \(g\circ f\) injective but \(g\) not injective.}
  
      Let
      \[
          A=(0,\infty),\qquad
          B=\mathbb{R},\qquad
          C=[0,\infty).
      \]
      Define
      \[
          f:A\longrightarrow B,\quad f(x)=x
          \quad(\text{the inclusion map}),
          \qquad
          g:B\longrightarrow C,\quad g(x)=|x|.
      \]
      \begin{itemize}
          \item \(g\) is \emph{not} injective because \(g(-1)=g(1)=1\) but \(-1\ne 1\).
          \item For \(x\in(0,\infty)\) we have \((g\circ f)(x)=g(x)=|x|=x\);
                thus \(g\circ f\) is the identity on \((0,\infty)\), and is therefore
                injective.
      \end{itemize}
      This furnishes the desired example.
  \end{enumerate}
  \end{solution}
  \begin{problem}
    Let \(f:A\to B\) and \(g:B\to C\) be functions.
    
    \begin{enumerate}[label=(\alph*)]
        \item If \(f\) and \(g\) are \emph{onto} (surjective), show that \(g\circ f\) is onto.
        \item If \(g\circ f\) is onto, show that \(g\) is onto.
        \item If \(g\circ f\) is onto and \(g\) is \emph{one–to–one} (injective), show that \(f\) is onto.
        \item Give an example where \(g\circ f\) is onto, but \(f\) is not.
    \end{enumerate}
    \end{problem}
    
    \begin{solution}
    Recall that a map \(h:X\to Y\) is \emph{onto} if for every \(y\in Y\)
    there exists \(x\in X\) with \(h(x)=y\).
    
    \begin{enumerate}[label=(\alph*)]
    %--------------------------------------------------------------
    \item \textbf{\(f\) and \(g\) surjective \(\Longrightarrow\) \(g\circ f\) surjective.}
    
        Fix any \(c\in C\).
        Surjectivity of \(g\) provides \(b\in B\) with \(g(b)=c\).
        Surjectivity of \(f\) provides \(a\in A\) with \(f(a)=b\).
        Then
        \[
           (g\circ f)(a)=g\bigl(f(a)\bigr)=g(b)=c,
        \]
        so every \(c\in C\) is hit by \(g\circ f\); hence \(g\circ f\) is onto.
    
    %--------------------------------------------------------------
    \item \textbf{\(g\circ f\) surjective \(\Longrightarrow\) \(g\) surjective.}
    
        Let \(c\in C\).
        Because \(g\circ f\) is onto, there exists \(a\in A\) with
        \((g\circ f)(a)=c\).
        Set \(b:=f(a)\in B\); then \(g(b)=c\), showing \(g\) is onto.
    
    %--------------------------------------------------------------
    \item \textbf{\(g\circ f\) surjective and \(g\) injective \(\Longrightarrow\) \(f\) surjective.}
    
        Take any \(b\in B\) and define \(c:=g(b)\in C\).
        Since \(g\circ f\) is onto, there exists \(a\in A\) such that
        \((g\circ f)(a)=c\); i.e.\ \(g\bigl(f(a)\bigr)=c=g(b)\).
        Because \(g\) is injective, we must have \(f(a)=b\).
        Thus every \(b\in B\) has a preimage under \(f\), proving \(f\) is onto.
    
    %--------------------------------------------------------------
    \item \textbf{Example where \(g\circ f\) is onto but \(f\) is not.}
    
        Let
        \[
           f:\mathbb{R}\longrightarrow\mathbb{R}^{2},\quad f(x)=(x,0),
           \qquad
           g:\mathbb{R}^{2}\longrightarrow\mathbb{R},\quad g(x,y)=x.
        \]
        Then \((g\circ f)(x)=g(x,0)=x\), the identity map on \(\mathbb{R}\),
        which is onto \(\mathbb{R}\).
        However, \(f\) maps \(\mathbb{R}\) into the \(x\)-axis of \(\mathbb{R}^{2}\)
        and therefore is \emph{not} onto \(\mathbb{R}^{2}\).
    \end{enumerate}
    \end{solution}
    \begin{solution}
      Let \(f:A\to B\) be any function.  
      For a subset \(X\subseteq A\) write \(X^{\prime}:=A\setminus X\)  
      and for \(Y\subseteq B\) write \(Y^{\prime}:=B\setminus Y\).
      
      \begin{enumerate}[label=(\alph*)]
      
      %--------------------------------------------------------------
      \item \textbf{\(f^{-1}\) respects unions.}
      
      For all \(B_{1},B_{2}\subseteq B\) we have
      \[
          f^{-1}(B_{1}\cup B_{2})
          =\{\,a\in A : f(a)\in B_{1}\cup B_{2}\,\}
          =\{\,a\in A : f(a)\in B_{1}\text{ or }f(a)\in B_{2}\,\}
          =f^{-1}(B_{1})\cup f^{-1}(B_{2}).
      \]
      
      %--------------------------------------------------------------
      \item \textbf{\(f^{-1}\) respects complements.}
      
      For every \(T\subseteq B\):
      \[
          \begin{aligned}
              a\in f^{-1}(T^{\prime})
              &\;\Longleftrightarrow\;
              f(a)\in T^{\prime}
              &&\text{(definition of the inverse image)}\\
              &\;\Longleftrightarrow\;
              f(a)\notin T\\
              &\;\Longleftrightarrow\;
              a\notin f^{-1}(T)\\
              &\;\Longleftrightarrow\;
              a\in\bigl(f^{-1}(T)\bigr)^{\prime}.
          \end{aligned}
      \]
      Hence \(f^{-1}(T^{\prime})=\bigl(f^{-1}(T)\bigr)^{\prime}\).
      
      %--------------------------------------------------------------
      \item \textbf{\(f\) respects unions.}
      
      For all \(A_{1},A_{2}\subseteq A\):
      \[
          f(A_{1}\cup A_{2})
          =\{\,f(a):a\in A_{1}\text{ or }a\in A_{2}\,\}
          =\{\,f(a):a\in A_{1}\,\}\cup\{\,f(a):a\in A_{2}\,\}
          =f(A_{1})\cup f(A_{2}).
      \]
      
      %--------------------------------------------------------------
      \item \textbf{Images and complements generally \emph{do not} commute.}
      
      \begin{enumerate}[label=\roman*.]
          \item \emph{A case where \(f^{-1}(A_{1}^{\prime})\nsubseteq(f(A_{1}))^{\prime}\).}
      
                Take \(A=\{0,1\}\), \(B=\{0\}\) and let
                \[
                    f:A\to B,\qquad f(0)=0,\;f(1)=0\quad(\text{a constant map}).
                \]
                Put \(A_{1}=\{0\}\).
                Then
                \[
                    f(A_{1})=\{0\},\qquad (f(A_{1}))^{\prime}=\varnothing,
                    \qquad A_{1}^{\prime}=\{1\},\qquad
                    f^{-1}(A_{1}^{\prime})=\{1\},
                \]
                so \(f^{-1}(A_{1}^{\prime})\not\subseteq(f(A_{1}))^{\prime}\).
      
          \item \emph{A case where \((f(A_{1}))^{\prime}\nsubseteq f^{-1}(A_{1}^{\prime})\).}
      
                Let \(A=\{0\}\), \(B=\{0,1\}\) and define
                \[
                    f:A\to B,\qquad f(0)=0.
                \]
                Again choose \(A_{1}=\{0\}\).
                Then
                \[
                    f(A_{1})=\{0\},\qquad (f(A_{1}))^{\prime}=\{1\},\qquad
                    A_{1}^{\prime}=\varnothing,\qquad f^{-1}(A_{1}^{\prime})=\varnothing,
                \]
                giving \((f(A_{1}))^{\prime}\not\subseteq f^{-1}(A_{1}^{\prime})\).
      \end{enumerate}
      
      These two counter-examples show that neither inclusion  
      \(f^{-1}(A_{1}^{\prime})\subseteq(f(A_{1}))^{\prime}\) nor  
      \((f(A_{1}))^{\prime}\subseteq f^{-1}(A_{1}^{\prime})\) holds in general.
      \end{enumerate}
      \end{solution}
      \begin{solution}
        Let $R\subseteq A\times A$ be a relation that is \emph{reflexive} and \emph{transitive}.
        Define
        \[
           \widetilde R\;:=\;\bigl\{(a,b)\in A\times A : (a,b)\in R\ \text{and}\ (b,a)\in R\bigr\}.
        \]
        
        %%%%%%%%%%%%%%%%%%%%%%%%%%%%%%%%%%%%%%%%%%%%%%%%%%%%%%%%%%%%%%%%%%%%%%%%%%%%%%%%
        \begin{enumerate}[label=(\alph*)]
        
        %-------------------------------------------------------------------------------
        \item \textbf{$\widetilde R$ is an equivalence relation on $A$.}
        
        \begin{enumerate}[label=\textbf{\roman*.},wide=0pt, itemsep=4pt]
          \item \emph{Reflexive.}  
                Because $R$ is reflexive, $(a,a)\in R$ for every $a\in A$.
                Hence $(a,a)\in\widetilde R$ and $\widetilde R$ is reflexive.
        
          \item \emph{Symmetric.}  
                Suppose $(a,b)\in\widetilde R$.  
                Then $(a,b)\in R$ \emph{and} $(b,a)\in R$.  
                Interchanging $a$ and $b$ we see that $(b,a)\in\widetilde R$; thus
                $\widetilde R$ is symmetric.
        
          \item \emph{Transitive.}  
                Let $(a,b),(b,c)\in\widetilde R$.  
                This means $(a,b),(b,c)\in R$ and $(b,a),(c,b)\in R$.
                By transitivity of $R$ we have
                \[
                    (a,c)\in R
                    \quad\text{and}\quad
                    (c,a)\in R.
                \]
                Hence $(a,c)\in\widetilde R$, so $\widetilde R$ is transitive.
        \end{enumerate}
        Since $\widetilde R$ is reflexive, symmetric, and transitive, it is an
        \emph{equivalence relation} on $A$.
        
        %-------------------------------------------------------------------------------
        \item \textbf{A partial order on the $\widetilde R$--equivalence classes.}
        
        For $a\in A$ denote by
        \(
           S_a:=\{\,x\in A : (a,x)\in \widetilde R\,\}
        \)
        the $\widetilde R$--equivalence class of $a$.
        Define a binary relation $\le$ on the set
        \(\mathcal{S}:=\{S_a:a\in A\}\) by
        \[
           S_a\;\le\;S_b
           \;\;\Longleftrightarrow\;\;
           (a,b)\in R.
        \]
        
        \smallskip
        \emph{Well-definedness.}  
        If $a'\in S_a$ and $b'\in S_b$ then $(a,a')\in\widetilde R$ and
        $(b,b')\in\widetilde R$; hence $(a,a'),(a',a),(b,b'),(b',b)\in R$.
        Assuming $(a,b)\in R$, transitivity of $R$ yields
        \((a',b')\in R\).  
        Thus the definition of $\le$ does not depend on the chosen representatives.
        
        \bigskip
        We verify that $\le$ is a \emph{partial order} on $\mathcal{S}$.
        
        \begin{enumerate}[label=\textbf{\roman*.},wide=0pt, itemsep=4pt]
          \item \emph{Reflexive.}  
                For any class $S_a$ we have $(a,a)\in R$ (reflexivity of $R$),
                so $S_a\le S_a$.
        
          \item \emph{Antisymmetric.}  
                Suppose $S_a\le S_b$ and $S_b\le S_a$.
                Then $(a,b)\in R$ and $(b,a)\in R$, i.e.\ $(a,b)\in\widetilde R$,
                whence $S_a=S_b$.
        
          \item \emph{Transitive.}  
                If $S_a\le S_b$ and $S_b\le S_c$, then $(a,b)\in R$ and $(b,c)\in R$.
                By transitivity of $R$, $(a,c)\in R$, so $S_a\le S_c$.
        \end{enumerate}
        Therefore $\le$ is a partial order on the set of
        $\widetilde R$–equivalence classes.
        \end{enumerate}
        \end{solution}
\end{document}
