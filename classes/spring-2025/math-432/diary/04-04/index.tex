\documentclass[12pt]{article}

% Packages
\usepackage[margin=1in]{geometry}
\usepackage{amsmath,amssymb,amsthm}
\usepackage{enumitem}
\usepackage{hyperref}
\usepackage{xcolor}
\usepackage{import}
\usepackage{xifthen}
\usepackage{pdfpages}
\usepackage{transparent}
\usepackage{listings}
\DeclareMathOperator{\Log}{Log}
\DeclareMathOperator{\Arg}{Arg}

\lstset{
    breaklines=true,         % Enable line wrapping
    breakatwhitespace=false, % Wrap lines even if there's no whitespace
    basicstyle=\ttfamily,    % Use monospaced font
    frame=single,            % Add a frame around the code
    columns=fullflexible,    % Better handling of variable-width fonts
}

\newcommand{\incfig}[1]{%
    \def\svgwidth{\columnwidth}
    \import{./Figures/}{#1.pdf_tex}
}
\theoremstyle{definition} % This style uses normal (non-italicized) text
\newtheorem{solution}{Solution}
\newtheorem{proposition}{Proposition}
\newtheorem{problem}{Problem}
\newtheorem{lemma}{Lemma}
\newtheorem{theorem}{Theorem}
\newtheorem{remark}{Remark}
\newtheorem{note}{Note}
\newtheorem{definition}{Definition}
\newtheorem{example}{Example}
\newtheorem{corollary}{Corollary}
\theoremstyle{plain} % Restore the default style for other theorem environments
%

% Theorem-like environments
% Title information
\title{MATH 432 HW 8}
\author{Jerich Lee}
\date{\today}

\begin{document}

\maketitle

\begin{problem}[1]
    
\end{problem}
\begin{solution}
    Let $(M,D)$ be a metric space, and let $x\in M$ be a fixed point. Suppose $r,s$ are real numbers with $0 < r < s$. Define the set
\[
A(x,r,s) \;=\; \{\,y\in M : r < D(x,y) < s \}.
\]
We want to show that $A(x,r,s)$ is open in $M$. By definition of openness in a metric space, it suffices to show that for each $y\in A(x,r,s)$, there exists some $\varepsilon>0$ such that the open ball
\[
B(y,\varepsilon) \;=\; \{\, z \in M : D(y,z) < \varepsilon \}
\]
is contained in $A(x,r,s)$.

\medskip

\noindent \textbf{Step 1.} Pick any point $y\in A(x,r,s)$. Then by definition:
\[
r < D(x,y) < s.
\]

\noindent \textbf{Step 2.} Define
\[
\alpha \;=\; \min\bigl\{\,D(x,y) - r,\; s - D(x,y)\bigr\}.
\]
Note that both $D(x,y) - r$ and $s - D(x,y)$ are strictly positive (since $r < D(x,y) < s$), so $\alpha>0$.

\medskip

\noindent \textbf{Step 3.} We claim that if $z \in M$ satisfies $D(y,z) < \alpha$, then $z \in A(x,r,s)$, i.e., $r < D(x,z) < s$.

\smallskip

\noindent \emph{(a) Lower bound: $D(x,z) > r$.}

Using the reverse triangle inequality,
\[
D(x,y) \; \le\; D(x,z) + D(z,y).
\]
Hence
\[
D(x,z) \;\ge\; D(x,y) - D(y,z).
\]
Since $D(y,z) < \alpha$ and $\alpha \leq D(x,y) - r$, it follows that
\[
D(x,z) \;>\; D(x,y) - \alpha \;\ge\; D(x,y) - (D(x,y) - r) \;=\; r.
\]
Thus $D(x,z) > r$.

\smallskip

\noindent \emph{(b) Upper bound: $D(x,z) < s$.}

Using the usual triangle inequality,
\[
D(x,z) \;\le\; D(x,y) + D(y,z).
\]
Since $D(y,z) < \alpha$ and $\alpha \leq s - D(x,y)$, it follows that
\[
D(x,z) \;<\; D(x,y) + \alpha \;\le\; D(x,y) + \bigl(s - D(x,y)\bigr) \;=\; s.
\]
Hence $D(x,z) < s$.

\medskip

\noindent \textbf{Step 4.} Putting parts (a) and (b) together shows:
\[
r \;<\; D(x,z) \;<\; s,
\]
which means $z \in A(x,r,s)$. Since $z$ was any point with $D(y,z)<\alpha$, this proves
\[
B(y,\alpha) \;\subseteq\; A(x,r,s).
\]

\medskip

\noindent \textbf{Conclusion.} We have shown that for every $y \in A(x,r,s)$ there exists an $\alpha>0$ such that $B(y,\alpha) \subset A(x,r,s)$. Hence $A(x,r,s)$ is open in $M$. \qed
\end{solution}
\begin{problem}[2]
    
\end{problem}
\begin{solution}
    \medskip
\noindent
\underline{\textit{Case 1: $M$ is a \emph{discrete} metric space.}}

\smallskip
\noindent
If $M$ is discrete, then \emph{every} subset of $M$ is open.  (Recall that in a discrete metric space, all singleton sets $\{x\}$ are open; hence arbitrary unions of singletons are open as well.)

Since $M$ is infinite, choose any partition of $M$ into two \emph{infinite} subsets, say
\[
  M \;=\; A \,\cup\, B
\]
where $A$ and $B$ are disjoint, each infinite.  Define
\[
  U \;=\; A.
\]
Because $A\subseteq M$ is a subset of a discrete metric space, $U=A$ is open.  By construction:
\[
  U \;=\; A \quad\text{is infinite,}
  \quad\text{and}\quad
  M\setminus U \;=\; B \quad\text{is infinite.}
\]
Hence $U$ is the required open set with infinite complement.

\bigskip
\noindent
\underline{\textit{Case 2: $M$ is \emph{not} a discrete metric space.}}

\smallskip
\noindent
If $M$ is not discrete, then there exists at least one \emph{limit point} $x \in M$.  By definition of limit point, every open ball around $x$ contains infinitely many distinct points.  That is,
\[
  \forall r>0,\quad B(x,r) \;=\; \{\,y\in M : D(x,y)<r\}
  \quad\text{is infinite.}
\]
We now argue that one can choose $r>0$ so that $B(x,r)$ does \emph{not} cover all of $M$.  Indeed, if $D$ is an unbounded metric, then there exist points arbitrarily far from $x,$ so certainly not all of $M$ can lie in some small ball $B(x,r).$  If instead $M$ is bounded with \emph{diameter} $d = \sup\{D(a,b) : a,b\in M\}$, then simply pick $r = \tfrac{d}{2}$ (or any number strictly between $0$ and $d$).  Because $d$ is the \emph{supremum} of distances, there must be some point $y\in M$ with $D(x,y)$ close to $d,$ so $y\notin B(x,r).$

Hence, we obtain an $r>0$ such that:
\[
  B(x,r)\quad\text{is infinite}
  \quad\text{and}\quad
  M \setminus B(x,r)\;\neq\;\varnothing.
\]
But in fact $M \setminus B(x,r)$ must also be infinite: either the space is unbounded and thus has infinitely many points outside $B(x,r)$, or (in the bounded case) by picking $y_1,y_2,\dots$ all nearly at distance $d$ from $x$, infinitely many such points lie outside a smaller ball.

Hence we can take
\[
  U \;=\; B(x,r).
\]
This $U$ is open by definition of an open ball, it is infinite, and its complement $M\setminus U$ is also infinite.  This completes the proof in the second case.

\medskip
\noindent
\textbf{Conclusion.} In both cases (discrete or not), we have produced an open set
$U \subseteq M$ with the property that $U$ is infinite and $M \setminus U$ is infinite.
\qed
\end{solution}
\begin{problem}[3]
    
\end{problem}
\begin{solution}
    \noindent
    \textbf{(a) Monotonicity of closure: $A \subset B \implies \overline{A} \subset \overline{B}$.}
    
    \smallskip
    \noindent
    \emph{Proof.} Assume $A \subset B$. We want to show $\overline{A} \subset \overline{B}$.
    \begin{itemize}
    \item Take any point $x \in \overline{A}$. By definition of closure, \emph{every} open ball (or open set) around $x$ intersects $A$.
    \item Since $A \subset B$, any open ball around $x$ that meets $A$ must also meet $B$. Hence $x$ is also in the closure of $B$.
    \end{itemize}
    Thus $x \in \overline{B}$, proving $\overline{A} \subset \overline{B}$. \qed
    
    \bigskip
    
    \noindent
    \textbf{(b) Union property of closures: $\overline{A \cup B} = \overline{A} \cup \overline{B}$.}
    
    \smallskip
    \noindent
    \emph{Proof.} We prove the two inclusions:
    
    \begin{enumerate}
    \item[\textbf{(i)}] $\overline{A \cup B} \supset \overline{A} \cup \overline{B}$.
    
    \smallskip
    Any set is contained in its own union, i.e.\ $A \subset A \cup B$ and $B \subset A \cup B$. Hence by part (a) (monotonicity),
    \[
    \overline{A} \;\subset\; \overline{A \cup B}
    \quad\text{and}\quad
    \overline{B} \;\subset\; \overline{A \cup B}.
    \]
    Taking the union of both sides,
    \[
    \overline{A} \,\cup\, \overline{B}
    \;\subset\;
    \overline{A \cup B}.
    \]
    
    \item[\textbf{(ii)}] $\overline{A \cup B} \subset \overline{A} \cup \overline{B}$.
    
    \smallskip
    Suppose, for contradiction, there exists a point $x \in \overline{A \cup B}$ such that 
    \[
    x \;\notin\; \overline{A} \cup \overline{B}.
    \]
    Then $x \notin \overline{A}$ and $x \notin \overline{B}$. By definition of closure, this means:
    \[
    \exists\;U_x \;\text{(open) with } x \in U_x, \text{ such that } U_x \cap A = \varnothing,
    \]
    and similarly
    \[
    \exists\;V_x \;\text{(open) with } x \in V_x, \text{ such that } V_x \cap B = \varnothing.
    \]
    Hence the intersection $U_x \cap V_x$ is an open neighborhood of $x$ that meets neither $A$ nor $B$, so it meets neither $A \cup B$. Thus
    \[
    (U_x \cap V_x)\;\cap\;(A \cup B) \;=\;\varnothing,
    \]
    contradicting $x \in \overline{A \cup B}$ (which would require \emph{every} open set around $x$ to intersect $A \cup B$).
    
    Therefore no such $x$ can exist, and we conclude
    \[
    \overline{A \cup B} \;\subset\; \overline{A} \,\cup\, \overline{B}.
    \]
    \end{enumerate}
    
    Since both inclusions hold, we get
    \[
    \overline{A \cup B}
    \;=\;
    \overline{A} \,\cup\, \overline{B}.
    \]
    \qed
    
    \bigskip
    
    \noindent
    \textbf{(c) Intersection property: does $\overline{A \cap B} = \overline{A} \cap \overline{B}$ always hold?}
    
    \smallskip
    \noindent
    \emph{Answer:} In general, we do \emph{not} have equality. However, we always have
    \[
    \overline{A \cap B}
    \;\subset\;
    \overline{A} \,\cap\, \overline{B}.
    \]
    \emph{Proof of the inclusion:} If $x \in \overline{A \cap B}$, then every open neighborhood of $x$ intersects $A \cap B$, and thus meets \emph{both} $A$ and $B$. Hence every open neighborhood of $x$ intersects $A$ and also intersects $B$, implying $x \in \overline{A}$ and $x \in \overline{B}$. Thus $x \in \overline{A} \cap \overline{B}$.
    
    \smallskip
    \noindent
    \emph{Why not equality in general?} A standard counterexample in $\mathbb{R}$ with the usual metric is:
    \[
    A \;=\;(0,1), \quad B \;=\;(1,2).
    \]
    Then
    \[
    A \cap B \;=\;\varnothing \quad\Longrightarrow\quad
    \overline{A \cap B} \;=\;\varnothing,
    \]
    while
    \[
    \overline{A} \;=\;[0,1],
    \quad
    \overline{B} \;=\;[1,2],
    \quad
    \overline{A} \cap \overline{B} \;=\;\{\,1\,\}.
    \]
    Thus
    \[
    \varnothing \;=\;\overline{A \cap B}
    \;\neq\;
    \{\,1\}
    \;=\;
    \overline{A} \cap \overline{B}.
    \]
    Hence $\overline{A \cap B} = \overline{A} \cap \overline{B}$ does \emph{not} hold in general.     
\end{solution}
\begin{problem}[4]
    
\end{problem}
\begin{solution}
    \smallskip
    \noindent
    Consider the function $f : M \to \mathbb{R}$ given by 
    \[
    f(y) \;=\; D(y,x).
    \]
    This function is continuous because the metric $D$ itself satisfies the triangle inequality; more explicitly, for any $y,z \in M$,
    \[
    |f(y) - f(z)| \;=\; \bigl|D(y,x) - D(z,x)\bigr|
    \;\le\; D(y,z).
    \]
    Hence if $D(y,z)$ is small, then $|f(y)-f(z)|$ is also small, showing continuity of $f$.
    
    \bigskip
    \noindent
    \textbf{(a)}~\emph{Closed ball is closed.}
    
    \smallskip
    \noindent
    The set
    \[
    \overline{B}_r(x)
    \;=\;
    \{\,y \in M : D(y,x) \le r\}
    \;=\;
    f^{-1}\bigl((-\infty,r]\bigr).
    \]
    Since $(-\infty,r]$ is a closed subset of $\mathbb{R}$, its preimage under a continuous map is closed in $M$. Thus $\overline{B}_r(x)$ is closed.
    
    \bigskip
    \noindent
    \textbf{(b)}~\emph{Sphere is closed.}
    
    \smallskip
    \noindent
    The sphere of radius $r$ is
    \[
    \{\,y \in M : D(y,x) = r\}
    \;=\;
    f^{-1}\bigl(\{\,r\,\}\bigr).
    \]
    Since $\{\,r\,\}$ is closed in $\mathbb{R}$, its preimage under $f$ is closed in $M$. Hence the sphere is closed.
    
    \bigskip
    \noindent
    \textbf{(c)}~\emph{Closed annulus is closed.}
    
    \smallskip
    \noindent
    The closed annulus is
    \[
    A(x;r,s)
    \;=\;
    \{\,y \in M : r \le D(y,x) \le s\}
    \;=\;
    f^{-1}\bigl([\,r,s\,]\bigr).
    \]
    Since $[\,r,s\,]$ is a closed interval in $\mathbb{R}$, again its preimage under the continuous map $f$ is closed in $M$. Therefore $A(x;r,s)$ is closed.
    
    \bigskip
    \noindent
    \textbf{Conclusion.} In each case, the given set is the inverse image of a closed set in $\mathbb{R}$ under the continuous function $y \mapsto D(y,x)$. Hence it is closed in $M$. \quad $\Box$     
\end{solution}
\begin{problem}[5]
    
\end{problem}
\begin{solution}
    We break it down into two inequalities:

\medskip
\noindent
\textbf{Step 1:} 
\(
\mathrm{diam}(A) \;\le\; \mathrm{diam}(\overline{A}).
\)
Since $A \subseteq \overline{A}$, any pair of points $x,y \in A$ are also in $\overline{A}$. Hence
\[
\sup\{\,D(x,y)\colon x,y\in A\}
\;\le\;
\sup\{\,D(x,y)\colon x,y\in \overline{A}\}.
\]
This means $\mathrm{diam}(A) \le \mathrm{diam}(\overline{A})$.

\medskip
\noindent
\textbf{Step 2:}
\(
\mathrm{diam}(\overline{A}) \;\le\; \mathrm{diam}(A).
\)
Take any $x,y \in \overline{A}$. By definition of closure, there exist sequences $(x_n)$ and $(y_n)$ in $A$ such that $x_n \to x$ and $y_n \to y$. The distance function $D$ is continuous, so
\[
D(x,y)
\;=\;
\lim_{n\to\infty} D(x_n,y_n).
\]
But for each $n$, $D(x_n,y_n) \le \mathrm{diam}(A)$ (since $x_n,y_n \in A$). Hence taking the limit as $n\to\infty$,
\[
D(x,y)
\;\le\;
\mathrm{diam}(A).
\]
Because $x,y$ were arbitrary in $\overline{A}$, we conclude
\[
\sup\{\,D(x,y)\colon x,y\in \overline{A}\}
\;\le\;
\mathrm{diam}(A),
\]
i.e.\ $\mathrm{diam}(\overline{A}) \le \mathrm{diam}(A)$.

\medskip
\noindent
\textbf{Conclusion.}
Combining both inequalities yields
\[
\mathrm{diam}(A)
\;=\;
\mathrm{diam}(\overline{A}),
\]
as required. \quad $\Box$

\end{solution}
\end{document}
