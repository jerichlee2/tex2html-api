\documentclass[12pt]{article}

% Packages
\usepackage[margin=1in]{geometry}
\usepackage{amsmath,amssymb,amsthm}
\usepackage{enumitem}
\usepackage{hyperref}
\usepackage{xcolor}
\usepackage{import}
\usepackage{xifthen}
\usepackage{pdfpages}
\usepackage{transparent}
\usepackage{listings}


\lstset{
    breaklines=true,         % Enable line wrapping
    breakatwhitespace=false, % Wrap lines even if there's no whitespace
    basicstyle=\ttfamily,    % Use monospaced font
    frame=single,            % Add a frame around the code
    columns=fullflexible,    % Better handling of variable-width fonts
}

\newcommand{\incfig}[1]{%
    \def\svgwidth{\columnwidth}
    \import{./Figures/}{#1.pdf_tex}
}
\theoremstyle{definition} % This style uses normal (non-italicized) text
\newtheorem{solution}{Solution}
\newtheorem{proposition}{Proposition}
\newtheorem{problem}{Problem}
\newtheorem{lemma}{Lemma}
\newtheorem{theorem}{Theorem}
\newtheorem{remark}{Remark}
\newtheorem{note}{Note}
\newtheorem{definition}{Definition}
\newtheorem{example}{Example}
\theoremstyle{plain} % Restore the default style for other theorem environments
%

% Theorem-like environments
% Title information
\title{}
\author{Jerich Lee}
\date{\today}

\begin{document}

\maketitle
\section*{Maximal Linearly Independent Sets via Zorn's Lemma}

\noindent
\textbf{Claim.} \emph{Every vector space admits a maximal linearly independent set. In fact, such a set is a basis for the space.}

\bigskip
\noindent
\textbf{Proof (using Zorn's Lemma).}

\medskip
\noindent
\textbf{Step 1. Construct the partially ordered set.}\\
Let $V$ be a vector space over a field $\mathbf{F}$. Define 
\[
\mathcal{S} \;=\; \{\,S \subseteq V \;\mid\; S \text{ is linearly independent}\}.
\]
We partially order $\mathcal{S}$ by \emph{set inclusion}. That is, for $S_1, S_2 \in \mathcal{S}$, we say 
\[
S_1 \le S_2 \quad \Longleftrightarrow \quad S_1 \subseteq S_2.
\]

\medskip
\noindent
\textbf{Step 2. Chains have upper bounds.}\\
A \emph{chain} in $\mathcal{S}$ is a totally ordered subset $\mathcal{C} \subseteq \mathcal{S}$. Define
\[
S^* \;=\; \bigcup_{S \in \mathcal{C}} S.
\]
We claim $S^* \in \mathcal{S}$. Indeed, suppose there is a nontrivial linear combination of finitely many vectors in $S^*$ that sums to zero:
\[
\alpha_1 v_1 + \cdots + \alpha_k v_k \;=\; 0,
\]
where each $v_j \in S^*$. By total ordering, there is a single $S \in \mathcal{C}$ that contains all the $v_j$. Since $S$ is linearly independent, the only solution is $\alpha_1 = \cdots = \alpha_k = 0$. Hence $S^*$ is also linearly independent and is therefore an upper bound for $\mathcal{C}$ in $\mathcal{S}$.

\medskip
\noindent
\textbf{Step 3. Apply Zorn's Lemma.}\\
By hypothesis, $\mathcal{S}$ is nonempty (it contains the empty set, which is vacuously linearly independent), and every chain in $\mathcal{S}$ has an upper bound. Therefore, by Zorn's Lemma, there exists a \emph{maximal element} in $\mathcal{S}$. Call that maximal element $B$. Thus $B$ is linearly independent, and for any $S \supseteq B$ with $S \in \mathcal{S}$, we must have $S=B$. In other words, $B$ is \emph{maximal linearly independent}.

\medskip
\noindent
\textbf{Step 4. A maximal independent set spans $V$.}\\
We claim $B$ spans $V$. Suppose not. Then there is some $v \in V \setminus \mathrm{span}(B)$. In that case,
\[
B \,\cup\, \{v\}
\]
would still be linearly independent (since $v$ is outside the span of $B$). Hence $B \cup \{v\} \in \mathcal{S}$ and strictly contains $B$, contradicting the maximality of $B$. Therefore no such $v$ exists, and so $B$ spans $V$.

\medskip
\noindent
Since $B$ is linearly independent and spans $V$, it is indeed a \emph{basis} of $V$. This completes the proof.

\begin{lemma}[Lemma 1]
    \label{lemma:1}
    \textit{Let $L$ be a partially ordered set in which every chain has a 
    least upper bound.  Then there cannot exist a function 
    $g : L \to L$ such that for every $x \in L$, we have $g(x) > x$ 
    \emph{and} there is no element strictly between $x$ and $g(x)$.}
    \end{lemma}
    
    \begin{proof}[Proof (by contradiction)]
    \textbf{Step 1. Assume the existence of $g$.}
    
    Suppose, for contradiction, that there is a function 
    $g : L \to L$ with the property
    \[
       \forall x \in L,\quad x < g(x)
       \quad\text{and no $y \in L$ satisfies } x < y < g(x).
    \]
    We aim to derive a contradiction from this assumption.
    
    \medskip
    
    \textbf{Step 2. Restrict to the set above a chosen element.}
    
    Choose some $z_0 \in L$.  Let
    \[
       L_0 \;=\; \{\,a \in L : a \ge z_0\}.
    \]
    Since $z_0 \le a$ for all $a \in L_0$, and $g$ raises each element, 
    we may work only inside $L_0$.  By relabeling, we may simply assume 
    that \emph{$z_0$ is the minimum element of $L$}.  (If not originally so, 
    we just rename $L_0$ by $L$.)
    
    \medskip
    
    \textbf{Step 3. Define a \emph{tower}.}
    
    Following terminology used by Halmos, we call a subset 
    $A \subseteq L$ a \emph{tower} if it satisfies:
    \begin{itemize}
    \item[(a)] $z_0 \in A$;
    \item[(b)] $g(A) \,\subseteq\, A$;  (i.e.\ for every $a \in A$, we have $g(a) \in A$);
    \item[(c)] whenever $C \subseteq A$ is a chain, the least upper bound of 
       $C$ (which exists by hypothesis) is also in $A$.
    \end{itemize}
    
    \medskip
    
    \textbf{Step 4. Intersect all towers to form $T$.}
    
    Since $L$ itself is easily seen to be a tower (check the properties),
    we know there is at least one tower in $\mathcal{P}(L)$.  Let
    \[
       T \;=\; \bigcap \{\text{all towers in $L$}\}.
    \]
    It is routine to check (by definition) that $T$ itself is a tower:
    \begin{itemize}
    \item[(i)] $z_0 \in T$ because $z_0 \in A$ for every tower $A$.
    \item[(ii)] if $x \in T$ then $x \in A$ for every tower $A$, hence 
       $g(x) \in A$ for every tower $A$, so $g(x) \in T$. 
    \item[(iii)] if $C \subseteq T$ is a chain, then $C$ is contained in 
       every tower $A$.  By each tower's property (c), the least upper bound 
       of $C$ belongs to each $A$, hence belongs to $T$.
    \end{itemize}
    
    \medskip
    
    \textbf{Step 5. Plan of proof: show $T$ is a chain, reach a contradiction.}
    
    The crucial step is to prove that $T$ itself is a \emph{chain}.  Once we 
    know $T$ is a chain, let $t$ be its least upper bound (which must lie 
    in $T$ by property~(c)).  Then $g(t) \in T$ again by property~(b).  
    But $g(t) > t$ contradicts the fact that $t$ is already an upper bound 
    of $T$.  Hence no such $g$ can exist.
    
    \medskip
    
    \textbf{Step 6. Define $V$ to capture ``comparability'' with all of $T$.}
    
    To show $T$ is a chain, we must show that any two elements of $T$ 
    are comparable.  Equivalently, define
    \[
       V \;=\; \{\,b \in T : b \text{ is comparable with every element of } T\}.
    \]
    Clearly $V \subseteq T$.  If we can show $V$ itself is a tower, 
    then $T \subseteq V$ (because $T$ is the \emph{intersection} of 
    all towers).  Thus we would get $T = V$, and so \emph{every} 
    pair of points in $T$ is comparable.  That is, $T$ is a chain.
    
    \medskip
    
    \textbf{Step 7. Show that $V$ is a tower.}
    
    We must check the three defining properties of a tower 
    (cf.\ (a)--(c) above), but for the set~$V$.
    
    \textbf{(a)} Since $z_0 \in T$, and $z_0$ is trivially comparable 
    with every element of $T$, we have $z_0 \in V$.
    
    \textbf{(c)} If $\{w_i\}$ is a chain in $V$, let $u$ be its least 
    upper bound in $L$.  We must show $u \in V$.  By definition of $V$, 
    every $w_i$ in the chain is comparable with all of $T$, hence 
    $u$ (as their least upper bound) is also forced to be in~$T$ 
    (property~(c) for $T$ itself).  We now argue that $u$ is comparable 
    with all of $T$:  each $w_i$ is comparable with any $t \in T$, and 
    since $u$ is an upper bound for the chain $\{w_i\}$, it follows $u$ 
    and $t$ are also comparable.  Thus $u \in V$.  
    
    \textbf{(b)} The key step is: given $b \in V$, show $g(b) \in V$.  
    By definition, $b \in T$ and $b$ is comparable with every element of $T$.  
    We must check that $g(b)$ (which is in $T$ by property~(b) of $T$) 
    is also comparable with every element of $T$.  
    
    \medskip
    
    \textit{Construction of a sub-tower $W$.}  
    Put
    \[
       W \;=\; \{\, w \in T : w \le b \;\text{ or }\; g(b) \le w \}.
    \]
    In words, $W$ consists of elements of $T$ that lie below or above the 
    point $b$ and its ``image'' $g(b)$.  We claim $W$ is a tower.  
    Then, since $b \in V$ means $b$ is comparable with everything in $T$, 
    the same will force $g(b)$ to be comparable with everything in $T$, 
    i.e.\ $g(b) \in V$.
    
    \textit{Check $W$ is a tower.}  
    \begin{itemize}
    \item[(a)] We have $z_0 \le b$ by assumption, so $z_0 \in W$.
    \item[(c)] If $\{w_i\} \subseteq W$ is a chain, let $y$ be its least 
       upper bound in $L$.  We need to show $y \in W$.  
       Each $w_i$ is in $T$, and $T$ is a tower, so $y \in T$.  
       Now we must check that $y \le b$ or $g(b) \le y$.  
       By contradiction, if $b < y < g(b)$, that would insert an element 
       between $b$ and $g(b)$, contradicting the hypothesis on $g$.  
       Hence either $y \le b$ or $g(b) \le y$, so $y \in W$.  
    \item[(b)] For any $w \in W$, we check that $g(w) \in W$.  
       This is exactly the same argument: either $g(w) \le b$ or 
       $g(b) \le g(w)$ must hold (again using that no element can 
       lie strictly between $w$ and $g(w)$, and transitivity).  
       So $g(w) \in W$.  
    \end{itemize}
    
    Hence $W$ is a tower.  But $T$ was defined as the \emph{intersection} 
    of \emph{all} towers, so $T \subseteq W$.  In particular, 
    $g(b) \in W \subseteq T$.  Then by the definition of $W$ and the fact 
    that $b$ is comparable with everything in $T$, $g(b)$ is also forced 
    to be comparable with everything in $T$.  So $g(b) \in V$.
    
    Thus property~(b) for $V$ is verified.
    
    \medskip
    
    \textbf{Step 8. Conclude $V = T$ and hence $T$ is a chain.}
    
    We have shown that $V$ satisfies properties (a),(b),(c), so $V$ is a tower.  
    By definition, $T$ is the intersection of all towers, so $T \subseteq V$.  
    But also $V \subseteq T$ by definition, hence $V = T$.  Therefore 
    \emph{every} pair of elements in $T$ is comparable.  In other words, 
    $T$ is a chain.
    
    \medskip
    
    \textbf{Step 9. Final contradiction.}
    
    Since $T$ is a chain, its least upper bound $t$ belongs to $T$.  
    But then $g(t) \in T$ also, by property~(b) of $T$.  However, $g(t) > t$ 
    contradicts the fact that $t$ is an upper bound for $T$.  
    Thus our original assumption that such a $g$ exists must be false.
    
    We have completed the proof of Lemma~\ref{lemma:1}.
    \end{proof}
    
    \bigskip
    
    \begin{lemma}[Lemma 2]
    \label{lemma:2}
    \textit{Let $L$ be a partially ordered set in which every chain has a least 
    upper bound. Then there cannot exist a function $g : L \to L$ satisfying 
    $g(x) > x$ for all $x \in L$ (i.e.\ we do \emph{not} require the 
    ``no-element-in-between'' condition).}
    \end{lemma}
    
    \begin{proof}[Proof idea]
    We adapt the proof of Lemma~\ref{lemma:1} by moving to a new 
    partially ordered set $\mathcal{S}$ whose elements are \emph{chains} in $L$.  
    We order $\mathcal{S}$ by inclusion: $C_1 \le C_2$ iff $C_1 \subseteq C_2$.  
    
    \medskip
    
    \textbf{Step 1. Define a function $h: \mathcal{S} \to \mathcal{S}$.}
    
    For a chain $C \in \mathcal{S}$:
    \[
       h(C) \;=\; 
       \begin{cases}
       C \,\cup\, \{g(x)\}, & \text{if $C$ has a top element $x$,}\\
       C \,\cup\, \{\sup(C)\}, & \text{if $C$ has no top element (the sup exists in $L$).}
       \end{cases}
    \]
    Note that by hypothesis each chain in $L$ has a least upper bound in $L$.  
    Hence $h(C)$ is obtained from $C$ by adjoining exactly one new element 
    (either $g(x)$ or the least upper bound of $C$).
    
    \medskip
    
    \textbf{Step 2. Check that $\mathcal{S}$ has least upper bounds.}
    
    In $\mathcal{S}$, a \emph{chain of chains} (i.e.\ a collection 
    $\{C_i\}_{i \in I}$ where each $C_i \subseteq L$) can be shown 
    to have a least upper bound under inclusion (it is the union of the $C_i$).  
    Hence $\mathcal{S}$ is again a partially ordered set in which every chain 
    has a least upper bound.
    
    \medskip
    
    \textbf{Step 3. Apply Lemma~\ref{lemma:1} to $\mathcal{S}$ and $h$.}
    
    If there were a function $g : L \to L$ with $g(x) > x$ for all $x \in L$, 
    then we obtain $h : \mathcal{S} \to \mathcal{S}$ with the analogous 
    property (``$h(C)$ is always strictly bigger than $C$ in the sense of 
    inclusion'').  Lemma~\ref{lemma:1} now tells us such an $h$ cannot exist 
    in a partially ordered set with all chains having a least upper bound.  
    Hence no such $g$ can exist on $L$.  
    
    This completes the proof of Lemma~\ref{lemma:2}.
    \end{proof}

    \section*{Why Lemma 2 does not require ``immediate jumps''}

The core reason is that in \textbf{Lemma\,2}, we start with an \emph{arbitrary}
strictly-increasing function 
\[
   g : L \,\longrightarrow\, L 
   \quad\text{with}\quad 
   x < g(x)\quad\text{for all }x\in L,
\]
\emph{but} we do not require ``no element between $$x$$ and $$g(x)$$.'' 
We then convert this general $$g$$ into a new function 
\[
   h : \mathcal{S} \,\longrightarrow\, \mathcal{S},
\]
where:
\begin{itemize}
\item $$\mathcal{S}$$ is the set of all \emph{chains} in $$L$$ (ordered by inclusion),
\item for each chain $$C\subseteq L$$, 
\[
   h(C)\;=\;
   \begin{cases}
     C \cup \{\;g(x)\}, & \text{if $$C$$ has a greatest element $$x$$;}\\
     C \cup \{\;\sup C\}, & \text{if $$C$$ has no greatest element (its sup exists in $$L$$).}
   \end{cases}
\]
\end{itemize}

\medskip

\noindent
\textbf{Key observations:}
\begin{enumerate}
\item $$\mathcal{S}$$ is again a partially ordered set (by inclusion) in which
every chain has a least upper bound (the union of chains).
\item Each application of $$h$$ strictly enlarges $$C$$. In the partial order
of set inclusion, we have $$C < h(C)$$ and there is no chain strictly in
between $$C$$ and $$h(C)$$. Hence $$h$$ behaves like a ``covering'' map.
\end{enumerate}

Thus, in $$\mathcal{S}$$, the function $$h$$ \emph{does} have the immediate-jump property
(for chains). \textbf{Lemma\,1} directly applies here: it says no function can
``cover'' every element of a poset in which every chain has an LUB. 
Since we derive a contradiction in $$\mathcal{S}$$, it follows that our initial
assumption (the existence of a strictly-increasing $$g$$ on $$L$$) must be false.

\medskip

Therefore, \textbf{Lemma\,2} does not require $$g(x)$$ to be an
\emph{immediate} successor of $$x$$. By working in the higher-level poset
$$\mathcal{S}$$, we \emph{force} an ``immediate jump'' condition on $$h$$, 
which then contradicts Lemma\,1. Hence any strictly-increasing $$g$$ is ruled out, 
whether or not it features immediate jumps in the original set $$L$$.
\[
C \;\subsetneq\; C\cup\{b{\prime}\}
\quad\text{and}\quad
C\cup\{b{\prime}\} \;\subsetneq\; C\cup\{b\},
\]


\section*{Lemma 2 Walkthrough (with Discussion of Intermediate Elements)}

\begin{lemma}[Lemma 2]
\label{lemma:2}
\textit{Let $$L$$ be a partially ordered set in which every chain has a least 
upper bound.  Then there cannot exist a function $$g : L \to L$$ satisfying 
$$g(x) > x$$ for all $$x \in L$$.}
\end{lemma}

\begin{proof}[Sketch of Proof]
\noindent
\textbf{Goal:} Show that no function $$g\colon L\to L$$ can satisfy 
\[
   x < g(x)
   \quad\text{for every x in L},
\]
under the hypothesis that \emph{every chain in $$L$$ has a least upper bound (LUB)}.

\medskip

\noindent
\textbf{Step 1. Form the poset $$\mathcal{S}$$ of all chains in $$L$$.}

\begin{itemize}
\item An element of $$\mathcal{S}$$ is a \textit{chain} $$C\subseteq L$$. 
  (A chain means any two distinct elements in $$C$$ are comparable in $$L$$.)
\item We partially order $$\mathcal{S}$$ by \emph{set inclusion}: for two chains 
  $$C_1,C_2\subseteq L$$,
  \[
    C_1 \;\le\; C_2 \quad\Longleftrightarrow\quad 
    C_1 \;\subseteq\; C_2.
  \]
\end{itemize}

\noindent
\textbf{Claim:} $$\mathcal{S}$$ also has the property that \textit{every chain in $$\mathcal{S}$$ has a least upper bound}. 
Indeed, a \emph{chain of chains} $$\{C_i\}_{i\in I} \subset \mathcal{S}$$ has a natural LUB: simply take
\[
  \bigcup_{i\in I} C_i 
  \;\;\; (\subseteq L).
\]
Because each $$C_i$$ is a chain in $$L$$, any two points in the union come from some $$C_i$$ or $$C_j$$, so they remain comparable. Hence the union is itself a chain in $$L$$. This makes 
$$\bigcup_{i\in I} C_i$$ an \emph{upper bound} for all $$C_i$$, and in fact the \emph{least} one under inclusion.

\bigskip

\noindent
\textbf{Step 2. Construct a function $$h\colon \mathcal{S}\to \mathcal{S}$$ using $$g$$.}

Assume, for contradiction, that such a strictly‐increasing function $$g$$ does exist on $$L$$.  
We will define $$h$$ on $$\mathcal{S}$$ by adding exactly one element to each chain:
\[
   h(C) \;=\;
   \begin{cases}
     C \,\cup\, \{g(x)\}, & \text{if $C$ has a top element $x \in L$,}\\[6pt]
     C \,\cup\, \{\sup(C)\}, & \text{if $C$ has no greatest element (its supremum exists).}
   \end{cases}
\]

In either case, $$h(C)\supsetneq C$$. In the poset $$\mathcal{S}$$ (ordered by \emph{inclusion}), this means $$C < h(C)$$.  
Moreover, because we are adding a single point that “completes” a jump (namely $$g(x)$$ or $$\sup C$$), there is \emph{no intermediate chain} $$D$$ with
\[
   C \;\subsetneq\; D \;\subsetneq\; h(C).
\]
Hence $$h$$ has the "cover-like" property from Lemma 1.

\bigskip

\noindent
\textbf{Step 3. Why not use an intermediate $$\boldsymbol{b}$$ with $$x<b<g(x)$$?}

A natural question is: \emph{why do we pick exactly $$g(x)$$ rather than some 
$$b$$ in the interval $$(x,g(x))$$?} The reason is that \textbf{we need $$h(C)$$ to cover $$C$$ in the inclusion order}, i.e.\ there should be no chain strictly between $$C$$ and $$h(C)$$.

- If you pick an intermediate point $$b$$ with 
  \[
    x < b < g(x),
  \]
  then very often there can be another point $$b'$$ in $$(x,b)$$ or $$(b,\,g(x))$$. 
  If that happens, one can form a chain $$C \cup \{\,b'\}$$ (assuming $$b'$$ is comparable with all of $$C$$), and you get
  \[
    C
    \;\subsetneq\;
    C \cup \{\,b'\}
    \;\subsetneq\;
    C \cup \{\,b\}.
  \]
  That would destroy the “cover” property. So the “safe choice” is 
  \emph{all the way up} to $$g(x)$$ or else $$\sup(C)$$ so that there is no “room” left in between.

In other words, we specifically ensure $$\textit{no}$$ strictly larger chain than $$C$$ but strictly smaller than $$h(C)$$. This is crucial for applying Lemma 1, which rules out precisely that style of "covering map" in posets that have the LUB property.

\bigskip

\noindent
\textbf{Step 4. Apply Lemma\,1 in $$\boldsymbol{\mathcal{S}}$$.}

By the property of $$\mathcal{S}$$ (every chain in $$\mathcal{S}$$ has an LUB), 
Lemma 1 says there cannot be a function that "covers" every element in $$\mathcal{S}$$. 
But $$h$$ is exactly such a covering function ($$C < h(C)$$ and no chain in between). 
Hence a contradiction arises in $$\mathcal{S}$$. 

Therefore, our initial assumption---that $$g\colon L\to L$$ is strictly increasing
($$x<g(x)$$ for all $$x$$)---must be false.

\bigskip

\noindent
\textbf{Conclusion.} 
No strictly‐increasing function $$g$$ can exist on $$L$$. This completes the proof 
of \textbf{Lemma\,\ref{lemma:2}}.
\end{proof}
\end{document}
