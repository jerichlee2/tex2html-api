\documentclass[12pt]{article}

% Packages
\usepackage[margin=1in]{geometry}
\usepackage{amsmath,amssymb,amsthm}
\usepackage{enumitem}
\usepackage{hyperref}
\usepackage{xcolor}
\usepackage{import}
\usepackage{xifthen}
\usepackage{pdfpages}
\usepackage{transparent}
\usepackage{listings}
\usepackage{tikz}
\usepackage{physics}
\usepackage{siunitx}
\usepackage{booktabs}
\usepackage{cancel}
  \usetikzlibrary{calc,patterns,arrows.meta,decorations.markings}


\DeclareMathOperator{\Log}{Log}
\DeclareMathOperator{\Arg}{Arg}

\lstset{
    breaklines=true,         % Enable line wrapping
    breakatwhitespace=false, % Wrap lines even if there's no whitespace
    basicstyle=\ttfamily,    % Use monospaced font
    frame=single,            % Add a frame around the code
    columns=fullflexible,    % Better handling of variable-width fonts
}

\newcommand{\incfig}[1]{%
    \def\svgwidth{\columnwidth}
    \import{./Figures/}{#1.pdf_tex}
}
\theoremstyle{definition} % This style uses normal (non-italicized) text
\newtheorem{solution}{Solution}
\newtheorem{proposition}{Proposition}
\newtheorem{problem}{Problem}
\newtheorem{lemma}{Lemma}
\newtheorem{theorem}{Theorem}
\newtheorem{remark}{Remark}
\newtheorem{note}{Note}
\newtheorem{definition}{Definition}
\newtheorem{example}{Example}
\newtheorem{corollary}{Corollary}
\theoremstyle{plain} % Restore the default style for other theorem environments
%

% Theorem-like environments
% Title information
\title{MATH-432 HW 4}
\author{Jerich Lee}
\date{\today}

\begin{document}

\maketitle

\begin{solution}
  \begin{theorem}
    Let $(L,\le)$ be a lattice in which \emph{every} chain has an upper bound.
    Then $L$ possesses a \emph{unique} maximal element.
    \end{theorem}
    
    \begin{proof}
    We argue in carefully itemised steps.
    
    \medskip
    \textbf{Step 1 (Existence via Zorn’s Lemma).}
    Because every chain (totally ordered subset) of $L$ has an upper bound
    in $L$, Zorn’s lemma guarantees the existence of at least one maximal
    element of $L$.
    Fix such an element and denote it by $m$.
    
    \medskip
    \textbf{Step 2 (Assume two maximal elements and derive a contradiction).}
    Suppose, for the sake of uniqueness, that $a_1,a_2\in L$ are \emph{both}
    maximal elements (we do not assume $a_1=a_2$).
    
    \medskip
    \textbf{Step 3 (Use the lattice join).}
    In a lattice the \emph{join} (least upper bound) $a_1\vee a_2$ exists
    and satisfies
    \[
      a_1\;\le\;a_1\vee a_2
      \quad\text{and}\quad
      a_2\;\le\;a_1\vee a_2.
    \]
    
    \medskip
    \textbf{Step 4 (Maximality forces equality).}
    Because $a_1$ is maximal and $a_1\le a_1\vee a_2$, we must have
    $a_1\vee a_2\le a_1$, hence $a_1\vee a_2=a_1$.
    Applying the same argument to $a_2$ yields $a_1\vee a_2=a_2$.
    Consequently
    \[
      a_1 = a_1\vee a_2 = a_2.
    \]
    
    \medskip
    \textbf{Step 5 (Conclusion).}
    The assumption of two distinct maximal elements led to equality,
    so the maximal element is unique.
    \end{proof}
    \begin{theorem}
      \label{thm:maximal-antichain}
      Every partially ordered set $(L,\le)$ contains a \textbf{maximal
      totally unordered subset} (equivalently, a \emph{maximal antichain}).
      \end{theorem}
      
\end{solution}

\begin{solution}
  \begin{proof}
    We give a detailed, Zorn--lemma style construction.
    
    \bigskip
    \textbf{Step 1 (Build a poset of antichains).}
    Let
    \[
      M \;:=\;
      \{\,A\subseteq L \mid A\ \text{is totally unordered (an antichain)}\}.
    \]
    Partially order $M$ by inclusion:
    $A_1\preceq A_2 \;\Longleftrightarrow\; A_1\subseteq A_2$.
    
    \bigskip
    \textbf{Step 2 (Chains in $M$ have upper bounds).}
    Let
    \(
      C=\{A_\alpha : \alpha\in I\}
      \subseteq M
    \)
    be a chain, i.e.\ for any $\alpha,\beta\in I$ we have
    $A_\alpha\subseteq A_\beta$ or $A_\beta\subseteq A_\alpha$.
    Define the union
    \[
      A_0 \;:=\; \bigcup_{\alpha\in I} A_\alpha.
    \]
    
    \smallskip
    \emph{Claim: $A_0$ is an antichain (hence $A_0\in M$).}
    
    Take any $a,b\in A_0$.
    Then $a\in A_{\alpha}$ and $b\in A_{\beta}$ for some
    $\alpha,\beta\in I$.
    Because $C$ is a chain, assume without loss of generality
    $A_{\alpha}\subseteq A_{\beta}$.
    Thus $a,b\in A_{\beta}$, and $A_{\beta}$ is totally unordered,
    so neither $a<b$ nor $b<a$ in $L$.
    Hence $A_0$ is totally unordered.
    
    \smallskip
    Furthermore, $A_\alpha\subseteq A_0$ for every $\alpha$,
    so $A_0$ is an \textbf{upper bound} of the chain $C$ in $(M,\preceq)$.
    
    \bigskip
    \textbf{Step 3 (Invoke Zorn’s lemma).}
    Every chain in $(M,\preceq)$ has an upper bound in $M$
    (by Step~2), so Zorn’s lemma applies:
    there exists a \emph{maximal} element $A^\ast\in M$.
    
    \bigskip
    \textbf{Step 4 (Interpretation).}
    By definition of $M$,
    $A^\ast$ is a totally unordered subset of $L$,
    and its maximality in $M$ means:
    \[
      A^\ast\subseteq B\in M \;\Longrightarrow\; B=A^\ast.
    \]
    In other words, $A^\ast$ is a \emph{maximal antichain} of $L$.
    
    \bigskip
    \textbf{Step 5 (Conclusion).}
    Thus every partially ordered set $(L,\le)$ possesses
    at least one maximal totally unordered subset, as required.
    \end{proof}
\end{solution}

    \begin{solution}
      \begin{theorem}[Extension (Szpilrajn) theorem]
        \label{thm:poset-extension}
        Let $(L,\le)$ be a partially ordered set.
        There exists a \emph{linear order} (total order) $\preceq$ on $L$
        such that
        \[
          x\le y \;\Longrightarrow\; x\preceq y
          \qquad(x,y\in L).
        \]
        Equivalently, every poset can be extended to a chain.
        \end{theorem}
        
        \begin{proof}
        We give a step–by–step construction that relies on Zorn’s lemma.
        
        \bigskip
        \textbf{Step 1 (Encode partial orders as subsets of $L\times L$).}
        A binary relation \(R\subseteq L\times L\) is a \emph{partial order}
        iff it is reflexive, anti‑symmetric, and transitive.
        Denote by
        \[
          \mathcal P \;=\;
          \bigl\{\,R \subseteq L\times L
                  \mid
                  \le\;\subseteq R
                  \text{ and $R$ is a partial order}
           \bigr\}
        \]
        the collection of \emph{all} partial orders on $L$
        that \emph{extend} the given $\le$.
        
        \bigskip
        \textbf{Step 2 (Partially order $\mathcal P$ by inclusion).}
        For $R_1,R_2\in\mathcal P$ write $R_1\preccurlyeq R_2$ iff
        $R_1\subseteq R_2$.
        This turns $\mathcal P$ itself into a poset.
        
        \bigskip
        \textbf{Step 3 (Chains in $\mathcal P$ have an upper bound).}
        Let
        \(
          \mathcal C=\{R_\alpha : \alpha\in I\}\subseteq\mathcal P
        \)
        be a chain (totally ordered by inclusion).
        Define the union
        \[
          R_0\;:=\;\bigcup_{\alpha\in I} R_\alpha.
        \]
        
        \emph{Claim: $R_0\in\mathcal P$.}
        
        \begin{itemize}
          \item \emph{Reflexivity.}
                Each $R_\alpha$ is reflexive, so $(x,x)\in R_\alpha$ for every
                $x\in L$ and some $\alpha$; hence $(x,x)\in R_0$.
          \item \emph{Anti‑symmetry.}
                Suppose $(x,y)\in R_0$ and $(y,x)\in R_0$.
                Then $(x,y)\in R_{\alpha}$ and $(y,x)\in R_{\beta}$ for some
                $\alpha,\beta$.
                Because $\mathcal C$ is a chain, assume $R_{\alpha}\subseteq R_{\beta}$.
                Then both pairs lie in $R_{\beta}$, and anti‑symmetry of
                $R_{\beta}$ forces $x=y$.
          \item \emph{Transitivity.}
                If $(x,y)\in R_0$ and $(y,z)\in R_0$, choose
                $\alpha,\beta$ with $(x,y)\in R_\alpha$ and $(y,z)\in R_\beta$.
                Without loss of generality $R_\alpha\subseteq R_\beta$,
                so both pairs live in $R_\beta$ and transitivity there
                gives $(x,z)\in R_\beta\subseteq R_0$.
        \end{itemize}
        Hence $R_0$ is a partial order containing $\le$,
        so $R_0\in\mathcal P$ and clearly $R_\alpha\subseteq R_0$
        for all $\alpha$.
        Thus $R_0$ is an \emph{upper bound} of the chain $\mathcal C$.
        
        \bigskip
        \textbf{Step 4 (Invoke Zorn’s lemma).}
        Every chain in $\mathcal P$ has an upper bound in $\mathcal P$ (Step 3);
        therefore, by Zorn’s lemma, $\mathcal P$ possesses a
        \emph{maximal element}, say $R^\ast$.
        
        \bigskip
        \textbf{Step 5 ($R^\ast$ is a total order).}
        Assume, for contradiction, that $R^\ast$ is \emph{not} total:
        there exist $a,b\in L$ with $a\neq b$ such that
        \(
          (a,b)\notin R^\ast \text{ and } (b,a)\notin R^\ast.
        \)
        
        \smallskip
        \emph{Extend $R^\ast$.}
        Define
        \[
          S\;:=\;R^\ast \cup \bigl\{(a,b)\bigr\}
                    \cup \bigl\{(x,b) : (x,a)\in R^\ast\bigr\}
                    \cup \bigl\{(a,y) : (b,y)\in R^\ast\bigr\}.
        \]
        Intuitively, we declare $a\preceq b$ and preserve transitivity by
        “pushing” all predecessors of $a$ below $b$
        and all successors of $b$ above $a$.
        
        One checks directly that $S$ is still
        reflexive, anti‑symmetric, and transitive, and that $R^\ast\subsetneq S$.
        Hence $S\in\mathcal P$, contradicting the maximality of $R^\ast$.
        
        \smallskip
        Therefore such incomparable $a,b$ cannot exist, and $R^\ast$ is a
        \emph{total} (linear) order on $L$ extending $\le$.
        
        \bigskip
        \textbf{Step 6 (Set the desired chain order).}
        Put $\preceq:=R^\ast$.
        Then $(L,\preceq)$ is a chain and $\le\subseteq\preceq$, completing
        the extension.
        
        \end{proof}
    \end{solution}

    \begin{solution}
      \begin{theorem}
        \label{thm:lattice-complete}
        Let $(L,\le)$ be a lattice in which \emph{every} chain
        has both a least upper bound and a greatest lower bound.
        Then $L$ is \emph{complete}; that is, every subset
        $A\subseteq L$ possesses a least upper bound
        $\displaystyle\bigvee A$ and a greatest lower bound
        $\displaystyle\bigwedge A$ in $L$.
        \end{theorem}
        
        \begin{proof}
        We give an explicit construction of $\bigvee A$;
        the dual argument for $\bigwedge A$ is obtained by
        reversing all inequalities.
        
        \smallskip
        \textbf{Step 1 (A top element exists).}
        Problem~1 (proved earlier) shows that under the
        present hypothesis $L$ has a
        \emph{unique maximal element}~$z$ (the “top” of $L$).
        Consequently every subset of $L$ has at least one
        upper bound, namely~$z$.
        
        \smallskip
        \textbf{Step 2 (The poset of upper bounds of $A$).}
        Fix any subset $A\subseteq L$ and define
        \[
          U\;:=\;
          \bigl\{\,u\in L \mid \forall\,a\in A,\; a\le u\bigr\}.
        \]
        Because $z\in U$, the set $U$ is non–empty.
        Endow $U$ with the restriction of~$\le$;
        then $(U,\le)$ is a partially ordered set.
        
        \smallskip
        \textbf{Step 3 (Chains in $U$ have lower bounds \emph{inside} $U$).}
        Let $C\subseteq U$ be a chain.
        Since $C$ is also a chain in~$L$, the hypothesis
        provides its greatest lower bound
        \[
          \ell := \bigwedge C \in L.
        \]
        We claim that $\ell\in U$.
        Indeed, fix $a\in A$.
        Because each $u\in C$ is an upper bound of $A$, we have $a\le u$.
        Hence $a$ is a lower bound of $C$.
        By the definition of $\ell$ we then have $a\le\ell$,
        so $\ell$ is an \emph{upper} bound of $A$, i.e.\ $\ell\in U$.
        Moreover, $\ell\le u$ for every $u\in C$,
        so $\ell$ is a lower bound of the chain $C$ \emph{within} $U$.
        
        \smallskip
        \textbf{Step 4 (Apply the dual form of Zorn’s lemma).}
        A standard equivalent of Zorn’s lemma states:
        
        \begin{quote}
        \emph{If a partially ordered set $(P,\preceq)$ is such that every
        chain in $P$ has a lower bound in $P$, then $P$ contains a
        minimal element.}
        \end{quote}
        
        Because Step~3 verifies the premise for $(U,\le)$,
        there exists a \emph{minimal} element $m\in U$.
        
        \smallskip
        \textbf{Step 5 ($m$ is the least upper bound of $A$).}
        \begin{enumerate}[label=\textup{(\alph*)},wide,labelindent=0pt]
          \item \emph{$m$ is an upper bound of $A$.}
                This is true by construction since $m\in U$.
          \item \emph{Minimality implies leastness.}
                Suppose $u\in U$ is any upper bound of $A$.
                Because $m$ is minimal in $U$, we must have $m\le u$.
        \end{enumerate}
        Thus $m$ is below every other upper bound of $A$,
        so $m=\bigvee A$.
        
        \smallskip
        \textbf{Step 6 (Greatest lower bound).}
        Apply the same argument to the dual lattice $(L,\ge)$
        (or, equivalently, to the set of \emph{lower} bounds of $A$)
        to obtain a maximal element $g$, which is the greatest
        lower bound $\bigwedge A$.
        
        \smallskip
        \textbf{Step 7 (Conclusion).}
        Every subset $A\subseteq L$ admits both $\bigvee A$ and $\bigwedge A$;
        hence $L$ is complete.
        \end{proof}
    \end{solution}

    \begin{solution}
      \begin{theorem}
        Let $\frak c$ denote the cardinality of the real line $\Bbb R$.
        Then
        \[
           \frak c+\frak c \;=\; \frak c .
        \]
        (Here “$+$’’ is cardinal addition, i.e.\ the cardinality of a
        \emph{disjoint} union of two sets of size~$\frak c$.)
        \end{theorem}
        
        \begin{proof}
        We give an explicit, step--by--step argument that avoids the
        invoked Theorems~13 and~14.
        
        \medskip
        \textbf{Step 1 (Two convenient subsets of $\Bbb R$).}
        Put
        \[
          P \;:=\; (0,\infty), \qquad
          N \;:=\; (-\infty,0).
        \]
        The sets $P$ and $N$ are disjoint and
        \(
          P\cup N = \Bbb R\setminus\{0\}.
        \)
        
        \medskip
        \textbf{Step 2 ($|P|=\frak c$).}
        Define
        \[
          g\colon \Bbb R \longrightarrow P,
          \qquad
          g(x)=e^{x}.
        \]
        $g$ is bijective (continuous, strictly increasing, surjective onto~$P$),
        so $|P| = |\Bbb R| = \frak c$.
        
        \medskip
        \textbf{Step 3 ($|N|=\frak c$).}
        Similarly,
        \[
          h\colon \Bbb R \longrightarrow N,
          \qquad
          h(x) = -e^{x},
        \]
        is a bijection, hence $|N|=\frak c$.
        
        \medskip
        \textbf{Step 4 ($|P\cup N|=\frak c$).}
        Because $P\cup N = \Bbb R\setminus\{0\}$ differs from $\Bbb R$
        by just one point, we construct an explicit bijection
        \[
          \varphi\colon \Bbb R \longrightarrow \Bbb R\setminus\{0\},
          \qquad
          \varphi(x)=
          \begin{cases}
             x           & \text{if }x<0,\\[6pt]
             x+1         & \text{if }x\ge 0.
          \end{cases}
        \]
        $\varphi$ is injective, its image omits $0$, and for any
        $y\in\Bbb R\setminus\{0\}$ one checks a pre‑image exists,
        so $\varphi$ is a bijection.
        Thus $|P\cup N| = |\Bbb R\setminus\{0\}| = \frak c$.
        
        \medskip
        \textbf{Step 5 (Compute the cardinal sum).}
        Cardinal addition is defined by
        \[
          \frak c+\frak c \;=\; |P| + |N|
          \;=\; |\,P \sqcup N\,|
          \;=\; |P\cup N|
          \;=\; \frak c .
        \]
        (The middle equality uses that $P$ and $N$ are already disjoint.)
        
        \medskip
        \textbf{Step 6 (Conclusion).}
        Hence $\frak c+\frak c=\frak c$, completing the proof.
        \end{proof}      
    \end{solution}

    \begin{solution}
      \begin{theorem}
        Let $\{A_i\}_{i\in\Bbb N}$ be a countably infinite family of
        \emph{infinite} sets such that
        \(
           \lvert A_i\rvert = d
        \)
        for every $i\in\Bbb N$.
        Then
        \[
           \left\lvert\;\bigcup_{i=1}^{\infty} A_i\;\right\rvert = d .
        \]
        \end{theorem}
        
        \begin{proof}
        Write
        \(
          A \;:=\; \bigcup_{i=1}^{\infty} A_i.
        \)
        
        \medskip
        \textbf{Step 1 (Lower bound).}
        Because $A_1\subseteq A$ and
        $\lvert A_1\rvert=d$, we immediately have
        \[
           d \;\le\; \lvert A\rvert.
        \]
        
        \medskip
        \textbf{Step 2 (Arrange the $A_i$ disjointly).}
        If the $A_i$ overlap, replace each by
        \(
          A_i' := A_i\times\{i\}\subseteq A_i\times\Bbb N
        \);
        each $A_i'$ is in bijection with $A_i$
        (via $(a,i)\mapsto a$), so $\lvert A_i'\rvert=d$.
        These new sets are disjoint, and their union is still
        in bijection with~$A$.
        Thus \emph{without loss of generality} we may assume the original
        $\{A_i\}$ are disjoint.
        
        \medskip
        \textbf{Step 3 (A surjection from $A_1\times\Bbb N$ onto $A$).}
        For each $n\in\Bbb N$ fix a bijection
        \[
          f_n:A_1\longrightarrow A_n.
        \]
        Define
        \[
          f:A_1\times\Bbb N\longrightarrow A,
          \qquad
          f(a,n) := f_n(a).
        \]
        \begin{itemize}
          \item \emph{Well‑defined:} $f_n(a)\in A_n\subseteq A$.
          \item \emph{Surjective:} given any $x\in A$, choose the index
                $n$ with $x\in A_n$; because $f_n$ is onto $A_n$,
                there exists $a\in A_1$ with $f_n(a)=x$, so $f(a,n)=x$.
        \end{itemize}
        Hence $f$ is a surjection
        \(
          A_1\times\Bbb N \twoheadrightarrow A,
        \)
        so
        \[
           \lvert A\rvert \;\le\; \lvert A_1\times\Bbb N\rvert.
        \]
        
        \medskip
        \textbf{Step 4 (Cardinality of $A_1\times\Bbb N$).}
        Since $\lvert A_1\rvert=d$ and $\lvert\Bbb N\rvert=\aleph_0$,
        \[
          \lvert A_1\times\Bbb N\rvert = d\cdot\aleph_0.
        \]
        For any infinite cardinal $d\ge\aleph_0$,
        one has $d\cdot\aleph_0 = d$ (standard cardinal arithmetic).
        
        \medskip
        \textbf{Step 5 (Combine bounds).}
        We now have
        \[
           d \;\le\; \lvert A\rvert \;\le\; d,
        \]
        so $\lvert A\rvert=d$.
        
        \medskip
        \textbf{Step 6 (Conclusion).}
        Therefore
        \(
          \displaystyle
          \left\lvert\bigcup_{i=1}^{\infty} A_i\right\rvert=d,
        \)
        as claimed.
        \end{proof} 
    \end{solution}
\end{document}
