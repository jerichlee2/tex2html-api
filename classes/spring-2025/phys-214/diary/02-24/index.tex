\documentclass[12pt]{article}

% Packages
\usepackage[margin=.5in]{geometry}
\usepackage{amsmath,amssymb,amsthm}
\usepackage{enumitem}
\usepackage{hyperref}
\usepackage{xcolor}
\usepackage{import}
\usepackage{xifthen}
\usepackage{pdfpages}
\usepackage{transparent}
\usepackage{listings}


\lstset{
    breaklines=true,         % Enable line wrapping
    breakatwhitespace=false, % Wrap lines even if there's no whitespace
    basicstyle=\ttfamily,    % Use monospaced font
    frame=single,            % Add a frame around the code
    columns=fullflexible,    % Better handling of variable-width fonts
}

\newcommand{\incfig}[1]{%
    \def\svgwidth{\columnwidth}
    \import{./Figures/}{#1.pdf_tex}
}
\theoremstyle{definition} % This style uses normal (non-italicized) text
\newtheorem{solution}{Solution}
\newtheorem{proposition}{Proposition}
\newtheorem{problem}{Problem}
\newtheorem{lemma}{Lemma}
\newtheorem{theorem}{Theorem}
\newtheorem{remark}{Remark}
\newtheorem{note}{Note}
\newtheorem{definition}{Definition}
\newtheorem{example}{Example}
\theoremstyle{plain} % Restore the default style for other theorem environments
%

% Theorem-like environments
% Title information
\title{}
\author{Jerich Lee}
\date{\today}

\begin{document}

\maketitle
\section*{Step-by-Step Solution in \LaTeX{}}

\subsection*{1. Convert Photon Energy (eV) to Wavelength \(\lambda\)}

\textbf{(a) Convert the photon energy from eV to joules}

\[
E = 1.89 \,\text{eV}, \quad
1 \,\text{eV} = 1.602 \times 10^{-19} \,\text{J}.
\]

\[
E_{\text{J}} = 1.89 \,\text{eV} \times 1.602 \times 10^{-19} \,\frac{\text{J}}{\text{eV}}
= 3.03 \times 10^{-19} \,\text{J}.
\]

\textbf{(b) Use \(\displaystyle E = \frac{hc}{\lambda}\) to solve for \(\lambda\)}

\[
\lambda = \frac{hc}{E_{\text{J}}}.
\]

Using:
\[
h = 6.626 \times 10^{-34}\,\text{J}\cdot\text{s}, 
\quad
c = 3.00 \times 10^{8}\,\text{m/s},
\]

\[
\lambda 
= \frac{(6.626 \times 10^{-34}\,\text{J}\cdot\text{s})(3.00 \times 10^{8}\,\text{m/s})}{3.03 \times 10^{-19}\,\text{J}}
\approx 6.56 \times 10^{-7}\,\text{m}
= 656\,\text{nm}.
\]

\subsection*{2. Determine the Grating Spacing \(d\)}

The grating has \(6000\) lines per centimeter. The spacing \(d\) is the reciprocal:

\[
d = \frac{1}{6000 \,\text{lines/cm}} = 1.6667 \times 10^{-4} \,\text{cm}.
\]

Convert cm to m (\(1\,\text{cm} = 10^{-2}\,\text{m}\)):

\[
d = 1.6667 \times 10^{-4} \,\text{cm} \times 10^{-2}\,\text{m/cm} 
= 1.6667 \times 10^{-6} \,\text{m}.
\]

\subsection*{3. Apply the Diffraction Grating Equation}

For normal incidence and first-order diffraction (\(m=1\)):

\[
d \,\sin \theta = m \,\lambda \quad \Longrightarrow \quad d\,\sin \theta = \lambda 
\quad (\text{since } m=1).
\]

\[
\sin \theta = \frac{\lambda}{d}
= \frac{6.56 \times 10^{-7}\,\text{m}}{1.6667 \times 10^{-6}\,\text{m}}
\approx 0.393.
\]

\subsection*{4. Solve for \(\theta\)}

\[
\theta = \sin^{-1}(0.393) \approx 23^\circ.
\]

\[
\boxed{\theta \approx 23^\circ \text{ (first-order diffraction angle)}}
\]

\section*{Step-by-Step Solution}

\subsection*{Problem Statement}
\noindent
We have two closely spaced Sodium (Na) emission lines at 
\[
\lambda_1 = 589.00\,\text{nm} 
\quad \text{and} \quad 
\lambda_2 = 589.59\,\text{nm}.
\]
A diffraction grating (up to 3.0\,cm wide) is used in second order (\(m=2\)). 
We want to determine the minimum number of lines per centimeter required 
for the grating to \emph{just resolve} these two lines, assuming 
the grating is fully illuminated.

\subsection*{1. Compute the Required Resolving Power}
The separation between the two lines is
\[
\Delta \lambda 
= \lambda_2 - \lambda_1
= 589.59\,\text{nm} - 589.00\,\text{nm} 
= 0.59\,\text{nm}.
\]
Taking an approximate wavelength \(\lambda \approx 589\,\text{nm}\), 
the \emph{resolving power} \(R\) needed is
\[
R 
= \frac{\lambda}{\Delta \lambda}
= \frac{589\,\text{nm}}{0.59\,\text{nm}}
\approx 1000.
\]

\subsection*{2. Relate Resolving Power to Grating Parameters}
For a diffraction grating, the resolving power \(R\) is given by
\[
R = m \, N,
\]
where 
\begin{itemize}
    \item \(m\) is the diffraction order (here \(m = 2\)),
    \item \(N\) is the total number of illuminated slits (or lines).
\end{itemize}
Since \(R = 1000\) and \(m = 2\),
\[
1000 = 2 \, N
\quad \Longrightarrow \quad
N = 500.
\]
Thus, we need at least \(500\) slits to be illuminated by the light 
in order to achieve the required resolution.

\subsection*{3. Determine Lines per Centimeter}
If the grating is fully illuminated across its maximum width of \(3.0\,\text{cm}\), 
then the total number of illuminated lines \(N\) is
\[
N = (\text{lines/cm}) \times (3.0\,\text{cm}).
\]
Let \(L\) be the number of lines per centimeter. Then
\[
N = 3.0 \, L.
\]
We have \(N = 500\), so
\[
3.0 \, L = 500
\quad \Longrightarrow \quad
L = \frac{500}{3.0}
= 166.7 \,\text{lines/cm}.
\]
Rounding suitably,
\[
\boxed{L \approx 167\;\text{lines/cm}.}
\]

\subsection*{Final Answer}
\noindent
\emph{The minimum number of lines per centimeter required 
to just resolve the Na doublet in second order, assuming a 
3.0\,cm fully illuminated grating, is about \(167\)\,lines/cm.}


\section*{Bohr Model of the Hydrogen Atom: First Six Energy Levels}

\subsection*{1. Background}
According to the Bohr model, the electron in a hydrogen atom occupies discrete energy levels 
labeled by the principal quantum number \(n = 1, 2, 3, \dots\). 
The energy of each level is given (in electron volts) by the formula
\[
E_n = -\frac{13.6\,\text{eV}}{n^2},
\]
where \(13.6\,\text{eV}\) is the ground-state (lowest) energy magnitude for hydrogen.

\subsection*{2. Calculate Each Energy Level}
We want the first six energy levels: \(n = 1, 2, 3, 4, 5,\) and \(6\).

\paragraph{Level \(n=1\)}
\[
E_1 
= -\frac{13.6\,\text{eV}}{1^2} 
= -13.6\,\text{eV}.
\]

\paragraph{Level \(n=2\)}
\[
E_2 
= -\frac{13.6\,\text{eV}}{2^2} 
= -\frac{13.6\,\text{eV}}{4} 
= -3.40\,\text{eV}.
\]

\paragraph{Level \(n=3\)}
\[
E_3 
= -\frac{13.6\,\text{eV}}{3^2} 
= -\frac{13.6\,\text{eV}}{9} 
\approx -1.51\,\text{eV}.
\]

\paragraph{Level \(n=4\)}
\[
E_4 
= -\frac{13.6\,\text{eV}}{4^2} 
= -\frac{13.6\,\text{eV}}{16} 
= -0.85\,\text{eV}.
\]

\paragraph{Level \(n=5\)}
\[
E_5 
= -\frac{13.6\,\text{eV}}{5^2} 
= -\frac{13.6\,\text{eV}}{25} 
\approx -0.54\,\text{eV}.
\]

\paragraph{Level \(n=6\)}
\[
E_6 
= -\frac{13.6\,\text{eV}}{6^2} 
= -\frac{13.6\,\text{eV}}{36} 
\approx -0.38\,\text{eV}.
\]

\subsection*{3. Summary of Results}
\[
\begin{aligned}
E_1 &= -13.6\,\text{eV}, \\
E_2 &= -3.40\,\text{eV}, \\
E_3 &= -1.51\,\text{eV}, \\
E_4 &= -0.85\,\text{eV}, \\
E_5 &= -0.54\,\text{eV}, \\
E_6 &= -0.38\,\text{eV}.
\end{aligned}
\]

\noindent
These negative signs indicate that the electron is \emph{bound} to the nucleus; 
the zero of energy is defined as the electron being infinitely far from the nucleus. 
Higher \(n\) values correspond to \emph{less negative} (i.e., higher) energy levels.

\section*{Hydrogen Transitions: Energies, Wavelengths, and EM Regions}

Below are sample calculations and a summary table for the specified electronic transitions in the hydrogen atom. 
All energies are in eV, wavelengths in nm, and the electromagnetic (EM) region is indicated (UV, visible with approximate color, or IR).

\subsection*{1. Energy Differences and Wavelengths}
\noindent
The energy of a hydrogen level with principal quantum number \(n\) is:
\[
E_n = -\frac{13.6\,\text{eV}}{n^2}.
\]
A photon emitted by a transition from \(n_i\) (initial) to \(n_f\) (final) has energy
\[
E_{\gamma} \;=\; E_{n_i} - E_{n_f},
\]
and the corresponding wavelength (in nm) can be found using:
\[
\lambda \,[\text{nm}] \;=\; \frac{1240\,\text{eV}\cdot\text{nm}}{E_{\gamma}\,[\text{eV}]}.
\]

\subsection*{2. Transitions to \texorpdfstring{$n_f=1$}{n=1} (Lyman Series)}
\[
\begin{aligned}
&\textbf{(a)}\; n_i = 2 \;\longrightarrow\; n_f = 1: \\
&\quad E_{2} = -3.40\,\text{eV}, \quad E_{1} = -13.6\,\text{eV}, \\
&\quad E_{\gamma} = E_{2} - E_{1} = (-3.40) - (-13.6) = 10.2\,\text{eV}, \\
&\quad \lambda = \frac{1240}{10.2} \approx 121.6\,\text{nm} \quad (\text{UV region}). \\[6pt]
&\textbf{(b)}\; n_i = 3 \;\longrightarrow\; n_f = 1: \\
&\quad E_{3} = -1.51\,\text{eV}, \quad E_{1} = -13.6\,\text{eV}, \\
&\quad E_{\gamma} = E_{3} - E_{1} = (-1.51) - (-13.6) = 12.09\,\text{eV}, \\
&\quad \lambda = \frac{1240}{12.09} \approx 102.5\,\text{nm} \quad (\text{UV region}).
\end{aligned}
\]

\subsection*{3. Transitions to \texorpdfstring{$n_f=2$}{n=2} (Balmer Series)}
\[
\begin{aligned}
&\textbf{(a)}\; n_i = 3 \;\longrightarrow\; n_f = 2: \\
&\quad E_{3} = -1.51\,\text{eV}, \quad E_{2} = -3.40\,\text{eV}, \\
&\quad E_{\gamma} = E_{3} - E_{2} = (-1.51) - (-3.40) = 1.89\,\text{eV}, \\
&\quad \lambda = \frac{1240}{1.89} \approx 656\,\text{nm} \quad (\text{red, visible}). \\[6pt]
&\textbf{(b)}\; n_i = 4 \;\longrightarrow\; n_f = 2: \\
&\quad E_{4} = -0.85\,\text{eV}, \quad E_{2} = -3.40\,\text{eV}, \\
&\quad E_{\gamma} = E_{4} - E_{2} = (-0.85) - (-3.40) = 2.55\,\text{eV}, \\
&\quad \lambda = \frac{1240}{2.55} \approx 486\,\text{nm} \quad (\text{blue-green, visible}). \\[6pt]
&\textbf{(c)}\; n_i = 5 \;\longrightarrow\; n_f = 2: \\
&\quad E_{5} = -0.54\,\text{eV}, \quad E_{2} = -3.40\,\text{eV}, \\
&\quad E_{\gamma} = E_{5} - E_{2} = (-0.54) - (-3.40) = 2.86\,\text{eV}, \\
&\quad \lambda = \frac{1240}{2.86} \approx 434\,\text{nm} \quad (\text{violet, visible}). \\[6pt]
&\textbf{(d)}\; n_i = 6 \;\longrightarrow\; n_f = 2: \\
&\quad E_{6} = -0.38\,\text{eV}, \quad E_{2} = -3.40\,\text{eV}, \\
&\quad E_{\gamma} = E_{6} - E_{2} = (-0.38) - (-3.40) = 3.02\,\text{eV}, \\
&\quad \lambda = \frac{1240}{3.02} \approx 411\,\text{nm} \quad (\text{violet, near UV}).
\end{aligned}
\]

\subsection*{4. Transitions to \texorpdfstring{$n_f=3$}{n=3} (Paschen Series)}
\[
\begin{aligned}
&\textbf{(a)}\; n_i = 4 \;\longrightarrow\; n_f = 3: \\
&\quad E_{4} = -0.85\,\text{eV}, \quad E_{3} = -1.51\,\text{eV}, \\
&\quad E_{\gamma} = E_{4} - E_{3} = (-0.85) - (-1.51) = 0.66\,\text{eV}, \\
&\quad \lambda = \frac{1240}{0.66} \approx 1880\,\text{nm} \quad (\text{IR region}). \\[6pt]
&\textbf{(b)}\; n_i = 5 \;\longrightarrow\; n_f = 3: \\
&\quad E_{5} = -0.54\,\text{eV}, \quad E_{3} = -1.51\,\text{eV}, \\
&\quad E_{\gamma} = E_{5} - E_{3} = (-0.54) - (-1.51) = 0.97\,\text{eV}, \\
&\quad \lambda = \frac{1240}{0.97} \approx 1280\,\text{nm} \quad (\text{IR region}). \\[6pt]
&\textbf{(c)}\; n_i = 6 \;\longrightarrow\; n_f = 3: \\
&\quad E_{6} = -0.38\,\text{eV}, \quad E_{3} = -1.51\,\text{eV}, \\
&\quad E_{\gamma} = E_{6} - E_{3} = (-0.38) - (-1.51) = 1.13\,\text{eV}, \\
&\quad \lambda = \frac{1240}{1.13} \approx 1097\,\text{nm} \quad (\text{IR region}).
\end{aligned}
\]

\subsection*{5. Summary Table}

\noindent
\textbf{(a) To \(\,n_f = 1\) (Lyman Series, UV):}

\begin{tabular}{|c|c|c|c|c|}
\hline
Transition & $E_{n_i}$ (eV) & $E_{n_f}$ (eV) & $E_\gamma$ (eV) & $\lambda$ (nm) \\ \hline
$n_i=2 \to n_f=1$ & -3.40 & -13.6 & 10.2 & 122 (UV) \\ \hline
$n_i=3 \to n_f=1$ & -1.51 & -13.6 & 12.09 & 102 (UV) \\ \hline
\end{tabular}

\vspace{0.5cm}

\noindent
\textbf{(b) To \(\,n_f = 2\) (Balmer Series, visible or near UV):}

\begin{tabular}{|c|c|c|c|c|}
\hline
Transition & $E_{n_i}$ (eV) & $E_{n_f}$ (eV) & $E_\gamma$ (eV) & $\lambda$ (nm) \\ \hline
$n_i=3 \to n_f=2$ & -1.51 & -3.40 & 1.89 & 656 (red) \\ \hline
$n_i=4 \to n_f=2$ & -0.85 & -3.40 & 2.55 & 486 (blue-green) \\ \hline
$n_i=5 \to n_f=2$ & -0.54 & -3.40 & 2.86 & 434 (violet) \\ \hline
$n_i=6 \to n_f=2$ & -0.38 & -3.40 & 3.02 & 411 (violet) \\ \hline
\end{tabular}

\vspace{0.5cm}

\noindent
\textbf{(c) To \(\,n_f = 3\) (Paschen Series, IR):}

\begin{tabular}{|c|c|c|c|c|}
\hline
Transition & $E_{n_i}$ (eV) & $E_{n_f}$ (eV) & $E_\gamma$ (eV) & $\lambda$ (nm) \\ \hline
$n_i=4 \to n_f=3$ & -0.85 & -1.51 & 0.66 & 1880 (IR) \\ \hline
$n_i=5 \to n_f=3$ & -0.54 & -1.51 & 0.97 & 1280 (IR) \\ \hline
$n_i=6 \to n_f=3$ & -0.38 & -1.51 & 1.13 & 1097 (IR) \\ \hline
\end{tabular}


\section*{Transition from \(n_i=4\) to \(n_f=1\) (Hydrogen Atom)}

\subsection*{1. Energy Levels}
Recall that the energy of the hydrogen atom at principal quantum number \(n\) is given by
\[
E_n = -\frac{13.6\,\text{eV}}{n^2}.
\]
Hence,
\[
E_4 = -\frac{13.6\,\text{eV}}{4^2} = -\frac{13.6\,\text{eV}}{16} = -0.85\,\text{eV},
\quad
E_1 = -\frac{13.6\,\text{eV}}{1^2} = -13.6\,\text{eV}.
\]

\subsection*{2. Photon Energy for the Transition}
A photon emitted by the transition \(n_i = 4 \to n_f = 1\) has energy
\[
E_{\gamma} = E_{4} - E_{1}
= \bigl(-0.85\,\text{eV}\bigr) - \bigl(-13.6\,\text{eV}\bigr)
= 13.6\,\text{eV} - 0.85\,\text{eV}
= 12.75\,\text{eV}.
\]

\subsection*{3. Wavelength of the Emitted Photon}
Using the conversion 
\[
\lambda \,[\text{nm}] = \frac{1240\,\text{eV}\cdot\text{nm}}{E_{\gamma}\,[\text{eV}]},
\]
we find
\[
\lambda 
= \frac{1240}{12.75}
\approx 97.3\,\text{nm}.
\]
This wavelength is in the \textbf{vacuum ultraviolet (UV)} region.

\subsection*{4. Summary}
\[
\boxed{
\text{For } n_i=4 \to n_f=1:\;
E_{\gamma} \approx 12.75\,\text{eV},\;
\lambda \approx 97.3\,\text{nm}\; (\text{vacuum UV}).
}
\]

\section*{Calculations for a Diffraction Grating with 7500 lines/cm}

\subsection*{1. Slit Spacing}
\begin{itemize}
    \item The grating has 7500 lines per centimeter (\(\text{lines/cm}\)).
    \item To find the slit spacing \(d\), take the reciprocal of the line density 
    and convert centimeters to meters:
    \[
      d 
      = \frac{1}{7500\,\text{lines/cm}} 
      = \frac{1}{7500}\,\text{cm} 
      = \frac{1}{7500} \times 10^{-2}\,\text{m} 
      \;=\; 1.33 \times 10^{-6}\,\text{m}.
    \]
\end{itemize}

\subsection*{2. First-Order Maximum Angle}
\begin{itemize}
    \item The wavelength of the yellow light is \(\lambda = 589.3\,\text{nm} = 589.3 \times 10^{-9}\,\text{m}.\)
    \item For first-order diffraction (\(m=1\)) at normal incidence, the grating equation is:
    \[
      d \,\sin \theta = m \,\lambda 
      \quad\Longrightarrow\quad
      \sin \theta = \frac{\lambda}{d}.
    \]
    \item Substitute the values:
    \[
      \sin \theta 
      = \frac{589.3 \times 10^{-9}\,\text{m}}{1.33 \times 10^{-6}\,\text{m}}
      \;\approx\; 0.442.
    \]
    \item Solve for \(\theta\):
    \[
      \theta = \sin^{-1}(0.442) 
      \;\approx\; 26.2^\circ.
    \]
\end{itemize}

\subsection*{Final Answers}
\[
\boxed{
d \;\approx\; 1.33\times10^{-6}\,\text{m}, 
\quad
\theta_1 \;\approx\; 26.2^\circ.
}
\]

\section*{Predicting First-Order Angles for the Four Balmer Lines}

\subsection*{Given Data}
\begin{itemize}
  \item Hydrogen Balmer-series lines (transitions to $n_f = 2$):
    \[
      \begin{aligned}
      n_i = 3 \;\to\; 2 &: \quad \lambda \approx 656\,\text{nm},\\
      n_i = 4 \;\to\; 2 &: \quad \lambda \approx 486\,\text{nm},\\
      n_i = 5 \;\to\; 2 &: \quad \lambda \approx 434\,\text{nm},\\
      n_i = 6 \;\to\; 2 &: \quad \lambda \approx 410\,\text{nm}.
      \end{aligned}
    \]
  \item A diffraction grating with spacing $d$ (from previous calculation). 
    For example, if the grating is $7500\,\text{lines/cm}$, then
    \[
      d \;=\; \frac{1}{7500\,\text{lines/cm}}
             \;=\; 1.33 \times 10^{-6}\,\text{m}.
    \]
  \item First-order diffraction condition:
    \[
      d \,\sin\theta = \lambda
      \quad\Longrightarrow\quad
      \sin\theta = \frac{\lambda}{d}.
    \]
\end{itemize}

\subsection*{Calculations for Each Line (Example with $d = 1.33\times10^{-6}\,\text{m}$)}
\begin{enumerate}
  \item \textbf{$n_i=3 \to n_f=2$:} 
    \[
      \lambda = 656\,\text{nm} = 6.56\times10^{-7}\,\text{m}.
    \]
    \[
      \sin\theta = \frac{6.56\times10^{-7}\,\text{m}}{1.33\times10^{-6}\,\text{m}}
      \approx 0.49,
      \quad
      \theta \approx \sin^{-1}(0.49) \approx 29.5^\circ.
    \]

  \item \textbf{$n_i=4 \to n_f=2$:} 
    \[
      \lambda = 486\,\text{nm} = 4.86\times10^{-7}\,\text{m}.
    \]
    \[
      \sin\theta = \frac{4.86\times10^{-7}\,\text{m}}{1.33\times10^{-6}\,\text{m}}
      \approx 0.37,
      \quad
      \theta \approx 21.4^\circ.
    \]

  \item \textbf{$n_i=5 \to n_f=2$:} 
    \[
      \lambda = 434\,\text{nm} = 4.34\times10^{-7}\,\text{m}.
    \]
    \[
      \sin\theta = \frac{4.34\times10^{-7}\,\text{m}}{1.33\times10^{-6}\,\text{m}}
      \approx 0.33,
      \quad
      \theta \approx 19.0^\circ.
    \]

  \item \textbf{$n_i=6 \to n_f=2$:} 
    \[
      \lambda = 410\,\text{nm} = 4.10\times10^{-7}\,\text{m}.
    \]
    \[
      \sin\theta = \frac{4.10\times10^{-7}\,\text{m}}{1.33\times10^{-6}\,\text{m}}
      \approx 0.31,
      \quad
      \theta \approx 17.8^\circ.
    \]
\end{enumerate}

\subsection*{Summary Table (Example)}
\[
\begin{array}{c|c|c}
\textbf{Transition} & \lambda \,(\text{nm}) & \theta_1 \,(\text{degrees}) \\
\hline
3 \to 2 & 656 & 29.5 \\
4 \to 2 & 486 & 21.4 \\
5 \to 2 & 434 & 19.0 \\
6 \to 2 & 410 & 17.8 \\
\end{array}
\]

\noindent
\textit{Note: Your actual angles will depend on the specific grating spacing }$d$\textit{ 
from your lab's first calculation. Simply use }$\theta = \sin^{-1}\!\bigl(\lambda/d\bigr)$\textit{ 
for each wavelength.}
\end{document}
