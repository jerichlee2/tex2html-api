\documentclass[12pt]{article}

% Packages
\usepackage[margin=1in]{geometry}
\usepackage{amsmath,amssymb,amsthm}
\usepackage{enumitem}
\usepackage{hyperref}
\usepackage{xcolor}
\usepackage{import}
\usepackage{xifthen}
\usepackage{pdfpages}
\usepackage{transparent}
\usepackage{listings}


\lstset{
    breaklines=true,         % Enable line wrapping
    breakatwhitespace=false, % Wrap lines even if there's no whitespace
    basicstyle=\ttfamily,    % Use monospaced font
    frame=single,            % Add a frame around the code
    columns=fullflexible,    % Better handling of variable-width fonts
}

\newcommand{\incfig}[1]{%
    \def\svgwidth{\columnwidth}
    \import{./Figures/}{#1.pdf_tex}
}
\theoremstyle{definition} % This style uses normal (non-italicized) text
\newtheorem{solution}{Solution}
\newtheorem{proposition}{Proposition}
\newtheorem{problem}{Problem}
\newtheorem{lemma}{Lemma}
\newtheorem{theorem}{Theorem}
\newtheorem{remark}{Remark}
\newtheorem{note}{Note}
\newtheorem{definition}{Definition}
\newtheorem{example}{Example}
\theoremstyle{plain} % Restore the default style for other theorem environments
%

% Theorem-like environments
% Title information
\title{}
\author{Jerich Lee}
\date{\today}

\begin{document}

\maketitle
A one-dimensional harmonic oscillator potential contains an electron. The ground state energy is measured to be $10$ meV. 
\noindent
\begin{enumerate}
    \item What wavelength electromagnetic radiation will promote the electron from the ground state $(n=0)$ to the 6th excited ground state $(n=6)$? 
    \noindent
    \begin{enumerate}
        \item 124000 nm
        \item 10300 nm
        \item 124 nm
        \item 9.54 nm
        \item 9540 nm
    \end{enumerate}
    \item A muon is an elementary particle that is essentially a heavy electron. It has the same charge as the electron, but its mass is $m_{muon}=207 m_e = 1.88e-28= 105.66 \frac{MeV}{c^{2}}$. What is the energy of a muon held in the same potential energy function oas the electron in the previous problem (i.e., with the same effective spring constant), but in the $n=0$ state?
    \noindent
    \begin{enumerate}
        \item 1030 meV
        \item .348 meV
        \item 2070 meV
        \item 0.695 meV
        \item 144 meV
    \end{enumerate}
\end{enumerate}

\noindent \textbf{Problem Statement:}\\
A one-dimensional harmonic oscillator potential contains an electron. 
The ground-state energy is measured to be $E_{0} = 10\text{ meV}$. 

\noindent\textbf{Part (1).} 
Find the wavelength of the electromagnetic radiation that promotes the electron from the ground state ($n=0$) to the $n=6$ state.

\[
\underline{\text{Step 1: Relation between ground-state energy and } \omega.}
\]
For a quantum harmonic oscillator, the energy levels are given by:
\[
E_{n} \;=\; \hbar \omega \left(n + \tfrac12 \right).
\]
Since the ground state corresponds to $n=0$, its energy is
\[
E_{0} \;=\; \tfrac12 \,\hbar \omega.
\]
We know $E_{0} = 10\text{ meV}$, hence
\[
\tfrac12 \,\hbar \omega \;=\; 10\text{ meV} 
\quad\Longrightarrow\quad
\hbar \omega \;=\; 20\text{ meV}.
\]

\[
\underline{\text{Step 2: Energy of the }n=6\text{ state and the energy difference.}}
\]
The $n$th level has energy $E_{n} = \hbar \omega \left(n + \tfrac12\right)$.  
So for $n=6$,
\[
E_6 \;=\; \hbar \omega \left(6 + \tfrac12\right)
        \;=\; \hbar \omega \,(6.5).
\]
However, the transition from $n=0$ to $n=6$ depends on 
\[
\Delta E \;=\; E_6 - E_0 
           \;=\; \hbar\omega \,(6.5 - 0.5) 
           \;=\; 6\,\hbar \omega.
\]
Since $\hbar \omega = 20\text{ meV}$, we get
\[
\Delta E \;=\; 6 \times 20\text{ meV} 
            \;=\; 120\text{ meV}.
\]

\[
\underline{\text{Step 3: Convert } \Delta E \text{ to wavelength.}}
\]
We use the relation $E = \frac{hc}{\lambda}$.  In convenient units, 
\[
hc \;\approx\; 1240 \text{ eV}\cdot \text{nm}.
\]
Now, $120\text{ meV} = 0.120\text{ eV}$. Thus,
\[
\lambda 
   \;=\; \frac{hc}{\Delta E} 
   \;=\; \frac{1240\text{ eV}\cdot\text{nm}}{0.120\text{ eV}} 
   \;=\; \frac{1240}{0.120}\,\text{nm}
   \;\approx\; 10333\,\text{nm}.
\]
From the options given, the nearest is $\boxed{10300\text{ nm}.}$

\vspace{1em}
\noindent\textbf{Part (2).} 
A muon is essentially a heavy electron, having the same charge but mass $m_{muon} = 207\, m_e$. 
What is the energy of the muon in the same potential, in the $n=0$ state?

\[
\underline{\text{Step 4: Relation of }\omega\text{ to mass.}}
\]
The angular frequency of a 1D harmonic oscillator is given by 
\[
\omega = \sqrt{\frac{k}{m}},
\]
where $k$ is the spring constant.  Since the potential is unchanged, $k$ stays the same.

\[
\underline{\text{Step 5: Muon frequency.}}
\]
Replacing $m$ by $m_{muon} = 207\,m_e$, we get
\[
\omega_{muon} 
   \;=\; \sqrt{\frac{k}{m_{muon}}} 
   \;=\; \sqrt{\frac{k}{207\,m_e}}
   \;=\; \frac{1}{\sqrt{207}} \,\sqrt{\frac{k}{m_e}}
   \;=\; \frac{\omega_e}{\sqrt{207}},
\]
where $\omega_e$ is the electron's oscillator frequency.

\[
\underline{\text{Step 6: Muon ground-state energy.}}
\]
The ground-state energy for any harmonic oscillator is
\[
E_0 = \tfrac12 \,\hbar \,\omega.
\]
Hence, for the muon,
\[
E_{0}^{(muon)} 
   \;=\; \tfrac12 \,\hbar \,\omega_{muon} 
   \;=\; \tfrac12 \,\hbar \,\Bigl(\frac{\omega_e}{\sqrt{207}}\Bigr) 
   \;=\; \frac{1}{\sqrt{207}} \,\bigl(\tfrac12 \hbar \omega_e \bigr).
\]
We already know $\tfrac12 \hbar \omega_e = 10\text{ meV}$. Therefore,
\[
E_{0}^{(muon)} 
   \;=\; \frac{10\text{ meV}}{\sqrt{207}}.
\]
Numerically, $\sqrt{207} \approx 14.387$, so
\[
E_{0}^{(muon)} 
   \;\approx\; \frac{10\text{ meV}}{14.387}
   \;\approx\; 0.695\text{ meV}.
\]
From the multiple-choice list, the correct value is $\boxed{0.695\text{ meV}.}$

\vspace{1em}
\noindent \textbf{Final Answers:}\\
(1) $10300\text{ nm}$\\
(2) $0.695\text{ meV}$


\noindent
\textbf{Stern-Gerlach Experiment: Probability Calculation} 

\bigskip
\noindent
\textbf{Setup:} 
A silver atom with \emph{random} spin passes through:
\begin{enumerate}
    \item a $+y$ filter (i.e.\ spin up along the $y$-axis), 
    \item followed by a $-x$ filter (spin down along the $x$-axis).
\end{enumerate}
We want the probability that the atom passes \emph{both} filters.

\bigskip
\noindent
\textbf{Step 1: Passing the $+y$ filter.}\\
Since the initial spin state is completely random, the probability of measuring spin-up along $y$ (i.e.\ projecting onto $\lvert +y \rangle$) is
\[
P(\lvert +y\rangle) \;=\; \tfrac{1}{2}.
\]
After this measurement, the spin state \emph{collapses} to $\lvert +y\rangle$.

\bigskip
\noindent
\textbf{Step 2: Probability of then being spin-down along $x$.}\\
Next, the atom is sent through an $x$-axis Stern-Gerlach filter, keeping only the $\lvert -x\rangle$ output. 
We need
\[
P(\lvert -x\rangle \,\big\vert\, \lvert +y\rangle) 
\;=\; \bigl|\langle -x \mid +y\rangle \bigr|^2.
\]
A standard spin-$\tfrac12$ calculation shows 
\[
\lvert +y \rangle \;=\; 
    \frac{1}{\sqrt{2}}
    \bigl( \lvert +x\rangle + i\lvert -x\rangle \bigr),
\]
so 
\[
\langle -x \mid +y \rangle 
   = \frac{i}{\sqrt{2}} 
   \quad\Longrightarrow\quad
\bigl|\langle -x \mid +y \rangle \bigr|^2 
   = \tfrac{1}{2}.
\]
Hence, if the spin is already in $\lvert +y\rangle,$ the probability to measure it as $\lvert -x\rangle$ is $\frac12.$

\bigskip
\noindent
\textbf{Step 3: Combine the two stages.}\\
Since each of those probabilities is $\tfrac12,$ the overall probability of passing \emph{both} filters is
\[
P(\lvert +y\rangle)\;\times\;P(\lvert -x\rangle \mid \lvert +y\rangle)
\;=\;\tfrac12 \;\times\;\tfrac12 
\;=\;\boxed{\tfrac14}.
\]

\bigskip
\noindent
\textbf{Conclusion:} The probability that a random spin state passes through a $+y$ filter \emph{and} subsequently through a $-x$ filter is $\frac14.$

\noindent
\textbf{Computing } $\langle -x \mid +y \rangle$:

\bigskip
\noindent
\textbf{Step 1. Express the states in the $\lvert +z\rangle,\lvert -z\rangle$ basis.}\\
One common choice of phase conventions is:
\[
\lvert +y\rangle 
  \;=\; \frac{1}{\sqrt{2}} 
  \begin{pmatrix}
  1 \\[5pt]
  i
  \end{pmatrix},
\qquad
\lvert -x\rangle 
  \;=\; \frac{1}{\sqrt{2}}
  \begin{pmatrix}
  1 \\[5pt]
  -1
  \end{pmatrix}.
\]
Here, we use 
$\lvert +z\rangle = \begin{pmatrix} 1 \\ 0 \end{pmatrix}$ and 
$\lvert -z\rangle = \begin{pmatrix} 0 \\ 1 \end{pmatrix}$ 
as our standard basis.

\bigskip
\noindent
\textbf{Step 2. Take the inner product.}\\
The bra $\langle -x\vert$ is the conjugate transpose of $\lvert -x\rangle$, i.e.
\[
\langle -x\vert 
  \;=\; \frac{1}{\sqrt{2}} 
  \bigl(1,\,-1\bigr).
\]
Hence,
\[
\langle -x \mid +y \rangle
  \;=\; 
  \biggl(\frac{1}{\sqrt{2}} \bigl(1,\,-1\bigr)\biggr)
  \biggl(\frac{1}{\sqrt{2}}\begin{pmatrix}1\\[4pt] i\end{pmatrix}\biggr)
  \;=\;
  \frac{1}{2} \bigl(1 - i\bigr).
\]
Often, you will see this written (up to an overall phase) as 
\[
\frac{i}{\sqrt{2}},
\]
since $1 - i = \sqrt{2}\, e^{-\,i\pi/4}$.

\bigskip
\noindent
\textbf{Step 3. Check the magnitude.}\\
Either way,
\[
\bigl|\langle -x \mid +y \rangle \bigr|^2
  \;=\; \frac12,
\]
showing that the probability of measuring $\lvert -x\rangle$ given the spin is in $\lvert +y\rangle$ is~$1/2$.

\end{document}
