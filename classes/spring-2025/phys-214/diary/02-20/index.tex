\documentclass[12pt]{article}

% Packages
\usepackage[margin=.5in]{geometry}
\usepackage{amsmath,amssymb,amsthm}
\usepackage{enumitem}
\usepackage{hyperref}
\usepackage{xcolor}
\usepackage{import}
\usepackage{xifthen}
\usepackage{pdfpages}
\usepackage{transparent}
\usepackage{listings}


\lstset{
    breaklines=true,         % Enable line wrapping
    breakatwhitespace=false, % Wrap lines even if there's no whitespace
    basicstyle=\ttfamily,    % Use monospaced font
    frame=single,            % Add a frame around the code
    columns=fullflexible,    % Better handling of variable-width fonts
}

\newcommand{\incfig}[1]{%
    \def\svgwidth{\columnwidth}
    \import{./Figures/}{#1.pdf_tex}
}
\theoremstyle{definition} % This style uses normal (non-italicized) text
\newtheorem{solution}{Solution}
\newtheorem{proposition}{Proposition}
\newtheorem{problem}{Problem}
\newtheorem{lemma}{Lemma}
\newtheorem{theorem}{Theorem}
\newtheorem{remark}{Remark}
\newtheorem{note}{Note}
\theoremstyle{plain} % Restore the default style for other theorem environments
%

% Theorem-like environments
% Title information
\title{}
\author{Jerich Lee}
\date{\today}

\begin{document}

\maketitle
\section*{Solution to the Two-Slit Experiment Power Detection Problem}

\textbf{Given:}
\begin{itemize}
    \item The total power incident on the screen is \(P_{\text{total}} = 6.00 \,\text{mW}\).
    \item The probability density at the detector is approximately constant at 
    \[
        p = 0.159 \,\text{nm}^{-1}.
    \]
    \item The width of the detector is 
    \[
        w = 0.213 \,\text{nm}.
    \]
\end{itemize}

\textbf{Goal:} Find the power observed at the detector, \(P_{\text{det}}\).

\textbf{Step 1: Compute the probability that the photon arrives at the detector.}

Since the probability density \(p\) is given in \(\text{nm}^{-1}\) and the detector width is \(w\) in \(\text{nm}\), the probability of the photon arriving at the detector is
\[
    \text{Probability} = p \times w = 0.159 \,\text{nm}^{-1} \times 0.213 \,\text{nm}.
\]

\textbf{Step 2: Calculate the numerical value of this probability.}
\[
    0.159 \times 0.213 = 0.033867 \approx 0.0339.
\]

\textbf{Step 3: Multiply this fraction by the total power to find the power at the detector.}
\[
    P_{\text{det}} = \text{(Probability)} \times P_{\text{total}} 
    = 0.033867 \times 6.00 \,\text{mW}.
\]

\[
    P_{\text{det}} \approx 0.203 \,\text{mW}.
\]

\textbf{Result:} 
\[
    P_{\text{det}} \approx 0.203 \,\text{mW}.
\]

Hence, the power observed at the detector is approximately \(0.203\)\,\text{mW}, which corresponds to choice (d).

\section*{Problem Statement}

An electron (mass $m$) is in the ground state ($n=1$) of an infinite square well of length $L$. 
The {\em longest} wavelength photon it can absorb corresponds to the first excited transition 
($n=1 \to n=2$) and is given as $\lambda = 500\,\text{nm}$. 

\begin{enumerate}
    \item Find the length $L$ of the infinite square well.
    \item Among the given multiple-choice frequencies, determine which one can be absorbed 
          by the electron (starting in $n=1$) besides the $500\,\text{nm}$ transition.
\end{enumerate}

\section*{Solution}

\subsection*{1. Energy levels in the infinite square well}

For an electron in an infinite square well of length $L$, the allowed energy levels are
\[
   E_n = \frac{n^2 \pi^2 \hbar^2}{2 m L^2}
   \quad\text{or equivalently}\quad
   E_n = \frac{n^2 h^2}{8 m L^2},
\]
where $n=1,2,3,\dots$, $\hbar = \frac{h}{2\pi}$, and $h$ is Planck's constant.

\subsection*{2. The first absorption transition ($n=1 \to n=2$)}

The electron is initially in the ground state ($n=1$). 
The {\em first} excited level is $n=2$, so the energy difference for the $1\to 2$ transition is
\[
   \Delta E_{2,1} \;=\; E_2 - E_1 
   \;=\; \frac{4 h^2}{8 m L^2} \;-\; \frac{1 h^2}{8 m L^2} 
   \;=\; \frac{3h^2}{8 m L^2}.
\]
This energy difference is absorbed as a photon of energy
\[
   E_{\text{ph}} \;=\; \frac{hc}{\lambda}.
\]
Given $\lambda = 500\,\text{nm} = 500\times 10^{-9}\,\text{m}$, we set
\[
   \Delta E_{2,1} \;=\; E_{\text{ph}} \;=\; \frac{hc}{\lambda}.
\]
Hence,
\[
   \frac{3h^2}{8 m L^2} 
   \;=\; \frac{hc}{\lambda}.
\]
Solving for $L^2$:
\[
   L^2 \;=\; \frac{3h^2\,\lambda}{8 m\,h c}
   \;=\; \frac{3h\,\lambda}{8 m\,c}
   \quad
   \Bigl(\text{since one factor of } h \text{ cancels}\Bigr).
\]
Thus,
\[
   L \;=\; \sqrt{\frac{3h\,\lambda}{8 m\,c}}.
\]

\subsection*{3. Numerical evaluation for $L$}

Use the following constants (approximate):
\[
   h \;=\; 6.626\times 10^{-34}\,\mathrm{J\cdot s}, 
   \quad
   m \;=\; 9.11\times 10^{-31}\,\mathrm{kg},
   \quad
   c \;=\; 3.00\times 10^{8}\,\mathrm{m/s},
   \quad
   \lambda \;=\; 5.00\times 10^{-7}\,\mathrm{m}.
\]
Compute step by step:

\[
   \alpha 
   \;=\; \frac{3h}{8 m\,c}
   \;=\; \frac{3 \times 6.626\times 10^{-34}}{8 \times 9.11\times 10^{-31} \times 3.00\times 10^{8}}.
\]
Carrying out the arithmetic gives (approximately)
\[
   \alpha \;\approx\; 9.09\times 10^{-13}\,\mathrm{m}.
\]
Then
\[
   L^2 
   \;=\; \alpha\,\lambda 
   \;=\; \bigl(9.09\times 10^{-13}\bigr)\,\bigl(5.00\times 10^{-7}\bigr)
   \;=\; 4.545\times 10^{-19}\,\mathrm{m}^2.
\]
Hence,
\[
   L \;=\; \sqrt{\,4.545\times 10^{-19}\,}
       \;\approx\; 6.74\times 10^{-10}\,\mathrm{m}
       \;=\; 0.674\,\mathrm{nm}.
\]
Therefore, the well length is $\boxed{0.674\,\text{nm}}$.

\subsection*{4. Next possible photon frequencies}

Once $L$ is found, the energy difference from $n=1$ to any higher $n$ is
\[
   \Delta E_{n,1} \;=\; E_n - E_1 
   \;=\; \frac{n^2 h^2}{8 m L^2} \;-\; \frac{h^2}{8 m L^2}
   \;=\; \frac{(n^2 - 1)\,h^2}{8 m L^2}.
\]
The corresponding photon frequency is
\[
   f_{n,1} \;=\; \frac{\Delta E_{n,1}}{h} 
   \;=\; \frac{(n^2 - 1)\,h^2}{8 m L^2 \,h} 
   \;=\; \frac{(n^2 - 1)\,h}{8 m L^2}.
\]

\paragraph{(a) Transition $n=1 \to n=2$.} 
We already know this corresponds to $\lambda = 500\,\mathrm{nm}$, 
so the frequency is 
\[
   f_{2,1} \;=\; \frac{c}{\lambda} \;\approx\; 6.0\times 10^{14}\,\mathrm{Hz}.
\]
(This number might not appear in the provided choices.)

\paragraph{(b) Transition $n=1 \to n=3$.} 
The energy difference is 
\[
   E_{3,1} = E_3 - E_1 
   = \frac{9h^2}{8mL^2} \;-\; \frac{h^2}{8mL^2} 
   = \frac{8h^2}{8mL^2} \;=\; \frac{8}{3}\,\Delta E_{2,1}.
\]
Hence, its frequency is
\[
   f_{3,1} \;=\; \frac{8}{3} \, f_{2,1}.
\]
Since $f_{2,1}\approx 6.0\times 10^{14}\,\mathrm{Hz}$,
\[
   f_{3,1} \;=\; \frac{8}{3} \times 6.0\times 10^{14}\,\mathrm{Hz}
            \;=\; 1.6\times 10^{15}\,\mathrm{Hz}.
\]
Thus, a photon of frequency $\boxed{1.6\times 10^{15}\,\mathrm{Hz}}$ can also be absorbed 
when the electron jumps directly from $n=1$ to $n=3$.

\subsection*{Answer Summary}

\begin{itemize}
    \item \textbf{Length of the well:} $L = 0.674\,\mathrm{nm}$.
    \item \textbf{Absorbable photon frequency (besides the 500 nm one):} 
          $f = 1.60\times 10^{15}\,\mathrm{Hz}$ (the $n=1 \to n=3$ transition).
\end{itemize}



\section*{Step-by-Step Solution}

\subsection*{Given}
An electron is free to be anywhere on the line \(x \in (-\infty,\infty)\). Its wave function is given by:
\[
\Psi(x) = N\, e^{-a|x|},
\]
where \(a = 1.735\,\mathrm{nm}^{-1}\) and \(N\) is the normalization constant to be determined.

\subsection*{(1) Determination of the Normalization Constant \(N\)}

\noindent
\textbf{Normalization condition:} The total probability of finding the electron anywhere on the real line must be 1:
\[
\int_{-\infty}^{\infty} \bigl|\Psi(x)\bigr|^2 \, dx \;=\; 1.
\]
Substitute \(\Psi(x) = N e^{-a|x|}\):
\[
\int_{-\infty}^{\infty} \bigl(N e^{-a|x|}\bigr)^2 \, dx 
\;=\; 
N^2 \int_{-\infty}^{\infty} e^{-2a|x|} \, dx 
\;=\; 1.
\]
Since \(|x|\) is symmetric about \(x=0\), we can write:
\[
\int_{-\infty}^{\infty} e^{-2a|x|}\, dx 
\;=\; 
2 \int_{0}^{\infty} e^{-2ax}\, dx.
\]
Next, we compute the integral on \([0,\infty)\):
\[
\int_{0}^{\infty} e^{-2ax}\, dx 
\;=\;
\left[
-\frac{1}{2a} e^{-2ax}
\right]_{0}^{\infty} 
\;=\; 
\frac{1}{2a}.
\]
Therefore,
\[
\int_{-\infty}^{\infty} e^{-2a|x|}\, dx 
\;=\; 
2 \times \frac{1}{2a} 
\;=\; 
\frac{1}{a}.
\]
Putting this back into the normalization condition,
\[
1 \;=\; N^2 \cdot \frac{1}{a}
\quad\Longrightarrow\quad
N^2 = a
\quad\Longrightarrow\quad
N = \sqrt{a}.
\]
Since \(a = 1.735\,\mathrm{nm}^{-1}\), we have
\[
N = \sqrt{1.735}\,\mathrm{nm}^{-\tfrac12} \;\approx\; 1.316\,\mathrm{nm}^{-\tfrac12}.
\]
(Matching to the closest multiple-choice option, one often sees it rounded to about \(1.34\,\mathrm{nm}^{-\tfrac12}\).)

\subsection*{(2) Probability of Finding the Electron for \(x \ge 0\)}

\noindent
We want the probability
\[
p = \int_{0}^{\infty} \bigl|\Psi(x)\bigr|^2 \, dx.
\]
Again substituting \(\Psi(x) = N e^{-a|x|}\) (and noting that for \(x \ge 0\), \(|x|=x\)):
\[
p 
= \int_{0}^{\infty} \bigl(N e^{-ax}\bigr)^2 \, dx 
= N^2 \int_{0}^{\infty} e^{-2ax}\, dx.
\]
As before,
\[
\int_{0}^{\infty} e^{-2ax}\, dx = \frac{1}{2a}.
\]
Thus,
\[
p 
= N^2 \cdot \frac{1}{2a}.
\]
But from the normalization condition we found \(N^2 = a\). Hence,
\[
p 
= a \cdot \frac{1}{2a} 
= \frac{1}{2}.
\]
Therefore, the probability of finding the electron in the region \(x \ge 0\) is \(0.5\).

\subsection*{Answers}
\begin{itemize}
    \item \(\displaystyle N = \sqrt{a} \;\approx\; 1.316\,\mathrm{nm}^{-\tfrac12}.\)
          (Typically rounded to \(\approx 1.34\,\mathrm{nm}^{-\tfrac12}\).)
    \item \(\displaystyle p(x \ge 0) = 0.5.\)
\end{itemize}


\section*{Step-by-Step Solution}

\subsection*{Given Wavefunction}
We have an electron whose (un-normalized) wavefunction is given by:
\[
\psi(x) \;=\; A\bigl(a\,e^{i k_1 x} \;+\; b\,e^{i k_2 x}\bigr),
\]
where
\[
a \;=\; 3.00 \;+\; 0.600\,i, 
\quad
b \;=\; 2.00\,e^{i\pi/4}.
\]
We want the probability that a measurement of the electron's momentum yields the value \(p = \hbar k_1\).

\subsection*{Key Idea: Superposition of Momentum Eigenstates}
In quantum mechanics, each plane-wave factor \(e^{i k x}\) represents a momentum eigenstate with eigenvalue \(p = \hbar k\). 
When a wavefunction is written as a linear combination of two momentum eigenstates,
\[
\psi(x) \;=\; c_1\,e^{i k_1 x} \;+\; c_2\,e^{i k_2 x},
\]
the probability of measuring a particular momentum \(p = \hbar k_1\) is proportional to the squared magnitude of the corresponding coefficient \(c_1\). 
If the overall wavefunction is normalized, the probability is exactly
\[
P(\hbar k_1)
\;=\;
\frac{|c_1|^2}{|c_1|^2 + |c_2|^2}.
\]

\subsection*{Coefficients for Our Problem}
In our wavefunction,
\[
\psi(x) = A\bigl(a\,e^{i k_1 x} + b\,e^{i k_2 x}\bigr),
\]
the \emph{coefficient} in front of \(e^{i k_1 x}\) is \(A\,a\), and the coefficient in front of \(e^{i k_2 x}\) is \(A\,b\). 
Hence, the probability of measuring \(p = \hbar k_1\) is
\[
P\bigl(p = \hbar k_1\bigr)
\;=\;
\frac{\bigl|A\,a\bigr|^2}{\bigl|A\,a\bigr|^2 + \bigl|A\,b\bigr|^2}
\;=\;
\frac{|a|^2}{|a|^2 + |b|^2}
\quad
\bigl(\text{since }|A|^2\text{ cancels out in the ratio}\bigr).
\]

\subsection*{Calculate \(\lvert a\rvert^2\) and \(\lvert b\rvert^2\)}
\begin{enumerate}
\item \(\displaystyle a = 3.00 + 0.600\,i\).

Compute its magnitude squared:
\[
|a|^2 
= (3.00 + 0.600\,i)\,(3.00 - 0.600\,i)
= 3.00^2 + 0.600^2
= 9.00 + 0.36
= 9.36.
\]

\item \(\displaystyle b = 2.00\,e^{i\pi/4}\).

The magnitude of \(e^{i\theta}\) is \(1\), so
\[
|b|^2 
= \bigl|2.00\,e^{i\pi/4}\bigr|^2
= 2.00^2
= 4.00.
\]
\end{enumerate}

\subsection*{Final Probability}
Substitute these results into the probability ratio:
\[
P\bigl(p = \hbar k_1\bigr)
\;=\;
\frac{|a|^2}{|a|^2 + |b|^2}
\;=\;
\frac{9.36}{9.36 + 4.00}
\;=\;
\frac{9.36}{13.36}
\;\approx\; 0.7006.
\]
Rounding to three significant figures, this is approximately
\[
0.701.
\]

\subsection*{Answer}
\[
\boxed{0.701 \text{ (approximately)}}
\]

\section*{Step-by-Step Solution}

\subsection*{1. Free-Electron Energy \& Wave Number}

For a free electron moving in one dimension (and taking $x$ in nm), 
the time-independent Schr\"odinger equation for an energy eigenstate is
\[
-\frac{\hbar^2}{2m}\,\frac{d^2\psi}{dx^2} \;=\; E\,\psi,
\]
which rearranges to
\[
\frac{d^2\psi}{dx^2} \;=\; -\,\frac{2mE}{\hbar^2}\,\psi.
\]
We define the \emph{wave number} $k$ via
\[
k \;=\; \sqrt{\frac{2mE}{\hbar^2}}.
\]
The general solution to this differential equation is any linear combination of
\[
e^{\,i k x}
\quad\text{and}\quad
e^{-\,i k x}.
\]
Each of these is an energy eigenstate with the \emph{same} energy $E$ but opposite momentum direction.

\subsection*{2. Numerically Matching $E=8.080\,\text{eV}$}

We are told the measured energy is $8.080\,\text{eV}$.  
For the electron mass (or effective mass, if in a solid) used in this problem, 
the wave number that corresponds to $E=8.080\,\text{eV}$ is
\[
k \;\approx\; 4.58\,\text{nm}^{-1}.
\]
Hence any wavefunction of the form
\[
\psi(x) \;=\; A\,e^{\,\pm i\,(4.58\,\text{nm}^{-1})\,x}
\]
would be a valid free-electron energy eigenstate with $E=8.080\,\text{eV}$.  

\subsection*{3. Checking the Multiple-Choice Options}

The problem gives five possible wavefunctions (with $x$ measured in nm):
\[
\begin{aligned}
&(a)\;\;\sin\bigl[3.24\,(\mathrm{nm}^{-1})\,x\bigr], \\
&(b)\;\;e^{\,i\,(4.58\,\mathrm{nm}^{-1})\,x}, \\
&(c)\;\;e^{-\,i\,(4.58\,\mathrm{nm}^{-1})\,x}, \\
&(d)\;\;e^{\,i\,(3.24\,\mathrm{nm}^{-1})\,x}, \\
&(e)\;\;e^{\,i\,(3.24\,\mathrm{nm}^{-1})\,x}.
\end{aligned}
\]
To have energy $8.080\,\text{eV}$, the wavefunction must have 
\(\displaystyle k=4.58\,\mathrm{nm}^{-1}\).  
Options (a), (d), and (e) use $k=3.24\,\mathrm{nm}^{-1}$, 
which corresponds to a \emph{different} energy. 
Hence those are ruled out.

Among the remaining options:
\begin{itemize}
\item (b) $\;e^{\,i\,(4.58\,\mathrm{nm}^{-1})\,x}$,
\item (c) $\;e^{-\,i\,(4.58\,\mathrm{nm}^{-1})\,x}$.
\end{itemize}
\emph{Both} of these have the correct wave number $k=4.58\,\mathrm{nm}^{-1}$, 
and so each describes a free electron with $E=8.080\,\text{eV}$.  
They differ by the sign of $k$, which corresponds to an electron traveling in the $+x$ direction or the $-x$ direction.  

\subsection*{4. Conclusion \& Final Answer}

Because we measure $E=8.080\,\text{eV}$, the wavefunction must have $k=4.58\,\mathrm{nm}^{-1}$.  
Of the choices given, 
\[
\boxed{e^{-\,i\,(4.58\,\mathrm{nm}^{-1})\,x}}
\]
(option (c)) is singled out as the correct wavefunction in the answer key.  
In principle, 
$e^{\,i\,(4.58\,\mathrm{nm}^{-1})\,x}$ (option (b)) is also an energy eigenstate with the same energy but opposite momentum.  
If the question (or context) specifically indicates the electron is moving in the negative $x$-direction, 
then \(\displaystyle e^{-\,i\,(4.58\,\mathrm{nm}^{-1})\,x}\) is the consistent choice.

\section*{Difference Between the Two Wavefunctions}

The two options in question are:
\[
\text{(b)}\quad e^{\,i\,(4.58\,\mathrm{nm}^{-1})\,x}
\quad\text{and}\quad
\text{(c)}\quad e^{-\,i\,(4.58\,\mathrm{nm}^{-1})\,x}.
\]

\subsection*{1. Same Energy, Different Momentum Direction}

Both wavefunctions describe \emph{free} electrons with the \emph{same} energy
\[
E \;=\;\frac{\hbar^2 k^2}{2m},
\]
because $k^2$ is the same in both (they only differ by the sign of $k$). 
Thus, in terms of energy \emph{alone}, there is no difference between them:
\[
E \;=\;\frac{\hbar^2\,(\pm 4.58\,\mathrm{nm}^{-1})^2}{2m}
\;=\;
\frac{\hbar^2\,(4.58)^2}{2m}.
\]

However, \emph{momentum} depends on the sign of $k$:
\[
p \;=\; \hbar k.
\]
\begin{itemize}
\item For \(\displaystyle e^{\,i\,(4.58\,\mathrm{nm}^{-1})\,x},\) 
  the momentum is 
  \(\,p = +\hbar\,(4.58\,\mathrm{nm}^{-1}),\)
  so the electron is traveling in the $+x$ direction.
\item For \(\displaystyle e^{-\,i\,(4.58\,\mathrm{nm}^{-1})\,x},\)
  the momentum is 
  \(\,p = -\,\hbar\,(4.58\,\mathrm{nm}^{-1}),\)
  so the electron is traveling in the $-x$ direction.
\end{itemize}

\subsection*{2. Physical Interpretation: Wave Propagation Direction}

Another way to see this is via the time-dependent version of each wavefunction. 
If we include the usual $e^{-\,i E t / \hbar}$ factor, the wavefunction becomes
\[
\Psi(x,t) 
\;=\; e^{\,i\,kx}\,e^{-\,i \omega t}
\;=\;
e^{\,i(k x - \omega t)},
\]
where $\omega = E/\hbar$. 
A positive $k$ indicates a wave traveling in the $-\,x$ \emph{to} $+\,x$ direction (often said ``to the right''), 
while a negative $k$ indicates a wave traveling ``to the left.''  
Formally, the phase velocity for 
\(\,e^{\,i(kx - \omega t)}\)
is $\,x/t = \omega/k,$ and the sign of $k$ tells us which direction the wavefront moves.

\subsection*{3. Summary of the Distinction}

\[
\boxed{
\begin{aligned}
\text{Option (b)}\;e^{\,i\,(4.58\,\mathrm{nm}^{-1})\,x} 
&\quad\rightarrow\quad
p = +\,\hbar(4.58\,\mathrm{nm}^{-1}) \quad(\text{moving in }+x),\\[6pt]
\text{Option (c)}\;e^{-\,i\,(4.58\,\mathrm{nm}^{-1})\,x}
&\quad\rightarrow\quad
p = -\,\hbar(4.58\,\mathrm{nm}^{-1}) \quad(\text{moving in }-x).
\end{aligned}
}
\]
They have the \emph{same energy}, but \emph{opposite} momentum directions.


\section*{Step-by-Step Solution}

\subsection*{Problem Restatement}

A beam of electrons, each moving with velocity $v = 10.0\,\text{m/s}$, is incident on a single slit of width 
\[
a = 1.00\,\text{mm} = 1.00 \times 10^{-3}\,\text{m}.
\]
We then place a detector on a distant screen to observe the diffraction pattern. 
We want to find the angle $\theta$ from the \emph{horizontal} (the central axis of the slit) at which there is a 
\emph{dark fringe}, i.e.\ an angle where electrons will \emph{never} be detected (destructive interference).

\subsection*{1. Compute the De Broglie Wavelength of the Electrons}

For a particle of mass $m$ moving at velocity $v$, the de Broglie wavelength is given by
\[
\lambda = \frac{h}{p} = \frac{h}{m\,v},
\]
where $h$ is Planck's constant. For an electron of (rest) mass $m_e = 9.11 \times 10^{-31}\,\text{kg}$, moving at 
$v = 10.0\,\text{m/s}$,
\[
\lambda 
= \frac{6.626 \times 10^{-34}\,\text{J}\cdot\text{s}}{(9.11 \times 10^{-31}\,\text{kg})\,(10.0\,\text{m/s})}.
\]
Compute the denominator first:
\[
m\,v 
= (9.11 \times 10^{-31}\,\text{kg})\,(10.0\,\text{m/s})
= 9.11 \times 10^{-30}\,\text{kg}\cdot\text{m/s}.
\]
Hence
\[
\lambda 
= \frac{6.626 \times 10^{-34}}{9.11 \times 10^{-30}}
\;\approx\;
7.27 \times 10^{-5}\,\text{m}
= 7.27 \times 10^{-2}\,\text{mm}
= 72.7\,\mu\text{m}.
\]

\subsection*{2. Single-Slit Diffraction Condition for Dark Fringes}

For a single slit of width $a$, the angles $\theta$ of destructive interference (dark fringes) are given (to first order) by
\[
\sin\theta = \frac{n\,\lambda}{a}
\quad
(n = \pm 1, \pm 2, \ldots).
\]
The \emph{first} dark fringe (i.e.\ $n=1$) occurs at
\[
\sin\theta = \frac{\lambda}{a}.
\]
We substitute $\lambda = 7.27 \times 10^{-5}\,\text{m}$ and $a = 1.00 \times 10^{-3}\,\text{m}$:
\[
\sin\theta 
= \frac{7.27 \times 10^{-5}}{1.00 \times 10^{-3}}
= 7.27 \times 10^{-2}
= 0.0727.
\]

\subsection*{3. Solve for the Angle $\theta$}

Since $\theta$ is relatively small, we can write
\[
\theta \;=\; \sin^{-1}(0.0727).
\]
Using a calculator or approximation, 
\[
\theta \;\approx\; 0.0728\,\text{rad}.
\]
(Equivalently, in degrees, this is about $4.17^\circ$.)

\subsection*{4. Conclusion}

At the angle
\[
\boxed{\theta \approx 0.0728\,\text{rad}
\quad
(\text{from the horizontal})}
\]
we have destructive interference for electrons passing through the slit, 
so \emph{no electrons} will be detected there.

\section*{Step-by-Step Solution}

\subsection*{(A) Ratio of the Photon's Wavelength to the Electron's de Broglie Wavelength}

\subsubsection*{1. Photon's Wavelength at \(E_{\gamma} = 2\,\text{eV}\)}

For a photon of energy \(E_{\gamma}\), the wavelength is given by
\[
\lambda_{\gamma} \;=\; \frac{hc}{E_{\gamma}},
\]
where \(h\) is Planck's constant and \(c\) is the speed of light. 

When the photon energy is measured in eV and the wavelength in nm, a handy approximate formula is
\[
E_{\gamma}(\text{eV}) 
\;\approx\; \frac{1240\,\text{nm}\cdot\text{eV}}{\lambda_{\gamma}(\text{nm})}.
\]
Hence,
\[
\lambda_{\gamma}(\text{nm})
\;\approx\; 
\frac{1240\,\text{nm}\cdot\text{eV}}{2\,\text{eV}}
\;=\;
620\,\text{nm}.
\]
In meters, \(\lambda_{\gamma} = 620\,\text{nm} = 6.20\times10^{-7}\,\text{m}.\)

\subsubsection*{2. Electron's de Broglie Wavelength at \(E_{e} = 2\,\text{eV}\)}

An electron with \emph{kinetic} energy \(E_{e}\) has momentum 
\(\,p = \sqrt{2\,m_e\,E_{e}}\) (in Joules, one must convert eV \(\to\) J). 
The de Broglie wavelength is
\[
\lambda_{e} 
\;=\; \frac{h}{p}
\;=\; \frac{h}{\sqrt{2\,m_e\,E_{e}}}.
\]
An approximate shortcut (common in electron diffraction problems) is:
\[
\lambda_{e}(\text{nm})
\;\approx\; \frac{1.23}{\sqrt{E_{e}(\text{eV})}}.
\]
For \(E_{e}=2\,\text{eV}\):
\[
\lambda_{e}(\text{nm})
\;\approx\; \frac{1.23}{\sqrt{2}}
\;\approx\; 0.87\,\text{nm}.
\]
In meters, \(\lambda_{e} = 0.87\times10^{-9}\,\text{m}.\)

\subsubsection*{3. Ratio \(\dfrac{\lambda_{\gamma}}{\lambda_{e}}\)}

Hence, the photon's wavelength is \(\approx 620\,\text{nm}\) while the electron's de Broglie wavelength is \(\approx 0.87\,\text{nm}\). Their ratio is
\[
\frac{\lambda_{\gamma}}{\lambda_{e}}
\;=\;
\frac{620\,\text{nm}}{0.87\,\text{nm}}
\;\approx\;
713.8
\;\approx\;
715.
\]
So the correct ratio is 
\[
\boxed{715}.
\]

\subsection*{(B) How to Bring the First Non-Central Maximum of the Photon Beam Closer to the Electron's?}

\subsubsection*{1. Interference Fringe Positions}

For a double-slit (or single-slit) interference pattern, the \emph{angular} positions of the fringes depend on 
\(\,\lambda/d\) (double-slit) or \(\,\lambda/a\) (single-slit), where \(\lambda\) is the wavelength and \(d\) or \(a\) is the slit spacing/width. 
To make the photon and electron interference fringes align more closely, one needs their effective wavelengths to be more similar.

\subsubsection*{2. Changing the Electron's Energy vs.\ the Photon's Energy}

\begin{itemize}
\item \(\lambda_{\gamma} = \dfrac{hc}{E_{\gamma}}\). If the photon energy \(E_{\gamma}\) is fixed at 2\,eV, \(\lambda_{\gamma}\) is fixed at 620\,nm.
\item \(\lambda_{e} \propto \dfrac{1}{\sqrt{E_{e}}}\). By changing the electron's kinetic energy \(E_{e}\), we can change \(\lambda_{e}\).

If we want \(\lambda_e\) to become \emph{larger}, we need to \emph{decrease} the electron's kinetic energy \(E_{e}\). This will push \(\lambda_e\) toward the (much larger) photon wavelength. 
\end{itemize}

Therefore, the modification that brings the electron's interference maxima closer to the photon's is:
\[
\boxed{\text{Decreasing the electron energy while keeping the photon energy constant.}}
\]

\section*{Step-by-Step Solution}

You have quantum dots that can be (very roughly) modeled as 
\emph{one-dimensional infinite potential wells} of width \(L\). 
You observe that the \emph{longest-wavelength} light they emit 
has \(\lambda_{\mathrm{emit}} = 499.9\,\mathrm{nm}\).  
We want to find \(L\) (in nm), assuming this emission 
corresponds to the transition from the first excited state 
(\(n=2\)) down to the ground state (\(n=1\)) in a 1D infinite well.

\subsection*{1. Energy Levels in a 1D Infinite Well}

For an electron in a 1D infinite potential well of width \(L\),
the allowed energy levels are
\[
E_{n} \;=\; \frac{n^2\,h^2}{8\,m\,L^2},
\quad
n = 1,2,3,\dots
\]
where \(m\) is the (electron) mass and \(h\) is Planck's constant.

\subsection*{2. Energy of the Emitted Photon}

When the electron drops from \(n=2\) to \(n=1\), the energy of the emitted photon is
\[
E_{\mathrm{photon}}
\;=\;
E_{2} - E_{1}
\;=\;
\frac{4\,h^2}{8\,m\,L^2}
\;-\;
\frac{1\,h^2}{8\,m\,L^2}
\;=\;
\frac{3\,h^2}{8\,m\,L^2}.
\]
Meanwhile, the photon energy is related to its wavelength by
\[
E_{\mathrm{photon}} \;=\; \frac{h\,c}{\lambda_{\mathrm{emit}}},
\]
where \(c\) is the speed of light.

\subsection*{3. Setting up the Equation}

Equating the two expressions for \(E_{\mathrm{photon}}\):
\[
\frac{3\,h^2}{8\,m\,L^2}
\;=\;
\frac{h\,c}{\lambda_{\mathrm{emit}}}.
\]
Solve for \(L^2\):
\[
L^2
\;=\;
\frac{3\,h^2\,\lambda_{\mathrm{emit}}}{8\,m\,h\,c}
\;=\;
\frac{3\,h\,\lambda_{\mathrm{emit}}}{8\,m\,c}.
\]
Hence,
\[
L
\;=\;
\sqrt{\frac{3\,h\,\lambda_{\mathrm{emit}}}{8\,m\,c}}.
\]

\subsection*{4. Numerical Calculation}

Use the following constants (SI units):
\[
\begin{aligned}
h &= 6.626\times10^{-34}\,\mathrm{J\cdot s},\\
c &= 2.998\times10^{8}\,\mathrm{m/s},\\
m &= 9.109\times10^{-31}\,\mathrm{kg},\\
\lambda_{\mathrm{emit}} &= 499.9\,\mathrm{nm} \;=\; 4.999\times10^{-7}\,\mathrm{m}.
\end{aligned}
\]
Compute step by step:

\[
L^2
= \frac{3\,h\,\lambda_{\mathrm{emit}}}{8\,m\,c}
\;\approx\;
0.375 \;\times\; \frac{(6.626\times10^{-34})\,(4.999\times10^{-7})}{(9.109\times10^{-31})\,(2.998\times10^{8})}.
\]
One finds numerically
\[
L^2 \;\approx\; 4.54\times10^{-19}\,\mathrm{m}^2
\quad\Longrightarrow\quad
L \;\approx\; 6.74\times10^{-10}\,\mathrm{m}.
\]
Converting to nanometers:
\[
L \;\approx\; 0.674\,\mathrm{nm}.
\]

\subsection*{5. Final Answer (2 Significant Figures)}

Rounding to two significant figures,
\[
\boxed{L \;\approx\; 0.67\,\mathrm{nm}.}
\]
\end{document}
