\documentclass[12pt]{article}

% Packages
\usepackage[margin=1in]{geometry}
\usepackage{amsmath,amssymb,amsthm}
\usepackage{enumitem}
\usepackage{hyperref}
\usepackage{xcolor}
\usepackage{import}
\usepackage{xifthen}
\usepackage{pdfpages}
\usepackage{transparent}
\usepackage{listings}
\usepackage{siunitx}  % Load the package
\usepackage{siunitx} % For units like micrometers



\lstset{
    breaklines=true,         % Enable line wrapping
    breakatwhitespace=false, % Wrap lines even if there's no whitespace
    basicstyle=\ttfamily,    % Use monospaced font
    frame=single,            % Add a frame around the code
    columns=fullflexible,    % Better handling of variable-width fonts
}

\newcommand{\incfig}[1]{%
    \def\svgwidth{\columnwidth}
    \import{./Figures/}{#1.pdf_tex}
}
\theoremstyle{definition} % This style uses normal (non-italicized) text
\newtheorem{solution}{Solution}
\newtheorem{proposition}{Proposition}
\newtheorem{problem}{Problem}
\newtheorem{lemma}{Lemma}
\newtheorem{theorem}{Theorem}
\newtheorem{remark}{Remark}
\newtheorem{note}{Note}
\theoremstyle{plain} % Restore the default style for other theorem environments
%

% Theorem-like environments
% Title information
\title{}
\author{Jerich Lee}
\date{\today}

\begin{document}

\maketitle



\section*{Rayleigh Criterion for Resolving Two Lanterns}

We wish to find the minimum separation \( \Delta x \) between two lanterns so that 
a telescope of diameter \( D \) can resolve them as two distinct points of light at 
wavelength \( \lambda \).  We assume the telescope and lanterns are separated by a distance 
\( L \).  According to the Rayleigh criterion, the smallest resolvable angular separation 
\( \theta \) is given by
\[
    \theta_\mathrm{min} \;=\; 1.22 \,\frac{\lambda}{D}.
\]
Once we know \( \theta_\mathrm{min} \), the linear separation \( \Delta x \) required at 
distance \( L \) is 
\[
    \Delta x \;=\; \theta_\mathrm{min}\,L.
\]

\subsection*{Numerical Values}

\begin{itemize}
  \item Wavelength of the light:
    \[
       \lambda = 550\,\text{nm} 
               = 550 \times 10^{-9}\,\text{m}.
    \]
  \item Diameter of the telescope:
    \[
       D = 2\,\text{cm} 
          = 0.02\,\text{m}.
    \]
  \item Distance from lanterns to observer (in Concord, MA):
    \[
       L = 40\,\text{km} 
          = 4.0 \times 10^{4}\,\text{m}.
    \]
\end{itemize}

\subsection*{Step 1: Compute the Minimum Angular Separation}

\[
    \theta_\mathrm{min}
    \;=\; 1.22 \,\frac{ \lambda }{ D }
    \;=\; 1.22 \,\times \frac{ 550 \times 10^{-9}\,\mathrm{m} }{ 0.02\,\mathrm{m} }.
\]
First, compute the fraction:
\[
   \frac{550 \times 10^{-9}}{0.02}
   = 550 \times 10^{-9} \div 2 \times 10^{-2}
   = 275 \times 10^{-7}
   = 2.75 \times 10^{-5}.
\]
Hence,
\[
    \theta_\mathrm{min}
    = 1.22 \times 2.75 \times 10^{-5}
    = 3.355 \times 10^{-5}\,\mathrm{radians}.
\]

\subsection*{Step 2: Compute the Required Linear Separation}

Multiplying the minimum angular separation by the distance \( L \),
\[
   \Delta x
   = \theta_\mathrm{min} \times L
   = \left(3.355 \times 10^{-5}\right)
     \times \left(4.0 \times 10^4\,\mathrm{m}\right).
\]
\[
   \Delta x 
   = 3.355 \times 4.0 \times 10^{-5 + 4}
   = 13.42 \times 10^{-1}
   = 1.342\,\mathrm{m}.
\]

\subsection*{Conclusion}

The lanterns must be separated by about \(\boxed{1.34\,\mathrm{m}}\) or more 
in order for a \(2\,\mathrm{cm}\) diameter telescope, located \(40\,\mathrm{km}\) away, 
to resolve them as two distinct lights.


\section*{1.\;Reading a Wave Graph}
A typical wave graph might plot either displacement $y$ versus:
\begin{itemize}
  \item \textbf{Time} $t$ (showing wave ``oscillations'' as time progresses), or
  \item \textbf{Position} $x$ (showing a snapshot of the wave in space at a fixed time).
\end{itemize}

\subsection*{Key Wave Quantities}
\begin{itemize}
  \item \textbf{Amplitude} $A$:
    \[
      A = \max |y|,
    \]
    the maximum displacement from the equilibrium (zero) position on the wave graph.
    
  \item \textbf{Wavelength} $\lambda$:
    \[
      \lambda = \text{(distance between two adjacent points in phase)},
    \]
    e.g., the distance between consecutive crests or troughs on a plot of $y$ vs.\ $x$.
    
  \item \textbf{Period} $T$:
    \[
      T = \text{(time between repeating crests or troughs)},
    \]
    on a $y$ vs.\ $t$ graph.  Often we use frequency $f = 1/T$.
    
  \item \textbf{Intensity} $I$:
    \[
      I \propto A^2.
    \]
    If other factors (like distance from the source) are held constant, intensity is proportional
    to the square of the amplitude.
    
  \item \textbf{Wave Speed} $v$:
    \[
      v = \lambda\,f.
    \]
    If you have wavelength $\lambda$ and frequency $f$, the wave speed is their product.
\end{itemize}

\section*{2.\;Rayleigh Criterion and Telescope Resolution}

\subsection*{Rayleigh Criterion (for circular apertures)}
When trying to resolve two distant sources of light (stars, lanterns, etc.), the 
\emph{minimum resolvable angular separation} $\theta_\mathrm{min}$ for a telescope 
of diameter $D$ (assuming diffraction--limited resolution) is:
\[
  \theta_\mathrm{min} = 1.22\,\frac{\lambda}{D}.
\]
Here, $\lambda$ is the wavelength of the light and $D$ is the telescope aperture (diameter).

\subsection*{Case 1:\;Given $D$, find Minimum Separation}
If you already know the telescope diameter $D$, you can find the smallest separation 
$\Delta x$ of two sources (e.g., lanterns, or other point objects) that can be resolved 
at distance $L$ via
\[
  \Delta x = \theta_\mathrm{min} \times L
           = \left(1.22\,\frac{\lambda}{D}\right)\,L.
\]

\subsection*{Case 2:\;Given $\Delta x$ and $L$, find Telescope Diameter $D$}
If the problem instead tells you \emph{how far away} the objects are ($L$) and 
\emph{how far apart} they need to be resolved ($\Delta x$), you can rearrange the formula 
to solve for the \emph{required} telescope diameter $D$.  First note that for small angles,
\[
  \theta_\mathrm{min} \approx \frac{\Delta x}{L}.
\]
Then set
\[
  \frac{\Delta x}{L} = 1.22\,\frac{\lambda}{D}.
\]
Hence, the \textbf{required diameter} $D$ is
\[
  \boxed{
    D = 1.22\,\frac{\lambda\,L}{\Delta x}.
  }
\]

\section*{Example:\;Space Landing Problem}
Suppose a problem states:
\begin{itemize}
  \item The distance to a spacecraft is $L = \SI{4.0e5}{m}$ (just an example).
  \item We want to distinguish two details on the spacecraft that are $\Delta x = \SI{2}{m}$ apart.
  \item Light has wavelength $\lambda = \SI{550e-9}{m}$ (typical visible light).
\end{itemize}
We want the telescope diameter $D$ that can just resolve these details:
\[
  D = 1.22\,\frac{\lambda\,L}{\Delta x}
    = 1.22 \times \frac{(550\times10^{-9}\,\mathrm{m})(4.0\times10^5\,\mathrm{m})}%
                     {2\,\mathrm{m}}
    \approx \SI{0.1342}{m} \;\approx \SI{13.4}{cm}.
\]
So a telescope of about 13--14~cm in diameter would be needed to \emph{just} 
separate those two points under ideal diffraction‐limited conditions.


\section*{Phase Difference of Two Waves}

Consider two coherent wave sources that are in phase with each other 
(i.e., $\Delta \phi_{\,\text{at sources}} = 0$).  
We have:
\[
  \text{Wavelength: } \lambda = 2.72\,\mathrm{m}, 
  \quad
  d_1 = 1.01\,\mathrm{m}, 
  \quad
  d_2 = 1.46\,\mathrm{m}.
\]
The wave number $k$ is
\[
  k \;=\; \frac{2\pi}{\lambda}.
\]

\subsection*{Step 1: Find the Path Difference}
The path difference $\Delta r$ between the two waves is
\[
  \Delta r \;=\; \bigl|\,d_2 - d_1\bigr|
              \;=\; |\,1.46\,\mathrm{m} - 1.01\,\mathrm{m}\,|
              \;=\; 0.45\,\mathrm{m}.
\]

\subsection*{Step 2: Calculate the Phase Difference}
Since the sources are in phase at the start, 
the phase difference $\Delta \phi$ at point $P$ is determined solely by 
the path difference:
\[
  \Delta \phi 
  \;=\; k\,(\Delta r)
  \;=\; \left(\frac{2\pi}{\lambda}\right)\,\bigl(\Delta r\bigr).
\]
Plug in numbers:
\[
  k
  \;=\; \frac{2\pi}{2.72\,\mathrm{m}}
  \;\approx\; 2.31\,\mathrm{m}^{-1},
  \quad
  \Delta r = 0.45\,\mathrm{m}.
\]
Hence,
\[
  \Delta \phi 
  \;=\; 2.31\,\mathrm{m}^{-1} \times 0.45\,\mathrm{m}
  \;\approx\; 1.04\,\mathrm{radians}.
\]

\subsection*{Result}
Therefore, the phase difference of the two waves at the point $P$ is
\[
  \boxed{1.04\,\text{radians (approximately)}.}
\]


\section*{Given}
\[
  y(x,t) \;=\; \sin\bigl(kx - \omega t\bigr),
\]
where
\[
  k = 1.891\,\mathrm{m}^{-1},
  \quad 
  \omega = 1.386\,\mathrm{s}^{-1}.
\]
Waves moving to the right have positive velocity.

\section*{1.\;Wave Velocity}
For a standard traveling wave of the form $f(kx - \omega t)$, 
the phase velocity $v$ is given by
\[
  v \;=\; \frac{\omega}{k}.
\]
Hence,
\[
  v = \frac{1.386\,\mathrm{s}^{-1}}{1.891\,\mathrm{m}^{-1}}
    \;\approx\; 0.733\,\mathrm{m/s}.
\]
Since the argument is $kx - \omega t$ (with a minus sign in front of $\omega t$),
the wave travels in the $+x$ direction. Thus, 
\[
  \boxed{v \;\approx\; +\,0.733\,\mathrm{m/s}.}
\]

\section*{2.\;Phase Change for a Spatial Shift}
The wave phase at any point $(x,t)$ is
\[
  \Phi(x,t) \;=\; k\,x \;-\; \omega\,t.
\]
Holding time $t$ fixed, the phase change $\Delta \Phi$ when $x$ increases by 
$\Delta x = 0.212\,\mathrm{m}$ is
\[
  \Delta \Phi 
  \;=\; k\,\Delta x
  \;=\; \bigl(1.891\,\mathrm{m}^{-1}\bigr)\,\bigl(0.212\,\mathrm{m}\bigr).
\]
Numerically,
\[
  \Delta \Phi 
  \;\approx\; 1.891 \times 0.212
  \;\approx\; 0.401\,\mathrm{radians}.
\]
Hence, 
\[
  \boxed{\Delta \Phi \;\approx\; 0.401\,\mathrm{rad}.}
\]

\section*{Michelson Interferometer: Finding the Wavelength}

\subsection*{Given}
\begin{itemize}
\item When the two arms have equal length ($\Delta x = 0$), the intensity at the detector is 
  \[
    I_{\max} = 6.93 \,\text{W/m}^2
  \]
  (constructive interference).
\item Moving the right mirror by $\Delta x = 87\,\mu\mathrm{m}$ 
  reduces the intensity to
  \[
    I = \frac{I_{\max}}{4} = \frac{6.93}{4} \approx 1.73\,\text{W/m}^2.
  \]
\item Find the wavelength \(\lambda\).
\end{itemize}

\subsection*{Step 1: Relate Intensity Drop to Phase Difference}
For two beams of equal amplitude, the resulting intensity 
(as a function of their phase difference $\phi$) can be written as
\[
  I(\phi) \;=\; I_{\max}\,\cos^2\!\bigl(\phi/2\bigr).
\]
Since $I = I_{\max}/4$, we have
\[
  \frac{I_{\max}}{4} \;=\; I_{\max}\,\cos^2\!\bigl(\phi/2\bigr)
  \;\;\Longrightarrow\;\; \cos^2\!\bigl(\phi/2\bigr) = \tfrac14.
\]
Hence 
\[
  \cos\!\bigl(\phi/2\bigr) = \tfrac12
  \quad\Longrightarrow\quad 
  \phi/2 = \frac{\pi}{3}
  \quad\Longrightarrow\quad 
  \phi = \frac{2\pi}{3}.
\]

\subsection*{Step 2: Path Difference and Wavelength}
Moving the mirror by $\Delta x$ changes the path difference by 
\[
  d = 2\,\Delta x.
\]
The phase difference $\phi$ between the two beams is related to $d$ by
\[
  \phi \;=\; \frac{2\pi}{\lambda}\,d 
  \;=\; \frac{2\pi}{\lambda}\,\bigl(2\,\Delta x\bigr).
\]
We know $\phi = \tfrac{2\pi}{3}$, so
\[
  \frac{2\pi}{\lambda} \,\bigl(2\,\Delta x\bigr)
  \;=\; \frac{2\pi}{3}.
\]
Divide both sides by $2\pi$,
\[
  \frac{2\,\Delta x}{\lambda}
  \;=\; \frac{1}{3}
  \quad\Longrightarrow\quad
  \lambda = 6\,\Delta x.
\]
With $\Delta x = 87\,\mu\mathrm{m}$, we get
\[
  \lambda = 6 \times 87\,\mu\mathrm{m}
          = 522\,\mu\mathrm{m}.
\]

\subsection*{Conclusion}
\[
  \boxed{\lambda = 522\,\mu\mathrm{m}.}
\]


\section*{Problem}

How many photons of frequency $4.00 \times 10^{14} \, \text{Hz}$ should be emitted (in a beam) to change the velocity by $3 \, \text{m/s}$?

\begin{enumerate}
    \item $3.39 \times 10^{36}$
    \item $3.40 \times 10^{20}$
    \item $2.94 \times 10^{-21}$
    \item $\mathbf{3.39 \times 10^{28}}$ \quad \textbf{(Correct Answer)}
    \item $2.95 \times 10^{-37}$
\end{enumerate}
\section*{Follow-Up Question}

If we decrease the frequency of light, how would the number of photons needed change?
\begin{enumerate}
    \item Stay the same
    \item Increase
    \item Decrease \quad \textbf{(Correct Answer)}
\end{enumerate}

\noindent
\begin{enumerate}
  \item 
\end{enumerate}

\begin{align}
  \sum_{s=1}^{\infty} 
\end{align}
$\int_{-1}^{0} 1+x \,\mathrm{d}x + \int_{0}^{1}  \,\mathrm{d}x + \int_{1}^{2} 2-x \,\mathrm{d}x $ 
$\int_{-\infty}^{1} \sqrt{\frac{15}{56}}(1+x^{2})^{2}  \,\mathrm{d}x $ 
$1 =\int_{-\infty}^{\infty} (c_{1}+c_{1}ix^{2})(c_{1}-c_{1}ix^{2})     \,\mathrm{d}x $ 
\end{document}
