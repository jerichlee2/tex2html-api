\documentclass[12pt]{article}

% Packages
\usepackage[margin=1in]{geometry}
\usepackage{amsmath,amssymb,amsthm}
\usepackage{enumitem}
\usepackage{hyperref}
\usepackage{xcolor}
\usepackage{import}
\usepackage{xifthen}
\usepackage{pdfpages}
\usepackage{transparent}
\usepackage{listings}
\usepackage{tikz}

\DeclareMathOperator{\Log}{Log}
\DeclareMathOperator{\Arg}{Arg}

\lstset{
    breaklines=true,         % Enable line wrapping
    breakatwhitespace=false, % Wrap lines even if there's no whitespace
    basicstyle=\ttfamily,    % Use monospaced font
    frame=single,            % Add a frame around the code
    columns=fullflexible,    % Better handling of variable-width fonts
}

\newcommand{\incfig}[1]{%
    \def\svgwidth{\columnwidth}
    \import{./Figures/}{#1.pdf_tex}
}
\theoremstyle{definition} % This style uses normal (non-italicized) text
\newtheorem{solution}{Solution}
\newtheorem{proposition}{Proposition}
\newtheorem{problem}{Problem}
\newtheorem{lemma}{Lemma}
\newtheorem{theorem}{Theorem}
\newtheorem{remark}{Remark}
\newtheorem{note}{Note}
\newtheorem{definition}{Definition}
\newtheorem{example}{Example}
\newtheorem{corollary}{Corollary}
\theoremstyle{plain} % Restore the default style for other theorem environments
%

% Theorem-like environments
% Title information
\title{}
\author{Jerich Lee}
\date{\today}

\begin{document}

\maketitle
\[
\textbf{Given:} \quad C_V = \alpha T \quad\text{with}\quad \alpha = 0.3\,\frac{\text{J}}{\text{K}^2}.
\]

\[
\textbf{We want the change in entropy as the block cools from } T_i = 302\,\text{K} \text{ to } T_f = 250\,\text{K.}
\]
\[
\Delta S \;=\; \int_{T_i}^{T_f} \frac{C_V}{T}\, dT
\;=\; \int_{T_i}^{T_f} \frac{\alpha\,T}{T}\, dT
\;=\; \alpha \int_{T_i}^{T_f} dT
\;=\; \alpha \,\bigl[T\bigr]_{T_i}^{T_f}
\;=\; \alpha\,\bigl(T_f - T_i\bigr).
\]

\[
\text{Substitute } \alpha = 0.3\,\frac{\text{J}}{\text{K}^2}, \quad
T_i = 302\,\text{K}, \quad
T_f = 250\,\text{K}:
\]
\[
\Delta S
\;=\; 0.3\,\frac{\text{J}}{\text{K}^2}\times (250\,\text{K} - 302\,\text{K})
\;=\; 0.3 \times (-52)\,\text{J/K}
\;=\; -15.6\,\text{J/K}.
\]

\[
\text{To two significant figures, } \Delta S \approx \boxed{-16\,\text{J/K}}.
\]

\[
\textbf{Given:}
\quad m = 500\,\mathrm{g}
\quad (\text{silver, with atomic mass }107\,\mathrm{g/mol}),
\quad Q = 397\,\mathrm{J}.
\]

\[
\textbf{Goal:}
\quad \Delta T = \,?
\]

\[
\textbf{Step 1: Number of moles of silver}
\]
\[
n \;=\; \frac{m}{M}
\;=\;
\frac{500\,\mathrm{g}}{107\,\mathrm{g/mol}}
\;\approx\;4.67\,\mathrm{mol}.
\]

\[
\textbf{Step 2: Molar heat capacity for a metal at high temperature (Dulong-Petit)}
\]
\[
C_m \;\approx\; 3R 
\;=\;3 \times 8.314\,\frac{\mathrm{J}}{\mathrm{mol\,K}}
\;\approx\;24.94\,\frac{\mathrm{J}}{\mathrm{mol\,K}}.
\]

\[
\textbf{Step 3: Total (bulk) heat capacity of the 500 g silver sample}
\]
\[
C_{\text{total}}
\;=\;
n\,C_m
\;=\;
4.67\,\mathrm{mol}\times 24.94\,\frac{\mathrm{J}}{\mathrm{mol\,K}}
\;\approx\;116.5\,\frac{\mathrm{J}}{\mathrm{K}}.
\]

\[
\textbf{Step 4: Temperature change from absorbing } Q=397\,\mathrm{J}
\]
\[
\Delta T
\;=\;\frac{Q}{C_{\text{total}}}
\;=\;\frac{397\,\mathrm{J}}{116.5\,\mathrm{J/K}}
\;\approx\;3.41\,\mathrm{K}.
\]

\[
\textbf{Answer (2 sig figs): }\boxed{3.4\,\mathrm{K}}
\] 

\[
\textbf{Given Data:}
\]
- Mass of block 1: \(m_1 = 5\,\text{kg}\)
- Initial velocity of block 1: \(v_1 = +1\,\text{m/s}\)
- Mass of block 2: \(m_2 = 3\,\text{kg}\)
- Initial velocity of block 2: \(v_2 = -1\,\text{m/s}\)  (opposite direction)
- After collision: the blocks stick together (perfectly inelastic collision), moving as a single unit.
- System is thermally isolated (no heat exchange with the environment).

---

1. Momentum of the two-block system after the collision, \(p_f\)

Because **momentum is conserved**, the final momentum of the combined mass must equal the total initial momentum:

\[
p_i \;=\; m_1 v_1 \;+\; m_2 v_2 
\;=\; (5\,\mathrm{kg})(+1\,\mathrm{m/s})
\;+\; (3\,\mathrm{kg})(-1\,\mathrm{m/s})
\;=\; 5 \;-\; 3
\;=\; 2\,\mathrm{kg\cdot m/s}.
\]

In a perfectly inelastic collision, the two masses stick together and move with a common final velocity \(v_f\). Since momentum is conserved:

\[
p_f \;=\; (m_1 + m_2)\,v_f \;=\; p_i.
\]
\[
\implies (5 + 3)\,\mathrm{kg} \;\times\; v_f 
\;=\; 2\,\mathrm{kg\cdot m/s}.
\]
\[
\implies v_f 
\;=\; \frac{2}{8}\;\mathrm{m/s}
\;=\; 0.25\,\mathrm{m/s}.
\]
But the question about \(p_f\) (the final total momentum) is simply:
\[
p_f = 2\,\mathrm{kg\cdot m/s}.
\]
Hence, the momentum of the two-block system after the collision is:
\[
\boxed{2\,\mathrm{kg\cdot m/s}}.
\]

---

2. Change in internal energy, \(\Delta U\)

**Step 1: Compute initial kinetic energy, \(KE_i\).**

\[
KE_i 
\;=\; \tfrac{1}{2} m_1 v_1^2 \;+\;\tfrac{1}{2} m_2 v_2^2
\;=\; \tfrac{1}{2}(5)(1^2) \;+\; \tfrac{1}{2}(3)(1^2)
\;=\; 2.5 \;+\; 1.5
\;=\; 4.0\,\text{J}.
\]

**Step 2: Compute final kinetic energy, \(KE_f\).**

After collision, the combined mass is \(m_1 + m_2 = 8\,\text{kg}\), and the final velocity is \(v_f = 0.25\,\text{m/s}\). Thus:

\[
KE_f 
\;=\; \tfrac{1}{2}\,(m_1 + m_2)\,v_f^2
\;=\; \tfrac{1}{2}\,(8\,\text{kg})\,(0.25\,\text{m/s})^2.
\]
\[
(0.25\,\text{m/s})^2 = 0.0625\,\text{(m/s)}^2,
\]
\[
\implies KE_f
\;=\; \tfrac{1}{2}\,\bigl(8 \times 0.0625\bigr)\,\text{J}
\;=\; \tfrac{1}{2}\,(0.5)\,\text{J}
\;=\; 0.25\,\text{J}.
\]

**Step 3: Change in internal energy \(\Delta U\).**

By the conservation of total energy (in a thermally isolated system), any loss in mechanical (kinetic) energy appears as an increase in internal energy of the blocks (e.g., deformation, heat-like vibration, etc.). So:

\[
\Delta U 
\;=\; -(KE_f - KE_i) 
\;=\; KE_i \;-\; KE_f
\;=\; 4.0\,\text{J} \;-\; 0.25\,\text{J}
\;=\; 3.75\,\text{J}.
\]

Therefore, the internal energy of the two-block system **increases** by \(3.75\,\text{J}\). Hence:

\[
\boxed{\Delta U = +3.75\,\text{J}}.
\]

---

3. Change in entropy of the two-block system, \(\Delta S\)

This is a **perfectly inelastic** collision in an isolated system; it is an **irreversible** process. Irreversible processes are associated with an **increase** in entropy. Since some macroscopic kinetic energy was irreversibly converted into internal energy (random motion at the microscopic level), the entropy of the system must increase:

\[
\boxed{\Delta S > 0}.
\] 

\[
\textbf{Given:} \quad \Omega = A\,U,
\quad A = 8.7\,\mathrm{eV}^{-1},
\quad \text{System at constant volume, temperature } T = 281\,\mathrm{K}.
\]

\[
\textbf{We know that:} \quad S = k_B \ln(\Omega) 
            = k_B \ln\bigl(A U\bigr)
            = k_B \bigl[\ln(A) + \ln(U)\bigr].
\]
Taking the derivative at constant volume,
\[
\frac{\partial S}{\partial U}
= k_B \cdot \frac{1}{U}.
\]
From the thermodynamic relation,
\[
\frac{1}{T} 
= \left(\frac{\partial S}{\partial U}\right)_V 
= \frac{k_B}{U}.
\]
Hence,
\[
U = k_B\,T.
\]
Since we want \(U\) in eV, we use:
\[
k_B = 8.617 \times 10^{-5}\,\frac{\mathrm{eV}}{\mathrm{K}}.
\]
Thus, at \(T=281\,\mathrm{K}\):
\[
U 
= \bigl(8.617\times 10^{-5}\,\mathrm{eV/K}\bigr)\,\times 281\,\mathrm{K}
\;\approx\;0.0242\,\mathrm{eV}.
\]

\[
\boxed{U \approx 0.024\,\mathrm{eV}} 
\quad \text{(to three significant figures).}
\] 

\[
\textbf{Problem Statement:}
\]
A 300000-liter (i.e., \(3.0\times10^5\,\mathrm{L}\)) space station contains:
- 1000 moles of \(\mathrm{O_2}\) gas (diatomic, at room temperature we ignore vibration, only translational + rotational)
- 9000 moles of He gas (monatomic, only translational)
All at an initial temperature of \(20^\circ\mathrm{C}\) (\(T_i = 293\,\mathrm{K}\)). The system is at constant volume.

\[
\textbf{A)}\quad \text{What is the pressure inside the box?}
\]
\[
\textbf{B)}\quad \text{How much heat is required to raise the temperature to }30^\circ\mathrm{C}
             \text{ (i.e., }T_f = 303\,\mathrm{K}\text{), at constant volume?}
\]

---

Part A: Pressure inside the box

Use the ideal gas law in the form
\[
pV = nRT.
\]

1. **Total moles, \(n\):**  
   \[
   n_{\mathrm{total}} = n_{\mathrm{O_2}} + n_{\mathrm{He}}
   = 1000 + 9000
   = 10000\,\text{moles}.
   \]

2. **Convert volume to liters and temperature to Kelvin**  
   Volume \(V = 300000\,\mathrm{L}\).  
   Temperature \(T = 20^\circ\mathrm{C} = 293\,\mathrm{K}.\)

3. **Use \(R = 0.082057\,\mathrm{L\cdot atm\;mol^{-1}\;K^{-1}}\) to get \(p\) in atm:**  
   \[
   p = \frac{nRT}{V}
     = \frac{(10000)\,\bigl(0.082057\,\mathrm{L\cdot atm/(mol\cdot K)}\bigr)\,(293\,\mathrm{K})}
            {300000\,\mathrm{L}}.
   \]
   First compute \(nR\):
   \[
   nR = 10000 \times 0.082057 \approx 820.57.
   \]
   Then
   \[
   nRT \approx 820.57 \times 293 \;\approx\; 240{,}427\;\mathrm{L\cdot atm}.
   \]
   Finally divide by \(V=300000\,\mathrm{L}\):
   \[
   p \;=\; \frac{240{,}427}{300{,}000}
          \;\approx\; 0.80\,\mathrm{atm}.
   \]

\[
\boxed{p \approx 0.80\,\mathrm{atm}.}
\]

---

Part B: Heat required to raise the gas mixture from \(293\,\mathrm{K}\) to \(303\,\mathrm{K}\) at constant volume

At constant volume, the required heat \(Q\) equals the change in internal energy \(\Delta U\). For an ideal gas mixture,  
\[
Q = \Delta U = \sum_i n_i\, C_{V,i}\,\Delta T.
\]
Here \(\Delta T = 303 - 293 = 10\,\mathrm{K}\). We assume only translational and rotational degrees of freedom are excited:

1. **Helium (He), monatomic:**  
   - Degrees of freedom: 3 (translational only).  
   - Hence \(C_{V,\mathrm{He}} = \tfrac{3}{2}R.\)  
   - Number of moles \(n_{\mathrm{He}} = 9000.\)

   Contribution to heat:
   \[
   Q_{\mathrm{He}}
     = n_{\mathrm{He}}\,C_{V,\mathrm{He}}\,\Delta T
     = (9000)\,\Bigl(\tfrac{3}{2}R\Bigr)\,(10\,\mathrm{K})
     = (9000)\times(1.5\,R)\times(10)
     = 135{,}000\,R.
   \]

2. **Oxygen (\(\mathrm{O_2}\)), diatomic:**  
   - At room temperature, vibrational modes are (mostly) not excited, so dof = 3 (translation) + 2 (rotation) = 5 total.  
   - Hence \(C_{V,\mathrm{O_2}} = \tfrac{5}{2}R.\)  
   - Number of moles \(n_{\mathrm{O_2}} = 1000.\)

   Contribution to heat:
   \[
   Q_{\mathrm{O_2}}
     = n_{\mathrm{O_2}}\,C_{V,\mathrm{O_2}}\,\Delta T
     = (1000)\,\Bigl(\tfrac{5}{2}R\Bigr)\,(10\,\mathrm{K})
     = (1000)\times(2.5\,R)\times(10)
     = 25{,}000\,R.
   \]

3. **Total heat:**
   \[
   Q_{\mathrm{total}} = Q_{\mathrm{He}} + Q_{\mathrm{O_2}}
                      = 135{,}000\,R + 25{,}000\,R
                      = 160{,}000\,R.
   \]
   In Joules, use \(R \approx 8.314\,\mathrm{J/(mol\cdot K)}\):
   \[
   Q_{\mathrm{total}}
   \;\approx\; (160{,}000)\,\bigl(8.314\,\mathrm{J/(mol\,K)}\bigr)
   \;\approx\; 1.330 \times 10^6\,\mathrm{J}.
   \]
   More precisely:
   \[
   Q_{\mathrm{He}} = 135{,}000 \times 8.314 \approx 1{,}122{,}390\,\mathrm{J},
   \quad
   Q_{\mathrm{O_2}} = 25{,}000 \times 8.314 \approx 207{,}850\,\mathrm{J},
   \]
   \[
   Q_{\mathrm{total}} \approx 1{,}122{,}390 \;+\; 207{,}850 
                       = 1{,}330{,}240\,\mathrm{J}
                       \;\approx\; 1.33 \times 10^6\,\mathrm{J}.
   \]

\[
\boxed{Q \;\approx\; 1.33 \times 10^6\,\mathrm{J}.}
\]

---

\[
\textbf{Answers:}
\]
1. Pressure inside the box: \(\boxed{0.8\,\mathrm{atm}}\).
2. Heat needed to raise from \(20^\circ\mathrm{C}\) to \(30^\circ\mathrm{C}\) (at constant volume): \(\boxed{1.33\times 10^6\,\mathrm{J}}\).

\[
\textbf{Given:}
\]
- Copper container mass: \(m_\text{Cu} = 0.3\,\mathrm{kg}\).
- Specific heat of copper: \(c_\text{Cu} = 386\,\mathrm{J/(kg\cdot K)}\).
- Helium gas: \(n_\text{He} = 1.5\,\mathrm{mol}\).
- Molar heat capacity (at constant volume) for helium: \(C_{V,\text{He}} = 12.5\,\mathrm{J/(mol\cdot K)}\).
- Initial temperature of helium: \(T_{\text{He},i} = 101^\circ\mathrm{C} = 374.15\,\mathrm{K}\).
- Initial temperature of copper container: \(T_{\text{Cu},i} = 20^\circ\mathrm{C} = 293.15\,\mathrm{K}\).
- Thermally isolated system \(\implies\) no heat exchange with environment, so \(\Delta U_{\text{total}}=0\).

\[
\textbf{Goal: find the final equilibrium temperature } T_f.
\]

---

1. Energy (Heat) Balance

Because the system is isolated, the heat lost by the hotter component (helium) must equal the heat gained by the colder component (copper).  In symbols,

\[
Q_{\text{He}} + Q_{\text{Cu}} = 0
\quad\Longrightarrow\quad
n_\text{He}\,C_{V,\text{He}}\,(T_f - T_{\text{He},i})
\;+\;
m_\text{Cu}\,c_\text{Cu}\,(T_f - T_{\text{Cu},i})
\;=\;0.
\]

Note that \(T_{\text{He},i} > T_{\text{Cu},i}\), so helium cools down \((T_f - T_{\text{He},i})<0\), and copper warms up \((T_f - T_{\text{Cu},i})>0\).

---

2. Plug in the Numbers

1. Heat capacity of the copper container:
   \[
   C_{\text{Cu}}
   = m_{\text{Cu}}\; c_{\text{Cu}}
   = (0.3\,\mathrm{kg})\times(386\,\mathrm{J/(kg\cdot K)})
   \;=\;115.8\,\mathrm{J/K}.
   \]

2. Total (constant‐volume) heat capacity of the helium gas:
   \[
   C_{V,\text{He, total}}
   = n_{\text{He}}\; C_{V,\text{He}}
   = (1.5\,\mathrm{mol})\times(12.5\,\mathrm{J/(mol\cdot K)})
   \;=\;18.75\,\mathrm{J/K}.
   \]

Hence the energy‐balance equation becomes
\[
18.75\,(T_f - 374.15)
\;+\;
115.8\,(T_f - 293.15)
\;=\;0.
\]

---

3. Solve for \(T_f\)

Combine like terms:
\[
(18.75 + 115.8)\, T_f
\;-\;
\bigl(18.75\cdot 374.15 + 115.8\cdot 293.15\bigr)
\;=\;0.
\]
\[
(134.55)\, T_f
\;=\;
18.75 \times 374.15 \;+\; 115.8 \times 293.15.
\]

Calculate the right side numerically:

\[
18.75\times374.15 \;\approx\; 7015.31,
\quad
115.8\times293.15 \;\approx\;33946.77,
\]
\[
\text{So,}
\quad
134.55\,T_f 
\;=\; 7015.31 + 33946.77
\;=\;40962.08.
\]
Finally,
\[
T_f
\;=\;\frac{40962.08}{134.55}
\;\approx\;304.4\,\mathrm{K}.
\]

---

4. Conclusion

The final equilibrium temperature of the helium–copper system is

\[
\boxed{T_f \approx 304.4\;\mathrm{K}\quad (\text{about }31.3^\circ\mathrm{C}).
}
\] 
\end{document}
