\documentclass[12pt]{article}

% Packages
\usepackage[margin=1in]{geometry}
\usepackage{amsmath,amssymb,amsthm}
\usepackage{enumitem}
\usepackage{hyperref}
\usepackage{xcolor}
\usepackage{import}
\usepackage{xifthen}
\usepackage{pdfpages}
\usepackage{transparent}
\usepackage{listings}
\usepackage{tikz}
\usepackage{physics}
\usepackage{siunitx}
  \usetikzlibrary{calc,patterns,arrows.meta,decorations.markings}


\DeclareMathOperator{\Log}{Log}
\DeclareMathOperator{\Arg}{Arg}

\lstset{
    breaklines=true,         % Enable line wrapping
    breakatwhitespace=false, % Wrap lines even if there's no whitespace
    basicstyle=\ttfamily,    % Use monospaced font
    frame=single,            % Add a frame around the code
    columns=fullflexible,    % Better handling of variable-width fonts
}

\newcommand{\incfig}[1]{%
    \def\svgwidth{\columnwidth}
    \import{./Figures/}{#1.pdf_tex}
}
\theoremstyle{definition} % This style uses normal (non-italicized) text
\newtheorem{solution}{Solution}
\newtheorem{proposition}{Proposition}
\newtheorem{problem}{Problem}
\newtheorem{lemma}{Lemma}
\newtheorem{theorem}{Theorem}
\newtheorem{remark}{Remark}
\newtheorem{note}{Note}
\newtheorem{definition}{Definition}
\newtheorem{example}{Example}
\newtheorem{corollary}{Corollary}
\theoremstyle{plain} % Restore the default style for other theorem environments
%

% Theorem-like environments
% Title information
\title{}
\author{Jerich Lee}
\date{\today}

\begin{document}

\maketitle

\section*{Latent heat of vaporisation per \(\mathbf{H_2O}\) molecule}

Given (at \(100^{\circ}\mathrm{C},\,1\;\text{atm}\))  
\[
  L_{\text{water}\to\text{steam}}
  = 2256\;\text{kJ\,kg}^{-1}
  = 2.256\times10^{6}\;\text{J\,kg}^{-1}
\]
our goal is to convert this macroscopic value to the energy required for
\emph{one} water molecule.

\subsection*{1.  Convert “per kilogram’’ to “per mole’’}

\[
  \begin{aligned}
    M_{\,\mathrm{H_2O}}
    &= 18.01528\;\text{g\,mol}^{-1}
      = 0.01801528\;\text{kg\,mol}^{-1}, \\[6pt]
    L_{\text{per mol}}
    &= L_{\text{per kg}}\;M_{\,\mathrm{H_2O}} \\[4pt]
    &= \bigl(2.256\times10^{6}\;\text{J\,kg}^{-1}\bigr)
       \bigl(0.01801528\;\text{kg\,mol}^{-1}\bigr) \\[4pt]
    &\approx 4.06\times10^{4}\;\text{J\,mol}^{-1}.
  \end{aligned}
\]

\subsection*{2.  Convert “per mole’’ to “per molecule’’}

\[
  \begin{aligned}
    N_A &= 6.02214076\times10^{23}\;\text{mol}^{-1}, \\[6pt]
    \varepsilon_{\text{molecule}}
    &= \frac{L_{\text{per mol}}}{N_A} \\[4pt]
    &= \frac{4.064\times10^{4}\;\text{J\,mol}^{-1}}
            {6.02214076\times10^{23}\;\text{mol}^{-1}} \\[4pt]
    &\approx 6.75\times10^{-20}\;\text{J}.
  \end{aligned}
\]

\subsection*{3.  Convert Joules to electron-volts}

\[
  1\;\text{eV} = 1.602176634\times10^{-19}\;\text{J},
  \qquad
  \varepsilon_{\text{molecule}}
  = \frac{6.75\times10^{-20}\;\text{J}}{1.602176634\times10^{-19}\;\text{J/eV}}
  \approx 0.42\;\text{eV}.
\]

\[
  \boxed{
    \varepsilon_{\text{molecule}}
    \;\approx\;
    6.8\times10^{-20}\;\text{J}
    \;\;(\text{or}\;\;0.42\;\text{eV})
  }.
\]

Thus, vaporising a single \(\mathrm{H_2O}\) molecule requires only a
\(\sim\!\!10^{-19}\)-joule energy packet — about two-fifths of an
electron-volt.
\paragraph{Answer to part B}

Water boils when its saturated‐vapor pressure equals the external (atmospheric) pressure.  
At sea level this pressure is one standard atmosphere:

\[
  P_{\text{atm}}
  = 1\;\text{atm}
  = 1.01325\times10^{5}\;\text{Pa}.
\]

Hence the equilibrium vapour pressure of pure water at \(373\;\text{K}\) is

\[
  \boxed{P_{\text{vapor}}(373\;\text{K}) \;=\; 1.013\times10^{5}\;\text{Pa}}.
\]
\paragraph{Answer to part C}

At \(T = 273\;\text{K}\) (the melting point at \(1\;\text{atm}\)),  
solid ice \((\mathrm{s})\) and liquid water \((\mathrm{l})\) coexist in equilibrium:

\[
  \mu_{\mathrm{ice}}(T{=}273\;\text{K},P)
  \;=\;
  \mu_{\mathrm{water}}(T{=}273\;\text{K},P).
\]

Because each condensed phase is in equilibrium with \emph{the same} water–vapour phase \((\mathrm{g})\), their chemical potentials must also equal the chemical potential of that vapour:

\[
  \mu_{\mathrm{ice}} = \mu_{\mathrm{water}} = \mu_{\mathrm{vapour}} .
\]

The chemical potential of an ideal vapour depends on its pressure \(P_{\text{vap}}\) via  
\(\mu_{\mathrm{vapour}} = \mu^{\circ}(T) + k_B T \ln P_{\text{vap}}\).
Equality of \(\mu_{\mathrm{vapour}}\) for both equilibria therefore demands

\[
  \boxed{\,P_{\text{vap}}^{(\mathrm{ice})}(273\;\text{K})
         \;=\;
         P_{\text{vap}}^{(\mathrm{water})}(273\;\text{K})\,}.
\]

\noindent
Hence, at \(273\;\text{K}\) the saturated vapour pressure above pure ice
is exactly the same as the saturated vapour pressure above pure liquid
water.
\section*{Part D Clausius-Clapeyron on a \(\ln p\)-vs.-\(1/T\) plot}

\subsection*{1.  Starting point}

The simplified Clausius–Clapeyron form for a single molecule is  
\[
  p(T)\;=\;p_{0}\,\exp\!\Bigl(-\tfrac{L}{k_B\,T}\Bigr),
\]
where  

\[
  L = \text{latent heat per molecule},   \qquad 
  k_B = 1.380649\times10^{-23}\,\mathrm{J\,K^{-1}} .
\]

\subsection*{2.  Take the natural logarithm}

\[
  \boxed{\;
    \ln p
    \;=\;
    \ln p_{0}
    \;-\;\frac{L}{k_B}\,\Bigl(\frac{1}{T}\Bigr)
  \;} .
\]

\subsection*{3.  Implications for a \(\ln p\) vs.\ \(1/T\) graph}

\begin{itemize}
  \item The graph is a \emph{straight line}.  
        \(\ln p\) (vertical axis) plotted against \(1/T\) (horizontal axis) has  
        \[
          \text{slope} = -\,\frac{L}{k_B}, 
          \qquad
          \text{intercept at } \frac{1}{T}=0 \text{ is } \ln p_0 .
        \]
  \item Because the slope is negative, the line falls as \(1/T\) increases  
        (i.e.\ as \(T\) decreases).
\end{itemize}

\subsection*{4.  How to obtain \(L\) from the plot}

Measure the slope \(m\) of the straight line:

\[
  m = \frac{\Delta(\ln p)}{\Delta(1/T)} \;=\; -\,\frac{L}{k_B}.
\]

\[
  \boxed{\,L = -\,m\,k_B\,}.
\]

\noindent
Thus, multiplying the \emph{negative} slope by \(-k_B\) gives the latent heat per molecule.  
(If you prefer \(L\) per \(\mathrm{mol}\) or per \(\mathrm{kg}\), multiply by Avogadro’s number or divide by the molar mass, respectively.)
\section*{Part E Boiling point of water at \(2\;\text{atm}\)}

\subsection*{1.  Integrated Clausius–Clapeyron relation}

For two points on the coexistence curve,

\[
  \ln\!\Bigl(\tfrac{P_2}{P_1}\Bigr)
  \;=\;
  -\,\frac{L}{k_B}
  \!\left(
     \frac{1}{T_2} - \frac{1}{T_1}
  \right),
  \tag{1}
\]
where  
\(L\) = latent heat \emph{per molecule} (from part A)  
and \(k_B\) is Boltzmann’s constant.

\subsection*{2.  Insert known values}

\[
  \begin{aligned}
    P_1 &= 1\;\text{atm}, & T_1 &= 373\;\text{K},\\
    P_2 &= 2\;\text{atm}, & \ln\!\bigl(\tfrac{P_2}{P_1}\bigr) &= \ln 2 = 0.6931,\\
    L   &= 6.75\times10^{-20}\;\text{J}, &
    k_B &= 1.380649\times10^{-23}\;\text{J\,K}^{-1}.
  \end{aligned}
\]

\subsection*{3.  Solve eq.\,(1) for \(T_2\)}

\[
  \frac{1}{T_2}
  = \frac{1}{T_1}
    -\Bigl(\frac{k_B}{L}\Bigr)\,
     \ln\!\Bigl(\tfrac{P_2}{P_1}\Bigr)
  \;=\;
  \frac{1}{373}
  -\frac{1.380649\times10^{-23}}{6.75\times10^{-20}}\,
   (0.6931).
\]

\[
  \frac{1}{T_2}
  \approx 0.002680
          - 0.000142
  = 0.002538\;\text{K}^{-1},
  \qquad
  \Longrightarrow\quad
  T_2 \approx 3.94\times10^{2}\;\text{K}.
\]

\[
  \boxed{T_2 \;\approx\; 394\;\text{K}
         \;\;(= 121^{\circ}\text{C}) }.
\]

\subsection*{4.  Interpretation}

A pressure cooker that holds water vapour at \(2\;\text{atm}\) operates
around \(121^{\circ}\text{C}\), which shortens cooking times by raising the
boiling point relative to ordinary atmospheric conditions.
\section*{Part F Qualitative heat-capacity curve for an Einstein solid}

\subsection*{1.  Heat capacity from the Einstein energy}

The average internal energy of \(N\) independent 3-D Einstein oscillators is  

\[
  U(T)\;=\;\frac{3N\varepsilon}{e^{\varepsilon/k_BT}-1},
\qquad
  \varepsilon = \hbar\omega .
\]

Differentiating with respect to \(T\) gives the molar* heat capacity  

\[
  \boxed{%
    C(T)\;=\;\frac{\mathrm{d}U}{\mathrm{d}T}
      \;=\;
      3Nk_B
      \biggl(\frac{\varepsilon}{k_BT}\biggr)^{\!2}
      \frac{e^{\varepsilon/k_BT}}
           {\Bigl(e^{\varepsilon/k_BT}-1\Bigr)^{2}}
  }.
\]

\emph{Low-\(T\) limit (\(T\ll\varepsilon/k_B\)).}  
\(e^{\varepsilon/k_BT}\gg1\Rightarrow C\propto T^{-2}e^{-\varepsilon/k_BT}\).
Heat capacity is \emph{exponentially suppressed}, approaching \(0\).

\emph{High-\(T\) limit (\(T\gg\varepsilon/k_B\)).}
\(e^{\varepsilon/k_BT}\simeq1+\varepsilon/k_BT\Rightarrow C\to3Nk_B\).  
This is the Dulong–Petit plateau.

\subsection*{2.  Sketch of \(C(T)\)}

\begin{center}
\begin{tikzpicture}[scale=1.1]
  \draw[->] (0,0) -- (0,4.2) node[left] {$c$};
  \draw[->] (0,0) -- (7,0) node[below] {$T$};
  % plateau level
  \draw[dashed] (0,3.5) -- (7,3.5) node[right] {$3Nk_B$};
  % curve
  \draw[line width=1pt,domain=0:6.8,samples=120,smooth]
      plot(\x,{3.5*(1-exp(-2.3*\x))});
  % labels
  \node at (1.4,0.4) {$\displaystyle C\!\sim\! e^{-\varepsilon/k_BT}$};
  \node at (5.8,3.1) {$\displaystyle C\!\to\!3Nk_B$};
\end{tikzpicture}
\end{center}

\noindent
Key features to include in any hand-drawn sketch:

* Starts at \(C=0\) when \(T=0\).  
* Rises steeply (almost vertical) for \(T\gtrsim \varepsilon/5k_B\).  
* Smoothly approaches the horizontal asymptote \(C=3Nk_B\) from below.  

\subsection*{3.  How to interpret the curve}

* **Quantum regime (\(k_BT\ll\varepsilon\))** – few oscillators are thermally
  excited, so the solid stores very little energy; \(C\) is negligible.
* **Intermediate regime** – as \(T\) increases, more oscillators are
  excited and \(C\) rises rapidly.
* **Classical regime (\(k_BT\gg\varepsilon\))** – each of the
  \(3N\) harmonic degrees of freedom contributes \(k_B\) to \(C\),
  giving the constant Dulong–Petit value \(3Nk_B\).

---

\emph{*Molar quantity: if you prefer per mole, replace \(N\) by
Avogadro’s number \(N_A\), so the plateau is \(3R\).}
\end{document}
