\documentclass[12pt]{article}

% Packages
\usepackage[margin=1in]{geometry}
\usepackage{amsmath,amssymb,amsthm}
\usepackage{enumitem}
\usepackage{hyperref}
\usepackage{xcolor}
\usepackage{import}
\usepackage{xifthen}
\usepackage{pdfpages}
\usepackage{transparent}
\usepackage{listings}
\usepackage{tikz}
\usepackage{physics}
\usepackage{siunitx}
  \usetikzlibrary{calc,patterns,arrows.meta,decorations.markings}


\DeclareMathOperator{\Log}{Log}
\DeclareMathOperator{\Arg}{Arg}

\lstset{
    breaklines=true,         % Enable line wrapping
    breakatwhitespace=false, % Wrap lines even if there's no whitespace
    basicstyle=\ttfamily,    % Use monospaced font
    frame=single,            % Add a frame around the code
    columns=fullflexible,    % Better handling of variable-width fonts
}

\newcommand{\incfig}[1]{%
    \def\svgwidth{\columnwidth}
    \import{./Figures/}{#1.pdf_tex}
}
\theoremstyle{definition} % This style uses normal (non-italicized) text
\newtheorem{solution}{Solution}
\newtheorem{proposition}{Proposition}
\newtheorem{problem}{Problem}
\newtheorem{lemma}{Lemma}
\newtheorem{theorem}{Theorem}
\newtheorem{remark}{Remark}
\newtheorem{note}{Note}
\newtheorem{definition}{Definition}
\newtheorem{example}{Example}
\newtheorem{corollary}{Corollary}
\theoremstyle{plain} % Restore the default style for other theorem environments
%

% Theorem-like environments
% Title information
\title{}
\author{Jerich Lee}
\date{\today}

\begin{document}

\maketitle
Below is a fully self-contained LaTeX write-up that walks through the counting step-by-step and ends with the numerical answer.
You can copy-paste it directly into your document or LaTeX editor.

\begin{solution}
\textbf{System description.}  
We have three \emph{distinguishable} simple harmonic oscillators (SHOs).  
If oscillator $i$ has $n_i$ quanta of excitation, its energy is $E_i = n_i\,\varepsilon$ with
\(n_i = 0,1,2,\dots\).  
The total energy is fixed at
\[
U \;=\; 3\varepsilon
      \;\;\Longrightarrow\;\;
      n_1 + n_2 + n_3 = 3.
\]

\medskip
\textbf{Goal.}  
Count the number $\Omega(3)$ of ordered triples $(n_1,n_2,n_3)$ of
non–negative integers that satisfy the constraint above.

\medskip
\textbf{Method 1: “Stars and Bars”.}  
The standard combinatorial result for the number of non-negative integer
solutions to
\[
n_1 + n_2 + \dots + n_k = N
\]
is
\[
\binom{N+k-1}{k-1}.
\]
Here $k=3$ (three oscillators) and $N=3$ (three quanta), hence
\[
\boxed{\;
\Omega(3)
  = \binom{3+3-1}{3-1}
  = \binom{5}{2}
  = 10
\; }.
\]

\medskip
\textbf{Method 2: Explicit enumeration (check).}
\[
\begin{aligned}
(3,0,0),\;(0,3,0),\;(0,0,3) &\;\;\Rightarrow\; 3 \text{ states} \\[4pt]
(2,1,0),\;(2,0,1),\;(1,2,0),\;(0,2,1),\;(1,0,2),\;(0,1,2)
 &\;\;\Rightarrow\; 6 \text{ states} \\[4pt]
(1,1,1) &\;\;\Rightarrow\; 1 \text{ state}
\end{aligned}
\]
Summing these,
\[
3 + 6 + 1 = 10,
\]
in agreement with the stars-and-bars result.

\medskip
\textbf{Answer.}\quad
There are \(\boxed{10}\) microstates with total energy \(U = 3\varepsilon\).
\end{solution}

\begin{solution}
  \textbf{Given:} Three \emph{distinguishable} SHOs with  
  \[
  n_1+n_2+n_3 = 5, \qquad n_i \in \{0,1,2,\dots\}.
  \]
  
  %--------------------------------------------------------------------
  \subsection*{2) Total number of microstates for $U = 5\varepsilon$}
  
  The stars–and–bars formula for the number of non-negative integer
  solutions to \(n_1+\dots+n_k=N\) is
  \(\displaystyle \binom{N+k-1}{k-1}\).
  Here \(k=3\) and \(N=5\):
  \[
  \boxed{\;
  \Omega(5) \;=\; \binom{5+3-1}{3-1}
             \;=\; \binom{7}{2}
             \;=\; 21
  \;} .
  \]
  
  %--------------------------------------------------------------------
  \subsection*{3) Probability that oscillator \#1 has zero energy}
  
  Fix \(n_1 = 0\).  
  The remaining two oscillators must satisfy \(n_2+n_3=5\).
  Counting those microstates:
  \[
  \Omega\!\bigl(E_1=0,U=5\varepsilon\bigr)
    = \binom{5+2-1}{2-1}
    = \binom{6}{1}
    = 6.
  \]
  Hence
  \[
  \boxed{\;
  P\!\bigl(E_1=0\bigr)
    = \frac{\Omega(E_1=0,U=5\varepsilon)}{\Omega(U=5\varepsilon)}
    = \frac{6}{21}
    = \frac{2}{7}
    \approx 0.286
  \;} .
  \]
  
  %--------------------------------------------------------------------
  \subsection*{4) Probability that oscillator \#1 has energy $4\varepsilon$}
  
  Fix \(n_1 = 4\).  
  Then \(n_2+n_3 = 1\).  The number of solutions is
  \[
  \Omega\!\bigl(E_1=4\varepsilon,U=5\varepsilon\bigr)
    = \binom{1+2-1}{2-1}
    = \binom{2}{1}
    = 2.
  \]
  Therefore
  \[
  \boxed{\;
  P\!\bigl(E_1 = 4\varepsilon\bigr)
    = \frac{2}{21}
    \approx 0.0952
  \;} .
  \]
  \end{solution}
  \begin{solution}
    \textbf{Given data}
    \[
    f = 2.2\times10^{12}\,\mathrm{Hz},
    \qquad
    h = 6.626\times10^{-34}\,\mathrm{J\,s},
    \qquad
    k_B = 1.380\,649\times10^{-23}\,\mathrm{J\,K^{-1}} .
    \]
    
    \medskip
    \textbf{Boltzmann ratio for a harmonic oscillator}
    
    For non-degenerate levels the probability of the first excited state
    relative to the ground state is
    \[
    \frac{P_1}{P_0}
       = \frac{e^{-\varepsilon/(k_BT)}}{e^{0}}
       = e^{-\varepsilon/(k_BT)} .
    \]
    
    \medskip
    \textbf{Set the required ratio and solve for \(T\)}
    
    We want \(P_1/P_0 = \tfrac12\):
    \[
    e^{-\varepsilon/(k_BT)} = \frac12
    \;\;\Longrightarrow\;\;
    \frac{\varepsilon}{k_BT} = \ln 2
    \;\;\Longrightarrow\;\;
    T = \frac{\varepsilon}{k_B\ln 2}.
    \]
    
    \medskip
    \textbf{Insert the energy spacing \(\varepsilon = hf\)}
    
    \[
    \varepsilon = h f
               = (6.626\times10^{-34}\,\mathrm{J\,s})
                 (2.2\times10^{12}\,\mathrm{Hz})
               = 1.458\times10^{-21}\,\mathrm{J}.
    \]
    
    \[
    T
      = \frac{1.458\times10^{-21}\,\mathrm{J}}
             {(1.380\,649\times10^{-23}\,\mathrm{J\,K^{-1}})(\ln 2)}
      = 1.523\times10^{2}\,\mathrm{K}.
    \]
    
    \[
    \boxed{T \;\approx\; 1.5\times10^{2}\ \text{K} \;=\; 152\ \text{K}}
    \]
    \end{solution}
    \begin{solution}
      Throughout we use the same energy spacing 
      \(\varepsilon = h f = 1.458\times10^{-21}\,\mathrm{J}\) and keep
      \(\displaystyle k_B = 1.380\,649\times10^{-23}\,\mathrm{J\,K^{-1}}\).
      
      Let 
      \[
      T_1 = 152.3\ \mathrm{K}
      \]
      be the temperature found in part 1, so that  
      \(P_1/P_0 = \tfrac12\) at \(T_1\).
      
      \vspace{0.5em}
      %--------------------------------------------------------------------
      \subsection*{2)  Temperature reduced to \(\boldsymbol{T_2 = 0.1\,T_1}\)}
      
      \[
      T_2 = 0.10\,T_1 = 15.23\ \mathrm{K}, 
      \qquad
      \beta_2 = \frac1{k_B T_2}.
      \]
      
      The Boltzmann ratio is 
      \[
      \frac{P_1}{P_0} = e^{-\beta_2\varepsilon}.
      \]
      Because \(\beta_2 = 10\,\beta_1\) and \(\beta_1\varepsilon=\ln 2\),
      \[
      \boxed{\;
      \frac{P_1}{P_0}(T_2)
            = e^{-10\ln 2}
            = 2^{-10}
            = \frac1{1024}
            \approx 9.77\times10^{-4}
      \;} .
      \]
      
      \vspace{0.5em}
      %--------------------------------------------------------------------
      \subsection*{3)  Ratio \(\boldsymbol{P_2/P_1}\) at the same \(T_2\)}
      
      \[
      \frac{P_2}{P_1}
        = \frac{e^{-2\beta_2\varepsilon}}{e^{-\beta_2\varepsilon}}
        = e^{-\beta_2\varepsilon}
        = 2^{-10}
        \approx 9.77\times10^{-4}.
      \]
      
      Thus
      \[
      \boxed{\;
      \frac{P_2}{P_1}(T_2) = 2^{-10} \approx 9.77\times10^{-4}
      \;} .
      \]
      
      \vspace{0.5em}
      %--------------------------------------------------------------------
      \subsection*{4)  Mean energy per oscillator, expressed as \(\langle E\rangle /kT\)}
      
      For a quantum harmonic oscillator whose ground–state energy
      is taken as zero, the canonical partition function is
      \[
      Z(\beta)=\sum_{n=0}^{\infty}e^{-n\beta\varepsilon}
              =\frac1{1-e^{-\beta\varepsilon}}.
      \]
      Hence
      \[
      \langle E\rangle
         = -\frac{\partial}{\partial\beta}\ln Z
         = \frac{\varepsilon}{e^{\beta\varepsilon}-1}.
      \]
      Dividing by \(kT=1/\beta\) gives the dimensionless ratio
      \[
      \frac{\langle E\rangle}{kT}
         = \frac{\beta\varepsilon}{e^{\beta\varepsilon}-1}
         \equiv \frac{x}{e^{x}-1},
         \quad x:=\beta\varepsilon.
      \]
      
      At \(T_2\) we have \(x = \beta_2\varepsilon = 10\ln 2 = 6.931\).
      Therefore
      \[
      \boxed{\;
      \frac{\langle E\rangle}{kT_2}
        = \frac{6.931}{e^{6.931}-1}
        = \frac{6.931}{1024-1}
        \approx 6.78\times10^{-3}
      \;} .
      \]
      
      (Numerically, \(\langle E\rangle/kT_2 \approx 0.0068\); far below the
      equipartition value~1 because the temperature is well below the
      quantum energy spacing.)
      \end{solution}
      \begin{solution}
        \textbf{Data}  
        \[
        \varepsilon = 10^{-20}\,\text{J}, 
        \qquad 
        T = 320\ \text{K}, 
        \qquad 
        k_B = 1.380\,649\times10^{-23}\,\text{J\,K}^{-1}.
        \]
        
        The Boltzmann weight for level \(n\) is 
        \[
        P_n \;\propto\; e^{-n\varepsilon/(k_B T)} .
        \]
        
        %-------------------------------------------------------------
        \subsection*{1) Most–probable energy}
        
        Because the weights decrease monotonically with increasing \(n\),
        the \emph{ground state} (\(n=0\)) is always the most probable state
        for a non-degenerate harmonic oscillator in equilibrium.
        
        \[
        \boxed{\; E_{\text{most probable}} = 0\,\varepsilon \;}
        \]
        
        %-------------------------------------------------------------
        \subsection*{2) Ratio of probabilities \(\displaystyle \frac{P(E=2\varepsilon)}{P(E=0)}\)}
        
        \[
        \frac{P(E=2\varepsilon)}{P(E=0)}
           \;=\; \frac{e^{-2\varepsilon/(k_B T)}}{e^{0}}
           \;=\; e^{-2\varepsilon/(k_B T)} .
        \]
        
        Insert the numbers:
        \[
        \frac{2\varepsilon}{k_B T}
           = \frac{2\times10^{-20}\,\text{J}}
                  {(1.380\,649\times10^{-23}\,\text{J\,K}^{-1})(320\,\text{K})}
           = 4.53 .
        \]
        
        Hence
        \[
        \boxed{\;
        \frac{P(E=2\varepsilon)}{P(E=0)}
          = e^{-4.53}
          \approx 1.08\times10^{-2}
        \;}
        \]
        
        (That is, about \(1.1\%\).)
        \end{solution}
        \begin{solution}
          \textbf{Energy levels and degeneracies}
          
          \[
          \begin{array}{c|c|c}
          \text{level} & E_i\ (\text{eV}) & g_i\ (\text{degeneracy})\\ \hline
          \text{ground} & 0   & 1 \\
          \text{first excited} & 0.1 & 2
          \end{array}
          \]
          
          \medskip
          \textbf{Boltzmann factors at \(T = 300\ \text{K}\)}
          
          The Boltzmann constant in electron-volts is  
          \(k_B = 8.617\,333\,262\times10^{-5}\ \text{eV K}^{-1}\), so
          
          \[
          k_B T = (8.617\,333\,262\times10^{-5}\,\text{eV K}^{-1})(300\ \text{K})
                = 2.5852\times10^{-2}\ \text{eV}.
          \]
          
          The dimensionless ratio for the excited level is
          
          \[
          \frac{E_1}{k_B T} = \frac{0.10\ \text{eV}}{0.025852\ \text{eV}} = 3.868,
          \qquad
          e^{-E_1/(k_B T)} = e^{-3.868} = 2.0897\times10^{-2}.
          \]
          
          \medskip
          \textbf{Partition function}
          
          \[
          Z = g_0 e^{-E_0/(k_B T)} + g_1 e^{-E_1/(k_B T)}
            = 1\,(1) + 2\,(2.0897\times10^{-2})
            = 1.0418 .
          \]
          
          \medskip
          \textbf{Probability of occupying \emph{either} of the two excited states}
          
          \[
          P_\text{excited}
            = \frac{g_1 e^{-E_1/(k_B T)}}{Z}
            = \frac{2\,(2.0897\times10^{-2})}{1.0418}
            = 4.012\times10^{-2}.
          \]
          
          \[
          \boxed{\,P_\text{excited}\;\approx\;4.0\%\,}
          \]
          
          Thus, at \(T = 300\ \text{K}\) there is about a \(\mathbf{4\%}\) chance that the quantum
          dot is found in one of the two degenerate first-excited states at
          \(0.1\ \text{eV}\).
          \end{solution}
          \begin{solution}
            We again have two energy levels:
            
            \[
            E_0 = 0\ \text{eV},\; g_0 = 1,
            \qquad
            E_1 = 0.1\ \text{eV},\; g_1 = 2,
            \qquad
            \Delta E = E_1-E_0 = 0.1\ \text{eV}.
            \]
            
            The canonical (Boltzmann) probabilities are  
            
            \[
            P_0 = \frac{g_0\,e^{-E_0/(k_BT)}}{Z},\qquad
            P_1 = \frac{g_1\,e^{-E_1/(k_BT)}}{Z},
            \quad
            Z = g_0+g_1\,e^{-\Delta E/(k_BT)}.
            \]
            
            %------------------------------------------------------------------
            \subsection*{2)  High-temperature limit \(T\to\infty\)}
            
            As \(T\to\infty\), the factor \(e^{-\Delta E/(k_BT)}\to 1\), so  
            
            \[
            P_0(\infty)=\frac{g_0}{g_0+g_1}
                       =\frac{1}{1+2}
                       =\boxed{\dfrac13}\;\;(\text{about }33\%).
            \]
            
            %------------------------------------------------------------------
            \subsection*{3)  Temperature where half the dots are excited}
            
            Set \(P_1 = P_0\):
            
            \[
            \frac{g_1\,e^{-\Delta E/(k_B T_{1/2})}}
                 {g_0} = 1
            \;\;\Longrightarrow\;\;
            e^{-\Delta E/(k_B T_{1/2})} = \frac{g_0}{g_1} = \frac12
            \;\;\Longrightarrow\;\;
            \frac{\Delta E}{k_B T_{1/2}} = \ln 2 .
            \]
            
            Using \(k_B = 8.617\,333\times10^{-5}\,\text{eV\,K}^{-1}\),
            
            \[
            T_{1/2} = \frac{0.1\ \text{eV}}
                           {(8.617\,333\times10^{-5}\,\text{eV\,K}^{-1})\ln 2}
                     = 1.67\times10^{3}\ \text{K}.
            \]
            
            \[
            \boxed{\,T_{1/2}\;\approx\;1.7\times10^{3}\ \text{K}\,}
            \]
            
            %------------------------------------------------------------------
            \subsection*{4)  Entropy as \(T\to 0\)}
            
            At very low \(T\) every dot occupies its unique ground state
            (\(g_0=1\)).  
            For \(N\) independent dots the number of accessible microstates is
            \(\Omega = 1^{N}=1\).  
            The entropy is therefore
            
            \[
            S(T\!\to\!0) = k_B \ln \Omega = k_B\ln 1 = 0,
            \qquad
            \boxed{S = 0\ \text{J\,K}^{-1}}.
            \]
            
            (This is consistent with the third law of thermodynamics since the
            ground state is non-degenerate.)
            \end{solution}
            \begin{solution}
              \textbf{Energy levels and degeneracies considered}
              
              \[
              \begin{array}{c|c|c}
              \text{level} & E_i\ (\text{eV}) & g_i \\ \hline
              \text{1S (ground)} & 0      & 1 \\
              \text{2S}          & 10.2   & 1 \\
              \text{2P}          & 10.2   & 3   % \;(\text{three magnetic sub-levels})
              \end{array}
              \]
              
              (The 2P level is triply degenerate because \(m=-1,0,+1\).  
              Higher states are neglected as instructed.)
              
              \medskip
              \textbf{Boltzmann factor at \(T = 5900\;\text{K}\)}
              
              Boltzmann constant in electron-volts:
              \[
              k_B = 8.617\,333\,262\times10^{-5}\ \text{eV\,K}^{-1}.
              \]
              
              \[
              k_B T = (8.617\,333\,262\times10^{-5}\,\text{eV K}^{-1})(5900\ \text{K})
                     = 5.08\times10^{-1}\ \text{eV}.
              \]
              
              \[
              \frac{E_1}{k_B T} = \frac{10.2\ \text{eV}}{0.508\ \text{eV}} = 20.06,
              \qquad
              e^{-E_1/(k_B T)} = e^{-20.06} = 1.94\times10^{-9}.
              \]
              
              \medskip
              \textbf{Partition function (truncated)}
              
              \[
              Z = g_{1\text{S}}\;e^{-0}
                  + (g_{2\text{S}}+g_{2\text{P}})\,e^{-E_1/(k_B T)}
                = 1 + (1+3)\,(1.94\times10^{-9})
                = 1 + 7.76\times10^{-9}.
              \]
              
              \medskip
              \textbf{Fraction in the 2P states}
              
              \[
              P_{2\text{P}}
                = \frac{g_{2\text{P}}\,e^{-E_1/(k_B T)}}{Z}
                = \frac{3\,(1.94\times10^{-9})}{1 + 7.76\times10^{-9}}
                = 5.81\times10^{-9}.
              \]
              
              \[
              \boxed{\;
              \text{Fraction of H atoms in 2P at }T=5900\ \text{K}
                    \;\approx\; 6\times10^{-9}
              \;}
              \]
              
              So only about six parts in a billion of the hydrogen atoms populate a
              2P state at the Sun’s surface temperature.
              \end{solution}
              \begin{solution}
                Keep the same level scheme and degeneracies as before,
                
                \[
                \begin{array}{c|c|c}
                \text{level} & E_i\ (\text{eV}) & g_i \\ \hline
                \text{1S (ground)} & 0      & 1 \\
                \text{2S}          & 10.2   & 1 \\
                \text{2P}          & 10.2   & 3
                \end{array}
                \]
                
                and use the recommended Boltzmann constant  
                \(k_B = 8.617\,333\,262\times10^{-5}\ \text{eV\,K}^{-1}\).
                
                %------------------------------------------------------------------
                \subsection*{Sunspot temperature \(T = 4300\ \text{K}\)}
                
                \[
                k_B T = (8.617\,333\times10^{-5}\,\text{eV K}^{-1})(4300\ \text{K})
                       = 0.3705\ \text{eV}.
                \]
                
                \[
                \frac{E_1}{k_B T} = \frac{10.2\ \text{eV}}{0.3705\ \text{eV}} = 27.55,
                \qquad
                e^{-E_1/(k_B T)} = e^{-27.55} = 1.11\times10^{-12}.
                \]
                
                \medskip
                \textbf{Partition function (truncated)}
                
                \[
                Z = g_{1\text{S}}
                    + (g_{2\text{S}}+g_{2\text{P}})\,e^{-E_1/(k_B T)}
                  = 1 + (1+3)(1.11\times10^{-12})
                  = 1 + 4.44\times10^{-12}.
                \]
                
                \medskip
                \textbf{Fraction in the 2P manifold}
                
                \[
                P_{2\text{P}}
                  = \frac{g_{2\text{P}}\,e^{-E_1/(k_B T)}}{Z}
                  = \frac{3\,(1.11\times10^{-12})}{1 + 4.44\times10^{-12}}
                  = 3.33\times10^{-12}.
                \]
                
                \[
                \boxed{\;
                \text{Fraction of H atoms in 2P at }T = 4300\ \text{K}
                      \;\approx\; 3\times10^{-12}
                \;}
                \]
                
                So at a typical sunspot temperature only a few parts in a trillion of
                the hydrogen atoms populate the 2P states.
                \end{solution}
                \begin{solution}
                  A simple intrinsic semiconductor obeys  
                  
                  \[
                  R(T)\;\propto\; \exp\!\Bigl(\frac{E_g}{2k_B T}\Bigr),
                  \]
                  
                  because the conductivity varies as 
                  \(\sigma\propto\exp\!\bigl(-E_g/2k_B T\bigr)\) and  
                  \(R\propto 1/\sigma\).
                  
                  \vspace{0.4em}
                  \textbf{Form the resistance ratio}
                  
                  \[
                  \frac{R_1}{R_2}
                     = \exp\!\Bigl(\frac{E_g}{2k_B}\Bigr)
                       \exp\!\Bigl(\frac{1}{T_1}-\frac{1}{T_2}\Bigr),
                  \qquad
                  R_1 = 0.010\;\Omega,\;
                  R_2 = 2.591\times10^{-4}\;\Omega.
                  \]
                  
                  \[
                  \ln\!\Bigl(\tfrac{R_1}{R_2}\Bigr)
                    = \frac{E_g}{2k_B}\Bigl(\tfrac{1}{T_1}-\tfrac{1}{T_2}\Bigr).
                  \]
                  
                  \vspace{0.4em}
                  \textbf{Insert the numbers}
                  
                  \[
                  \frac{R_1}{R_2} = \frac{0.010}{2.591\times10^{-4}} = 38.6,
                  \qquad
                  \ln(38.6)=3.654.
                  \]
                  
                  \[
                  \frac{1}{T_1}-\frac{1}{T_2}
                     = \frac{1}{270\;\text{K}}-\frac{1}{300\;\text{K}}
                     = 3.7037\times10^{-4}\;\text{K}^{-1}.
                  \]
                  
                  \[
                  E_g
                    = 2k_B\,\frac{\ln(R_1/R_2)}{1/T_1-1/T_2}
                    = 2(8.617\times10^{-5}\,\text{eV K}^{-1})
                      \frac{3.654}{3.7037\times10^{-4}}
                    = 1.7\;\text{eV}\;(\text{to one decimal place}).
                  \]
                  
                  \[
                  \boxed{E_g \;\approx\; 1.7\ \text{eV}}
                  \]
                  \end{solution}
                  \begin{solution}
                    For an intrinsic semiconductor the resistivity (and hence the resistance
                    of a fixed‑geometry sample) follows
                    
                    \[
                    R(T)\;=\; R_0\,\exp\!\Bigl(\frac{E_g}{2k_B T}\Bigr),
                    \]
                    
                    so the \emph{ratio} of resistances at two temperatures is
                    
                    \[
                    \frac{R_2}{R_1}
                      = \exp\!\Bigl[\frac{E_g}{2k_B}\Bigl(\frac{1}{T_2}-\frac{1}{T_1}\Bigr)\Bigr].
                    \]
                    
                    \medskip
                    \textbf{Insert the data}
                    
                    \[
                    \begin{aligned}
                    E_g &= 1.1\ \text{eV}
                          = (1.1)(1.602\,176\,634\times10^{-19}\,\text{J})
                          = 1.762\,\times10^{-19}\ \text{J}, \\[6pt]
                    k_B &= 1.380\,649\times10^{-23}\ \text{J\,K}^{-1}, \\[6pt]
                    T_1 &= 300\ \text{K}, \qquad R_1 = 2\ \Omega, \\[6pt]
                    T_2 &= 77\ \text{K}.
                    \end{aligned}
                    \]
                    
                    \[
                    \frac{E_g}{2k_B} = 
                    \frac{1.762\times10^{-19}}{2(1.380\,649\times10^{-23})}
                    = 6.382\times10^{3}\ \text{K}.
                    \]
                    
                    \[
                    \frac{1}{T_2}-\frac{1}{T_1}
                      = \frac{1}{77\ \text{K}}-\frac{1}{300\ \text{K}}
                      = 9.654\times10^{-3}\ \text{K}^{-1}.
                    \]
                    
                    \[
                    \ln\!\Bigl(\tfrac{R_2}{R_1}\Bigr)
                      = (6.382\times10^{3}\ \text{K})(9.654\times10^{-3}\ \text{K}^{-1})
                      = 61.6
                    \;\;\Longrightarrow\;\;
                    \frac{R_2}{R_1}=e^{61.6}=5.74\times10^{26}.
                    \]
                    
                    \[
                    R_2 = R_1\,\frac{R_2}{R_1}
                         = (2\ \Omega)(5.74\times10^{26})
                         = 1.15\times10^{27}\ \Omega.
                    \]
                    
                    \[
                    \boxed{\;R(77\ \text{K})\;\approx\;1\times10^{27}\ \text{Ohms}\;}
                    \]
                    
                    (The resistance skyrockets because the thermal excitation
                    across the \(1.1\ \text{eV}\) silicon bandgap becomes
                    vanishingly small at liquid‑nitrogen temperature.)
                    \end{solution}
\end{document}
