\documentclass[12pt]{article}

% Packages
\usepackage[margin=1in]{geometry}
\usepackage{amsmath,amssymb,amsthm}
\usepackage{enumitem}
\usepackage{hyperref}
\usepackage{xcolor}
\usepackage{import}
\usepackage{xifthen}
\usepackage{pdfpages}
\usepackage{transparent}
\usepackage{listings}
\usepackage{tikz}
\usepackage{physics}
\usepackage{siunitx}

  \usetikzlibrary{calc,patterns,arrows.meta,decorations.markings}


\DeclareMathOperator{\Log}{Log}
\DeclareMathOperator{\Arg}{Arg}

\lstset{
    breaklines=true,         % Enable line wrapping
    breakatwhitespace=false, % Wrap lines even if there's no whitespace
    basicstyle=\ttfamily,    % Use monospaced font
    frame=single,            % Add a frame around the code
    columns=fullflexible,    % Better handling of variable-width fonts
}

\newcommand{\incfig}[1]{%
    \def\svgwidth{\columnwidth}
    \import{./Figures/}{#1.pdf_tex}
}
\theoremstyle{definition} % This style uses normal (non-italicized) text
\newtheorem{solution}{Solution}
\newtheorem{proposition}{Proposition}
\newtheorem{problem}{Problem}
\newtheorem{lemma}{Lemma}
\newtheorem{theorem}{Theorem}
\newtheorem{remark}{Remark}
\newtheorem{note}{Note}
\newtheorem{definition}{Definition}
\newtheorem{example}{Example}
\newtheorem{corollary}{Corollary}
\theoremstyle{plain} % Restore the default style for other theorem environments
%

% Theorem-like environments
% Title information
\title{}
\author{Jerich Lee}
\date{\today}

\begin{document}

\maketitle
\[
  Q = m\,L_f
     = (1000\;\text{g})\,(333.5\;\text{J g}^{-1})
     = 3.335 \times 10^{5}\ \text{J}.
\]

\[
  \boxed{Q \approx 3.34 \times 10^{5}\ \text{J}}
\]

\subsection*{Entropy changes for 1 kg of water/ice}

\bigskip
\begin{enumerate}
%-----------------------------------------------------------------
\item \textbf{Melting at $0^{\circ}$C}

Latent heat \(L_f = 333.5\;\text{J g}^{-1}\).

\[
Q_{\text{melt}} = (1000\;\text{g})(333.5\;\text{J g}^{-1}) = 3.335\times10^{5}\,\text{J}.
\]

Isothermal at \(T = 273.15\;\text{K}\):

\[
\Delta S_{\text{melt}} = \frac{Q_{\text{melt}}}{T}
                       = \frac{3.335\times10^{5}}{273.15}
                       \approx \boxed{1.22\times10^{3}\;\text{J K}^{-1}}.
\]

%-----------------------------------------------------------------
\item \textbf{Heating liquid water from $0^{\circ}$C to $100^{\circ}$C}

Specific heat (nearly constant) \(c = 4184\;\text{J kg}^{-1}\text{K}^{-1}\).

\[
\Delta S_{\text{heat}}
   = \int_{T_1}^{T_2} \frac{mc}{T}\,dT
   = mc\,\ln\!\bigl(T_2/T_1\bigr)
   = (4184)\,\ln\!\bigl(373.15/273.15\bigr)
   \approx \boxed{1.31\times10^{3}\;\text{J K}^{-1}}.
\]

%-----------------------------------------------------------------
\item \textbf{Boiling at $100^{\circ}$C}

Latent heat \(L_v = 2257\;\text{J g}^{-1}\).

\[
Q_{\text{boil}} = (1000\;\text{g})(2257\;\text{J g}^{-1}) = 2.257\times10^{6}\,\text{J}.
\]

Isothermal at \(T = 373.15\;\text{K}\):

\[
\Delta S_{\text{boil}} = \frac{Q_{\text{boil}}}{T}
                       = \frac{2.257\times10^{6}}{373.15}
                       \approx \boxed{6.05\times10^{3}\;\text{J K}^{-1}}.
\]
\end{enumerate}
\subsection*{Entropy change when \(1\;\text{kg}\) of liquid water is heated
from \(0^{\circ}\text{C}\) to \(100^{\circ}\text{C}\)}

\[
  c_p \simeq 4184\;\frac{\text{J}}{\text{kg}\,\text{K}}
  \quad(\text{assumed constant over this interval}).
\]

With no volume change, \(dU = m\,c_p\,dT\) and

\[
  \Delta S
  = \int_{T_1}^{T_2} \frac{1}{T}\,dU
  = m\,c_p \int_{T_1}^{T_2} \frac{dT}{T}
  = m\,c_p \ln\!\left(\frac{T_2}{T_1}\right).
\]

\[
  T_1 = 0^{\circ}\text{C} = 273.15\;\text{K},\qquad
  T_2 = 100^{\circ}\text{C} = 373.15\;\text{K},\qquad
  m = 1.00\;\text{kg}.
\]

\[
  \Delta S
  = (1.00)(4184)\,
    \ln\!\left(\frac{373.15}{273.15}\right)
  \approx 4184 \times 0.312
  \approx \boxed{1.3 \times 10^{3}\ \text{J K}^{-1}}.
\]

So warming the kilogram of liquid water from the melting point to the
boiling point increases its entropy by roughly
\(1.3\;\text{kJ\,K}^{-1}\).
\bigskip
\noindent
\textbf{Summary}  
\[
\Delta S_{\text{melt}} \approx 1.22\times10^{3}\;\text{J K}^{-1},\quad
\Delta S_{\text{heat}} \approx 1.31\times10^{3}\;\text{J K}^{-1},\quad
\Delta S_{\text{boil}} \approx 6.05\times10^{3}\;\text{J K}^{-1}.
\]
\subsection*{Adding \(10^{3}\,\text{J}\) of heat to \(10\;\text{kg}\) of water at its boiling point}

\[
  T = 373.15\;\text{K},\qquad
  p = 1.01325\times10^{5}\;\text{Pa},\qquad
  L_v = 2230\;\text{J g}^{-1},\qquad
  Q = 1.00\times10^{3}\;\text{J}.
\]

\begin{enumerate}
%-----------------------------------------------------------------
\item \textbf{Mass that vaporises}
      \[
        m = \frac{Q}{L_v}
            = \frac{1.00\times10^{3}}{2230}
            = 4.48\times10^{-1}\;\text{g}.
      \]

%-----------------------------------------------------------------
\item \textbf{Change in Gibbs free energy \((\Delta G)\)}

      At the boiling point the two phases are in equilibrium,  
      \(\mu_{\text{liq}} = \mu_{\text{vap}}\); hence for a reversible transfer  
      \(dG = \mu_{\text{vap}}\,dN - \mu_{\text{liq}}\,dN = 0\).

      \[
        \boxed{\;\Delta G = 0\ \text{J}\;}
      \]

%-----------------------------------------------------------------
\item \textbf{Change in entropy \((\Delta S)\)}

      \[
        \Delta S = \frac{Q}{T}
                 = \frac{1.00\times10^{3}}{373.15}
                 \approx \boxed{2.68\ \text{J K}^{-1}}.
      \]

%-----------------------------------------------------------------
\item \textbf{Change in enthalpy \((\Delta H)\)}

      At constant pressure: \(\displaystyle \Delta H = Q\).
      \[
        \boxed{\;\Delta H = +1.00\times10^{3}\ \text{J}\;}
      \]

%-----------------------------------------------------------------
\item \textbf{Change in volume \((\Delta V)\)}

      Mole number of vapour produced  
      \(n = m/M = 0.448\ \text{g}/18\ \text{g mol}^{-1}=2.49\times10^{-2}\ \text{mol}\).

      \[
        \Delta V
        = \frac{nRT}{p}
        = \frac{(2.49\times10^{-2})(8.314)(373.15)}
               {1.01325\times10^{5}}
        \approx \boxed{7.6\times10^{-4}\ \text{m}^{3}}.
      \]

%-----------------------------------------------------------------
\item \textbf{Change in internal energy \((\Delta U)\)}

      \[
        \Delta U = \Delta H - p\,\Delta V
                 = 1.00\times10^{3}
                   - (1.01325\times10^{5})(7.6\times10^{-4})
                 \approx \boxed{9.23\times10^{2}\ \text{J}}.
      \]
\end{enumerate}
\subsection*{Lowering the boiling point by \(1^{\circ}\text{C}\)}

\textbf{Given}\\
Atmospheric pressure \(P = 1.013\times10^{5}\,\text{Pa}\),  
boiling temperature \(T = 373.15\,\text{K}\),  
latent heat of vaporisation \(L_B = 2.256\times10^{6}\,\text{J kg}^{-1}\),  
molar mass \(M = 0.018\,\text{kg mol}^{-1}\).

\medskip
\textbf{1.  Clausius–Clapeyron in differential form}

For small changes,
\[
  \frac{dP}{dT} \;=\; \frac{L}{T\,\Delta v}.
\]
Taking the liquid volume per mole to be negligible,  
\(\Delta v \approx v_{\text{vap}} = \dfrac{RT}{P}\).  
Hence
\[
  \frac{dP}{dT}
  = \frac{L}{T}\,\frac{P}{RT}
  = \frac{P\,L}{R\,T^{2}}.
\]

\medskip
\textbf{2.  Convert latent heat to a molar basis}

\[
  L = L_B\,M
    = (2.256\times10^{6})(0.018)
    \approx 4.06\times10^{4}\,\text{J mol}^{-1}.
\]

\medskip
\textbf{3.  Pressure change for \(\Delta T = -1\,\text{K}\)}

\[
  \Delta P
  = \frac{P\,L}{R\,T^{2}}\;\Delta T
  = \frac{(1.013\times10^{5})(4.06\times10^{4})}
         {(8.314)(373.15)^{2}}\;(-1)
  \approx -\,3.6\times10^{3}\,\text{Pa}.
\]

\[
  \boxed{\;\Delta P \approx -3.6\ \text{kPa}\;}
\]

\noindent
So the pressure must be reduced by roughly \(3.6\;\text{kPa}\) (about \(3.5\%\) of an atmosphere) to lower water’s boiling point by \(1^{\circ}\text{C}\) near standard conditions.
\subsection*{Phase–transition data (per atom)}

\[
  \Delta h = 8.0\times 10^{-24}\;\text{J atom}^{-1},\qquad
  \Delta s = 8.0\times 10^{-24}\;\text{J K}^{-1}\!\text{ atom}^{-1}.
\]

Here “$\Delta$” denotes (liquid – solid).

%--------------------------------------------------------------------
\paragraph{1.  Latent heat per mole}

\[
  L_{\text{m}} 
  = \bigl(\Delta h\bigr)\,N_{\!A}
  = (8.0\times10^{-24})\,
    (6.022\,140\,76\times10^{23})
  \;\text{J mol}^{-1}
  \approx \boxed{4.8\;\text{J mol}^{-1}}.
\]

%--------------------------------------------------------------------
\paragraph{2.  Melting temperature}

At the equilibrium (melting) point, Gibbs free energies of the two phases
are equal, so
\[
  \Delta G = \Delta h - T_{\text{m}}\Delta s = 0
  \;\;\Longrightarrow\;\;
  T_{\text{m}} = \frac{\Delta h}{\Delta s}.
\]

Because the given numbers are identical,
\[
  T_{\text{m}} 
  = \frac{8.0\times10^{-24}}{8.0\times10^{-24}}
  = \boxed{1.0\;\text{K}}.
\]
% ---------------------------------------------------------------------------
%  Equilibrium solubility of water in oil: temperature shift
% ---------------------------------------------------------------------------
\paragraph{Data}
\[
c_1 = 9.0\times10^{-4}\;
      \Bigl(\text{water molecules per oil molecule at }T_1=283\;\text{K}\Bigr),
\qquad
\Delta E = 2.208\times10^{-20}\,\text{J},
\qquad
k_{\mathrm B}=1.380\,649\times10^{-23}\,\text{J\,K}^{-1},
\qquad
T_2 = 293\;\text{K}.
\]

\paragraph{Boltzmann expression for a very dilute ideal solute}
\[
c(T)=c_0\exp\!\Bigl(-\tfrac{\Delta E}{k_{\mathrm B}T}\Bigr).
\]

\paragraph{Ratio of concentrations at two temperatures}
\[
\frac{c_2}{c_1}
  =\exp\!\Bigl[-\frac{\Delta E}{k_{\mathrm B}}
                \Bigl(\frac{1}{T_2}-\frac{1}{T_1}\Bigr)\Bigr].
\]

\paragraph{Numerical evaluation}
\[
\begin{aligned}
\frac{1}{T_2}-\frac{1}{T_1} &=\frac{1}{293}-\frac{1}{283}
                              =-1.208\times10^{-4}\ \text{K}^{-1},\\[6pt]
\frac{\Delta E}{k_{\mathrm B}}
                            &=\frac{2.208\times10^{-20}}
                                   {1.380\,649\times10^{-23}}
                              =1.599\times10^{3}\ \text{K},\\[6pt]
\frac{c_2}{c_1}
  &=\exp\!\bigl[(1.599\times10^{3})(-1.208\times10^{-4})\bigr]
   =\exp(-0.193)=1.213.
\end{aligned}
\]

\paragraph{Concentration at \(T_2=293\;\text{K}\)}
\[
\boxed{%
  c_2 = c_1\,\frac{c_2}{c_1}
       =(9.0\times10^{-4})(1.213)
       \approx 1.09\times10^{-3}\;
       \text{water molecules per oil molecule}
}
\]
%%%%%%%%%%%%%%%%%%%%%%%%%%%%%%%%%%%%%%%%%%%%%%%%%%%%%%%%%%%%%%%%%%%%%%%%
\paragraph{Model}  At these very low concentrations the water in oil
behaves as an \emph{ideal solute}, so its equilibrium mole fraction (or
concentration) obeys

\[
c \;\propto\; \exp\!\Bigl(-\tfrac{\Delta E}{k_{\mathrm B}T}\Bigr),
\]

where \(\Delta E\) is the molecular energy cost of transferring one
water molecule from the liquid phase into the oil.

Let the two given solubilities be
\[
  c_1 = 0.90\;\text{g L}^{-1}\;(T_1=283\;\text{K}),\qquad
  c_2 = 1.092\;\text{g L}^{-1}\;(T_2=293\;\text{K}).
\]

=======================================================================
\subsection*{1. Energy cost per molecule}

\[
  \ln\!\frac{c_2}{c_1}
  \;=\;
  -\frac{\Delta E}{k_{\mathrm B}}
  \bigl(\tfrac{1}{T_2}-\tfrac{1}{T_1}\bigr)
  \quad\Longrightarrow\quad
  \boxed{%
    \Delta E
    = -\,\frac{k_{\mathrm B}\ln(c_2/c_1)}{\,1/T_2 - 1/T_1\,}
    \approx 2.2\times10^{-20}\ \text{J\;molecule}^{-1}}.
\]

=======================================================================
\subsection*{2. Solubility at \(T_3 = 321\;\text{K}\)}
% --------------------------------------------------------------------
%  Solubility at \(T_3 = 321\;\text{K}\)
% --------------------------------------------------------------------

The dilute-solute relation  
\[
  c(T)=c_0\exp\!\Bigl(-\tfrac{\Delta E}{k_{\mathrm B}T}\Bigr)
\]
gives, for two temperatures \(T_1\) and \(T_3\),

\[
  c_3
  = c_1
    \exp\!\Bigl[
      -\frac{\Delta E}{k_{\mathrm B}}
      \Bigl(\tfrac{1}{T_3}-\tfrac{1}{T_1}\Bigr)
    \Bigr].
\]

Numbers  
\(\;c_1 = 0.900\;\text{g L}^{-1},\;
  T_1 = 283\;\text{K},\;
  T_3 = 321\;\text{K},\;
  \Delta E = 2.208\times10^{-20}\;\text{J}\).

\[
  \frac{\Delta E}{k_{\mathrm B}}
    =\frac{2.208\times10^{-20}}{1.380\,649\times10^{-23}}
    = 1.599\times10^{3}\ \text{K},
\qquad
  \bigl(\tfrac{1}{T_3}-\tfrac{1}{T_1}\bigr)
    = -1.330\times10^{-4}\ \text{K}^{-1}.
\]

\[
  c_3
  = 0.900\,
    \exp\!\bigl[(1.599\times10^{3})(-1.330\times10^{-4})\bigr]
  = 0.900 \times 1.952
  = \boxed{1.76\;\text{g L}^{-1}}.
\]

=======================================================================
\subsection*{3. What happens if equilibrated oil at \(80^{\circ}\text{C}\)
is cooled to \(5^{\circ}\text{C}\)?}

Because the solubility drops as temperature decreases
(\(c \propto e^{-\Delta E/kT}\)), the oil will temporarily contain more
water than its new equilibrium value.  The excess water therefore
\textbf{condenses out of the oil}.  (First option.)
\[
  \text{Degeneracies and energies:}\quad
  \begin{cases}
    g_A = 1, & E_A = E_A,\\
    g_B = 2, & E_B = 3E_A,\\
    g_C = 3, & E_C = 5E_A.
  \end{cases}
\]

Given \(k_{\mathrm B}T = E_A\) (\( \Rightarrow E_A/k_{\mathrm B}T = 1\)).

\[
\begin{aligned}
\text{Boltzmann weights}&:\;
w_A = g_A e^{-E_A/kT}=e^{-1},\\
w_B &= g_B e^{-E_B/kT}=2e^{-3},\\
w_C &= g_C e^{-E_C/kT}=3e^{-5}.
\end{aligned}
\]

Partition sum  
\[
  Z = w_A + w_B + w_C
    = e^{-1}+2e^{-3}+3e^{-5}\approx0.488.
\]

Probability that the molecule has energy \(E_B\):
\[
  P(E_B)
  = \frac{w_B}{Z}
  = \frac{2e^{-3}}{e^{-1}+2e^{-3}+3e^{-5}}
  \approx \boxed{0.204}.
\]
\paragraph{Isothermal barometric relation}
For an ideal gas in hydrostatic equilibrium at constant temperature \(T\)

\[
  \frac{p(z)}{p(0)} \;=\;
  \exp\!\Bigl(-\frac{m g z}{k_{\mathrm B}T}\Bigr),
\]

where \(m\) is the particle mass and \(z\) the height above the reference
level.

\paragraph{Insert the data}
\[
  m = 3.0\times10^{-20}\,\text{kg},\qquad
  g = 9.81\ \text{m s}^{-2},\qquad
  z = h = 5.0\times10^{3}\,\text{m},\qquad
  T = 300\ \text{K}.
\]

\[
  \frac{m g h}{k_{\mathrm B}T}
  = \frac{(3.0\times10^{-20})(9.81)(5.0\times10^{3})}
         {(1.380\,649\times10^{-23})(300)}
  \approx 3.6\times10^{5}.
\]

\[
  \boxed{\frac{p_{\text{top}}}{p_{\text{bottom}}}
         = \exp(-3.6\times10^{5})
         \;\approx\; 10^{-1.6\times10^{5}}\;\text{(essentially zero).}}
\]
\paragraph{Pressure ratio in a \(5.0\times10^{3}\,\text{m}\) vertical tube
(isothermal, \(T = 300\;\text{K}\))}

For an ideal gas in hydrostatic equilibrium at constant \(T\)

\[
  \frac{p(h)}{p(0)}
  = \exp\!\Bigl(-\frac{m\,g\,h}{k_{\mathrm B}T}\Bigr),
\]

where  

* \(m = 3.0\times10^{-26}\ \text{kg}\) (mass per molecule)  
* \(g = 9.81\ \text{m s}^{-2}\)  
* \(h = 5.0\times10^{3}\ \text{m}\)  
* \(k_{\mathrm B}=1.380\,649\times10^{-23}\ \text{J K}^{-1}\)  
* \(T = 300\ \text{K}\).

\[
  \frac{mgh}{k_{\mathrm B}T}
  =\frac{(3.0\times10^{-26})(9.81)(5.0\times10^{3})}
        {(1.380\,649\times10^{-23})(300)}
  = 0.355.
\]

\[
  \boxed{\frac{p_{\text{top}}}{p_{\text{bottom}}}
        = e^{-0.355}\;\approx\;0.70 }.
\]

Thus the pressure at the top of the 5-km column is about **70 %** of the
pressure at the bottom.

\textbf{Interpretation:} with such massive molecules, the pressure at the
top of a \(5\;\text{km}\) column is negligibly small compared with the
pressure at the bottom.
\subsection*{Boltzmann statistics for electron spins}

\[
  \mu = 9.3\times10^{-24}\,\text{J T}^{-1},
  \qquad
  k_{\mathrm B}=1.380\,649\times10^{-23}\,\text{J K}^{-1}.
\]

--------------------------------------------------------------------
\paragraph{1.  Population ratio for 63  up}

\[
  \frac{N_\uparrow}{N_\downarrow}
  = \frac{0.63}{0.37}
  \approx \boxed{1.70}.
\]

--------------------------------------------------------------------
\paragraph{2.  Energy gap that yields this ratio at \(T = 24^{\circ}\text{C}=297\;\text{K}\)}

Boltzmann relation  
\(N_\uparrow/N_\downarrow = \exp\bigl(-\Delta E/kT\bigr)\) with  
\(\Delta E = E_\downarrow-E_\uparrow\) \(= 2\mu B\).

\[
  \Delta E
  = kT\,\ln\!\bigl(N_\uparrow/N_\downarrow\bigr)
  = (1.3806\times10^{-23})(297)\,\ln(1.70)
  \approx \boxed{2.18\times10^{-21}\ \text{J}}.
\]

--------------------------------------------------------------------
\paragraph{3.  Magnetic field that produces this splitting}

\[
  B = \frac{\Delta E}{2\mu}
    = \frac{2.18\times10^{-21}}{2(9.3\times10^{-24})}
    \approx \boxed{1.17\times10^{2}\ \text{T}}.
\]

\bigskip
So a field of roughly \(1.2\times10^{2}\,\text{tesla}\) would bias the spins so that 63 % point up at room temperature.

\subsection*{Two-state systems: \(N\) molecules, levels \(0\) and \(\varepsilon\)}

-------------------------------------------------------------------
\paragraph{1.  Partition function and internal energy}

\[
  Z = 1 + e^{-\varepsilon/kT},
  \qquad
  U(T) = N\,\varepsilon\,\frac{e^{-\varepsilon/kT}}{Z}
        = \frac{N\varepsilon}{e^{\varepsilon/kT}+1}.
\]

-------------------------------------------------------------------
\paragraph{2.  Heat capacity \(C(T)=\dv{U}{T}\)}

\[
  C(T)
  = N\,\frac{\varepsilon^{2}}{kT^{2}}\;
    \frac{e^{\varepsilon/kT}}
         {\bigl(1+e^{\varepsilon/kT}\bigr)^{2}}.
\]

-------------------------------------------------------------------
\paragraph{Numerical constants}

\[
  N = 6.022\times10^{23},\qquad
  \varepsilon = 2.07\times10^{-21}\,\text{J},\qquad
  k = 1.380649\times10^{-23}\,\text{J K}^{-1}.
\]

-------------------------------------------------------------------
\paragraph{3.  Results}

| \(T\) (K) | \(U(T)\) (J) | \(C(T)\) (J K\(^{-1}\)) |
|-----------|--------------|-------------------------|
| 200 | \(4.00\times10^{2}\) | \(1.02\) |
| 250 | \(4.42\times10^{2}\) | \(0.684\) |
|  20 | \(6.92\times10^{-1}\) | \(0.259\) |

-------------------------------------------------------------------
\textbf{Answers required by the form}

1. \(U(200\,\text{K}) \approx 4.0\times10^{2}\ \text{J}\)

2. \(C(200\,\text{K}) \approx 1.0\ \text{J K}^{-1}\)

3. \(C(250\,\text{K}) \approx 0.68\ \text{J K}^{-1}\)

4. \(C(20\,\text{K})  \approx 0.26\ \text{J K}^{-1}\)
\end{document}
