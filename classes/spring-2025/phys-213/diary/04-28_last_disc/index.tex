\documentclass[12pt]{article}

% Packages
\usepackage[margin=1in]{geometry}
\usepackage{amsmath,amssymb,amsthm}
\usepackage{enumitem}
\usepackage{hyperref}
\usepackage{xcolor}
\usepackage{import}
\usepackage{xifthen}
\usepackage{pdfpages}
\usepackage{transparent}
\usepackage{listings}
\usepackage{tikz}
\usepackage{physics}
\usepackage{siunitx}
  \usetikzlibrary{calc,patterns,arrows.meta,decorations.markings}


\DeclareMathOperator{\Log}{Log}
\DeclareMathOperator{\Arg}{Arg}

\lstset{
    breaklines=true,         % Enable line wrapping
    breakatwhitespace=false, % Wrap lines even if there's no whitespace
    basicstyle=\ttfamily,    % Use monospaced font
    frame=single,            % Add a frame around the code
    columns=fullflexible,    % Better handling of variable-width fonts
}

\newcommand{\incfig}[1]{%
    \def\svgwidth{\columnwidth}
    \import{./Figures/}{#1.pdf_tex}
}
\theoremstyle{definition} % This style uses normal (non-italicized) text
\newtheorem{solution}{Solution}
\newtheorem{proposition}{Proposition}
\newtheorem{problem}{Problem}
\newtheorem{lemma}{Lemma}
\newtheorem{theorem}{Theorem}
\newtheorem{remark}{Remark}
\newtheorem{note}{Note}
\newtheorem{definition}{Definition}
\newtheorem{example}{Example}
\newtheorem{corollary}{Corollary}
\theoremstyle{plain} % Restore the default style for other theorem environments
%

% Theorem-like environments
% Title information
\title{}
\author{Jerich Lee}
\date{\today}

\begin{document}

\maketitle
Below is a fully-formatted copy-and-paste LaTeX solution that walks through the problem step-by-step.
(Everything between the two triple-backtick blocks is ready to drop into your document.)

\section*{Solution: Three–state system in an isolated bath}

We work in the \emph{microcanonical} ensemble because the internal energy \(U\) is fixed.  
Throughout, \(k_B\) denotes Boltzmann’s constant.

\subsection*{Step 1:  Count the accessible microstates \(\Omega(U)\)}

\begin{itemize}
  \item \textbf{Energy \(U = 0\).}  
        Only the single ground state (energy \(0\)) is accessible:  
        \[
          \Omega(0)=1 .
        \]
        
  \item \textbf{Energy \(U = \Delta\).}  
        Two degenerate excited states (each at energy \(\Delta\)) are accessible:  
        \[
          \Omega(\Delta)=2 .
        \]
\end{itemize}

\subsection*{Step 2:  Compute the entropy \(S(U)=k_B \ln \Omega(U)\)}

\begin{align}
  S(0)      &= k_B \ln\bigl[\Omega(0)\bigr]      = k_B \ln 1 = 0 ,\\[6pt]
  S(\Delta) &= k_B \ln\bigl[\Omega(\Delta)\bigr] = k_B \ln 2 .
\end{align}

\subsection*{Step 3:  Estimate the temperature}

For a system with discrete energy levels we approximate the derivative  
\(\displaystyle \left(\frac{\partial S}{\partial U}\right)_{V,N}\) by a finite difference:
\[
  \frac{1}{T}
  \;\simeq\;
  \frac{S(\Delta)-S(0)}{\Delta}
  \;=\;
  \frac{k_B\ln 2 - 0}{\Delta}
  \;=\;
  \frac{k_B\ln 2}{\Delta}.
\]

\[
  \boxed{\, T \;=\; \dfrac{\Delta}{k_B \ln 2}\, }.
\]

\subsection*{Summary}

\[
  S(0)=0, 
  \qquad 
  S(\Delta)=k_B\ln 2,
  \qquad
  T=\frac{\Delta}{k_B\ln 2}.
\]

The entropy increases by \(k_B\ln 2\) when the system goes from the ground state (\(U=0\)) to one of the two degenerate excited states (\(U=\Delta\)), and the corresponding temperature is positive and finite.
\section*{Qualitative discussion of energy exchange between System \(A\) and System \(B\)}

\subsection*{1.  Statement of the set-up}

\begin{itemize}
  \item \textbf{System \(A\)} consists of \(n_A = 100\) identical three–level subsystems, each with
        energies \(\{0,\;\Delta_1,\;\Delta_1\}\).
        We denote by \(M_A\) the number of subsystems in \(A\) that are in an excited state
        (energy \(\Delta_1\)).  Initially, \(\boxed{M_A^{(0)} = 100}\), so
        \[
            U_A^{(0)} \;=\; M_A^{(0)} \,\Delta_1 = 100\,\Delta_1 .
        \]

  \item \textbf{System \(B\)} consists of \(n_B = 200\) identical three–level subsystems, each with
        energies \(\{0,\;\Delta_2,\;\Delta_2\}\) where \(\displaystyle \Delta_2 = \tfrac12\Delta_1\).
        Denote by \(M_B\) the number of excited subsystems in \(B\).
        Initially, \(\boxed{M_B^{(0)} = 0}\), so
        \[
            U_B^{(0)} \;=\; M_B^{(0)} \,\Delta_2 = 0 .
        \]

  \item The two systems are brought into weak thermal contact.
        Total energy is conserved:
        \[
            U_{\text{tot}}
            \;=\;
            U_A + U_B
            \;=\;
            100\,\Delta_1
            ,
            \qquad
            \text{(always)} .
        \]
\end{itemize}

\subsection*{2.  Entropy and temperature of a collection of three–level subsystems}

For a system with \(n\) copies and \(M\) excitations (\(0\le M\le n\)) the number of
micro\-states is\footnote{%
Each excited copy has two degenerate states, giving the factor \(2^{M}\);
the binomial factor counts which \(M\) copies are excited.}
\[
    \Omega(n,M)
    \;=\;
    \binom{n}{M}\,2^{M}.
\]
Using Stirling’s approximation for large \(n\), the entropy becomes
\[
  S(n,M)
  \;=\;
  k_B\ln\Omega
  \;\approx\;
  k_B
  \!\left[
      -M\ln\!\frac{M}{n}
      -(n-M)\ln\!\frac{n-M}{n}
      +M\ln 2
  \right].
\]

The microcanonical temperature follows from
\(
  \displaystyle
  \frac{1}{T}
  =
  \frac{\partial S}{\partial U}
  =
  \frac{1}{\Delta}\frac{\partial S}{\partial M},
\)
giving the compact formula
\[
  \boxed{%
    \frac{1}{T}
    =
    \frac{k_B}{\Delta}
    \,\ln\!\Bigl[\frac{2\,(n-M)}{M}\Bigr]
  } .
  \tag{$\ast$}
\]

\subsection*{3.  Initial temperatures}

\paragraph{System \(A\) (all copies excited).}
Here \(n=n_A=100\), \(M=M_A^{(0)}=100\Rightarrow n-M=0\).  
In Eq.\,($\ast$), the argument of the logarithm tends to \(0\), so
\[
  \frac{1}{T_A^{(0)}}\;\longrightarrow\; -\infty
  \quad\Longrightarrow\quad
  \boxed{T_A^{(0)}<0\ (\text{negative temperature})}.
\]

\paragraph{System \(B\) (no copies excited).}
Here \(n=n_B=200\), \(M=M_B^{(0)}=0\Rightarrow n-M=n\).  
The logarithm diverges to \(+\infty\), so
\[
  \frac{1}{T_B^{(0)}}\;\longrightarrow\; +\infty
  \quad\Longrightarrow\quad
  \boxed{T_B^{(0)}\approx 0^{+}\ (\text{very low positive temperature})}.
\]

\subsection*{4.  Direction of energy flow}

Negative temperatures are \emph{hotter} than any positive temperature:
entropy \emph{decreases} when energy is added at \(T<0\), so the system
tends to \emph{release} energy to increase total entropy.
Hence
\[
  A\;(T<0)\quad\longrightarrow\quad B\;(T\approx 0^{+}),
\]
\emph{energy will flow from system \(A\) to system \(B\).}

\subsection*{5.  Qualitative outcome at equilibrium}

\begin{itemize}
  \item Some of the \(100\,\Delta_1\) units of energy initially in \(A\)
        migrate to \(B\).
  \item \(M_A\) decreases from \(100\) and \(M_B\) increases from \(0\),
        until both systems share a \emph{common positive} temperature
        \(T_{\text{eq}}>0\) satisfying Eq.\,($\ast$) for each system.
  \item Because \(\Delta_2=\tfrac12\Delta_1\) is smaller and
        \(n_B>n_A\), \(B\) can accommodate many excitations at relatively
        low energy cost, so in the final state a \emph{majority} of the
        total energy resides in \(B\).
\end{itemize}

\subsection*{6.  Final verdict}

\[
  \boxed{\text{Heat flows from }A\text{ to }B\ \text{until}\ T_A=T_B>0.}
\]

System \(A\) cools (its population inversion is partially undone),
while system \(B\) warms up by accepting excitations.  The energy does
\emph{not} stay in \(A\); equilibrium requires redistributing it so that
the combined entropy is maximised.
\section*{Average internal energy \(\langle U\rangle\) of a three–state system in a heat bath}

We work in the \emph{canonical} ensemble, where the temperature \(T\) is fixed and  
\(\beta \equiv 1/(k_B T)\).

\subsection*{1.  Energies and degeneracies}

\[
  \begin{array}{c|c|c}
    \text{Level} & E_i & \text{Degeneracy } g_i\\ \hline
    0            & 0   & 1 \\
    1            & \Delta & 2
  \end{array}
\]

\subsection*{2.  Partition function}

\[
  Z
  \;=\;
  \sum_i g_i\,e^{-\beta E_i}
  \;=\;
  1 + 2\,e^{-\beta\Delta}.
\]

\subsection*{3.  Canonical average of the energy}

Using \(\langle U\rangle = -\partial(\ln Z)/\partial\beta\):

\begin{align}
  \langle U\rangle
  &= -\frac{\partial}{\partial\beta}\!\bigl[\ln(1+2e^{-\beta\Delta})\bigr]
     \\[6pt]
  &= \frac{2\Delta\,e^{-\beta\Delta}}{1+2e^{-\beta\Delta}}.
     \tag{*}
\end{align}

\subsection*{4.  Simplified closed form}

Multiply numerator and denominator of \((*)\) by \(e^{\beta\Delta}\):

\[
  \boxed{\;
    \langle U\rangle(T)
    \;=\;
    \frac{2\Delta}{e^{\beta\Delta}+2}
    \;=\;
    \frac{2\Delta}{\,e^{\displaystyle\Delta/(k_B T)} + 2}\; }.
\]

\subsection*{5.  Useful limiting cases}

\[
  \begin{aligned}
    T\to 0^{+}\!: &\quad e^{\Delta/(k_B T)}\!\to\!\infty
                   \;\Longrightarrow\; \langle U\rangle \to 0, \\[4pt]
    T\to\infty\!: &\quad e^{\Delta/(k_B T)}\!\to\!1
                   \;\Longrightarrow\; \langle U\rangle \to \dfrac{2}{3}\,\Delta .
  \end{aligned}
\]

Thus the mean energy rises smoothly from \(0\) at very low temperatures to
\(2\Delta/3\) in the high-temperature (equiprobable) limit.
\section*{Limits \(T\!\to\!0\) and \(T\!\to\!\infty\) for a three–state system}

Throughout, let \(\beta \equiv 1/(k_B T)\) and recall  
\[
  Z \;=\; 1 + 2\,e^{-\beta\Delta},
  \qquad
  \langle U\rangle(T)
  \;=\;
  \frac{2\Delta}{e^{\beta\Delta}+2}.
\]

The canonical entropy can be written in the compact form  
\[
  S(T)\;=\;k_B\bigl[\ln Z + \beta\,\langle U\rangle\bigr].
  \tag{1}
\]

%-------------------------------------------------
\subsection*{1.  Low–temperature limit \(T\to 0^{+}\;\;(\beta\to\infty)\)}

\begin{align}
  e^{-\beta\Delta} &\;\longrightarrow\; 0, & Z &\to 1, \\[4pt]
  \langle U\rangle &\;\to\; 0,            & S  &\to 0.
\end{align}

\[
  \boxed{\;
    \langle U\rangle \xrightarrow[T\to 0]{} 0, 
    \qquad
    S \xrightarrow[T\to 0]{} 0
  \;}
\]

The system resides entirely in its non–degenerate ground state,
so both the mean energy and the entropy vanish.

%-------------------------------------------------
\subsection*{2.  High–temperature limit \(T\to\infty\;\;(\beta\to 0)\)}

\begin{align}
  e^{-\beta\Delta} &\;\longrightarrow\; 1, &
  Z &\to 3, \\[4pt]
  \langle U\rangle &\;\to\;
     \frac{2\Delta}{1+2}= \frac{2}{3}\,\Delta,
  &
  S &\;\to\;
     k_B\ln 3 .
\end{align}

\[
  \boxed{\;
    \langle U\rangle \xrightarrow[T\to\infty]{} \dfrac{2}{3}\,\Delta,
    \qquad
    S \xrightarrow[T\to\infty]{} k_B\ln 3
  \;}
\]

At very high temperatures the three microstates become equiprobable
(probability \(1/3\) each), giving the maximal entropy  
\(S_{\max}=k_B\ln 3\) and a mean energy equal to the
equipartition value \(2\Delta/3\).
\subsection*{2.1  Ratio of line intensities as a thermometer}

Let  

\[
  E_A=-13.6\;\text{eV}, 
  \quad 
  E_B=-3.4\;\text{eV},
  \quad 
  E_C=-1.5\;\text{eV},
\qquad
  g_A=2,\; g_B=8,\; g_C=18 .
\]

In thermal equilibrium the population (probability) of state \(i\in\{A,B,C\}\) is  

\[
  P_i \;=\; \frac{g_i\,e^{-\beta E_i}}{\displaystyle\sum_{j=A,B,C} g_j\,e^{-\beta E_j}},
  \qquad 
  \beta \equiv \frac{1}{k_B T}.
\]

The intensity (photon count) of an emission line is proportional to the
number of atoms initially in the upper state of that transition.
Hence\footnote{%
  We assume the spontaneous–emission probabilities for the
  \(653\;\text{nm}\) (C\(\!\to\)B) and \(121\;\text{nm}\) (B\(\!\to\)A) lines
  are comparable, or that any difference is already absorbed into the
  experimental calibration.  The \(102\;\text{nm}\) line from C\(\!\to\)A
  is neglected as stated in the problem.}
%
\[
  \frac{N(653\;\text{nm})}{N(121\;\text{nm})}
  \;=\;
  \frac{P_C}{P_B}
  \;=\;
  \frac{g_C}{g_B}\,
  e^{-\beta\,(E_C-E_B)}.
\]

Insert the numbers:

\[
  E_C - E_B \;=\; (-1.5) - (-3.4) = 1.9\;\text{eV},
  \qquad
  \frac{g_C}{g_B} \;=\; \frac{18}{8}= \frac{9}{4}.
\]

\[
  \boxed{
  \displaystyle
  \frac{N(653\;\text{nm})}{N(121\;\text{nm})}
  \;=\;
  \frac{9}{4}\,
  \exp\!\left(
     -\frac{1.9\ \text{eV}}{k_B T}
  \right)}
  \;.
\]

With \(k_B = 8.617\times10^{-5}\ \text{eV/K}\), this expression lets one extract the
temperature \(T\) directly from the measured line-intensity ratio.
\subsection*{2.2  Extracting the temperature from the intensity ratio}

Starting point (derived in §2.1):
\[
   R \;\equiv\; 
   \frac{N(653\;\text{nm})}{N(121\;\text{nm})}
   \;=\;
   \frac{g_C}{g_B}
   \exp\!\Bigl[-\beta\,(E_C-E_B)\Bigr],
   \qquad 
   \beta \;=\; \frac{1}{k_B T}.
\]

\paragraph{1.  Isolate the Boltzmann factor}

\[
   \exp\!\Bigl[-\beta\,(E_C-E_B)\Bigr]
   \;=\;
   R\,\frac{g_B}{g_C}.
\]

\paragraph{2.  Take the natural logarithm}

\[
   -\beta\,(E_C-E_B)
   \;=\;
   \ln\!\Bigl[R\,\dfrac{g_B}{g_C}\Bigr].
\]

\paragraph{3.  Solve for \(T\)}

\[
   T
   \;=\;
   \frac{E_C-E_B}{k_B}\;
   \Biggl[\;
     -\,\ln\!\Bigl(R\,\dfrac{g_B}{g_C}\Bigr)
   \Biggr]^{-1}.
\]

\[
   \boxed{
     \displaystyle 
     T
     \;=\;
     \frac{E_C-E_B}{\,k_B\,\ln\!\bigl(\dfrac{g_C}{g_B\,R}\bigr)}
   }.
\]

\noindent
Here
\(
   R = N(653\;\text{nm})/N(121\;\text{nm})
\)
is the experimentally measured line–intensity ratio,
\(E_C-E_B\) is the energy difference between the upper levels
(\(1.9\;\text{eV}\) for hydrogen), \(g_C/g_B=18/8\), and
\(k_B = 8.617\times10^{-5}\;\text{eV/K}\).
\subsection*{2.3  Numerical temperatures from the measured line–intensity ratios}

Recall the general result from §2.2
\[
   T \;=\;
   \frac{E_C-E_B}{k_B\,\ln\!\bigl(\dfrac{g_C}{g_B\,R}\bigr)},
   \qquad
   R \;\equiv\; \frac{N(653\text{ nm})}{N(121\text{ nm})}.
\]

\[
  \begin{aligned}
    E_C-E_B &= 1.9\;\text{eV}, &
    k_B &= 8.617\times10^{-5}\;\text{eV/K}, &
    \frac{g_C}{g_B} &= \frac{18}{8} = 2.25 .
  \end{aligned}
\]

\paragraph{Star 1: \(R = 0.005\)}

\[
  T_1
  = \frac{1.9\;\text{eV}}
         {\,k_B\,\ln\!\bigl(\dfrac{2.25}{0.005}\bigr)}
  = \frac{1.9}
         {\,\bigl(8.617\times10^{-5}\bigr)\,
            \ln(450)}
  \approx \boxed{3.6\times10^{3}\ \text{K}}.
\]

\paragraph{Star 2: \(R = 0.038\)}

\[
  T_2
  = \frac{1.9\;\text{eV}}
         {\,k_B\,\ln\!\bigl(\dfrac{2.25}{0.038}\bigr)}
  = \frac{1.9}
         {\,\bigl(8.617\times10^{-5}\bigr)\,
            \ln(59.2)}
  \approx \boxed{5.4\times10^{3}\ \text{K}}.
\]

\[
  \boxed{T_1 \approx 3.6\ \text{kk}, \qquad T_2 \approx 5.4\ \text{kk}}
\]
(where “kk” denotes kilokelvin).
\subsection*{2.4  Why no hydrogen‐line emission is seen at room temperature}

\paragraph{1.  Thermal energy versus excitation gaps}

At room temperature \(T\simeq 300\;\text{K}\),
\[
  k_B T \;\approx\; (8.617\times10^{-5}\;\text{eV/K})(300\;\text{K})
  \;\approx\; 2.6\times10^{-2}\;\text{eV}.
\]

Compare with the excitation energies of atomic hydrogen:

\[
  \begin{aligned}
    A\!\to\!B: \;&\; \Delta E_{AB}=10.2\;\text{eV}, \\[4pt]
    B\!\to\!C: \;&\; \Delta E_{BC}= 1.9\;\text{eV}.
  \end{aligned}
\]

\paragraph{2.  Boltzmann population factors}

The probability that an atom thermally occupies an excited level \(i\)
relative to the ground state \(A\) is

\[
  \frac{P_i}{P_A} \;=\;
  \frac{g_i}{g_A}\,
  e^{-\Delta E_{Ai}/k_B T}.
\]

\[
  \begin{aligned}
    \frac{P_B}{P_A}
    &=
    \frac{\,8\,}{2}\,
    e^{-10.2/0.026}
    \;\approx\;
    4\,e^{-390}
    \;\approx\;
    0, \\[6pt]
    \frac{P_C}{P_A}
    &=
    \frac{18}{2}\,
    e^{-12.1/0.026}
    \;\approx\;
    9\,e^{-465}
    \;\approx\;
    0.
  \end{aligned}
\]

Both ratios are astronomically small: essentially \(\!>10^{-150}\).

\paragraph{3.  Consequence for spontaneous emission}

\begin{itemize}
  \item Virtually every hydrogen atom resides in the ground state \(A\).
  \item Spontaneous emission at 121\,nm (B\(\!\to\)A) or 653\,nm (C\(\!\to\)B)
        requires the atom to be \emph{initially} in \(B\) or \(C\),
        which almost never occurs.
  \item Hence the photon flux at these wavelengths is far below any
        realistic detection threshold.
\end{itemize}

\paragraph{4.  Additional suppression mechanisms}

\begin{itemize}
  \item At room temperature hydrogen is predominantly molecular
        (\(\mathrm{H_2}\)), removing the atomic energy levels altogether.
  \item Collisions in a dense gas more often cause
        \emph{non-radiative} de-excitation than photon emission.
\end{itemize}

\[
  \boxed{\text{Room-temperature thermal energy is too small to populate the excited states, so no detectable line emission occurs.}}
\]
\subsection*{3.1  Ratio of \(\mathrm{N_2}\) to \(\mathrm{O_2}\) molecules at sea level}

At sea level (height \(h=0\)), the Boltzmann factor is unity,
\(f(0)=\exp(-mgh/k_BT)=1\),
so the relative abundances are set entirely by the
composition of dry air:

\[
  x_{\mathrm{N_2}} = 0.80,
  \qquad
  x_{\mathrm{O_2}} = 0.20,
\]
where \(x_i\) denotes the mole (or number) fraction of species \(i\).

For a volume \(V\) at fixed \(T\) and \(P\)
the ideal-gas relation gives the same total number density for both
species, so

\[
  N_{\mathrm{N_2}} = x_{\mathrm{N_2}}\,N_{\text{tot}},
  \qquad
  N_{\mathrm{O_2}} = x_{\mathrm{O_2}}\,N_{\text{tot}}.
\]

Therefore
\[
  \boxed{\;
    \frac{N_{\mathrm{N_2}}}{N_{\mathrm{O_2}}}
    = \frac{x_{\mathrm{N_2}}}{x_{\mathrm{O_2}}}
    = \frac{0.80}{0.20}
    = 4
  \;} .
\]

\[
  \text{Hence the ratio is }N_{\mathrm{N_2}}:N_{\mathrm{O_2}} = 4:1.
\]
\subsection*{3.2  Density ratio between two altitudes}

Let the local number density of a species of mass \(m\) at height \(h\) be
\(\displaystyle n(h)=n_0\,e^{-mgh/k_BT}\),
where \(n_0\) is the density at \(h=0\).  
The number of particles contained in a fixed volume \(V\) at height \(h\) is then  

\[
  N(h)=n(h)\,V
       =n_0\,V\,e^{-mgh/k_BT}.
\]

For two heights \(h_1\) and \(h_2\) (\(h_2>h_1\) without loss of generality):

\begin{align}
  \frac{N(h_1)}{N(h_2)}
  &=\frac{n_0 V e^{-mgh_1/k_BT}}{n_0 V e^{-mgh_2/k_BT}}
    \\[6pt]
  &=\exp\!\Bigl[-\tfrac{mg}{k_BT}\,(h_1-h_2)\Bigr].
\end{align}

\[
  \boxed{\displaystyle
    \frac{N(h_1)}{N(h_2)}
    =\exp\!\Bigl[\tfrac{mg}{k_BT}\,(h_2-h_1)\Bigr]}
  \qquad(\text{same volume }V,\ \text{constant }T).
\]

In words: moving a volume \(V\) up by \(\Delta h=h_2-h_1\) reduces the particle count by the Boltzmann factor \(\exp[-mg\Delta h/k_BT]\).
\subsection*{3.3  \(\displaystyle \frac{N_{\mathrm{N_2}}(h_2)}{N_{\mathrm{O_2}}(h_2)}\): composition at a higher altitude}

Let \(h_1=0\) denote sea level and \(h_2>h_1\) the higher elevation.  
From part 3.2 the particle number in a fixed volume \(V\) obeys  

\[
  N_i(h) \;=\; N_i(h_1)\,
               \exp\!\Bigl[-\,m_i g (h-h_1)/k_B T\Bigr],
  \qquad
  i\in\{\mathrm{N_2},\mathrm{O_2}\}.
\]

Hence

\[
  \begin{aligned}
      \frac{N_{\mathrm{N_2}}(h_2)}{N_{\mathrm{O_2}}(h_2)}
      &=\frac{N_{\mathrm{N_2}}(h_1)
             e^{-m_{\mathrm{N_2}}g(h_2-h_1)/k_B T}}
            {N_{\mathrm{O_2}}(h_1)
             e^{-m_{\mathrm{O_2}}g(h_2-h_1)/k_B T}} \\[6pt]
      &=\frac{N_{\mathrm{N_2}}(h_1)}{N_{\mathrm{O_2}}(h_1)}
        \,\exp\!\Bigl[-(m_{\mathrm{N_2}}-m_{\mathrm{O_2}})\,
                       g (h_2-h_1)/k_B T\Bigr].
  \end{aligned}
\]

From part 3.1 the sea-level ratio is  

\[
  \frac{N_{\mathrm{N_2}}(h_1)}{N_{\mathrm{O_2}}(h_1)} = 4 .
\]

Therefore

\[
  \boxed{\;
    \frac{N_{\mathrm{N_2}}(h_2)}{N_{\mathrm{O_2}}(h_2)}
    = 4\,
      \exp\!\Bigl[
        \frac{(m_{\mathrm{O_2}}-m_{\mathrm{N_2}})\,g\,(h_2-h_1)}
             {k_B T}
      \Bigr]
  }.
\]

\emph{Interpretation.}  
Because \(\,m_{\mathrm{O_2}}>m_{\mathrm{N_2}}\), the exponent is positive,
so the \(\mathrm{N_2}/\mathrm{O_2}\) ratio increases with altitude:
the heavier \(\mathrm{O_2}\) molecules thin out more rapidly than the lighter
\(\mathrm{N_2}\) molecules.
\subsection*{3.4  Gas separation in a centrifuge}

A high–speed centrifuge replaces the gravitational acceleration \(g\) by a
much larger \emph{effective} acceleration  
\(a = \alpha g\) (with \(\alpha \approx 1.5\times10^{4}\) in this problem).
If the rotor tube has length \(\ell = 0.50\;\text{m}\) and is held at
\(T = 300\;\text{K}\), we can treat the “bottom’’ of the tube as
height \(h_1 = 0\) and the “top’’ as \(h_2 = \ell\).

\paragraph{1.  Percentage of \(\mathrm{O_2}\) at the top}

From part~3.3
\[
  \frac{N_{\mathrm{N_2}}(h_2)}{N_{\mathrm{O_2}}(h_2)}
  \;=\;
  4\,
  \exp\!\Bigl[
      \frac{(m_{\mathrm{O_2}}-m_{\mathrm{N_2}})\,a\,(h_2-h_1)}
           {k_B T}
  \Bigr].
\]

\[
  \begin{aligned}
    m_{\mathrm{N_2}} &= 28\,u = 28(1.6605\times10^{-27}) = 4.65\times10^{-26}\;\text{kg},\\[2pt]
    m_{\mathrm{O_2}} &= 32\,u = 32(1.6605\times10^{-27}) = 5.31\times10^{-26}\;\text{kg},\\[2pt]
    \Delta m         &= m_{\mathrm{O_2}}-m_{\mathrm{N_2}}
                      = 6.64\times10^{-27}\;\text{kg},\\[2pt]
    a                &= 15\,000\,g = 15\,000(9.81) = 1.47\times10^{5}\;\text{m/s}^{2},\\[2pt]
    k_B T            &= (1.3807\times10^{-23})(300) = 4.14\times10^{-21}\;\text{J},\\[2pt]
    \Delta h         &= 0.50\;\text{m}.
  \end{aligned}
\]

Exponent:
\[
  \frac{\Delta m\,a\,\Delta h}{k_B T}
  = \frac{(6.64\times10^{-27})(1.47\times10^{5})(0.50)}
         {4.14\times10^{-21}}
  \;\approx\; 0.12.
\]

Hence
\[
  \frac{N_{\mathrm{N_2}}(h_2)}{N_{\mathrm{O_2}}(h_2)}
  = 4\,e^{0.12}
  \approx 4.50.
\]

The mole fraction of \(\mathrm{O_2}\) at the top is therefore
\[
  x_{\mathrm{O_2}}(h_2)
  = \frac{1}{1 + N_{\mathrm{N_2}}/N_{\mathrm{O_2}}}
  = \frac{1}{1+4.50}
  \approx 0.182 \;(\text{or }18.2\%).
\]

\[
  \boxed{\text{Only about }18\%\text{ of the gas at the top is } \mathrm{O_2}\;
          (\text{down from }20\%\text{ at the bottom}).}
\]

\paragraph{2.  Why uranium–isotope separation requires very large, fast centrifuges}

\begin{itemize}
  \item For uranium hexafluoride, the relative mass difference between
        \(\mathrm{^{235}UF_6}\) (\(\approx349\,u\)) and
        \(\mathrm{^{238}UF_6}\) (\(\approx352\,u\))
        is barely \(0.9\%\).
  \item The separation factor in a centrifuge is
        \(\displaystyle
          \exp\!\bigl[\Delta m\,a\,\ell/(k_B T)\bigr]\),
        where \(\Delta m\) is this \emph{tiny} mass difference.
  \item To obtain an appreciable exponent one must raise either
        the path length \(\ell\) or the acceleration \(a = \omega^{2}r\).
        Practically this means:
        \begin{enumerate}
          \item very large rotor radii (\(r\)) to increase path length,
          \item extremely high angular speeds (\(\omega\)) to boost \(a\).
        \end{enumerate}
  \item Even so, each centrifuge stage gives only a modest enrichment;
        thousands of ultra-fast stages (a “cascade’’) are required for
        reactor-grade or weapons-grade material.
\end{itemize}

\[
  \boxed{\text{Tiny mass splittings }\Rightarrow\text{ need huge }a\text{ and long }\ell
         \;\Rightarrow\; large, very fast centrifuges.}
\]
\end{document}
