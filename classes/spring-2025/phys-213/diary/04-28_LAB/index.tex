\documentclass[12pt]{article}

% Packages
\usepackage[margin=1in]{geometry}
\usepackage{amsmath,amssymb,amsthm}
\usepackage{enumitem}
\usepackage{hyperref}
\usepackage{xcolor}
\usepackage{import}
\usepackage{xifthen}
\usepackage{pdfpages}
\usepackage{transparent}
\usepackage{listings}
\usepackage{tikz}
\usepackage{physics}
\usepackage{siunitx}
  \usetikzlibrary{calc,patterns,arrows.meta,decorations.markings}


\DeclareMathOperator{\Log}{Log}
\DeclareMathOperator{\Arg}{Arg}

\lstset{
    breaklines=true,         % Enable line wrapping
    breakatwhitespace=false, % Wrap lines even if there's no whitespace
    basicstyle=\ttfamily,    % Use monospaced font
    frame=single,            % Add a frame around the code
    columns=fullflexible,    % Better handling of variable-width fonts
}

\newcommand{\incfig}[1]{%
    \def\svgwidth{\columnwidth}
    \import{./Figures/}{#1.pdf_tex}
}
\theoremstyle{definition} % This style uses normal (non-italicized) text
\newtheorem{solution}{Solution}
\newtheorem{proposition}{Proposition}
\newtheorem{problem}{Problem}
\newtheorem{lemma}{Lemma}
\newtheorem{theorem}{Theorem}
\newtheorem{remark}{Remark}
\newtheorem{note}{Note}
\newtheorem{definition}{Definition}
\newtheorem{example}{Example}
\newtheorem{corollary}{Corollary}
\theoremstyle{plain} % Restore the default style for other theorem environments
%

% Theorem-like environments
% Title information
\title{}
\author{Jerich Lee}
\date{\today}

\begin{document}

\maketitle
%%%%%%%%%%%%%%%%%%%%%%%%%%%%%%%%%%%%%%%%%%%%%%%%%%%%%%%%%%%%%%%%%%%%%%%%%%%
%  ANALYSIS SHEET — “Clausius–Clapeyron”  (copy-and-paste LaTeX template) %
%%%%%%%%%%%%%%%%%%%%%%%%%%%%%%%%%%%%%%%%%%%%%%%%%%%%%%%%%%%%%%%%%%%%%%%%%%%

\begin{enumerate}
  %-----------------------------------------------------------------------
  \item \textbf{Expected slope on a \(\ln p\) vs.\(1/T\) plot}
  
  The integrated Clausius–Clapeyron relation for a single molecule is  
  \[
    \ln p \;=\; \ln p_0 \;-\; \frac{L}{k_B}\,\Bigl(\frac{1}{T}\Bigr),
  \]
  so the \emph{slope} of the straight line is
  \[
    \boxed{\,m_{\text{theory}} = -\dfrac{L}{k_B}\,}.
  \]
  
  %-----------------------------------------------------------------------
  \item \textbf{Does the experimental data follow a straight line on the
        semi‐log plot?}
  
  \begin{itemize}
    \item[] \rule{0.93\linewidth}{0.4pt}
    \item[] \rule{0.93\linewidth}{0.4pt}
  \end{itemize}
  
  %-----------------------------------------------------------------------
  \item \textbf{Pick two convenient points on the line}
  
  \[
    \begin{aligned}
      \text{Point 1:}\;& (P_1,\;T_1) \;=\;(\;\rule{2cm}{0.15mm},\;
                                           \rule{2cm}{0.15mm}\;)\;,\\
      \text{Point 2:}\;& (P_2,\;T_2) \;=\;(\;\rule{2cm}{0.15mm},\;
                                           \rule{2cm}{0.15mm}\;)\;.
    \end{aligned}
  \]
  
  %-----------------------------------------------------------------------
  \item \textbf{Calculate the experimental slope}
  
  \[
    m_{\text{exp}}
    \;=\;
    \frac{\ln P_2 - \ln P_1}{\dfrac{1}{T_2} - \dfrac{1}{T_1}}
    \;=\;
    \rule{5cm}{0.15mm}
  \]
  
  %-----------------------------------------------------------------------
  \item \textbf{Estimate the uncertainty in the slope}
  
  (e.g.\ by drawing the steepest and shallowest reasonable straight lines
  through the error bars)
  
  \[
    \sigma_{m}
    \;=\;
    \rule{4cm}{0.15mm}
  \]
  
  %-----------------------------------------------------------------------
  \item \textbf{Determine the latent heat per molecule}
  
  \[
    L_{\text{exp}}
    \;=\;
    -\,m_{\text{exp}}\,k_B
    \;=\;
    (\rule{2.5cm}{0.15mm})\;\text{J}
    \quad\Longrightarrow\quad
    \frac{L_{\text{exp}}}{1.60218\times10^{-19}}
    \;=\;
    (\rule{2.5cm}{0.15mm})\;\text{eV}.
  \]
  
  %-----------------------------------------------------------------------
  \item \textbf{Comparison with the book value \(L_{\text{book}} = 0.42\;\text{eV}\)}
  
  \[
    \text{Percent difference}
    \;=\;
    \frac{\lvert L_{\text{exp}}-L_{\text{book}}\rvert}
         {L_{\text{book}}}\times100\%
    \;=\;
    \rule{3cm}{0.15mm}\%.
  \]
  
  \vspace*{0.5em}
  Comment on agreement (within uncertainties? systematic offsets?):
  \begin{itemize}
    \item[] \rule{0.93\linewidth}{0.4pt}
    \item[] \rule{0.93\linewidth}{0.4pt}
  \end{itemize}
  
  \end{enumerate}
  \subsection*{Predictions}

\begin{enumerate}
  \item \textbf{Immersing the room-temperature aluminium cylinder in liquid\,N\textsubscript{2}.}
        \begin{itemize}
          \item The cylinder is initially at roughly \(T_{\text{room}}\simeq 295\;\text{K}\), 
                whereas liquid nitrogen is at its boiling point
                \(T_{\text{LN}_{2}}\simeq 77\;\text{K}\).
          \item Heat therefore flows rapidly \emph{from} the aluminium \emph{into} the LN\(_2\),
                providing the latent heat of vaporisation that allows the nitrogen to boil.
          \item \emph{Prediction:} the mass of LN\(_2\) in the Styrofoam cup will
                \textbf{decrease much faster} than before the cylinder was added.
                You will observe vigorous bubbling and a visible cloud of cold
                nitrogen gas above the cup until the cylinder cools to \(77\;\text{K}\).
        \end{itemize}

  \item \textbf{Transferring the now-cold cylinder to a chilled water bath.}
        \begin{itemize}
          \item The aluminium has been cooled to \(77\;\text{K}\), far below the
                freezing point of water (\(273\;\text{K}\)).
          \item When it enters the water bath the temperature gradient is reversed:
                heat flows from the water into the cylinder.
          \item \emph{Prediction:}
                \begin{itemize}
                  \item A thin layer of ice will form almost instantly on the cylinder’s surface
                        (the water in direct contact freezes).
                  \item The bulk water temperature will drop; if the bath is small enough, it may
                        approach the freezing point or even begin to freeze.
                  \item As the cylinder warms toward \(0^\circ\text{C}\), the rate of ice formation
                        slows and eventually stops once thermal equilibrium is reached.
                \end{itemize}
          \item The dramatic contrast—rapid LN\(_2\) boil-off first, then rapid ice formation—illustrates how the direction of heat flow depends solely on relative temperature, not on the substance involved.
        \end{itemize}
\end{enumerate}
Below is a ready-to-paste LaTeX block that answers every item in the “Questions” panel.

\subsection*{Analysis}

\begin{enumerate}
  \item \textbf{Comparison of measured and tabulated latent heats \& error sources}

        Your experimental value for the latent heat per molecule,
        \(
           L_{\text{exp}}\;(\text{from the slope})
        \),
        should be compared with the book value  
        \(
           L_{\text{book}}\simeq 0.42\;\text{eV}\;(6.8\times10^{-20}\,\mathrm{J})
        \).
        Typical discrepancies and their origins include
        \begin{itemize}
          \item \emph{Heat losses to the surroundings}  
                (conduction through the cup walls, convection, radiation).
          \item \emph{Evaporation not accounted for} while transferring LN\(_2\)
                or while weighing the Styrofoam cup.
          \item \emph{Finite heat capacity of the aluminium cylinder and cup}  
                —often neglected in the simple energy balance.
          \item \emph{Temperature gradients} inside the LN\(_2\); the
                thermometer reads the bulk value, not the film temperature at
                the surface of the cylinder.
          \item \emph{Scale calibration and timing errors} when determining the
                rate of mass loss.
        \end{itemize}
        Any one of these can push the fitted slope away from the ideal
        \(-L/k_B\).

  \item \textbf{Why does the LN\(_2\) boil \emph{most vigorously} just before
        reaching equilibrium?}

        Initially the hot aluminium is wrapped in a thick vapour film
        (Leidenfrost layer).  
        Nitrogen \emph{gas} has a very low thermal conductivity, so
        heat transfer (and thus boil-off) is throttled despite the large
        temperature difference.

        As the cylinder cools, the vapour layer collapses and liquid
        nitrogen comes into direct contact with the metal.
        Liquid N\(_2\) conducts heat \(\sim\! 100\) times better than its
        vapour, producing a sudden surge of heat flow.
        The result is a last, rapid burst of boiling that subsides
        only when the cylinder’s temperature falls to \(T_{\text{LN}_2}\).

  \item \textbf{Why does water freeze more slowly once an ice layer has formed?}

        The freshly deposited ice coats the aluminium with an insulating
        shell whose thermal conductivity
        (\(\kappa_{\text{ice}}\approx 2\,\mathrm{W\,m^{-1}K^{-1}}\))
        is an order of magnitude lower than that of aluminium
        (\(\kappa_{\text{Al}}\approx 200\,\mathrm{W\,m^{-1}K^{-1}}\)).
        Heat must now diffuse through this ice barrier before it can
        reach the cold metal core, dramatically reducing the cooling rate of
        the still-liquid water and slowing further ice growth.
\end{enumerate}
\end{document}
