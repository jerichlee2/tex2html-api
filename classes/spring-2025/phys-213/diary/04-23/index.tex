\documentclass[12pt]{article}

% Packages
\usepackage[margin=1in]{geometry}
\usepackage{amsmath,amssymb,amsthm}
\usepackage{enumitem}
\usepackage{hyperref}
\usepackage{xcolor}
\usepackage{import}
\usepackage{xifthen}
\usepackage{pdfpages}
\usepackage{transparent}
\usepackage{listings}
\usepackage{tikz}
\usepackage{physics}
\usepackage{siunitx}

  \usetikzlibrary{calc,patterns,arrows.meta,decorations.markings}


\DeclareMathOperator{\Log}{Log}
\DeclareMathOperator{\Arg}{Arg}

\lstset{
    breaklines=true,         % Enable line wrapping
    breakatwhitespace=false, % Wrap lines even if there's no whitespace
    basicstyle=\ttfamily,    % Use monospaced font
    frame=single,            % Add a frame around the code
    columns=fullflexible,    % Better handling of variable-width fonts
}

\newcommand{\incfig}[1]{%
    \def\svgwidth{\columnwidth}
    \import{./Figures/}{#1.pdf_tex}
}
\theoremstyle{definition} % This style uses normal (non-italicized) text
\newtheorem{solution}{Solution}
\newtheorem{proposition}{Proposition}
\newtheorem{problem}{Problem}
\newtheorem{lemma}{Lemma}
\newtheorem{theorem}{Theorem}
\newtheorem{remark}{Remark}
\newtheorem{note}{Note}
\newtheorem{definition}{Definition}
\newtheorem{example}{Example}
\newtheorem{corollary}{Corollary}
\theoremstyle{plain} % Restore the default style for other theorem environments
%

% Theorem-like environments
% Title information
\title{}
\author{Jerich Lee}
\date{\today}

\begin{document}

\maketitle
% --------------------------------------------------------------------------
% Clear, step-by-step explanation of the “Phases” vignette
% (place inside any article document)

\subsection*{Understanding Phase Diagrams via Chemical Potential}

\paragraph{1.  Free energy of a multi-phase system.}
For a fixed temperature \(T\) and pressure \(p\) we write the Gibbs free
energy of \(N\) water molecules split between two phases
(liquid \(\ell\) and gas \(g\)) as
\[
  G_{\mathrm{tot}}
  \;=\;
  \mu_{\ell}(p,T)\,N_{\ell} \;+\; \mu_{g}(p,T)\,N_{g},
  \qquad
  N_{\ell}+N_{g}=N_{\mathrm{w}} .
\]
Here \(\mu_{\alpha}(p,T)\) is the \emph{chemical potential}
(Gibbs free energy per molecule) of phase~\(\alpha\).
Because \(p\) and \(T\) are imposed by the surroundings,
\(\mu_{\ell}\) and \(\mu_{g}\) are just two \emph{numbers};
the \emph{only} variables the system can adjust are
\(N_{\ell}\) and \(N_{g}\).

\paragraph{2.  Minimising \(G_{\mathrm{tot}}\) picks the stable phase.}
\begin{itemize}
  \item If \(\mu_{\ell} < \mu_{g}\), the derivative
        \(\partial G_{\mathrm{tot}}/\partial N_{\ell} = \mu_{\ell}-\mu_{g}\)
        is \emph{negative}.  Decreasing \(G_{\mathrm{tot}}\)
        therefore means converting every molecule to the liquid:
        \(N_{\ell}=N_{\mathrm{w}}\), \(N_{g}=0\).
  \item Conversely, if \(\mu_{g} < \mu_{\ell}\) the gas is favoured:
        \(N_{g}=N_{\mathrm{w}}\), \(N_{\ell}=0\).
  \item If \(\mu_{\ell} = \mu_{g}\) the derivative vanishes and
        \(G_{\mathrm{tot}}\) is \emph{flat} along the
        \((N_{\ell},N_{g})\)-direction.  Any mixture of the two phases
        has the same \(G_{\mathrm{tot}}\);
        liquid and vapour may therefore coexist at equilibrium.
\end{itemize}

\paragraph{3.  Phase-diagram interpretation.}
\begin{enumerate}[label=\textbf{\arabic*.}, itemsep=0.4em]
  \item The \emph{phase boundaries} (solid curves in Fig.\,2)
        are the \((p,T)\) pairs satisfying
        \(\mu_{\ell}(p,T)=\mu_{g}(p,T)\)
        (or \(\mu_{\ell}=\mu_{s}\) for solid/liquid, \(\mu_{g}=\mu_{s}\) for
        solid/gas).  Crossing such a line swaps the order of the
        two chemical potentials and thus swaps the stable phase.
  \item At the \emph{triple point} all three chemical potentials coincide,
        \(\mu_{\ell}=\mu_{g}=\mu_{s}\); ice, water, and vapour may all
        coexist.
  \item Beyond the \emph{critical point} the distinction between liquid
        and gas disappears: their chemical potentials remain equal even
        though the densities merge into a single \emph{supercritical} phase.
\end{enumerate}

\paragraph{4.  Why we seldom see coexistence in everyday life.}
At ordinary atmospheric pressure $\SI{101}{kPa}$ the equality
\(\mu_{\ell}=\mu_{g}\) holds only at \(T=\SI{100}{\text{celsius} }\).
At room temperature \(\mu_{\ell}<\mu_{g}\), so liquid water is stable and
any vapour eventually condenses; below $\SI{0}{\text{celsius} }$,
\(\mu_{s}<\mu_{\ell}\) and ice is ultimately favoured.
Coexistence is thus confined to
\emph{one-dimensional} curves in the two-dimensional \((p,T)\)-plane,
making it a comparatively rare occurrence outside controlled
laboratory conditions.

\bigskip
\noindent
\textbf{Key takeaway:} \emph{A phase diagram is nothing more than a map
of which phase has the lowest chemical potential at each \((p,T)\).
Lines and points where two or more chemical potentials coincide mark
where phases can coexist.}
\paragraph{Problem.}
An ideal gas of He atoms is heated from \(T_1 = 200\;\mathrm{K}\) to \(T_2 = 400\;\mathrm{K}\).
The root–mean–squared speed is defined by
\[
  v_{\mathrm{rms}}
  \;=\;
  \sqrt{\langle |{\bf v}|^{2}\rangle }.
\]
By what factor does \(v_{\mathrm{rms}}\) increase?

\paragraph{Solution (step–by–step).}
\begin{enumerate}
  \item For a monatomic ideal gas,
        \[
          v_{\mathrm{rms}}
          \;=\;
          \sqrt{\frac{3k_{\mathrm B}T}{m}},
        \]
        where \(k_{\mathrm B}\) is Boltzmann’s constant and \(m\) is the particle mass.
        Hence \(v_{\mathrm{rms}}\propto\sqrt{T}\) for a fixed species.

  \item Form the ratio of the rms speeds at the two temperatures:
        \[
          \frac{v_{\mathrm{rms},2}}{v_{\mathrm{rms},1}}
          \;=\;
          \sqrt{\frac{T_2}{T_1}}
          \;=\;
          \sqrt{\frac{400\;\mathrm{K}}{200\;\mathrm{K}}}
          \;=\;
          \sqrt{2}.
        \]

  \item Therefore, when the temperature is doubled, the rms speed
        increases by a factor of \(\boxed{\sqrt{2}}\).
\end{enumerate}
\paragraph{Problem.}
A helium balloon is initially at \(T_1 = \SI{23}{\celsius}\) with volume
\(V_1 = \SI{0.0042}{\meter^{3}}\).
The balloon is flexible, so its internal pressure always equals the external
(atmospheric) pressure \(p = \SI{101.3}{\kilo\pascal}\).
After lying in the sun the volume increases by \(12\%\).
Find the new temperature \(T_2\).

\paragraph{Solution (step by step).}

\begin{enumerate}
  \item \textbf{Convert the initial temperature to kelvin.}
        \[
          T_1 = 23^{\circ}\mathrm{C}
          = 23 + 273.15
          = \SI{296.15}{\kelvin}.
        \]

  \item \textbf{State the relationship for an isobaric ideal-gas process.}\\
        Because the pressure is constant and the number of moles does not
        change, Charles’s law applies:
        \[
          \frac{V}{T} = \text{constant}
          \quad\Longrightarrow\quad
          \frac{V_2}{V_1} = \frac{T_2}{T_1}.
        \]

  \item \textbf{Insert the \(12\%\) volume increase.}
        \[
          V_2 = 1.12\,V_1
          \quad\Longrightarrow\quad
          T_2 = 1.12\,T_1.
        \]

  \item \textbf{Compute \(T_2\).}
        \[
          T_2
          = 1.12 \times \SI{296.15}{\kelvin}
          = \SI{331.7}{\kelvin}.
        \]

  \item \textbf{Convert back to degrees Celsius (if desired).}
        \[
          T_2
          = 331.7 - 273.15
          \approx \boxed{\SI{58.5}{\celsius}}.
        \]
\end{enumerate}

\noindent
\textbf{Answer:} The balloon’s temperature rises to approximately
\(\boxed{T_2 \approx 3.32\times 10^{2}\;\text{K} \;(\sim\!59^{\circ}\text{C})}\).
\paragraph{Work done \emph{by} the gas during the isobaric expansion.}

\begin{enumerate}
  \item \textbf{Initial and final volumes.}
        \[
          V_1 = \SI{0.0042}{\meter^{3}}, 
          \qquad
          V_2 = V_1 \,(1+0.12) = 1.12\,V_1
        \]
        \[
          \Delta V = V_2 - V_1 = 0.12\,V_1
                   = 0.12 \times \SI{0.0042}{\meter^{3}}
                   = \SI{5.04e-4}{\meter^{3}}.
        \]

  \item \textbf{Constant external pressure.}
        \[
          p = \SI{101.3}{\kilo\pascal} = \SI{101.3e3}{\pascal}.
        \]

  \item \textbf{Isobaric work (positive when done \emph{by} the gas).}
        \[
          W_{\text{by}} = p\,\Delta V
          = (\SI{101.3e3}{\pascal}) \times (\SI{5.04e-4}{\meter^{3}})
          \approx \SI{5.1e1}{\joule}.
        \]
\end{enumerate}


  \[
    \boxed{%
      W_{\text{by}} \;\approx\; \SI{5.1e1}{\joule}%
    }
\]
\paragraph{Concept check.}
For an ideal gas the root–mean–squared speed is
\[
  v_{\mathrm{rms}}
  = \sqrt{\frac{3k_{\mathrm B}T}{m}},
\]
which depends \emph{only} on the absolute temperature \(T\) and the molecular mass \(m\).

\begin{enumerate}[label=\textbf{\alph*.}]
  \item Compressing the gas to half its volume (at fixed \(T\)) \emph{raises the pressure} according to Boyle’s law, \(pV=\text{const.}\)
  \item However, because \(T\) is specified to remain constant, the average kinetic energy per particle \(\tfrac{3}{2}k_{\mathrm B}T\) is unchanged.
  \item Therefore \(v_{\mathrm{rms}}\) is unaffected.
\end{enumerate}

\[
  \boxed{\text{\(v_{\mathrm{rms}}\) would remain the same.}}
\]
\subsection*{Equilibrium partition of volume}

\paragraph{Given data}
\[
  N_A = 1~\text{mol}, \qquad
  N_B = 2~\text{mol}, \qquad
  V_{\text{tot}} = 1~\text{m}^{3}, \qquad
  b = 4\times 10^{-4}~\text{m}^{3}\!\big/\text{mol}.
\]

\paragraph{Entropy of each compartment}
\[
  S_A = N_A k_B \ln\!\bigl(V_A - bN_A\bigr) + f(U_A,N_A),
  \qquad
  S_B = N_B k_B \ln(V_B) + f(U_B,N_B),
\]
where the energy–dependent terms \(f\) do not involve the volumes.
With the movable, impermeable membrane \(V_A+V_B = V_{\text{tot}}\) is fixed.

\paragraph{Maximising the total entropy}
Set \(S_{\text{tot}} = S_A + S_B\) and vary \(V_A\) while holding
\(V_{\text{tot}}\) constant.

\[
  \frac{\partial S_{\text{tot}}}{\partial V_A}
  = \frac{\partial S_A}{\partial V_A}
    + \frac{\partial S_B}{\partial V_A}
  = N_A k_B \frac{1}{V_A - bN_A}
    - N_B k_B \frac{1}{V_{\text{tot}}-V_A}
  \stackrel{!}{=} 0 .
\]

\textbf{Cancel \(k_B\)} and rearrange:
\[
  \frac{N_A}{V_A - bN_A} \;=\; \frac{N_B}{V_{\text{tot}}-V_A}.
\]

Insert the numerical values:

\[
  \frac{1}{V_A - 4\times10^{-4}}
  = \frac{2}{1 - V_A}.
\]

Cross-multiply:
\[
  1 - V_A \;=\; 2\,V_A - 8\times10^{-4}
  \;\;\Longrightarrow\;\;
  3\,V_A = 1 + 8\times10^{-4}.
\]

Hence
\[
  V_A = \frac{1.0008}{3} \,\text{m}^{3}
       \approx 0.3336~\text{m}^{3}.
\]

\paragraph{Result}
\[
  \boxed{\,V_A^{\ast} \;\approx\; 3.34\times10^{-1}\ \text{m}^{3}\,}
\]
and therefore \(V_B^{\ast} = V_{\text{tot}} - V_A^{\ast} \approx 0.6664\;\text{m}^{3}\).
The tiny excluded volume (\(bN_A \sim 4\times10^{-4}\,\text{m}^{3}\)) causes
only a negligible shift from the naïve one-third / two-thirds split.
\paragraph{Concept.}
For an ideal gas the state equation is
\[
  pV = nRT.
\]
During an \emph{isochoric} process \(V=\text{const}\) (and \(n\) is fixed),
so the pressure is directly proportional to the absolute temperature:
\[
  \frac{p_2}{p_1} = \frac{T_2}{T_1}.
\]
Therefore, to \textbf{increase} the pressure at constant volume we must
\emph{increase the temperature}.

\paragraph{Choice.}
\begin{itemize}
  \item[\(\square\)] Cool down the gas
  \item[\(\boxtimes\)] \textbf{Heat up the gas} \hfill (correct)
  \item[\(\square\)] Push a piston down into the gas \emph{(changes volume)}
  \item[\(\square\)] Pull a piston out, increasing container size
                      \emph{(changes volume)}
\end{itemize}
Heating the gas raises \(T\), which by \(p\propto T\) raises \(p\) while
leaving \(V\) unchanged—exactly the desired isochoric pressure increase.
\paragraph{Entropy change from volume term only (adiabatic, ideal Ar gas)}

\begin{enumerate}
  \item \textbf{Number of moles.}\\
        Initial state: 
        \(p_i = \SI{100}{\kilo\pascal},\;
          V_i = \SI{0.01}{\meter^{3}},\;
          T_i = \SI{300}{\kelvin}\).
        Using \(n = \dfrac{p_i V_i}{RT_i}\):
        \[
          n 
          = \frac{\SI{100e3}{\pascal}\times \SI{0.01}{\meter^{3}}}
                 {\SI{8.314}{\joule\per\mole\per\kelvin}\times \SI{300}{\kelvin}}
          \approx \SI{0.40}{\mole}.
        \]

  \item \textbf{Volume ratio.}\\
        \(V_f = \SI{0.025}{\meter^{3}}\), so
        \[
          \frac{V_f}{V_i} = \frac{0.025}{0.01} = 2.5.
        \]

  \item \textbf{Entropy contribution from volume change.}\\
        For an \(\alpha\)-ideal gas
        \(\displaystyle
          \Delta S_V = nR\ln\!\bigl(V_f/V_i\bigr).
        \)
        Hence
        \[
          \Delta S_V
          = (\SI{0.40}{\mole})
            \,(\SI{8.314}{\joule\per\mole\per\kelvin})
            \,\ln(2.5)
          \approx \boxed{\SI{3.05}{\joule\per\kelvin}}.
        \]
\end{enumerate}
\subsection*{Argon expansion: sanity check}

\paragraph{Given data}
\[
  p_i = \SI{100}{\kilo\pascal},\;
  T_i = \SI{300}{\kelvin},\;
  V_i = \SI{0.010}{\metre^{3}},\;
  V_f = \SI{0.025}{\metre^{3}}.
\]

Argon is monatomic, so \(N_{\text{DOF}} = 3\) and
\[
  \alpha = \frac{N_{\text{DOF}}}{2} = \frac{3}{2}, 
  \qquad
  \gamma = \frac{\alpha+1}{\alpha} = \frac{5}{3}.
\]

%%%%%%%%%%%%%%%%%%%%%%%%%%%%%%%%%%%%%%%%%%%%%%%%%%%%%%%%%%%%%%%%%%%%%%%%%%%%
\paragraph{1.  Moles of gas (needed for \(\Delta S\))}
\[
  n
  = \frac{p_i V_i}{RT_i}
  = \frac{(100\times10^{3})\,(0.010)}
         {8.314\,(300)}
  \simeq 0.401\;\text{mol}.
\]

%%%%%%%%%%%%%%%%%%%%%%%%%%%%%%%%%%%%%%%%%%%%%%%%%%%%%%%%%%%%%%%%%%%%%%%%%%%%
\paragraph{2.  Final temperature \(T_f\) (adiabatic law)}
For a reversible adiabatic process
\(T V^{\gamma-1} = \text{const}\):
\[
  T_f
  = T_i \!\left(\frac{V_i}{V_f}\right)^{\gamma-1}
  = 300 \left(\frac{0.010}{0.025}\right)^{2/3}
  \approx \boxed{\SI{163}{\kelvin}}.
\]

%%%%%%%%%%%%%%%%%%%%%%%%%%%%%%%%%%%%%%%%%%%%%%%%%%%%%%%%%%%%%%%%%%%%%%%%%%%%
\paragraph{3.  Entropy change from the \emph{temperature} term}
\[
  \Delta S_T
  = \alpha\,nR \ln\!\Bigl(\tfrac{T_f}{T_i}\Bigr)
  = (1.5)(0.401)(8.314)\,\ln\!\Bigl(\tfrac{163}{300}\Bigr)
  \approx \boxed{-\,\SI{3.05}{\joule\per\kelvin}}.
\]

%%%%%%%%%%%%%%%%%%%%%%%%%%%%%%%%%%%%%%%%%%%%%%%%%%%%%%%%%%%%%%%%%%%%%%%%%%%%
\paragraph{4.  Entropy change from the \emph{volume} term}
\[
  \Delta S_V
  = nR \ln\!\Bigl(\tfrac{V_f}{V_i}\Bigr)
  = (0.401)(8.314)\,\ln(2.5)
  \approx \boxed{+\SI{3.05}{\joule\per\kelvin}}.
\]

%%%%%%%%%%%%%%%%%%%%%%%%%%%%%%%%%%%%%%%%%%%%%%%%%%%%%%%%%%%%%%%%%%%%%%%%%%%%
\paragraph{5.  Total entropy change}
\[
  \Delta S_{\text{total}}
  = \Delta S_T + \Delta S_V
  \approx -3.05 + 3.05
  \approx \boxed{0\;(\text{within rounding})}.
\]

\noindent
\textbf{Interpretation:} for a slow, perfectly insulating (reversible adiabatic)
expansion, the entropy produced by the volume increase is exactly cancelled
by the entropy decrease from cooling, so the net change in entropy is zero—
consistent with \(S = \text{const}\) for a reversible adiabatic path.
\paragraph{Problem recap}
Two identical samples of \({N2}\) at 
\(T_i = 30^{\circ}\text{C}=303.15\;\text{K}\) and \(p=\SI{1}{atm}\)  
fill containers of volume \(V_i=\SI{0.5}{L}=5.0\times10^{-4}\,\text{m}^3\).

* **Container X** is \emph{rigid} (\(V=\text{const}\)).
* **Container Y** has a frictionless piston so the gas stays at \(p=\text{const}\).

Both gases are warmed by \(\Delta T = 10\;\text{K}\).  
\(Q_X\) and \(Q_Y\) are the heats required; find \(Q_X-Q_Y\).

---------------------------------------------------------------------------
\paragraph{1.  Moles of nitrogen}
\[
  n \;=\; \frac{pV}{RT}
        = \frac{(1.013\times10^{5}\,\text{Pa})(5.0\times10^{-4}\,\text{m}^3)}
               {(8.314\,\text{J mol}^{-1}\,\text{K}^{-1})(303.15\,\text{K})}
        \;\approx\; 0.0201\;\text{mol}.
\]

---------------------------------------------------------------------------
\paragraph{2.  Heat at constant volume vs.\ pressure}
For an ideal gas  
\(Q = nC\,\Delta T\) where \(C=C_V\) (rigid) or \(C=C_P\) (piston).

Hence
\[
  Q_X - Q_Y
  = nC_V\Delta T - nC_P\Delta T
  = n(C_V-C_P)\Delta T
  = -\,nR\,\Delta T
  \quad(\because C_P - C_V = R).
\]

---------------------------------------------------------------------------
\paragraph{3.  Numerical value}
\[
  Q_X - Q_Y
  = -\,(0.0201\;\text{mol})(8.314\;\text{J mol}^{-1}\text{K}^{-1})(10\;\text{K})
  \approx \boxed{-\,1.67\;\text{J}}.
\]
\[
\begin{aligned}
C_P
  &= \left(\frac{\partial H}{\partial T}\right)_P
   = \left(\frac{\partial U}{\partial T}\right)_P
     + \left(\frac{\partial (pV)}{\partial T}\right)_P \\[6pt]
  &= \underbrace{\left(\frac{\partial U}{\partial T}\right)_V}_{C_V}
     + \underbrace{\left(\frac{\partial U}{\partial V}\right)_T
                    \left(\frac{\partial V}{\partial T}\right)_P}_{\;=\,0}
     + \frac{\partial (RT)}{\partial T} \\[6pt]
  &= C_V + R .
\end{aligned}
\]
% ---------------------------------------------------------------------------
%  Detailed step-by-step derivation of  C_P - C_V = R   (one mole, ideal gas)
% ---------------------------------------------------------------------------

\paragraph{1.  Start with the definition of enthalpy}
\[
  H(T,p)\;=\;U(T,V)\;+\;pV.
\]

For an \emph{ideal} gas, the state equation is \(pV = RT\) (one mole) and
the internal energy depends only on temperature: \(U = U(T)\).

\bigskip
\paragraph{2.  Differentiate \(H\) with respect to \(T\) at constant pressure}
\[
  C_P \;\equiv\;
  \left(\frac{\partial H}{\partial T}\right)_P
  \;=\;
  \left(\frac{\partial U}{\partial T}\right)_P
  \;+\;
  \left(\frac{\partial (pV)}{\partial T}\right)_P .
\]

\bigskip
\paragraph{3.  Expand the first term with the chain rule}
\[
  \begin{aligned}
  \left(\frac{\partial U}{\partial T}\right)_P
      &=\left(\frac{\partial U}{\partial T}\right)_V
        +\left(\frac{\partial U}{\partial V}\right)_T
         \underbrace{\left(\frac{\partial V}{\partial T}\right)_P}_{\displaystyle (*)}\\[6pt]
      &=\underbrace{\left(\dfrac{\partial U}{\partial T}\right)_V}_{C_V}\;
        +\;
        0\times(*) .
  \end{aligned}
\]

The last factor vanishes because \(U\) for an ideal gas is *independent of*
\(V\); hence \(\left(\partial U/\partial V\right)_T = 0\).

\bigskip
\paragraph{4.  Evaluate the second term}
\[
  \left(\frac{\partial (pV)}{\partial T}\right)_P
  =\frac{\partial (RT)}{\partial T}
  = R,
\]
because \(pV = RT\) and \(R\) is a constant.

\bigskip
\paragraph{5.  Collect the pieces}
\[
  \boxed{\,C_P
    = C_V + R\,}.
\]

The difference \(C_P - C_V = R\) is therefore the constant extra energy
per kelvin that must be supplied at constant pressure to provide the
expansion work \(p\,dV\) of one mole of ideal gas.
% ---------------------------------------------------------------------------
%  What does the sub-script “\(P\)” on a partial derivative mean?
% ---------------------------------------------------------------------------

When a thermodynamic function depends on several independent variables,
we must specify which ones are held fixed during differentiation.
For example, for enthalpy \(H=H(T,p)\),

\[
  \left(\frac{\partial H}{\partial T}\right)_{\!P}
  \quad\text{means “differentiate }H\text{ with respect to }T
            \text{ \emph{while keeping the pressure } }p \text{ fixed.”}
\]

The sub-script lists every variable that is \emph{held constant}; all
variables not shown are allowed to vary consistently with the state
relation(s).

\bigskip
\paragraph{Ideal-gas illustration (one mole)}
\[
  H(T,p)=U(T)+pV,
  \qquad pV = RT,
  \qquad U=U(T).
\]

\[
\begin{aligned}
\left(\frac{\partial H}{\partial T}\right)_{\!P}
 &= \underbrace{\left(\frac{\partial U}{\partial T}\right)_{P}}_{=\,C_V}
    +\left(\frac{\partial (pV)}{\partial T}\right)_{P} \\[6pt]
 &= C_V + \frac{\partial (RT)}{\partial T}
  = C_V + R,
\end{aligned}
\]

because \(pV=RT\) and \(R\) is constant.  
If instead the volume were held fixed,

\[
  \left(\frac{\partial H}{\partial T}\right)_{V}=C_V,
\]

since \(p\,dV=0\).  The difference arises solely from the different
constraint indicated by the sub-script.
---------------------------------------------------------------------------
\paragraph{Interpretation}
The constant-pressure sample (Y) needs an extra \(nR\Delta T\) of energy to
supply the \(p\Delta V\) work done while expanding, so \(Q_Y\) exceeds \(Q_X\)
by \(\sim1.7\;\text{J}\) for this small system.
\subsection*{“Almost–ideal” gas with excluded volume}

The entropy model is
\[
  S = nR\ln\!\bigl(V - nb\bigr)+\text{const},
\]
so at equilibrium
\[
  \frac{p}{T} \;=\;
  \pdv{S}{V} = \frac{nR}{V-nb}
  \;\Longrightarrow\;
  p\bigl(V-nb\bigr)=nRT.
\]
This is the usual ideal–gas relation with the volume replaced by the
\emph{free} volume \(V-nb\).

%--------------------------------------------------------------------
\paragraph{1.  Find \(T\): \(b=8\times10^{-4}\,\text{m}^{3}\!/\text{mol},
                           n=30.2\;\text{mol},
                           V=1.00\;\text{m}^{3},
                           p=101300\;\text{Pa}\)}

\[
  V-nb
  = 1.00 - (30.2)(8.0\times10^{-4})
  = 1.00 - 0.02416
  = 0.97584\;\text{m}^{3}.
\]

\[
  T
  = \frac{p\bigl(V-nb\bigr)}{nR}
  = \frac{(1.013\times10^{5})(0.97584)}
         {(30.2)(8.314)}
  \approx \boxed{3.94\times10^{2}\;\text{K}}
  \;(\text{about }394\;\text{K}).
\]

%--------------------------------------------------------------------
\paragraph{2.  Find \(V\): \(b=8\times10^{-4}\,\text{m}^{3}\!/\text{mol},
                           n=1.50\;\text{mol},
                           T=350\;\text{K},
                           p=101300\;\text{Pa}\)}

Re-arrange the same equation for \(V\):
\[
  V = nb + \frac{nRT}{p}.
\]

\[
  nb = (1.50)(8.0\times10^{-4}) = 0.00120\;\text{m}^{3},
  \qquad
  \frac{nRT}{p}
  = \frac{(1.50)(8.314)(350)}{1.013\times10^{5}}
  = 0.04305\;\text{m}^{3}.
\]

\[
  V
  = 0.00120 + 0.04305
  \approx \boxed{4.43\times10^{-2}\;\text{m}^{3}}
  \;(\text{about }0.0443\;\text{m}^{3}).
\]

\medskip
\textit{Common slip:} if you neglect the excluded volume \(nb\) in part 2 you
get \(V \approx 0.0431\;\text{m}^{3}\), which is slightly low (your entry
\(0.0432\;\text{m}^{3}\) shows exactly that omission).  Always add the \(nb\)
term for the “almost–ideal” gas.
\subsection*{Excluded–volume gas: pressure before and after isothermal expansion}

The equation of state is
\[
  p\bigl(V - Nb\bigr) = NkT,
\]
where \(b\) is the excluded volume \emph{per molecule}.

\paragraph{Constants and data}
\[
  \begin{aligned}
    n &= 3.00\;\text{mol},
    &
    N &= nN_A = 3(6.022\times10^{23}) = 1.807\times10^{24}\;\text{molecules}, \\[4pt]
    b &= 1.3\times10^{-28}\;\text{m}^{3}\!/\text{molecule},
    &
    k &= 1.380649\times10^{-23}\;\text{J K}^{-1}, \\[4pt]
    T &= 300\;\text{K}.
  \end{aligned}
\]

--------------------------------------------------------------------
\paragraph{1. Initial pressure \((V_i = 0.001\,\text{m}^{3})\)}

\[
  V_i - Nb \;=\;
  0.001 - (1.807\times10^{24})(1.3\times10^{-28})
  = 7.65\times10^{-4}\;\text{m}^{3}.
\]

\[
  p_i
  = \frac{NkT}{V_i - Nb}
  = \frac{(1.807\times10^{24})(1.380649\times10^{-23})(300)}
         {7.65\times10^{-4}}
  \approx \boxed{9.8\times10^{6}\;\text{Pa}}.
\]

--------------------------------------------------------------------
\paragraph{2. Final pressure \((V_f = 0.002\,\text{m}^{3})\)}

\[
  V_f - Nb
  = 0.002 - 2.35\times10^{-4}
  = 1.77\times10^{-3}\;\text{m}^{3}.
\]

\[
  p_f
  = \frac{NkT}{V_f - Nb}
  = \frac{(1.807\times10^{24})(1.380649\times10^{-23})(300)}
         {1.77\times10^{-3}}
  \approx \boxed{4.2\times10^{6}\;\text{Pa}}.
\]

--------------------------------------------------------------------
\paragraph{Check}  
Because the expansion is isothermal, \(NkT\) is constant, so doubling the
container volume more than halves the pressure; the excluded volume term
\(Nb\) reduces the available space in both cases, giving
\(p_f/p_i \approx (V_i-Nb)/(V_f-Nb) \approx 0.43\), consistent with the
numerical results.
\paragraph{Work done during the isothermal expansion}

For the excluded-volume gas the equation of state is  
\(p(V-Nb)=NkT\), so along an isotherm

\[
  p \;=\; \frac{NkT}{V-Nb}.
\]

The reversible work is  

\[
  W = \int_{V_i}^{V_f} p \,\dd V
      = NkT\int_{V_i}^{V_f} \frac{\dd V}{V-Nb}
      = NkT \,\ln\!\Bigl(\frac{V_f-Nb}{V_i-Nb}\Bigr).
\]

\[
  \begin{aligned}
    N &= nN_A = 3(6.022\times10^{23})
         = 1.807\times10^{24}\;\text{molecules},\\[4pt]
    Nb &= (1.807\times10^{24})(1.3\times10^{-28})
          = 2.35\times10^{-4}\;\text{m}^{3},\\[6pt]
    V_i-Nb &= 0.00100-0.000235 = 7.65\times10^{-4}\;\text{m}^{3},\\
    V_f-Nb &= 0.00200-0.000235 = 1.77\times10^{-3}\;\text{m}^{3}.
  \end{aligned}
\]

\[
  NkT = (1.807\times10^{24})(1.380\,649\times10^{-23})(300)
       = 2.50\times10^{4}\;\text{J}.
\]

\[
  W
  = (2.50\times10^{4})\,
    \ln\!\Bigl(\tfrac{1.77\times10^{-3}}{7.65\times10^{-4}}\Bigr)
  \approx \boxed{6.3\times10^{3}\;\text{J}}.
\]

\[
  W \;\approx\; 6.26 \times 10^{3}\,\text{J} \quad(\text{consistent with }6.27\text{e3}).
\]
\paragraph{Answer.}
The \(\mathbf{1{,}600^{\circ}\!\text{C}}\) design would have the higher
\emph{theoretical maximum} efficiency.

\paragraph{Why?}
For any heat engine the Carnot limit is
\[
  \eta_{\text{max}}
  = 1 - \frac{T_C}{T_H},
\]
where \(T_H\) is the hot‐reservoir temperature and \(T_C\) the cold‐reservoir
temperature (both in kelvin).  
With \(T_C\) fixed, a larger \(T_H\) makes the fraction
\(T_C/T_H\) smaller, hence \(1-T_C/T_H\) larger.  
Therefore raising the furnace temperature from \(1{,}400^{\circ}\text{C}\)
to \(1{,}600^{\circ}\text{C}\) increases the maximum possible efficiency.
\subsection*{Isothermal compression of an ideal gas}

Given  
\(n = 2.5\ \text{mol},\;
 T = 310\ \text{K},\;
 V_i = 0.500\ \text{m}^3\).

\paragraph{1. Initial pressure}
\[
  p_i = \frac{nRT}{V_i}
       = \frac{(2.5)(8.314\ \text{J mol}^{-1}\text{K}^{-1})(310\ \text{K})}
              {0.500\ \text{m}^{3}}
       \approx \boxed{1.29\times10^{4}\ \text{Pa}}.
\]

\paragraph{2. Final volume at \(p_f = 1.546404\times10^{4}\ \text{Pa}\) (isothermal)}
\[
  V_f = \frac{nRT}{p_f}
       = \frac{(2.5)(8.314)(310)}
              {1.546404\times10^{4}}
       \approx \boxed{4.17\times10^{-1}\ \text{m}^{3}}.
\]

\paragraph{3. Work \emph{done on} the gas}
For a reversible isothermal change,
\[
  W_{\text{by}} = nRT\ln\!\Bigl(\frac{V_f}{V_i}\Bigr),
  \quad
  W_{\text{on}} = -\,W_{\text{by}}.
\]

\[
  W_{\text{by}}
  = (2.5)(8.314)(310)\,
    \ln\!\Bigl(\tfrac{0.4167}{0.5000}\Bigr)
  = -1.17\times10^{3}\ \text{J},
\]

\[
  \boxed{W_{\text{on}} \;\approx\; +1.17\times10^{3}\ \text{J}}.
\]

(The positive sign indicates work is done \emph{on} the gas during compression.)
%---------------------------------------------------------------------------
\paragraph{Reservoir temperatures (convert to kelvin)}
\[
  T_H = 60^{\circ}\text{C} + 273.15 = 333.15\;\text{K},
  \qquad
  T_C = -15^{\circ}\text{C} + 273.15 = 258.15\;\text{K}.
\]

%---------------------------------------------------------------------------
\paragraph{Carnot relations}
For a reversible Carnot engine
\[
  \frac{Q_C}{Q_H} = \frac{T_C}{T_H},
  \qquad
  \eta_{\max} = 1 - \frac{T_C}{T_H}.
\]

Given \(Q_H = 600\;\text{J}\):

%---------------------------------------------------------------------------
\begin{enumerate}
  \item \textbf{Heat expelled to the cold reservoir}
        \[
          Q_C
          = Q_H \frac{T_C}{T_H}
          = (600)\frac{258.15}{333.15}
          \approx \boxed{4.65\times10^{2}\ \text{J}}.
        \]

  \item \textbf{Work done by the engine in 27 min}
        \[
          W_{\text{by}}
          = Q_H - Q_C
          = 600 - 464.9
          \approx \boxed{1.35\times10^{2}\ \text{J}}.
        \]

  \item \textbf{Maximum (Carnot) efficiency}
        \[
          \eta_{\max}
          = 1 - \frac{T_C}{T_H}
          = 1 - \frac{258.15}{333.15}
          \approx \boxed{0.225}\;(\text{or }22.5\%).
        \]
\end{enumerate}

\medskip
\noindent
These represent the \emph{best-case} limits; any real engine will be less efficient and expel more than \(4.65\times10^{2}\,\text{J}\) of heat for every \(600\,\text{J}\) it absorbs from the \(60^{\circ}\text{C}\) reservoir.
\paragraph{Step-by-step reasoning}

\begin{enumerate}
  \item \textbf{Convert reservoir temperatures to kelvin.}
        \[
          T_H = 400^{\circ}\text{C}+273.15 = 673.15\;\text{K},
          \qquad
          T_C =  30^{\circ}\text{C}+273.15 = 303.15\;\text{K}.
        \]

  \item \textbf{Ideal Carnot efficiency (no leak).}
        \[
          \eta_{\text{Carnot}}
          = 1 - \frac{T_C}{T_H}
          = 1 - \frac{303.15}{673.15}
          \approx 0.550.
        \]

  \item \textbf{Relate work to heat entering the engine.}\\
        With \(\eta_{\text{Carnot}}=0.550\),
        \[
          W_{\text{by}} = \eta_{\text{Carnot}}\;Q_{H1}
                        = 0.550\,Q_{H1}.
        \]

  \item \textbf{Overall efficiency (including the leak) is given as}
        \(\eta_{\text{overall}} = 0.25\):
        \[
          \eta_{\text{overall}}
          = \frac{W_{\text{by}}}{Q_{H1}+Q_{H2}}
          = 0.25.
        \]
        Substitute \(W_{\text{by}}\) from step 3:
        \[
          0.25\,(Q_{H1}+Q_{H2}) = 0.550\,Q_{H1}.
        \]

  \item \textbf{Solve for the ratio \(Q_{H1}/Q_{H2}\).}
        \[
          0.25Q_{H1} + 0.25Q_{H2} = 0.550Q_{H1}
          \quad\Longrightarrow\quad
          0.25Q_{H2} = 0.300Q_{H1}
          \quad\Longrightarrow\quad
          \frac{Q_{H1}}{Q_{H2}} = \frac{0.25}{0.30}
          = \frac{5}{6}
          \approx 0.83.
        \]
\end{enumerate}

\[
  \boxed{\dfrac{Q_{H1}}{Q_{H2}} \;\approx\; 0.83} \qquad
  (\text{i.e.\ the leak carries about }20\%\text{ more heat than the engine.})
\]
%--------------------------------------------------------------------
\subsection*{Reversible refrigerator between $60^{\circ}$C and $-15^{\circ}$C}

\begin{enumerate}
  \item \textbf{Convert reservoir temperatures to kelvin.}
        \[
          T_H = 60^{\circ}\mathrm{C} + 273.15 = 333.15\;\mathrm{K},
          \qquad
          T_C = -15^{\circ}\mathrm{C} + 273.15 = 258.15\;\mathrm{K}.
        \]

  \item \textbf{Carnot coefficient of performance (COP).}
        \[
          \text{COP}_{\text{Carnot}}
          = \frac{Q_C}{W_{\text{on}}}
          = \frac{1}{T_H/T_C - 1}
          = \frac{1}{333.15/258.15 - 1}
          \approx 3.44.
        \]

  \item \textbf{Heat extracted from the cold reservoir.}\\
        Work input is given as \(W_{\text{on}} = 1100\ \text{J}\):
        \[
          Q_C
          = \text{COP}_{\text{Carnot}}\; W_{\text{on}}
          = 3.44 \times 1100\ \text{J}
          \;\approx\;
          \boxed{3.78 \times 10^{3}\ \text{J}}.
        \]

  \item \textbf{Heat expelled to the hot reservoir.}
        \[
          Q_H
          = Q_C + W_{\text{on}}
          \approx 3.78 \times 10^{3} + 1.10 \times 10^{3}
          \;\approx\;
          \boxed{4.88 \times 10^{3}\ \text{J}}.
        \]

  \item \textbf{Coefficient of performance (already obtained).}
        \[
          K \equiv \text{COP}_{\text{Carnot}}
          \;\approx\;
          \boxed{3.44}.
        \]
\end{enumerate}
\begin{problem}
  Suppose two rigid containers, $A$ and $B$, each hold the \emph{same} non-ideal gas at a common temperature $T$ and an externally fixed pressure $P$.
  After we open a valve that allows particle exchange but \emph{not} heat or work transfer, we observe a net flow of particles from container $B$ to container $A$.
  Determine the relationship between the chemical potentials $\mu_A$ and $\mu_B$ of the gas in the two containers \emph{at the instant the valve is opened}.  
  \end{problem}
  
  \begin{solution}
  \begin{enumerate}
      \item \textbf{Equilibrium condition.}  
            For systems that can exchange particles at fixed $T$ and $P$, thermodynamic equilibrium requires the Gibbs free energy to be minimized.  
            Since $G = N\mu$ for a single species, equilibrium demands
            \[
                \mu_A \;=\; \mu_B.
            \]
      \item \textbf{Direction of spontaneous particle flow.}  
            If the chemical potentials initially differ, particles spontaneously flow from the region of \emph{higher} chemical potential to the region of \emph{lower} chemical potential in order to decrease the total Gibbs free energy.
            Hence the observed flow $B \to A$ implies
            \[
                \mu_B \;>\; \mu_A.
            \]
      \item \textbf{Conclusion.}  
            Therefore, at the moment we open the valve and witness particles moving from $B$ to $A$, the inequality
            \[
                \boxed{\mu_A \;<\; \mu_B}
            \]
            must hold.  
            (Once enough particles have transferred so that $\mu_A=\mu_B$, the net flow will cease and equilibrium is reached.)
  \end{enumerate}
  \end{solution}
\subsection*{Ideal (and almost-ideal) gas in a $T,p,N$ ensemble}

%%%%%%%%%%%%%%%%%%%%%%%%%%%%%%%%%%%%%%%%%%%%%%%%%%%%%%%%%%%%%%%%%%%%
\paragraph{1.  Derivative of $G$ with respect to $V$ (fixed $T,p,N$)}

For one mole of an \emph{ideal} gas  
\(G = U + pV - TS\) with \(U=U(T)\) only and  
\(S = Nk\ln V + \text{const}\).

\[
  \pdv{G}{V}_{T,p,N}
  = p - T\!\left(\pdv{S}{V}\right)_{T,N}
  = p \;-\; T\,\frac{Nk}{V}
  = p - \frac{nRT}{V},
\]
where we used \(Nk = nR\).

---
\begin{problem}
  One mole of an \emph{ideal} gas is in equilibrium at a fixed temperature $T$ (e.g.\ 300 K) and a fixed pressure $p$ (e.g.\ $10^{6}\,\text{Pa}$).
  At those conditions, evaluate the derivative of the Gibbs free energy with respect to volume:
  \[
  \left.\frac{\partial G}{\partial V}\right|_{T,p,N}\;=\;\;?
  \]
  (Report the result in $\text{J}\,\text{m}^{-3}$.)
  \end{problem}
  
  \begin{solution}
  \begin{enumerate}
      \item \textbf{Thermodynamic identity for $G$.}  
            For a simple, \emph{single‐component} system,
            \[
                \boxed{\;\mathrm{d}G \;=\; -S\,\mathrm{d}T \;+\; V\,\mathrm{d}p \;+\; \mu\,\mathrm{d}N\;}
            \]
            where $S$ is entropy, $V$ the volume, $\mu$ the chemical potential, and $N$ the amount of substance.
      \item \textbf{Hold $T$, $p$, and $N$ fixed.}  
            Under the stated conditions $\mathrm{d}T=\mathrm{d}p=\mathrm{d}N=0$, so the total differential reduces to
            \[
                \mathrm{d}G\;=\;0.
            \]
            Hence $G$ is \emph{constant} as long as $T$, $p$, and $N$ are fixed.
      \item \textbf{Derivative with respect to $V$.}  
            Because $G$ is a function only of $(T,p,N)$, it has \emph{no explicit dependence on $V$} when those variables are treated as the independent set.  Therefore
            \[
                \boxed{\;\left.\frac{\partial G}{\partial V}\right|_{T,p,N}\;=\;0\;}
                \qquad\bigl(\text{units: } \text{J}\,\text{m}^{-3}\bigr).
            \]
            In words, once $T$ and $p$ are fixed, varying $V$ is not an independent operation for an ideal gas (because $pV = NRT$ already constrains $V$); consequently the partial derivative vanishes.
  \end{enumerate}
  \end{solution}

%%%%%%%%%%%%%%%%%%%%%%%%%%%%%%%%%%%%%%%%%%%%%%%%%%%%%%%%%%%%%%%%%%%%
\paragraph{2.  Equilibrium volume for $T=300$ K and $p=10^{6}$ Pa}

Setting the derivative to zero gives the ideal-gas relation

\[
  p = \frac{nRT}{V}
  \quad\Longrightarrow\quad
  V_{\text{eq}} = \frac{nRT}{p}.
\]

For one mole:

\[
  V_{\text{eq}}
  = \frac{(1.00\;\text{mol})(8.314\;\text{J mol}^{-1}\text{K}^{-1})(300\;\text{K})}
         {1.0\times10^{6}\;\text{Pa}}
  \approx \boxed{2.49\times10^{-3}\;\text{m}^{3}}.
\]

---

%%%%%%%%%%%%%%%%%%%%%%%%%%%%%%%%%%%%%%%%%%%%%%%%%%%%%%%%%%%%%%%%%%%%
\paragraph{3.  Gas with an excluded volume $b$}

If the entropy’s volume term is \(S = Nk\ln(V - Nb)\) 
(\(b = 1.1\times10^{-28}\,\text{m}^{3}\) per molecule), then

\[
  \pdv{G}{V}_{T,p,N}
  = p - \frac{nRT}{V - Nb}.
\]

Equilibrium (\(\pdv*{G}{V}=0\)) now requires

\[
  p = \frac{nRT}{V - Nb}
  \quad\Longrightarrow\quad
  \boxed{\,V = \frac{nRT}{p} + N b\,}.
\]

Numerically (one mole, same $T$ and $p$):

\[
  N b = N_A b = (6.022\times10^{23})(1.1\times10^{-28})
       = 6.62\times10^{-5}\;\text{m}^{3},
\]
\[
  V_{\text{eq}}
  = 2.49\times10^{-3} + 6.62\times10^{-5}
  \approx \boxed{2.56\times10^{-3}\;\text{m}^{3}}.
\]

\medskip
\noindent
\textit{Keep the algebraic result}
\[
  V = \frac{nRT}{p} + N b
\]
for later exercises.
\[
  \left( \pdv{S(U,V)}{V} \right)_{U}
  \;=\;
  \frac{p}{T}.
\]

\textbf{Derivation (one line).}\;
With $dU=0$ (holding $U$ fixed), the fundamental relation
$T\,dS = dU + p\,dV$ simplifies to
$T\,dS = p\,dV$, so $dS/dV = p/T$ at constant $U$.

\begin{align*}
  \textbf{Given:}\qquad 
  & S=S(U,V)\quad\text{(fixed particle number)} \\[4pt]
  &\displaystyle dS \;=\;
    \left(\frac{\partial S}{\partial U}\right)_{V}\! dU
    +\left(\frac{\partial S}{\partial V}\right)_{U}\! dV
    \tag{1}
  \end{align*}
  
  \begin{enumerate}
  \item \textbf{Use the Fundamental Relation of Thermodynamics} (equilibrium):
    \[
      TdS \;=\; dU + p\,dV .
    \]
  
  \item \textbf{Solve for \(dS\)} by dividing by \(T\):
    \[
      dS \;=\; \frac{1}{T}\,dU + \frac{p}{T}\,dV .
      \tag{2}
    \]
  
  \item \textbf{Match coefficients of \(dU\) and \(dV\)} between (1) and (2):
    \[
      \left(\frac{\partial S}{\partial U}\right)_{V} = \frac{1}{T},
      \qquad
      \left(\frac{\partial S}{\partial V}\right)_{U} = \frac{p}{T}.
    \]
  
  \item \textbf{Answer}:
    \[
      \boxed{\displaystyle\left(\frac{\partial S}{\partial V}\right)_{U} = \frac{p}{T}}.
    \]
  \end{enumerate}
  Below is a complete, copy-paste-ready LaTeX walkthrough for part 2.

\begin{align*}
\textbf{Objective:}\qquad 
\left(\frac{\partial U}{\partial S}\right)_V \;=\; ?
\end{align*}

\begin{enumerate}
\item \textbf{Treat \(U\) as a function of \(S\) and \(V\).}  
      In equilibrium we may write \(U = U(S,V)\).  
      Its total differential is
      \[
          dU
          \;=\;
          \left(\frac{\partial U}{\partial S}\right)_V dS
          +\left(\frac{\partial U}{\partial V}\right)_S dV .
          \tag{1}
      \]

\item \textbf{Insert the Fundamental Relation of Thermodynamics}  
      (valid in equilibrium):
      \[
          TdS = dU + p\,dV
          \;\;\Longrightarrow\;\;
          dU = T\,dS - p\,dV .
          \tag{2}
      \]

\item \textbf{Compare coefficients of differentials in (1) and (2).}  
      Setting \(dV=0\) (i.e.\ taking the partial derivative at constant \(V\)),
      \[
          \left(\frac{\partial U}{\partial S}\right)_V dS \;=\; T\,dS
          \quad\Longrightarrow\quad
          \boxed{\displaystyle
          \left(\frac{\partial U}{\partial S}\right)_V = T }.
      \]
      (For completeness, the \(dV\)-terms show 
      \(\bigl(\frac{\partial U}{\partial V}\bigr)_S = -p\).)
\end{enumerate}
\begin{align*}
  \textbf{Objective:}\qquad 
  \left(\frac{\partial U}{\partial V}\right)_S \;=\; ?
  \end{align*}
  
  \begin{enumerate}
  \item \textbf{Regard \(U\) as a function of \((S,V)\):}
        \[
          dU
          \;=\;
          \left(\frac{\partial U}{\partial S}\right)_V dS
          +\left(\frac{\partial U}{\partial V}\right)_S dV .
          \tag{1}
        \]
  
  \item \textbf{Use the Fundamental Relation of Thermodynamics (equilibrium):}
        \[
          TdS \;=\; dU + p\,dV
          \;\;\Longrightarrow\;\;
          dU = T\,dS - p\,dV .
          \tag{2}
        \]
  
  \item \textbf{Hold \(S\) constant (\(dS=0\)).}  
        From (2) this gives
        \[
          dU\big|_{dS=0} \;=\; -p\,dV .
        \]
  
  \item \textbf{Identify the coefficient of \(dV\).}  
        Comparing with (1) when \(dS=0\):
        \[
          \left(\frac{\partial U}{\partial V}\right)_S dV \;=\; -p\,dV
          \quad\Longrightarrow\quad
          \boxed{\displaystyle
          \left(\frac{\partial U}{\partial V}\right)_S = -\,p }.
        \]
  \end{enumerate}
  \begin{align*}
    \textbf{Objective:}\qquad
    \left(\frac{\partial F}{\partial V}\right)_T
    \end{align*}
    
    \begin{enumerate}
    \item \textbf{Define the Helmholtz free energy.}
          \[
            F(T,V) \;=\; U - T\,S .
          \]
    
    \item \textbf{Take the total differential of \(F\).}
          \[
            dF
            \;=\;
            dU - T\,dS - S\,dT .
            \tag{1}
          \]
    
    \item \textbf{Insert the Fundamental Relation of Thermodynamics (equilibrium):}
          \[
            T\,dS \;=\; dU + p\,dV
            \;\;\Longrightarrow\;\;
            dU \;=\; T\,dS - p\,dV .
            \tag{2}
          \]
    
    \item \textbf{Substitute (2) into (1).}
          \[
            dF
            \;=\;
            \bigl(T\,dS - p\,dV\bigr) - T\,dS - S\,dT
            \;=\;
            -\,p\,dV - S\,dT .
            \tag{3}
          \]
    
    \item \textbf{Hold temperature \(T\) constant (\(dT = 0\)).}
          Under this condition, (3) reduces to
          \[
            dF\big|_{dT=0} \;=\; -\,p\,dV .
          \]
    
    \item \textbf{Identify the coefficient of \(dV\).}
          Comparing \(dF = \bigl(\partial F/\partial V\bigr)_T\,dV\) with the result above,
          \[
            \boxed{\displaystyle
              \left(\frac{\partial F}{\partial V}\right)_T = -\,p }.
          \]
    \end{enumerate}
    \begin{align*}
      \textbf{Objective:}\qquad 
      \left(\frac{\partial F}{\partial T}\right)_V
      \end{align*}
      
      \begin{enumerate}
      \item \textbf{Helmholtz free energy definition.}
            \[
              F(T,V) \;=\; U - T\,S .
            \]
      
      \item \textbf{Take the total differential of \(F\).}
            \[
              dF
              \;=\;
              dU - T\,dS - S\,dT .
              \tag{1}
            \]
      
      \item \textbf{Insert the Fundamental Relation of Thermodynamics (equilibrium).}
            \[
              T\,dS = dU + p\,dV
              \;\;\Longrightarrow\;\;
              dU = T\,dS - p\,dV .
              \tag{2}
            \]
      
      \item \textbf{Substitute (2) into (1).}
            \[
              dF
              \;=\;
              (T\,dS - p\,dV) - T\,dS - S\,dT
              \;=\;
              -p\,dV - S\,dT .
              \tag{3}
            \]
      
      \item \textbf{Hold the volume \(V\) constant (\(dV = 0\)).}
            Under this condition, (3) becomes
            \[
              dF\big|_{dV = 0} \;=\; -\,S\,dT .
            \]
      
      \item \textbf{Identify the coefficient of \(dT\).}
            Comparing \(dF = (\partial F/\partial T)_V\,dT\) with the above result,
            \[
              \boxed{\displaystyle
                \left(\frac{\partial F}{\partial T}\right)_V = -\,S } .
            \]
      \end{enumerate}
      \begin{align*}
        \textbf{Objective:}\qquad
        \left(\frac{\partial G}{\partial T}\right)_p
        \end{align*}
        
        \begin{enumerate}
        \item \textbf{Define the Gibbs free energy.}
              \[
                G(T,p)\;=\; U \;-\; T\,S \;+\; p\,V .
              \]
        
        \item \textbf{Take the total differential of \(G\).}
              \[
                dG
                \;=\;
                dU \;-\; T\,dS \;-\; S\,dT \;+\; p\,dV \;+\; V\,dp .
                \tag{1}
              \]
        
        \item \textbf{Insert the Fundamental Relation of Thermodynamics (equilibrium).}
              \[
                T\,dS \;=\; dU \;+\; p\,dV
                \;\;\Longrightarrow\;\;
                dU = T\,dS \;-\; p\,dV .
                \tag{2}
              \]
        
        \item \textbf{Substitute (2) into (1).}
              \[
                dG
                \;=\;
                \bigl(T\,dS - p\,dV\bigr)
                - T\,dS
                - S\,dT
                + p\,dV
                + V\,dp
                \;=\;
                -\,S\,dT
                + V\,dp .
                \tag{3}
              \]
        
        \item \textbf{Hold the pressure \(p\) constant (\(dp = 0\)).}
              Under this condition, (3) reduces to
              \[
                dG\big|_{dp = 0} \;=\; -\,S\,dT .
              \]
        
        \item \textbf{Identify the coefficient of \(dT\).}
              Comparing \(dG = \bigl(\partial G/\partial T\bigr)_p\,dT\) with the result above,
              \[
                \boxed{\displaystyle
                  \left(\frac{\partial G}{\partial T}\right)_p = -\,S } .
              \]
        \end{enumerate}
        \begin{align*}
          \textbf{Objective:}\qquad
          \left(\frac{\partial G}{\partial p}\right)_T
          \end{align*}
          
          \begin{enumerate}
          \item \textbf{Gibbs free energy definition.}
                \[
                  G(T,p)\;=\; U \;-\; T\,S \;+\; p\,V .
                \]
          
          \item \textbf{Take the total differential of \(G\).}
                \[
                  dG
                  \;=\;
                  dU \;-\; T\,dS \;-\; S\,dT \;+\; p\,dV \;+\; V\,dp .
                  \tag{1}
                \]
          
          \item \textbf{Insert the Fundamental Relation of Thermodynamics (equilibrium).}
                \[
                  T\,dS \;=\; dU \;+\; p\,dV
                  \;\;\Longrightarrow\;\;
                  dU = T\,dS \;-\; p\,dV .
                  \tag{2}
                \]
          
          \item \textbf{Substitute (2) into (1).}
                \[
                  dG
                  \;=\;
                  \bigl(T\,dS - p\,dV\bigr)
                  - T\,dS
                  - S\,dT
                  + p\,dV
                  + V\,dp
                  \;=\;
                  -\,S\,dT
                  + V\,dp .
                  \tag{3}
                \]
          
          \item \textbf{Hold the temperature \(T\) constant (\(dT = 0\)).}
                With \(dT = 0\), equation (3) simplifies to
                \[
                  dG\big|_{dT=0} \;=\; V\,dp .
                \]
          
          \item \textbf{Identify the coefficient of \(dp\).}
                Comparing \(dG = \bigl(\partial G/\partial p\bigr)_T\,dp\) with the result above,
                \[
                  \boxed{\displaystyle
                    \left(\frac{\partial G}{\partial p}\right)_T = V } .
                \]
          \end{enumerate}
          Below is a self-contained LaTeX walkthrough you can copy straight into your editor and compile.

\begin{align*}
\textbf{Data:}\;
&n = 1\;\text{mol},\;
T = 300\;\text{K},\;
p_1 = 4.0 \times 10^{5}\;\text{Pa},\;
p_2 = 2.0 \times 10^{5}\;\text{Pa}. \\[6pt]
\textbf{Goal:}\;
&\Delta G \;\;\text{for an \emph{isothermal} ideal-gas expansion ( } T=\text{const}\text{).}
\end{align*}

\begin{enumerate}
\item \textbf{Start from the thermodynamic definition of \(G\):}
  \[
     G \;=\; U - TS + pV .
  \]

  Hence for any change between states 1 and 2,
  \[
     \Delta G \;=\; \Delta U \;-\; T\,\Delta S \;+\; \Delta(pV).
  \]

\item \textbf{Internal energy change.}  
      For an ideal gas \(U(T)\) depends only on \(T\); here \(T\) is constant, so
      \[
        \boxed{\Delta U = 0 } .
      \]

\item \textbf{Entropy change for an isothermal ideal-gas expansion.}  
      Using \(pV = nRT\) and \(S = nR\ln V + \text{const}\),
      \[
         \Delta S
         \;=\;
         nR\ln\!\frac{V_2}{V_1}
         \;=\;
         nR\ln\!\frac{p_1}{p_2},
         \quad
         \bigl(V\propto 1/p \text{ at fixed } T\bigr).
      \]
      With \(p_1/p_2 = 4.0\times10^{5}/2.0\times10^{5}=2\),
      \[
         \Delta S
         \;=\;
         (1\;\text{mol})(8.314\;\tfrac{\text{J}}{\text{mol\,K}})\,
         \ln 2
         \;=\;
         5.76\;\text{J\,K}^{-1}.
      \]

\item \textbf{Change in \(pV\).}  
      Because \(pV = nRT\) and both \(n\) and \(T\) are constant,
      \[
        p_1V_1 = p_2V_2 = nRT
        \;\;\Longrightarrow\;\;
        \boxed{\Delta(pV)=0 } .
      \]

\item \textbf{Assemble the pieces.}
      \[
        \Delta G
        \;=\;
        0
        \;-\;
        T\,\Delta S
        \;+\;
        0
        \;=\;
        -\,T\,\Delta S
        \;=\;
        -\,(300\;\text{K})(5.76\;\text{J\,K}^{-1})
        \;=\;
        -1.73\times10^{3}\;\text{J}.
      \]

\item \textbf{Result.}
      \[
        \boxed{\displaystyle \Delta G = -1.73 \times 10^{3}\ \text{J}}
      \]
      The Gibbs free energy decreases by \(1.73\;\text{kJ}\) during the
      isothermal expansion.
\end{enumerate}
\end{document}
