\documentclass[12pt]{article}

% Packages
\usepackage[margin=1in]{geometry}
\usepackage{amsmath,amssymb,amsthm}
\usepackage{enumitem}
\usepackage{hyperref}
\usepackage{xcolor}
\usepackage{import}
\usepackage{xifthen}
\usepackage{pdfpages}
\usepackage{transparent}
\usepackage{listings}
\usepackage{tikz}
\usepackage{physics}
\usepackage{siunitx}
\usepackage{booktabs}
\usepackage{cancel}
  \usetikzlibrary{calc,patterns,arrows.meta,decorations.markings}


\DeclareMathOperator{\Log}{Log}
\DeclareMathOperator{\Arg}{Arg}

\lstset{
    breaklines=true,         % Enable line wrapping
    breakatwhitespace=false, % Wrap lines even if there's no whitespace
    basicstyle=\ttfamily,    % Use monospaced font
    frame=single,            % Add a frame around the code
    columns=fullflexible,    % Better handling of variable-width fonts
}

\newcommand{\incfig}[1]{%
    \def\svgwidth{\columnwidth}
    \import{./Figures/}{#1.pdf_tex}
}
\theoremstyle{definition} % This style uses normal (non-italicized) text
\newtheorem{solution}{Solution}
\newtheorem{proposition}{Proposition}
\newtheorem{problem}{Problem}
\newtheorem{lemma}{Lemma}
\newtheorem{theorem}{Theorem}
\newtheorem{remark}{Remark}
\newtheorem{note}{Note}
\newtheorem{definition}{Definition}
\newtheorem{example}{Example}
\newtheorem{corollary}{Corollary}
\theoremstyle{plain} % Restore the default style for other theorem environments
%

% Theorem-like environments
% Title information
\title{}
\author{Jerich Lee}
\date{\today}

\begin{document}

\maketitle

%-----------------------------------------------------------------
%  Chemical–potential criterion for phase equilibrium
%-----------------------------------------------------------------

\textbf{Given at \(T=300\;\text{K},\;p=1\;\text{atm}\):}
\[
N_L^{\,(0)} = 5\;\text{mol},\qquad
N_S^{\,(0)} = 2\;\text{mol},
\]
\[
\mu_L = 5.0\times10^{-20}\;\text{J},\qquad
\mu_S = 9.0\times10^{-21}\;\text{J}.
\]

\[
\mu_S < \mu_L
\quad\Longrightarrow\quad
\text{solid phase is thermodynamically preferred.}
\]

\bigskip
\textbf{Total Gibbs free energy} of the two‑phase system:
\[
G = N_L\mu_L + N_S\mu_S,
\qquad N_L+N_S = N_{\text{tot}} = 7\;\text{mol}.
\]

Because \(\mu_S\) is a strict minimum, \(G\) decreases whenever a mole
of liquid converts to solid:
\[
\Delta G = \mu_S - \mu_L < 0.
\]

Hence the global minimum of \(G\) is reached when
\[
N_S = N_{\text{tot}},\qquad N_L = 0.
\]

\bigskip
\[
\boxed{N_S^{\;(\text{eq})} = 7\ \text{moles}}
\]

(All material freezes, corresponding to answer \textbf{(c)}.)
%---------------------------------------------------------------
%  Work done by a monatomic ideal gas when heated at constant \(n\)
%---------------------------------------------------------------

\textbf{Data}

\[
n = 9\;\text{mol},\quad
T_i = 250\;\text{K},\quad
T_f = 280\;\text{K},\quad
Q = 3523\;\text{J}.
\]

\[
\Delta T = T_f - T_i = 30\;\text{K}.
\]

\textbf{1.\ Change in internal energy \(\Delta U\)}

For a monatomic ideal gas (only translational d.o.f.):
\[
\Delta U = \frac{3}{2}\,nR\,\Delta T
         = \frac32(9)(8.314)(30)\,\text{J}
         \approx 3.37\times10^{3}\;\text{J}.
\]

\textbf{2.\ First law to get the work done \emph{by} the gas}

\[
Q = \Delta U + W
\quad\Longrightarrow\quad
W = Q - \Delta U
  = 3523\;\text{J} - 3.37\times10^{3}\;\text{J}
  \approx 1.56\times10^{2}\;\text{J}.
\]

\[
\boxed{W \;\approx\; 1.56\times10^{2}\ \text{J}}
\]

\bigskip
Therefore the work done \emph{by} the gas during the expansion corresponds to answer **(e)**.
%------------------------------------------------------------------
%  Problem 22 – \(C_p-C_v\) for a non‑ideal monatomic gas
%------------------------------------------------------------------
\textbf{Equation of state}
\[
 (P+a)\,V = nRT, \qquad  a = 0.10\;\text{atm}\;(=\text{constant}).
\]

\textbf{Useful identity (always true)}
\[
\boxed{\;
  C_p - C_v
  = T\left(\frac{\partial P}{\partial T}\right)_{V}
      \left(\frac{\partial V}{\partial T}\right)_{P}
\;}
\tag{1}
\]

---

1  Compute the derivatives in (1)

1.  At fixed \(V\):
    \[
      \left(\frac{\partial P}{\partial T}\right)_{V}
      = \frac{nR}{V}.
    \]

2.  Solve the EOS for \(V\) at fixed \(P\):
    \[
      V = \frac{nRT}{P+a}
      \;\;\Longrightarrow\;\;
      \left(\frac{\partial V}{\partial T}\right)_{P}
      = \frac{nR}{P+a}
      = \frac{V}{T}.
    \]

---

2  Insert into (1)

\[
C_p - C_v
  = T\left(\frac{nR}{V}\right)\!\left(\frac{V}{T}\right)
  = \boxed{\,nR\,}.
\]

Remarkably, the constant pressure shift \(+a\) cancels; the
difference \(C_p-C_v\) is the same as for an ideal gas.

---

3  Numeric value for \(n = 7\) mol

\[
C_p - C_v = nR = (7)(8.314\ \text{J\,mol}^{-1}\text{K}^{-1})
           \approx 5.82\times10^{1}\ \text{J\,K}^{-1}.
\]

---

\[
\boxed{C_p - C_v \;\approx\; 5.82 \times 10^{1}\ \text{J K}^{-1}}
\]

This matches option **(a)** in the list.
%-----------------------------------------------------------------
%  Entropy of fusion for “Element X” at \(T = 488\;\text{K}\)
%-----------------------------------------------------------------

\section*{1.  Data (as stated)}

\[
\begin{aligned}
T &= 488\;\text{K}, \\
L &= 121\;\text{J g}^{-1}\quad\text{(latent heat of fusion)},\\
M &= 35\;\text{g mol}^{-1},\\
\Delta v &= 0.4\;\text{m}^{3}\,\text{kg}^{-1},\\
P &= 1.0731\times10^{4}\;\text{Pa}.
\end{aligned}
\]

\bigskip
\section*{2.  Convert the latent heat to a molar \emph{enthalpy} of fusion}

\[
\boxed{\;
  \Delta H_{\text{fus}}
     = L\,M
     = (121)(35)
     = 4.235\times10^{3}\;\text{J mol}^{-1}
\;}
\]
(Laboratory measurements of latent heat are done at constant \(P\), so
they already give an \emph{enthalpy} change.)

\bigskip
\section*{3.  Entropy change at equilibrium}

At the melting temperature the Gibbs energy change vanishes:
\( \Delta G = \Delta H - T\Delta S = 0\).

Hence
\[
\boxed{\;
  \Delta S = \frac{\Delta H_{\text{fus}}}{T}
           = \frac{4.235\times10^{3}}{488}
           \approx 8.68\;\text{J mol}^{-1}\text{K}^{-1}.
\;}
\]

\bigskip
\section*{4.  Why the \(P\,\Delta V\) term is \emph{not} added}

Because \(L\) was measured at constant pressure, it is already
\(\Delta H\), not \(\Delta U\).
Adding \(P\,\Delta V\) again would double‑count that contribution.
(If \(L\) had been reported as an \emph{internal‑energy} change, one
would indeed correct by \(P\,\Delta V\).)

\bigskip
\section*{5.  None of the listed choices matches \(8.7\;\text{J mol}^{-1}\text{K}^{-1}\)}

The multiple‑choice values given—\(4.03, 1.88, 0.248, 147\;\text{J mol}^{-1}\text{K}^{-1}\)
and \(7.08\times10^{-3}\)—are inconsistent with the thermodynamically
correct result of
\[
\boxed{\Delta S_{\text{liq}-\text{sol}} \simeq 8.7\;\text{J mol}^{-1}\text{K}^{-1}}.
\]

\textit{Conclusion:} the answer key (or one of the provided numbers)
contains a typographical error; the physics is unambiguous.
%-----------------------------------------------------------------
%  Internal energy of fusion  (\Delta U_\text{fus})
%-----------------------------------------------------------------
\section*{1.\  Data (from the problem)}
\[
\begin{aligned}
\Delta H_\text{fus} &= L\,M
                    = (121\;\text{J\,g}^{-1})(35\;\text{g\,mol}^{-1})
                    = 4.235\times10^{3}\;\text{J\,mol}^{-1},\\
P &= 1.0731\times10^{4}\;\text{Pa},\\
\Delta v &= 0.40\;\text{m}^3\!\,\text{kg}^{-1},\\
M &= 0.035\;\text{kg\,mol}^{-1}.
\end{aligned}
\]

\section*{2.\  Convert the specific volume change to a molar volume change}

\[
\boxed{\;
  \Delta V_m
  = \Delta v\,M
  = (0.40)(0.035)
  = 1.40\times10^{-2}\;\text{m}^3\,\text{mol}^{-1}.
\;}
\]

\section*{3.\  Use \(\Delta H = \Delta U + P\Delta V\)}

\[
\Delta U_\text{fus}
  = \Delta H_\text{fus} - P\,\Delta V_m
  = 4.235\times10^{3}
    \;-\;
    (1.0731\times10^{4})(1.40\times10^{-2})
  \;\text{J\,mol}^{-1}.
\]

\[
P\,\Delta V_m
  = 1.0731\times10^{4}\times1.40\times10^{-2}
  \simeq 1.50\times10^{2}\;\text{J\,mol}^{-1}.
\]

\[
\boxed{\;
  \Delta U_\text{fus}
  \approx 4.235\times10^{3} - 1.50\times10^{2}
  = 4.09\times10^{3}\;\text{J\,mol}^{-1}
\;}
\]

So the internal energy increases by about  
\[
\boxed{\; \Delta U_\text{fus}\;\approx\; 4.1\ \text{kJ mol}^{-1}. \;}
\]

\section*{4.\  Relation to the entropy result}

\[
\Delta S_\text{fus} = \frac{\Delta H_\text{fus}}{T}
                    = \frac{4.235\times10^{3}}{488}
                    \approx 8.68\;\text{J mol}^{-1}\text{K}^{-1},
\]
as you already found.  The difference
\(\Delta H - \Delta U = P\Delta V\) is the mechanical work required to
push back the external pressure while the substance expands on melting.
\end{document}
