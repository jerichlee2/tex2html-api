\documentclass[12pt]{article}

% Packages
\usepackage[margin=1in]{geometry}
\usepackage{amsmath,amssymb,amsthm}
\usepackage{enumitem}
\usepackage{hyperref}
\usepackage{xcolor}
\usepackage{import}
\usepackage{xifthen}
\usepackage{pdfpages}
\usepackage{transparent}
\usepackage{listings}
\usepackage{tikz}

\DeclareMathOperator{\Log}{Log}
\DeclareMathOperator{\Arg}{Arg}

\lstset{
    breaklines=true,         % Enable line wrapping
    breakatwhitespace=false, % Wrap lines even if there's no whitespace
    basicstyle=\ttfamily,    % Use monospaced font
    frame=single,            % Add a frame around the code
    columns=fullflexible,    % Better handling of variable-width fonts
}

\newcommand{\incfig}[1]{%
    \def\svgwidth{\columnwidth}
    \import{./Figures/}{#1.pdf_tex}
}
\theoremstyle{definition} % This style uses normal (non-italicized) text
\newtheorem{solution}{Solution}
\newtheorem{proposition}{Proposition}
\newtheorem{problem}{Problem}
\newtheorem{lemma}{Lemma}
\newtheorem{theorem}{Theorem}
\newtheorem{remark}{Remark}
\newtheorem{note}{Note}
\newtheorem{definition}{Definition}
\newtheorem{example}{Example}
\newtheorem{corollary}{Corollary}
\theoremstyle{plain} % Restore the default style for other theorem environments
%

% Theorem-like environments
% Title information
\title{}
\author{Jerich Lee}
\date{\today}

\begin{document}

\maketitle
%--------------------------
% Chapter 1
%--------------------------
\noindent \textbf{Chapter 1: Origins of Mechanical Energy}

\noindent A. Kinetic Energy and Work \dotfill 1\\
\noindent B. Extension to Many-Particle Systems \dotfill 3\\
\noindent C. Internal Energy \dotfill 5\\
\noindent D. Potential Energy \dotfill 7\\
\noindent E. Vibrational Energy---Kinetic plus Potential \dotfill 8

\vspace{1em}

%--------------------------
% Chapter 2
%--------------------------
\noindent \textbf{Chapter 2: Irreversibility and the Second Law of Thermodynamics}

\noindent A. Thermal Energy \dotfill 13\\
\noindent B. Irreversibility of Many-Body Systems \dotfill 14\\
\noindent C. Entropy and the Approach to Equilibrium \dotfill 16\\
\noindent D. Entropy Maximization and the Calculus of Several Variables \dotfill 19

\vspace{1em}

%--------------------------
% Chapter 3
%--------------------------
\noindent \textbf{Chapter 3: Kinetic Theory of the Ideal Gas}

\noindent A. Common Particles \dotfill 21\\
\noindent B. Pressure and Kinetic Energy \dotfill 22\\
\noindent C. Equipartition Theorem \dotfill 23\\
\noindent D. Equipartition Applied to a Solid \dotfill 25\\
\noindent E. Ideal Gas Law \dotfill 26\\
\noindent F. Distribution of Energies in a Gas \dotfill 27

\vspace{1em}

%--------------------------
% Chapter 4
%--------------------------
\noindent \textbf{Chapter 4: Ideal-Gas Heat Engines}

\noindent A. The First Law of Thermodynamics \dotfill 31\\
\noindent B. Quasi-static Processes and State Functions \dotfill 32

\vspace{1em}

%--------------------------
% Chapter 5
%--------------------------
\noindent \textbf{Chapter 5: Statistical Processes I: Two-State Systems}

\noindent A. Macrostates and Microstates \dotfill 45\\
\noindent B. Multiple Spins \dotfill 46\\
\noindent C. The Random Walk Problem---Diffusion of Particles \dotfill 50\\
\noindent D. Heat Conduction \dotfill 54

\vspace{1em}

%--------------------------
% Chapter 6
%--------------------------
\noindent \textbf{Chapter 6: Statistical Processes II: Entropy and the Second Law}

\noindent A. Meaning of Equilibrium \dotfill 59\\
\noindent B. Objects in Multiple Bins \dotfill 60\\
\noindent C. Application to a Gas of Particles \dotfill 62\\
\noindent D. Volume Exchange and Entropy \dotfill 64\\
\noindent E. Indistinguishable Particles \dotfill 68\\
\noindent F. Maximum Entropy in Equilibrium \dotfill 68

\vspace{1em}

%--------------------------
% Chapter 7
%--------------------------
\noindent \textbf{Chapter 7: Energy Exchange}

\noindent A. Model System for Exchanging Energy \dotfill 73\\
\noindent B. Thermal Equilibrium and Absolute Temperature \dotfill 78\\
\noindent C. Equipartition Revisited \dotfill 79\\
\noindent D. Why Energy Flows from Hot to Cold \dotfill 80\\
\noindent E. Entropy of the Ideal Gas---Temperature Dependence \dotfill 82

\vspace{1em}

%--------------------------
% Chapter 8
%--------------------------
\noindent \textbf{Chapter 8: Boltzmann Distribution}

\noindent A. Concept of a Thermal Reservoir \dotfill 87\\
\noindent B. The Boltzmann Factor \dotfill 88\\
\noindent C. Paramagnetism \dotfill 91\\
\noindent D. Elasticity in Polymers \dotfill 92\\
\noindent E. Harmonic Oscillator \dotfill 95

\vspace{1em}

%--------------------------
% Chapter 9
%--------------------------
\noindent \textbf{Chapter 9: Distributions of Molecules and Photons}

\noindent A. Applying the Boltzmann Factor \dotfill 99\\
\noindent B. Particle States in a Classical Gas \dotfill 100\\
\noindent C. Maxwell-Boltzmann Distribution \dotfill 102\\
\noindent D. Photons \dotfill 103\\
\noindent E. Thermal Radiation \dotfill 105\\
\noindent F. Global Warming \dotfill 107

\vspace{1em}

%--------------------------
% Chapter 10
%--------------------------
\noindent \textbf{Chapter 10: Work and Free Energy}

\noindent A. Heat Flow and Entropy \dotfill 111\\
\noindent B. Ideal Heat Engines \dotfill 112\\
\noindent C. Free Energy and Available Work \dotfill 114\\
\noindent D. Free Energy Minimum in Equilibrium \dotfill 115\\
\noindent E. Principle of Minimum Free Energy \dotfill 116\\
\noindent F. Paramagnetism---the Free Energy Approach \dotfill 117

\vspace{1em}

%--------------------------
% Chapter 11
%--------------------------
\noindent \textbf{Chapter 11: Equilibrium Between Particles I}

\noindent A. Free Energy and the Chemical Potential \dotfill 121\\
\noindent B. Absolute Entropy of an Ideal Gas \dotfill 125\\
\noindent C. Chemical Potential of an Ideal Gas \dotfill 128\\
\noindent D. Law of Atmospheres \dotfill 129\\
\noindent E. Chemical Potentials \dotfill 130

\vspace{1em}

%--------------------------
% Chapter 12
%--------------------------
\noindent \textbf{Chapter 12: Equilibrium Between Particles II}

\noindent A. Ionization of Atoms \dotfill 135\\
\noindent B. Chemical Equilibrium in gases \dotfill 137\\
\noindent C. Carrier Densities in a semiconductor \dotfill 138\\
\noindent D. Doped Semiconductors \dotfill 139\\

\vspace{1em}

%--------------------------
% Chapter 13
%--------------------------
\noindent \textbf{Chapter 13: Adsorption of Atoms and Phase Transitions}

\noindent A. Adsorption of Atoms on a Solid Surface \dotfill 145\\
\noindent B. Oxygen in Myoglobin \dotfill 147\\
\noindent C. Why Gases Condense \dotfill 148\\
\noindent D. Vapor Pressure of a Solid \dotfill 149\\
\noindent E. Solid/Liquid/Gas Phase Transitions \dotfill 151\\
\noindent F. Model of Liquid--Gas Condensation \dotfill 154

\vspace{1em}

%--------------------------
% Chapter 14
%--------------------------
\noindent \textbf{Chapter 14: Processes at Constant Pressure}

\noindent A. Gibbs Free Energy \dotfill 157\\
\noindent B. Vapor Pressures of Liquids---General Aspects \dotfill 160\\
\noindent C. Chemical Reactions at Constant Pressure \dotfill 161

\end{document}
