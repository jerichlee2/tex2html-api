\documentclass[12pt]{article}

% Packages
\usepackage[margin=1in]{geometry}
\usepackage{amsmath,amssymb,amsthm}
\usepackage{enumitem}
\usepackage{hyperref}
\usepackage{xcolor}
\usepackage{import}
\usepackage{xifthen}
\usepackage{pdfpages}
\usepackage{transparent}
\usepackage{listings}
\usepackage{tikz}
\usepackage{physics}
\usepackage{siunitx}
\usepackage{booktabs}
\usepackage{cancel}
  \usetikzlibrary{calc,patterns,arrows.meta,decorations.markings}


\DeclareMathOperator{\Log}{Log}
\DeclareMathOperator{\Arg}{Arg}

\lstset{
    breaklines=true,         % Enable line wrapping
    breakatwhitespace=false, % Wrap lines even if there's no whitespace
    basicstyle=\ttfamily,    % Use monospaced font
    frame=single,            % Add a frame around the code
    columns=fullflexible,    % Better handling of variable-width fonts
}

\newcommand{\incfig}[1]{%
    \def\svgwidth{\columnwidth}
    \import{./Figures/}{#1.pdf_tex}
}
\theoremstyle{definition} % This style uses normal (non-italicized) text
\newtheorem{solution}{Solution}
\newtheorem{proposition}{Proposition}
\newtheorem{problem}{Problem}
\newtheorem{lemma}{Lemma}
\newtheorem{theorem}{Theorem}
\newtheorem{remark}{Remark}
\newtheorem{note}{Note}
\newtheorem{definition}{Definition}
\newtheorem{example}{Example}
\newtheorem{corollary}{Corollary}
\theoremstyle{plain} % Restore the default style for other theorem environments
%

% Theorem-like environments
% Title information
\title{}
\author{Jerich Lee}
\date{\today}

\begin{document}

\maketitle
\begin{solution}
  \textbf{Equation of state}\\[-4pt]
  \[
  (P+a)V=nRT,\qquad a=\text{constant},\qquad\text{monatomic gas (equipartition).}
  \]
  
  \textbf{1. General identity}\\[-4pt]
  For any single–component fluid,
  \[
  C_P-C_V
     =T\bigl(\partial_T P\bigr)_V\bigl(\partial_T V\bigr)_P.
  \]
  
  \textbf{2. Required derivatives}\\[-4pt]
  \[
  \bigl(\partial_T P\bigr)_V
     =\frac{nR}{V},
  \qquad
  V=\frac{nRT}{P+a}
     \;\;\Longrightarrow\;\;
  \bigl(\partial_T V\bigr)_P
     =\frac{nR}{P+a}.
  \]
  
  \textbf{3. Compute the product}\\[-4pt]
  \[
  T\bigl(\partial_T P\bigr)_V\bigl(\partial_T V\bigr)_P
     =T\left(\frac{nR}{V}\right)\left(\frac{nR}{P+a}\right)
     =\frac{T\,n^2R^2}{V(P+a)}
     =\frac{T\,n^2R^2}{\displaystyle nRT}\;(=nR)
     =nR.
  \]
  
  \textbf{4. Numerical value for \(n=7\)}\\[-4pt]
  \[
  C_P-C_V=nR=(7)(8.314\ \text{J\,mol}^{-1}\text{K}^{-1})
          =5.82\times10^{1}\ \text{J\,K}^{-1}.
  \]
  
  \[
  \boxed{C_P-C_V=5.82\times10^{1}\ \text{J\,K}^{-1}}\quad(\text{option (a)})
  \]
  \end{solution}
  \begin{solution}
    \[
    \textbf{Equation of state\,:}\qquad
    (P+a)V = nRT,\qquad a=0.10\;\text{atm}. \\[4pt]
    \textbf{Internal energy assumption\,:}\qquad
    \displaystyle U = \frac{3}{2}\,nRT + \phi(V), 
    \]
    with $\phi(V)$ still to be determined.
    
    % ------------------------------------------------
    \subsection*{1.\;Heat‑capacity identities (from the snapshot)}
    \[
    C \;\equiv\; \frac{dQ}{dT},\qquad
    C_V \;=\;\Bigl(\frac{dU}{dT}\Bigr)_{V},\qquad
    C_P \;=\;\Bigl(\frac{dU}{dT}\Bigr)_{P} + P\Bigl(\frac{dV}{dT}\Bigr)_{P}.
    \]
    
    % ------------------------------------------------
    \subsection*{2.\;Express $C_P-C_V$ }
    Write $U(T,V)$ explicitly in the total derivative at constant $P$:
    \[
    \Bigl(\frac{dU}{dT}\Bigr)_{P}
       =\Bigl(\frac{\partial U}{\partial T}\Bigr)_{V}
       +\Bigl(\frac{\partial U}{\partial V}\Bigr)_{T}
        \Bigl(\frac{\partial V}{\partial T}\Bigr)_{P}.
    \]
    
    Hence
    \[
    C_P-C_V
      =\Bigl[\;P+\Bigl(\frac{\partial U}{\partial V}\Bigr)_{T}\Bigr]
        \Bigl(\frac{\partial V}{\partial T}\Bigr)_{P}.
    \tag{1}
    \]
    
    % ------------------------------------------------
    \subsection*{3.\;Find the \emph{internal pressure} $\displaystyle
    \pi_T\equiv\Bigl(\frac{\partial U}{\partial V}\Bigr)_{T}$}
    
    A standard Maxwell‑relation result for any simple fluid is
    \[
    \boxed{\;
    \pi_T
       \;=\;
       T\Bigl(\frac{\partial P}{\partial T}\Bigr)_{V} - P
    \;} .
    \]
    
    Apply the given EOS:
    \[
    P = \frac{nRT}{V} - a
          \;\;\Longrightarrow\;\;
    \Bigl(\frac{\partial P}{\partial T}\Bigr)_{V} = \frac{nR}{V}.
    \]
    
    Therefore
    \[
    \pi_T
      = T\frac{nR}{V} -\Bigl(\frac{nRT}{V}-a\Bigr)
      = a.   \tag{2}
    \]
    (The fluid’s “internal pressure’’ is the same constant $a$ that appears in the equation of state.)
    
    % ------------------------------------------------
    \subsection*{4.\;Obtain $(\partial V/\partial T)_P$}
    
    From \((P+a)V=nRT\):
    \[
    V = \frac{nRT}{P+a}
       \;\;\Longrightarrow\;\;
    \Bigl(\frac{\partial V}{\partial T}\Bigr)_{P}
       = \frac{nR}{P+a}. \tag{3}
    \]
    
    % ------------------------------------------------
    \subsection*{5.\;Assemble $C_P-C_V$}
    
    Insert (2) and (3) into (1):
    \[
    C_P-C_V
       = \bigl[P+\pi_T\bigr]
         \Bigl(\frac{\partial V}{\partial T}\Bigr)_{P}
       = (P+a)\,\frac{nR}{P+a}
       = nR. \tag{4}
    \]
    
    % ------------------------------------------------
    \subsection*{6.\;Numerical value for \(n=7\)}
    
    \[
    C_P-C_V
       = nR
       = 7\,(8.314\ \text{J\,mol}^{-1}\text{K}^{-1})
       \approx 5.82\times10^{1}\ \text{J\,K}^{-1}.
    \]
    
    \[
    \boxed{C_P-C_V = 5.82\times10^{1}\,\text{J K}^{-1}}
    \]
    \end{solution}
    \subsection*{Step 2.  Express $C_{P}-C_{V}$ in terms of partial derivatives}

\paragraph{1.  Start from the snapshot identities}
\[
C_V = \Bigl(\frac{dU}{dT}\Bigr)_{V},
\qquad
C_P = \Bigl(\frac{dU}{dT}\Bigr)_{P} 
      + P\Bigl(\frac{dV}{dT}\Bigr)_{P}.
\]

\paragraph{2.  Isolate the difference}
\[
C_P-C_V
  = \Bigl(\frac{dU}{dT}\Bigr)_{P}
  - \Bigl(\frac{dU}{dT}\Bigr)_{V}
  + P\Bigl(\frac{dV}{dT}\Bigr)_{P}.
\tag{A}
\]

\paragraph{3.  Expand \(\bigl(dU/dT\bigr)_{P}\) via the chain rule at constant \(P\)}
Treat \(U\) as a function of two independent variables \(T\) and \(V\):
\[
U = U(T,V).
\]
Along a path of \emph{constant} pressure \(P\), the total derivative of \(U\) with respect to \(T\) reads
\[
\Bigl(\frac{dU}{dT}\Bigr)_{P}
  = \Bigl(\frac{\partial U}{\partial T}\Bigr)_{V}
  + \Bigl(\frac{\partial U}{\partial V}\Bigr)_{T}\,
    \Bigl(\frac{\partial V}{\partial T}\Bigr)_{P}.
\tag{B}
\]

\paragraph{4.  Substitute (B) into (A) and cancel the common term}
Insert (B) into (A):
\[
C_P-C_V
  = \Bigl[\Bigl(\frac{\partial U}{\partial T}\Bigr)_{V}
          + \Bigl(\frac{\partial U}{\partial V}\Bigr)_{T}
            \Bigl(\frac{\partial V}{\partial T}\Bigr)_{P}\Bigr]
    - \Bigl(\frac{\partial U}{\partial T}\Bigr)_{V}
    + P\Bigl(\frac{\partial V}{\partial T}\Bigr)_{P}.
\]
The two \(\bigl(\partial U/\partial T\bigr)_{V}\) terms cancel:
\[
C_P-C_V
  = \Bigl[\;
      P
      + \Bigl(\frac{\partial U}{\partial V}\Bigr)_{T}\Bigr]\,
    \Bigl(\frac{\partial V}{\partial T}\Bigr)_{P}.
\tag{C}
\]

\paragraph{5.  Final compact form}
\[
\boxed{%
C_P - C_V
  = \Bigl[
      P
      + \Bigl(\frac{\partial U}{\partial V}\Bigr)_{T}\Bigr]\,
    \Bigl(\frac{\partial V}{\partial T}\Bigr)_{P}}
\]
Equation (C) is exactly the expression quoted in the notes.  All that remains (Step 3 in the previous derivation) is to evaluate the two bracketed factors for the specific equation of state \( (P+a)V=nRT \).
\subsection*{Deriving Eq.\,(B): total derivative along a constant–\(P\) path}

\paragraph{1.  Treat \(U\) as a function of two \emph{independent} variables}
\[
U = U(T,V).
\]

\paragraph{2.  Write the total differential of \(U\)}
\[
dU
  = \Bigl(\frac{\partial U}{\partial T}\Bigr)_{V}\,dT
  + \Bigl(\frac{\partial U}{\partial V}\Bigr)_{T}\,dV.
\tag{1}
\]

\paragraph{3.  Impose the constraint “along constant \(P\)”}

* Because \(P\) is held fixed, \(V\) cannot vary independently—it becomes a single‑valued function of \(T\): \(V=V(T)\bigl|\_{P}\).  
  (Use the equation of state to compute that function later.)

* Therefore, along this path
  \[
    dV = \Bigl(\frac{\partial V}{\partial T}\Bigr)_{P}\,dT.
  \tag{2}
  \]

\paragraph{4.  Substitute (2) into (1) and divide by \(dT\)}

\[
\boxed{\;
\Bigl(\frac{dU}{dT}\Bigr)_{P}
  = \Bigl(\frac{\partial U}{\partial T}\Bigr)_{V}
  + \Bigl(\frac{\partial U}{\partial V}\Bigr)_{T}
    \Bigl(\frac{\partial V}{\partial T}\Bigr)_{P}}
\]

The first term is the “explicit” \(T\) dependence of \(U\) at fixed \(V\);  
the second term captures the “indirect” \(T\) dependence of \(U\) that flows through the volume \(V(T)\) required to keep \(P\) constant.
\paragraph{Why does $V$ become a single–valued function of $T$ when $P$ is fixed?}

\begin{enumerate}
\item \textbf{Thermodynamic surface.}\;
   For any simple compressible substance the state variables satisfy an
   \emph{equation of state}
   \[
     f(P,T,V)=0
     \qquad\bigl(\text{with $n$ fixed}\bigr).
   \]
   Geometrically, this defines a two‑dimensional surface in the $P$–$T$–$V$ phase‑space.

\item \textbf{Intersection with a constant‑$P$ plane.}\;
   Fixing the pressure at some value $P=P_0$ amounts to slicing that surface with the plane
   \[
     P = P_0.
   \]
   The intersection is a \emph{curve} in the $T$–$V$ plane described by
   \[
     g(T,V)\equiv f(P_0,T,V)=0.
   \]

\item \textbf{Implicit‑function theorem.}\;
   Provided
   \(\displaystyle \Bigl(\partial g/\partial V\Bigr)_{T}\neq0,\)
   the implicit‑function theorem guarantees that the curve can be written locally as
   \[
     V = V(T)\bigl|_{P=P_0},
   \]
   i.e.\ volume becomes a \emph{single‑valued} function of temperature along that path.

   The derivative along the path is then
   \[
     \Bigl(\frac{dV}{dT}\Bigr)_{P}
       = -\,\frac{\displaystyle\Bigl(\partial g/\partial T\Bigr)_{V}}
                {\displaystyle\Bigl(\partial g/\partial V\Bigr)_{T}}
       = \Bigl(\frac{\partial V}{\partial T}\Bigr)_{P}.
   \]

\item \textbf{Explicit example (ideal gas).}\;
   For the ideal‑gas EOS $PV=nRT$,
   holding $P=P_0$ gives
   \[
     V=\frac{nRT}{P_0}\;\;\Longrightarrow\;\;
     \Bigl(\frac{\partial V}{\partial T}\Bigr)_{P}
       =\frac{nR}{P_0},
   \]
   confirming that $V$ is determined \emph{solely} by~$T$ when $P$ is fixed.

\end{enumerate}

\medskip
\noindent
\textbf{Physical intuition.}\;
Keeping the external (or internal, for a reversible path) pressure fixed means the
system is always mechanically balanced at that pressure.
If you now \emph{tweak the temperature}, the only way the state can remain on the
constant‑$P$ slice of the thermodynamic surface is for the volume to adjust
\emph{uniquely} to the new $T$. Hence $V$ cannot “choose” a second, independent value—%
it is locked to $T$ by the imposed constraint $P=P_0$.
\[
% ---- Total differential of U(T,V) with no constraint ----
dU
  = \Bigl(\frac{\partial U}{\partial T}\Bigr)_{V}\,dT
  + \Bigl(\frac{\partial U}{\partial V}\Bigr)_{T}\,dV.
\tag{1}
\]
\[
% ---- Impose the constant‑P path and keep T as the independent variable ----
\text{Along }P=\text{const},\;V = V(T)\;\Longrightarrow\;
dV = \Bigl(\frac{\partial V}{\partial T}\Bigr)_{P}\,dT.
\]
% ---- Substitute into (1) ----
\[
\Bigl(\frac{dU}{dT}\Bigr)_{P}
  = \Bigl(\frac{\partial U}{\partial T}\Bigr)_{V}
  + \Bigl(\frac{\partial U}{\partial V}\Bigr)_{T}
    \Bigl(\frac{\partial V}{\partial T}\Bigr)_{P}.
\tag{B}
\]
% -------------------------------------------------------------
\subsection*{Step 3.  Evaluate the internal pressure
            $\displaystyle\pi_{T}\equiv
            \Bigl(\frac{\partial U}{\partial V}\Bigr)_{T}$}

\paragraph{3.1 Fundamental thermodynamic identity}

For a single‑component fluid,
\[
dU = T\,dS - P\,dV.
\tag{3.1}
\]

\paragraph{3.2 Take the partial derivative $(\partial/\partial V)_{T}$ of (3.1)}

Holding \(T\) fixed \((dT=0)\),
\[
\Bigl(\frac{\partial U}{\partial V}\Bigr)_{T}
  = T\Bigl(\frac{\partial S}{\partial V}\Bigr)_{T} - P.
\tag{3.2}
\]

The left‑hand side is, by definition, the
\emph{internal pressure}
\[
\boxed{\pi_{T} \;=\;
       \Bigl(\frac{\partial U}{\partial V}\Bigr)_{T}}.
\tag{3.3}
\]

\paragraph{3.3 Replace $\displaystyle(\partial S/\partial V)_{T}$
          by a Maxwell relation}

Start from the Helmholtz free energy
\(F\equiv U-TS\).  
Its differential is
\[
dF = -S\,dT - P\,dV,
\]
so the mixed partials give the Maxwell relation
\[
\boxed{\Bigl(\frac{\partial S}{\partial V}\Bigr)_{T}
       = \Bigl(\frac{\partial P}{\partial T}\Bigr)_{V}.}
\tag{3.4}
\]

\paragraph{3.4 Insert (3.4) into (3.2)}

\[
\pi_{T}
  = T\Bigl(\frac{\partial P}{\partial T}\Bigr)_{V} - P.
\]
This is the “standard Maxwell‑relation result’’ quoted in the notes.

\paragraph{3.5 Apply the specific equation of state}

Given
\[
\boxed{(P+a)\,V = nRT},
\]
\[
P = \frac{nRT}{V} - a
     \quad\Longrightarrow\quad
\Bigl(\frac{\partial P}{\partial T}\Bigr)_{V}
     = \frac{nR}{V}.
\]

\paragraph{3.6 Compute $\pi_{T}$}

\[
\pi_{T}
  = T\frac{nR}{V}
    -\Bigl(\frac{nRT}{V} - a\Bigr)
  = a.
\tag{3.5}
\]

\[
\boxed{\;\pi_{T}=a\;}
\]

\emph{Interpretation}:  
the constant \(a\) represents a temperature‑independent, attractive “cohesive’’ pressure.  
Because \(\pi_{T}\) measures how the internal energy changes when you compress the system \emph{isothermally}, a purely attractive contribution of magnitude \(a\) shows up unchanged in \(\pi_{T}\).
\begin{solution}
  \textbf{Coexistence‐line geometry}\\
  In the sketch the \emph{Phase I–Phase II} coexistence curve slopes \emph{downward} as
  temperature increases (negative slope\footnote{%
  Explicitly $\mathrm{d}P/\mathrm{d}T<0$ because the line descends to the
  right.}).
  
  \bigskip
  \textbf{Clapeyron (or Gibbs) equation}\\
  Along any two–phase equilibrium
  \[
  \frac{\mathrm{d}P}{\mathrm{d}T}
     =\frac{\Delta S}{\Delta V},
     \qquad
     \Delta S \equiv S_{\text{II}}-S_{\text{I}},\;
     \Delta V \equiv V_{\text{II}}-V_{\text{I}}
     \;\;( \text{per particle} ).
  \]
  
  \bigskip
  \textbf{Sign of $\Delta S$}\\
  Moving from Phase I to Phase II requires adding heat (the latent heat of
  transition), so $\Delta S>0$.
  
  \bigskip
  \textbf{Consequence of a negative slope}\\
  Because $\mathrm{d}P/\mathrm{d}T<0$ while $\Delta S>0$,
  the quotient must be negative; therefore
  \[
  \Delta V < 0
  \quad\Longrightarrow\quad
  V_{\text{II}}<V_{\text{I}}.
  \]
  
  \bigskip
  \textbf{Density comparison}\\
  Smaller molar (or per‐particle) volume means greater density, so
  
  \[
  \boxed{\text{Phase II is \textit{denser} than Phase I.}}
  \]
  
  \bigskip
  \textbf{Answer}\\
  \[
  \text{(a) Phase II is denser than Phase I.}
  \]
  All other listed statements depend on properties (\(\Delta S\) or
  \(\Delta V\)) that the diagram does \emph{not} fix uniquely.
  \end{solution}
  \begin{solution}
    \textbf{Start from the given thermodynamic identity}
    \[
    d\mu \;=\; \frac{V}{N}\,dP \;-\; \frac{S}{N}\,dT
    \quad
    \left(
      \mu\equiv\text{chemical potential per particle},\;
      \frac{V}{N}\equiv\text{molar (or per‐particle) volume},\;
      \frac{S}{N}\equiv\text{entropy per particle}
    \right).
    \]
    
    \bigskip
    \textbf{Apply the equality of chemical potentials on a coexistence line}\\
    Along a line where \emph{Phase I} and \emph{Phase II} coexist,
    \(\mu_{\text{I}} = \mu_{\text{II}}\) at every point.  
    Therefore their differentials are equal:
    \[
    d\mu_{\text{I}} = d\mu_{\text{II}}.
    \]
    
    Insert the identity for each phase and subtract:
    \[
    \frac{V_{\text{I}}}{N}\,dP - \frac{S_{\text{I}}}{N}\,dT
    =
    \frac{V_{\text{II}}}{N}\,dP - \frac{S_{\text{II}}}{N}\,dT.
    \]
    
    \bigskip
    \textbf{Isolate the coexistence‐curve slope}\\
    Rearrange the last equation:
    \[
    \bigl(V_{\text{II}} - V_{\text{I}}\bigr)\,dP
    =
    \bigl(S_{\text{II}} - S_{\text{I}}\bigr)\,dT.
    \]
    Define
    \(\displaystyle
      \Delta V \equiv V_{\text{II}}-V_{\text{I}},\;
      \Delta S \equiv S_{\text{II}}-S_{\text{I}}
    \)
    (per particle) and divide by \(dT\):
    \[
    \boxed{\;
      \frac{dP}{dT}
      = \frac{\Delta S}{\Delta V}
    \;} .
    \]
    This is the \emph{Clapeyron relation}, now obtained directly from the
    given \(d\mu\) formula.
    
    \bigskip
    \textbf{Use the diagram’s information}\\
    * The Phase I–Phase II coexistence curve slopes \emph{downward} in the
      \(P\!-\!T\) plane: \(dP/dT<0.\)
    * A phase change requires absorbing latent heat, so
      \(\Delta S = S_{\text{II}}-S_{\text{I}} > 0.\)
    
    Because \(dP/dT = \Delta S / \Delta V\) is negative while \(\Delta S\) is
    positive, the denominator must be negative:
    \[
    \Delta V < 0
    \quad\Longrightarrow\quad
    V_{\text{II}} < V_{\text{I}}.
    \]
    
    \bigskip
    \textbf{Density comparison}\\
    Smaller volume per particle means greater density, hence
    \[
    \boxed{\text{Phase II is denser than Phase I.}}
    \]
    
    \bigskip
    \textbf{Answer}\\
    \[
    \text{(a) Phase II is denser than Phase I.}
    \]
    \end{solution}
    %------------------------------------------------------------------
% Thermodynamic cycle – step‑by‑step solution
%------------------------------------------------------------------

\textbf{Given data}

\[
\begin{array}{c|cc}
\text{Point} & V\;(\text{m}^3) & P\;(\text{Pa}) \\ \hline
1 & 1.2 & 1027 \\
2 & 1.4 & 1460 \\
3 & 1.4 & 1534 \\
4 & 1.2 & 1156
\end{array}
\qquad
n = 2\;\text{mol}, \quad
R = 8.314\;\text{J\,mol}^{-1}\text{K}^{-1}, \quad
C_v = \tfrac{5}{2}R\quad(\text{diatomic})
\]

%------------------------------------------------------------------
\section*{1. Work done by the gas during segment $1\rightarrow2$}

The path $1\rightarrow2$ is a straight line on the $P$–$V$ diagram  
(linear variation of pressure with volume).  
For a linear path the work equals the area of the trapezoid:

\[
\begin{aligned}
W_{1\rightarrow2}
      &= \int_{V_1}^{V_2} P\,dV
      = \tfrac{1}{2}\bigl(P_1 + P_2\bigr)\bigl(V_2 - V_1\bigr) \\[6pt]
      &= \tfrac{1}{2}\bigl(1027 + 1460\bigr)\,\text{Pa}\;(1.4 - 1.2)\,\text{m}^3 \\[6pt]
      &= 1243.5\;\text{Pa}\times 0.20\;\text{m}^3 \\[4pt]
      &\approx 2.49\times10^{2}\;\text{J}.
\end{aligned}
\]

\[
\boxed{W_{1\rightarrow2} \;\approx\; +2.49 \times 10^{2}\ \text{J}}
\]

(Positive sign $\Rightarrow$ work done \emph{by} the gas.)

%------------------------------------------------------------------
\section*{2. Heat added to the gas during segment $4\rightarrow1$}

Segment $4\rightarrow1$ is vertical ($V=$ const.), so

\[
W_{4\rightarrow1} \;=\; \int P\,dV = 0.
\]

For an ideal gas at constant volume

\[
Q_{4\rightarrow1} \;=\; \Delta U \;=\; nC_v\,(T_1 - T_4).
\]

\[
\begin{aligned}
T_4 &= \frac{P_4 V_4}{nR}
     = \frac{1156\;\text{Pa}\times 1.2\;\text{m}^3}{(2)(8.314\;\text{J mol}^{-1}\text{K}^{-1})}
     \approx 83.4\;\text{K},\\[6pt]
T_1 &= \frac{P_1 V_1}{nR}
     = \frac{1027\;\text{Pa}\times 1.2\;\text{m}^3}{(2)(8.314\;\text{J mol}^{-1}\text{K}^{-1})}
     \approx 74.1\;\text{K},\\[6pt]
\Delta T &= T_1 - T_4 \;=\; 74.1 - 83.4 \;=\; -9.3\;\text{K},\\[6pt]
\Delta U &= nC_v\Delta T
         = (2)\!\left(\tfrac{5}{2}R\right)(-9.3\;\text{K}) \\[4pt]
         &= 41.57\;\text{J K}^{-1}\times(-9.3\;\text{K}) \\[4pt]
         &\approx -3.9\times10^{2}\;\text{J}.
\end{aligned}
\]

\[
\boxed{Q_{4\rightarrow1} \;\approx\; -3.9 \times 10^{2}\ \text{J}}
\]

(Negative sign $\Rightarrow$ $3.9\times10^{2}$ J of heat \emph{leaves} the gas; none is added.)
%--------------------------------------------------------------
%  Latent–heat problem (entropy of fusion)
%--------------------------------------------------------------
\textbf{Data}

\[
\begin{aligned}
T        &= 488\;\text{K}, &
P        &= 1.0731\times10^{4}\;\text{Pa},\\
L        &= 121\;\text{J\,g}^{-1}, &
M        &= 35\;\text{g\,mol}^{-1},\\
\Delta v &= 0.4\;\text{m}^{3}\,\text{kg}^{-1}.
\end{aligned}
\]

\bigskip
\textbf{1.\ Convert latent heat to a molar basis}

\[
\boxed{L_{\text{molar}} = L\,M}
      = (121\;\text{J\,g}^{-1})(35\;\text{g\,mol}^{-1})
      = 4.235\times10^{3}\;\text{J\,mol}^{-1}.
\]

\bigskip
\textbf{2.\ Entropy change at constant \emph{T} and \emph{P}}

For a phase transition in equilibrium, \( \Delta G = 0 \) and
\[
\Delta H = T\,\Delta S
\quad\Longrightarrow\quad
\boxed{\;\Delta S = \dfrac{\Delta H}{T}\;}.
\]

Assuming the quoted \(L\) is the enthalpy of fusion (the usual experimental definition),

\[
\Delta S
  = \frac{4.235\times10^{3}\;\text{J\,mol}^{-1}}
         {488\;\text{K}}
  = 8.68\;\text{J\,mol}^{-1}\,\text{K}^{-1}.
\]

\bigskip
\textbf{3.\ Small correction if $L$ were an \emph{internal‑energy} value}

If instead \(L\) represented an internal‑energy change, the enthalpy would be
\[
\Delta H = L_{\text{molar}} + P\,\Delta V_{\text{molar}},
\]
where
\[
\Delta V_{\text{molar}}
   = \Delta v\,M
   = (0.4\;\text{m}^{3}\,\text{kg}^{-1})(0.035\;\text{kg\,mol}^{-1})
   = 1.40\times10^{-2}\;\text{m}^{3}\,\text{mol}^{-1}.
\]
Hence
\[
P\,\Delta V_{\text{molar}}
   = (1.0731\times10^{4}\;\text{Pa})(1.40\times10^{-2}\;\text{m}^{3}\,\text{mol}^{-1})
   \approx 1.50\times10^{2}\;\text{J\,mol}^{-1}.
\]
Adding this to \(L_{\text{molar}}\) raises \(\Delta S\) only slightly,
to about \(9.0\;\text{J\,mol}^{-1}\,\text{K}^{-1}\).

\bigskip
\textbf{4.\ Numerical result vs. the multiple‑choice list}

Either way, the physically consistent answer is

\[
\boxed{\Delta S \approx 8.7\;\text{J\,mol}^{-1}\,\text{K}^{-1}}.
\]

None of the five options given (\(4.03,\;1.88,\;0.248,\;\dots\)) equals this value, so it looks as though the answer key (or one of the supplied numbers) contains a misprint.  The derivation above follows standard thermodynamic definitions and the data exactly as provided.
%-----------------------------------------------------------------
%  Irreversible heat‑engine problem (entropy increase per cycle)
%-----------------------------------------------------------------

\textbf{Given data}

\[
T_H = 532\;\text{K}, \qquad
T_C = 346\;\text{K}, \qquad
Q_H = 155\;\text{J}, \qquad
\varepsilon = 0.238.
\]

%-----------------------------------------------------------------
\section*{1.\ Work output of the engine}

\[
W = \varepsilon\,Q_H
   = (0.238)(155\ \text{J})
   = 36.9\ \text{J}.
\]

%-----------------------------------------------------------------
\section*{2.\ Heat rejected to the cold reservoir}

\[
Q_C = Q_H - W
     = 155\ \text{J} - 36.9\ \text{J}
     = 118.1\ \text{J}.
\]

%-----------------------------------------------------------------
\section*{3.\ Total entropy change per cycle}

Because the engine itself is assumed reversible (no internal losses),
the only entropy change comes from the two reservoirs:

\[
\Delta S_{\text{tot}}
  = \frac{Q_C}{T_C} - \frac{Q_H}{T_H}.
\]

Insert the numbers:

\[
\begin{aligned}
\Delta S_{\text{tot}}
  &= \frac{118.1\;\text{J}}{346\;\text{K}}
     \;-\;
     \frac{155\;\text{J}}{532\;\text{K}} \\[6pt]
  &= 0.341\;\text{J\,K}^{-1}
     \;-\;
     0.291\;\text{J\,K}^{-1} \\[6pt]
  &= 0.0498\;\text{J\,K}^{-1}.
\end{aligned}
\]

\[
\boxed{\Delta S_{\text{tot}} \;\approx\; 4.98 \times 10^{-2}\ \text{J\,K}^{-1}}
\]

(The positive value confirms the cycle is irreversible, in accord with
the second law.)
%--------------------------------------------------------------
%  Phase–diagram question: Which statement is true?
%  (Return‑to‑text LaTeX version)
%--------------------------------------------------------------

\textbf{Correct choice:} \textbf{(e)} \emph{Phase III has more entropy per particle than Phase II.}

\bigskip
\hrule
\bigskip

\section*{Step–by–step thermodynamic reasoning}

\begin{enumerate}
\item[]%
\item \textbf{Clausius–Clapeyron relation for a coexistence line}
      \[
        \frac{dP}{dT}\;=\;\frac{\Delta S}{\Delta V},
        \qquad
        \Delta S = S_\beta - S_\alpha,\;\;
        \Delta V = V_\beta - V_\alpha .
      \]

\item \textbf{Apply it to the II--III boundary}\\[2pt]
      The II--III line in the diagram slopes up and to the right, so
      \[
        \left(\frac{dP}{dT}\right)_{\mathrm{II\text{--}III}} > 0.
      \]

\item \textbf{Sign of \(\Delta V\)}\\[2pt]
      At a given temperature just \emph{below} the coexistence line,
      Phase II is the \emph{higher‑pressure} phase.
      Higher pressure favours the phase with \emph{smaller} molar
      volume, hence
      \[
        V_{\mathrm{II}} < V_{\mathrm{III}}
        \quad\Longrightarrow\quad
        \Delta V = V_{\mathrm{III}} - V_{\mathrm{II}} > 0.
      \]

\item \textbf{Deduce the sign of \(\Delta S\)}\\[2pt]
      Since both \(dP/dT\) and \(\Delta V\) are positive,
      the Clapeyron relation gives
      \[
        \Delta S = S_{\mathrm{III}} - S_{\mathrm{II}} > 0
        \;\;\Longrightarrow\;\;
        S_{\mathrm{III}} > S_{\mathrm{II}} .
      \]
\end{enumerate}

\bigskip
\hrule
\bigskip

\section*{Why the other options are incorrect}

\begin{itemize}
  \item \textbf{(a)} and \textbf{(d)}  
        Along the I--III line, the slope is also
        \(dP/dT>0\) but Phase I is on the \emph{higher}‑pressure side,
        leading to \(S_{\mathrm{III}} > S_{\mathrm{I}}\),
        not the reverse claimed in (a) or (d).

  \item \textbf{(b)}  
        On the I--II line, Phase II sits at the higher pressure,
        implying \(V_{\mathrm{II}}<V_{\mathrm{I}}\);
        statement (b) asserts the opposite density ordering.

  \item \textbf{(c)}  
        From the II--III argument above we have
        \(V_{\mathrm{III}} > V_{\mathrm{II}}\),
        so Phase III is \emph{less} dense, contradicting (c).
\end{itemize}
%---------------------------------------------------------------
%  Local gravity on Pluto from a scale–height measurement
%---------------------------------------------------------------
\textbf{Data}

\[
\frac{n(h)}{n_0}=0.75,\qquad h=10.0\;\text{km},\qquad
T=50\;\text{K},\qquad
\text{molecule: } \mathrm{N_2}\;(M = 28\ \text{u})
\]

\[
k_B = 1.380\,649\times10^{-23}\;\text{J K}^{-1},\qquad
m_u = 1.660\,539\times10^{-27}\;\text{kg}.
\]

%---------------------------------------------------------------
\section*{1.\ Barometric (isothermal) law}

For an isothermal, ideal gas in hydrostatic equilibrium,

\[
\boxed{\;
  \frac{n(h)}{n_0}
     = \exp\!\left(-\frac{mgh}{k_B T}\right)
\;}
\tag{1}
\]

where \(m\) is the mass of one gas molecule.

%---------------------------------------------------------------
\section*{2.\ Solve (1) for \(g\)}

\[
\begin{aligned}
\ln\!\left(\frac{n(h)}{n_0}\right)
      &= -\frac{mgh}{k_B T}
      \quad\Longrightarrow\quad
      g
      = -\frac{k_B T}{m h}\,
        \ln\!\left(\frac{n(h)}{n_0}\right).
\end{aligned}
\tag{2}
\]

%---------------------------------------------------------------
\section*{3.\ Insert numbers}

Molecular mass (in kg):

\[
m = 28\,m_u
    = 28\,(1.660\,539\times10^{-27})\;\text{kg}
    = 4.6494\times10^{-26}\;\text{kg}.
\]

Natural logarithm:

\[
\ln(0.75) = -0.28768.
\]

Now put everything into (2):

\[
\begin{aligned}
g
&= -\frac{(1.380\,649\times10^{-23}\,\text{J K}^{-1})(50\;\text{K})}
         {(4.6494\times10^{-26}\,\text{kg})(10\,000\;\text{m})}
     \;\ln(0.75) \\[6pt]
&= \frac{6.9032\times10^{-22}}{4.6494\times10^{-22}}
     \times 0.28768 \\[6pt]
&= 1.485 \times 0.28768
 = 0.427\;\text{m s}^{-2}.
\end{aligned}
\]

\[
\boxed{g_{\text{Pluto}} \approx 0.427\ \text{m s}^{-2}}
\]

\bigskip
\noindent
\emph{Interpretation.}  
The relatively shallow decline (only 25 % drop in number density over
10 km) implies a weak gravitational field, and the computed value is
fully consistent with modern spacecraft measurements of Pluto’s surface
gravity (~0.62 \% of Earth’s).
%---------------------------------------------------------------
%  Why the substance at point Q will freeze
%  (copy‑paste LaTeX explanation)
%---------------------------------------------------------------

\section*{Thermodynamic reasoning}

\subsection*{1.\ What a $\mu$--$T$ diagram tells us}
\begin{itemize}
  \item Each curve gives the \emph{molar} chemical potential $\mu$ of one
        phase (solid, liquid, gas) at a fixed pressure.
  \item \textbf{At any temperature, the stable phase is the one with the
        \emph{lowest} $\mu$.}\;%
        Minimising $\mu$ minimises the Gibbs free energy
        $G = N\mu$ for a one‑component system.
\end{itemize}

\subsection*{2.\ Position of point Q}
\begin{itemize}
  \item Point~Q lies on the \emph{dashed} curve (call it the liquid).
  \item Drawing a vertical line down to the \emph{solid} curve shows that
        $\mu_{\text{solid}} < \mu_{\text{liquid}}$ at the same~$T$.
  \item Therefore the liquid at Q is \emph{metastable}: the system can
        lower $G$ by converting to solid.
\end{itemize}

\subsection*{3.\ Long‑time outcome at constant $T$ and $P$}
Because $T$ and $P$ are held fixed,
the only route to a lower $\mu$ is a phase change.
Any thermal fluctuation that nucleates a bit of solid is favoured and
will grow, so \textbf{the substance freezes.}

\subsection*{4.\ Why the other options are excluded}
\begin{center}
\begin{tabular}{@{}ll@{}}
\toprule
Option & Reason it cannot happen \\ \midrule
Sublimation / boiling &
  Gas curve has the \emph{highest} $\mu$, so gas is least stable. \\
Melting &
  Would require $\mu_{\text{liquid}} < \mu_{\text{solid}}$, opposite to
  the diagram. \\
Nothing happens &
  A metastable state can persist briefly, but \emph{``after a long
  time''} the system always reaches the lowest‑$\mu$ phase. \\ \bottomrule
\end{tabular}
\end{center}

\bigskip
\noindent
\textbf{Conclusion:}\; After sufficient time the substance will
\boxed{\text{freeze}.}
%--------------------------------------------------------------------
%  Binding energy of an N$_2$ molecule in liquid nitrogen
%--------------------------------------------------------------------
\textbf{Given}

\[
\begin{aligned}
Q &= 4480\ \text{J} 
       &&\text{(heat supplied, equal to enthalpy change $\Delta H$)},\\
m &= 20\ \text{g}, \\
M &= 28\ \text{g mol}^{-1} 
       &&\text{(molar mass of N$_2$)},\\
T &= 77\ \text{K}, \\
p &= 1\ \text{atm}\approx 1.013\times10^{5}\ \text{Pa},\\
R &= 8.314\ \text{J mol}^{-1}\text{K}^{-1}.
\end{aligned}
\]

\section*{1.\ Moles of nitrogen}

\[
n = \frac{m}{M}
    = \frac{20\ \text{g}}{28\ \text{g mol}^{-1}}
    = 0.7143\ \text{mol}.
\]

\section*{2.\ $pV$ term for the ideal gas}

For an ideal gas,
\[
pV = nRT.
\]
At the boiling temperature
\[
nRT
  = (0.7143\ \text{mol})(8.314\ \text{J mol}^{-1}\text{K}^{-1})(77\ \text{K})
  = 4.57\times10^{2}\ \text{J}.
\]

\section*{3.\ Change in internal energy}

Because the volume of the liquid is negligible,
\[
\Delta U = \Delta H - nRT
         = 4480\ \text{J} - 457\ \text{J}
         = 4.02\times10^{3}\ \text{J}.
\]

\section*{4.\ Number of molecules}

\[
N = nN_A
    = (0.7143\ \text{mol})(6.022\times10^{23}\ \text{mol}^{-1})
    = 4.30\times10^{23}\ \text{molecules}.
\]

\section*{5.\ Binding energy per molecule}

\[
\boxed{
  \varepsilon
  = \frac{\Delta U}{N}
  = \frac{4.02\times10^{3}\ \text{J}}{4.30\times10^{23}}
  = 9.35\times10^{-21}\ \text{J}
}
\]

\bigskip
Thus the binding (cohesive) energy of a nitrogen molecule in the liquid
is approximately
\[
\varepsilon \simeq 9.4\times10^{-21}\ \text{J},
\]
matching option \textbf{(b)}.
%---------------------------------------------------------------
%  Entropy change of a copper sphere quenched in water
%---------------------------------------------------------------

\textbf{Data}

\[
\begin{aligned}
m_{\mathrm{Cu}} &= 5.0\;\text{kg}, &
c_{\mathrm{Cu}} &= 385\;\text{J\,kg}^{-1}\text{K}^{-1}, &
T_{\mathrm{Cu},i} &= 200^{\circ}\text{C}=473.15\;\text{K},\\[4pt]
m_{\mathrm{H_2O}} &= 50.0\;\text{kg}, &
c_{\mathrm{H_2O}} &= 4182\;\text{J\,kg}^{-1}\text{K}^{-1}, &
T_{\mathrm{H_2O},i} &= 5^{\circ}\text{C}=278.15\;\text{K}.
\end{aligned}
\]

No heat is lost to the surroundings; the copper + water form an isolated
system that comes to a common final temperature \(T_f\).

%---------------------------------------------------------------
\section*{1.\ Find the equilibrium temperature \(T_f\)}

Energy balance (heat lost by Cu = heat gained by water):

\[
m_{\mathrm{Cu}}\,c_{\mathrm{Cu}}\,(T_{\mathrm{Cu},i}-T_f)
   = m_{\mathrm{H_2O}}\,c_{\mathrm{H_2O}}\,(T_f-T_{\mathrm{H_2O},i}).
\]

Insert the numbers:

\[
(5)(385)(473.15-T_f)
   = (50)(4182)(T_f-278.15).
\]

Solve for \(T_f\):

\[
T_f \approx 280\;\text{K}\quad(\,7^{\circ}\text{C}\,).
\]

(With more significant figures \(T_f = 280.1\ \text{K}\).)

%---------------------------------------------------------------
\section*{2.\ Entropy change of the \emph{copper} only}

For a body whose specific heat \(c\) is effectively constant,

\[
\Delta S_{\mathrm{Cu}}
   = m_{\mathrm{Cu}}\,c_{\mathrm{Cu}}
     \int_{T_{\mathrm{Cu},i}}^{T_f}\frac{dT}{T}
   = m_{\mathrm{Cu}}\,c_{\mathrm{Cu}}
     \ln\!\left(\frac{T_f}{T_{\mathrm{Cu},i}}\right).
\]

\[
\begin{aligned}
\Delta S_{\mathrm{Cu}}
 &= (5.0\;\text{kg})(385\;\text{J\,kg}^{-1}\text{K}^{-1})
    \ln\!\bigl(\tfrac{280.1}{473.15}\bigr) \\[4pt]
 &= 1925\;\text{J K}^{-1}\times
    \bigl(-0.523\bigr) \\[4pt]
 &\approx -1.0\times10^{3}\;\text{J K}^{-1}.
\end{aligned}
\]

\[
\boxed{\;\Delta S_{\mathrm{Cu}}\;\approx\;-1.0\times10^{3}\ \text{J K}^{-1}\;}
\]

(The negative sign means the copper \emph{loses} entropy as it cools.)
%------------------------------------------------------------------
%  Why we substitute $pV = nRT$ in the $\Delta U$ expression
%------------------------------------------------------------------

\section*{1.\ First--law differential versus state functions}

The first law for an \emph{infinitesimal} change of state is  
\[
dU = \delta Q - p\,dV .
\]
If the \emph{path} happens to satisfy $dV = 0$ (e.g.\ an ideal
isochoric process), then the $p\,dV$ term indeed vanishes \emph{along
that path}.  The vaporisation problem, however, involves a
\emph{finite} expansion from liquid to gas, so $dV \neq 0$ on that
path and the above differential is not the most convenient tool.

\bigskip
\section*{2.\ Enthalpy identity}

Define the molar enthalpy
\[
H \;=\; U + pV .
\]
Because $H$ is a \emph{state function}, for any initial $(i)$ and final
$(f)$ states we have the exact algebraic relation
\[
\boxed{\;
  \Delta U
  = \Delta H - \Delta(pV)
  = (H_f - H_i) - (p_fV_f - p_iV_i)
\;}.
\tag{1}
\]

Equation~(1) needs no knowledge of the path; it depends only on the
end‑point states.

\bigskip
\section*{3.\ Applying Eq.\,(1) to nitrogen vaporisation}

\begin{itemize}
  \item \emph{Initial state (liquid):}\quad
        $V_i \approx 0$, hence $p_i V_i \approx 0$.
  \item \emph{Final state (gas at 1\,atm, $T=77$\,K):}\quad
        For an ideal gas
        \[
          p_fV_f = nRT .
        \]
\end{itemize}

Therefore
\[
\Delta(pV) = p_fV_f - p_iV_i \;\approx\; nRT - 0 = nRT ,
\]
and Eq.\,(1) becomes
\[
\boxed{\;
  \Delta U = \Delta H - nRT
\;}
\]
with $\Delta H = Q_{\text{vap}}$ supplied by the heater (latent heat).

\bigskip
\section*{4.\ Contrast with a true constant--volume process}

In the copper–water cooling example,
the \emph{path itself} obeys $dV = 0$, so integrating the
first‑law differential along that path gives directly
\[
\Delta U = \int \delta Q \quad (\text{since } p\,dV = 0).
\]
No $nRT$ term appears because the system volume never changes.

\bigskip
\section*{5.\ Summary}

\begin{itemize}
  \item $pV = nRT$ is invoked \emph{only} to evaluate the
        \emph{state‑function difference} $\Delta(pV)$ between the gas
        and liquid states.
  \item It has nothing to do with assuming $dV = 0$; in fact the term
        is crucial precisely because $V$ \emph{does} change during
        vaporisation.
\end{itemize}
%--------------------------------------------------------------------
%  Two heat‑capacity/equipartition questions
%--------------------------------------------------------------------

\section*{Problem 1.  Number of atoms in the aluminium block}

\begin{itemize}
\item Measured heat capacity of the block at $T=300\,$K:
      \[
      C_{\text{block}} = 15\;\text{J\,K}^{-1}.
      \]
\item High‑temperature (equipartition) limit for a crystalline solid:  
      each atom contributes $3k_B$ to the heat capacity (Dulong–Petit).
\[
C_{\text{per atom}} = 3k_B
                    = 3\,(1.380\,649\times10^{-23}\,\text{J\,K}^{-1})
                    = 4.1419\times10^{-23}\,\text{J\,K}^{-1}.
\]
\item Number of atoms:
\[
N = \frac{C_{\text{block}}}{3k_B}
    = \frac{15}{4.1419\times10^{-23}}
    \approx 3.62\times10^{23}\;\text{atoms}.
\]
\end{itemize}

\[
\boxed{N \simeq 3.6\times10^{23}\ \text{atoms}}
\]

(The choice $3.62\times10^{23}$ is therefore correct.)

%--------------------------------------------------------------------
\section*{Problem 2.  Identifying the gas from the equilibrium temperature}

\vspace{-0.5em}
\[
\begin{array}{lcl}
T_{\text{block},i} &=& 300\;\text{K},\\
T_{\text{gas},i}   &=& 400\;\text{K},\\
T_f               &=& 362\;\text{K},\\
C_{\text{block}}  &=& 15\;\text{J\,K}^{-1},\\
n_{\text{gas}}    &=& 1\;\text{mol}.
\end{array}
\]

\subsection*{(a) Energy balance}

Because the container is thermally isolated,
\[
\Delta Q_{\text{block}} + \Delta Q_{\text{gas}} = 0.
\]

\[
C_{\text{block}}\,(T_f - 300)
   + n\,C_{V,\text{gas}}\,(T_f - 400) = 0.
\]

Insert the numbers:
\[
15\,(362 - 300)
   + 1\,C_{V,\text{gas}}\,(362 - 400) = 0
\quad\Longrightarrow\quad
C_{V,\text{gas}}
   = 15\,\frac{62}{38} \simeq 24.5\;\text{J\,mol}^{-1}\text{K}^{-1}.
\]

\subsection*{(b) Match with equipartition values}

\[
\begin{array}{lcc}
\text{Gas} & C_V/R & C_V\;(\text{J\,mol}^{-1}\text{K}^{-1}) \\ \hline
\text{Ar (monatomic)} & \tfrac{3}{2} & 12.5 \\
\text{H}_2\ \text{(diatomic)} & \tfrac{5}{2} & 20.8 \\
\text{NH}_3\ \text{(non‑linear polyatomic)} & 3 & 24.9
\end{array}
\]

The calculated $C_{V,\text{gas}}\approx 24.5$ J mol\(^{-1}\) K\(^{-1}\)
matches the non‑linear polyatomic value \(3R\).  Among the options, this
corresponds to \(\mathrm{NH_3}\).

\[
\boxed{\text{The gas could be NH}_3.}
\]
%-----------------------------------------------------------------
%  Gallium at its triple point: which phase is favoured
%  when pressure is increased slightly?
%-----------------------------------------------------------------

\section*{Given data at the triple point}

\[
T_{\text{tp}} = 302\;\text{K}, \qquad
p_{\text{tp}} = 101\;\text{kPa}.
\]

\[
\begin{array}{lcl}
\rho_{\text{solid}} &=& 5.91\;\text{g\,cm}^{-3}, \\
\rho_{\text{liquid}} &=& 6.05\;\text{g\,cm}^{-3}, \\
\rho_{\text{gas}}   &=& 0.116\;\text{g\,cm}^{-3}.
\end{array}
\]

\section*{Key thermodynamic idea}

At fixed temperature \(T\), the phase with the \emph{lowest molar
Gibbs free energy} \(G = \mu\) is the stable one.  Infinitesimally
away from the triple point, a change in pressure favours the phase
with the \textbf{smallest molar volume} \(V_{\!m}\), because

\[
\left(\frac{\partial \mu}{\partial p}\right)_T
   = V_{\!m}.
\]

\[
\boxed{\;
  \mu(p+\mathrm{d}p) = \mu(p_{\text{tp}}) + V_{\!m}\,\mathrm{d}p
\;}
\qquad(\mathrm{d}p>0).
\]

Thus increasing \(p\) lowers the Gibbs energy of the
high‑density (small‑volume) phase most strongly.

\section*{Compare molar volumes}

Since \(V_{\!m} = M/\rho\) (\(M\) = molar mass, same for all phases),
ordering by \(\rho\) is equivalent to ordering by~\(V_{\!m}\):

\[
\rho_{\text{liquid}} > \rho_{\text{solid}} \gg \rho_{\text{gas}}
\quad\Longrightarrow\quad
V_{m,\text{liquid}} < V_{m,\text{solid}} \ll V_{m,\text{gas}}.
\]

\section*{Conclusion}

Because the liquid has the \emph{highest density}
(\(\rho = 6.05\;\text{g\,cm}^{-3}\)), it possesses the \emph{smallest}
molar volume and therefore its chemical potential decreases most when
\(p\) is increased.  Consequently,

\[
\boxed{\text{A slight pressure increase stabilises the \emph{liquid}
phase.}}
\]
%------------------------------------------------------------------
%  Shift of the boiling point of water for a tiny pressure change
%------------------------------------------------------------------

\section*{1.\  Start from equality of chemical potentials}

Along the coexistence curve the molar chemical potentials of liquid
and vapour are equal: $\mu_{\ell}(T,p)=\mu_{g}(T,p)$.  
Differentiate that identity:

\[
\frac{V_{\ell}}{N}\,dp-\frac{S_{\ell}}{N}\,dT
  =\frac{V_{g}}{N}\,dp-\frac{S_{g}}{N}\,dT
\quad\Longrightarrow\quad
\boxed{\,\Delta V_m\,dp = \Delta S_m\,dT\,},
\]
where $\Delta V_m\!\equiv V_{g,m}-V_{\ell,m}$ and
$\Delta S_m\!\equiv S_{g,m}-S_{\ell,m}$.

\section*{2.\  Simplify the volume difference}

The molar volume of liquid water
$V_{\ell,m}\approx\!1.8\times10^{-5}\,\text{m}^3$  
is tiny compared with the vapour volume, so  
$\Delta V_m\simeq V_{g,m}$.

For an ideal gas at the atmospheric boiling point
($T_0 = 373.15\,$K,\; $p_0 = 1.013\times10^{5}\,$Pa):

\[
V_{g,m}
  = \frac{RT_0}{p_0}
  = \frac{(8.314\;\text{J\,mol}^{-1}\text{K}^{-1})(373.15\;\text{K})}
         {1.013\times10^{5}\;\text{Pa}}
  = 3.06\times10^{-2}\;\text{m}^{3}\!\!\;\text{mol}^{-1}.
\]

\section*{3.\  Entropy difference per mole}

Given $\displaystyle
\frac{\Delta S}{m}=6.05\times10^{3}\;\frac{\text{J}}{\text{kg\,K}}$  
and molar mass $M=0.018\;\text{kg\,mol}^{-1}$,

\[
\Delta S_m = \left(\frac{\Delta S}{m}\right) M
            = (6.05\times10^{3})(0.018)
            = 1.09\times10^{2}\;\text{J\,mol}^{-1}\text{K}^{-1}.
\]

\section*{4.\  Small-pressure formula}

From $\Delta V_m\,dp = \Delta S_m\,dT$:

\[
dT = \frac{\Delta V_m}{\Delta S_m}\,dp.
\]

Insert numbers with  
$dp=-0.16\;\text{Pa}$ (a \emph{decrease} in pressure):

\[
dT
  = \frac{3.06\times10^{-2}}{1.09\times10^{2}}(-0.16)
  = -4.5\times10^{-5}\;\text{K}.
\]

\section*{5.\  Result}

\[
\boxed{\;|\Delta T|\;\approx\;4.5\times10^{-5}\ \text{K}\;}
\]

A pressure change of only $0.16$ Pa (about $1.6\times10^{-6}$ of an
atmosphere) shifts the boiling point by \emph{just a few
$10^{-5}$ kelvin}.  The temperature change is so small that it is
practically undetectable—certainly nowhere near 34 K.
%-----------------------------------------------------------------
%  1‑D quantum harmonic oscillator in a thermal bath
%-----------------------------------------------------------------

\section*{Given}
\[
f = 7.0\times10^{11}\ \text{Hz},
\qquad
\varepsilon = h f,
\qquad
E_0 = 0,
\qquad
E_1 = \varepsilon .
\]

\[
h = 6.626\,070\,15\times10^{-34}\ \text{J\,s},
\quad
k_B = 1.380\,649\times10^{-23}\ \text{J\,K}^{-1}.
\]

%-----------------------------------------------------------------
\section*{(a) Temperature at which $P_1/P_0 = 1/3$}

Boltzmann factors for a single oscillator:

\[
P_n \propto e^{-E_n/(k_B T)} .
\]

Hence
\[
\frac{P_1}{P_0}
   = \exp\!\bigl[-(E_1-E_0)/(k_B T)\bigr]
   = e^{-\varepsilon/(k_B T)} .
\]

Set $P_1/P_0 = 1/3$:

\[
\frac{1}{3} = e^{-\varepsilon/(k_B T)}
\quad\Longrightarrow\quad
T = \frac{\varepsilon}{k_B \ln 3}.
\]

Numeric value of the level spacing:

\[
\varepsilon = h f
            = (6.626\,070\,15\times10^{-34})
              (7.0\times10^{11})
            = 4.638\times10^{-22}\ \text{J}.
\]

\[
T = \frac{4.638\times10^{-22}}
         {(1.380\,649\times10^{-23})(\ln 3)}
  = \frac{4.638\times10^{-22}}
         {1.512\times10^{-23}}
  \approx 30.6\ \text{K}.
\]

\[
\boxed{T \simeq 30.6\ \text{K}}
\]

%-----------------------------------------------------------------
\section*{(b) Behaviour of the ratio $P_1/P_0$ as $T\to\infty$}

\[
\frac{P_1}{P_0} = e^{-\varepsilon/(k_B T)}
\;\xrightarrow[T\to\infty]{}\;
e^{\,0} = 1.
\]

Thus at extremely high temperature the ground and first‑excited states
become equally populated:

\[
\boxed{\displaystyle \lim_{T\to\infty}\frac{P_1}{P_0}=1 }.
\]
\end{document}
