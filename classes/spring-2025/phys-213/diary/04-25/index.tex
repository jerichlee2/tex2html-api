\documentclass[12pt]{article}

% Packages
\usepackage[margin=1in]{geometry}
\usepackage{amsmath,amssymb,amsthm}
\usepackage{enumitem}
\usepackage{hyperref}
\usepackage{xcolor}
\usepackage{import}
\usepackage{xifthen}
\usepackage{pdfpages}
\usepackage{transparent}
\usepackage{listings}
\usepackage{tikz}
\usepackage{physics}
\usepackage{siunitx}
  \usetikzlibrary{calc,patterns,arrows.meta,decorations.markings}


\DeclareMathOperator{\Log}{Log}
\DeclareMathOperator{\Arg}{Arg}

\lstset{
    breaklines=true,         % Enable line wrapping
    breakatwhitespace=false, % Wrap lines even if there's no whitespace
    basicstyle=\ttfamily,    % Use monospaced font
    frame=single,            % Add a frame around the code
    columns=fullflexible,    % Better handling of variable-width fonts
}

\newcommand{\incfig}[1]{%
    \def\svgwidth{\columnwidth}
    \import{./Figures/}{#1.pdf_tex}
}
\theoremstyle{definition} % This style uses normal (non-italicized) text
\newtheorem{solution}{Solution}
\newtheorem{proposition}{Proposition}
\newtheorem{problem}{Problem}
\newtheorem{lemma}{Lemma}
\newtheorem{theorem}{Theorem}
\newtheorem{remark}{Remark}
\newtheorem{note}{Note}
\newtheorem{definition}{Definition}
\newtheorem{example}{Example}
\newtheorem{corollary}{Corollary}
\theoremstyle{plain} % Restore the default style for other theorem environments
%

% Theorem-like environments
% Title information
\title{}
\author{Jerich Lee}
\date{\today}

\begin{document}

\maketitle
\begin{align}
  \textbf{Given data}\\[2pt]
  &n = 2\;\text{mol}, \qquad 
  T = 290\;\text{K}, \qquad 
  V_i = 2\;\text{m}^3,\qquad 
  V_f = 1\;\text{m}^3.\\[6pt]
  \textbf{Isothermal work for an ideal gas}\\[2pt]
  &\displaystyle W_{\text{by}} = \int_{V_i}^{V_f} P\,dV = nRT\int_{V_i}^{V_f}\frac{dV}{V}
        = nRT\ln\!\left(\frac{V_f}{V_i}\right).\\[6pt]
  \textbf{Work \emph{on} the gas} \\[2pt]
  &\displaystyle W_{\text{on}} = -\,W_{\text{by}}
        = nRT\ln\!\left(\frac{V_i}{V_f}\right), \qquad 
        \text{since } W_{\text{on}} = -W_{\text{by}}.\\[6pt]
  \textbf{Numerical substitution}\\[2pt]
  &\displaystyle W_{\text{on}} 
     = (2\;\text{mol})(8.314\;\text{J mol}^{-1}\text{K}^{-1})(290\;\text{K})
       \ln\!\left(\frac{2}{1}\right) \\[4pt]
  &\displaystyle \phantom{W_{\text{on}} }
     = (4820\;\text{J})\times 0.6931 \\[4pt]
  &\displaystyle \phantom{W_{\text{on}} }
     \approx 3.34\times 10^{3}\;\text{J}.\\[6pt]
  \boxed{\,W_{\text{on}}\ \approx\ 3.34\times 10^{3}\ \text{J}\,}
  \end{align}
  \begin{align}
    \textbf{Data}\\[2pt]
    &n = 1\;\text{mol (H}_2\text{)}, 
    \quad V_1 = 1\;\text{m}^3,\;
    \quad V_2 = 0.4\;\text{m}^3,\;
    \quad p_1 = 1\;\text{atm}.\\[8pt]
    \textbf{1.\ Degrees of freedom and } \gamma \\[2pt]
    &N_{\text{DOF}} = 3\;(\text{translation}) + 2\;(\text{rotation}) = 5,\\[4pt]
    &\displaystyle 
    \gamma \;=\; \frac{N_{\text{DOF}}/2 + 1}{N_{\text{DOF}}/2}
              \;=\; \frac{5/2 + 1}{5/2}
              \;=\; \frac{3.5}{2.5}
              \;=\; 1.4.\\[10pt]
    \textbf{2.\ Final pressure from } p_1V_1^{\gamma}=p_2V_2^{\gamma}\\[4pt]
    &\displaystyle 
    p_2 \;=\; p_1\!\left(\frac{V_1}{V_2}\right)^{\gamma}
            \;=\; 1\;\text{atm}\,(2.5)^{1.4}
            \;\approx\; 3.61\;\text{atm}.\\[10pt]
    \textbf{3.\ Adiabatic work }(1\!\to\!2)\\[2pt]
    &\displaystyle 
    W \;=\; \int_{V_1}^{V_2} p\,dV
          \;=\; \frac{p_2V_2 - p_1V_1}{1-\gamma},\\[4pt]
    &\displaystyle 
    W \;=\; \frac{(3.61\;\text{atm})(0.4\;\text{m}^3) - (1\;\text{atm})(1\;\text{m}^3)}
                  {1 - 1.4}
         \;\approx\; -1.11\;\text{atm}\,\text{m}^3.\\[10pt]
    \textbf{4.\ Convert to joules}\\[2pt]
    &1\;\text{atm}\,\text{m}^3 = 101\,325\;\text{J},\\[4pt]
    &|W| = 1.11 \times 101\,325\;\text{J}
          \;\approx\; 1.1\times 10^{5}\;\text{J}.\\[8pt]
    \boxed{|W_{1\to 2}| \;\approx\; 1.1\times 10^{5}\ \text{J}}
    \end{align}
    \begin{align}
      \textbf{Adiabatic relation for an ideal gas} \\[4pt]
      & P V^{\gamma} = \text{constant} \;\;(\equiv C). \\[10pt]
      \textbf{1.~~Express the pressure as a function of volume} \\[4pt]
      & P = C V^{-\gamma}. \\[10pt]
      \textbf{2.~~Work for a quasi-static adiabatic path} \\[4pt]
      & W \;=\; \int_{V_1}^{V_2} P\,dV \;=\; \int_{V_1}^{V_2} C\,V^{-\gamma}\,dV. \\[10pt]
      \textbf{3.~~Integrate} \\[4pt]
      & W \;=\; C \int_{V_1}^{V_2} V^{-\gamma}\,dV
            \;=\; C\left[ \frac{V^{1-\gamma}}{1-\gamma} \right]_{V_1}^{V_2} \\[4pt]
      &\phantom{W} \;=\; \frac{C}{1-\gamma}\bigl(V_2^{1-\gamma} - V_1^{1-\gamma}\bigr). \\[10pt]
      \textbf{4.~~Replace the constant \(C\)} \\[4pt]
      & C = P_1 V_1^{\gamma} = P_2 V_2^{\gamma}. \\[10pt]
      \textbf{Using \(C = P_2 V_2^{\gamma}\)} \\[4pt]
      & W \;=\; \frac{P_2 V_2^{\gamma}}{1-\gamma}\,V_2^{\,1-\gamma}
               \;-\; \frac{P_2 V_2^{\gamma}}{1-\gamma}\,V_1^{\,1-\gamma} \\[4pt]
      &\phantom{W} \;=\; \frac{P_2 V_2 - P_2 V_2^{\gamma}V_1^{1-\gamma}}{1-\gamma}. \\[8pt]
      \textbf{But}\; P_2 V_2^{\gamma} = P_1 V_1^{\gamma} 
      \;\Longrightarrow\; P_2 V_2^{\gamma}V_1^{1-\gamma}=P_1V_1, \\[4pt]
      \therefore\; & W \;=\; \frac{P_2 V_2 - P_1 V_1}{1-\gamma}. \\[12pt]
      \boxed{\,W \;=\; \displaystyle \frac{P_2 V_2 - P_1 V_1}{1 - \gamma}\,}
      \end{align}
      \begin{align}
        \text{Adiabatic relation:}\qquad 
          & PV^{\gamma} = \text{const} 
             \;\;\Longrightarrow\;\;
             TV^{\gamma-1}= \text{const} \\[6pt]
        %
        \text{Relating the temperatures at states 3 and 4:}\qquad
          & T_3\,V_3^{\,\gamma-1} = T_4\,V_1^{\,\gamma-1} \\[4pt]
          & \Rightarrow 
            \frac{T_4}{T_3} 
            =\biggl(\frac{V_3}{V_1}\biggr)^{\gamma-1} \\[10pt]
        %
        \text{Root‐mean‐square speed:}\qquad
          & v_{\text{rms}} = \sqrt{\frac{3k_B T}{m}} \\[4pt]
          & \Rightarrow
            \frac{v_{\text{rms},\,4}}{v_{\text{rms},\,3}}
            = \sqrt{\frac{T_4}{T_3}}
            = \biggl(\frac{V_3}{V_1}\biggr)^{(\gamma-1)/2} \\[10pt]
        %
        \text{Numerical values for diatomic }(\mathrm{H_2})\text{:}\qquad
          & \gamma = \frac{7}{5} = 1.4,
            \quad \gamma - 1 = \frac{2}{5} \\[4pt]
          & \frac{V_3}{V_1} = \frac{0.5\;\text{m}^3}{1\;\text{m}^3} 
                             = \tfrac12 \\[10pt]
        %
        \text{Final ratio of rms speeds:}\qquad
          & \frac{v_{\text{rms},\,4}}{v_{\text{rms},\,3}}
            = \bigl(\tfrac12\bigr)^{\,(\gamma-1)/2}
            = \bigl(\tfrac12\bigr)^{\,1/5}
            = 2^{-1/5} \\[6pt]
        %
        & \boxed{\,v_{\text{rms,\,final}}
                 \;=\;
                 2^{-1/5}\,v_{\text{rms,\,initial}}\,}
        \end{align}
        \begin{align}
          \textbf{1.\ Helmholtz free energy at fixed $T$ and $n$} \qquad
            & F(V) = U - TS
                    = \frac{3}{2}\,nRT + \frac{a n^{2}}{V}
                      - T\,nR\ln V                                           \label{eq:F}\\[8pt]
          %
          \textbf{2.\ Mechanical equilibrium (isothermal compression)}\qquad
            &\text{Minimise }G(V)=F(V)+P_{\text{ext}}V
              \;\Longrightarrow\;
              \left(\frac{\partial G}{\partial V}\right)_{T,n}=0\\[2pt]
            &\hspace{1.7cm}
              \frac{\partial F}{\partial V} + P_{\text{ext}} = 0
              \;\;\Longrightarrow\;\;
              P_{\text{ext}} = -\,\frac{\partial F}{\partial V}\Big|_{T,n}     \label{eq:Pext}\\[10pt]
          %
          \textbf{3.\ Evaluate the derivative of $F$} \\[2pt]
            & \frac{\partial F}{\partial V}
                = -\,\frac{a n^{2}}{V^{2}} \;-\; T\,nR\,\frac{1}{V}           \label{eq:dFdV}\\[6pt]
          %
          \textbf{4.\ Insert (\ref{eq:dFdV}) into (\ref{eq:Pext})} \\[2pt]
            & P_{\text{ext}}
                = \frac{a n^{2}}{V^{2}} + \frac{T\,nR}{V}                     \label{eq:Pformula}\\[12pt]
          %
          \textbf{5.\ Numerical substitution} \\[2pt]
            & n = 3.3\;\text{mol}, \quad a = 20.18\;\text{J\,m}^{3}\!/\text{mol}^{2},
              \quad V = 0.152\;\text{m}^{3}, \quad
              T = 211.8\;\text{K}, \quad
              R = 8.314\;\text{J\,mol}^{-1}\text{K}^{-1} \\[4pt]
          %
            & P_{\text{ext}}
                = \frac{(20.18)(3.3)^{2}}{(0.152)^{2}}
                  \;+\;
                  \frac{(211.8)(3.3)(8.314)}{0.152} \\[6pt]
            & \phantom{P_{\text{ext}}}
                \approx 9.5 \times 10^{3}\;\text{Pa}
                  \;+\;
                  3.8 \times 10^{4}\;\text{Pa} \\[4pt]
            & \phantom{P_{\text{ext}}}
                \approx 4.8 \times 10^{4}\;\text{Pa} \\[10pt]
          %
          \boxed{\,P_{\text{ext}} \;\approx\; 4.8 \times 10^{4}\ \text{Pa}
                 \;\;(\text{about } 0.47\ \text{atm})\,}
          \end{align}
          
          % ----------  Ratio of rms speeds: He vs. N2  ----------
% (compact layout so every line stays inside the margins)

% 1. rms speed of a single species
\begin{align}
  v_{\text{rms}}
    &= \sqrt{\frac{3 k_B T}{m}}
  \end{align}
  
  % 2. Ratio at the same temperature
  \begin{align}
  \frac{v_{\text{rms,He}}}{v_{\text{rms,N}_2}}
    &= \sqrt{\frac{m_{N_2}}{m_{\text{He}}}}
  \end{align}
  
  % 3. Molar–to–molecule masses (Avogadro factor cancels)
  \begin{align}
  m = \frac{M}{N_A}, \qquad
  M_{\text{He}} = 4.0\times10^{-3}\,\text{kg\,mol}^{-1}, \qquad
  M_{N_2} = 2.8\times10^{-2}\,\text{kg\,mol}^{-1}
  \end{align}
  
  % 4. Numerical evaluation
  \begin{align}
  \frac{v_{\text{rms,He}}}{v_{\text{rms,N}_2}}
    &= \sqrt{\frac{M_{N_2}}{M_{\text{He}}}}
     = \sqrt{\frac{28}{4}}
     = \sqrt{7}
     \approx 2.65
  \end{align}
  
  \[
  \boxed{\displaystyle
    \frac{v_{\text{rms,He}}}{v_{\text{rms,N}_2}} \;\approx\; 2.65
  }
  \]
  % ----------  Why the particle counts do not affect the rms-speed ratio  ----------

\paragraph{Equipartition in a mixture.}
In equilibrium, \emph{every} species in the mixture shares the same
temperature~\(T\);
only their molecular masses differ.
The root–mean–square speed of \emph{one particle} of any species is
\[
v_{\text{rms}}
   = \sqrt{\frac{3k_B T}{m}},
\]
where \(m\) is the mass of one particle of that species.
The formula contains no factor that depends on how \emph{many}
such particles are present.

\paragraph{Including the given counts.}
Let \(N_{\text{He}}=3.0\times10^{23}\) and
\(N_{N_2}=6.0\times10^{23}\).
If we wrote the rms speed in terms of molar mass \(M\),
\[
v_{\text{rms}}
   = \sqrt{\frac{3 R T}{M}},
\]
nothing changes, because \(R=N_A k_B\) and
\(\,M=m N_A\); again, \(N_{\text{He}}\) or \(N_{N_2}\) never enter.

\paragraph{Ratio of rms speeds.}
Hence the ratio remains
\[
\frac{v_{\text{rms,He}}}{v_{\text{rms,}N_2}}
   = \sqrt{\frac{M_{N_2}}{M_{\text{He}}}}
   = \sqrt{\frac{28}{4}}
   = \sqrt{7}
   \approx 2.65.
\]

\[
\boxed{\displaystyle
  \frac{v_{\text{rms,He}}}{v_{\text{rms,}N_2}}\approx 2.65
}
\]

\paragraph{Take-away.}
The \emph{number} of particles influences macroscopic quantities
like total energy or pressure, but not the
\emph{per-particle} rms speed, so those \(3\times10^{23}\) He atoms
and \(6\times10^{23}\) \(N_2\) molecules do not affect the ratio.

% ----------  Entropy–change signs for a Stirling cycle  ----------

\textbf{Stirling‐cycle heat exchanges}  
\begin{itemize}
  \item \textbf{1 $\;$Isothermal expansion at $T_h$}\\[-4pt]
        Heat $Q_1>0$ enters the \emph{working gas} from the hot reservoir.  
        For the reservoir  
        \[
            \Delta S_{\text{hot},1}
              = \frac{Q_{\text{res}}}{T_h}
              = \frac{-Q_1}{T_h}<0.
        \]
  \item \textbf{2 $\;$Isochoric cooling}  
        Heat is transferred only between the gas and the regenerator 
        (no reservoir involved):  
        \[
            \Delta S_{\text{hot},2}=0, \qquad
            \Delta S_{\text{cold},2}=0.
        \]
  \item \textbf{3 $\;$Isothermal compression at $T_c$}\\[-4pt]
        Heat $Q_3<0$ leaves the gas and enters the cold reservoir.  
        For the reservoir  
        \[
            \Delta S_{\text{cold},3}
              = \frac{-Q_3}{T_c}>0.
        \]
  \item \textbf{4 $\;$Isochoric heating}  
        Again, heat is exchanged only with the regenerator, so  
        \[
            \Delta S_{\text{hot},4}=0, \qquad
            \Delta S_{\text{cold},4}=0.
        \]
\end{itemize}

\textbf{Conclusion}  
During the isothermal legs,
\[
    \boxed{\;
        \Delta S_{\text{hot},1}<0, \qquad 
        \Delta S_{\text{cold},3}>0
    \;}
\]
while the isochores leave both reservoirs’ entropies unchanged.
Therefore, the correct option is  
\[
\text{(a) }\; \Delta S_{\text{hot},1}<0,\; \Delta S_{\text{cold},3}>0.
\]
% ----------  Maximum indoor temperature for a heat-pump system ----------
% (All equations kept narrow enough to fit typical page margins)

\begin{align}
  \textbf{Given data:}\qquad
    &Q_{\text{leak}} = 24\;\text{kW}, \quad
     W_{\text{in}}   = 4.8\;\text{kW}, \quad
     T_c            = 257.15\;\text{K}
  \end{align}
  
  \begin{align}
  \textbf{1.\  Required heating rate}\qquad
    Q_h &= Q_{\text{leak}}
         = 24\;\text{kW}
  \end{align}
  
  \begin{align}
  \textbf{2.\  Coefficient of performance (observed)}\qquad
    \text{COP}_{\text{obs}}
    &= \frac{Q_h}{W_{\text{in}}}
    = \frac{24\;\text{kW}}{4.8\;\text{kW}}
    = 5
  \end{align}
  
  \begin{align}
  \textbf{3.\  Ideal (Carnot) COP for a heat pump}\qquad
    \text{COP}_{\text{Carnot}}
    &= \frac{T_h}{T_h - T_c}
  \end{align}
  
  \begin{align}
  \textbf{4.\  Set }\text{COP}_{\text{Carnot}} = \text{COP}_{\text{obs}}\qquad
    \frac{T_h}{T_h - T_c} &= 5
    \label{eq:cop_equate}
  \end{align}
  
  \begin{align}
  \text{Solve \eqref{eq:cop_equate}:}\qquad
    T_h &= 5\,(T_h - T_c)\\
    4T_h &= 5T_c\\
    T_h &= \frac{5}{4}\,T_c
         = 1.25 \times 257.15\;\text{K}\\
    T_h &\approx 321\;\text{K}
  \end{align}
  
  \begin{align}
  \textbf{5.\  Convert to Celsius}\qquad
    T_h &= 321\;\text{K} - 273.15
         \approx 48.3^{\circ}\text{C}
  \end{align}
  
  \[
  \boxed{
    T_h^{\max} \approx 321\ \text{K}\;(\;48^{\circ}\text{C}\;)
  }
  \]
  
  \emph{Interpretation:}  
  With a $4.8\,$kW electrical input and a $24\,$kW heat loss, an \textbf{ideal}
  heat pump could maintain an indoor temperature of roughly
  $48^{\circ}\text{C}$ ($321\,$K) when the outdoor air is at $257\,$K.
  Real‐world heat pumps have lower COPs, so the attainable indoor
  temperature would be lower in practice.
\end{document}
