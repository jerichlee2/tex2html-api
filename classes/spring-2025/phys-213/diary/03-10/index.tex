\documentclass[12pt]{article}

% Packages
\usepackage[margin=1in]{geometry}
\usepackage{amsmath,amssymb,amsthm}
\usepackage{enumitem}
\usepackage{hyperref}
\usepackage{xcolor}
\usepackage{import}
\usepackage{xifthen}
\usepackage{pdfpages}
\usepackage{transparent}
\usepackage{listings}


\lstset{
    breaklines=true,         % Enable line wrapping
    breakatwhitespace=false, % Wrap lines even if there's no whitespace
    basicstyle=\ttfamily,    % Use monospaced font
    frame=single,            % Add a frame around the code
    columns=fullflexible,    % Better handling of variable-width fonts
}

\newcommand{\incfig}[1]{%
    \def\svgwidth{\columnwidth}
    \import{./Figures/}{#1.pdf_tex}
}
\theoremstyle{definition} % This style uses normal (non-italicized) text
\newtheorem{solution}{Solution}
\newtheorem{proposition}{Proposition}
\newtheorem{problem}{Problem}
\newtheorem{lemma}{Lemma}
\newtheorem{theorem}{Theorem}
\newtheorem{remark}{Remark}
\newtheorem{note}{Note}
\newtheorem{definition}{Definition}
\newtheorem{example}{Example}
\theoremstyle{plain} % Restore the default style for other theorem environments
%

% Theorem-like environments
% Title information
\title{Lecture 1}
\author{Jerich Lee}
\date{\today}

\begin{document}

\maketitle
\subsubsection*{Internal Energy, Work, and Heat}
Main equations:
\begin{align}
    ME_i + U_i = ME_f + U_f \\[10pt] 
    dU = dQ-pdV
\end{align}

\begin{definition}
   Mechanical Energy: (macroscopic kinetic energy and potential energy) — this is usually ignored in thermo!
\end{definition}
\begin{definition}
    Macroscopic quantities:
   \noindent
   \begin{enumerate}
    \item U: internal energy
    \item Q: heat
    \item V: volume
    \item p: pressure
   \end{enumerate} 
   
\end{definition}
\subsubsection*{Work and Heat}
\begin{align}
    \Delta U = W_{on} + Q
\end{align}
Work is the energy we see get put into the system macroscopically. \\
Heat transfer $Q$ is spontaneous movement of energy from hot objects to cold objects (microscopic).

\subsubsection*{Heat Transfer}
Heat flows \emph{spontaneously} from a high temp object to a low temp object.
\begin{align}
    \Delta U_{cold} = Q \\[10pt] 
    \Delta U_{hot} = -Q
\end{align}
\subsubsection*{Work On}
\begin{align}
    dW_{on} = -pdV \\[10pt] 
    W_{on} = -\int_{}^{} p \,\mathrm{d}V \\[10pt] 
    W_{by} = \int_{}^{} p \,\mathrm{d}V = -W_{on} 
\end{align}
Work on system (apply force that changes its volume), then we increase its internal energy.

\subsubsection*{Differential Notation}

Usually only know derivatives:
\begin{align}
    \Delta U = \int_{t_{i}}^{t_{f}} \frac{dU}{dt} \,\mathrm{d}t = U(t_{f})- U(t_{i})= U_{f}-U_{i}\\[10pt] 
    \Delta U = \int_{U_{i}}^{U_{f}}  \,\mathrm{d}U \\[10pt] 
    dU = dQ - pdV \label{1st}
\end{align}
\autoref{1st} is the First Law of Thermodynamics.
\end{document}
