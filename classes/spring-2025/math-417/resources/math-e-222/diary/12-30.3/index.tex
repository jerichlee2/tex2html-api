\documentclass[12pt]{article}

% Packages
\usepackage[margin=1in]{geometry}
\usepackage{amsmath,amssymb,amsthm}
\usepackage{enumitem}
\usepackage{hyperref}
\usepackage{xcolor}
\usepackage{import}
\usepackage{xifthen}
\usepackage{pdfpages}
\usepackage{transparent}
\usepackage{listings}


\lstset{
    breaklines=true,         % Enable line wrapping
    breakatwhitespace=false, % Wrap lines even if there's no whitespace
    basicstyle=\ttfamily,    % Use monospaced font
    frame=single,            % Add a frame around the code
    columns=fullflexible,    % Better handling of variable-width fonts
}

\newcommand{\incfig}[1]{%
    \def\svgwidth{\columnwidth}
    \import{./Figures/}{#1.pdf_tex}
}
\theoremstyle{definition} % This style uses normal (non-italicized) text
\newtheorem{solution}{Solution}
\newtheorem{proposition}{Proposition}
\newtheorem{problem}{Problem}
\newtheorem{lemma}{Lemma}
\newtheorem{theorem}{Theorem}
\newtheorem{remark}{Remark}
\newtheorem{note}{Note}
\theoremstyle{plain} % Restore the default style for other theorem environments
%

% Theorem-like environments
% Title information
\title{Generalities on groups; Symmetric groups on n letters; A stabilizer subgroup; The subgroups of Z; Cyclic subgroups gen by element}
\author{Jerich Lee}
\date{\today}

\begin{document}

\maketitle
\subsubsection*{HW: 2.1.5, 2.1.7, 2.2.1, 2.2.15, 2.2.20(a)} 
\subsubsection*{Reading: 2.1, 2.2} 

\begin{problem}[2.1.5]
    Assume that the equation $xyz=1$ holds in a group $G$ . Does it follow that $yzx=1$? That $yxz=1$?
\end{problem}
\begin{solution}
    
\end{solution}
\begin{problem}[2.1.7]
    Let $S$ be any set. Prove that the law of composition defined by $ab=a$ is associative.
\end{problem}
\begin{solution}
    
\end{solution}
\begin{problem}[2.2.1]
   Determine the elements of the cyclic group generated by the matrix 
   \begin{align}
    \begin{bmatrix}
        1 &  1 \\
        -1 &  0 \\
    \end{bmatrix}
   \end{align} 
   explicitly.
\end{problem}
\begin{solution}
    
\end{solution}
\begin{problem}[2.2.15]
   \noindent
   \begin{enumerate}
    \item In the definition of subgroup, the identity element in $H$ is required to be the identity of $G$ . One might require only that $H$ have an identity element, not that it is the same as the identity in $G$ . Show that if $H$ has an identity at all, then it is the identity in $G$, so this definition would be equivalent to the one given.
    \item Show the analogous thing for inverses.
   \end{enumerate} 
\end{problem}
\begin{solution}
    
\end{solution}
\begin{problem}[2.2.20]
   \noindent
   \begin{enumerate}
    \item Let $a,b$ be elements of an abelian group of orders $m,n$ respectively. What can you say about the order of their product $ab$?
    \item Show by example that the product of elements of finite order in a nonabelian group need not have finite order.
   \end{enumerate} 
\end{problem}
\begin{solution}
    
\end{solution}
\end{document}
