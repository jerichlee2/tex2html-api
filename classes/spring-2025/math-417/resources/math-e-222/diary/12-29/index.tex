\documentclass[12pt]{article}

% Packages
\usepackage[margin=1in]{geometry}
\usepackage{amsmath,amssymb,amsthm}
\usepackage{enumitem}
\usepackage{hyperref}
\usepackage{xcolor}
\usepackage{import}
\usepackage{xifthen}
\usepackage{pdfpages}
\usepackage{transparent}
\usepackage{listings}


\lstset{
    breaklines=true,         % Enable line wrapping
    breakatwhitespace=false, % Wrap lines even if there's no whitespace
    basicstyle=\ttfamily,    % Use monospaced font
    frame=single,            % Add a frame around the code
    columns=fullflexible,    % Better handling of variable-width fonts
}

\newcommand{\incfig}[1]{%
    \def\svgwidth{\columnwidth}
    \import{./Figures/}{#1.pdf_tex}
}
\theoremstyle{definition} % This style uses normal (non-italicized) text
\newtheorem{solution}{Solution}
\newtheorem*{proposition}{Proposition}
\newtheorem{problem}{Problem}
\newtheorem{lemma}{Lemma}
\newtheorem{theorem}{Theorem}
\theoremstyle{plain} % Restore the default style for other theorem environments
%

% Theorem-like environments
% Title information
\title{HW 1—MATH E-222: Abstract Algebra I}
\author{Jerich Lee}
\date{\today}

\begin{document}

\maketitle
\noindent
Read $\S 1.1$ and pages $38-42$ in Artin.  
\begin{problem}[1.1.7]
    Find a formula for 
    \begin{align}
        \begin{bmatrix}
        1 & 1 &  1 \\
        0 & 1 &  1 \\
        0 & 0 &  1 \\
    \end{bmatrix}^{n}
    \end{align}, and prove it by induction. 
\end{problem}
\begin{solution}

    \begin{proof}
         I initially thought to use the following series:
    $\sum_{k=1}^{n} (-1)^{k}k^{2}$. But I realized I was being silly and instead used the following:
    \begin{align}
     \begin{bmatrix}
        1 & n &  \frac{n(n+1)}{2} \\
         0 & 1 & n  \\
         0& 0 & 1  \\
    \end{bmatrix}+\begin{bmatrix}
        0 & 1 &n+1   \\
         0&0  &1  \\
         0&0  &0   \\
    \end{bmatrix}=\begin{bmatrix}
        1 & n &  \frac{n(n+1)}{2} \\
         0 & 1 & n  \\
         0& 0 & 1  \\
    \end{bmatrix}\cdot \begin{bmatrix}
        1 & 1 &  1 \\
        0 & 1 &  1 \\
        0 & 0 &  1 \\
    \end{bmatrix} 
    \end{align}
    \end{proof}
    \end{solution}
\begin{problem}[1.1.16]
A square matrix $A$ is called \emph{nilpotent} if $A^{k}=0$ for some $k>0$. Prove that if $A$ is nilpotent, then $I+A$ is invertible.   
\end{problem}
\begin{solution}
\begin{align}
    \underbrace{\begin{bmatrix}
    1 &  0 \\
    0 &  1 \\
   \end{bmatrix}}_{\mathbf{I} }+ \begin{bmatrix}
    a &b   \\
     c&d   \\
   \end{bmatrix} = \begin{bmatrix}
    a+1 &  b \\
    c & d+1  \\
   \end{bmatrix}
\end{align}
      \begin{align}
    det(I+A)= (a+1)(b+1)-cb \\[10pt] 
    =ad+a+d+1-cb = 0\\[10pt] 
    ad-cb = -(a+d+1)\\[10pt] 
    =det(A) \\[10pt] 
    \neq 0 
   \end{align}  
   Need to show that if $A$ is nilpotent then $\det(A)=0$. 
\end{solution}
\begin{problem}[1.1.17]
   \begin{enumerate}
    \item Find infinitely many matrices $B$ such that $BA=I_2$ when 
    \begin{align}
        A= \begin{bmatrix}
            2 &  3 \\
            1 &  2 \\
            2 &  5 \\
        \end{bmatrix}
    \end{align}
    \item Prove that there is no matrix $C$ such that $AC=I_3$. 
   \end{enumerate} 
\end{problem}
\begin{solution}

\end{solution}
\end{document}
