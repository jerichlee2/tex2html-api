\documentclass[12pt]{article}

% Packages
\usepackage[margin=1in]{geometry}
\usepackage{amsmath,amssymb,amsthm}
\usepackage{enumitem}
\usepackage{hyperref}
\usepackage{xcolor}
\usepackage{import}
\usepackage{xifthen}
\usepackage{pdfpages}
\usepackage{transparent}
\usepackage{listings}


\lstset{
    breaklines=true,         % Enable line wrapping
    breakatwhitespace=false, % Wrap lines even if there's no whitespace
    basicstyle=\ttfamily,    % Use monospaced font
    frame=single,            % Add a frame around the code
    columns=fullflexible,    % Better handling of variable-width fonts
}

\newcommand{\incfig}[1]{%
    \def\svgwidth{\columnwidth}
    \import{./Figures/}{#1.pdf_tex}
}
\theoremstyle{definition} % This style uses normal (non-italicized) text
\newtheorem{solution}{Solution}
\newtheorem*{proposition}{Proposition}
\newtheorem{problem}{Problem}
\newtheorem{lemma}{Lemma}
\newtheorem{theorem}{Theorem}
\newtheorem{note}{Note}
\theoremstyle{plain} % Restore the default style for other theorem environments
%

% Theorem-like environments
% Title information
\title{Equivalence relations; Cosets; Examples}
\author{Jerich Lee}
\date{\today}

\begin{document}

\maketitle
\begin{note}
    The group $ S_n $ consists of all permutations of $ \{1, 2, \dots, n\} $. Each permutation is determined by $ n $ choices for the image of $ 1 $, then $ n-1 $ choices for the image of $ 2 $, and so on. Hence:
    \begin{align}
        |S_n| \;=\; n!.
    \end{align}
    Define:
    \begin{align}
        f \colon S_n \to \{\pm1\}
    \end{align}
    by:
    \begin{align}
        f(\sigma) \;=\; 
        \begin{cases}
            +1, & \text{if $\sigma$ is an even permutation}, \\
            -1, & \text{if $\sigma$ is an odd permutation}.
        \end{cases}
    \end{align}
    This is well-defined (the parity of transposition decompositions is invariant) and can be easily checked to be a group homomorphism.
    \begin{enumerate}
        \item The identity permutation is even ($f(e) = +1$).
        \item Any transposition $(i\, j)$ is odd ($f((i\, j)) = -1$).
    \end{enumerate}
    Thus, $f$ is onto $\{\pm1\}$.

    By definition:
    \begin{align}
        \ker(f) \;=\; \{\sigma \in S_n \mid f(\sigma) = +1\} 
        \;=\; \text{(even permutations)} 
        \;=\; A_n.
    \end{align}

    Since $f$ is surjective:
    \begin{align}
        |S_n| \;=\; |\ker(f)| \times |f(S_n)|
        \;\;\Longrightarrow\;\;
        n! \;=\; |A_n| \times 2 
        \;\;\Longrightarrow\;\;
        |A_n| \;=\; \frac{n!}{2}.
    \end{align} 
\end{note}

\begin{note}
    A group $ G $ is cyclic if there exists an element $ g \in G $ such that:
    \begin{align}
        G = \langle g \rangle = \{ g^n : n \in \mathbb{Z} \}.
    \end{align}
    Equivalently (in additive notation):
    \begin{align}
        G = \langle a \rangle = \{ na : n \in \mathbb{Z} \}.
    \end{align}
    Examples of cyclic groups include:
    \begin{align}
        \mathbb{Z}, \quad \mathbb{Z}_n, \quad \text{the } n\text{-th roots of unity in } \mathbb{C}.
    \end{align}
    Note:
    \begin{align}
        \text{All cyclic groups are abelian, but not all abelian groups are cyclic.}
    \end{align}
\end{note}

\begin{note}
    Consider $ S_3 $:
    \begin{align}
        S_3 = \{\mathrm{id}, (12), (13), (23), (123), (132)\}.
    \end{align}
    Properties of $ S_3 $:
    \begin{enumerate}
        \item Order: $ |S_3| = 3! = 6 $.
        \item Operation: composition of permutations.
        \item Properties: non-abelian, no element of order $6$ (hence not cyclic).
    \end{enumerate}
\end{note}


\subsubsection*{Reading} $\S$$\S$  2.5, 2.6
\subsubsection*{HW:} 2.5.1, 2.5.6, 2.6.2, 2.6.4, 2.6.5

\subsubsection*{Reading for next lecture:} $\S$ 2.9

\begin{problem}[2.5.1]
    Prove that the nonempty fibres of a map form a partition of the domain.
\end{problem}
\begin{solution}
    
\end{solution}
\begin{problem}[2.5.6]
    \noindent
   \begin{enumerate}
    \item Prove that the relation $x$ conjugate to $y$ in a group $G$ is an equivalence relation on $G$ .
    \item Describe the elements $a$ whose conjugacy class ($=$ equivalence class) consists of the element $a$ alone.
   \end{enumerate} 
\end{problem}
\begin{solution}
    
\end{solution}
\begin{problem}[2.6.2]
   Prove directly that distinct cosets do not overlap. 
\end{problem}
\begin{solution}
    
\end{solution}
\begin{problem}[2.6.4]
   Give an example showing that left cosets and right cosets of $GL_2(\mathbb{{R}})$ in $GL_2(\mathbb{{C}})$ are not always equal.
\end{problem}
\begin{solution}
    
\end{solution}
\begin{problem}[2.6.5]
   Let $H,K$ be subgroups of a group $G$ of orders $3,5$ respectfully. Prove that $H \cap K=\left\{ 1 \right\} $  . 
\end{problem}
\begin{solution}
    
\end{solution}
\end{document}
