\documentclass[12pt]{article}

% Packages
\usepackage[margin=1in]{geometry}
\usepackage{amsmath,amssymb,amsthm}
\usepackage{enumitem}
\usepackage{hyperref}
\usepackage{xcolor}
\usepackage{import}
\usepackage{xifthen}
\usepackage{pdfpages}
\usepackage{transparent}
\usepackage{listings}


\lstset{
    breaklines=true,         % Enable line wrapping
    breakatwhitespace=false, % Wrap lines even if there's no whitespace
    basicstyle=\ttfamily,    % Use monospaced font
    frame=single,            % Add a frame around the code
    columns=fullflexible,    % Better handling of variable-width fonts
}

\newcommand{\incfig}[1]{%
    \def\svgwidth{\columnwidth}
    \import{./Figures/}{#1.pdf_tex}
}
\theoremstyle{definition} % This style uses normal (non-italicized) text
\newtheorem{solution}{Solution}
\newtheorem{proposition}{Proposition}
\newtheorem{problem}{Problem}
\newtheorem{lemma}{Lemma}
\newtheorem{theorem}{Theorem}
\newtheorem{remark}{Remark}
\newtheorem{note}{Note}
\theoremstyle{plain} % Restore the default style for other theorem environments
%

% Theorem-like environments
% Title information
\title{Kernels, normality; Examples; Centers and inner autos}
\author{Jerich Lee}
\date{\today}

\begin{document}

\maketitle
\subsubsection*{HW: 1.4.5} 
\subsubsection*{Reading: 1.4} 

\begin{problem}[]
   Let $V$ denote the Klein $4$-group. Show that $\text{Aut}(V)$ is isomorphic to $S_3$.  
\end{problem}
\begin{solution}
    
\end{solution}
\begin{problem}[]
   Define $f:\text{GL}_{n} (\mathbb{{R}})\to \text{GL}_n(\mathbb{{R}})  $  by $f(A)=^{t}A^{-1}$ (where $^{t}A$ is the transpose of $A$). Show that $f$ is an automorphism, but not an inner automorphism for $n\geq 1$. 
\end{problem}
\begin{solution}
   How do I solve $^{T}A^{-1}= gAg^{-1}$ to show that $f:\text{GL}_{n} (\mathbb{{R}})\to \text{GL}_n(\mathbb{{R}})  $  by $f(A)=^{t}A^{-1}$ (where $^{t}A$ is the transpose of $A$). Show that $f$ is an automorphism, but not an inner automorphism for $n\geq 1$?
\end{solution}
\begin{problem}[1.4.5]
    Prove that the transpose of a permutation matrix $P$ is its inverse.

\end{problem}
\begin{solution}
    
\end{solution}
\end{document}
