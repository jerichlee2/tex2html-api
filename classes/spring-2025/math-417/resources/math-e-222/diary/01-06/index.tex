\documentclass[12pt]{article}

% Packages
\usepackage[margin=1in]{geometry}
\usepackage{amsmath,amssymb,amsthm}
\usepackage{enumitem}
\usepackage{hyperref}
\usepackage{xcolor}
\usepackage{import}
\usepackage{xifthen}
\usepackage{pdfpages}
\usepackage{transparent}
\usepackage{listings}

\lstset{
    breaklines=true,         % Enable line wrapping
    breakatwhitespace=false, % Wrap lines even if there's no whitespace
    basicstyle=\ttfamily,    % Use monospaced font
    frame=single,            % Add a frame around the code
    columns=fullflexible,    % Better handling of variable-width fonts
}

\newcommand{\incfig}[1]{%
    \def\svgwidth{\columnwidth}
    \import{./Figures/}{#1.pdf_tex}
}
\theoremstyle{definition} % Normal (non-italicized) text
\newtheorem{solution}{Solution}
\newtheorem{proposition}{Proposition}
\newtheorem{problem}{Problem}
\newtheorem{lemma}{Lemma}
\newtheorem{theorem}{Theorem}
\newtheorem{remark}{Remark}
\newtheorem{note}{Note}
\theoremstyle{plain} % Restore the default style for other theorem environments

% Title information
\title{Congruence mod n; (Z/nZ)*}
\author{Jerich Lee}
\date{\today}

\begin{document}
\maketitle
Do exercises $2.9.2, 2.9.4, 2.9.5, 2.9.8$
Read $\S 2.10$  
\begin{problem}[2.9.2]
   \noindent
   \begin{enumerate}
    \item Prove that the square $a^{2}$ of an integer $a$ is congruent to $0$ or $1$ modulo $4$.
    \item What are the possible values of $a^{2}$ modulo $8$?    
   \end{enumerate} 
\end{problem}
\begin{solution}
    
\end{solution}
\begin{problem}[2.9.4]
   Prove that every integer $a$ is congruent to the sum of its decimal digits modulo $9$.  
\end{problem}
\begin{solution}
    
\end{solution}
\begin{problem}[2.9.5]
   Solve the congruence $2x \equiv 5$:
   \noindent
   \begin{enumerate}
    \item modulo $9$
    \item modulo $6$  
   \end{enumerate}  
\end{problem}
\begin{solution}
    
\end{solution}
\begin{problem}[2.9.8]
   Use Proposition (2.6) to prove the Chinese Remainder Theorem: Let $m,n,a,b$ be integers, and assume that the greatest common divisor of $m$ and $n$ is $1$. Then there is an integer $x$ such that $x \equiv a$ (modulo $m$) and $x \equiv b$(modulo $n$).
   \begin{proposition}[2.6]
    Let $a,b$ be integers, not both zero, and let $d$ be the positive integer which generates the subgroup $a\mathbb{{Z}}+b\mathbb{{Z}}$. Then
    \noindent
    \begin{enumerate}
        \item $d$ can be written in the form $d=ar + bs$ for some integers $r$ and $s$ .
        \item $d$ divides $a$ and $b$ .
        \item If an integer $e$ divides $a$ and $b$, it also divides $d$ .
    \end{enumerate} 
   \end{proposition}     
\end{problem}
\begin{solution}
  
\begin{problem}[2.9.8]
Use Proposition (2.6) to prove the Chinese Remainder Theorem: Let $m,n,a,b$ be integers, and assume that $\gcd(m,n) = 1$. Then there is an integer $x$ such that 
\begin{align}
x \equiv a \ (\mathrm{mod}\ m)
\quad \text{and} \quad
x \equiv b \ (\mathrm{mod}\ n).
\end{align}
\end{problem}

\begin{proposition}[2.6]
Let $a,b$ be integers, not both zero, and let $d$ be the positive integer which generates the subgroup $a\mathbb{Z} + b\mathbb{Z}$. Then:
\begin{enumerate}
\item $d$ can be written in the form $d = ar + bs$ for some integers $r$ and $s$.
\item $d$ divides $a$ and $b$.
\item If an integer $e$ divides $a$ and $b$, it also divides $d$.
\end{enumerate}
\end{proposition}

\begin{theorem}[Chinese Remainder Theorem for $\gcd(m,n) = 1$]
Suppose $\gcd(m,n) = 1$. Let $m,n,a,b$ be integers. Then there exists an integer $x$ satisfying
\begin{align}
\begin{cases}
x \equiv a \pmod{m},\\[5pt]
x \equiv b \pmod{n}.
\end{cases}
\end{align}
\end{theorem}


\section*{Statement of the Problem}

\begin{theorem}[Chinese Remainder Theorem for Two Moduli]
Let $m$ and $n$ be coprime positive integers, and let $a,b$ be integers. The system of congruences
\begin{align}
\begin{cases}
x \equiv a \pmod{m}, \\[6pt]
x \equiv b \pmod{n}
\end{cases}
\end{align}
has a unique solution modulo $mn$. In fact, there is an explicit formula:
\begin{align}
x \;\equiv\; a \;+\; m\bigl((b - a)\,m^{-1} \bmod n\bigr)
\;\pmod{mn},
\end{align}
where $m^{-1}$ is the multiplicative inverse of $m$ modulo $n$, i.e.\ an integer satisfying
\begin{align}
m \cdot m^{-1} \equiv 1 \pmod{n}.
\end{align}
\end{theorem}

\section*{Step-by-Step Explanation}

\subsection*{1. Invertibility}

Since $\gcd(m,n) = 1$, we know that there exists an integer $m^{-1}$ such that
\begin{align}
m \cdot m^{-1} \;\equiv\; 1 \pmod{n}.
\end{align}
This $m^{-1}$ is called the \emph{multiplicative inverse of $m$ modulo $n$}.
\subsection*{1. Invertibility}

Since $\gcd(m,n) = 1$, we know from elementary number theory (specifically, Bezout's identity) that there exist integers $x, y$ such that
\begin{align}
m \,x + n \,y \;=\; 1.
\end{align}
Rearranging this, we get
\begin{align}
m \,x \;=\; 1 - n\,y.
\end{align}
Now, consider what happens \emph{modulo} $n$. We see
\begin{align}
m \,x 
\;\equiv\; 1 - n\,y 
\;\equiv\; 1
\pmod{n}.
\end{align}
Hence $m \,x \equiv 1 \pmod{n}$, which shows $x$ is a multiplicative inverse of $m$ modulo $n$. We typically denote this inverse by $m^{-1}$, so we write
\begin{align}
m \cdot m^{-1} \;\equiv\; 1 \pmod{n}.
\end{align}
This integer $m^{-1}$ is called the \emph{multiplicative inverse of $m$ modulo $n$}.

\begin{remark}
In practice, we often find $m^{-1}$ using the \emph{Extended Euclidean Algorithm}. The algorithm not only computes the greatest common divisor of $m$ and $n$, but also finds the Bezout coefficients $x$ and $y$ such that
\begin{align}
m x + n y = \gcd(m,n).
\end{align}
Since $\gcd(m,n)=1$, it follows that
\begin{align}
m x + n y = 1,
\end{align}
which immediately yields $m\,x \equiv 1 \pmod{n}$. Thus $m^{-1} \equiv x \pmod{n}$.
\end{remark}
\subsection*{2. Motivating the Formula for $x$}

We wish to solve:
\begin{align}
\begin{cases}
x \equiv a \pmod{m}, \\
x \equiv b \pmod{n}.
\end{cases}
\end{align}

\begin{enumerate}
\item Since $x \equiv a \pmod{m}$, we can write
\begin{align}
x = a + m\,k
\end{align}
for some integer $k$.  

\item Substituting $x = a + m\,k$ into $x \equiv b \pmod{n}$ gives:
\begin{align}
a + m\,k \;\equiv\; b \pmod{n} 
\quad \Longrightarrow \quad 
m\,k \;\equiv\; b - a \pmod{n}.
\end{align}
Because $m$ has an inverse $m^{-1}$ modulo $n$, we can multiply the above congruence by $m^{-1}$ to obtain
\begin{align}
k \;\equiv\; (b - a)\, m^{-1} \pmod{n}.
\end{align}
\end{enumerate}

Hence $k$ is determined modulo $n$.

\subsection*{3. Constructing the Solution}

From the above, we choose
\begin{align}
k \;\equiv\; (b - a)\, m^{-1} \pmod{n},
\end{align}
and hence we \emph{define}
\begin{align}
x \;=\; a + m \bigl((b-a)\,m^{-1} \bmod n\bigr).
\end{align}
This ensures $x\equiv a \pmod{m}$ by construction.

\subsection*{4. Verifying the Two Congruences}

\paragraph{(i) Modulo $m$:}
\begin{align}
x = a + m \bigl((b-a)\,m^{-1}\bigr) 
\;\equiv\; a \;+\; 0
\;\equiv\; a 
\pmod{m},
\end{align}
since $m\cdot(\dots)$ vanishes modulo $m$.

\paragraph{(ii) Modulo $n$:}
We use $m\,m^{-1} \equiv 1 \pmod{n}$. Then
\begin{align}
m \bigl((b-a)\,m^{-1}\bigr) 
\;\equiv\; (b-a)\,\underbrace{(m \,m^{-1})}_{\equiv 1 \,(\bmod\,n)} 
\;\equiv\; b-a \pmod{n}.
\end{align}
Hence
\begin{align}
x 
\;=\;
a + m\bigl((b-a)\,m^{-1}\bigr)
\;\equiv\; 
a + (b-a)
\;=\;
b \pmod{n}.
\end{align}
Thus $x\equiv a \pmod{m}$ and $x\equiv b \pmod{n}$, as required.

\subsection*{5. Uniqueness Modulo $mn$}

Suppose $x$ and $x'$ both solve the system:
\begin{align}
\begin{cases}
x \equiv a \pmod{m}, & x \equiv b \pmod{n},\\
x' \equiv a \pmod{m}, & x' \equiv b \pmod{n}.
\end{cases}
\end{align}
Then
\begin{align}
x' \equiv x \pmod{m}
\quad\text{and}\quad
x' \equiv x \pmod{n}.
\end{align}
Hence $m\mid (x'-x)$ and $n\mid (x'-x)$.  Because $\gcd(m,n) = 1$, it follows that $mn \mid (x'-x)$. Thus
\begin{align}
x' \equiv x \pmod{mn}.
\end{align}
Consequently, the solution is unique up to adding multiples of $mn$. 

\medskip

\noindent
\textbf{Therefore, there is exactly one solution in the ring of integers modulo $mn$.}

\end{solution}
\end{document}
