\documentclass[12pt]{article}

% Packages
\usepackage[margin=1in]{geometry}
\usepackage{amsmath,amssymb,amsthm}
\usepackage{enumitem}
\usepackage{hyperref}
\usepackage{xcolor}
\usepackage{import}
\usepackage{xifthen}
\usepackage{pdfpages}
\usepackage{transparent}
\usepackage{listings}
\usepackage{tikz}
\usepackage{physics}
\usepackage{siunitx}
\usepackage{booktabs}
\usepackage{cancel}
  \usetikzlibrary{calc,patterns,arrows.meta,decorations.markings}


\DeclareMathOperator{\Log}{Log}
\DeclareMathOperator{\Arg}{Arg}

\lstset{
    breaklines=true,         % Enable line wrapping
    breakatwhitespace=false, % Wrap lines even if there's no whitespace
    basicstyle=\ttfamily,    % Use monospaced font
    frame=single,            % Add a frame around the code
    columns=fullflexible,    % Better handling of variable-width fonts
}

\newcommand{\incfig}[1]{%
    \def\svgwidth{\columnwidth}
    \import{./Figures/}{#1.pdf_tex}
}
\theoremstyle{definition} % This style uses normal (non-italicized) text
\newtheorem{solution}{Solution}
\newtheorem{proposition}{Proposition}
\newtheorem{problem}{Problem}
\newtheorem{lemma}{Lemma}
\newtheorem{theorem}{Theorem}
\newtheorem{remark}{Remark}
\newtheorem{note}{Note}
\newtheorem{definition}{Definition}
\newtheorem{example}{Example}
\newtheorem{corollary}{Corollary}
\theoremstyle{plain} % Restore the default style for other theorem environments
%

% Theorem-like environments
% Title information
\title{finals prep}
\author{Jerich Lee}
\date{\today}

\begin{document}

\maketitle
%-----------------------------------------------------------------
%  Problem 35(c) –  Inverses in the group \((G,\star)\)
%  where \(G=\{-1,0,1\}\) and
%  \[
%     a\star b=
%     \begin{cases}
%        a+b+1, & a+b\le 0,\\[4pt]
%        a+b-2, & a+b> 0 .
%     \end{cases}
%  \]
%-----------------------------------------------------------------

\subsection*{Step 1.  Recall (from part b) the identity element}

We showed in part (b) that
\[
e=-1,
\qquad
a\star(-1)=(-1)\star a=a\quad\forall\,a\in G.
\]

\subsection*{Step 2.  Set up the inverse condition}

For each \(a\in G\) we need \(b\in G\) such that
\[
a\star b = e = -1 .
\]

\subsection*{Step 3.  Solve \(a\star b=-1\) using the
piece-wise definition}

\[
a\star b =
\begin{cases}
a+b+1, & a+b \le 0,\\
a+b-2, & a+b > 0 .
\end{cases}
\]

\[
a\star b = -1 \;\Longrightarrow\;
\begin{cases}
a+b+1 = -1, & a+b \le 0,\\
a+b-2 = -1, & a+b > 0,
\end{cases}
\;\Longrightarrow\;
\begin{cases}
b = -2-a, & a+b \le 0,\\
b = 1-a,  & a+b > 0.
\end{cases}
\]

We test these candidates for each \(a\) and keep the one that lies in
\(G=\{-1,0,1\}\) \emph{and} satisfies its associated inequality.

\begin{center}
\renewcommand{\arraystretch}{1.3}
\begin{tabular}{c|ccl}
\(a\) & candidate from \(\;b=-2-a\) &
candidate from \(\;b=1-a\) & valid inverse \\ \hline
\(-1\) & \(b=-2-(-1)=-1\in G\) \(\checkmark\) & \(b=1-(-1)=2\notin G\) & \(-1\) \\[2pt]
\(0\)  & \(b=-2\notin G\) & \(b=1-0=1\in G,\;a+b=1>0\) \(\checkmark\) & \(1\) \\[2pt]
\(1\)  & \(b=-3\notin G\) & \(b=1-1=0\in G,\;a+b=1>0\) \(\checkmark\) & \(0\)
\end{tabular}
\end{center}

\subsection*{Step 4.  Summary of the inverse operation}

\[
a^{-1}=
\begin{cases}
-1, & a=-1,\\
\;\;1, & a=0,\\
\;\;0, & a=1.
\end{cases}
\]

Equivalently, the inverse table is

\[
\begin{array}{c|ccc}
a & -1 & 0 & 1 \\ \hline
a^{-1} & -1 & 1 & 0
\end{array}
\]

\subsection*{Step 5.  Quick check}

Compute \(a\star a^{-1}\):

\[
\begin{aligned}
-1\star(-1)&=-1 &&(\text{identity})\\
0\star 1   &=(-1) &&(\text{using }0+1>0)\\
1\star 0   &=(-1) &&(\text{using }1+0>0)
\end{aligned}
\]
and similarly \(a^{-1}\star a=-1\) for each \(a\).
Thus the inverses are correct.
%---------------------------------------------------------------
%  Inverses in the multiplicative group  \(\Bbb Z_{18}^{\!*}\)
%---------------------------------------------------------------
\[
\Bbb Z_{18}^{\!*}
   \;=\;
   \{\,a\in\{1,2,\dots,17\}\mid\gcd(a,18)=1\,\}
   \;=\;
   \{1,5,7,11,13,17\}.
\]

For each \(a\) we need \(a^{-1}\) satisfying
\[
a\,a^{-1}\equiv 1 \pmod{18}.
\]

\[
\renewcommand{\arraystretch}{1.3}
\begin{array}{c|c}
a & a^{-1}\pmod{18}\\\hline
 1 & 1 \\[2pt]
 5 & 11 \quad(\,5\!\times\!11 = 55 \equiv 1\,)\\
 7 & 13 \quad(\,7\!\times\!13 = 91 \equiv 1\,)\\
11 & 5  \quad(\,11\!\times\!5 = 55 \equiv 1\,)\\
13 & 7  \quad(\,13\!\times\!7 = 91 \equiv 1\,)\\
17 & 17 \quad(\,17\!\times\!17 = 289 \equiv 1\,)
\end{array}
\]

Hence the inverse pairs are
\[
1\leftrightarrow 1,\quad
5\leftrightarrow 11,\quad
7\leftrightarrow 13,\quad
17\leftrightarrow 17.
\]

All elements now have their multiplicative inverses in
\(\Bbb Z_{18}^{\!*}\), completing the task.
%-----------------------------------------------------------------
%  Problem 41. Problem 2 – Modular–exponentiation practice
%-----------------------------------------------------------------

\newcommand{\pow}[2]{#1^{#2}}

\begin{enumerate}[label=(\alph*)]

%------------------ (a) ------------------
\item  \(\displaystyle \pow{3}{100}\pmod{10}\).

      Because \(\gcd(3,10)=1\) we may use Euler’s theorem
      \(\pow{3}{\varphi(10)}\equiv1\pmod{10}\) with
      \(\varphi(10)=4\):

      \[
        3^{100}
          =\bigl(3^{4}\bigr)^{25}
          \equiv 1^{25}\equiv\boxed{1}\pmod{10}.
      \]

%------------------ (b) ------------------
\item  \(\displaystyle \pow{5}{60}\pmod{7}\).

      The modulus \(7\) is prime, so \(\varphi(7)=6\) and
      \(5^{6}\equiv1\pmod{7}\):
      \[
        5^{60}
          =\bigl(5^{6}\bigr)^{10}
          \equiv 1^{10}\equiv\boxed{1}\pmod{7}.
      \]

%------------------ (c) ------------------
\item  \(\displaystyle \pow{400}{60}\pmod{61}\).

      First reduce the base:
      \(400\equiv400-6\!\times\!61=34\pmod{61}\).

      Since \(61\) is prime,
      \(34^{60}\equiv1\pmod{61}\) (Fermat’s little theorem):

      \[
        400^{60}\equiv34^{60}\equiv\boxed{1}\pmod{61}.
      \]

%------------------ (d) ------------------
\item  Last digit of \(17^{50}\).

      Last digits correspond to residues mod \(10\).
      Replace \(17\) by \(7\):

      \[
        7^{1}\equiv7,\;
        7^{2}\equiv9,\;
        7^{3}\equiv3,\;
        7^{4}\equiv1\pmod{10},
      \]
      so the cycle length is \(4\).
      Because \(50\equiv2\pmod{4}\),

      \[
        17^{50}\equiv7^{50}\equiv7^{2}\equiv9\pmod{10}.
      \]

      The last digit is \(\boxed{9}\).

\end{enumerate}
%-----------------------------------------------------------------
%  Every subgroup of \((\Bbb Z,+)\) is cyclic of the form \(k\Bbb Z\)
%-----------------------------------------------------------------

\begin{theorem}
  Let \(H\le(\Bbb Z,+)\).  Then there exists a (unique) integer
  \(k\ge0\) such that
  \[
        H = k\Bbb Z := \{\,ka : a\in\Bbb Z\,\}.
  \]
  \end{theorem}
  
  \begin{proof}
  \textbf{Step 1.  Handle the trivial subgroup.}
  
  If \(H=\{0\}\) we may take \(k=0\), because \(0\Bbb Z=\{0\}\).
  
  \vspace{2ex}
  \textbf{Step 2.  Assume \(H\neq\{0\}\) and find the least positive element.}
  
  Because \(H\) is non‑empty and closed under additive inverses,
  it contains some positive integers.
  Let
  \[
        k := \min\bigl\{\,h\in H : h>0\,\bigr\}.
  \]
  Existence of the minimum follows from the well‑ordering of
  \(\Bbb Z_{\ge0}\).
  
  \vspace{2ex}
  \textbf{Step 3.  Show \(k\Bbb Z\subseteq H\).}
  
  For any \(a\in\Bbb Z\) we have \(ka = a\!+\!\dots+ a\)
  (\(a\) times) or its negative,
  and \(H\) is closed under integer addition,
  so \(ka\in H\).
  Thus \(k\Bbb Z\subseteq H\).
  
  \vspace{2ex}
  \textbf{Step 4.  Show \(H\subseteq k\Bbb Z\).}
  
  Take an arbitrary \(h\in H\) with \(h>0\).
  Apply the division algorithm:
  \[
        h = qk + r,
        \quad 0\le r < k,
        \quad q\in\Bbb Z.
  \]
  
  Since \(qk\in k\Bbb Z\subseteq H\) and \(H\) is a subgroup,
  the difference \(r = h - qk\) also lies in \(H\).
  By construction \(0\le r < k\), but \(k\) is the \emph{smallest}
  positive element of \(H\); hence we must have \(r=0\).
  Therefore \(h = qk\in k\Bbb Z\).
  
  The same argument with \(-h\) covers negative elements, so
  every \(h\in H\) is a multiple of \(k\):
  \(H\subseteq k\Bbb Z\).
  
  \vspace{2ex}
  \textbf{Step 5.  Combine the inclusions.}
  
  We have shown \(k\Bbb Z\subseteq H\) and \(H\subseteq k\Bbb Z\),
  hence \(H = k\Bbb Z\).
  
  \vspace{2ex}
  \textbf{Uniqueness.}
  If \(k\Bbb Z = k'\Bbb Z\) with \(k,k'>0\),
  then \(k\in k'\Bbb Z\) implies \(k'|k\) and vice‑versa,
  forcing \(k=k'\).
  \end{proof}
  
  \begin{remark}
  Thus the lattice of subgroups of \((\Bbb Z,+)\) is in one‑to‑one
  correspondence with the set of non‑negative integers:
  \[
        \{\,\{0\}=0\Bbb Z,\;1\Bbb Z,\;2\Bbb Z,\;3\Bbb Z,\dots\}.
  \]
  \end{remark}
\end{document}
