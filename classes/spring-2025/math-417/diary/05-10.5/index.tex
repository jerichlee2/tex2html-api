\documentclass[12pt]{article}

% Packages
\usepackage[margin=1in]{geometry}
\usepackage{amsmath,amssymb,amsthm}
\usepackage{enumitem}
\usepackage{hyperref}
\usepackage{xcolor}
\usepackage{import}
\usepackage{xifthen}
\usepackage{pdfpages}
\usepackage{transparent}
\usepackage{listings}
\usepackage{tikz}
\usepackage{physics}
\usepackage{siunitx}
\usepackage{booktabs}
\usepackage{cancel}
  \usetikzlibrary{calc,patterns,arrows.meta,decorations.markings}


\DeclareMathOperator{\Log}{Log}
\DeclareMathOperator{\Arg}{Arg}
\DeclareMathOperator{\F}{F}


\lstset{
    breaklines=true,         % Enable line wrapping
    breakatwhitespace=false, % Wrap lines even if there's no whitespace
    basicstyle=\ttfamily,    % Use monospaced font
    frame=single,            % Add a frame around the code
    columns=fullflexible,    % Better handling of variable-width fonts
}

\newcommand{\incfig}[1]{%
    \def\svgwidth{\columnwidth}
    \import{./Figures/}{#1.pdf_tex}
}
\theoremstyle{definition} % This style uses normal (non-italicized) text
\newtheorem{solution}{Solution}
\newtheorem{proposition}{Proposition}
\newtheorem{problem}{Problem}
\newtheorem{lemma}{Lemma}
\newtheorem{theorem}{Theorem}
\newtheorem{remark}{Remark}
\newtheorem{note}{Note}
\newtheorem{definition}{Definition}
\newtheorem{example}{Example}
\newtheorem{corollary}{Corollary}
\theoremstyle{plain} % Restore the default style for other theorem environments
%

% Theorem-like environments
% Title information
\title{}
\author{Jerich Lee}
\date{\today}

\begin{document}

\maketitle
%------------------------------------------------------------
%  Core‑Group Cheat Sheet  (for a Math 417 final)
%  ‑‑ copy‑and‑paste LaTeX code ‑‑
%  (Assumes amsmath / amssymb packages; TikZ not required here)
%------------------------------------------------------------

\begin{center}
  \Large\bfseries Core Groups \& Facts to Know by Heart
  \end{center}
  
  \bigskip
  %%%%%%%%%%%%%%%%%%%%%%%%%%%%%%%%%%%%%%%%%%%%%%%%%%%%%%%%%%%%%%
  \section*{1.\;Cyclic groups $C_{n}\cong\Bbb Z_{n}$}
  
  \begin{tabular}{|l|l|}
  \hline
  \textbf{Fact} & \textbf{What to remember}\\\hline
  Order               & $n$\\
  Generator           & any element of order $n$ (e.g.\ $1\pmod n$)\\
  Subgroups           & exactly one for each divisor $d\mid n$; it is $\langle n/d\rangle$\\
  Structure           & Every finite abelian group is a direct product of cyclic groups\\\hline
  \end{tabular}
  
  \bigskip
  %%%%%%%%%%%%%%%%%%%%%%%%%%%%%%%%%%%%%%%%%%%%%%%%%%%%%%%%%%%%%%
  \section*{2.\;Dihedral groups $D_{2n}$}
  
  \[
     D_{2n}= \langle r,s \mid r^{\,n}=s^{2}=1,\; srs=r^{-1}\rangle
  \]
  \begin{itemize}
    \item Order $2n$.
    \item Center $\{1\}$ if $n$ is odd; $\{1,r^{n/2}\}$ if $n$ is even.
    \item $n$ reflections, each of order $2$.
  \end{itemize}
  
  \bigskip
  %%%%%%%%%%%%%%%%%%%%%%%%%%%%%%%%%%%%%%%%%%%%%%%%%%%%%%%%%%%%%%
  \section*{3.\;Symmetric/Alternating groups}
  
  \begin{tabular}{|l|c|p{7cm}|}
  \hline
  Group & Order & Key facts\\\hline
  $S_{n}$ & $n!$ & Generated by adjacent transpositions $(\,i\; i+1\,)$.\\
  $A_{n}$ & $n!/2$ & Simple for $n\ge 5$; $A_{4}$ has normal Klein subgroup $V_{4}$.\\\hline
  \end{tabular}
  %------------------------------------------------------------
%  The symmetric group $S_{3}$ in cycle notation
%------------------------------------------------------------

\begin{align*}
  S_{3}
    \;=\;
    \Bigl\{
        e,\, (12),\, (13),\, (23),\, (123),\, (132)
    \Bigr\},
  \end{align*}
  
  \noindent where  
  
  \[
     e      =(1)(2)(3),\quad
     (12)   =(1\;2),\quad
     (13)   =(1\;3),\quad
     (23)   =(2\;3),\quad
     (123)  =(1\;2\;3),\quad
     (132)  =(1\;3\;2).
  \]
  
  \smallskip
  \textbf{Orders of the elements}
  \[
  \begin{array}{c|c}
  \text{element} & \text{order}\\\hline
  e             & 1\\
  (12),(13),(23) & 2\\
  (123),(132)    & 3
  \end{array}
  \]
  
  \smallskip
  \textbf{Cayley table} (composition is read left\,\(\circ\)\,right).
  
  \[
  \renewcommand{\arraystretch}{1.25}
  \begin{array}{c|cccccc}
  \circ & e & (12) & (13) & (23) & (123) & (132)\\\hline
  e      & e & (12) & (13) & (23) & (123) & (132)\\
  (12)   & (12) & e & (132) & (123) & (13) & (23)\\
  (13)   & (13) & (123) & e & (132) & (23) & (12)\\
  (23)   & (23) & (132) & (123) & e & (12) & (13)\\
  (123)  & (123) & (13) & (23) & (12) & (132) & e\\
  (132)  & (132) & (23) & (12) & (13) & e & (123)
  \end{array}
  \]
  
  \smallskip
  \textbf{Structure summary}
  \begin{itemize}
    \item $|S_{3}| = 6$.
    \item Non-abelian; smallest non-abelian group.
    \item Presentation: $\displaystyle S_{3} = \langle\,s,t \mid s^{2}=t^{3}=e,\; sts = t^{2}\,\rangle$,  
          where one may take $s=(12)$, $t=(123)$.
    \item Conjugacy classes:  
          $\{e\}$, \(\{(12),(13),(23)\}\), \(\{(123),(132)\}\).
  \end{itemize}
  %------------------------------------------------------------
  \bigskip
  %%%%%%%%%%%%%%%%%%%%%%%%%%%%%%%%%%%%%%%%%%%%%%%%%%%%%%%%%%%%%%
  \section*{4.\;Klein four group $V_{4}\cong\Bbb Z_{2}\oplus\Bbb Z_{2}$}
  
  \begin{itemize}
    \item Every non‑identity element has order $2$.
    \item $\operatorname{Aut}(V_{4})\cong S_{3}$.
  \end{itemize}
  
  \bigskip
  %%%%%%%%%%%%%%%%%%%%%%%%%%%%%%%%%%%%%%%%%%%%%%%%%%%%%%%%%%%%%%
  \section*{5.\;Quaternion group $Q_{8}$}
  
  \[
     Q_{8}= \{\pm1,\pm i,\pm j,\pm k\},\qquad i^{2}=j^{2}=k^{2}=ijk=-1
  \]
  \begin{itemize}
    \item Center $=\{\pm1\}$.
    \item Only element of order $2$ is $-1$.
  \end{itemize}
  
  \bigskip
  %%%%%%%%%%%%%%%%%%%%%%%%%%%%%%%%%%%%%%%%%%%%%%%%%%%%%%%%%%%%%%
  \section*{6.\;Unit groups $\Bbb Z_{n}^{\times}$}
  
  \begin{itemize}
    \item Order $\varphi(n)$.
    \item $\Bbb Z_{p}^{\times}$ (prime $p$) is cyclic.
    \item Examples: $\Bbb Z_{8}^{\times}\cong C_{2}\times C_{2}$ (not cyclic); 
          $\Bbb Z_{12}^{\times}\cong C_{2}\times C_{2}$.
  \end{itemize}
  
  \bigskip
  %%%%%%%%%%%%%%%%%%%%%%%%%%%%%%%%%%%%%%%%%%%%%%%%%%%%%%%%%%%%%%
  \section*{7.\;General linear group $GL_{2}(\F_{p})$}
  
  \[
     |GL_{2}(\F_{p})|=(p^{2}-1)(p^{2}-p)=p(p-1)^{2}(p+1).
  \]
  
  Acts on the $2$‑dimensional space $\F_{p}^{2}$; useful for Sylow and automorphism problems.
  
  \bigskip
  %%%%%%%%%%%%%%%%%%%%%%%%%%%%%%%%%%%%%%%%%%%%%%%%%%%%%%%%%%%%%%
  \section*{8.\;Direct‑product cheat lines}
  
  \[
     C_{m}\times C_{n}\cong C_{mn}\iff\gcd(m,n)=1.
  \qquad
     G\times H \text{ abelian } \iff G \text{ and } H \text{ abelian}.
  \]
  
  \bigskip
  %%%%%%%%%%%%%%%%%%%%%%%%%%%%%%%%%%%%%%%%%%%%%%%%%%%%%%%%%%%%%%
  \section*{9.\;Quick classification templates}
  
  \begin{itemize}
    \item $|G|=p$ (prime) $\;\Longrightarrow\;$ $G$ cyclic, simple.
    \item $|G|=p^{2}$ $\;\Longrightarrow\;$ $G\cong C_{p^{2}}\;$ or $\;C_{p}\times C_{p}$.
    \item $|G|=2p$ ($p$ odd) $\;\Longrightarrow\;$ $G\cong C_{2p}$ or $D_{2p}$.
    \item If $n_{p}=1$ in Sylow’s theorem, the Sylow $p$‑subgroup is normal.
  \end{itemize}
  
  \bigskip
  %%%%%%%%%%%%%%%%%%%%%%%%%%%%%%%%%%%%%%%%%%%%%%%%%%%%%%%%%%%%%%
  \section*{10.\;Standard presentations}
  
  \[
  \begin{aligned}
  C_{n}\! &: \langle a\mid a^{n}=1\rangle,\\
  D_{2n}\! &: \langle r,s\mid r^{n}=s^{2}=1,\;srs=r^{-1}\rangle,\\
  S_{3}\!  &: \langle s,t\mid s^{2}=t^{3}=1,\;sts=t^{2}\rangle,\\
  Q_{8}\!  &: \langle i,j\mid i^{4}=1,\;i^{2}=j^{2},\; jij^{-1}=i^{-1}\rangle.
  \end{aligned}
  \]
  
  \bigskip
  %%%%%%%%%%%%%%%%%%%%%%%%%%%%%%%%%%%%%%%%%%%%%%%%%%%%%%%%%%%%%%
  \section*{11.\;Ring/field reminders}
  
  \begin{itemize}
    \item \textbf{Finite integral domain $\Rightarrow$ field.}
    \item $(\Bbb Z/n\Bbb Z)^{\times}$ is cyclic $\iff n=2,4,p^{k},2p^{k}$ with $p$ odd prime.
    \item Wilson: $(p-1)!\equiv-1\pmod p$. \quad
          Fermat: $a^{p-1}\equiv1\pmod p$ if $p\nmid a$.
  \end{itemize}
  
  \bigskip
  %%%%%%%%%%%%%%%%%%%%%%%%%%%%%%%%%%%%%%%%%%%%%%%%%%%%%%%%%%%%%%
  \section*{12.\;Automorphism quick facts}
  
  \begin{tabular}{|l|c|}
  \hline
  Group $G$ & $|\mathrm{Aut}(G)|$ \\\hline
  $C_{n}$                & $\varphi(n)$\\
  $V_{4}$                & $6$\\
  $\mathbb{Z}_{p}^{\,n}$         & $(p^{n}-1)(p^{n}-p)\cdots(p^{n}-p^{\,n-1})$\\
  $Q_{8}$                & $24\;(\cong S_{4})$\\\hline
  \end{tabular}
  
  \bigskip
  %%%%%%%%%%%%%%%%%%%%%%%%%%%%%%%%%%%%%%%%%%%%%%%%%%%%%%%%%%%%%%
  \paragraph{How to use the sheet.}
  Make sure you can
  \begin{enumerate}
    \item state a presentation for each group,
    \item give orders of key subgroups/elements,
    \item recognise abelian / cyclic / dihedral / quaternion structure,
    \item run a Sylow count in seconds.
  \end{enumerate}
  
  \medskip
  \hrule
  \medskip
  \centerline{\textbf{Good luck on your final!}}
  %------------------------------------------------------------
\end{document}
