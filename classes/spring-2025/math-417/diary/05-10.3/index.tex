\documentclass[12pt]{article}

% Packages
\usepackage[margin=1in]{geometry}
\usepackage{amsmath,amssymb,amsthm}
\usepackage{enumitem}
\usepackage{hyperref}
\usepackage{xcolor}
\usepackage{import}
\usepackage{xifthen}
\usepackage{pdfpages}
\usepackage{transparent}
\usepackage{listings}
\usepackage{tikz}
\usepackage{physics}
\usepackage{siunitx}
\usepackage{booktabs}
\usepackage{cancel}
  \usetikzlibrary{calc,patterns,arrows.meta,decorations.markings}


\DeclareMathOperator{\Log}{Log}
\DeclareMathOperator{\Arg}{Arg}

\lstset{
    breaklines=true,         % Enable line wrapping
    breakatwhitespace=false, % Wrap lines even if there's no whitespace
    basicstyle=\ttfamily,    % Use monospaced font
    frame=single,            % Add a frame around the code
    columns=fullflexible,    % Better handling of variable-width fonts
}

\newcommand{\incfig}[1]{%
    \def\svgwidth{\columnwidth}
    \import{./Figures/}{#1.pdf_tex}
}
\theoremstyle{definition} % This style uses normal (non-italicized) text
\newtheorem{solution}{Solution}
\newtheorem{proposition}{Proposition}
\newtheorem{problem}{Problem}
\newtheorem{lemma}{Lemma}
\newtheorem{theorem}{Theorem}
\newtheorem{remark}{Remark}
\newtheorem{note}{Note}
\newtheorem{definition}{Definition}
\newtheorem{example}{Example}
\newtheorem{corollary}{Corollary}
\theoremstyle{plain} % Restore the default style for other theorem environments
%

% Theorem-like environments
% Title information
\title{MATH 417 Practice Final Exam 3}
\author{Jerich Lee}
\date{\today}

\begin{document}

\maketitle
%------------------------------------------------------------
%  Practice Final Exam #3 – MATH 417 (Comprehensive)
%  13 questions • no calculators • show full work
%------------------------------------------------------------
\newcommand{\Z}{\mathbb Z}
\newcommand{\Q}{\mathbb Q}
\newcommand{\R}{\mathbb R}

\bigskip
\begin{problem}
  On the set \(\R\) define
  \[
      x\ast y \;:=\; x+y+xy.
  \]
  \begin{enumerate}
      \item[(a)] Show that \((\R,\ast)\) is a group.
      \item[(b)] Find the identity element and the inverse of an arbitrary
                \(x\in\R\).
      \item[(c)] Determine the subgroup generated by \(\tfrac12\).
  \end{enumerate}
\end{problem}

\bigskip
\begin{problem}
  Let \(G\) act on a finite set \(X\) on the left and let
  \(\rho:G\to S_{X}\) be the associated permutation representation.
  \begin{enumerate}
      \item[(a)] Prove that \(\rho\) is a group homomorphism.
      \item[(b)] Show that \(\ker\rho\) equals the kernel of the action
                (the largest normal subgroup fixing every element of \(X\)).
  \end{enumerate}
\end{problem}

\bigskip
\begin{problem}
  \begin{enumerate}
      \item[(a)] List all elements of \(\Z_{21}^{\times}\) and their orders.
      \item[(b)] Is \(\Z_{21}^{\times}\) cyclic?  Justify.
      \item[(c)] Compute \(\prod_{a\in\Z_{21}^{\times}}a\pmod{21}\).
  \end{enumerate}
\end{problem}

\bigskip
\begin{problem}
  Let \(|G|=108=2^{2}\cdot3^{3}\).
  Show that:
  \begin{enumerate}
      \item[(a)] \(G\) has a normal Sylow \(3\)-subgroup;
      \item[(b)] \(G\) is not necessarily abelian (give an example);
      \item[(c)] \(G\) has at least one subgroup of order \(18\).
  \end{enumerate}
\end{problem}

\bigskip
\begin{problem}
  Let \(E=\Q(\sqrt[3]{2},\,\omega)\) where
  \(\omega=e^{2\pi i/3}\).
  \begin{enumerate}
      \item[(a)] Compute \([E:\Q]\).
      \item[(b)] Show that \(E/\Q\) is a Galois extension and identify its
                Galois group (up to isomorphism).
  \end{enumerate}
\end{problem}

\bigskip
\begin{problem}
  Prove that
  \[
      SL_{2}(\R)/\{\pm I\}\;\cong\;PSL_{2}(\R),
  \]
  and explain why the quotient is still non-abelian.
\end{problem}

\bigskip
\begin{problem}
  Let
  \(R=\Z_{5}[x]/(x^{2}+2)\).
  \begin{enumerate}
      \item[(a)] Show that \(x^{2}+2\) is irreducible over \(\Z_{5}\).
      \item[(b)] Conclude that \(R\) is a field of order \(25\).
      \item[(c)] Find the multiplicative inverse of \(x+(x^{2}+2)\) in \(R\).
  \end{enumerate}
\end{problem}

\bigskip
\begin{problem}
  \begin{enumerate}
      \item[(a)] Determine \(\operatorname{Aut}(\Z_{12})\) explicitly.
      \item[(b)] Show it is isomorphic to \(\Z_{2}\oplus\Z_{2}\).
  \end{enumerate}
\end{problem}

\bigskip
\begin{problem}
  Let \(H\le S_{4}\) be the subgroup generated by the transposition \((12)\).
  Compute:
  \begin{enumerate}
      \item[(a)] the centralizer \(C_{S_{4}}(H)\);
      \item[(b)] the normalizer \(N_{S_{4}}(H)\);
      \item[(c)] the index \([N_{S_{4}}(H):C_{S_{4}}(H)]\) and interpret
                it in terms of group actions.
  \end{enumerate}
\end{problem}

\bigskip
\begin{problem}
  Solve simultaneously
  \[
      x\equiv 3\pmod{8},\qquad
      x\equiv 4\pmod{9},\qquad
      x\equiv 5\pmod{7},
  \]
  and determine \(\operatorname{ord}_{\Z_{504}}(x)\).
\end{problem}

\bigskip
\begin{problem}
  Let \(D_{10}=\langle r,s\mid r^{5}=s^{2}=1,\;srs=r^{-1}\rangle\).
  \begin{enumerate}
      \item[(a)] List all conjugacy classes of \(D_{10}\).
      \item[(b)] Verify the class equation.
  \end{enumerate}
\end{problem}

\bigskip
\begin{problem}
  In a commutative ring \(R\) with identity, define
  \[
      \mathfrak N=\{\,x\in R:x^{k}=0\text{ for some }k\ge1\}.
  \]
  Prove that \(\mathfrak N\) is an ideal (the nilradical of \(R\)).
\end{problem}

\bigskip
\begin{problem}
  \begin{enumerate}[label=(\alph*)]
      \item Every finite subgroup of the multiplicative group \(\mathbb{C}^{\times}\)
            is cyclic.
      \item If \(G\) is cyclic, then every subgroup of \(G\) is characteristic.
      \item The ring \(\Z_{6}\) is an integral domain.
      \item There exists a non-trivial group in which every element is conjugate to every other.
      \item Any two groups of order \(49\) are isomorphic.
  \end{enumerate}
  Circle \textbf{T} or \textbf{F} for each.
\end{problem}
%------------------------------------------------------------
\end{document}
