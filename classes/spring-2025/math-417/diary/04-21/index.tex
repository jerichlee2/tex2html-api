\documentclass[12pt]{article}

% Packages
\usepackage[margin=1in]{geometry}
\usepackage{amsmath,amssymb,amsthm}
\usepackage{enumitem}
\usepackage{hyperref}
\usepackage{xcolor}
\usepackage{import}
\usepackage{xifthen}
\usepackage{pdfpages}
\usepackage{transparent}
\usepackage{listings}
\usepackage{tikz}
  \usetikzlibrary{calc,patterns,arrows.meta,decorations.markings}


\DeclareMathOperator{\Log}{Log}
\DeclareMathOperator{\Arg}{Arg}

\lstset{
    breaklines=true,         % Enable line wrapping
    breakatwhitespace=false, % Wrap lines even if there's no whitespace
    basicstyle=\ttfamily,    % Use monospaced font
    frame=single,            % Add a frame around the code
    columns=fullflexible,    % Better handling of variable-width fonts
}

\newcommand{\incfig}[1]{%
    \def\svgwidth{\columnwidth}
    \import{./Figures/}{#1.pdf_tex}
}
\theoremstyle{definition} % This style uses normal (non-italicized) text
\newtheorem{solution}{Solution}
\newtheorem{proposition}{Proposition}
\newtheorem{problem}{Problem}
\newtheorem{lemma}{Lemma}
\newtheorem{theorem}{Theorem}
\newtheorem{remark}{Remark}
\newtheorem{note}{Note}
\newtheorem{definition}{Definition}
\newtheorem{example}{Example}
\newtheorem{corollary}{Corollary}
\theoremstyle{plain} % Restore the default style for other theorem environments
%

% Theorem-like environments
% Title information
\title{}
\author{Jerich Lee}
\date{\today}

\begin{document}

\maketitle
\paragraph{Polynomial rings in several variables.}
Let $R$ be a (commutative) ring.  
Adjoining a single indeterminate $X$ produces the polynomial ring $R[X]$.  
Adjoining a second indeterminate $Y$ to this ring gives
\[
  R[X][Y],
\]
whose elements are polynomials in $Y$ with coefficients taken from $R[X]$.
Because $X$ and $Y$ are independent, we suppress the nesting and write
\[
  R[X][Y] \;=\; R[X,Y].
\]

More generally, for $n\ge 1$ we define inductively
\[
  R[X_1,\dots,X_n]
  \;=\;
  R[X_1,\dots,X_{n-1}][X_n].
\]

Every polynomial $f\in R[X_1,\dots,X_n]$ has a \emph{unique} finite expansion
\[
  f
  \;=\;
  \sum_{i_1,\dots,i_n\ge 0}
    a_{i_1\dots i_n}\,X_1^{\,i_1}\cdots X_n^{\,i_n},
  \qquad
  a_{i_1\dots i_n}\in R.
\]

Each product
\[
  m_{(i_1,\dots,i_n)}
  \;=\;
  X_1^{\,i_1}\cdots X_n^{\,i_n}
\]
is called a \emph{monomial}.  The collection of all such monomials forms an $R$‑basis of the ring $R[X_1,\dots,X_n]$.

In the sequel we shall mostly work with the case where $R$ is a field, as this allows us to divide by any non‑zero coefficient and simplifies many arguments.
\subsection{Characteristic of an Entire Ring}

Let \(R\) be an \emph{entire} (i.e.\ non‑trivial, commutative, with no zero‑divisors) ring.
Because \((R,+)\) is an abelian group, every element \(a\in R\) has an additive order:
the least \(k>0\) such that \(ka=0\), or \(\infty\) if no such \(k\) exists.

\begin{theorem}
In an entire ring \(R\), the additive order of every non‑zero element is the same.
If this common order is finite, then it is a \emph{prime} number.
\end{theorem}

\begin{proof}
\textbf{(Equality of orders).}\;
Choose a non‑zero element \(a\in R\) of \emph{minimal} finite additive order and
denote this order by \(k>1\).  Then \(ka=0\).
Multiplying by \(1_R\) on the left gives
\[
  (k\cdot1_R)a = 0.
\]
Since \(R\) has no zero‑divisors and \(a\neq0\), we must have \(k\cdot1_R=0\).
For any $b\in R$ we now obtain
\[
  kb = (k\cdot1_R)b = 0,
\]
so every element has order dividing $k$.  
Minimality of \(k\) forces every non‑zero element to have \emph{exactly} order~\(k\).
If no element has finite order, all non‑zero elements share infinite order.

\medskip\noindent
\textbf{(Primality of the order).}\;
Assume the common additive order \(k\) is finite.
Write \(k=rs\) with \(1<r,s<k\).
Then
\[
  0 = k\cdot1_R = (rs)1_R = (r1_R)(s1_R).
\]
Because \(R\) is entire, \(r1_R=0\) or \(s1_R=0\),
contradicting the minimality of \(k=\operatorname{ord}(1_R)\).
Hence \(k\) has no proper factorisation and is therefore prime.
\end{proof}
\paragraph{Concrete example.}
Let the base ring be \(R=\mathbf{Q}\subseteq S=\mathbf{R}\) and adjoin \(\alpha=\sqrt2\).
By definition
\[
  \mathbf{Q}[\sqrt2]
  \;=\;
  \left\{\,f(\sqrt2)\;\middle|\;f\in\mathbf{Q}[X]\right\}.
\]

\begin{itemize}
  \item Since \((\sqrt2)^2 = 2\), any higher power of \(\sqrt2\) reduces to a
        \(\mathbf{Q}\)-linear combination of \(1\) and \(\sqrt2\).
  \item Consequently
        \[
          \boxed{\,\mathbf{Q}[\sqrt2]=\{\,a+b\sqrt2\mid a,b\in\mathbf{Q}\,\}\,}
        \]
        is a subring (indeed a field) between \(\mathbf{Q}\) and \(\mathbf{R}\).
  \item For any subring \(T\subseteq\mathbf{R}\) containing
        \(\mathbf{Q}\cup\{\sqrt2\}\) we have \(a+b\sqrt2\in T\), hence
        \(\mathbf{Q}[\sqrt2]\subseteq T\).
\end{itemize}

Thus \(\mathbf{Q}[\sqrt2]\) is the smallest subring of \(\mathbf{R}\) containing
both \(\mathbf{Q}\) and \(\sqrt2\).
% ------------------------------------------------------------
%  Adjoining elements — bracket notation R[α]
%  (Requires the amsmath package for \middle, \text, \boxed)
% ------------------------------------------------------------
\subsection*{Adjoining Elements:  The Bracket Notation $R[\alpha]$}

Let $R$ be a subring of a larger ring $S$ and let $\alpha\in S$.  
The \emph{ring obtained from $R$ by adjoining $\alpha$} is written
\[
  R[\alpha]
  \;=\;
  \left\{\, f(\alpha)\;\middle|\; f\in R[X] \right\},
\]
i.e.\ all polynomial expressions in $\alpha$ whose coefficients lie in $R$.
Equivalently, $R[\alpha]$ is the \textbf{smallest subring of $S$} that contains
both $R$ and the element~$\alpha$.

\paragraph{Example: $\mathbf{Q}[\sqrt{2}]$.}
Inside the ambient ring $\mathbf{R}$ take $R=\mathbf{Q}$ and
$\alpha=\sqrt{2}$.  Because $(\sqrt{2})^{2}=2$, every higher power of
$\sqrt{2}$ reduces to a $\mathbf{Q}$–linear combination of $1$ and $\sqrt{2}$.
Hence
\[
  \boxed{%
    \mathbf{Q}[\sqrt{2}]
      \;=\;
      \{\,a+b\sqrt{2}\mid a,b\in\mathbf{Q}\,\}}
\]
is a \emph{quadratic field} sitting between $\mathbf{Q}$ and $\mathbf{R}$.

\paragraph{Several generators.}
Given $\alpha_{1},\dots,\alpha_{n}\in S$,
\[
  R[\alpha_{1},\dots,\alpha_{n}]
  \;=\;
  \left\{\, f(\alpha_{1},\dots,\alpha_{n})
           \;\middle|\;
           f\in R[X_{1},\dots,X_{n}] \right\}.
\]
For example,
\[
  \mathbf{Q}\!\left[\sqrt{2},\,i\right]
  =\bigl\{\,a+b\sqrt{2}+ci+di\sqrt{2}\mid a,b,c,d\in\mathbf{Q}\bigr\}.
\]

\paragraph{Brackets vs.\ parentheses.}
Square brackets \emph{do not} force reciprocals of new elements.  
If one also wants to close under division, write
\[
  R(\alpha)
  \;=\;
  \text{the smallest \emph{field} containing } R[\alpha].
\]
When $R$ is already a field and $\alpha$ is \emph{algebraic} over $R$
(e.g.\ $\sqrt{2}$ over $\mathbf{Q}$), one actually has $R[\alpha]=R(\alpha)$.
For a \emph{transcendental} element (e.g.\ $\pi$) the two differ.
\end{document}
