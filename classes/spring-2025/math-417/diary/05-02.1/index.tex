\documentclass[12pt]{article}

% Packages
\usepackage[margin=1in]{geometry}
\usepackage{amsmath,amssymb,amsthm}
\usepackage{enumitem}
\usepackage{hyperref}
\usepackage{xcolor}
\usepackage{import}
\usepackage{xifthen}
\usepackage{pdfpages}
\usepackage{transparent}
\usepackage{listings}
\usepackage{tikz}
\usepackage{physics}
\usepackage{siunitx}
\usepackage{cancel}   % provides \cancel, \bcancel, \xcancel, …
  \usetikzlibrary{calc,patterns,arrows.meta,decorations.markings}


\DeclareMathOperator{\Log}{Log}
\DeclareMathOperator{\Arg}{Arg}

\lstset{
    breaklines=true,         % Enable line wrapping
    breakatwhitespace=false, % Wrap lines even if there's no whitespace
    basicstyle=\ttfamily,    % Use monospaced font
    frame=single,            % Add a frame around the code
    columns=fullflexible,    % Better handling of variable-width fonts
}

\newcommand{\incfig}[1]{%
    \def\svgwidth{\columnwidth}
    \import{./Figures/}{#1.pdf_tex}
}
\theoremstyle{definition} % This style uses normal (non-italicized) text
\newtheorem{solution}{Solution}
\newtheorem{proposition}{Proposition}
\newtheorem{problem}{Problem}
\newtheorem{lemma}{Lemma}
\newtheorem{theorem}{Theorem}
\newtheorem{remark}{Remark}
\newtheorem{note}{Note}
\newtheorem{definition}{Definition}
\newtheorem{example}{Example}
\newtheorem{corollary}{Corollary}
\theoremstyle{plain} % Restore the default style for other theorem environments
%

% Theorem-like environments
% Title information
\title{}
\author{Jerich Lee}
\date{\today}

\begin{document}

\maketitle
\[
\begin{aligned}
\chi_1-\chi_2
  &=\!\bigl(1+\sqrt{2}+\sqrt{3+\sqrt2}\bigr)
    -\bigl(1-\sqrt{2}+\sqrt{3-\sqrt2}\bigr) \\[2pt]
  &= (1-1) + \sqrt2 + \sqrt2
     +\sqrt{3+\sqrt2}-\sqrt{3-\sqrt2}\\[2pt]
  &= 2\sqrt2 + \Bigl(\sqrt{3+\sqrt2}-\sqrt{3-\sqrt2}\Bigr),
\\[12pt]
\chi_3-\chi_4
  &=\!\bigl(1+\sqrt{2}-\sqrt{3+\sqrt2}\bigr)
    -\bigl(1-\sqrt{2}-\sqrt{3-\sqrt2}\bigr) \\[2pt]
  &= (1-1) + \sqrt2 + \sqrt2
     -\sqrt{3+\sqrt2}+\sqrt{3-\sqrt2}\\[2pt]
  &= 2\sqrt2 + \Bigl(-\sqrt{3+\sqrt2}+\sqrt{3-\sqrt2}\Bigr).
\\[12pt]
\text{Add the two differences:}\quad
\chi_1-\chi_2+\chi_3-\chi_4
  &=\Bigl[2\sqrt2+\bigl(\sqrt{3+\sqrt2}-\sqrt{3-\sqrt2}\bigr)\Bigr] \\
  &\quad+\Bigl[2\sqrt2+\bigl(-\sqrt{3+\sqrt2}+\sqrt{3-\sqrt2}\bigr)\Bigr]\\[4pt]
  &= 4\sqrt2
    +\cancel{\sqrt{3+\sqrt2}}
    -\cancel{\sqrt{3-\sqrt2}}
    -\cancel{\sqrt{3+\sqrt2}}
    +\cancel{\sqrt{3-\sqrt2}}\\[4pt]
  &= 4\sqrt2.
\end{aligned}
\]

Hence
\[
\boxed{\ \chi_{1}-\chi_{2}+\chi_{3}-\chi_{4}-4\sqrt2 = 0\ }.
\]\[
  \begin{aligned}
  \chi_1-\chi_2
    &=\!\bigl(1+\sqrt{2}+\sqrt{3+\sqrt2}\bigr)
      -\bigl(1-\sqrt{2}+\sqrt{3-\sqrt2}\bigr) \\[2pt]
    &= (1-1) + \sqrt2 + \sqrt2
       +\sqrt{3+\sqrt2}-\sqrt{3-\sqrt2}\\[2pt]
    &= 2\sqrt2 + \Bigl(\sqrt{3+\sqrt2}-\sqrt{3-\sqrt2}\Bigr),
  \\[12pt]
  \chi_3-\chi_4
    &=\!\bigl(1+\sqrt{2}-\sqrt{3+\sqrt2}\bigr)
      -\bigl(1-\sqrt{2}-\sqrt{3-\sqrt2}\bigr) \\[2pt]
    &= (1-1) + \sqrt2 + \sqrt2
       -\sqrt{3+\sqrt2}+\sqrt{3-\sqrt2}\\[2pt]
    &= 2\sqrt2 + \Bigl(-\sqrt{3+\sqrt2}+\sqrt{3-\sqrt2}\Bigr).
  \\[12pt]
  \text{Add the two differences:}\quad
  \chi_1-\chi_2+\chi_3-\chi_4
    &=\Bigl[2\sqrt2+\bigl(\sqrt{3+\sqrt2}-\sqrt{3-\sqrt2}\bigr)\Bigr] \\
    &\quad+\Bigl[2\sqrt2+\bigl(-\sqrt{3+\sqrt2}+\sqrt{3-\sqrt2}\bigr)\Bigr]\\[4pt]
    &= 4\sqrt2
      +\cancel{\sqrt{3+\sqrt2}}
      -\cancel{\sqrt{3-\sqrt2}}
      -\cancel{\sqrt{3+\sqrt2}}
      +\cancel{\sqrt{3-\sqrt2}}\\[4pt]
    &= 4\sqrt2.
  \end{aligned}
  \]
  
  Hence
  \[
  \boxed{\ \chi_{1}-\chi_{2}+\chi_{3}-\chi_{4}-4\sqrt2 = 0\ }.
  \]
\end{document}
