\documentclass[12pt]{article}

% Packages
\usepackage[margin=1in]{geometry}
\usepackage{amsmath,amssymb,amsthm}
\usepackage{enumitem}
\usepackage{hyperref}
\usepackage{xcolor}
\usepackage{import}
\usepackage{xifthen}
\usepackage{pdfpages}
\usepackage{transparent}
\usepackage{listings}
\usepackage{tikz}
\usepackage{physics}
\usepackage{siunitx}
  \usetikzlibrary{calc,patterns,arrows.meta,decorations.markings}


\DeclareMathOperator{\Log}{Log}
\DeclareMathOperator{\Arg}{Arg}

\lstset{
    breaklines=true,         % Enable line wrapping
    breakatwhitespace=false, % Wrap lines even if there's no whitespace
    basicstyle=\ttfamily,    % Use monospaced font
    frame=single,            % Add a frame around the code
    columns=fullflexible,    % Better handling of variable-width fonts
}

\newcommand{\incfig}[1]{%
    \def\svgwidth{\columnwidth}
    \import{./Figures/}{#1.pdf_tex}
}
\theoremstyle{definition} % This style uses normal (non-italicized) text
\newtheorem{solution}{Solution}
\newtheorem{proposition}{Proposition}
\newtheorem{problem}{Problem}
\newtheorem{lemma}{Lemma}
\newtheorem{theorem}{Theorem}
\newtheorem{remark}{Remark}
\newtheorem{note}{Note}
\newtheorem{definition}{Definition}
\newtheorem{example}{Example}
\newtheorem{corollary}{Corollary}
\newtheorem{explanation}{Explanation}
\theoremstyle{plain} % Restore the default style for other theorem environments
%

% Theorem-like environments
% Title information
\title{MATH-417}
\author{Jerich Lee}
\date{\today}

\begin{document}

\maketitle
% ------------------------------------------------------------
%  Why “\(a\) does **not** divide \(r\)” once \(r\notin(a)\)
% ------------------------------------------------------------
\paragraph{Divisibility  $\boldsymbol{\Longleftrightarrow}$ membership in \((a)$.}

For any non–zero element \(a\) in an integral domain \(R\) and any
\(x\in R\),

\[
  \boxed{\;
     a \mid x
     \quad\Longleftrightarrow\quad
     x \in (a)
     \;=\;\{\,ra \mid r\in R\,\}.
  \;}
\]

* “\(a\mid x\)” means \(\exists\,r\in R\) with \(x = ra\);\;
  but that is exactly the condition \(x\in(a)\).
* Conversely, if \(x\in(a)\) there \emph{is} such an \(r\), so \(a\mid x\).

\medskip
\paragraph{Applying this to the proof.}

* You chose \(a\in I\) with minimal \(\varphi(a)\).  
* If \(I\neq(a)\), by definition of inequality of sets  
  \(\exists\,r\in I\) with \(r\notin(a)\).
* By the boxed equivalence, \(r\notin(a)\;\Longrightarrow\;a\nmid r\).

That single logical step is what the proof records as
“\emph{This means that \(a\) does not divide \(r\).}’’  
No hidden subtleties—just the dictionary between divisibility
and membership in the principal ideal generated by \(a\).
% ---------------------------------------------------------------------------
%  Theorem.  A Principal Ideal Domain (PID) satisfies the ACC on ideals
% ---------------------------------------------------------------------------
\begin{theorem}[Ascending Chain Condition in a PID]
  If \(R\) is a principal ideal domain, then every ascending chain of ideals
  \[
     I_{1}\;\subseteq\; I_{2}\;\subseteq\; I_{3}\;\subseteq\;\cdots
  \]
  is \emph{stationary}; i.e.\ there exists an index \(n\) such that
  \(I_{m}=I_{n}\) for all \(m\ge n\).
  \end{theorem}
  
  \begin{proof}
  \textbf{1.\;Form the union of the chain.}\;
  Let
  \[
     I \;:=\; \bigcup_{k\ge1} I_{k}.
  \]
  We first verify that \(I\) is an ideal of \(R\).
  
  \smallskip
  \emph{(i) Non–emptiness.}\;
  Since each \(I_k\) is an ideal, \(0\in I_k\subseteq I\).
  
  \emph{(ii) Closed under addition and negatives.}\;
  Take \(r,s\in I\).  Then \(r\in I_i\) and \(s\in I_j\) for some \(i,j\).
  Without loss of generality assume \(i\le j\); hence \(r\in I_j\).
  Because \(I_j\) is an ideal, \(r+s\in I_j\subseteq I\) and \(-r\in I_j\subseteq I\).
  
  \emph{(iii) Absorption.}\;
  If \(r\in I\) (so \(r\in I_i\) for some \(i\)) and \(a\in R\),
  then \(ar\in I_i\subseteq I\).
  
  Thus \(I\) is indeed an ideal of \(R\).
  
  \bigskip
  \textbf{2.\;Use the PID property to pick a single generator.}\;
  Because \(R\) is a PID, there exists an element \(b\in R\) such that
  \[
       I = (b).
  \]
  By definition of \(I\) we must have \(b\in I\); therefore \(b\in I_n\)
  for some index \(n\).
  
  \bigskip
  \textbf{3.\;Show the chain stabilises at } \(I_n\).
  
  \emph{Containment \(I\subseteq I_n\).}\;
  Since \(b\in I_n\) and \(I_n\) is an ideal,
  \((b)\subseteq I_n\); i.e.\ \(I\subseteq I_n\).
  
  \emph{Containment \(I_n\subseteq I\).}\;
  Trivial because \(I\) is the union of all \(I_k\).
  
  Hence \(I=I_n\).
  
  \bigskip
  \textbf{4.\;Conclude stationarity.}\;
  Because the sequence is ascending,
  \(
     I_n \subseteq I_{n+1} \subseteq I_{n+2}\subseteq\cdots
  \)
  and we have just shown \(I_{n}=I\).
  But \(I_{n+1}\subseteq I\) as well, so \(I_{n+1}=I_{n}\).
  Inductively it follows that
  \[
     I_{m}=I_{n}\quad\text{for all } m\ge n.
  \]
  Thus the chain stabilises, proving the Ascending Chain Condition for ideals in a PID.
  \end{proof}
  
  \bigskip
  \noindent\emph{Key idea.}\;
  The union \(I\) is an ideal because the chain is ascending;  
  the PID property supplies a single generator \(b\), which must appear in
  some stage of the chain and “locks’’ every subsequent ideal equal to
  \((b)\).  This finite ‘locking index’ is precisely the stationarity point.

  \begin{definition}
    Let $R$ be a ring with identity $1_R\neq 0_R$.  
    An element $u\in R$ is called a \emph{unit} if there exists $v\in R$ such that $uv=vu=1_R$.  
    An element $a\in R$ is a \emph{non-unit} if it is \textbf{not} a unit; equivalently, no element of $R$ serves as a multiplicative inverse of $a$.
\end{definition}
\begin{definition}
  Let $R$ be an integral domain.  
  A \emph{non-zero}, \emph{non-unit} element $a\in R$ is called \emph{irreducible} if whenever
  \[
      a = b\,c \qquad(b,c\in R),
  \]
  then at least one of $b$ or $c$ is a unit in $R$.  Equivalently, $a$ cannot be expressed as a product of two non-units.
\end{definition}
\begin{example}
  \textbf{In the ring $\mathbb{Z}$:}  
  The integer $7$ is irreducible.  
  It is non-zero and not a unit (the only units in $\mathbb{Z}$ are $\pm1$).  
  Suppose $7 = a\,b$ for $a,b\in\mathbb{Z}$.  
  Then $|a|,|b| \in \{1,7\}$, so at least one of $a$ or $b$ must be a unit ($\pm1$).  
  Hence $7$ satisfies the definition of an irreducible element.

  \medskip
  \textbf{In the polynomial ring $\mathbb{Q}[x]$:}  
  The polynomial $x^2 + 1$ is irreducible.  
  It is non-constant (thus non-unit) and has no roots in $\mathbb{Q}$; if it factored as
  \[
      x^2 + 1 = f(x)\,g(x), \qquad f,g\in\mathbb{Q}[x],
  \]
  then one of $f$ or $g$ would have to be a non-constant polynomial of degree $1$, contradicting the absence of rational roots.  
  Therefore one of the factors must be a unit in $\mathbb{Q}[x]$, so $x^2+1$ is irreducible.
\end{example}
\begin{example}[]
  Assume, for contradiction, that $x^{2}+1$ admits a non-trivial factorisation in the polynomial ring $\mathbb{Q}[x]$:
  \begin{align}
      x^{2}+1 \;=\; f(x)\,g(x), \qquad f,g \in \mathbb{Q}[x], \; \deg f,\deg g \ge 1 .
  \end{align}

  \begin{enumerate}[label=\textbf{Step \arabic*:}]
      \item \textbf{Necessity of a linear factor.}  
            Because $\deg f + \deg g = 2$, at least one factor must be linear.  
            Without loss of generality, write $f(x) = x - r$ with $r \in \mathbb{Q}$.
            
      \item \textbf{A rational root would follow.}  
            If $x - r$ divides $x^{2}+1$, then substituting $x = r$ yields 
            \begin{align}
                r^{2} + 1 \;=\; 0 
                \quad\Longrightarrow\quad
                r^{2} = -1 .
            \end{align}

      \item \textbf{Contradiction.}  
            The equation $r^{2} = -1$ has \emph{no} solution in $\mathbb{Q}$.  
            Hence no such rational root $r$ exists.
  \end{enumerate}

  Since assuming a factorisation forces the existence of a rational root—contradicting the fact that $x^{2}+1$ has none—the polynomial cannot factor non-trivially.  
  Therefore $x^{2}+1$ is irreducible in $\mathbb{Q}[x]$.
\end{example}
\begin{remark}[Why the ``non--unit'' condition is necessary]
  In any ring (or integral domain) \emph{units} already possess a
  multiplicative inverse.  Consequently, if we omitted the requirement
  that an irreducible element be a non--unit, every unit would
  \emph{trivially} satisfy the factorisation condition:
  \[
      u = (u)\cdot 1 ,
      \qquad\text{with both factors units in $R$.}
  \]
  Thus all units would be labelled ``irreducible,'' rendering the
  notion useless for describing the genuine ``building blocks'' of
  multiplication.

  Requiring an irreducible element to be \emph{non--unit} filters out
  these trivial cases and singles out the elements that cannot be
  broken down into smaller non-invertible factors.  These are the
  true atomic elements that play the central role in unique
  factorisation theories (e.g.\ in UFDs every non-zero, non-unit is a
  product of irreducibles, unique up to units and order).
\end{remark}
\begin{explanation}[Why every unit vacuously meets the “irreducible”
  factorisation test]
      Let $R$ be a ring with identity and let $u\in R^{\times}$ be a
      \textbf{unit}, i.e.\ there exists $u^{-1}\in R$ with
      $uu^{-1}=u^{-1}u = 1_R$.
  
      \medskip
      \textbf{(i)  The trivial factorisation \(\displaystyle u = u\cdot 1\).}
      \[
          u \;=\; \underbrace{u}_{\text{unit}}\;\cdot\;
                  \underbrace{1}_{\text{unit}} .
      \]
      Both factors are units, so the “irreducible” condition
      (“\emph{whenever \(a = bc\), at least one of \(b,c\) is a unit}”)
      is satisfied \emph{automatically}.
  
      \medskip
      \textbf{(ii)  Any factorisation forces the factors to be units.}\\
      Suppose we have \(\displaystyle u = b\,c\) with \(b,c\in R\).
      Multiply on the left by \(u^{-1}\):
      \[
          1_R = u^{-1}u = u^{-1}(b\,c) = (u^{-1}b)\,c .
      \]
      Hence \(c\) possesses the left–inverse \(u^{-1}b\), and (in a ring
      with identity) that is enough to make \(c\) a unit.
      A symmetric argument shows \(b\) is also a unit.  
      Therefore \emph{every} factorisation of a unit already satisfies
      the “one factor is a unit” requirement.
  
      \medskip
      \textbf{Conclusion.}  
      If we did \emph{not} exclude units from the definition, then
      \emph{all} units would be labelled “irreducible,” diluting the
      concept.  By insisting that an irreducible element be a
      \emph{non–unit}, we ensure the term refers only to the genuine
      “atomic” (non-invertible) building blocks of the ring.
  \end{explanation}
  \begin{theorem}
    If $R$ is a principal ideal domain (PID), then every non-zero,
    non-unit element of $R$ can be written as a \emph{finite} product of
    irreducible elements.
\end{theorem}

\begin{proof}[Step-by-step]
    We break the argument into two main parts.

    \begin{enumerate}[label=\textbf{Part \arabic*:}, leftmargin=1.5em]

        %--------------------------------------------------------------
        \item \textbf{Every non-zero, non-unit has \emph{at least one}
              irreducible divisor.}

              \begin{enumerate}[label=\arabic*. , leftmargin=2em]
                  \item Let $a\in R$ be non-zero and not a unit.
                        If $a$ is already irreducible, we are done.

                  \item Otherwise, there exist non-units $b_{1},a_{1}\in R$
                        with $a = b_{1}\,a_{1}$.
                        Because $b_{1}$ is \emph{not} a unit, we obtain a
                        \emph{strict} containment of principal ideals:
                        \begin{align}
                            (a) \subsetneq (a_{1}).
                        \end{align}
                        % \textbf{Expanded explanation of Step~1.2}

\begin{enumerate}[label=\arabic*. , leftmargin=2em]
  \item[] \textbf{Context.}  
        We have a non–zero, non–unit element $a\in R$ (with $R$ a PID) 
        and we are assuming $a$ is \emph{not} irreducible.  
        By definition of irreducibility, this means
        \[
            \exists\,b_{1},a_{1}\in R\quad 
            \text{such that}\quad
            a = b_{1}a_{1},
            \quad
            b_{1},a_{1}\ \text{are \emph{non–units}.}
        \]

  \item \textbf{Why $(a)\subseteq(a_{1})$.}  
        The principal ideal $(a_{1})$ is the set
        \[
            (a_{1}) \;=\; \{\,a_{1}r \mid r\in R\,\}.
        \]
        Since $a = b_{1}a_{1}$, the element $a$ belongs to $(a_{1})$,
        hence every multiple of $a$ (that is, every element of $(a)$)
        also lies in $(a_{1})$.  Therefore
        \[
            (a)\subseteq(a_{1}).
        \]

  \item \textbf{Why the containment is \emph{strict}.}  
        Suppose, for contradiction, that $(a)=(a_{1})$.  
        In a PID two principal ideals are equal \emph{iff} their
        generators differ by a unit, so there exists a unit
        $u\in R^{\times}$ with
        \[
            a \;=\; u\,a_{1}.
        \]
        Combining this with the earlier factorisation
        $a = b_{1}a_{1}$ gives
        \[
            b_{1}a_{1} \;=\; u\,a_{1}
            \quad\Longrightarrow\quad
            (b_{1}-u)a_{1}=0.
        \]
        Because $R$ is an integral domain and $a_{1}\neq 0$, we must
        have $b_{1}=u$, i.e.\ $b_{1}$ is a \emph{unit}.  
        This contradicts the fact that $b_{1}$ was chosen to be a
        non–unit.  Hence $(a)\neq(a_{1})$, and we conclude
        \[
            (a)\subsetneq(a_{1}).
        \]
\end{enumerate}

                  \item If $a_{1}$ is irreducible we are finished; if not,
                        factor again:
                        \[
                            a_{1} = b_{2}\,a_{2},\qquad
                            b_{2},a_{2}\text{ non-units},
                        \]
                        yielding another strict containment
                        $(a_{1}) \subsetneq (a_{2})$.

                  \item Iterating produces an \emph{ascending} chain of
                        ideals
                        \[
                            (a) \subsetneq (a_{1}) \subsetneq (a_{2})
                            \subsetneq \cdots .
                        \]
                        In a PID every ascending chain of ideals
                        stabilises (ACC).  Hence the process must stop
                        after finitely many steps, and some $a_{k}$ is
                        irreducible.
              \end{enumerate}

        %--------------------------------------------------------------
        \item \textbf{Every non-zero, non-unit factors into
              \emph{finitely many} irreducibles.}

              \begin{enumerate}[label=\arabic*. , leftmargin=2em]
                  \item If $a$ is irreducible we are done.  
                        Otherwise write
                        \[
                            a = p_{1}\,c_{1},
                        \]
                        where $p_{1}$ is irreducible and
                        $c_{1}$ is a non-unit.  Then
                        $(a) \subsetneq (c_{1})$.

                  \item If $c_{1}$ is irreducible we are finished.
                        Otherwise factor again:
                        \[
                            c_{1} = p_{2}\,c_{2},
                            \qquad p_{2}\text{ irreducible},
                            \;c_{2}\text{ non-unit},
                        \]
                        so $(c_{1}) \subsetneq (c_{2})$.

                  \item Repeating gives a strictly ascending chain
                        \[
                            (a) \subsetneq (c_{1}) \subsetneq (c_{2})
                            \subsetneq \cdots .
                        \]
                        The ACC in a PID forces this chain to become
                        stationary: for some $r$, $c_{r}$ is irreducible.

                  \item Collecting all the irreducible factors we have
                        obtained,
                        \[
                            a = p_{1}\,p_{2}\,\dots\,p_{r}\,c_{r},
                        \]
                        a \emph{finite} product of irreducibles, as
                        required.
              \end{enumerate}

    \end{enumerate}
\end{proof}
\begin{definition}
  For a ring $R$ with identity $1_R\neq 0_R$, the set
  \[
      R^{\*}\;=\;\{\,u\in R \mid \exists\,v\in R\text{ such that }uv=vu=1_R\,\}
  \]
  is called the \emph{group of units} (or \emph{unit group}) of $R$.
  Thus $R^{\*}$ (often written $R^{\times}$) consists precisely of
  those elements of $R$ that have a multiplicative inverse, with the
  operation being the ring multiplication.
\end{definition}
\begin{proposition}
  Let $R$ be a principal ideal domain (PID) and let 
  $a,b\in R$.  
  Then the principal ideals $(a)$ and $(b)$ are equal
  \[
      (a) \;=\; (b)
  \]
  \emph{iff} their generators differ by a unit; i.e.\ there exists a
  unit $u\in R^{\*}$ such that
  \[
      a \;=\; u\,b
      \quad\Longleftrightarrow\quad
      b \;=\; u^{-1}a .
  \]
\end{proposition}

\begin{proof}
  We prove both directions.

  \medskip
  \noindent\textbf{($\Rightarrow$)  Equality of ideals $\;\Longrightarrow\;$ generators differ by a unit.}

  Assume $(a)=(b)$.  
  By definition of principal ideals this means
  \[
      a = r\,b 
      \quad\text{and}\quad
      b = s\,a
      \qquad\text{for some } r,s\in R .
  \]
  Substituting the second equation into the first gives
  \[
      a \;=\; r\,(s\,a) \;=\; (rs)\,a .
  \]
  Because $R$ is an integral domain (every PID is), we may cancel
  the non-zero factor $a$ to obtain
  \[
      rs \;=\; 1_R .
  \]
  Thus $r$ (and likewise $s$) is a \emph{unit} in $R$.  
  Setting $u:=r\in R^{\*}$ we have $a=u\,b$, as desired.

  \medskip
  \noindent\textbf{($\Leftarrow$)  Generators differ by a unit $\;\Longrightarrow\;$ equality of ideals.}

  Now suppose $a = u\,b$ with $u\in R^{\*}$.  
  Because $u$ has an inverse $u^{-1}$, we have
  \[
      (a) \;=\; (u\,b)
            \;=\; \{\,u\,b\,r \mid r\in R\,\}
            \;=\; \{\,b\,(u\,r) \mid r\in R\,\}
            \;=\; (b),
  \]
  where the third equality uses that multiplication by the unit $u$
  is a bijection on~$R$ (so $u\,r$ ranges over \emph{all} of $R$ as
  $r$ does).  Hence the ideals coincide.

  \medskip
  Combining the two implications completes the proof.
\end{proof}
\end{document}
