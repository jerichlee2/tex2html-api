\documentclass[12pt]{article}

% Packages
\usepackage[margin=1in]{geometry}
\usepackage{amsmath,amssymb,amsthm}
\usepackage{enumitem}
\usepackage{hyperref}
\usepackage{xcolor}
\usepackage{import}
\usepackage{xifthen}
\usepackage{pdfpages}
\usepackage{transparent}
\usepackage{listings}
\usepackage{tikz}
\usepackage{physics}
\usepackage{siunitx}
\usepackage{booktabs}
\usepackage{cancel}
  \usetikzlibrary{calc,patterns,arrows.meta,decorations.markings}


\DeclareMathOperator{\Log}{Log}
\DeclareMathOperator{\Arg}{Arg}


\lstset{
    breaklines=true,         % Enable line wrapping
    breakatwhitespace=false, % Wrap lines even if there's no whitespace
    basicstyle=\ttfamily,    % Use monospaced font
    frame=single,            % Add a frame around the code
    columns=fullflexible,    % Better handling of variable-width fonts
}

\newcommand{\incfig}[1]{%
    \def\svgwidth{\columnwidth}
    \import{./Figures/}{#1.pdf_tex}
}
\theoremstyle{definition} % This style uses normal (non-italicized) text
\newtheorem{solution}{Solution}
\newtheorem{proposition}{Proposition}
\newtheorem{problem}{Problem}
\newtheorem{lemma}{Lemma}
\newtheorem{theorem}{Theorem}
\newtheorem{remark}{Remark}
\newtheorem{note}{Note}
\newtheorem{definition}{Definition}
\newtheorem{example}{Example}
\newtheorem{corollary}{Corollary}
\theoremstyle{plain} % Restore the default style for other theorem environments
%

% Theorem-like environments
% Title information
\title{MATH 417 Practice Final Exam 1}
\author{Jerich Lee}
\date{\today}

\begin{document}

\maketitle
%------------------------------------------------------------
%  Practice Final Exam – MATH 417: Introduction to Abstract Algebra
%  13 questions • no calculators • show all reasoning
%------------------------------------------------------------
\newcommand{\Z}{\mathbb Z}
\newcommand{\Q}{\mathbb Q}
\newcommand{\R}{\mathbb R}

\bigskip
\begin{problem}
  Let \(G=\{0,1,2,3\}\) and define
  \[
      a\star b \;:=\; a+b+1\pmod{4}.
  \]
  \begin{enumerate}
      \item[(a)] Fill in the Cayley table of \((G,\star)\).
      \item[(b)] Determine the identity element.
      \item[(c)] Find the inverse of each element.
      \item[(d)] Verify associativity by checking one non-trivial triple of elements.
  \end{enumerate}
\end{problem}

\bigskip
\begin{problem}
  Let \(G\) be a group and define a relation on \(G\) by
  \(a\sim b \iff \exists\,g\in G\text{ such that }b=gag^{-1}\).
  \begin{enumerate}
      \item[(a)] Prove that \(\sim\) is an equivalence relation
                (\emph{conjugacy}).
      \item[(b)] Show that elements lying in the same conjugacy class
                have the same order.
  \end{enumerate}
\end{problem}

\bigskip
\begin{problem}
  Let
  \[
      C_{8}\;=\;\bigl\{e^{2\pi i k/8}:k=0,1,\dots,7\bigr\}
      \subset \mathbb{C}^{\times}.
  \]
  \begin{enumerate}
      \item[(a)] Prove that \((C_{8},\cdot)\) is a cyclic (hence abelian) group.
      \item[(b)] List all subgroups of \(C_{8}\) and draw the subgroup lattice.
  \end{enumerate}
\end{problem}

\bigskip
\begin{problem}
  \begin{enumerate}
      \item[(a)] Show that any subgroup of index \(2\) in a group is normal.
      \item[(b)] Provide an explicit example of a subgroup of index \(3\)
                that is \emph{not} normal, and justify your choice.
  \end{enumerate}
\end{problem}

\bigskip
\begin{problem}
  Determine \emph{all} group homomorphisms
  \(\varphi:\Z_{15}\longrightarrow\Z_{6}\)
  and count how many distinct homomorphisms there are.
\end{problem}

\bigskip
\begin{problem}
  Let \(G\) be a group of order \(63=3^{2}\cdot7\).
  \begin{enumerate}
      \item[(a)] Use Sylow’s theorems to show that \(G\) has a unique
                Sylow \(7\)-subgroup, which is therefore normal.
      \item[(b)] Is \(G\) necessarily abelian?  Prove your claim or give a
                counter-example.
  \end{enumerate}
\end{problem}

\bigskip
\begin{problem}
  Classify, up to isomorphism, all abelian groups of order \(84\).
\end{problem}

\bigskip
\begin{problem}
  Consider the multiplicative group \(\Z_{20}^{\ast}\).
  \begin{enumerate}
      \item[(a)] Compute the order of every element of \(\Z_{20}^{\ast}\).
      \item[(b)] Identify all generators of \(\Z_{20}^{\ast}\).
  \end{enumerate}
\end{problem}

\bigskip
\begin{problem}
  \begin{enumerate}
      \item[(a)] Prove Wilson’s theorem: for an odd prime \(p\),
                \((p-1)!\equiv -1\pmod{p}\).
      \item[(b)] Use part (a) to evaluate \(12!\pmod{13}\).
  \end{enumerate}
\end{problem}

\bigskip
\begin{problem}
  \begin{enumerate}
      \item[(a)] Prove that every finite integral domain is a field.
      \item[(b)] Give a concrete example of a finite ring with identity
                that \emph{is not} an integral domain, and list all of its
                zero divisors.
  \end{enumerate}
\end{problem}

\bigskip
\begin{problem}
  Let \(\Z[i]\) be the ring of Gaussian integers.
  \begin{enumerate}
      \item[(a)] Show that \(\Z[i]\) is a Euclidean domain with respect
                to the norm \(N(a+bi)=a^{2}+b^{2}\).
      \item[(b)] Compute \(\gcd(4+5i,\;7+2i)\) in \(\Z[i]\), expressing it
                as a Gaussian integer of minimal norm.
  \end{enumerate}
\end{problem}

\bigskip
\begin{problem}
  Let \(F\) be the (unique) field with \(16\) elements.
  \begin{enumerate}
      \item[(a)] Show that \(F\) contains exactly three subfields and
                describe each of them explicitly.
      \item[(b)] How many elements of order \(15\) does \(F^{\times}\) have?
  \end{enumerate}
\end{problem}

\bigskip
\begin{problem}
  Let \(p\) be a prime and consider the elementary abelian group
  \(V=\Z_{p}\oplus\Z_{p}\).
  \begin{enumerate}
      \item[(a)] Prove that \(\operatorname{Aut}(V)\cong GL_{2}(\Z_{p})\).
      \item[(b)] Compute \(|\operatorname{Aut}(V)|\) as an explicit
                function of \(p\).
  \end{enumerate}
\end{problem}
%------------------------------------------------------------
\end{document}
