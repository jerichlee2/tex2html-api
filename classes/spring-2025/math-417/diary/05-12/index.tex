\documentclass[12pt]{article}

% Packages
\usepackage[margin=1in]{geometry}
\usepackage{amsmath,amssymb,amsthm}
\usepackage{enumitem}
\usepackage{hyperref}
\usepackage{xcolor}
\usepackage{import}
\usepackage{xifthen}
\usepackage{pdfpages}
\usepackage{transparent}
\usepackage{listings}
\usepackage{tikz}
\usepackage{physics}
\usepackage{siunitx}
\usepackage{booktabs}
\usepackage{cancel}
  \usetikzlibrary{calc,patterns,arrows.meta,decorations.markings}

\DeclareMathOperator{\Cl}{Cl}
\DeclareMathOperator{\Log}{Log}
\DeclareMathOperator{\Arg}{Arg}

\lstset{
    breaklines=true,         % Enable line wrapping
    breakatwhitespace=false, % Wrap lines even if there's no whitespace
    basicstyle=\ttfamily,    % Use monospaced font
    frame=single,            % Add a frame around the code
    columns=fullflexible,    % Better handling of variable-width fonts
}

\newcommand{\incfig}[1]{%
    \def\svgwidth{\columnwidth}
    \import{./Figures/}{#1.pdf_tex}
}
\theoremstyle{definition} % This style uses normal (non-italicized) text
\newtheorem{solution}{Solution}
\newtheorem{proposition}{Proposition}
\newtheorem{problem}{Problem}
\newtheorem{lemma}{Lemma}
\newtheorem{theorem}{Theorem}
\newtheorem{remark}{Remark}
\newtheorem{note}{Note}
\newtheorem{definition}{Definition}
\newtheorem{example}{Example}
\newtheorem{corollary}{Corollary}
\newtheorem{procedure}{Procedure}

\newtheorem{explanation}{Explanation}

\theoremstyle{plain} % Restore the default style for other theorem environments
%

% Theorem-like environments
% Title information
\title{}
\author{Jerich Lee}
\date{\today}

\begin{document}

\maketitle
\begin{proposition}
  Let \(p\) be a prime number and let \(G\) be a finite group with
  \(\lvert G\rvert = p^{2}\).
  Then the center \(Z(G)\) is non-trivial; in fact 
  \[
     \lvert Z(G)\rvert \in \{p,\;p^{2}\}.
  \]
  \end{proposition}
  
  \begin{proof}
  Because \(\lvert G\rvert = p^{2}\) is a power of the prime \(p\),
  \(G\) is a \(p\)--group.
  A standard fact about \(p\)--groups is that their centers are non-trivial; nevertheless, we provide a direct argument for this special case using the class equation.
  
  \medskip
  \noindent\textbf{1.  The class equation.}
  Write the class equation of \(G\):
  \[
     \lvert G\rvert
        \;=\;
        \lvert Z(G)\rvert
        \;+\;
        \sum_{i=1}^{k}\,
        \lvert\Cl_{G}(g_{i})\rvert,
  \]
  where each \(g_{i}\notin Z(G)\) and
  \(\Cl_{G}(g_{i})=\{\,xg_{i}x^{-1}:x\in G\,\}\) is the conjugacy
  class of \(g_{i}\).
  
  \medskip
  \noindent\textbf{2.  Sizes of the conjugacy classes.}
  For any \(g\in G\) the size of its conjugacy class satisfies
  \[
     \lvert\Cl_{G}(g)\rvert
        = \frac{\lvert G\rvert}{\lvert C_{G}(g)\rvert},
  \]
  where \(C_{G}(g)=\{\,x\in G : xg=gx\,\}\) is the centralizer of \(g\).
  Hence \(\lvert\Cl_{G}(g)\rvert\) divides \(\lvert G\rvert=p^{2}\).
  If \(g\notin Z(G)\) then \(C_{G}(g)\neq G\), so
  \(\lvert C_{G}(g)\rvert\) must equal \(p\) (it cannot be \(1\) because
  \(e\in C_{G}(g)\)).
  Consequently
  \[
     \lvert\Cl_{G}(g)\rvert
        =\frac{p^{2}}{p}=p .
  \]
  
  \medskip
  \noindent\textbf{3.  Applying the class equation.}
  Assume, for contradiction, that \(Z(G)=\{e\}\),
  so \(\lvert Z(G)\rvert=1\).
  Then every non-central element lies in some conjugacy class of size \(p\).
  Because each such class has \(p\) elements and they are pairwise
  disjoint, the class equation gives
  \[
     p^{2}=\underbrace{1}_{\lvert Z(G)\rvert}
            \;+\;
            k\,p
     \;\;\Longrightarrow\;\;
     p^{2}-1 = k\,p .
  \]
  Dividing both sides by \(p\) yields
  \(p-1 = k\), so \(k=p-1\).
  But then the total number of elements outside the center is
  \(k\,p = (p-1)p = p^{2}-p\),
  and together with the central element \(e\) this accounts for exactly
  \(p^{2}-p+1\) elements—\emph{strictly fewer} than \(p^{2}\) when
  \(p>1\).
  This contradiction shows that our assumption was false; hence
  \(\lvert Z(G)\rvert>1\).
  
  \medskip
  \noindent\textbf{4.  Determining \(\lvert Z(G)\rvert\).}
  Since \(\lvert Z(G)\rvert\) divides \(\lvert G\rvert=p^{2}\) and is
  greater than \(1\), it must be \(p\) or \(p^{2}\).
  If \(\lvert Z(G)\rvert=p^{2}\), then \(Z(G)=G\) and \(G\) is abelian.
  Otherwise \(\lvert Z(G)\rvert=p\).
  
  \medskip
  \noindent\textbf{Conclusion.}
  In every case, the center \(Z(G)\) is non-trivial, completing the
  proof.
  \end{proof}
  \begin{explanation}
    \textbf{What does “cyclic’’ mean?}  
    A group \(G\) is called \emph{cyclic} if there exists an element
    \(g\in G\) such that every element of \(G\) can be written as a power
    (of \(g\)) in multiplicative notation or a multiple (of \(g\)) in
    additive notation:
    \[
       G \;=\; \langle g\rangle
         \;=\;\{\,g^{k} : k\in\mathbb{Z}\,\}.
    \]
    Such an element \(g\) is a \emph{generator} of \(G\).
    
    \medskip
    Below are the most common strategies for proving that a concrete group
    is cyclic.  Pick whichever fits the information you know about the
    group in question.
    
    \begin{enumerate}[label=\textbf{(\arabic*)},leftmargin=2.2em]
    
      \item \textbf{Find an element whose order equals
            \(\lvert G\rvert\) (finite case).}\\
            Thanks to Lagrange’s theorem, every element order divides
            \(\lvert G\rvert\).  
            Hence if you can exhibit an element \(g\) with
            \(\operatorname{ord}(g)=\lvert G\rvert\), then
            \(\langle g\rangle\) already contains \(\lvert G\rvert\)
            distinct elements—so it \emph{must} be all of \(G\).
    
            \smallskip
            \textit{Typical tools:}
            \begin{itemize}
              \item Cauchy’s theorem (for prime divisors of \(\lvert G\rvert\));
              \item the structure theorem for finite abelian groups;
              \item counting arguments that show some element attains the
                    maximum possible order.
            \end{itemize}
    
      \item \textbf{Show that \(G\) has at most one subgroup of each order
            dividing \(\lvert G\rvert\).}\\
            A classical theorem says:
            \[
               \text{“A finite group is cyclic}
               \;\Longleftrightarrow\;
               \text{it has exactly one subgroup of every divisor of
               its order.”}
            \]
            Thus, if you can prove the uniqueness of subgroups of each
            possible order (for example via group actions or counting),
            cyclicity follows automatically.
    
      \item \textbf{Exploit the structure of abelian groups.}\\
            If \(G\) is \emph{finite abelian}, write its primary
            decomposition
            \(
                G \;\cong\;
                C_{p_{1}^{\,a_{1}}}\times\cdots\times
                C_{p_{k}^{\,a_{k}}}.
            \)
            Choose generators \(g_{i}\) of each cyclic
            factor; then the element
            \(
                g \;:=\;
                g_{1}\,g_{2}\cdots g_{k}
            \)
            has order
            \(
                \operatorname{lcm}\bigl(
                      p_{1}^{\,a_{1}},\dots,p_{k}^{\,a_{k}}
                \bigr)=\lvert G\rvert,
            \)
            so \(G=\langle g\rangle\).
            (The same idea works Sylow–group–by–Sylow–group if you have
            already shown each Sylow subgroup is cyclic.)
    
      \item \textbf{Infinite groups:  build an isomorphism with
            \((\mathbb{Z},+)\).}\\
            Show there is a surjective homomorphism
            \(f:\mathbb{Z}\twoheadrightarrow G\);  
            its kernel is of the form \(n\mathbb{Z}\), so by the First
            Isomorphism Theorem,
            \(
               G\cong \mathbb{Z}/n\mathbb{Z}
               \;(n<\infty)
               \quad\text{or}\quad
               G\cong \mathbb{Z}
               \;(n=0).
            \)
            Either way \(G\) is cyclic, generated by \(f(1)\).
    
      \item \textbf{Small–order tricks.}\\
            For very small orders there are only a handful of possible
            groups, so classification works quickly:
    
            \begin{center}
            \begin{tabular}{c|l}
               \(\lvert G\rvert\) & Result\\\hline
               \(p\) (prime)      & \(G\) is automatically cyclic.\\
               \(p^{2}\)          & \(G\cong C_{p^{2}}\) or
                                    \(C_{p}\times C_{p}\);
                                    find an element of order \(p^{2}\) to
                                    settle the first case.\\
               \(4,6,8,10,\dots\) & Exploit generators of known
                                    presentations, e.g.\ \(D_{2n}\) is
                                    not cyclic because every element has
                                    order dividing \(n\).
            \end{tabular}
            \end{center}
    
    \end{enumerate}
    
    \medskip
    \noindent\textbf{Key takeaway.}
    \emph{To prove a group is cyclic, your main objective is to locate one
    element whose powers (or multiples) exhaust the entire group.}
    Most techniques above are simply different ways to guarantee such an
    element exists.
    \end{explanation}
    \begin{procedure}[Determining the conjugacy classes of the dihedral group]
      \textbf{Goal.}
      Given an integer \(n\ge 2\), let
      \[
         D_{2n}
            \;=\;
            \bigl\langle\,r,s
            \;\bigm|\;
              r^{n}=e,\;
              s^{2}=e,\;
              srs=r^{-1}
            \bigr\rangle
            \quad
            (\lvert D_{2n}\rvert = 2n).
      \]
      We describe a practical, step–by–step method for listing all conjugacy
      classes of \(D_{2n}\).
      
      \medskip
      \begin{enumerate}[label=\textbf{Step \arabic*.},leftmargin=2.8em]
      
      \item \textbf{Separate the two types of elements.}
            \[
               \underbrace{\langle r\rangle
                 \;=\;\{e,r,r^{2},\dots,r^{n-1}\}}_{\text{rotations}}
               \quad\sqcup\quad
               \underbrace{
                 \bigl\{\,s r^{k}\;:\;0\le k\le n-1\bigr\}
               }_{\text{reflections}}.
            \]
            Conjugacy relations interact differently inside each part, so
            treat rotations and reflections separately.
      
      \item \textbf{Handle the rotations.}
            \begin{itemize}
               \item Because \(r\) commutes with every
                     power of itself, conjugating \(r^{k}\) by a rotation
                     does nothing.
               \item Conjugating \(r^{k}\) by a reflection flips the
                     exponent:
                     \(
                       s\,r^{k}\,s^{-1}=r^{-k}=r^{n-k}.
                     \)
            \end{itemize}
            Therefore
            \[
               \Cl\bigl(r^{k}\bigr)=
               \begin{cases}
                  \{\,r^{k},\,r^{n-k}\,\},
                     & k\not\equiv -k\pmod{n};\\[4pt]
                  \{\,r^{k}\,\}, & k\equiv -k\pmod{n}.
               \end{cases}
            \]
            Concretely:
            \begin{itemize}
               \item If \(n\) is odd:
                     the non-trivial rotations split into
                     \((n-1)/2\) two-element classes
                     \(\{r^{k},r^{n-k}\}\) \((1\le k\le (n-1)/2)\).
               \item If \(n\) is even:
                     the rotation \(r^{n/2}\) is central, so
                     \(\{r^{n/2}\}\) is a singleton class;
                     the remaining \(n/2-1\) pairs
                     \(\{r^{k},r^{n-k}\}\) form the other classes.
            \end{itemize}
      
      \item \textbf{Handle the reflections.}
            Write every reflection as \(s r^{k}\).
            Conjugating by a rotation gives
            \[
               r^{m}\,(s r^{k})\,r^{-m}=s r^{k-2m},
            \]
            so rotation conjugation acts by \(\;k\mapsto k-2m\pmod{n}\).
      
            \vspace{2pt}
            \begin{itemize}
               \item If \(n\) is \emph{odd}, the map \(m\mapsto 2m\)
                     is a bijection on \(\mathbb{Z}/n\mathbb{Z}\); hence
                     the orbit of any \(k\) is all of \(\mathbb{Z}/n\mathbb{Z}\).
                     \[
                        \boxed{\text{All \(n\) reflections form one class.}}
                     \]
               \item If \(n\) is \emph{even}, the map \(m\mapsto 2m\)
                     hits only the even residues.
                     Thus the reflections split into two orbits:
                     \[
                       \begin{aligned}
                          \mathcal{R}_{0}
                             &=\{\,s r^{k}\;:\;k\text{ even}\},\\
                          \mathcal{R}_{1}
                             &=\{\,s r^{k}\;:\;k\text{ odd}\}.
                       \end{aligned}
                     \]
                     Each set has size \(n/2\) and is a conjugacy class.
                     (Geometrically these correspond to axes through
                     opposite vertices versus opposite edges of the regular
                     \(n\)-gon.)
            \end{itemize}
      
      \item \textbf{Collect the results.}
            \[
               \boxed{
                  \begin{aligned}
                     &\text{Identity: }\{e\};\\
                     &\text{Rotations: as in Step 2;}\\
                     &\text{Reflections: as in Step 3.}
                  \end{aligned}
               }
            \]
      
      \end{enumerate}
      
      \noindent\textbf{Quick summary of class counts.}
      
      \begin{center}
      \renewcommand{\arraystretch}{1.3}
      \begin{tabular}{c|c|c}
         & \(n\) odd & \(n\) even\\\hline
         rotation classes &
            \(1\;+\;\dfrac{n-1}{2}\)
               & \(2\;+\;\bigl(\dfrac{n}{2}-1\bigr)\)\\
         reflection classes & \(1\) & \(2\)\\\hline
         \textbf{total classes} &
            \(\dfrac{n+3}{2}\) &
            \(\dfrac{n}{2}+3\)
      \end{tabular}
      \end{center}
      \end{procedure}
      \begin{remark}[Conjugacy classes of the quaternion group \(Q_{8}\)]
        Recall
        \[
           Q_{8}
              \;=\;
              \bigl\{\,\pm1,\;\pm i,\;\pm j,\;\pm k\bigr\},
              \qquad
              i^{2}=j^{2}=k^{2}=ijk=-1.
        \]
        Its center is
        \(
           Z(Q_{8})=\{\pm1\},
        \)
        so each element of \(Z(Q_{8})\) forms a singleton conjugacy class.
        For the remaining elements one checks that
        \[
           x^{-1}\,i\,x
              \in\{\pm i\},
           \quad
           x^{-1}\,j\,x
              \in\{\pm j\},
           \quad
           x^{-1}\,k\,x
              \in\{\pm k\}
           \quad
           \text{for all }x\in Q_{8}.
        \]
        Hence the conjugacy classes are
        \[
           \boxed{
           \{\,1\},
           \;\{-1\},
           \;\{\,i,-i\},
           \;\{\,j,-j\},
           \;\{\,k,-k\}.
           }
        \]
        There are \(5\) classes in total:
        two of size \(1\) (the central elements) and three of size \(2\).
        \end{remark}
\end{document}
