\documentclass[12pt]{article}

% Packages
\usepackage[margin=1in]{geometry}
\usepackage{amsmath,amssymb,amsthm}
\usepackage{enumitem}
\usepackage{hyperref}
\usepackage{xcolor}
\usepackage{import}
\usepackage{xifthen}
\usepackage{pdfpages}
\usepackage{transparent}
\usepackage{listings}
\usepackage{tikz}
\usepackage{physics}
\usepackage{siunitx}
\usepackage{booktabs}
\usepackage{cancel}
  \usetikzlibrary{calc,patterns,arrows.meta,decorations.markings}


\DeclareMathOperator{\Log}{Log}
\DeclareMathOperator{\Arg}{Arg}
\DeclareMathOperator{\Gal}{Gal}
\DeclareMathOperator{\Aut}{Aut}



\lstset{
    breaklines=true,         % Enable line wrapping
    breakatwhitespace=false, % Wrap lines even if there's no whitespace
    basicstyle=\ttfamily,    % Use monospaced font
    frame=single,            % Add a frame around the code
    columns=fullflexible,    % Better handling of variable-width fonts
}

\newcommand{\incfig}[1]{%
    \def\svgwidth{\columnwidth}
    \import{./Figures/}{#1.pdf_tex}
}
\theoremstyle{definition} % This style uses normal (non-italicized) text
\newtheorem{solution}{Solution}
\newtheorem{proposition}{Proposition}
\newtheorem{problem}{Problem}
\newtheorem{lemma}{Lemma}
\newtheorem{theorem}{Theorem}
\newtheorem{remark}{Remark}
\newtheorem{note}{Note}
\newtheorem{definition}{Definition}
\newtheorem{example}{Example}
\newtheorem{corollary}{Corollary}
\theoremstyle{plain} % Restore the default style for other theorem environments
%

% Theorem-like environments
% Title information
\title{MATH 417 Practice Final Exam 3 Solutions}
\author{Jerich Lee}
\date{\today}

\begin{document}

\maketitle
\begin{problem}
  On the set \(\Bbb R\) define the binary operation
  \[
     x\*y \;:=\; x+y+xy .
  \]
  \begin{enumerate}[]
     \item Show that \(\bigl(\Bbb R\setminus\{-1\},\*\bigr)\) is a group.
     \item Find the identity element and the inverse of an arbitrary
           \(x\in\Bbb R\setminus\{-1\}\).
     \item Determine the subgroup generated by \(\tfrac12\).
  \end{enumerate}
  \end{problem}
  
  \begin{solution}
  Throughout write
  \[
     G=\Bbb R\setminus\{-1\},
     \qquad
     x\*y = x+y+xy.
  \]
  A useful observation is
  \[
     x\*y
       = x+y+xy
       = (1+x)(1+y)-1,
  \]
  so the map
  \[
     \varphi:G\longrightarrow\Bbb R^{\times},
     \qquad
     \varphi(x)=1+x,
  \]
  satisfies
  \(\varphi(x\*y)=\varphi(x)\varphi(y)\).
  
  %---------------------------------------------------------------
  \bigskip
  \textbf{(a)  Group axioms.}
  
  \emph{Closure.}
  If \(x,y\in G\) then \(1+x\neq0\neq1+y\); hence
  \((1+x)(1+y)\neq0\) and \(x\*y=(1+x)(1+y)-1\neq-1\), so \(x\*y\in G\).
  
  \emph{Associativity.}
  \[
  \bigl(x\*y\bigr)\*z
     =(x+y+xy)+z+(x+y+xy)z
     =(1+x)(1+y)(1+z)-1
     =x\*\bigl(y\*z\bigr).
  \]
  
  \emph{Identity.}
  Solve \(x\*e=x\):
  \[
     x+e+xe=x
     \;\Longrightarrow\;
     e(1+x)=0\quad(\forall x\in G)\;\Longrightarrow\; e=0.
  \]
  Since \(0\neq-1\), we have \(e=0\in G\).
  
  \emph{Inverses.}
  Given \(x\in G\), seek \(y\) with \(x\*y=0\):
  \[
     x+y+xy=0
     \;\Longrightarrow\;
     y(1+x)=-x
     \;\Longrightarrow\;
     y=\frac{-x}{1+x}.
  \]
  Because \(1+x\neq0\), this \(y\in G\) and is the unique inverse of \(x\).
  
  Thus \(\bigl(G,\*\bigr)\) is a group.
  
  %---------------------------------------------------------------
  \bigskip
  \textbf{(b)  Summary of identity and inverse.}
  \[
     \boxed{\,e=0,\qquad
            x^{-1}= \dfrac{-x}{1+x}\;\;(x\neq-1).}
  \]
  
  %---------------------------------------------------------------
  \bigskip
  \textbf{(c)  Subgroup generated by \(\tfrac12\).}
  
  Because \(\varphi\) is a \emph{group isomorphism}
  \(\bigl(G,\*\bigr)\cong\bigl(\Bbb R^{\times},\cdot\bigr)\),
  the subgroup
  \(\langle\tfrac12\rangle_{\*}\) corresponds to the multiplicative
  subgroup generated by
  \[
     \varphi\!\Bigl(\tfrac12\Bigr)=1+\tfrac12=\tfrac32 .
  \]
  Hence
  \[
     \bigl\langle\tfrac12\bigr\rangle_{\*}
        = \Bigl\{\,\bigl(\tfrac32\bigr)^{n}-1\;\Bigm|\;n\in\Bbb Z\Bigr\}.
  \]
  
  \[
     \boxed{\,\langle\tfrac12\rangle_{\*}
        \;=\;\{\,(\frac32)^{\,n}-1 : n\in\Bbb Z\,\}.}
  \]
  Every element is obtained by repeated “\(*\)”–multiplication or
  \(*\)–inversion of \(\tfrac12\).
  \end{solution}
  \begin{problem}
    Let a group \(G\) act on a finite set \(X\) on the left and let  
    \[
       \rho \;:\; G \longrightarrow S_{X}, 
       \qquad 
       \rho(g)(x)=g\!\cdot\!x ,
    \]
    be the associated permutation representation.
    \begin{enumerate}[]
       \item Prove that \(\rho\) is a group homomorphism.
       \item Show that \(\ker\rho\) equals the \emph{kernel of the action}
             (the largest normal subgroup of \(G\) that fixes every element of \(X\)).
    \end{enumerate}
    \end{problem}
    
    \begin{solution}
    \textbf{Notation.}
    Throughout \(g,h\in G\) and \(x\in X\).
    The group action axioms are
    \[
       e\!\cdot\!x = x,
       \qquad
       (gh)\!\cdot\!x = g\!\cdot\!\bigl(h\!\cdot\!x\bigr).
    \]
    
    %-----------------------------------------------------------------
    \bigskip
    \textbf{(a)  \(\rho\) is a homomorphism.}
    
    Compute \(\rho(gh)\) and \(\rho(g)\,\rho(h)\) as permutations of \(X\):
    \[
       \rho(gh)(x) \;=\; (gh)\!\cdot\!x
                     \;=\; g\!\cdot\!\bigl(h\!\cdot\!x\bigr)
                     \;=\; \rho(g)\!\bigl(\rho(h)(x)\bigr)
                     \;=\; \bigl(\rho(g)\,\rho(h)\bigr)(x).
    \]
    Since the two permutations coincide on every \(x\in X\),
    \[
       \boxed{\;\rho(gh)=\rho(g)\,\rho(h)\;} ,
    \]
    so \(\rho\) is indeed a group homomorphism \(G\to S_{X}\).
    
    %-----------------------------------------------------------------
    \bigskip
    \textbf{(b)  \(\ker\rho\) is the kernel of the action.}
    
    \[
    \ker\rho
       =\{\,g\in G \mid \rho(g)=\mathrm{id}_{X}\}
       =\{\,g\in G \mid g\!\cdot\!x = x\;\text{for all }x\in X\}.
    \]
    Thus \(\ker\rho\) consists \emph{precisely} of those elements of \(G\)
    that fix every point of \(X\); by definition, this subgroup is called
    the \emph{kernel of the action}.  Denote it
    \(\mathrm{Fix}_{G}(X)\).  We have shown
    \[
       \boxed{\;\ker\rho \;=\; \mathrm{Fix}_{G}(X)\;} .
    \]
    
    \smallskip
    \emph{Normality and maximality.}
    For any \(g\in G\) and \(k\in\ker\rho\),
    \[
       \bigl(gkg^{-1}\bigr)\!\cdot\!x
          = g\!\cdot\!\bigl(k\!\cdot\!(g^{-1}\!\cdot\!x)\bigr)
          = g\!\cdot\!\bigl(g^{-1}\!\cdot\!x\bigr)
          = x,
    \]
    so \(gkg^{-1}\in\ker\rho\).  Hence \(\ker\rho\lhd G\).  
    Any subgroup of \(G\) that fixes every element of \(X\) is contained in
    \(\ker\rho\), making it the \emph{largest} normal subgroup with this
    property.
    
    \end{solution}
    \begin{solution}
      Write  
      \[
         \Bbb Z_{21}^{\times}
         =\{\,a\in\Bbb Z_{21}\mid\gcd(a,21)=1\,\},
         \qquad
         21=3\cdot7,\;\; \varphi(21)=12 .
      \]
      
      %-----------------------------------------------------------------
      \bigskip
      \textbf{(a)  Elements and their orders.}
      
      Via the Chinese remainder theorem  
      \(\Bbb Z_{21}^{\times}\cong\Bbb Z_{3}^{\times}\times\Bbb Z_{7}^{\times}
              \cong C_{2}\times C_{6}\).
      For every \(a\) list its image \((\overline a_{\,3},\overline a_{\,7})\)
      and take
      \(\operatorname{ord}_{21}(a)=\mathrm{lcm}
            \bigl(\operatorname{ord}_{3}(\overline a_{\,3}),
                  \operatorname{ord}_{7}(\overline a_{\,7})\bigr)\):
      
      \[
      \renewcommand{\arraystretch}{1.25}
      \begin{array}{c|c|c|c}
      a & a\pmod3 & a\pmod7 & \operatorname{ord}_{21}(a)\\\hline
       1 & 1 & 1 & 1\\
       2 & 2 & 2 & 6\\
       4 & 1 & 4 & 3\\
       5 & 2 & 5 & 6\\
       8 & 2 & 1 & 2\\
      10 & 1 & 3 & 6\\
      11 & 2 & 4 & 6\\
      13 & 1 & 6 & 2\\
      16 & 1 & 2 & 3\\
      17 & 2 & 3 & 6\\
      19 & 1 & 5 & 6\\
      20 & 2 & 6 & 2
      \end{array}
      \]
      
      %-----------------------------------------------------------------
      \bigskip
      \textbf{(b)  Is \(\Bbb Z_{21}^{\times}\) cyclic?}
      
      No.  
      The structure \(C_{2}\times C_{6}\) has no element of order
      \(\mathrm{lcm}(2,6)=12\); the largest order is \(6\).  
      Hence the group is \emph{not} cyclic.
      
      %-----------------------------------------------------------------
      \bigskip
      \textbf{(c)  Product \(\displaystyle\prod_{a\in\Bbb Z_{21}^{\times}} a
                  \pmod{21}\).}
      
      In any abelian group, the product of all elements equals the product of
      the self–inverse elements, because every element pairs with its inverse
      and contributes \(1\) to the product.
      
      Self–inverse units modulo \(21\) are those of order \(1\) or \(2\):
      \[
         \{\,1,\,8,\,13,\,20\,\}.
      \]
      \[
         1\cdot8\cdot13\cdot20
           \equiv 8\cdot13\cdot20
           \equiv 20\cdot20
           \equiv 400
           \equiv 1 \pmod{21}.
      \]
      
      \[
         \boxed{\,\displaystyle\prod_{a\in\Bbb Z_{21}^{\times}} a \equiv 1
                \pmod{21}\, }.
      \]
      \end{solution}
      \begin{problem}
        Let \(|G| = 108 = 2^{2}\cdot 3^{3}\).
        \begin{enumerate}[]
           \item Show that \(G\) possesses a \emph{normal} Sylow–\(3\) subgroup.
           \item Explain why \(G\) need not be abelian (give an explicit example).
           \item Prove that \(G\) has at least one subgroup of order \(18\).
        \end{enumerate}
        \end{problem}
        
        \begin{solution}
        %-----------------------------------------------------------------
        \bigskip
        \textbf{(a)  A normal Sylow–\(3\) subgroup.}
        
        Let \(n_{3}\) be the number of Sylow–\(3\) subgroups.
        Sylow’s congruence and divisibility conditions give
        \[
           n_{3}\;\mid\;2^{2}=4,
           \qquad
           n_{3}\equiv 1 \pmod{3}
           \quad\Longrightarrow\quad
           n_{3}\in\{1,4\}.
        \]
        
        \emph{Rule out \(n_{3}=4\).}
        Each Sylow–\(3\) subgroup \(P\) has order \(27\) and is a
        \(3\)-group, hence has non-trivial centre.
        Two distinct Sylow \(3\)-subgroups intersect in a subgroup
        of order at least \(3\).
        Counting the \emph{non-identity} elements contained in four such
        subgroups gives
        \[
           4\,(27-1) - 6\,(3-1) = 4\cdot26-6\cdot2 = 104-12 = 92,
        \]
        already exceeding the total of \(108-1=107\) non-identity elements of
        \(G\).  
        (Here \(6\) is the number of pairwise intersections.)
        Hence \(n_{3}\neq4\); therefore \(n_{3}=1\).
        
        Thus the unique Sylow–\(3\) subgroup \(P\) is normal:
        \[
           \boxed{\,P\trianglelefteq G,\;|P|=27.}
        \]
        
        %-----------------------------------------------------------------
        \bigskip
        \textbf{(b)  \(G\) need not be abelian.}
        
        Take the direct product
        \[
           D_{12}\times\Bbb Z_{9},
        \]
        where \(D_{12}=\langle r,s\mid r^{6}=s^{2}=1,\,srs=r^{-1}\rangle\) has
        order \(12\) and is non-abelian, while \(\Bbb Z_{9}\) is cyclic of order
        \(9\).
        The product has order \(12\cdot9=108\) and is non-abelian because
        \(D_{12}\) already contains non-commuting elements.
        
        \[
           \boxed{\;
              \text{Example: } D_{12}\times\Bbb Z_{9}\ (\text{order }108)
              \text{ is not abelian.}}
        \]
        
        %-----------------------------------------------------------------
        \bigskip
        \textbf{(c)  Existence of a subgroup of order \(18\).}
        
        Let \(Q\le G\) be a Sylow–\(2\) subgroup (\(|Q|=4\)).
        Because \(P\lhd G\), conjugation by any element of \(Q\) induces an
        automorphism of \(P\).  Automorphisms of a \(3\)-group have order a
        power of \(3\); hence an element \(q\in Q\) satisfies \(q^{2}=1\) and
        acts on \(P\) by an automorphism of order dividing \(2\).
        In particular the fixed-point set
        \[
           C_{P}(q)=\{\,x\in P\mid qxq^{-1}=x\,\}
        \]
        is non–trivial.  
        Choose \(x\in C_{P}(q)\) with \(|x|=9\).  
        Then
        \[
           H=\langle x,q\rangle
        \]
        has order \(9\cdot2 = 18\) because \(x\) and \(q\) have coprime orders
        and \(\langle x\rangle\lhd H\).
        
        \[
           \boxed{\,G \text{ contains a subgroup }H\cong C_{9}\rtimes C_{2}
                  \text{ of order }18.}
        \]
        
        \end{solution}
        \begin{problem}
          Let \(E=\Bbb Q\bigl(\sqrt[3]{2},\,\omega\bigr)\) where
          \(\displaystyle \omega=e^{2\pi i/3}=\tfrac{-1+\sqrt{-3}}{2}\) is a
          primitive cube–root of unity.
          \begin{enumerate}[]
             \item Compute the degree \([E:\Bbb Q]\).
             \item Show that \(E/\Bbb Q\) is a Galois extension and identify
                   \(\Gal(E/\Bbb Q)\) up to isomorphism.
          \end{enumerate}
          \end{problem}
          
          \begin{solution}
          %-----------------------------------------------------------------
          \bigskip
          \textbf{(a)  The degree \([\!E:\Bbb Q]\).}
          
          \smallskip
          \emph{Step 1: adjoining \(\sqrt[3]{2}\).}  
          The polynomial \(f(x)=x^{3}-2\) is irreducible over \(\Bbb Q\) by
          Eisenstein (with prime \(2\)), so
          \[
             [\,\Bbb Q(\sqrt[3]{2}):\Bbb Q\,]=3.
          \]
          
          \smallskip
          \emph{Step 2: adjoining \(\omega\).}  
          The minimal polynomial of \(\omega\) over \(\Bbb Q\) is
          \(x^{2}+x+1\), hence
          \(
             [\,\Bbb Q(\omega):\Bbb Q\,]=2.
          \)
          
          \smallskip
          \emph{Step 3: compositum degree.}  
          Since \(\sqrt[3]{2}\notin\Bbb Q(\omega)\)
          (\(\Bbb Q(\omega)\subset\Bbb R\) is false),
          and \(\omega\notin\Bbb Q(\sqrt[3]{2})\)
          (the latter is totally real),
          the two subfields intersect trivially:
          \[
             \Bbb Q(\sqrt[3]{2})\cap\Bbb Q(\omega)=\Bbb Q.
          \]
          Therefore
          \[
             [E:\Bbb Q]
                =[\,\Bbb Q(\sqrt[3]{2},\omega):\Bbb Q\,]
                =[\,\Bbb Q(\sqrt[3]{2}):\Bbb Q\,]\;
                 [\,\Bbb Q(\omega):\Bbb Q\,]
                =3\cdot2
                =6.
          \]
          
          \[
             \boxed{\, [E:\Bbb Q]=6 \,}.
          \]
          
          %-----------------------------------------------------------------
          \bigskip
          \textbf{(b)  Galois and its Galois group.}
          
          \smallskip
          \emph{Splitting field.}  
          The polynomial \(x^{3}-2\) has roots
          \[
             \sqrt[3]{2},\qquad
             \omega\sqrt[3]{2},\qquad
             \omega^{2}\sqrt[3]{2}.
          \]
          All of them lie in \(E\), so \(E\) is the \emph{splitting field} of
          \(x^{3}-2\) over \(\Bbb Q\).
          
          \smallskip
          \emph{Normal + separable \(\Longrightarrow\) Galois.}  
          A splitting field of a separable polynomial (here, \(x^{3}-2\)) is
          normal and separable; hence \(E/\Bbb Q\) is \emph{Galois}.
          
          \smallskip
          \emph{Galois group structure.}  
          Any automorphism of \(E\) must send \(\sqrt[3]{2}\) to one of the three
          roots and must send \(\omega\) to \(\omega\) or \(\omega^{2}\).
          Hence there are at most \(3\cdot2=6\) automorphisms.
          Since \([\!E:\Bbb Q]=6\), all possible maps occur and
          \[
             |\Gal(E/\Bbb Q)|=6.
          \]
          
          Label
          \(\sigma:\sqrt[3]{2}\mapsto\omega\sqrt[3]{2}\) (fixing \(\omega\)) and
          \(\tau:\omega\mapsto\omega^{2}\) (fixing \(\sqrt[3]{2}\)).
          Then
          \(
             \sigma^{3}=1,\;
             \tau^{2}=1,\;
             \tau\sigma\tau^{-1}=\sigma^{-1},
          \)
          the defining relations of the symmetric group \(S_{3}\).
          
          \[
             \boxed{\, \Gal(E/\Bbb Q)\;\cong\;S_{3}. \;}
          \]
          \end{solution}
          \begin{solution}
            Recall the standard notation
            \[
               SL_{2}(\Bbb R)
                  =\Bigl\{\;A\in M_{2}(\Bbb R)\;\Bigm|\;\det A=1\Bigr\},
               \qquad
               PSL_{2}(\Bbb R)
                  =SL_{2}(\Bbb R)\big/\{\pm I\}.
            \]
            
            %---------------------------------------------------------------
            \bigskip
            \textbf{1.\;The subgroup \(\{\pm I\}\) is central in \(SL_{2}(\Bbb R)\).}
            
            Indeed \(AI=\!IA=A\) for all \(A\in SL_{2}(\Bbb R)\) and
            \((-I)\) also commutes with every matrix.  
            Hence \(\{\pm I\}\le Z\bigl(SL_{2}(\Bbb R)\bigr)\), so the quotient
            \(SL_{2}(\Bbb R)/\{\pm I\}\) is well defined.
            
            %---------------------------------------------------------------
            \bigskip
            \textbf{2.\;Canonical epimorphism onto \(PSL_{2}(\Bbb R)\).}
            
            Define
            \[
               \pi : SL_{2}(\Bbb R) \;\longrightarrow\; PSL_{2}(\Bbb R),
               \qquad
               \pi(A)=\overline{A}\;=\;\{\pm A\}.
               \tag{\(*\)}
            \]
            \begin{itemize}
               \item \emph{Homomorphism:} 
                     \(\pi(AB)=\overline{\,AB\,}=\overline{A}\,\overline{B}
                             =\pi(A)\pi(B)\).
               \item \emph{Surjectivity:} 
                     Every coset \(\overline{A}\) has a representative \(A\in SL_{2}(\Bbb R)\).
               \item \emph{Kernel:}
                     \(\ker\pi=\{\,A\in SL_{2}(\Bbb R)\mid\overline{A}=\overline{I}\}
                               =\{\pm I\}.
                     \)
            \end{itemize}
            
            By the First Isomorphism Theorem,
            \[
               SL_{2}(\Bbb R)\big/\{\pm I\}
                  \;\cong\;
               PSL_{2}(\Bbb R).
            \]
            \[
               \boxed{\;SL_{2}(\Bbb R)/\{\pm I\}\;\simeq\;PSL_{2}(\Bbb R).\;}
            \]
            
            %---------------------------------------------------------------
            \bigskip
            \textbf{3.\;Why \(PSL_{2}(\Bbb R)\) is still non-abelian.}
            
            Choose
            \[
               A=\begin{pmatrix}1&1\\0&1\end{pmatrix},
               \qquad
               B=\begin{pmatrix}1&0\\1&1\end{pmatrix},
               \qquad
               A,B\in SL_{2}(\Bbb R).
            \]
            Direct computation shows
            \[
               AB=\begin{pmatrix}2&1\\1&1\end{pmatrix},
               \quad
               BA=\begin{pmatrix}1&1\\1&2\end{pmatrix},
               \quad
               AB\neq BA,
               \quad
               AB\neq -BA.
            \]
            Therefore the cosets \(\overline{A},\overline{B}\in PSL_{2}(\Bbb R)\)
            \emph{still} fail to commute:
            \[
               \overline{A}\,\overline{B}\;=\;\overline{AB}
               \;\neq\;\overline{BA}\;=\;\overline{B}\,\overline{A}.
            \]
            
            Hence \(PSL_{2}(\Bbb R)\) remains non-abelian even after modding out the
            central subgroup \(\{\pm I\}\).
            
            \[
               \boxed{\;PSL_{2}(\Bbb R)\ \text{ is non-abelian.}\;}
            \]
            \end{solution}
            \begin{problem}
              Let  
              \[
                 R \;=\; \Bbb Z_{5}[x]\big/\bigl(x^{2}+2\bigr)
                        \;=\;
                 \Bbb Z_{5}[x]/(x^{2}+2).
              \]
              \begin{enumerate}[]
                 \item Show that \(x^{2}+2\) is irreducible over \(\Bbb Z_{5}\).
                 \item Conclude that \(R\) is a field of order \(25\).
                 \item Find the multiplicative inverse of \(\,x+(x^{2}+2)\) in \(R\).
              \end{enumerate}
              \end{problem}
              
              \begin{solution}
              %-----------------------------------------------------------------
              \bigskip
              \textbf{(a)  Irreducibility of \(x^{2}+2\) over \(\Bbb Z_{5}\).}
              
              A quadratic over a field is reducible iff it has a root.
              Check all residues modulo \(5\):
              \[
                 \begin{aligned}
                 0^{2}+2 &= 2\not\equiv0\pmod5,\\
                 1^{2}+2 &= 3\not\equiv0\pmod5,\\
                 2^{2}+2 &= 6\equiv1\not\equiv0\pmod5,\\
                 3^{2}+2 &= 11\equiv1\not\equiv0\pmod5,\\
                 4^{2}+2 &= 18\equiv3\not\equiv0\pmod5.
                 \end{aligned}
              \]
              No root exists, hence \(x^{2}+2\) is \emph{irreducible} in
              \(\Bbb Z_{5}[x]\).
              
              %-----------------------------------------------------------------
              \bigskip
              \textbf{(b)  \(R\) is a field of order \(25\).}
              
              Because \(x^{2}+2\) is irreducible, the ideal \((x^{2}+2)\) is maximal,
              so the quotient \(R\) is a field.  
              Every element of \(R\) can be represented by a degree–\(\le1\) polynomial
              \(a+bx\) with \(a,b\in\Bbb Z_{5}\).
              There are \(5^{2}=25\) such pairs, so
              \[
                 |R|=25.
              \]
              
              %-----------------------------------------------------------------
              \bigskip
              \textbf{(c)  Inverse of \(x\) in \(R\).}
              
              In \(R\) we have the defining relation
              \[
                 x^{2} = -2 \;\equiv\; 3 \pmod5.
              \]
              
              Seek \(u=a+bx\in R\) with \(x\*u = 1\):
              \[
                 x(a+bx)=ax + b x^{2} = a x + 3b.
              \]
              To equal \(1\) (which has no \(x\) term) we require
              \[
                 a = 0,
                 \qquad
                 3b \equiv 1 \pmod5
                 \;\Longrightarrow\;
                 b \equiv 2 \pmod5.
              \]
              
              Hence
              \[
                 u = 2x \quad\text{and}\quad
                 x\cdot 2x = 2x^{2} = 2\cdot 3 = 6 \equiv 1\pmod5.
              \]
              
              \[
                 \boxed{\;
                   \bigl(x+(x^{2}+2)\bigr)^{-1}
                     = 2x +(x^{2}+2)\;\in R
                 \;}
              \]
              \end{solution}
              \begin{solution}
                Write $\Bbb Z_{12}=\langle\bar 1\rangle=\{\bar 0,\bar1,\dots,\bar{11}\}$,
                where bars denote residue classes modulo $12$.
                
                %---------------------------------------------------------------
                \bigskip
                \textbf{(a)  The automorphism group $\Aut(\Bbb Z_{12})$.}
                
                Because $\Bbb Z_{12}$ is cyclic, every automorphism is determined by the
                image of its generator $\bar1$.  
                An element $\bar k$ generates the whole group iff
                $\gcd(k,12)=1$, i.e.\ $k\in\{1,5,7,11\}$.
                
                \[
                \Aut(\Bbb Z_{12})
                   =\bigl\{\;\varphi_{k} \mid k\in\{1,5,7,11\}\bigr\},
                   \qquad
                   \varphi_{k}(\bar m)=\overline{km}\pmod{12}.
                \]
                
                Composition corresponds to multiplication of the indices:
                \[
                   \varphi_{k}\circ\varphi_{\ell}=\varphi_{k\ell},
                   \qquad k,\ell\in\{1,5,7,11\}.
                \]
                
                \[
                \renewcommand{\arraystretch}{1.15}
                \begin{array}{c|cccc}
                \circ & \varphi_{1} & \varphi_{5} & \varphi_{7} & \varphi_{11}\\\hline
                \varphi_{1} & \varphi_{1} & \varphi_{5} & \varphi_{7} & \varphi_{11}\\
                \varphi_{5} & \varphi_{5} & \varphi_{1} & \varphi_{11} & \varphi_{7}\\
                \varphi_{7} & \varphi_{7} & \varphi_{11} & \varphi_{1} & \varphi_{5}\\
                \varphi_{11}& \varphi_{11}& \varphi_{7} & \varphi_{5} & \varphi_{1}
                \end{array}
                \]
                
                %---------------------------------------------------------------
                \bigskip
                \textbf{(b)  Isomorphism $\Aut(\Bbb Z_{12})\cong\Bbb Z_{2}\oplus\Bbb Z_{2}$.}
                
                \smallskip
                \emph{(i) Via Euler units.}\;
                The map
                \(
                   \Phi:\Aut(\Bbb Z_{12})\longrightarrow (\Bbb Z/12)^{\times},
                   \;
                   \varphi_{k}\mapsto\bar k,
                \)
                is an isomorphism of groups.  
                Since
                \[
                   (\Bbb Z/12)^{\times}=\{\bar1,\bar5,\bar7,\bar{11}\},
                   \quad
                   \bar5^{2}=\bar7^{2}=\bar{11}^{2}=\bar1,
                \]
                every non-identity element has order $2$.  
                A group of order $4$ in which every non-identity element has order $2$
                is the Klein four group.
                
                \[
                   \boxed{\;
                      \Aut(\Bbb Z_{12})\;\cong\;(\Bbb Z/12)^{\times}
                      \;\cong\;\Bbb Z_{2}\;\oplus\;\Bbb Z_{2}.
                   \;}
                \]
                
                \smallskip
                \emph{(ii) Explicit splitting via the C.R.T.}\;
                The Chinese remainder theorem gives
                \[
                   (\Bbb Z/12)^{\times}
                      \;\cong\;
                   (\Bbb Z/4)^{\times}\times(\Bbb Z/3)^{\times}
                      =\{\bar1,\bar3\}\times\{\bar1,\bar2\}
                      \;\cong\;\Bbb Z_{2}\times\Bbb Z_{2}.
                \]
                Under this correspondence
                \[
                   \bar1\longleftrightarrow(1,1),\;
                   \bar5\longleftrightarrow(1,2),\;
                   \bar7\longleftrightarrow(3,1),\;
                   \bar{11}\longleftrightarrow(3,2).
                \]
                Thus
                \(
                   \varphi_{5},\varphi_{7}
                \)
                form an elementary‐abelian generating set of rank $2$, confirming the
                isomorphism.
                
                \end{solution}
                \begin{problem}
                  Let \(H\le S_{4}\) be the subgroup generated by the transposition
                  \((12)\), i.e.\ \(H=\{\,e,(12)\,\}\).
                  \begin{enumerate}[]
                     \item Compute the centraliser \(C_{S_{4}}(H)\).
                     \item Compute the normaliser \(N_{S_{4}}(H)\).
                     \item Find the index \(;N_{S_{4}}(H):C_{S_{4}}(H)\); and interpret the
                           result via group actions.
                  \end{enumerate}
                  \end{problem}
                  
                  \begin{solution}
                  Throughout write \(S_{4}=\langle(12),(13),(14),(23),(24),(34)\rangle\).
                  
                  %---------------------------------------------------------------
                  \bigskip
                  \textbf{(a)  The centraliser \(C_{S_{4}}(H)\).}
                  
                  An element \(\sigma\in S_{4}\) centralises \(H\) iff
                  \(\sigma(12)\sigma^{-1}=(12)\).
                  Conjugation sends a transposition to the transposition of the displaced
                  points, so \(\sigma(12)\sigma^{-1}=(\sigma(1)\,\sigma(2))\).
                  Hence \(\sigma\) must send the \emph{unordered} pair \(\{1,2\}\) to
                  itself, i.e.
                  \[
                     \sigma\{1,2\}=\{1,2\}.
                  \]
                  Thus \(\sigma\) may
                  \begin{itemize}
                     \item fix \(1\) and \(2\) or swap them, and
                     \item independently fix or swap \(3\) and \(4\).
                  \end{itemize}
                  Consequently
                  \[
                     C_{S_{4}}(H)
                        =\{\,e,\;(12),\;(34),\;(12)(34)\,\}
                        \;\cong\;C_{2}\times C_{2},
                        \qquad |C_{S_{4}}(H)|=4.
                  \]
                  
                  %---------------------------------------------------------------
                  \bigskip
                  \textbf{(b)  The normaliser \(N_{S_{4}}(H)\).}
                  
                  The normaliser consists of all \(\sigma\) such that
                  \(\sigma H\sigma^{-1}=H\),
                  equivalently
                  \(\sigma(12)\sigma^{-1}\in H\).
                  Since \(H\) contains only \(e\) and \((12)\),
                  this again forces \(\sigma\{1,2\}=\{1,2\}\).
                  There is no further restriction on the images of \(3\) and \(4\).
                  
                  Hence \(N_{S_{4}}(H)\) is the same set described in~(a):
                  \[
                     N_{S_{4}}(H)=C_{S_{4}}(H),
                     \qquad |N_{S_{4}}(H)|=4.
                  \]
                  
                  %---------------------------------------------------------------
                  \bigskip
                  \textbf{(c)  The index \(\bigl[N_{S_{4}}(H):C_{S_{4}}(H)\bigr]\).}
                  
                  Because the two subgroups coincide,
                  \[
                     \bigl[N_{S_{4}}(H):C_{S_{4}}(H)\bigr]=1.
                  \]
                  
                  \emph{Interpretation.}  
                  For any subgroup \(H\le G\) there is a natural homomorphism
                  \[
                     \theta:N_{G}(H)\;\longrightarrow\;\Aut(H),
                     \qquad
                     \theta(\sigma)\;:\;h\longmapsto\sigma h\sigma^{-1}.
                  \]
                  Its kernel is \(C_{G}(H)\), so
                  \(
                     N_{G}(H)/C_{G}(H)\;\cong\;\operatorname{im}\theta
                     \subseteq\Aut(H).
                  \)
                  Here \(H\cong C_{2}\) has trivial automorphism group
                  (\(\Aut(C_{2})=\{1\}\)),
                  so \(\theta\) is the \emph{zero} map and \(N_{S_{4}}(H)=C_{S_{4}}(H)\).
                  
                  Equivalently, the stabiliser of the transposition \((12)\) under
                  conjugation is already its centraliser; conjugation induces \emph{no}
                  non-trivial automorphisms of \(H\).
                  
                  \end{solution}
                  \begin{solution}
                    \[
                       \text{Solve }\;
                       \begin{cases}
                          x\equiv 3 \pmod 8,\\
                          x\equiv 4 \pmod 9,\\
                          x\equiv 5 \pmod 7,
                       \end{cases}
                       \qquad
                       \text{and compute }\operatorname{ord}_{\Bbb Z_{504}^{\times}}(x).
                    \]
                    
                    %-----------------------------------------------------------------
                    \bigskip
                    \textbf{1.\;Chinese remainder theorem.}
                    
                    Because the moduli \(8,9,7\) are pairwise coprime,
                    there is a unique solution modulo their product \(8\cdot9\cdot7=504\).
                    
                    \[
                    \begin{aligned}
                    x &\equiv 3 \pmod 8
                         &&\Longrightarrow\;x=3+8k,\\[2pt]
                    x &\equiv 4 \pmod 9
                         &&\Longrightarrow\;3+8k\equiv4\pmod 9
                                          \;\Longrightarrow\;8k\equiv1\pmod 9
                                          \;\Longrightarrow\;k\equiv8\pmod 9;\\[2pt]
                    &\hspace{7.5cm}
                    k=8+9\ell, \quad x=3+8(8+9\ell)=67+72\ell,\\[4pt]
                    x &\equiv 5 \pmod 7
                         &&\Longrightarrow\;67+72\ell\equiv5\pmod 7
                                          \;\Longrightarrow\;72\ell\equiv5-4\pmod 7\\
                    &&&\Longrightarrow\;72\ell\equiv1\pmod 7
                                          \;\Longrightarrow\;2\ell\equiv1\pmod 7
                                          \;\Longrightarrow\;\ell\equiv4\pmod 7.
                    \end{aligned}
                    \]
                    
                    Hence \(\ell=4+7m\) and
                    \[
                       x = 67+72(4+7m)=355+504m\qquad(m\in\Bbb Z).
                    \]
                    The smallest positive solution is  
                    
                    \[
                       \boxed{x\equiv355\pmod{504}}.
                    \]
                    
                    %-----------------------------------------------------------------
                    \bigskip
                    \textbf{2.\;Multiplicative order of \(x\) modulo \(504\).}
                    
                    Since \(\gcd(355,504)=1\), \(x=355\) lies in the unit group
                    \(\Bbb Z_{504}^{\times}\).
                    Euler’s totient gives
                    \(\varphi(504)=504\bigl(1-\tfrac12\bigr)\bigl(1-\tfrac13\bigr)
                                  \bigl(1-\tfrac17\bigr)=144\);
                    thus \(\operatorname{ord}(x)\mid144\).
                    
                    \[
                       x^{2}\equiv 25\pmod{504}\neq1,\qquad
                       x^{3}\equiv307\pmod{504}\neq1,\qquad
                       x^{6}\equiv1\pmod{504}.
                    \]
                    Since no smaller divisor of \(6\) works, the multiplicative order is
                    
                    \[
                       \boxed{\operatorname{ord}_{\Bbb Z_{504}^{\times}}(355)=6}.
                    \]
                    
                    \end{solution}
                    \begin{solution}
                      Recall the dihedral group of order \(10\)
                      \[
                         D_{10}\;=\;
                         \bigl\langle\,r,s \;\bigm|\; r^{5}=s^{2}=1,\; srs=r^{-1}\bigr\rangle,
                      \qquad
                         |D_{10}|=10.
                      \]
                      
                      %-----------------------------------------------------------------
                      \bigskip
                      \textbf{(a)  Conjugacy classes in \(D_{10}\).}
                      
                      \[
                      \renewcommand{\arraystretch}{1.2}
                      \begin{array}{c|l|l}
                      \text{Class} & \text{Representatives} & \text{Reasoning}\\\hline
                      C_{1} & \{1\} &
                              \text{identity} \\[2pt]
                      C_{2} & \{\,r,\;r^{4}\,\} &
                              r^{k}\sim r^{-k}\text{ under }srs=r^{-1} \\[2pt]
                      C_{3} & \{\,r^{2},\;r^{3}\,\} &
                              likewise, r^{2}\sim r^{-2}=r^{3} \\[2pt]
                      C_{4} & \{\,s,\;sr,\;sr^{2},\;sr^{3},\;sr^{4}\,\} &
                              n=5\text{ odd }\Longrightarrow\text{ all reflections conjugate}
                      \end{array}
                      \]
                      
                      Hence the full list of conjugacy classes is
                      \[
                         \boxed{
                           \{1\},\;
                           \{r,r^{4}\},\;
                           \{r^{2},r^{3}\},\;
                           \{s,sr,sr^{2},sr^{3},sr^{4}\}.
                         }
                      \]
                      
                      %-----------------------------------------------------------------
                      \bigskip
                      \textbf{(b)  The class equation.}
                      
                      Add the sizes of the classes:
                      \[
                         |C_{1}| + |C_{2}| + |C_{3}| + |C_{4}|
                         \;=\; 1 + 2 + 2 + 5
                         \;=\; 10
                         \;=\; |D_{10}|.
                      \]
                      Thus the class equation
                      \(
                         |D_{10}| = \sum_{C\;\text{class}} |C|
                      \)
                      is verified for \(D_{10}\).
                      
                      \end{solution}
                      \begin{solution}
                        Let $R$ be a commutative ring with $1\!\neq\!0$ and define the (set of) **nilpotent elements**
                        \[
                           \mathfrak N
                              \;=\;
                           \bigl\{\,x\in R \;\bigm|\; \exists\,k\ge1\text{ with }x^{k}=0\bigr\}.
                        \]
                        We prove that $\mathfrak N$ is an \emph{ideal} of $R$.
                        
                        \bigskip
                        \textbf{1.\;Closed under addition.}
                        
                        Take $x,y\in\mathfrak N$ with $x^{m}=0$ and $y^{n}=0$ for some
                        $m,n\ge1$.  
                        Consider the binomial expansion (commutativity is used!)
                        \[
                           (x+y)^{m+n}
                             \;=\;
                           \sum_{j=0}^{m+n}
                              \binom{m+n}{j}\,
                              x^{\,j}y^{\,m+n-j}.
                        \]
                        In each summand either $j\ge m$ or $m+n-j\ge n$, so
                        $x^{\,j}=0$ or $y^{\,m+n-j}=0$ respectively; hence every term vanishes
                        and
                        \[
                           (x+y)^{\,m+n}=0
                           \quad\Longrightarrow\quad
                           x+y\in\mathfrak N .
                        \]
                        
                        \bigskip
                        \textbf{2.\;Closed under additive inverses.}
                        
                        If $x^{k}=0$ then $(-x)^{k}=(-1)^{k}x^{k}=0$, so $-x\in\mathfrak N$.
                        
                        \bigskip
                        \textbf{3.\;Absorbs multiplication by arbitrary elements.}
                        
                        Let $r\in R$ and $x\in\mathfrak N$ with $x^{k}=0$.
                        Then
                        \[
                           (rx)^{k}=r^{k}x^{k}=r^{k}\!\cdot0=0,
                        \]
                        so $rx\in\mathfrak N$.
                        
                        \bigskip
                        \textbf{Conclusion.}
                        The set $\mathfrak N$ is closed under addition, contains additive
                        inverses, and is closed under multiplication by any element of $R$;
                        therefore
                        \[
                           \boxed{\;\mathfrak N\text{ is an ideal of }R\;}
                        \]
                        — the \emph{nilradical} of the ring.
                        \end{solution}
                        %--------------------------------------------------
%  Compact T/F table (no tabularx)
%--------------------------------------------------
\newcolumntype{L}[1]{>{\raggedright\arraybackslash}p{#1}} % left-aligned paragraph col.

\begin{table}[h]
\small                       % slightly smaller font
\setlength{\tabcolsep}{4pt}  % tighter column padding
\renewcommand{\arraystretch}{1.15}

\begin{tabular}{|L{5.7cm}|c|L{7.3cm}|}
\hline
\textbf{Statement} & \textbf{T/F} & \textbf{Brief justification} \\ \hline
Every finite subgroup of $\Bbb C^\times$ is cyclic. &
\textbf{T} &
Finite subgroups are the $n$-th roots of unity; these are generated by a primitive root $\zeta_n=e^{2\pi i/n}$. \\ \hline

If $G$ is cyclic, every subgroup of $G$ is characteristic. &
\textbf{T} &
A cyclic group has a \emph{unique} subgroup of each divisor of $|G|$. Automorphisms preserve element orders, so they must fix each subgroup. \\ \hline

$\Bbb Z_6$ is an integral domain. &
\textbf{F} &
Zero divisors exist: $\bar 2\cdot\bar 3=\bar 0$ while $\bar 2,\bar 3\neq\bar 0$. \\ \hline

There is a non-trivial group in which every element is conjugate to every other. &
\textbf{F} &
The identity is always central, so it forms its own conjugacy class; hence no non-trivial group can have only one class. \\ \hline

Any two groups of order $49$ are isomorphic. &
\textbf{F} &
Order $49=7^{2}$: there are two abelian groups, $\Bbb Z_{49}$ and $\Bbb Z_{7}\oplus\Bbb Z_{7}$, which are not isomorphic. \\ \hline
\end{tabular}
\end{table}
\end{document}
