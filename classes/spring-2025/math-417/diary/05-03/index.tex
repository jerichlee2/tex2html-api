\documentclass[12pt]{article}

% Packages
\usepackage[margin=1in]{geometry}
\usepackage{amsmath,amssymb,amsthm}
\usepackage{enumitem}
\usepackage{hyperref}
\usepackage{xcolor}
\usepackage{import}
\usepackage{xifthen}
\usepackage{pdfpages}
\usepackage{transparent}
\usepackage{listings}
\usepackage{tikz}
\usepackage{tikz-cd}
\usepackage{physics}
\usepackage{siunitx}
\usepackage{cancel}
  \usetikzlibrary{calc,patterns,arrows.meta,decorations.markings}


\DeclareMathOperator{\Log}{Log}
\DeclareMathOperator{\Arg}{Arg}

\lstset{
    breaklines=true,         % Enable line wrapping
    breakatwhitespace=false, % Wrap lines even if there's no whitespace
    basicstyle=\ttfamily,    % Use monospaced font
    frame=single,            % Add a frame around the code
    columns=fullflexible,    % Better handling of variable-width fonts
}

\newcommand{\incfig}[1]{%
    \def\svgwidth{\columnwidth}
    \import{./Figures/}{#1.pdf_tex}
}
\theoremstyle{definition} % This style uses normal (non-italicized) text
\newtheorem{solution}{Solution}
\newtheorem{proposition}{Proposition}
\newtheorem{problem}{Problem}
\newtheorem{lemma}{Lemma}
\newtheorem{theorem}{Theorem}
\newtheorem{remark}{Remark}
\newtheorem{note}{Note}
\newtheorem{definition}{Definition}
\newtheorem{example}{Example}
\newtheorem{examples}{Examples}
\newtheorem{nonexamples}{Non-Examples}
\newtheorem{remarks}{Remarks}
\newtheorem{corollary}{Corollary}
\theoremstyle{plain} % Restore the default style for other theorem environments
%

% Theorem-like environments
% Title information
\title{}
\author{Jerich Lee}
\date{\today}

\begin{document}

\maketitle
\begin{definition}[Integral Domain]
  A \textbf{commutative ring} \(R\) with identity \(1\neq 0\) is called an
  \emph{integral domain} if it contains \textbf{no zero‐divisors}; that is,
  for all \(a,b\in R\),
  \[
     ab = 0 \quad\Longrightarrow\quad a = 0 \ \text{or}\ b = 0.
  \]
  Equivalently, the cancellation law holds for non–zero elements:
  \[
     a\neq 0,\; ab = ac \;\Longrightarrow\; b = c
     \quad\text{for all }a,b,c\in R.
  \]
  \end{definition}
  
  \begin{examples}
     \begin{enumerate}
        \item \(\mathbb{Z}\) (the integers) and, more generally, every
              subring of a field, e.g.\ \(\mathbb{Z}[i]\subset \mathbb{C}\).
        \item Polynomial rings over integral domains, e.g.\
              \(\mathbb{Z}[X]\) or \(k[X_1,\dots,X_n]\) for a field \(k\).
        \item Any field \(F\) itself is an integral domain
              (zero divisors cannot exist because every non–zero element is
              invertible).
     \end{enumerate}
  \end{examples}
  
  \begin{nonexamples}
     \begin{enumerate}
        \item The ring \(\mathbb{Z}/6\mathbb{Z}\):
              \(2\cdot 3 = 0\) in this ring, yet \(2\neq 0\) and \(3\neq 0\).
        \item The product ring \(\mathbb{Z}\times\mathbb{Z}\):
              \((1,0)\cdot(0,1) = (0,0)\).
     \end{enumerate}
  \end{nonexamples}
  
  \begin{remarks}
     \begin{itemize}
        \item If \(R\) is an integral domain, its fraction field
              \(\operatorname{Frac}(R)\) (constructed by formally adjoining
              denominators) exists and embeds \(R\) into a field.
        \item Many classical results—unique factorisation, Euclidean
              algorithm, Gauss’s lemma—are developed first inside integral
              domains.
     \end{itemize}
  \end{remarks}
  \begin{corollary}[Rational Root Test]\label{cor:rational_root_test}
    Let
    \[
       f(X)=a_{0}+a_{1}X+\dots+a_{n}X^{n}\in\mathbb{Z}[X],
       \qquad a_{n}\neq 0.
    \]
    Assume \(f\) has a \emph{rational} zero \(\dfrac{\alpha}{\beta}\) written
    in lowest terms (\(\gcd(\alpha,\beta)=1,\ \beta>0\)).  
    Then
    \[
       \beta\mid a_{n}
       \quad\text{and}\quad
       \alpha\neq 0\;\Longrightarrow\; \alpha\mid a_{0}.
    \]
    Consequently, if the polynomial is \emph{monic} (\(a_{n}=1\)),
    every rational zero is in fact an \emph{integer}.
    \end{corollary}
    
    \begin{proof}
    \textbf{Step 1.\;Divisibility in \(\mathbb{Q}[X]\).}
    Since \(f\!\bigl(\tfrac{\alpha}{\beta}\bigr)=0\), the linear polynomial
    \[
       X-\frac{\alpha}{\beta}
    \]
    divides \(f\) in the polynomial ring \(\mathbb{Q}[X]\); that is,
    \(\exists\,g\in\mathbb{Q}[X]\) with
    \[
       f=(X-\tfrac{\alpha}{\beta})\,g.
    \]
    
    \smallskip
    \textbf{Step 2.\;Clearing denominators.}
    Multiply the factor \(X-\tfrac{\alpha}{\beta}\) by \(\beta\) to obtain
    the \emph{primitive} (coefficients have gcd \(1\)) polynomial
    \[
       \beta X-\alpha\in\mathbb{Z}[X].
    \]
    Because \(\beta X-\alpha\) is just a unit
    (\(\beta^{-1}\in\mathbb{Q}^{\!*}\)) multiple of
    \(X-\tfrac{\alpha}{\beta}\), we still have a divisibility in
    \(\mathbb{Q}[X]\):
    \[
       (\beta X-\alpha)\mid f \quad\text{in }\mathbb{Q}[X].
    \]
    
    \smallskip
    \textbf{Step 3.\;Gauss’s lemma.}
    The polynomial \(\beta X-\alpha\) is \emph{primitive}.
    Gauss’s lemma (or the theorem proved earlier) states that if a primitive
    polynomial divides \(f\) in \(\mathbb{Q}[X]\), then it already divides
    \(f\) in \(\mathbb{Z}[X]\).
    Hence there exists \(h\in\mathbb{Z}[X]\) such that
    \[
       f=(\beta X-\alpha)\,h.
    \]
    
    \smallskip
    \textbf{Step 4.\;Leading and constant coefficients.}
    Write
    \[
       h(X)=b_{0}+b_{1}X+\dots+b_{n-1}X^{\,n-1}\qquad(b_{n-1}\neq 0).
    \]
    Comparing coefficients in
    \(f(X)=(\beta X-\alpha)h(X)\) gives:
    
    \begin{itemize}
       \item \emph{Leading term: }
             \(\displaystyle
             a_{n}= \beta\cdot b_{\,n-1}\;\Longrightarrow\;
             \beta\mid a_{n}.
             \)
       \item \emph{Constant term: }
             \(\displaystyle
             a_{0}=(-\alpha)\cdot b_{0}
             \;\Longrightarrow\;
             \alpha\mid a_{0}\quad(\text{provided }\alpha\neq 0).
             \)
    \end{itemize}
    
    \smallskip
    \textbf{Step 5.\;The monic case.}
    If \(a_{n}=1\) (i.e.\ \(f\) is monic) then
    \(\beta\mid 1\mathbb{R}ightarrow\beta=1\).
    Hence the only possible rational zeros are of the form
    \(\tfrac{\alpha}{1}=\alpha\in\mathbb{Z}\).
    
    \end{proof}
    \begin{theorem}[Eisenstein’s Criterion]
      Let 
      \[
         f(x)=a_{0}+a_{1}x+a_{2}x^{2}+\dots+a_{n}x^{n}\in\mathbb{Z}[x]\setminus\{0\},
         \qquad n\ge 1.
      \]
      Suppose there exists a prime number \(p\in\mathbb{N}\) such that 
      \begin{enumerate}
         \item[(i)] \(p\nmid a_{n}\);  \hfill           % 1st hypothesis
         \item[(ii)] \(p \mid a_{i}\) for every \(0\le i<n\); \hfill % 2nd hypothesis
         \item[(iii)] \(p^{2}\nmid a_{0}\).                       % 3rd hypothesis
      \end{enumerate}
      Then \(f\) is irreducible over \(\mathbb{Q}\).
      \end{theorem}
      
      \begin{proof}[Step--by--step proof]
      We argue by contradiction.
      
      \smallskip
      \textbf{1.\;Reduction to \(\mathbb{Z}[x]\).}  
      If \(f\) were reducible over \(\mathbb{Q}\) it would factor as  
      \(f=gh\) with \(g,h\in\mathbb{Q}[x]\) and \(\deg g,\deg h>0\).  
      By clearing denominators and removing any unit factor, Gauss’s lemma
      yields a factorisation \(f=\tilde g\,\tilde h\) with  
      \(\tilde g,\tilde h\in\mathbb{Z}[x]\) and  
      \(\deg\tilde g,\deg\tilde h>0\).  
      Thus it suffices to show that no such decomposition exists inside
      \(\mathbb{Z}[x]\).  
      For simplicity rename the integral factors again \(g,h\).
      
      \smallskip
      \textbf{2.\;Coefficient notation.}  
      Write  
      \[
         g(x)=b_{0}+b_{1}x+\dots+b_{r}x^{r},
         \qquad
         h(x)=c_{0}+c_{1}x+\dots+c_{s}x^{s},
         \qquad r+s=n .
      \]
      
      \smallskip
      \textbf{3.\;First and leading coefficients.}  
      Because \(f=gh\) we have
      \[
         a_{0}=b_{0}c_{0},
         \qquad
         a_{n}=b_{r}c_{s}.
      \]
      From (ii) and (iii) we know \(p\mid a_{0}\) but \(p^{2}\nmid a_{0}\);
      thus exactly one of \(b_{0},c_{0}\) is divisible by \(p\).
      WLOG assume
      \[
         p\mid b_{0}
         \quad\text{and}\quad
         p\nmid c_{0}.
      \]
      Condition (i) says \(p\nmid a_{n}\), so \(p\nmid b_{r}\) and \(p\nmid c_{s}\).
      
      \smallskip
      \textbf{4.\;At least one non--initial coefficient of \(g\) is not
      divisible by \(p\).}  
      Because \(p\mid b_{0}\) but \(p\nmid b_{r}\), there exists an index
      \(0\le j\le r\) for which \(p\nmid b_{j}\).
      Let
      \[
         i:=\min\{\,j\ge 0 : p\nmid b_{j}\,\}.
      \]
      By construction \(1\le i\le r < n\) and
      \[
         p\mid b_{0},\dots,b_{i-1},
         \qquad
         p\nmid b_{i}.
      \]
      
      \smallskip
      \textbf{5.\;The key coefficient relation.}  
      Compare the coefficient of \(x^{i}\) in the product \(f=gh\):
      \[
         a_{i}=b_{i}c_{0}+b_{i-1}c_{1}+b_{i-2}c_{2}+\dots+b_{0}c_{i}.
      \]
      
      \smallskip
      \textbf{6.\;Divisibility analysis.}  
      
      \begin{itemize}
      \item By choice of \(i\), we have \(p\nmid b_{i}\).
      \item From Step~2 we have \(p\nmid c_{0}\).
      \item For every \(j<i\) we have \(p\mid b_{j}\) (minimality of \(i\)).
      \end{itemize}
      
      Hence in the expression for \(a_{i}\)
      
      \[
         b_{i}c_{0}\;\;\text{is \emph{not} divisible by }p,
         \qquad
         b_{k}c_{i-k}\;\;\text{is divisible by }p\;\;\text{for all }k<i.
      \]
      
      Therefore \(a_{i}\not\equiv 0\pmod p\).
      
      \smallskip
      \textbf{7.\;Contradiction.}  
      But hypothesis (ii) says \(p\mid a_{i}\) for every \(i<n\),
      in particular \(p\mid a_{i}\).
      This contradicts the conclusion of Step~6 that \(p\nmid a_{i}\).
      
      \smallskip
      \textbf{8.\;Conclusion.}  
      The assumption that \(f\) factors non-trivially in \(\mathbb{Q}[x]\)
      leads to a contradiction.  
      Hence \(f\) is irreducible over \(\mathbb{Q}\).
      \end{proof}
      \paragraph{Definition (Monic Polynomial).}
Let $R$ be a commutative ring (e.g.\ $\mathbb{Z},\ \mathbb{Q},\ \mathbb{R},\ \mathbb{C}$) and let  

\[
   f(x)\;=\;a_{n}x^{\,n}+a_{n-1}x^{\,n-1}+\cdots+a_{1}x+a_{0}
   \;\in\;R[x], 
   \qquad a_{n}\neq 0 .
\]

The polynomial $f$ is called \emph{monic} if its \emph{leading coefficient} $a_{n}$ equals $1$:
\[
   f \text{ is monic } \;\Longleftrightarrow\; a_{n}=1 .
\]

\medskip
\textbf{Examples}
\[
   x^{3}-2x+5,\quad
   x^{4}+7x^{2}-x+1,\quad
   x-1
\]
are monic over $\mathbb{Z}$ (and hence over any larger coefficient ring), because the coefficient in front of the highest power of $x$ is $1$.

\medskip
\textbf{Non–example}
\[
   3x^{2}+x+4
\]
is \emph{not} monic, since its leading coefficient is $3\neq 1$.

\medskip
\textbf{Why the term “monic”?}  
The prefix \textit{mono–} (meaning “single” or “one”) highlights that the highest–degree term has a single (unit) coefficient equal to $1$.
\begin{proposition}
  Let $E/F$ be a field extension and let $\alpha\in E$ be \emph{algebraic} over $F$.
  Define
  \[
       F[\alpha]\;:=\;\bigl\{\,f(\alpha)\;:\;f\in F[x]\,\bigr\}\;\subseteq\;E.
  \]
  Then
  \[
       F[\alpha]\,=\,F(\alpha),
  \]
  i.e.\ the \textbf{ring} generated by $\alpha$ is already a \textbf{field}.
  Moreover
  \[
       [\,F(\alpha):F\,]=\deg\!\bigl(m_{\alpha,F}\bigr),
  \]
  where $m_{\alpha,F}(x)$ denotes the minimal polynomial of $\alpha$ over $F$.
  \end{proposition}
  
  \begin{proof}[Step--by--step]
  \textbf{1.\;Finite--dimensional $F$–vector space structure.}
  Because $\alpha$ is algebraic, let
  \[
       m_{\alpha,F}(x)=x^{n}+b_{n-1}x^{\,n-1}+\dots+b_{1}x+b_{0}\in F[x],
       \qquad b_{n}=1,
  \]
  be its monic minimal polynomial ($n:=\deg m_{\alpha,F}$).  
  Since $m_{\alpha,F}(\alpha)=0$ we may always reduce any power
  $\alpha^{k}$ with $k\ge n$ to a $F$–linear combination of
  $1,\alpha,\dots,\alpha^{\,n-1}$.  
  Hence every $f(\alpha)\in F[\alpha]$ is an $F$–linear combination of the
  set
  \[
       \mathcal{B}:=\{\,1,\alpha,\dots,\alpha^{\,n-1}\}.
  \]
  Thus $\mathrm{Span}_{F}(\mathcal{B})=F[\alpha]$ and
  $\dim_{F}F[\alpha]\le n<\infty$.
  
  \medskip
  \textbf{2.\;Linear independence of $\mathcal{B}$.}
  Assume a dependence relation
  \[
       a_{0}+a_{1}\alpha+\dots+a_{n-1}\alpha^{\,n-1}=0,
       \qquad a_{i}\in F.
  \]
  Then $g(x):=a_{0}+a_{1}x+\dots+a_{n-1}x^{\,n-1}\in F[x]$
  satisfies $g(\alpha)=0$.
  Minimality of $m_{\alpha,F}$ forces $g=0$, hence all $a_{i}=0$.
  Therefore $\mathcal{B}$ is an $F$–basis and
  \[
       \dim_{F}F[\alpha]=n.
  \]
  
  \medskip
  \textbf{3.\;Showing $F[\alpha]$ is a field.}
  Let $\beta=f(\alpha)$ with $f\in F[x]$ and $\beta\neq0$.
  Because the powers $1,\beta,\dots,\beta^{n}$ live in the
  $n$–dimensional $F$–space $F[\alpha]$, they are $F$–linearly dependent:
  there exist $c_{0},\dots,c_{n}\in F$, \emph{not all zero}, with
  \[
       c_{0}+c_{1}\beta+\dots+c_{n}\beta^{\,n}=0.
  \]
  Choose such a relation with \emph{minimal} $k:=\max\{i:c_{i}\neq0\}$; then $c_{k}\neq0$ and we may divide
  by it to obtain
  \[
       1+\tilde b_{1}\beta+\dots+\tilde b_{k}\beta^{\,k}=0
       \quad\Longrightarrow\quad
       1=\beta\bigl(-\tilde b_{1}-\tilde b_{2}\beta-\dots-\tilde b_{k}\beta^{\,k-1}\bigr).
  \]
  The right--hand factor lies in $F[\alpha]$, so $\beta^{-1}\in F[\alpha]$.
  Every non–zero element therefore has a multiplicative inverse:
  $F[\alpha]$ is a field.
  
  \medskip
  \textbf{4.\;Equality $F[\alpha]=F(\alpha)$.}
  By definition $F[\alpha]\subseteq F(\alpha)$ and $F(\alpha)$ is a field.
  Step~3 shows $F[\alpha]$ is itself a field containing $\alpha$,
  hence it must \emph{equal} the smallest such field, namely $F(\alpha)$.
  
  \medskip
  \textbf{5.\;Degree formula.}
  Because $\{1,\alpha,\dots,\alpha^{\,n-1}\}$ is an $F$–basis of
  $F(\alpha)$, we have
  \[
       [\,F(\alpha):F\,]=n=\deg m_{\alpha,F},
  \]
  as claimed.
  \end{proof}\begin{proposition}
    Let $R$ be a \textbf{Euclidean domain}\footnote{%
    All Euclidean domains are PIDs and hence UFDs, but we shall only
    use the fact that every two elements admit a greatest common divisor
    which can be expressed as a linear combination (Bezout identity).}
    and let $a\in R$ be an \emph{irreducible} (hence non–unit) element.
    Then the principal ideal
    \[
       (a)\;:=\;\{\,ar \mid r\in R\,\}\;\subset R
    \]
    is a \emph{maximal} ideal of~$R$.
    \end{proposition}
    
    \begin{proof}[Step--by--step]
    \textbf{1.\;$(a)$ is proper.}  
    Because $a$ is not a unit, $1\notin(a)$, so $(a)\neq R$.
    
    \medskip
    \textbf{2.\;Proper ideals contain no units.}  
    If $I\subset R$ is an ideal and contains a unit $u\in R^{\times}$,
    then $1=u^{-1}u\in I$, hence $I=R$.
    Conse\-quently every proper ideal of $R$ contains \emph{no} units.
    
    \medskip
    \textbf{3.\;Assume $(a)$ is \underline{not} maximal.}  
    Then there exists a proper ideal
    \[
       (a)\;\subsetneq\;I\;\subsetneq\;R.
    \]
    Choose $b\in I\setminus(a)$; necessarily $b\neq 0$ and $b$ is \emph{not}
    a unit (by Step~2).
    
    \medskip
    \textbf{4.\;Greatest common divisor of $a$ and $b$.}  
    Because $R$ is Euclidean (hence a UFD), the gcd
    \[
       d\;:=\;\gcd(a,b)
    \]
    exists and may be written as a linear combination
    (Bezout identity):
    \[
       \exists\,u,v\in R\quad ua+vb=d.
    \]
    
    \medskip
    \textbf{5.\;$d=1$.}  
    Since $a$ is irreducible and $a\nmid b$ (because $b\notin(a)$),
    the only common divisors of $a$ and $b$ are units.
    Hence $\gcd(a,b)=1$, so $d=1$ in Step~4.
    
    \medskip
    \textbf{6.\;The contradiction.}  
    Because $a\in(a)\subset I$ and $b\in I$, the linear combination
    $ua+vb=1$ lies in $I$ (ideals are closed under
    ring multiplication and addition).  
    Thus $1\in I$, contradicting the assumption that $I$ is proper.
    
    \medskip
    \textbf{7.\;Conclusion.}  
    No proper ideal can sit strictly between $(a)$ and $R$,
    so $(a)$ is maximal.
    \end{proof}
    \begin{example}[A concrete splitting field over $\mathbb{Q}$]
      Consider the irreducible\footnote{%
      Irreducibility follows from Eisenstein’s criterion with the prime
      $p=2$.} polynomial
      \[
            f(x)\;=\;x^{3}\;-\;2\;\in\;\mathbb{Q}[x].
      \]
      
      \medskip
      \textbf{1.\;Roots in $\mathbb{C}$.}\;
      Its three complex roots are
      \[
            \alpha_{1}= \sqrt[3]{2},\qquad
            \alpha_{2}= \sqrt[3]{2}\,\zeta_{3},\qquad
            \alpha_{3}= \sqrt[3]{2}\,\zeta_{3}^{\,2},
            \quad\text{where}\;\;
            \zeta_{3}:=e^{2\pi i/3}=\frac{-1+\sqrt{-3}}{2}.
      \]
      
      \medskip
      \textbf{2.\;Generating the splitting field.}\;
      The \emph{smallest} field that contains \emph{all}
      roots of $f$ and $\mathbb{Q}$ is
      \[
            K\;=\;\mathbb{Q}\bigl(\,\sqrt[3]{2},\;\zeta_{3}\bigr).
      \]
      It is called the \emph{splitting field} of $f$ over~$\mathbb{Q}$.
      
      \medskip
      \textbf{3.\;Tower of extensions and degree.}\;
      Form the tower
      \[
            \mathbb{Q} \;\subset\; \mathbb{Q}\bigl(\sqrt[3]{2}\bigr) 
              \;\subset\; \mathbb{Q}\bigl(\sqrt[3]{2},\zeta_{3}\bigr)=K.
      \]
      \begin{itemize}
        \item $\deg\bigl(\mathbb{Q}(\sqrt[3]{2})/\mathbb{Q}\bigr)=3$
              because $x^{3}-2$ is irreducible.
        \item $\zeta_{3}$ satisfies $x^{2}+x+1=0$, irreducible over
              $\mathbb{Q}(\sqrt[3]{2})$ (the polynomial has no root there),  
              so $\deg\bigl(K/\mathbb{Q}(\sqrt[3]{2})\bigr)=2$.
      \end{itemize}
      Hence by the tower law
      \[
            [K:\mathbb{Q}]=3\cdot 2=6.
      \]
      
      \medskip
      \textbf{4.\;Splitting.}\;
      Inside $K$ we indeed have
      \[
            f(x)\;=\;
            \bigl(x-\sqrt[3]{2}\bigr)
            \bigl(x-\sqrt[3]{2}\zeta_{3}\bigr)
            \bigl(x-\sqrt[3]{2}\zeta_{3}^{\,2}\bigr),
      \]
      so $f$ \emph{splits completely} into linear factors over~$K$.
      No proper intermediate field of~$\mathbb{{Q}}$ already contains
      all three roots, so $K$ is the splitting field.
      
      \medskip
      \textbf{5.\;Galois group (optional).}\;
      Because $f$ is separable and $K/\mathbb{{Q}}$ is the splitting field,
      $K/\mathbb{{Q}}$ is a Galois extension.
      One can show
      \[
            \mathrm{Gal}(K/\mathbb{{Q}})\;\cong\;S_{3},
      \]
      generated by 
      $\;\sigma:\sqrt[3]{2}\!\mapsto\!\sqrt[3]{2}\zeta_{3},\;
        \zeta_{3}\!\mapsto\!\zeta_{3}\;$
      and
      $\tau:\sqrt[3]{2}\!\mapsto\!\sqrt[3]{2},\;
        \zeta_{3}\!\mapsto\!\zeta_{3}^{\,2}$.
      
      \bigskip
      \noindent
      Thus $K=\mathbb{{Q}}\!\bigl(\sqrt[3]{2},\zeta_{3}\bigr)$ is a classic
      \emph{splitting field} example: it is the minimal field over
      $\mathbb{{Q}}$ in which the polynomial $x^{3}-2$ factors into
      linear terms.
      \end{example}
      % --------------------------------------------------------------------------
%  Splitting–field existence: adjoining (any) one root
% --------------------------------------------------------------------------
\begin{theorem}[Adjoining a root]\label{thm:adjoin–root}
  Let \(F\) be a field and let \(f\in F[x]\) be a non–constant
  polynomial.  
  There exists a field extension \(E/F\) and an element
  \(\alpha\in E\) such that \(f(\alpha)=0\).
  Moreover, \(E\) can be chosen to be a \emph{finite} (i.e.\ algebraic)
  extension of~\(F\).
  \end{theorem}
  
  \begin{proof}[Step–by–step construction of \(E\)]
  \textbf{1.\;Reduce to an irreducible factor.}
  If \(f\) is reducible, choose one of its irreducible factors
  \[
        p(x)\;\bigm|\;f(x), \qquad
        \deg p = n\ge 1 .
  \]
  Since \(p(\alpha)=0\) will automatically force \(f(\alpha)=0\),
  it suffices to adjoin a root of \(p\).
  Hence from now on \emph{assume \(f\) itself is irreducible} of degree
  \(n\).
  
  \smallskip
  \textbf{2.\;The quotient ring \(F[x]/(f)\).}
  Because \(F[x]\) is a Euclidean domain,
  the ideal \(\bigl(f(x)\bigr)\subset F[x]\) generated by an
  irreducible element is \emph{maximal}.
  Consequently
  \[
        E \;:=\; F[x]\bigl/\!\bigl(f(x)\bigr)
  \]
  is a \emph{field}.  
  Denote by
  \(
        \pi : F[x]\to E,\;
        g(x)\longmapsto g(x)+\bigl(f(x)\bigr)
  \)
  the natural quotient map.
  
  \smallskip
  \textbf{3.\;Embedding \(F\hookrightarrow E\).}
  For every \(\lambda\in F\subset F[x]\) we have
  \(
        \pi(\lambda)=\lambda+\bigl(f(x)\bigr).
  \)
  Because \(\pi\) has trivial kernel, this injects \(F\) into \(E\);
  thus we may view \(E\) as a field \emph{extension} of \(F\).
  
  \smallskip
  \textbf{4.\;A root of \(f\) inside \(E\).}
  Let
  \[
        \alpha\;:=\;x+\bigl(f(x)\bigr)\;\in\;E ,
  \]
  (the coset of the indeterminate \(x\)).
  Then, in \(E\),
  \[
        f(\alpha)\;=\;\pi\!\bigl(f(x)\bigr)=
        f(x)+\bigl(f(x)\bigr)=0,
  \]
  so \(\alpha\) is indeed a root of \(f\).
  
  \smallskip
  \textbf{5.\;Finiteness of the extension.}
  Write
  \[
        \mathcal{B}\;:=\;\{\,1,\alpha,\alpha^{2},\dots,\alpha^{n-1}\,\}\;\subset E.
  \]
  Because \(f\) is monic of degree \(n\) and satisfies
  \(f(\alpha)=0\), every power \(\alpha^{k}\) with \(k\ge n\) can be
  reduced (by polynomial long division) to an \(F\)-linear
  combination of the entries of \(\mathcal{B}\).
  Thus \(\text{Span}_{F}(\mathcal{B}) = E\).
  
  To see the elements of \(\mathcal{B}\) are independent, suppose
  \(
        a_{0}+a_{1}\alpha+\dots+a_{n-1}\alpha^{n-1}=0
  \)
  with \(a_{i}\in F\).  
  Applying \(\pi^{-1}\) gives a polynomial of degree \(<n\) that
  vanishes at \(\alpha\); but \(f\) is \emph{minimal}, so all
  coefficients must vanish.  
  Hence \(\mathcal{B}\) is a basis and
  \[
        [E:F]=n<\infty .
  \]
  
  \smallskip
  \textbf{6.\;Conclusion.}
  The field \(E:=F[x]/\bigl(f(x)\bigr)\) is finite over \(F\)
  and contains a root \(\alpha\) of \(f\).
  \end{proof}
  % --------------------------------------------------------------------------
  %
  %  (Remarks, if desired, can be added here.)
  %
  % --------------------------------------------------------------------------
  % ---------------------------------------------------------------------------
%  Examples of Principal Ideal Domains (PIDs)
% ---------------------------------------------------------------------------
\begin{example}
  \item \textbf{The integers \(\boldsymbol{\mathbb{Z}}\).}  
        Every ideal of \(\mathbb{Z}\) has the form 
        \((n)\!=\!n\mathbb{Z}\) for some \(n\in\mathbb{Z}\); hence \(\mathbb{Z}\) is a PID.  
        (In fact \(\mathbb{Z}\) is Euclidean with the absolute value norm.)
  
  \item \textbf{Polynomial rings in one variable over a field.}  
        For any field \(k\) the ring \(k[x]\) is Euclidean 
        (hence a PID) with Euclidean function
        \(\deg\colon k[x]\smallsetminus\{0\}\to\mathbb{N}\cup\{-\infty\}\).  
        Typical concrete instances:
        \[
           \mathbb{Q}[x],\qquad
           \mathbb{F}_{p}[x]\;(p\text{ prime}),\qquad
           \mathbb{R}[x],\qquad
           \mathbb{C}[x].
        \]
  
  \item \textbf{Gaussian integers \(\boldsymbol{\mathbb{Z}[i]}\).}  
        \(\mathbb{Z}[i]=\{a+bi\mid a,b\in\mathbb{Z}\}\) is Euclidean
        for the norm \(N(a+bi)=a^{2}+b^{2}\).  
        Hence \(\mathbb{Z}[i]\) is a PID (and a UFD).
  
  \item \textbf{Eisenstein integers \(\boldsymbol{\mathbb{Z}[\omega]}\).}  
        Here \(\omega=e^{2\pi i/3}=\tfrac{-1+\sqrt{-3}}{2}\).
        The norm \(N(a+b\omega)=a^{2}-ab+b^{2}\) makes \(\mathbb{Z}[\omega]\)
        Euclidean, so \(\mathbb{Z}[\omega]\) is a PID.
  
  \item \textbf{\(p\)-adic integers \(\boldsymbol{\mathbb{Z}_{p}}\).}  
        For a prime \(p\) the ring
        \[
           \mathbb{Z}_{p}\;=\;\varprojlim_{n}\mathbb{Z}/p^{n}\mathbb{Z}
        \]
        is a \emph{discrete valuation ring};
        its non–zero ideals are exactly \((p^{n})\;(n\ge0)\),
        each generated by a power of~\(p\).  
        Hence \(\mathbb{Z}_{p}\) is a PID, \emph{but not} Euclidean.
  
  \item \textbf{Formal power–series rings \(\boldsymbol{k[[x]]}\).}  
        For any field \(k\) the power–series ring
        \(k[[x]]\) is again a discrete valuation ring
        with unique non–zero prime ideal \((x)\):
        every ideal equals \((x^{n})\) for some \(n\ge0\).  
        Thus \(k[[x]]\) is a PID (though not Euclidean).
  
  \item \textbf{Quadratic integer rings with class number \(1\).}  
        The rings of integers of the imaginary quadratic fields
        \[
           \mathbb{Q}\!\bigl(\sqrt{-d}\bigr),\qquad
           d\in\{1,2,3,7,11,19,43,67,163\},
        \]
        are PIDs.  
        For example
        \(
           \mathcal{O}_{\mathbb{Q}(\sqrt{-19})}=
           \mathbb{Z}\bigl[\tfrac{1+\sqrt{-19}}{2}\bigr]
        \)
        is a PID, even though no known Euclidean norm exists on it.
  
  \end{example}
  
  \bigskip
  \noindent
  \textbf{Remarks.}
  \begin{itemize}
  \item Every Euclidean domain is a PID, so Euclidean examples are
        plentiful (items~1–4 above).
  \item Discrete valuation rings (items~5–6) provide PIDs that are
        \emph{not} Euclidean, illustrating that the Euclidean property
        is stronger than principality.
  \item The quadratic–integer examples in item~7 show that there exist
        PIDs which are neither Euclidean nor DVRs in the strict sense,
        highlighting the subtle spectrum of properties between
        “Euclidean’’ and “PID’’.
  \end{itemize}
  % ---------------------------------------------------------------------------
%  Adjoining one root of a polynomial — every detail spelled out
% ---------------------------------------------------------------------------
\subsection*{Theorem (Every polynomial has a root in a finite extension)}
Let \(F\) be a field and \(f\in F[X]\) a non–constant polynomial.
Then there exists a \emph{finite} field extension \(E/F\) together
with an element \(\alpha\in E\) such that
\[
     f(\alpha)=0.
\]

\bigskip
\textbf{Proof \;(full expansion).}

\medskip
\textbf{1.\;Why \(F[X]\) is Euclidean (hence a PID and UFD).}
For every non–zero polynomial \(g\in F[X]\) define
\(\delta(g):=\deg g\in\mathbb{N}\).
The usual polynomial long–division shows that for any
\(a,b\in F[X]\) with \(b\neq0\) we can find
\(q,r\in F[X]\) such that
\(a=bq+r\) and either \(r=0\) or \(\deg r<\deg b\).
Thus \(\delta\) is a Euclidean function and \(F[X]\) is a 
\emph{Euclidean domain}.
Consequences:

\begin{itemize}
\item Every ideal of \(F[X]\) is principal (PID).
\item Irreducible elements generate \emph{maximal} ideals.
\item Unique factorisation of polynomials (UFD).
\end{itemize}

\medskip
\textbf{2.\;Reduce to an irreducible factor.}
If \(f\) is reducible, pick an irreducible factor \(p\mid f\).
A root of \(p\) will automatically be a root of \(f\), so we may
\emph{replace \(f\) by an irreducible factor}.
Henceforth assume \(f\) itself is irreducible of degree
\(n\ge1\).

\medskip
\textbf{3.\;Form the quotient field \(E:=F[X]/(f)\).}

\begin{enumerate}[label=\alph*)]
\item Because \(f\) is irreducible, the ideal \((f)\subset F[X]\)
      is \emph{maximal}.  
      Consequently the quotient
      \[
          E\;:=\;F[X]\bigl/(f)
      \]
      is a \emph{field}.
\item Denote the canonical projection by
      \(
          \pi:F[X]\twoheadrightarrow E,\quad
          g\mapsto g+(f).
      \)
\item The kernel of \(\pi\) is exactly \((f)\); hence \(\pi\) is
      injective on the subring \(F\subset F[X]\).
      Via the embedding
      \[
           F \;\hookrightarrow\;E,\qquad
           \lambda\longmapsto\lambda+(f),
      \]
      we view \(E\) \emph{as a field extension of \(F\)}.
\end{enumerate}

\medskip
\textbf{4.\;The element \(\alpha:=X+(f)\) is a root of \(f\).}

Let
\[
     \alpha\;:=\;X+(f)\;\in\;E.
\]
Because \(f(X)\in(f)\), we have
\[
     f(\alpha)=\pi\!\bigl(f(X)\bigr)=f(X)+(f)=0\quad\text{in }E.
\]
Thus \(\alpha\) is the desired root.

\medskip
\textbf{5.\;Proving \(E/F\) is finite.}

We need to show that \(E\) has finite dimension as an \(F\)-vector
space.

\begin{enumerate}[label=\alph*)]
\item \emph{Candidate spanning set.}  
      Consider the \(n\) elements
      \[
           1+(f),\;X+(f),\;X^{2}+(f),\;\dots,\;X^{n-1}+(f)
      \]
      and abbreviate them by
      \(\{1,\alpha,\alpha^{2},\dots,\alpha^{n-1}\}\subset E\).
\item \emph{Why they span.}  
      Take an arbitrary element of \(E\); it is of the form
      \(g(X)+(f)\) for some \(g\in F[X]\).
      Perform polynomial long–division of \(g\) by \(f\):
      \[
           g(X)=q(X)f(X)+r(X),\qquad \deg r<\deg f=n.
      \]
      Applying \(\pi\) gives
      \(
         g(X)+(f)=r(X)+(f).
      \)
      But
      \[
           r(X)+(f)=a_{0}+a_{1}\alpha+\dots+a_{n-1}\alpha^{n-1},
           \qquad a_{i}\in F,
      \]
      showing every element of \(E\) is in the \(F\)-span of our
      \(n\)-element set.
\item \emph{Why they are linearly independent.}  
      Suppose
      \(
           a_{0}+a_{1}\alpha+\dots+a_{n-1}\alpha^{n-1}=0
      \)
      in \(E\).
      Pull this back to \(F[X]\):
      \(
           a_{0}+a_{1}X+\dots+a_{n-1}X^{n-1}\in(f).
      \)
      Hence \(f\mid\bigl(a_{0}+a_{1}X+\dots+a_{n-1}X^{n-1}\bigr)\),
      but the latter has degree \(<n\), impossible unless all
      \(a_{i}=0\).
\end{enumerate}
Therefore
\[
     \bigl\{1,\alpha,\dots,\alpha^{n-1}\bigr\}
     \text{ is an }F\text{-basis of }E,
     \qquad
     [E:F]=n<\infty.
\]

\medskip
\textbf{6.\;Conclusion.}
We have produced
\begin{itemize}
\item a finite field extension \(E/F\)
      of degree \(\le\deg f\);
\item an element \(\alpha\in E\) satisfying \(f(\alpha)=0\).
\end{itemize}
Hence every polynomial in \(F[X]\) acquires a root in some
finite extension of \(F\).\qedhere

% ---------------------------------------------------------------------------
%  End of expanded proof
% ---------------------------------------------------------------------------
\paragraph{Definition (Maximal Ideal).}
Let \(R\) be a commutative ring with unity \(1\neq 0\).
An ideal \( \mathfrak{m}\subset R \) is called \emph{maximal} if it is
proper (i.e.\ \( \mathfrak{m}\neq R \)) and there exists
\emph{no} ideal strictly between \( \mathfrak{m} \) and \( R \):
\[
   \mathfrak{m}\subsetneq I\subseteq R
   \;\;\Longrightarrow\;\;
   I = R .
\]

\medskip
\paragraph{Equivalent Characterisation.}
An ideal \( \mathfrak{m}\subset R \) is maximal  
\(\Longleftrightarrow\)
the quotient ring \( R/\mathfrak{m} \) is a \emph{field}.
(Proof: \(R/\mathfrak{m}\) is always a ring; maximality of
\( \mathfrak{m} \) guarantees that every non-zero coset
has a multiplicative inverse, and conversely.)

\medskip
\paragraph{Key Properties.}
\begin{enumerate}
  \item Every maximal ideal is \emph{prime}, but not every prime ideal
        is maximal (unless \(R\) is a PID or a principal Artinian ring).
  \item Maximal ideals always exist in any non-zero ring with unity;
        Zorn’s Lemma guarantees this.
  \item In a PID (e.g.\ \(\mathbb{Z}\) or \(k[x]\) with \(k\) a field),
        irreducible elements generate maximal ideals.
\end{enumerate}

\medskip
\paragraph{Concrete Examples.}
\begin{itemize}
  \item \(\mathbf{Integers:}\)
        For a prime number \(p\),
        \[
           (p)\;=\;p\mathbb{Z}\;\subset\;\mathbb{Z}
        \]
        is maximal because \(\mathbb{Z}/(p)\cong\mathbb{F}_{p}\)
        is a field.

  \item \(\mathbf{Univariate\;polynomials:}\)
        If \(k\) is a field and \(f(x)\in k[x]\) is \emph{irreducible},
        then
        \[
           (f)\;\subset\;k[x]
        \]
        is maximal; the quotient \(k[x]/(f)\) is a field extension of
        \(k\) of degree \(\deg f\).

  \item \(\mathbf{Multivariate\;case:}\)
        For \(k=\mathbb{C}\) and a point \(a=(a_{1},\dots,a_{n})\in k^{n}\),
        the ideal
        \[
           \mathfrak{m}_{a}
           \;:=\;
           (\,x_{1}-a_{1},\dots,x_{n}-a_{n}\,)
           \;\subset\;k[x_{1},\dots,x_{n}]
        \]
        is maximal because
        \(k[x_{1},\dots,x_{n}]/\mathfrak{m}_{a}\cong k\)
        (evaluation at the point \(a\)).

  \item \(\mathbf{p-adic integers:}\)
        In \(\mathbb{Z}_{p}\) the unique maximal ideal is
        \((p)\); the residue field is \(\mathbb{F}_{p}\).
\end{itemize}

\medskip
\paragraph{Non-Example.}
In \(\mathbb{Z}\) the ideal \((6)\) is \emph{not} maximal,
because \((6)\subsetneq(2)\subsetneq\mathbb{Z}\).

\medskip
\paragraph{Why Maximal Ideals Matter.}
\begin{itemize}
  \item They classify simple (one-point) algebraic “quotient spaces’’
        of a ring.
  \item In algebraic geometry, maximal ideals of
        \(k[x_{1},\dots,x_{n}]\) correspond to points in the affine space
        \(k^{n}\) (Hilbert’s Nullstellensatz when \(k\) is algebraically closed).
  \item Via the Jacobson radical, maximal ideals control the “topology”
        of \(\operatorname{Spec} R\) and the semisimple structure of
        modules over~\(R\).
\end{itemize}
\begin{theorem}[A first step toward a splitting field]
  Let \(F\) be a field and let \(f\in F[X]\) be a non–constant polynomial.
  Then there exists a field extension \(E/F\) in which \(f\) has a root.
  Moreover the extension \(E/F\) can be chosen to be finite.
  \end{theorem}
  
  \begin{proof}
  \textbf{Step 1.  Work in the Euclidean domain \(F[X]\).}  
  Because \(F\) is a field, the polynomial ring \(F[X]\) is a Euclidean
  domain (the Euclidean function is the usual degree).  Hence \(F[X]\) is
  a principal ideal domain (PID) and therefore a unique-factorisation
  domain (UFD).  In particular, every non–constant polynomial can be
  factored into irreducibles.  Replacing \(f\) by one of its irreducible
  factors if necessary, we \emph{assume from now on that \(f\) itself is
  irreducible}.  Write
  \[
     \deg f = n\ge 1 .
  \]
  
  \textbf{Step 2.  Form a natural candidate for the extension field.}  
  Consider the prime (hence maximal) ideal \((f)\subseteq F[X]\).
  Because \((f)\) is maximal, the quotient
  \[
     E \;:=\; F[X]\big/ (f)
  \]
  is a field.  We obtain a natural ring homomorphism
  \[
     \iota: F \;\longrightarrow\; E, \qquad
     \lambda \;\longmapsto\; \lambda + (f),
  \]
  by regarding a constant \(\lambda\in F\) as the constant polynomial
  \(\lambda\in F[X]\) and then reducing modulo \((f)\).
  The kernel of \(\iota\) is trivial, so \(\iota\) is injective and we can
  \emph{identify \(F\) with its image inside \(E\)}.  Thus \(E/F\) is a
  field extension.
  
  \textbf{Step 3.  Exhibit an explicit root of \(f\) in \(E\).}  
  Denote by
  \[
     \alpha \;:=\; X + (f) \;\in\; E
  \]
  the image of the indeterminate \(X\in F[X]\) in the quotient.
  Because \(X\) is the residue class of the polynomial \(X\),
  we may \emph{evaluate \(f\) at \(\alpha\)} inside \(E\) by simply
  reducing the polynomial \(f(X)\) modulo \((f)\):
  \[
     f(\alpha)
     \;=\;
     f\bigl(X+(f)\bigr)
     \;=\;
     f(X) + (f).
  \]
  But \(f(X)\in (f)\) by definition of the ideal, so
  \(f(X) + (f) = (f)\), which is the zero element of \(E\).
  Therefore
  \[
     f(\alpha)=0 \quad\text{in }E,
  \]
  i.e.\ \(\alpha\) is a \emph{root} of \(f\) in the extension \(E\).
  
  \medskip
  \emph{Why is \((f)\) the additive identity in \(E\)?}  
  Elements of \(E=F[X]/(f)\) are congruence classes \(g(X)+(f)\).
  The class \((f)\) itself is represented by the polynomial \(f(X)\).  
  Because \(f(X)\equiv 0 \pmod{(f)}\), the class \((f)\) acts exactly like
  the zero polynomial when we add classes:
  \[
     \bigl(g(X)+(f)\bigr) \;+\; (f) 
     \;=\; g(X) + f(X) + (f) 
     \;=\; g(X) + (f).
  \]
  Hence \((f)\) is the unique additive identity of the quotient ring.
  Since a root of a polynomial is defined by the vanishing of that
  polynomial, the calculation above shows \(f(\alpha)=0\).
  
  \textbf{Step 4.  Show that \(E/F\) is finite.}  
  Because \(\deg f = n\), observe that every residue class
  \(g(X)+(f)\in E\) has a representative of degree \(<n\):
  apply the Euclidean algorithm to write \(g(X)=q(X)f(X)+r(X)\) with
  either \(r=0\) or \(\deg r < n\).  In \(E\) we have
  \(g(X)+(f)=r(X)+(f)\).  Consequently the set
  \[
     \mathcal{B}\;=\;\bigl\{\,1+(f),\,X+(f),\,
            X^{2}+(f),\ldots,X^{n-1}+(f)\,\bigr\}
  \]
  spans \(E\) as a vector space over \(F\).  Hence
  \(\dim_{F}E\le n\), and \(E/F\) is a \emph{finite} extension
  (of degree at most \(n\)).
  
  \textbf{Conclusion.}  
  The field \(E=F[X]/(f)\) is finite over \(F\) and contains an element
  \(\alpha=X+(f)\) with \(f(\alpha)=0\).  This completes the proof.
  \end{proof}
  \pagebreak
  \section*{Faithful versus Injective---what is the difference?}

\subsection*{1.  Two contexts where the words occur}

\begin{enumerate}
   \item \textbf{Maps between algebraic objects}  
         Here “injective’’ makes literal sense: a homomorphism 
         \( \varphi\colon A \to B \) is \emph{injective} 
         (a \emph{monomorphism} in the category of sets)  
         if \( \varphi(a_1)=\varphi(a_2)\Rightarrow a_1=a_2\).

   \item \textbf{Representations / actions / functors}  
         Words like “faithful module”, “faithful group action” or
         “faithful functor’’ describe \emph{how well the source object
         is reflected inside the target category}.  
\end{enumerate}

\subsection*{2.  Group homomorphisms and representations}

\begin{definition}
   Let \(G\) and \(H\) be groups and \( \rho\colon G \to H\) a group
   homomorphism.
   \begin{enumerate}
      \item \(\rho\) is \textbf{injective} if it is injective as a map
            of underlying sets.
      \item \(\rho\) (or the corresponding action of \(G\) on \(H\))
            is \textbf{faithful} if its kernel is trivial:
            \(
               \ker\rho=\{e_G\}.
            \)
   \end{enumerate}
\end{definition}

\paragraph{Equivalence in this setting.}
For group homomorphisms these two notions coincide:

\[
   \rho \text{ injective}
   \quad\Longleftrightarrow\quad
   \ker\rho=\{e_G\}
   \quad\Longleftrightarrow\quad
   \rho \text{ faithful}.
\]

Indeed, a group homomorphism is injective
iff its kernel is trivial.  

\emph{Take-away}:  
“Faithful’’ and “injective’’ are synonymous for \emph{homomorphisms of
groups, rings, vector spaces, \dots}.  

\subsection*{3.  Functors}

\begin{definition}
   A functor \(F\colon \mathcal{C}\to\mathcal{D}\) is
   \textbf{faithful} if for every pair of objects \(X,Y\) in
   \(\mathcal{C}\) the function
   \[
      F_{X,Y}\colon \operatorname{Hom}_{\mathcal{C}}(X,Y)
         \;\longrightarrow\;
         \operatorname{Hom}_{\mathcal{D}}(F(X),F(Y))
   \]
   is injective.
\end{definition}

\begin{itemize}
   \item Faithfulness \emph{ignores objects}:  
         two different objects may still be mapped to the \emph{same}
         object in \(\mathcal{D}\).
   \item Hence a faithful functor \textbf{need not be injective on
         objects}.  (Think of the forgetful functor
         \(\mathbf{Grp}\to\mathbf{Set}\): it is faithful, but many
         non-isomorphic groups become the same underlying set.)
\end{itemize}

So for functors “faithful’’ is strictly weaker than any reasonable
notion of “injective’’ you might impose on objects.

\subsection*{4.  Modules}

For a (left) \(R\)-module \(M\):

\[
   M \text{ is \textbf{faithful}}
   \;\Longleftrightarrow\;
   \operatorname{Ann}_R(M)=\{\,0\}
\]

(the only element of \(R\) acting as \(0\) on \(M\) is \(0\) itself).
There is \emph{no} underlying set-map here, so “injective’’ does not
make sense in the usual way; instead one compares with the module‐theoretic
property of being an \textbf{injective module}, which concerns
extensions of homomorphisms and is unrelated to faithfulness.

\subsection*{5.  Summary table}

\[
\begin{array}{|c|c|c|}
\hline
\text{Context} & \text{Faithful means\ldots} & \text{Relation to ``injective''} \\
\hline
\text{Group (or ring) hom.\ } \rho\colon G\to H 
   & \ker\rho=\{e\} & \text{Exactly equivalent.}\\
\hline
\text{Functor } F\colon\mathcal{C}\to\mathcal{D}
   & F_{X,Y}\text{ injective on each Hom-set} 
   & \text{Weaker; may identify objects.}\\
\hline
\text{Module } M \text{ over } R 
   & \operatorname{Ann}_R(M)=0 & \text{No direct link to ``injective module''.}\\
\hline
\end{array}
\]

\bigskip
\noindent\emph{Rule of thumb:}  
For plain homomorphisms into an \emph{automorphism group} (e.g.\ group
actions, linear representations) “faithful’’ \(\Longleftrightarrow\)
“injective’’.  
In more elaborate categorical settings “faithful’’ only captures
injectivity on \emph{morphisms}, not on objects.
\section*{What \emph{is} a Module?}

\subsection*{1.  The core idea}

A \textbf{module} generalises the notion of a vector space by allowing
the “scalars’’ to come from an arbitrary (possibly non-commutative) ring
instead of a field.

\begin{definition}[Left $R$-module]
Let $R$ be a (unital) ring.  
A \emph{left $R$-module} is an abelian group $\,(M,+)\,$ equipped with a
map
\[
   R\times M \;\longrightarrow\; M,
   \qquad (r,m)\longmapsto r\cdot m,
\]
satisfying, for all $r,s\in R$ and $m,n\in M$,
\begin{align*}
   r\cdot(m+n)             &= r\cdot m + r\cdot n &\text{(distributivity over $M$)},\\
   (r+s)\cdot m            &= r\cdot m + s\cdot m &\text{(distributivity over $R$)},\\
   (rs)\cdot m             &= r\cdot(s\cdot m)    &\text{(associativity)},\\
   1_R\!\cdot m            &= m                   &\text{(unit action)}.
\end{align*}
\end{definition}

\noindent
\emph{Right $R$-modules} are defined analogously, except that scalars
act on the \emph{right}: $m\cdot r$.

\medskip
\textbf{Vector spaces} are precisely the modules whose scalars come from
a division ring (in particular, a field).

\subsection*{2.  Basic examples}

\begin{enumerate}
   \item \emph{Abelian groups} $\;=\;$ $\mathbb Z$-modules  
         (scalars are integers; scalar multiplication is iterated
         addition).

   \item Any \emph{vector space} $V$ over a field $F$ is an $F$-module.

   \item The ring $R$ itself, with multiplication as the action,
         is a left (and right) $R$-module, often denoted ${}_R R$ or
         $R_R$.

   \item \emph{Ideals} of $R$ are submodules of the regular module
         ${}_R R$.

   \item For a topological space $X$, the set $C(X,\Bbb R)$ of
         continuous real-valued functions is a module over the ring
         $\Bbb R$ (and also over $C(X,\Bbb R)$ itself).
\end{enumerate}

\subsection*{3.  Module morphisms and submodules}

\begin{definition}
A homomorphism of left $R$-modules is a group homomorphism
$\varphi:M\to N$ satisfying $\varphi(r\cdot m)=r\cdot\varphi(m)$.
\end{definition}

\noindent
Submodules, kernels, images, and quotients are defined exactly as in
linear algebra, but keeping the $R$-action compatible.

\subsection*{4.  Free, finitely generated, and torsion modules}

\begin{itemize}
   \item A \emph{free} $R$-module has a basis: it is isomorphic to
         $R^{(I)}$ for some index set $I$.
   \item It is \emph{finitely generated} if $I$ can be chosen finite.
   \item Over an integral domain $R$, an element $m\in M$ is
         \emph{torsion} if $rm=0$ for some non-zero $r\in R$.
\end{itemize}

\subsection*{5.  Why modules matter}

\begin{enumerate}
   \item They provide a common language for linear algebra,
         group theory (\(\mathbb Z\)-modules), and algebraic topology
         (e.g.\ homology groups).
   \item Many structural properties of rings are best understood via
         their module categories (projective, injective, simple,
         Noetherian, Artinian, \dots).
   \item Representation theory: a group representation is precisely a
         module over the group ring $\Bbb C[G]$ (or any field).
\end{enumerate}

\subsection*{6.  A quick “dictionary’’}

\[
\begin{array}{|c|c|}
\hline
\text{Vector-space term} & \text{Module analogue} \\ \hline
\text{Field } F & \text{Ring } R \\ \hline
\text{Vector space } V & R\text{-module } M \\ \hline
\dim_F V & \text{Rank (if free)} \\ \hline
\text{Linear map} & \text{$R$-module homomorphism} \\ \hline
\text{Basis} & \text{Module basis (free case)} \\ \hline
\end{array}
\]

\bigskip
\noindent
\textbf{Take-away:}  
If you are happy with vector spaces, just replace “field’’ by “ring’’
and allow that not every non-zero scalar is invertible.  
All the familiar linear-algebra constructions still make sense, but new
phenomena (torsion, non-free modules, one-sided actions, etc.) appear
and enrich the theory.
\section*{Concrete Examples of Galois Groups}

Below are several standard field extensions together with their Galois-
groups.  Each example is fully explicit, so you can verify the Galois
condition (\emph{normal} $+$ \emph{separable}) and compute the group
structure directly.

\bigskip
\begin{enumerate}
%--------------------------------------------------------------------
\item \textbf{Quadratic extensions.}

      \[
         K \;=\; \Bbb Q\!\left(\sqrt d\right), 
         \qquad d\in\Bbb Z \text{ square-free}.
      \]

      Every automorphism of $K$ over $\Bbb Q$ is determined by the image
      of $\sqrt d$, and the only two possibilities are
      $\sqrt d \mapsto \pm\sqrt d$.  Hence 

      \[
         \operatorname{Gal}(K/\Bbb Q)
           \;\cong\; C_2
           \quad\text{(cyclic of order $2$)}.
      \]

%--------------------------------------------------------------------
\item \textbf{Cyclotomic extensions.}

      Let $\zeta_n := e^{2\pi i/n}$ be a primitive $n$-th root of unity
      and set $K=\Bbb Q(\zeta_n)$.  Complex conjugation sends 
      $\zeta_n\mapsto\zeta_n^{-1}$ and, more generally, every integer
      $a\in(\Bbb Z/n\Bbb Z)^{\times}$ induces an automorphism

      \[
         \sigma_a : \zeta_n \longmapsto \zeta_n^{\,a}.
      \]

      The map $a\mapsto\sigma_a$ is an isomorphism, so

      \[
         \operatorname{Gal}\bigl(\Bbb Q(\zeta_n)/\Bbb Q\bigr)
            \;\cong\; (\Bbb Z/n\Bbb Z)^{\times}.
      \]

      \begin{itemize}
         \item $n=3$: $\;(\Bbb Z/3\Bbb Z)^{\times}\cong C_2$.
         \item $n=4$: $\;(\Bbb Z/4\Bbb Z)^{\times}\cong C_2$.
         \item $n=5$: $\;(\Bbb Z/5\Bbb Z)^{\times}\cong C_4$.
      \end{itemize}

%--------------------------------------------------------------------
\item \textbf{Finite fields.}

      For a prime $p$ and integer $n\ge1$ let 
      $K=\Bbb F_{p^{\,n}}$ and $F=\Bbb F_p$.  
      The Frobenius map
      \[
         \varphi : x \longmapsto x^{p}
      \]
      is an automorphism of $K$ fixing $F$; its $n$-th iterate is the
      identity and those iterates give all automorphisms.  Thus

      \[
         \operatorname{Gal}\bigl(\Bbb F_{p^{\,n}}/\Bbb F_p\bigr)
            \;\cong\; C_n.
      \]

%--------------------------------------------------------------------
\item \textbf{The splitting field of $x^{3}-2$ over $\Bbb Q$.}

      Factor in $\Bbb C$:
      \[
         x^{3}-2 = (x-\sqrt[3]{2})
                    \bigl(x-\sqrt[3]{2}\,\zeta_3\bigr)
                    \bigl(x-\sqrt[3]{2}\,\zeta_3^{\,2}\bigr),
         \qquad \zeta_3 = e^{2\pi i/3}.
      \]
      The splitting field is 
      $K=\Bbb Q\!\bigl(\sqrt[3]{2},\zeta_3\bigr)$,  
      and one checks that $[K:\Bbb Q]=6$.  
      A convenient set of generators is
      \[
         \sigma:\sqrt[3]{2}\mapsto\zeta_3\sqrt[3]{2},
         \quad\zeta_3\mapsto\zeta_3,
         \qquad
         \tau:   \sqrt[3]{2}\mapsto\sqrt[3]{2},
         \quad\zeta_3\mapsto\zeta_3^{\,2}.
      \]
      These satisfy the relations of the symmetric group $S_3$, so

      \[
         \operatorname{Gal}\bigl(\Bbb Q\bigl(\sqrt[3]{2},\zeta_3\bigr)
/\Bbb Q\bigr)\;\cong\;S_3.
      \]

%--------------------------------------------------------------------
\item \textbf{The splitting field of $x^{4}-2$ over $\Bbb Q$.}

      A root is $\sqrt[4]{2}$; 
      adjoining $i$ is necessary for all four roots.  
      One obtains $K=\Bbb Q\!\bigl(\sqrt[4]{2},i\bigr)$ with $[K:\Bbb Q]=8$.
      Presenting the Galois group:

      \[
          \sigma : \sqrt[4]{2}\mapsto i\sqrt[4]{2},\; i\mapsto i,
          \quad
          \tau   : \sqrt[4]{2}\mapsto \sqrt[4]{2},\; i\mapsto -i,
      \]
      gives relations $\sigma^{4}=\tau^{2}=1$ and
      $\tau\sigma\tau=\sigma^{-1}$.  
      Hence
      \[
         \operatorname{Gal}(K/\Bbb Q)\;\cong\;D_{4}
      \]
      (the dihedral group of order $8$).

%--------------------------------------------------------------------
\item \textbf{Generic separable quartic.}

      Let $f(x)$ be an irreducible separable quartic with
      non-square discriminant.  
      Then its splitting field $K$ over $\Bbb Q$ has Galois group

      \[
         \operatorname{Gal}(K/\Bbb Q)\;\cong\;S_4.
      \]

      (Typical example: $f(x)=x^{4}-x+1$.)

%--------------------------------------------------------------------
\item \textbf{An Artin–Schreier extension in characteristic $p$.}

      For a prime $p$ take $F=\Bbb F_p(t)$ and 
      consider $K=F\bigl(y\bigr)$ where $y$ satisfies
      \[
         y^{p}-y = t.
      \]
      The map $y\mapsto y+c$ ($c\in\Bbb F_p$) fixes $F$ and produces
      all automorphisms.  Thus

      \[
         \operatorname{Gal}(K/F)\;\cong\;C_p.
      \]

\end{enumerate}

\bigskip
\noindent
\textbf{Take-away.}\;
Every finite group occurs as a Galois group over \(\Bbb Q\)
(Shafarevich’s theorem), but the concrete examples above already cover
the most common structures encountered in an introductory course:
cyclic, dihedral, symmetric, and automorphism groups of finite fields.
\subsection*{Why Galois automorphisms preserve algebraic relations among the roots}

Let \(E/F\) be a (finite) Galois extension and let  
\[
   f(X)\;=\;a_nX^n+\dots+a_1X+a_0\;\in\;F[X], 
   \qquad a_i\in F,
\]
have roots \(r_1,\dots,r_n\in E\) (so \(E\) is, for instance, the
splitting field of \(f\)).  
Write \(G=\operatorname{Gal}(E/F)\).

\bigskip
\textbf{1.  Automorphisms fix the coefficients.}\;
By definition every \(\sigma\in G\) satisfies \(\sigma(\lambda)=\lambda\)
for all \(\lambda\in F\).  Hence
\[
   \sigma\!\bigl(a_i\bigr)=a_i
   \quad\text{for each }i.
\]

\bigskip
\textbf{2.  A single–variable relation.}\;
Because \(f(r_k)=0\) we have
\[
   0
   \;=\;
   \sigma\!\bigl(f(r_k)\bigr)
   \;=\;
   \sigma\!\Bigl(a_n r_k^{\,n}
                 +\dots+a_1 r_k + a_0\Bigr)
   \;=\;
   a_n\bigl(\sigma r_k\bigr)^{\!n}
      +\dots+a_1\bigl(\sigma r_k\bigr)+a_0
   \;=\;
   f\!\bigl(\sigma r_k\bigr).
\]
Therefore \(\sigma r_k\) is also a root of \(f\).  
\emph{Conclusion: \(G\) permutes the set \(\{r_1,\dots,r_n\}\).}

\bigskip
\textbf{3.  Many–variable algebraic relations.}\;
More generally take any polynomial
\[
   g(X_1,\dots,X_m)\;\in\;F[X_1,\dots,X_m].
\]
Suppose the chosen roots satisfy
\[
   g(r_{i_1},\dots,r_{i_m})\;=\;0.
\]
Applying \(\sigma\in G\) and using \(\sigma|_F=\operatorname{id}\) gives
\[
   0
   \;=\;
   \sigma\!\bigl(g(r_{i_1},\dots,r_{i_m})\bigr)
   \;=\;
   g\!\bigl(\sigma r_{i_1},\dots,\sigma r_{i_m}\bigr).
\]
Hence \emph{every polynomial relation over \(F\) that the roots satisfy
is preserved under every automorphism in \(G\)}.

\bigskip
\textbf{4.  The conceptual picture.}
\begin{itemize}
   \item Each \(\sigma\in G\) is an \(F\)-algebra automorphism of \(E\);
         it acts \emph{compatibly} with addition and multiplication
         \((\sigma(a+b)=\sigma a+\sigma b,\;\sigma(ab)=\sigma a\,\sigma b)\)
         and leaves every element of \(F\) fixed.
   \item Algebraic relations are precisely the vanishing of polynomials
         with coefficients in \(F\).  Because \(\sigma\) fixes those
         coefficients and respects ring operations, the vanishing is
         preserved after applying \(\sigma\).
   \item In particular, the minimal polynomial of a single root over
         \(F\) remains the same after applying \(\sigma\), so \(\sigma\)
         must send the root to \emph{another} root of that polynomial.
\end{itemize}

\bigskip
\textbf{5.  A quick example.}  
Let \(f(X)=X^3-2\in\Bbb Q[X]\) with roots  
\(r_1=\sqrt[3]{2},\;r_2=\sqrt[3]{2}\,\zeta_3,\;
  r_3=\sqrt[3]{2}\,\zeta_3^{\,2}\)
in the splitting field \(E=\Bbb Q(\sqrt[3]{2},\zeta_3)\).
For \(\sigma\in\operatorname{Gal}(E/\Bbb Q)\) defined by
\(\sigma(\sqrt[3]{2})=\sqrt[3]{2}\,\zeta_3\) and
\(\sigma(\zeta_3)=\zeta_3,\) we compute
\[
   f\!\bigl(\sigma(\sqrt[3]{2})\bigr)
   \;=\;
   f\!\bigl(\sqrt[3]{2}\,\zeta_3\bigr)
   \;=\;
   0,
\]
so \(\sigma\) indeed sends one root to another root, respecting the
relation \(X^3-2=0\) over \(\Bbb Q\).

\bigskip
\noindent
\emph{Key take-away.}\;
Because Galois automorphisms fix the coefficient field \(F\) and respect
ring operations, they carry any algebraic dependence with coefficients
in \(F\) to a new dependence with the \emph{same} coefficients—hence
they must permute the roots while preserving \emph{all}
\(F\)-polynomial relations those roots satisfy.
\section*{Noetherian Rings: Definition, Characterisations, and Examples}

\subsection*{1.  Definition}

\begin{definition}[Noetherian ring]
A (commutative, unital) ring \(R\) is called \emph{Noetherian} if it
satisfies \emph{any} (hence all) of the following equivalent
conditions:
\begin{enumerate}
   \item Every ascending chain of ideals stabilises:
         \[
            I_1 \;\subseteq\; I_2 \;\subseteq\; I_3 \;\subseteq\;\cdots
         \quad\Longrightarrow\quad
            \exists N \text{ such that } I_N = I_{N+1} = \cdots.
         \]
   \item Every ideal of \(R\) is \textbf{finitely generated}.
   \item (Maximal condition) Every non-empty set of ideals of \(R\)
         has a maximal element with respect to inclusion.
\end{enumerate}
\end{definition}

\emph{Equivalence.}  
\((1)\!\iff(2)\) is due to Emmy Noether (1921)---hence the name.
Condition \((3)\) is Zorn–Hausdorff’s reformulation.

\subsection*{2.  Fundamental properties}

\begin{itemize}
   \item \textbf{Ideals and quotients.}
         If \(R\) is Noetherian and \(I\lhd R\), then both \(I\) and
         \(R/I\) are Noetherian.

   \item \textbf{Finite generation.}
         If \(R\) is Noetherian and \(S\) is a finitely generated
         \(R\)-algebra, then \(S\) is Noetherian
         (Hilbert Basis Theorem).

   \item \textbf{Modules.}
         An \(R\)-module is called Noetherian if it satisfies the
         ascending-chain condition on submodules.  Over a Noetherian
         ring, submodules of finitely generated modules are again
         finitely generated.

   \item \textbf{Localisation.}
         Any localisation \(S^{-1}R\) of a Noetherian ring is
         Noetherian.

   \item \textbf{Dimension theory.}
         Noetherian rings admit Krull dimension; many finiteness
         results (e.g.\ primary decomposition) rely on Noetherianness.
\end{itemize}

\subsection*{3.  Classical examples}

\begin{enumerate}
   \item \(\mathbb{Z}\) \textbf{and} every PID (principal ideal domain)
         are Noetherian, because every ideal is generated by one
         element.

   \item Any \textbf{field} \(k\) is Noetherian
         (the only ideals are \(0\) and \(k\)).

   \item Polynomial rings in finitely many variables over a Noetherian
         ring:
         \[
            R \text{ Noetherian}\;\Longrightarrow\;
            R[x_1,\dots,x_n] \text{ Noetherian}
         \]
         by the Hilbert Basis Theorem.

   \item Finite rings such as \(\mathbb{Z}/n\mathbb{Z}\) are trivially
         Noetherian (there are only finitely many ideals).

   \item The coordinate ring of an affine algebraic variety
         \(k[x_1,\dots,x_n]/I\) is Noetherian whenever \(k\) is a field
         (again via Hilbert).
\end{enumerate}

\subsection*{4.  Non-examples}

\begin{enumerate}
   \item The polynomial ring in \emph{infinitely} many variables over a
         field:
         \[
            k[x_1,x_2,x_3,\dots]
         \]
         is \textbf{not} Noetherian.  
         Example ascending chain:
         \(
            (x_1)\subset(x_1,x_2)\subset(x_1,x_2,x_3)\subset\cdots.
         \)

   \item The ring of all real-valued continuous functions on
         \([0,1]\), \(C([0,1],\mathbb{R})\), fails to be Noetherian:
         ideals like \(\{f: f(0)=0\}\subset\{f: f([0,\tfrac12])=0\}
         \subset\cdots\) form a non-stationary ascending chain.

   \item The power-series ring \(k[[x_1,x_2,\dots]]\) in infinitely many
         variables (compare with the finitely generated case
         \(k[[x]]\), which \emph{is} Noetherian).
\end{enumerate}

\subsection*{5.  Why Noetherian rings matter}

\begin{itemize}
   \item They ensure \textbf{finiteness}: algorithms terminate, bases
         exist, and decompositions (primary, Gröbner, etc.) are finite.
   \item In algebraic geometry, coordinate rings of affine varieties
         are Noetherian, making it possible to translate geometric
         questions into algebraic ones.
   \item In commutative algebra, many deep theorems
         (Krull’s principal ideal theorem, Lasker–Noether primary
         decomposition, Cohen–Macaulay and regular local rings) assume
         Noetherianness as a starting point.
\end{itemize}

\bigskip
\noindent
\textbf{Take-away.}\;
A Noetherian ring is one in which “nothing requires infinitely many
generators’’.  This finiteness condition is the bedrock on which modern
commutative algebra and algebraic geometry are built.
\section*{Solvability by Radicals, Solvable Galois Groups, and the Quintic}

%%%%%%%%%%%%%%%%%%%%%%%%%%%%%%%%%%%%%%%%%%%%%%%%%%%%%%%%%%%%%%%%%%%%%%%%
\subsection*{1.  From radicals to groups: Galois’ criterion}

\begin{theorem}[Galois]
   Let \(f\in\Bbb Q[X]\) be a separable polynomial and let
   \(K_f\) be its splitting field over \(\Bbb Q\).
   Then
   \[
      \boxed{\;
         f\text{ is solvable by radicals }
         \Longleftrightarrow
         \operatorname{Gal}(K_f/\Bbb Q)\text{ is a \emph{solvable} group}
      \;}
   \]
\end{theorem}

\emph{Idea of the proof.}\;
Adjoining an \(n\)-th root corresponds to adjoining a root of the
cyclotomic polynomial \(X^n-1\), whose Galois group is abelian.
Building a radical tower therefore produces a chain of field extensions
whose successive Galois groups are abelian.  
Conversely, if the Galois group is solvable one can reverse this
process and peel off successive abelian quotients, realising each step
by extracting radicals.

%%%%%%%%%%%%%%%%%%%%%%%%%%%%%%%%%%%%%%%%%%%%%%%%%%%%%%%%%%%%%%%%%%%%%%%%
\subsection*{2.  What does “solvable group’’ mean?}

\begin{definition}
   A finite group \(G\) is \emph{solvable} if it admits a
   \textbf{composition series}
   \[
      1 = G_0 \;\lhd\; G_1 \;\lhd\;\dots\;\lhd\; G_r = G
   \]
   whose successive \emph{composition factors}
   \(G_{i}/G_{i-1}\) are all \textbf{abelian}.
   Equivalently, \(G\) is solvable iff its \emph{derived series}
   \(
      G\triangleright G^{(1)}\triangleright G^{(2)}\triangleright\cdots
   \)
   terminates in the trivial group, where
   \(G^{(i+1)}=[G^{(i)},G^{(i)}]\).
\end{definition}

\noindent
A non-abelian \emph{simple} group has no non-trivial normal subgroups
and therefore cannot occur as an abelian factor in \emph{any}
composition series.  The minimal obstruction to solvability is thus
the appearance of a non-abelian simple composition factor.

%%%%%%%%%%%%%%%%%%%%%%%%%%%%%%%%%%%%%%%%%%%%%%%%%%%%%%%%%%%%%%%%%%%%%%%%
\subsection*{3.  The symmetric groups}

\[
\begin{array}{|c|c|c|}
\hline
n & S_n\ \text{solvable?} & \text{Reason} \\ \hline
2 & \text{Yes} & S_2\cong C_2 \\ \hline
3 & \text{Yes} & 1\lhd A_3\cong C_3\lhd S_3,\ 
                 S_3/A_3\cong C_2 \\ \hline
4 & \text{Yes} & 1\lhd V_4\lhd A_4\lhd S_4,\ 
                 \text{all factors abelian} \\ \hline
5 & \textbf{No} & A_5\ \text{is simple and non-abelian} \\ \hline
n\ge6 & \textbf{No} & A_n\ \text{simple, non-abelian} \\ \hline
\end{array}
\]

For \(n\ge5\) the alternating group \(A_n\) is non-abelian and simple.
Because \(A_n\lhd S_n\) with abelian quotient \(S_n/A_n\cong C_2\),
every composition series of \(S_n\ (n\ge5)\) contains \(A_n\) as a
factor, hence \(S_n\) is \emph{not} solvable.

\medskip
\textbf{Composition series of \(S_5\).}
\[
   1 \;<\; C_2 \;<\; A_5 \;<\; S_5,
\qquad
\text{factors}\;=\;\{C_2,\ A_5\}.
\]
Because \(A_5\) is non-abelian, \(S_5\) fails to be solvable.

%%%%%%%%%%%%%%%%%%%%%%%%%%%%%%%%%%%%%%%%%%%%%%%%%%%%%%%%%%%%%%%%%%%%%%%%
\subsection*{4.  A concrete quintic with Galois group \(\boldsymbol{S_5}\)}

Take
\[
   f(X)\;=\;X^{5}-9X+3\;\in\;\Bbb Q[X].
\]
\begin{enumerate}
   \item \emph{Irreducible over \(\Bbb Q\).}\;
         Eisenstein with \(p=3\) after the shift \(X\mapsto X+1\), or
         via the rational-root test ($\pm1,\,\pm3$ fail).
   \item \emph{Exactly three real roots.}\;
         \(f\) is odd and \(f' = 5X^{4}-9\) has exactly two real roots
         \(\pm(9/5)^{1/4}\); an easy analysis shows the real graph
         crosses the \(x\)-axis three times.
   \item \emph{Discriminant non-square.}\;
         Compute \(\Delta(f)=-2^{4}\cdot3^{6}\cdot5^{5}\), which is not
         a square in \(\Bbb Q\).
\end{enumerate}

\noindent
\emph{Conclusion.}\;
The splitting field \(K_f\) is of degree \(120\) over \(\Bbb Q\) and
\[
   \operatorname{Gal}(K_f/\Bbb Q)\;\cong\;S_5.
\]
(The standard proof uses the \emph{transitive subgroup test}:
irreducibility \(\Rightarrow\) transitivity;  
\(3\) real roots \(\Rightarrow\) the Galois group contains an odd
permutation;  
non-square discriminant \(\Rightarrow\) it is not contained in \(A_5\);  
together with a \(5\)-cycle from reduction modulo a suitable prime, one
obtains the full \(S_5\).)

%%%%%%%%%%%%%%%%%%%%%%%%%%%%%%%%%%%%%%%%%%%%%%%%%%%%%%%%%%%%%%%%%%%%%%%%
\subsection*{5.  Why the quintic is (usually) unsolvable by radicals}

Because \(S_5\) is \emph{not} solvable, any quintic whose Galois group
is (isomorphic to) \(S_5\) fails Galois’ solvability criterion.
Therefore there exists \emph{no} formula involving only field
operations and radical extractions that expresses its roots in terms of
the coefficients.

\paragraph{Historical note.}
Galois’ revolutionary insight (1830) was precisely this translation:

\[
   \text{``Formula with radicals''}\;
   \Longleftrightarrow\;
   \text{``Galois group solvable''}.
\]

The existence of a non-abelian simple composition factor in \(S_5\)
provided the first rigorous proof that \emph{no quintic analogue of the
quadratic, cubic, or quartic formula can exist in general}.

%%%%%%%%%%%%%%%%%%%%%%%%%%%%%%%%%%%%%%%%%%%%%%%%%%%%%%%%%%%%%%%%%%%%%%%%
\subsection*{6.  Why degrees \(\boldsymbol{\le4}\) pose no problem}

For \(n=2,3,4\) the full symmetric group \(S_n\) is solvable; more
precisely,
\[
   S_2\cong C_2,\quad
   S_3\cong D_3,\quad
   S_4 \text{ has the chain }1<V_4<A_4<S_4.
\]
Hence \emph{every} polynomial of degree \(\le4\) has a solvable Galois
group (a subgroup of a solvable group is solvable) and is therefore
solvable by radicals—exactly the classical formulas you meet in
algebra.

\bigskip
\noindent
\textbf{Take-away.}\;
The quintic barrier is not about ``degree $5$'' per se; it is about the
emergence of the first non-abelian simple group \(A_5\) inside the
symmetric group.  Once such a factor appears, radical formulas are
doomed to fail.
%%%%%%%%%%%%%%%%%%%%%%%%%%%%%%%%%%%%%%%%%%%%%%%%%%%%%%%%%%%%%%%%%%%%%%%%
\subsection*{What the bullet really says (and why it matters)}

\paragraph{Set--up.}
You have a tower of fields
\[
   K_0 \;\subseteq\; K_1 \;\subseteq\; K_2 \;\subseteq\; \dots \;\subseteq\; K_m
\]
with $K_0 = F$ (typically $F=\Bbb Q$).  
The bullet states that for each intermediate step $0\le i<m$ we pass
from $K_i$ to $K_{i+1}$ by adjoining a single element
\[
   \alpha_i
   \quad\text{where}\quad
   \alpha_i^{\,n_i}=b_i\in K_i^{\phantom{|}}.
\]
Formally,
\[
   K_{i+1} \;=\; K_i(\alpha_i) 
               \;\cong\;
               K_i[X]\Big/\bigl(X^{n_i}-b_i\bigr).
\]

\paragraph{Interpretation.}
Each extension $K_{i+1}/K_i$ is generated by \emph{extracting an
$n_i$-th root} of some element $b_i\in K_i$.
Such an extension is called a \textbf{radical extension},
and a tower built entirely from such steps is a \emph{radical tower}.
If \emph{every} root of your original polynomial lies in the top field
$K_m$, we say that the polynomial is \textbf{solvable by radicals}.

\paragraph{Why the polynomial $x^{n_i}-b_i$.}
Choosing a minimal polynomial of the special form
\(
   X^{n_i}-b_i
\)
guarantees two things:

\begin{enumerate}
   \item \emph{Conceptual clarity.}
         We are literally “taking an $n_i$-th root’’; no more complicated
         algebraic relation is introduced at this step.
   \item \emph{Group--theoretic consequence.}
         The Galois group of $X^{n_i}-b_i$ (after adjoining the
         necessary $n_i$-th roots of unity) is \textbf{abelian}
         (isomorphic to a subgroup of the semidirect product
         $C_{n_i}\rtimes C_{\varphi(n_i)}$).  
         Hence each step in the tower contributes an \emph{abelian
         factor} to the global Galois group, a cornerstone of the
         “solvable $\Leftrightarrow$ solvable-by-radicals’’ theorem.
\end{enumerate}

\paragraph{Typical shape of a radical tower.}
\[
   \begin{aligned}
      K_0 &= \Bbb Q, \\[4pt]
      K_1 &= K_0\bigl(\sqrt[n_0]{b_0}\bigr), \\[4pt]
      K_2 &= K_1\bigl(\sqrt[n_1]{b_1}\bigr), \\[4pt]
      &\;\;\vdots \\[4pt]
      K_m &= K_{m-1}\bigl(\sqrt[n_{m-1}]{\,b_{m-1}}\bigr).
   \end{aligned}
\]
Often one also adjoins the relevant roots of unity to ensure
\emph{normality}, producing a \emph{Galois} radical tower:
\[
   K_i^{\;\prime} = K_i\bigl(\zeta_{n_i}\bigr)
   \quad\subseteq\quad
   K_{i+1}^{\;\prime}
             =K_{i}^{\;\prime}\bigl(\sqrt[n_i]{b_i}\bigr).
\]

\paragraph{Concrete example.}
Solve $X^3-2=0$ by radicals.

\[
   \begin{aligned}
      K_0 &= \Bbb Q, \\[2pt]
      K_1 &= \Bbb Q\!\bigl(\sqrt[3]{2}\bigr) 
            \quad\bigl(\text{adjoin root of }X^3-2\bigr), \\[2pt]
      K_2 &= K_1\!\bigl(\zeta_3\bigr)
            \quad\bigl(\text{adjoin primitive $3$rd root of unity
                      so the extension becomes Galois}\bigr).
   \end{aligned}
\]
Here $m=2$, $n_0=3$, $b_0=2$, and $n_1=1$ (trivial step for $\zeta_3$ if
we choose to view it as a $1$-st root of itself).  
The Galois group
$\operatorname{Gal}(K_2/\Bbb Q)\cong S_3$ has a composition series
$1\lhd C_3\lhd S_3$ with abelian factors, confirming solvability by
radicals.

\paragraph{Big picture.}
Building a tower with steps of the form
\(
   K_{i+1}=K_i\bigl(\sqrt[n_i]{b_i}\bigr)
\)
is precisely how one “peels off’’ the derived series of a solvable
Galois group.  
Conversely, if you \emph{cannot} arrange such a tower (because the
Galois group contains a non-abelian simple factor like $A_5$), the
polynomial is \emph{not} solvable by radicals.
%%%%%%%%%%%%%%%%%%%%%%%%%%%%%%%%%%%%%%%%%%%%%%%%%%%%%%%%%%%%%%%%%%%%%%%%
%%%%%%%%%%%%%%%%%%%%%%%%%%%%%%%%%%%%%%%%%%%%%%%%%%%%%%%%%%%%%%%%%%%%%%%%
\subsection*{A closer look at the tower for $X^{3}-2=0$}

Recall the radical tower we wrote:

\[
   K_0=\Bbb Q
   \;\subset\;
   K_1=\Bbb Q\!\bigl(\sqrt[3]{2}\bigr)
   \;\subset\;
   K_2=K_1(\zeta_3)
        =\Bbb Q\!\bigl(\sqrt[3]{2},\zeta_3\bigr),
   \qquad
   \zeta_3=e^{2\pi i/3}.
\]

\begin{itemize}
   \item $K_1/K_0$ is generated by a \emph{single} radical step
         $\sqrt[3]{2}$, so $[K_1:\Bbb Q]=3$.
   \item $K_2/K_1$ is obtained by adjoining $\zeta_3$, which is a root
         of $X-\,\zeta_3\in K_1[X]$ (formally a “$1^{\text{st}}$‐root’’
         step).  
         The real reason for adjoining $\zeta_3$ is to make the top
         field \emph{normal} over $\Bbb Q$, i.e.\ to create the full
         \textbf{splitting field}.
   \item $[K_2:K_1]=2$, so $[K_2:\Bbb Q]=3\cdot2=6$.
\end{itemize}

%%%%%%%%%%%%%%%%%%%%%%%%%%%%%%%%%%%%%%%%%%%%%%%%%%%%%%%%%%%%%%%%%%%%%%%%
\subsection*{The Galois group and its generators}

Label the three roots of $X^{3}-2$ in $K_2$ by
\[
   r_1=\sqrt[3]{2},\quad
   r_2=\zeta_3\sqrt[3]{2},\quad
   r_3=\zeta_3^{\,2}\sqrt[3]{2}.
\]

\begin{align*}
   \sigma :\;
      &r_1\mapsto r_2,\;
        r_2\mapsto r_3,\;
        r_3\mapsto r_1,
      &&
      (\text{$3$-cycle}) \\[4pt]
   \tau :\;
      &r_1\mapsto r_1,\;
        r_2\mapsto r_3,\;
        r_3\mapsto r_2,
      &&
      (\text{transposition, complex conjugation}).
\end{align*}

These automorphisms satisfy  
$\sigma^{3}=1,\;\tau^{2}=1,\;\tau\sigma\tau=\sigma^{-1}$,  
so
\[
   G\;=\;\operatorname{Gal}(K_2/\Bbb Q)
   \;\cong\;S_3.
\]

%%%%%%%%%%%%%%%%%%%%%%%%%%%%%%%%%%%%%%%%%%%%%%%%%%%%%%%%%%%%%%%%%%%%%%%%
\subsection*{The subgroup $C_3=\langle\sigma\rangle$}

\[
   C_3=\{1,\sigma,\sigma^{2}\}
   \quad\text{(order $3$ cyclic)}.
\]

\paragraph{Why is $C_3\lhd S_3$?}

\begin{itemize}
   \item \emph{Group–theoretic reason:}  
         $C_3$ has index $2$ in $S_3$; any subgroup of index $2$ is
         automatically normal because left and right cosets coincide.
   \item \emph{Galois–theoretic translation:}  
         Normality means that for every $\gamma\in G$
         we have $\gamma C_3\gamma^{-1}=C_3$.  
         Equivalently, the fixed field of $C_3$
         \[
             L := K_2^{C_3}
                \;=\;
                \{\,x\in K_2 : \sigma x = x\,\}
         \]
         is \emph{stable} under the whole Galois group; hence $L/\Bbb Q$
         is itself a normal (Galois) extension.
\end{itemize}

%%%%%%%%%%%%%%%%%%%%%%%%%%%%%%%%%%%%%%%%%%%%%%%%%%%%%%%%%%%%%%%%%%%%%%%%
\subsection*{Field diagram and correspondence}

\[
\begin{tikzcd}[row sep=small,column sep=large]
           & K_2=\Bbb Q(\sqrt[3]{2},\zeta_3) \arrow[-]{dl}[swap]{C_2}
                                             \arrow[-]{dr}{C_3}
           & \\
  L=K_2^{C_3}=\Bbb Q(\zeta_3)
           &                                     &
  K_1=\Bbb Q(\sqrt[3]{2}) 
           \\
           & \Bbb Q \arrow[-]{ul}[swap]{S_3/C_3\cong C_2}
                      \arrow[-]{ur}{S_3/C_2\cong S_3}
           &
\end{tikzcd}
\]

*   The vertical arrows are field extensions;  
    the labels give the corresponding quotient groups via the
    \emph{Fundamental Theorem of Galois Theory}.  
*   Because $C_3\lhd S_3$, the fixed field $L$ is Galois over $\Bbb Q$
    with Galois group $S_3/C_3\cong C_2$ (a quadratic extension).
*   The top extension $K_2/L$ has Galois group $C_3$ and degree $3$.

%%%%%%%%%%%%%%%%%%%%%%%%%%%%%%%%%%%%%%%%%%%%%%%%%%%%%%%%%%%%%%%%%%%%%%%%
\subsection*{Composition series and solvability}

\[
   1 \;\lhd\; C_3 \;\lhd\; S_3,
   \qquad
   S_3/C_3 \cong C_2,\;
   C_3/1 \cong C_3.
\]

\begin{itemize}
   \item Each factor (\(C_3\) and \(C_2\)) is \emph{abelian}, so
         \(S_3\) is a \textbf{solvable} group.
   \item This mirrors the radical tower:
         \begin{center}
            \(\displaystyle
              \underbrace{K_2/L}_{\text{degree $3$}}
              \quad\text{and}\quad
              \underbrace{L/\Bbb Q}_{\text{degree $2$}}
            \)
         \end{center}
         each obtained by adjoining a single radical 
         ($\sqrt[3]{2}$ and $\sqrt{-3}$ respectively).
\end{itemize}

%%%%%%%%%%%%%%%%%%%%%%%%%%%%%%%%%%%%%%%%%%%%%%%%%%%%%%%%%%%%%%%%%%%%%%%%
\paragraph{In everyday words.}
Saying “$C_3$ is a \emph{normal} subgroup of $S_3$’’ in this context is
equivalent to saying that the quadratic subfield
$L=\Bbb Q(\zeta_3)=\Bbb Q\!\bigl(\sqrt{-3}\bigr)$
is \emph{fixed} by the entire Galois group and is thus itself a Galois
extension of $\Bbb Q$.  
That normal subgroup (and the abelian quotient $C_2$) provides exactly
the abelian “building block’’ needed in the composition series, matching
the fact that the cubic can be solved by successively adjoining radicals.
%%%%%%%%%%%%%%%%%%%%%%%%%%%%%%%%%%%%%%%%%%%%%%%%%%%%%%%%%%%%%%%%%%%%%%%%
\begin{align}
  &\textbf{Set--up:}\qquad 
     K_1 \;=\; \mathbb{Q}\!\bigl(\sqrt[3]{2}\bigr), 
     \qquad \alpha \;:=\; \sqrt[3]{2}.
  \\[6pt]
  &\textbf{Goal:}\qquad 
     \bigl[K_1:\mathbb{Q}\bigr] \;=\; ?
  \end{align}
  
  \bigskip
  \hrule
  \bigskip
  
  \section*{1.  Field\,–\,extension degree and minimal polynomial}
  
  The extension degree is the vector–space dimension
  \[
     [K_1:\mathbb{Q}] 
     \;=\; \dim_{\mathbb{Q}}\bigl(K_1\bigr).
  \]
  For any algebraic element $\alpha$, that dimension equals the degree of
  its \emph{minimal polynomial} $\mu_{\alpha,\mathbb{Q}}(X)$:
  \[
     [\,\mathbb{Q}(\alpha):\mathbb{Q}\,] 
     \;=\; \deg\!\bigl(\mu_{\alpha,\mathbb{Q}}\bigr).
  \]
  Hence we must identify the minimal polynomial of
  $\alpha=\sqrt[3]{2}$ over~$\mathbb{Q}$.
  
  \bigskip
  \section*{2.  The polynomial $X^{3}-2$ is irreducible}
  
  \[
     f(X)\;:=\;X^{3}-2 \;\in\; \mathbb{Q}[X].
  \]
  \begin{itemize}
     \item \emph{Why $f(\alpha)=0$:}\;
           Clearly $\alpha^{3}-2=0$ by construction.
     \item \emph{Why $f$ is irreducible over $\mathbb{Q}$:}
           \begin{enumerate}
              \item \textbf{Rational-root test.}\;
                    Any rational root must divide the constant term $\pm2$,
                    so candidates are $\pm1,\pm2$.
                    None of them satisfies $x^{3}-2=0$.
              \item \textbf{Eisenstein (optional).}\;
                    After the linear shift $X\mapsto X+1$,
                    \[
                       f(X+1) = X^{3}+3X^{2}+3X-1,
                    \]
                    which is Eisenstein with prime $2$; thus $f$ is
                    irreducible, and so is $X^{3}-2$.
           \end{enumerate}
  \end{itemize}
  
  Because $f$ is monic and irreducible, it \emph{is} the minimal
  polynomial of $\alpha$ over $\mathbb{Q}$.
  
  \bigskip
  \section*{3.  Compute the extension degree}
  
  \[
     \deg\bigl(\mu_{\alpha,\mathbb{Q}}\bigr)
     \;=\;\deg(X^{3}-2)
     \;=\;3
     \;\Longrightarrow\;
     [K_1:\mathbb{Q}]
     \;=\;3.
  \]
  
  \bigskip
  \hrule
  \bigskip
  
  \section*{4.  Explicit $\mathbb{Q}$–basis}
  
  A convenient $\mathbb{Q}$-basis of $K_1$ is
  \[
     \bigl\{\,1,\;\alpha,\;\alpha^{2}\,\bigr\}.
  \]
  
  \bigskip
  \noindent
  \textbf{Final answer:}\;
  \[
     \boxed{\; [K_1:\mathbb{Q}] = 3 \;}
  \]
  %%%%%%%%%%%%%%%%%%%%%%%%%%%%%%%%%%%%%%%%%%%%%%%%%%%%%%%%%%%%%%%%%%%%%%%%
\subsection*{What does it mean for \(G_i/G_{i-1}\) to be \emph{abelian}?}

\begin{definition}[Abelian group]
A group \(H\) is \emph{abelian} if its operation is commutative:
\[
   \forall\,x,y\in H \qquad xy = yx .
\]
\end{definition}

Now suppose \(G_{i-1}\lhd G_i\) are groups.  
The \textbf{quotient group} \(G_i/G_{i-1}\) consists of the left cosets
\(\,gG_{i-1}\,\) with the operation
\[
   (gG_{i-1})(hG_{i-1}) = (gh)G_{i-1}.
\]

The quotient \(G_i/G_{i-1}\) is \emph{abelian} iff
\[
   (gG_{i-1})(hG_{i-1})=(hG_{i-1})(gG_{i-1})
   \quad\Longleftrightarrow\quad
   gh\,G_{i-1}=hg\,G_{i-1}.
\]
Equivalently, every commutator \([g,h]=g^{-1}h^{-1}gh\) already lies in
\(G_{i-1}\), so the “commutativity defect’’ is killed in the quotient.

%%%%%%%%%%%%%%%%%%%%%%%%%%%%%%%%%%%%%%%%%%%%%%%%%%%%%%%%%%%%%%%%%%%%%%%%
\subsection*{A concrete example: the solvable group \(S_3\)}

Consider the symmetric group
\[
   G_2=S_3
   \;\supset\;
   G_1=A_3=\{\,\text{even permutations}\,\}
   \;\supset\;
   G_0=\{1\}.
\]

\[
\begin{array}{|c|c|c|}
\hline
\text{Step} & \text{Quotient} & \text{Is it abelian?} \\ \hline
G_2/G_1 & S_3/A_3\;\cong\;C_2 & \text{Yes (cyclic of order $2$)} \\ \hline
G_1/G_0 & A_3/1\;\cong\;C_3 & \text{Yes (cyclic of order $3$)} \\ \hline
\end{array}
\]

* \(C_2\) and \(C_3\) are cyclic, hence abelian.
* Thus each factor \(G_i/G_{i-1}\) in the chain is abelian, making
  \(S_3\) a \emph{solvable} group.

\paragraph{Explicit verification for the top quotient.}
Because \(A_3\lhd S_3\) and \([S_3:A_3]=2\), the quotient
\(S_3/A_3=\{\;A_3,\;(12)A_3\;\}\) has only two elements and therefore
must be abelian (any group of order \(2\) is automatically commutative).

%%%%%%%%%%%%%%%%%%%%%%%%%%%%%%%%%%%%%%%%%%%%%%%%%%%%%%%%%%%%%%%%%%%%%%%%
\subsection*{Another easy illustration: a cyclic chain inside \(\Bbb Z/8\Bbb Z\)}

Take
\[
   G_2=\Bbb Z/8\Bbb Z,\qquad
   G_1=4\Bbb Z/8\Bbb Z\cong C_2,\qquad
   G_0=\{0\}.
\]

\[
\begin{array}{|c|c|}
\hline
\text{Quotient} & \text{Structure} \\ \hline
G_2/G_1 & (\Bbb Z/8)/(\,4\Bbb Z/8) \;\cong\; C_4 \\ \hline
G_1/G_0 & C_2 \\ \hline
\end{array}
\]

Both \(C_4\) and \(C_2\) are cyclic, hence abelian.  
So again \(G_i/G_{i-1}\) are all abelian.

%%%%%%%%%%%%%%%%%%%%%%%%%%%%%%%%%%%%%%%%%%%%%%%%%%%%%%%%%%%%%%%%%%%%%%%%
\subsection*{Take-away}

Saying \emph{“\(G_i/G_{i-1}\) is abelian’’} simply means that, after
modding out by \(G_{i-1}\), the remaining cosets commute with each
other.  Concrete chains like
\(
   1\lhd A_3\lhd S_3
\)
or
\(
   0\lhd 4\Bbb Z/8\Bbb Z\lhd\Bbb Z/8\Bbb Z
\)
illustrate how each successive quotient can indeed be an abelian group.
\end{document}
