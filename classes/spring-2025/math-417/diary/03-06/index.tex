\documentclass[12pt]{article}

% Packages
\usepackage[margin=1in]{geometry}
\usepackage{amsmath,amssymb,amsthm}
\usepackage{enumitem}
\usepackage{hyperref}
\usepackage{xcolor}
\usepackage{import}
\usepackage{xifthen}
\usepackage{pdfpages}
\usepackage{transparent}
\usepackage{listings}


\lstset{
    breaklines=true,         % Enable line wrapping
    breakatwhitespace=false, % Wrap lines even if there's no whitespace
    basicstyle=\ttfamily,    % Use monospaced font
    frame=single,            % Add a frame around the code
    columns=fullflexible,    % Better handling of variable-width fonts
}

\newcommand{\incfig}[1]{%
    \def\svgwidth{\columnwidth}
    \import{./Figures/}{#1.pdf_tex}
}
\theoremstyle{definition} % This style uses normal (non-italicized) text
\newtheorem{solution}{Solution}
\newtheorem{proposition}{Proposition}
\newtheorem{problem}{Problem}
\newtheorem{lemma}{Lemma}
\newtheorem{theorem}{Theorem}
\newtheorem{remark}{Remark}
\newtheorem{note}{Note}
\newtheorem{definition}{Definition}
\newtheorem{example}{Example}
\theoremstyle{plain} % Restore the default style for other theorem environments
%

% Theorem-like environments
% Title information
\title{MATH 417: HW 5}
\author{Jerich Lee}
\date{\today}

\begin{document}

\maketitle
\begin{problem}[]
    
\end{problem}
\begin{solution}
 \textbf{Solution to: How many (pairwise non-isomorphic) abelian groups of order 
\ (a)\ 15,\ (b)\ 16,\ (c)\ 17,\ (d)\ 18\ are there?}

\bigskip

\noindent
\textbf{Step 1. Factor each order into prime powers.}
\[
\begin{aligned}
15 &= 3^1 \cdot 5^1, \\
16 &= 2^4, \\
17 &= 17^1, \\
18 &= 2^1 \cdot 3^2.
\end{aligned}
\]

\noindent
\textbf{Step 2. Use the Fundamental Theorem of Finite Abelian Groups.}\\
Any finite abelian group of order $n$ decomposes as a direct product of its $p$-primary (Sylow) components.  Concretely, if
\[
n \;=\; p_1^{\alpha_1}\,\cdot\,p_2^{\alpha_2}\,\cdots\,p_k^{\alpha_k},
\]
then an abelian group of order $n$ is isomorphic to a direct product
\[
G \;\cong\; G_{p_1}\;\times\;\cdots\;\times\;G_{p_k},
\]
where $G_{p_i}$ is an abelian $p_i$-group of order $p_i^{\alpha_i}$.

\bigskip

\noindent
\textbf{Step 3. Classification of abelian groups of prime-power order.}\\
For a prime $p$ and a positive integer $m$, the number of \emph{distinct} abelian groups of order $p^m$ (up to isomorphism) equals the number of partitions of $m$.  Equivalently:
\[
\text{Number of abelian groups of order }p^m \;=\; \text{number of ways to write }m \text{ as an ordered sum of positive integers.}
\]

\bigskip

\noindent
\textbf{Step 4. Combine the $p$-primary components.}\\
If $n = p_1^{\alpha_1}\,\cdots\,p_k^{\alpha_k}$ and $a_i$ is the number of distinct abelian groups of order $p_i^{\alpha_i}$, then the total number of abelian groups of order $n$ is
\[
a_1 \;\times\; a_2 \;\times\;\cdots\;\times\; a_k,
\]
since each Sylow component is chosen independently and then combined by direct product.

\bigskip

\noindent
\textbf{Step 5. Apply this to each specific order.}

\begin{enumerate}
\item[(a)] \textbf{Order 15 = $3^1 \cdot 5^1$.}

Each prime power is $3^1$ and $5^1$.  The number of abelian groups of order $3^1$ is the number of partitions of $1$, which is $1$.  Similarly, for order $5^1$ there is also only $1$ possibility.  
\[
\text{Total number} = 1 \;\times\; 1 = 1.
\]
Hence there is exactly one abelian group of order 15 (namely $\mathbb{Z}_{15}$).

\bigskip

\item[(b)] \textbf{Order 16 = $2^4$.}

We count the number of partitions of $4$:
\[
4,\quad 3 + 1,\quad 2 + 2,\quad 2 + 1 + 1,\quad 1 + 1 + 1 + 1.
\]
There are $5$ partitions, so there are $5$ non-isomorphic abelian groups of order $16$.  Concretely, they can be listed as
\[
\mathbb{Z}_{16}, \quad
\mathbb{Z}_8 \times \mathbb{Z}_2, \quad
\mathbb{Z}_4 \times \mathbb{Z}_4, \quad
\mathbb{Z}_4 \times \mathbb{Z}_2 \times \mathbb{Z}_2, \quad
\mathbb{Z}_2 \times \mathbb{Z}_2 \times \mathbb{Z}_2 \times \mathbb{Z}_2.
\]
Thus the answer for (b) is $5$.

\bigskip

\item[(c)] \textbf{Order 17 = $17^1$.}

Since $17$ is prime, the only abelian group of that order is the cyclic group $\mathbb{Z}_{17}$.  Thus there is exactly $1$ abelian group of order $17$.

\bigskip

\item[(d)] \textbf{Order 18 = $2^1 \cdot 3^2$.}

The number of abelian groups of order $2^1$ is the number of partitions of $1$, which is $1$.  
The number of abelian groups of order $3^2$ is the number of partitions of $2$, which is $2$ (namely $2$ and $1+1$).  Corresponding groups for order $9$ are:
\[
\mathbb{Z}_9 \quad \text{and} \quad \mathbb{Z}_3 \times \mathbb{Z}_3.
\]
So the total number of abelian groups of order $18$ is
\[
1 \;\times\; 2 = 2.
\]
These can be realized as:
\[
\mathbb{Z}_2 \;\times\; \mathbb{Z}_9 \;\cong\; \mathbb{Z}_{18}, 
\quad\text{and}\quad
\mathbb{Z}_2 \;\times\; (\mathbb{Z}_3 \times \mathbb{Z}_3).
\]

\end{enumerate}

\bigskip

\noindent
\textbf{Final Answers:}
\[
\text{(a) }1, \quad
\text{(b) }5, \quad
\text{(c) }1, \quad
\text{(d) }2.
\]   
\end{solution}
\begin{problem}[]
    
\end{problem}
\begin{solution}
    \textbf{Solution to Problem 2: Decomposition into disjoint cycles}

    \bigskip
    
    All permutations are multiplied from right to left. For each part, we track each element 
    through the product of cycles.
    
    \bigskip
    
    \noindent
    \textbf{(a) }$(12)(12345)$
    
    \medskip
    
    \noindent
    Apply $(12345)$ first, then $(12)$.  We see:
    \[
    \begin{aligned}
    1 &\xrightarrow{(12345)} 2 \;\xrightarrow{(12)} 1,\\
    2 &\xrightarrow{(12345)} 3 \;\xrightarrow{(12)} 3,\\
    3 &\xrightarrow{(12345)} 4 \;\xrightarrow{(12)} 4,\\
    4 &\xrightarrow{(12345)} 5 \;\xrightarrow{(12)} 5,\\
    5 &\xrightarrow{(12345)} 1 \;\xrightarrow{(12)} 2.
    \end{aligned}
    \]
    Hence the net permutation sends
    \[
    1 \to 1,\quad 2 \to 3,\quad 3 \to 4,\quad 4 \to 5,\quad 5 \to 2.
    \]
    In disjoint cycles, this is $(2\,3\,4\,5)$ (and $1$ is fixed).  
    
    \bigskip
    
    \noindent
    \textbf{(b) }$(14)(12345)$
    
    \medskip
    
    \noindent
    Again, apply $(12345)$ first, then $(14)$:
    \[
    \begin{aligned}
    1 &\xrightarrow{(12345)} 2 \;\xrightarrow{(14)} 2,\\
    2 &\xrightarrow{(12345)} 3 \;\xrightarrow{(14)} 3,\\
    3 &\xrightarrow{(12345)} 4 \;\xrightarrow{(14)} 1,\\
    4 &\xrightarrow{(12345)} 5 \;\xrightarrow{(14)} 5,\\
    5 &\xrightarrow{(12345)} 1 \;\xrightarrow{(14)} 4.
    \end{aligned}
    \]
    Thus the net permutation sends
    \[
    1 \to 2,\quad 2 \to 3,\quad 3 \to 1,\quad 4 \to 5,\quad 5 \to 4.
    \]
    In disjoint cycle form, this is $(1\,2\,3)(4\,5)$.
    
    \bigskip
    
    \noindent
    \textbf{(c) }$(12)(13)(14)$
    
    \medskip
    
    \noindent
    Here we apply $(14)$, then $(13)$, then $(12)$.  Track each number in $\{1,2,3,4\}$:
    
    \[
    \begin{aligned}
    1 &\xrightarrow{(14)} 4 \;\xrightarrow{(13)} 4 \;\xrightarrow{(12)} 4,\\
    4 &\xrightarrow{(14)} 1 \;\xrightarrow{(13)} 3 \;\xrightarrow{(12)} 3,\\
    3 &\xrightarrow{(14)} 3 \;\xrightarrow{(13)} 1 \;\xrightarrow{(12)} 2,\\
    2 &\xrightarrow{(14)} 2 \;\xrightarrow{(13)} 2 \;\xrightarrow{(12)} 1.
    \end{aligned}
    \]
    Hence
    \[
    1 \to 4,\quad 4 \to 3,\quad 3 \to 2,\quad 2 \to 1.
    \]
    That is the 4-cycle $(1\,4\,3\,2)$.
    
    \bigskip
    
    \noindent
    \textbf{(d) }$(13)(1234)(13)$
    
    \medskip
    
    \noindent
    First apply the rightmost $(13)$, then $(1234)$, then the leftmost $(13)$.  We get:
    \[
    \begin{aligned}
    1 &\xrightarrow{(13)} 3 \;\xrightarrow{(1234)} 4 \;\xrightarrow{(13)} 4,\\
    4 &\xrightarrow{(13)} 4 \;\xrightarrow{(1234)} 1 \;\xrightarrow{(13)} 3,\\
    3 &\xrightarrow{(13)} 1 \;\xrightarrow{(1234)} 2 \;\xrightarrow{(13)} 2,\\
    2 &\xrightarrow{(13)} 2 \;\xrightarrow{(1234)} 3 \;\xrightarrow{(13)} 1.
    \end{aligned}
    \]
    So
    \[
    1 \to 4,\quad 4 \to 3,\quad 3 \to 2,\quad 2 \to 1,
    \]
    again yielding the 4-cycle $(1\,4\,3\,2)$.
    
    \bigskip
    
    \noindent
    \textbf{Final answers:}
    \[
    \begin{aligned}
    &(a)\ (12)(12345) \;=\; (2\,3\,4\,5),\\
    &(b)\ (14)(12345) \;=\; (1\,2\,3)(4\,5),\\
    &(c)\ (12)(13)(14) \;=\; (1\,4\,3\,2),\\
    &(d)\ (13)(1234)(13) \;=\; (1\,4\,3\,2).
    \end{aligned}
    \]
     
\end{solution}

\begin{problem}[]
    
\end{problem}
\begin{solution}
    \textbf{Solution to Problem 3: All elements of $A_4$ and their orders}

    \bigskip
    
    \noindent
    \textbf{1.\ Definition and size of $A_4$.}\\
    The group $A_4$ is the subgroup of the symmetric group $S_4$ consisting of all even permutations on the set $\{1,2,3,4\}$.  It has $12$ elements in total.
    
    \bigskip
    
    \noindent
    \textbf{2.\ Types of even permutations in $S_4$.}\\
    An even permutation can be:
    \begin{itemize}
    \item the identity permutation (no transpositions),
    \item a product of two disjoint transpositions (since each transposition is odd, the product of two of them is even),
    \item or a 3-cycle (which can be written as two transpositions).
    \end{itemize}
    In $S_4$:
    \begin{itemize}
    \item There is exactly $1$ identity permutation.
    \item There are $3$ distinct elements that are products of two disjoint transpositions in $S_4$:
    \[
    (12)(34),\quad (13)(24),\quad (14)(23).
    \]
    All of these are even, hence in $A_4$.
    \item There are $8$ distinct 3-cycles in $S_4$, and all 3-cycles are even permutations, hence they all lie in $A_4$.  These are:
    \[
    (123),\ (132),\ (124),\ (142),\ (134),\ (143),\ (234),\ (243).
    \]
    \end{itemize}
    
    \bigskip
    
    \noindent
    \textbf{3.\ Listing all elements of $A_4$.}\\
    Thus we can list \emph{all} $12$ elements of $A_4$ as follows:
    \[
    \begin{aligned}
    &A_4 = \{\, e,\ (12)(34),\ (13)(24),\ (14)(23),\\
    &\qquad (123),\ (132),\ (124),\ (142),\ (134),\ (143),\ (234),\ (243)\}.
    \end{aligned}
    \]
    
    \bigskip
    
    \noindent
    \textbf{4.\ Orders of the elements.}
    \begin{itemize}
    \item The \emph{identity} $e$ has order $1$.
    \item Each product of two disjoint transpositions, e.g.\ $(12)(34)$, is its own inverse and therefore has order $2$.
    \item Each 3-cycle (e.g.\ $(123)$) has order $3$, since $(abc)^3$ is the identity in $S_4$.
    \end{itemize}
    
    \noindent
    Hence we get the following classification of elements by order:
    \[
    \begin{aligned}
    \text{Order }1 &: \quad e,\\
    \text{Order }2 &: \quad (12)(34),\ (13)(24),\ (14)(23),\\
    \text{Order }3 &: \quad (123),\ (132),\ (124),\ (142),\ (134),\ (143),\ (234),\ (243).
    \end{aligned}
    \]
    
    \bigskip
    
    \noindent
    \textbf{Final Answer.}\\
    All $12$ elements of $A_4$ are as listed above; there is $1$ element of order $1$, $3$ elements of 
    order $2$, and $8$ elements of order $3$.
     
\end{solution}

\begin{problem}[]
    
\end{problem}
\begin{solution}
    \textbf{Solution to Problem 4}

    \bigskip
    
    Throughout, let $D_n$ denote the dihedral group of order $2n$, that is, the group of symmetries
    of a regular $n$-gon.  Hence $D_3$ has $6$ elements, $D_6$ has $12$ elements, and $D_{12}$ has $24$ elements.
    
    \bigskip
    
    \noindent
    \textbf{(a) $D_3 \cong S_3$.}
    
    \medskip
    
    \noindent
    An equilateral triangle has $3$ vertices.  A symmetry of the triangle is determined by how it permutes
    these vertices.  Thus there is a natural map
    \[
    \phi: D_3 \;\longrightarrow\; S_3
    \]
    taking each symmetry to the corresponding permutation of the triangle's vertices.
    It is straightforward to check that this map is a group homomorphism (it respects composition of symmetries),
    is injective (different symmetries act differently on vertices),
    and is surjective (every permutation of the $3$ vertices can be realized by some symmetry).
    Hence $\phi$ is a bijective homomorphism, i.e.\ an isomorphism.  Therefore $D_3 \cong S_3$.
    
    \bigskip
    
    \noindent
    \textbf{(b) $D_{12}$ is not isomorphic to $S_4$.}
    
    \medskip
    
    \noindent
    Both $D_{12}$ and $S_4$ have order $24$, so one might suspect an isomorphism.  However, an easy invariant
    is the \emph{element orders} in each group.  In $D_{12}$, the rotation by $\tfrac{360^\circ}{12} = 30^\circ$
    has order $12$.  Hence there is an element of order $12$ in $D_{12}$.
    
    By contrast, in $S_4$, the maximum possible order of any permutation is $4$ (for example, a $4$-cycle).
    There are no elements of order $12$ in $S_4$.  Because the two groups do not share the same element orders,
    they cannot be isomorphic.
    
    \bigskip
    
    \noindent
    \textbf{(c) $D_{6}$ is not isomorphic to $A_4$.}
    
    \medskip
    
    \noindent
    Both $D_{6}$ and $A_4$ have order $12$, so again one checks other invariants.
    In $D_{6}$, the rotation by $\tfrac{360^\circ}{6} = 60^\circ$ has order $6$.
    Hence $D_{6}$ contains an element of order $6$.
    
    In $A_4$, all non-identity elements have order $2$ (the product of two disjoint transpositions) or
    order $3$ (a $3$-cycle).  There is \emph{no} element of order $6$ in $A_4$.  Hence $D_{6}$ cannot be
    isomorphic to $A_4$.
    
    \bigskip
    
    \noindent
    \textbf{(d) $D_3 \times \mathbb{Z}_2$ is not isomorphic to $A_4$.}
    
    \medskip
    
    \noindent
    Observe that $D_3 \cong S_3$, so
    \[
    D_3 \times \mathbb{Z}_2 \;\cong\; S_3 \times \mathbb{Z}_2,
    \]
    which has $6 \times 2 = 12$ elements.  Again, compare element orders:
    \begin{itemize}
    \item In $S_3 \times \mathbb{Z}_2$, if $(\sigma,x)$ is such that $\sigma$ is a $3$-cycle
      and $x$ is the non-identity element of $\mathbb{Z}_2$, then $(\sigma,x)$ has order
      $\mathrm{lcm}\bigl(\mathrm{ord}(\sigma),\mathrm{ord}(x)\bigr) = \mathrm{lcm}(3,2) = 6.$
      Thus $S_3 \times \mathbb{Z}_2$ has elements of order $6$.
    \item By contrast, as above, $A_4$ has no elements of order $6$.
    \end{itemize}
    Hence there can be no isomorphism between $D_3 \times \mathbb{Z}_2$ and $A_4$.
    
    \bigskip
    
    \noindent
    \textbf{Conclusion:} We have shown the required (non-)isomorphisms by comparing element orders
    (or by constructing explicit isomorphisms in the one case where it exists).
     
\end{solution}
\end{document}
