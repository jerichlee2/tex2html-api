\documentclass[12pt]{article}

% Packages
\usepackage[margin=1in]{geometry}
\usepackage{amsmath,amssymb,amsthm}
\usepackage{enumitem}
\usepackage{hyperref}
\usepackage{xcolor}
\usepackage{import}
\usepackage{xifthen}
\usepackage{pdfpages}
\usepackage{transparent}
\usepackage{listings}
\usepackage{tikz}
\usepackage{physics}
\usepackage{siunitx}
  \usetikzlibrary{calc,patterns,arrows.meta,decorations.markings}


\DeclareMathOperator{\Log}{Log}
\DeclareMathOperator{\Arg}{Arg}

\lstset{
    breaklines=true,         % Enable line wrapping
    breakatwhitespace=false, % Wrap lines even if there's no whitespace
    basicstyle=\ttfamily,    % Use monospaced font
    frame=single,            % Add a frame around the code
    columns=fullflexible,    % Better handling of variable-width fonts
}

\newcommand{\incfig}[1]{%
    \def\svgwidth{\columnwidth}
    \import{./Figures/}{#1.pdf_tex}
}
\theoremstyle{definition} % This style uses normal (non-italicized) text
\newtheorem{solution}{Solution}
\newtheorem{proposition}{Proposition}
\newtheorem{problem}{Problem}
\newtheorem{lemma}{Lemma}
\newtheorem{theorem}{Theorem}
\newtheorem{remark}{Remark}
\newtheorem{note}{Note}
\newtheorem{definition}{Definition}
\newtheorem{example}{Example}
\newtheorem{corollary}{Corollary}
\theoremstyle{plain} % Restore the default style for other theorem environments
%

% Theorem-like environments
% Title information
\title{MATH-417}
\author{Jerich Lee}
\date{\today}

\begin{document}

\maketitle
\begin{proposition}
  Let $R$ be an integral domain and let $p\in R$.  
  If $p$ is \emph{prime}, then $p$ is \emph{irreducible}.
  \end{proposition}
  
  \begin{proof}
  Write $p=ab$ for some $a,b\in R$.  
  Because $p$ is prime, $p\mid a$ or $p\mid b$.  
  Assume $p\mid a$ (the other case is analogous), so $a=pc$ for some $c\in R$.  
  Then
  \[
  p=ab=(pc)b=p(cb).
  \]
  Since $R$ is an integral domain and $p\neq 0$, we may cancel $p$ to obtain $1=cb$.  
  Thus $b$ is a unit, and $p$ is irreducible.
  \end{proof}
  
  \medskip
  
  \begin{theorem}
  Let $R$ be a principal ideal domain (PID) and let $p\in R$.  
  Then
  \[
    p \text{ irreducible} \;\Longleftrightarrow\; p \text{ prime}.
  \]
  \end{theorem}
  
  \begin{proof}
  The direction ``prime $\Rightarrow$ irreducible'' holds in every integral
  domain by the preceding proposition.  
  We prove the converse.
  
  \smallskip
  Assume $p$ is irreducible.  
  Since $p$ is not a unit, $(p)$ is a proper ideal.  
  Suppose $(p)\subsetneq I\subsetneq R$ for some ideal $I$.  
  Because $R$ is a PID, $I$ is principal, say $I=(a)$.  
  Then $(p)\subseteq(a)$ implies $p=ab$ for some $b\in R$.  
  Irreducibility of $p$ forces $a$ or $b$ to be a unit.  
  If $b$ were a unit, $(a)=(p)$, contradicting $(p)\subsetneq I$; hence $a$
  is a unit and $I=R$.  
  Therefore, no ideal lies strictly between $(p)$ and $R$, so $(p)$ is
  \emph{maximal}.  
  In any commutative ring, maximal ideals are prime, hence $(p)$ is a prime
  ideal.  
  Consequently, $p$ is a prime element.
  
  Combining with the previous proposition, we conclude that $p$ is prime
  iff $p$ is irreducible in a PID.
  \end{proof}
  \begin{theorem}
    \label{thm:UFD-characterisation}
    An integral domain $R$ is a unique factorisation domain (UFD) \emph{iff}
    \begin{enumerate}
      \item\label{UFD-existence}
        every non--zero, non--unit $a\in R$ can be written as a finite
        product of irreducible elements (\emph{complete factorisation});
      \item\label{UFD-prime}
        every irreducible element of $R$ is prime.
    \end{enumerate}
    \end{theorem}
    
    \begin{proof}
    \textbf{$\Longrightarrow$}\;
    If $R$ is a UFD, property~\ref{UFD-existence} is part of the definition.
    To prove~\ref{UFD-prime}, let $p\in R$ be irreducible and suppose
    $p\mid ab$ with $a,b\neq0$.
    Since $R$ is a UFD we have unique factorisations
    \[
      a = u\,p_1^{\alpha_1}\cdots p_r^{\alpha_r},
      \qquad
      b = v\,p_1^{\beta_1}p_2^{\beta_2}\cdots p_r^{\beta_r},
    \]
    where $u,v\in R^\times$, the $p_i$ are pairwise non--associated
    irreducibles, and $p\sim p_1$.
    Then
    \[
      ab
       = uv\,p_1^{\alpha_1+\beta_1}\cdots p_r^{\alpha_r+\beta_r}.
    \]
    Because $p\mid ab$, the exponent $\alpha_1+\beta_1$ is positive, so
    either $\alpha_1>0$ or $\beta_1>0$, i.e.\ $p\mid a$ or $p\mid b$.
    Thus $p$ is prime.
    
    \smallskip
    \textbf{$\Longleftarrow$}\;
    Assume \ref{UFD-existence} and \ref{UFD-prime}.
    To show $R$ is a UFD, we must prove uniqueness of factorisation
    up to order and association.
    
    Let
    \[
      c = a_1\cdots a_r = b_1\cdots b_s
    \]
    be two factorisations of the same non--unit $c$ into irreducibles.
    We proceed by induction on $r$.
    
    \emph{Base $r=1$.}
    Then $a_1$ divides $b_1\cdots b_s$; by
    \ref{UFD-prime} it divides some $b_j$.
    Relabelling, suppose $j=1$.
    Irreducibility forces $a_1$ and $b_1$ to be associated.
    If $s>1$ we would have a unit equal to a product of irreducibles,
    contradiction, so $s=1$ and the two factorizations coincide.
    
    \emph{Inductive step $r>1$.}
    Again $a_1\mid b_1\cdots b_s$, so $a_1\mid b_j$ for some $j$; rename
    so $j=1$.
    Then $b_1=u\,a_1$ for a unit $u$, and
    \[
      u^{-1}a_2\cdots a_r = b_2\cdots b_s.
    \]
    Absorbing the unit into $a_2$, both sides are now factorizations of the
    same element with $r-1$ and $s-1$ irreducibles, respectively.
    By the induction hypothesis $r-1=s-1$ and, after re-ordering,
    each $a_i$ ($i\ge2$) is associated to $b_i$.
    Including $i=1$ shows $r=s$ and completes the proof.
    \end{proof}
    \begin{theorem}
      Every principal\-ideal domain (PID) is a unique factorisation domain (UFD).
      \end{theorem}
      
      \begin{proof}
      By the standard characterisation, an integral domain is a UFD iff
      \begin{enumerate}
        \item[(i)] every non--zero, non--unit element can be factored into
                   irreducibles, and
        \item[(ii)] every irreducible element is prime.
      \end{enumerate}
      
      \medskip
      \noindent\textbf{(i) Existence of factorisations.}
      Because a PID is Noetherian, any strictly increasing chain of principal
      ideals stabilises.
      Given $a\neq0$ that is not a unit, either $a$ is irreducible or it can be
      written $a=bc$ with $b,c$ non--units and $(a)\subsetneq(b),(c)$.
      Iterating and using the ascending chain condition yields a finite
      factorisation $a=p_1\cdots p_r$ into irreducibles.
      
      \smallskip
      \noindent\textbf{(ii) Irreducible implies prime.}
      Let $p$ be irreducible.
      If $(p)\subsetneq I\subsetneq R$ with $I=(a)$, then $p=ab$.
      Irreducibility forces $a$ or $b$ to be a unit, whence $I=R$.
      Thus $(p)$ is maximal, and maximal ideals are prime; hence $p$ is prime.
      
      \smallskip
      With (i) and (ii) verified, $R$ satisfies the UFD conditions.
      \end{proof}
      
      \begin{theorem}
      Every Euclidean ring is a UFD.
      \end{theorem}
      
      \begin{proof}
      Given a Euclidean function $\delta\colon R\setminus\{0\}\to\mathbb N$,
      let $I\neq\{0\}$ be an ideal of $R$ and pick $d\in I$ with minimal
      $\delta(d)$.
      For any $x\in I$, division gives $x=qd+r$ with either $r=0$ or
      $\delta(r)<\delta(d)$.
      Minimality forces $r=0$, so $x\in(d)$ and $I=(d)$.
      Hence every ideal is principal and $R$ is a PID.
      By the previous theorem every PID is a UFD, so every Euclidean ring is a
      UFD.
      \end{proof}
      \paragraph{Why $(p)$ is proper when $p$ is irreducible.}
Recall that an element $p\in R$ is \emph{irreducible} precisely when
\begin{enumerate}
  \item $p$ is \emph{not} a unit, and
  \item $p=ab\ \Longrightarrow\ a\text{ or }b\text{ is a unit.}$
\end{enumerate}

Because of condition~(1), $p$ has no inverse in $R$.
Hence $1\notin(p)=\{\,rp : r\in R\}$; otherwise $1=rp$ would exhibit an
inverse $r=p^{-1}$.
Consequently $(p)\subsetneq R$, so $(p)$ is a \emph{proper} ideal.
\medskip
\noindent
Now assume $p_{1}\mid ab$.
Write unique factorisations in the UFD:
\[
  a = u\,p_{1}^{\alpha_{1}}p_{2}^{\alpha_{2}}\cdots p_{r}^{\alpha_{r}},
  \qquad
  b = v\,p_{1}^{\beta_{1}}p_{2}^{\beta_{2}}\cdots p_{r}^{\beta_{r}},
\]
with $u,v\in R^{\times}$ and $\alpha_i,\beta_i\in\mathbb N\cup\{0\}$.
Hence
\[
  ab = uv\,
       p_{1}^{\alpha_{1}+\beta_{1}}
       p_{2}^{\alpha_{2}+\beta_{2}}\cdots
       p_{r}^{\alpha_{r}+\beta_{r}}.
\]

Because $p_{1}\mid ab$, the element $p_{1}$ occurs \emph{at least once}
in the factorisation of $ab$; equivalently
\[
  \alpha_{1}+\beta_{1} \;\ge 1.
\]
But $\alpha_{1},\beta_{1}$ are non-negative integers, so
$\alpha_{1}+\beta_{1}\ge1$ implies
\[
  \alpha_{1}>0 \quad\text{or}\quad \beta_{1}>0.
\]
If $\alpha_{1}>0$, then $p_{1}^{\alpha_{1}}\mid a$ and hence $p_{1}\mid a$.
If $\beta_{1}>0$, then $p_{1}^{\beta_{1}}\mid b$ and hence $p_{1}\mid b$.
Thus
\[
  p_{1}\mid ab
  \;\Longrightarrow\;
  p_{1}\mid a \;\text{or}\; p_{1}\mid b,
\]
so $p_{1}$ is \emph{prime}.
Since $a_{1}\mid b_{1}$ we have $b_{1}=a_{1}u$ for some $u\in R$.
Both $a_{1}$ and $b_{1}$ are irreducible, hence neither is a unit.
Because $b_{1}$ is irreducible, any factorisation of $b_{1}$ must involve
a unit, so $u$ must be that unit; i.e.\ $u\in R^{\times}$.
Multiplying by $u^{-1}$ gives $a_{1}=u^{-1}b_{1}$, whence
$b_{1}\mid a_{1}$ and $a_{1}$, $b_{1}$ are associated.
Applying the division algorithm in $F[X]$ we have
\[
  p(x)=q(x)\bigl(x-\alpha\bigr)+r,
  \qquad \deg r<1,
\]
so $r$ is a constant, say $r=c\in F$.
Evaluating at $x=\alpha$ gives
\[
  p(\alpha)=q(\alpha)\,(\alpha-\alpha)+c = c.
\]
If $r\neq0$ (i.e.\ $c\neq0$), then $p(\alpha)=c\neq0$, contradicting the
assumption that $\alpha$ is a root of $p$.
Hence $r=0$, and the equality becomes $p(x)=q(x)(x-\alpha)$; thus
\[
  x-\alpha \;\mid\; p(x).
\]
%--------------------------------------------------------------------
%  Gauss' Lemma (UFD version) – detailed step-by-step proof
%--------------------------------------------------------------------
\begin{theorem}[Gauss' Lemma]\label{thm:GaussLemmaUFD}
  Let $R$ be a unique-factorisation domain (UFD).  
  If $f,g\in R[X]$ are \emph{primitive} polynomials, then their product
  $fg$ is again primitive.
  \end{theorem}
  
  \begin{proof}
  \textbf{Notation.}
  Write
  \[
    f=\sum_{i=0}^{m}a_i\,x^{\,i},\qquad
    g=\sum_{j=0}^{n}b_j\,x^{\,j},\qquad
    a_i,b_j\in R,
  \]
  and recall that \emph{primitive} means
  \(
  \gcd(a_0,\dots,a_m)=\gcd(b_0,\dots,b_n)=1.
  \)
  
  \medskip
  \noindent
  \textbf{Step 1 –\,Reduction to a contradiction.}
  
  \begin{enumerate}[label=\textbf{(\arabic*)},itemsep=6pt]
  \item
  Assume, for contradiction, that $fg$ is \emph{not} primitive.
  Then its coefficients have a non-trivial common divisor.
  Choose an \emph{irreducible} element
  \(
    \pi\in R
  \)
  together with a polynomial
  \(
    h=\sum_{k=0}^{m+n}c_k x^{\,k}\in R[X]
  \)
  such that
  \[
    fg=\pi\,h.
  \]
  
  \item
  Because $f$ and $g$ are primitive, $\pi$ divides \emph{not all} $a_i$ and
  \emph{not all} $b_j$.
  Pick the \underline{smallest} indices
  \[
    r=\min\{\,i :\pi\nmid a_i\,\},\qquad
    s=\min\{\,j :\pi\nmid b_j\,\}.
  \]
  By construction $\pi\nmid a_r$ \emph{and} $\pi\nmid b_s$, while
  $\pi\mid a_i$ for $i<r$ and $\pi\mid b_j$ for $j<s$.
  \end{enumerate}
  
  \medskip
  \noindent
  \textbf{Step 2 –\,Look at the coefficient of $x^{\,r+s}$ in $fg$.}
  
  \begin{enumerate}[label=\textbf{(\arabic*)},start=3,itemsep=6pt]
  \item
  Expand the product:
  \(
    fg=\sum_{k=0}^{m+n}c_k x^{\,k},
  \)
  so $c_{r+s}$ is the coefficient of $x^{\,r+s}$ in $fg$.
  Concretely,
  \[
    c_{r+s}=a_0b_{r+s}+a_1b_{r+s-1}+\dots+a_{r+s}b_0.
  \]
  
  \item
  Multiply by $\pi$ using $fg=\pi h$:
  \[
    \pi\,c_{r+s}
      =a_0b_{r+s}+a_1b_{r+s-1}+\dots+a_rb_s+\dots+a_{r+s}b_0.
  \]
  
  \item
  Solve the above for the \emph{special} product $a_r b_s$:
  \[
    a_r b_s
      =\pi\,c_{r+s}-a_0b_{r+s}-\dots-a_{r+s}b_0
      \;=\;\lambda\,c_{r+s}-a_0b_{r+s}-\dots-a_{r+s}b_0,
  \]
  where $\lambda=1$ (we just rewrote, but introducing $\lambda$ keeps the
  next sentence short).
  \end{enumerate}
  
  \medskip
  \noindent
  \textbf{Step 3 –\,Divisibility analysis of the right-hand side.}
  
  \begin{enumerate}[label=\textbf{(\arabic*)},start=6,itemsep=6pt]
  \item
  Every term \emph{except} $a_r b_s$ contains a factor $a_i$ with $i<r$ or
  a factor $b_j$ with $j<s$.
  By the minimality of $r$ and $s$ these $a_i$ or $b_j$ are all divisible
  by $\pi$.
  Hence each term on the right-hand side is divisible by $\pi$, which
  implies
  \[
    \pi \mid a_r b_s.
  \]
  
  \item
  Because $R$ is a UFD, the irreducible $\pi$ is automatically
  \emph{prime}.
  Thus $\pi\mid a_r$ or $\pi\mid b_s$.
  
  \item
  But our choice of $r$ and $s$ guaranteed
  \(
    \pi\nmid a_r \text{ and } \pi\nmid b_s,
  \)
  a contradiction.
  \end{enumerate}
  
  \medskip
  \noindent
  \textbf{Conclusion.}
  Our assumption that $fg$ is not primitive leads to a contradiction, so
  $fg$ \emph{must} be primitive.
  \end{proof}
  \paragraph{Why every term other than $a_r b_s$ is divisible by $\pi$.}

Recall the choice of the indices
\[
  r=\min\{\,i \mid \pi\nmid a_i\,\},
  \qquad
  s=\min\{\,j \mid \pi\nmid b_j\,\}.
\]
By definition of these minima:
\[
  \boxed{\;
    \pi\mid a_i \ \text{for all } i<r
    \quad\text{and}\quad
    \pi\mid b_j \ \text{for all } j<s.
  \;}
\tag{$\ast$}
\]

The coefficient of $x^{\,r+s}$ in the product $fg$
is
\[
  c_{r+s}
  \;=\;
  \sum_{i=0}^{r+s} a_i\,b_{\,r+s-i}
  \;=\;
  \underbrace{a_0 b_{r+s}}_{i=0}
  \;+\;
  \underbrace{a_1 b_{r+s-1}}_{i=1}
  \;+\;\dots+\;
  \underbrace{a_r b_s}_{i=r}
  \;+\;\dots+\;
  \underbrace{a_{r+s} b_0}_{i=r+s}.
\]

Take an arbitrary summand $a_i b_{\,r+s-i}$ \emph{different} from
$a_r b_s$.
We must show it is divisible by $\pi$.
There are two mutually exclusive possibilities:

\begin{enumerate}[label=\textbf{Case \arabic*:}, leftmargin=12pt]
\item $i<r$.  
      Then $(\ast)$ gives $\pi\mid a_i$, so
      $\pi\mid a_i b_{\,r+s-i}$.
\item $i>r$.  
      In that case $r+s-i < s$ (because $i>r$), so
      $\pi\mid b_{\,r+s-i}$ by $(\ast)$, and again
      $\pi\mid a_i b_{\,r+s-i}$.
\end{enumerate}

In \textbf{both} cases the product is a multiple of $\pi$.
Hence \emph{every} term in the sum for $\pi c_{r+s}$,
\[
  \pi c_{r+s}
  \;=\;
  a_0 b_{r+s}
  +\dots+
  a_r b_s
  +\dots+
  a_{r+s} b_0,
\]
is divisible by $\pi$ \emph{except possibly} the single product
$a_r b_s$.
This justifies the statement that
\[
  \text{“By the minimality of $r$ and $s$ these $a_i$ or $b_j$ are all
  divisible by $\pi$.”}
\]
\[
  r=\min\{\,i : \pi\nmid a_i\,\}
  \quad\Longrightarrow\quad
  \begin{cases}
    \pi\mid a_i & (i<r),\\
    \pi\nmid a_r & (i=r).
  \end{cases}
\]

(The same with \(s\) and the $b_j$’s.)
Choosing \emph{max} instead of \emph{min} would not guarantee
$\pi\mid a_i$ for $i<r$, so the proof would fail.
\end{document}
