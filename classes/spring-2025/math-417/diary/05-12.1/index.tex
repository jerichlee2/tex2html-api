\documentclass[12pt]{article}

% Packages
\usepackage[margin=1in]{geometry}
\usepackage{amsmath,amssymb,amsthm}
\usepackage{enumitem}
\usepackage{hyperref}
\usepackage{xcolor}
\usepackage{import}
\usepackage{xifthen}
\usepackage{pdfpages}
\usepackage{transparent}
\usepackage{listings}
\usepackage{tikz}
\usepackage{physics}
\usepackage{siunitx}
\usepackage{booktabs}
\usepackage{cancel}
  \usetikzlibrary{calc,patterns,arrows.meta,decorations.markings}


\DeclareMathOperator{\Log}{Log}
\DeclareMathOperator{\Arg}{Arg}

\lstset{
    breaklines=true,         % Enable line wrapping
    breakatwhitespace=false, % Wrap lines even if there's no whitespace
    basicstyle=\ttfamily,    % Use monospaced font
    frame=single,            % Add a frame around the code
    columns=fullflexible,    % Better handling of variable-width fonts
}

\newcommand{\incfig}[1]{%
    \def\svgwidth{\columnwidth}
    \import{./Figures/}{#1.pdf_tex}
}
\theoremstyle{definition} % This style uses normal (non-italicized) text
\newtheorem{solution}{Solution}
\newtheorem{proposition}{Proposition}
\newtheorem{problem}{Problem}
\newtheorem{lemma}{Lemma}
\newtheorem{theorem}{Theorem}
\newtheorem{remark}{Remark}
\newtheorem{note}{Note}
\newtheorem{definition}{Definition}
\newtheorem{example}{Example}
\newtheorem{corollary}{Corollary}
\theoremstyle{plain} % Restore the default style for other theorem environments
%

% Theorem-like environments
% Title information
\title{}
\author{Jerich Lee}
\date{\today}

\begin{document}

\maketitle
\begin{solution}
  \textbf{Step 1.\;Size of the group.}
  \[
  \bigl|\operatorname{GL}_{2}(\mathbb{Z}_{2})\bigr|
     =(2^{2}-1)(2^{2}-2)=3\cdot2=6.
  \]
  
  \textbf{Step 2.\;A complete list of matrices.}  Writing matrices by columns,
  \[
  \renewcommand{\arraystretch}{1.2}
  \begin{array}{lll}
  I=\begin{pmatrix}1&0\\0&1\end{pmatrix},&
  A=\begin{pmatrix}1&1\\0&1\end{pmatrix},&
  B=\begin{pmatrix}1&0\\1&1\end{pmatrix},\\[4pt]
  C=\begin{pmatrix}0&1\\1&0\end{pmatrix},&
  D=\begin{pmatrix}1&1\\1&0\end{pmatrix},&
  E=\begin{pmatrix}0&1\\1&1\end{pmatrix}.
  \end{array}
  \]
  
  \textbf{Step 3.\;Orders (quick calculation).}
  \[
  \begin{aligned}
  \operatorname{ord}(I)&=1,\\
  \operatorname{ord}(A)=\operatorname{ord}(B)=\operatorname{ord}(C)&=2,\\
  \operatorname{ord}(D)=\operatorname{ord}(E)&=3.
  \end{aligned}
  \]
  
  \textbf{Step 4.\;Characteristic polynomials.}\;
  For every \(M\in G\),
  \[
  \chi_{M}(x)=x^{2}+\operatorname{tr}(M)\,x+1
  \quad(\text{since }\det M=1 \text{ in } \mathbb{Z}_{2}).
  \]
  \[
  \operatorname{tr}(I)=\operatorname{tr}(A)=\operatorname{tr}(B)
            =\operatorname{tr}(C)=0,
  \qquad
  \operatorname{tr}(D)=\operatorname{tr}(E)=1.
  \]
  
  \smallskip
  \textbf{Step 5.\;Conjugacy classes.}
  
  \[
  \renewcommand{\arraystretch}{1.4}
  \begin{array}{|c|c|c|c|c|}
  \hline
  \text{Class} & \text{Size} & \text{Order of elems.} &
  \text{Characteristic polynomial} & \text{Representative(s)} \\
  \hline
  C_{1} & 1 & 1 & x^{2}+1 & I \\
  \hline
  C_{2} & 3 & 2 & x^{2}+1 & 
     \left\{\,A,\;B,\;C\,\right\} \\
  \hline
  C_{3} & 2 & 3 & x^{2}+x+1 & 
     \left\{\,D,\;E\,\right\} \\
  \hline
  \end{array}
  \]
  
  \textit{Why they split this way.}
  \begin{itemize}
    \item The class of the identity is always singleton.
    \item All matrices with minimal polynomial \((x+1)^{2}\) but \(\neq I\)
          are conjugate (they are the “reflections’’ in the
          $S_{3}\cong G$ picture).  There are exactly three such matrices.
    \item The remaining two matrices have irreducible polynomial
          \(x^{2}+x+1\) and order \(3\); they form a single conjugacy class
          (the “3-cycles’’ of \(S_{3}\)).
  \end{itemize}
  
  \[
  \boxed{\;
     \text{There are three conjugacy classes: } 
     \{I\},\;
     \{A,B,C\},\;
     \{D,E\}.
  \;}
  \]
  \end{solution}
  \begin{theorem}
    For every positive integer \(n\) the additive group 
    \(G=\mathbb{Q}/\mathbb{Z}\) contains a \emph{unique} subgroup of order \(n\).
    \end{theorem}
    
    \begin{proof}
    Throughout, write the group additively and let 
    \(\overline{q}=q+\mathbb{Z}\) denote the coset of \(q\in\mathbb{Q}\).
    
    \smallskip
    \textbf{1.\;Every element of \(\mathbb{Q}/\mathbb{Z}\) has finite order.}  
    If \(q=\tfrac ab\) with \(a\in\mathbb{Z}\), \(b\in\mathbb{N}\) and
    \(\gcd(a,b)=1\), then
    \[
       b\;\overline{q}= \overline{\,b\cdot\tfrac ab\,}= \overline{a}=0 ,
    \]
    so \(\operatorname{ord}(\overline{q})\mid b\).
    In fact \(\operatorname{ord}(\overline{q})=b\); otherwise a smaller
    positive multiple of \(q\) would be an integer, contradicting minimality
    of \(b\).
    Thus every element has order equal to the denominator of its reduced
    fraction.
    
    \smallskip
    \textbf{2.\;Existence of a subgroup of order \(n\).}  
    Consider
    \[
       H_{n}\;:=\;\bigl\langle\,\overline{1/n}\,\bigr\rangle
                \;=\;
                \Bigl\{
                   \overline{0},\,\overline{1/n},\,\overline{2/n},
                   \dots,\overline{(n-1)/n}
                \Bigr\}.
    \]
    Because \(n\,\overline{1/n}=0\) and no smaller positive multiple
    vanishes, \(\operatorname{ord}(\overline{1/n})=n\); hence \(H_{n}\)
    is cyclic of order \(n\).
    
    \smallskip
    \textbf{3.\;Uniqueness.}  
    Let \(K\le\mathbb{Q}/\mathbb{Z}\) be \emph{any} subgroup of order \(n\).
    Finite subgroups of an abelian group are cyclic, so
    \(K=\langle\overline{q}\rangle\) for some \(\overline{q}\) with
    \(\operatorname{ord}(\overline{q})=n\).
    Write \(q\) in lowest terms as \(q=\tfrac mn\) (\(\gcd(m,n)=1\)).
    Because \(\gcd(m,n)=1\), there exists \(s\in\mathbb{Z}\) with
    \(ms\equiv1\pmod{n}\).  Then
    \[
       s\,\overline{q}
         \;=\;
         \overline{\tfrac{s\,m}{n}}
         \;=\;
         \overline{\tfrac1n}\;\in\;K,
    \]
    so \(\overline{1/n}\in K\).  
    Therefore \(H_{n}\subseteq K\).
    But \(|H_{n}|=|K|=n\), forcing \(H_{n}=K\).
    
    \smallskip
    \textbf{4.\;Conclusion.}  
    For each \(n\ge1\) there is exactly one subgroup of order \(n\), namely
    \[
       \boxed{\;
         H_{n}\;=\;
         \bigl\langle\,\overline{1/n}\,\bigr\rangle
         \;\cong\;
         C_{n}}.
    \]
    \end{proof}
    \begin{theorem}
      Let
      \[
         H \;=\; \bigl\{\,A\in\operatorname{GL}_{2}(\mathbb{C})
                    \;\big|\;
                    \det(A)\in\mathbb{R}_{>0}\bigr\}, 
         \qquad
         B \;=\; S^{1}
              \;=\;\bigl\{\,z\in\mathbb{C}\;\big|\;|z|=1\bigr\}.
      \]
      Then the quotient group
      \(\operatorname{GL}_{2}(\mathbb{C})/H\) is (canonically) isomorphic to
      \(B\).
      \end{theorem}
      
      \begin{proof}
      \textbf{1.\;Reduce to determinants.}
      The determinant gives a surjective group homomorphism
      \[
         \det \;:\;
         \operatorname{GL}_{2}(\mathbb{C}) \;\longrightarrow\;
         \mathbb{C}^{\times},
         \qquad
         A \;\longmapsto\; \det(A).
      \]
      Because \(H=\det^{-1}(\mathbb{R}_{>0})\), the map
      \[
         \varphi \;:\;
         \operatorname{GL}_{2}(\mathbb{C})/H
         \;\longrightarrow\;
         \mathbb{C}^{\times}/\mathbb{R}_{>0},
         \qquad
         AH \;\longmapsto\; \det(A)\,\mathbb{R}_{>0},
      \]
      is well defined, surjective, and a group homomorphism whose kernel is
      trivial.  Hence
      \[
         \operatorname{GL}_{2}(\mathbb{C})/H
         \;\cong\;
         \mathbb{C}^{\times}/\mathbb{R}_{>0}.
      \]
      
      \smallskip
      \textbf{2.\;Identify \(\mathbb{C}^{\times}/\mathbb{R}_{>0}\) with \(S^{1}\).}
      Every \(z\in\mathbb{C}^{\times}\) can be written uniquely as
      \(z=re^{i\theta}\) with \(r\in\mathbb{R}_{>0}\) and \(\theta\in(-\pi,\pi]\).
      Multiplication by a positive real rescales \(r\) but leaves the phase
      \(e^{i\theta}\) unchanged.  Define
      \[
         \psi \;:\;
         \mathbb{C}^{\times}/\mathbb{R}_{>0}
         \;\longrightarrow\;
         S^{1},
         \qquad
         z\,\mathbb{R}_{>0}
         \;\longmapsto\;
         \frac{z}{|z|}.
      \]
      \begin{itemize}
        \item \emph{Well defined:}  If \(z_{1}=r\,z_{2}\) with \(r>0\) then
              \(z_{1}/|z_{1}|=z_{2}/|z_{2}|\).
        \item \emph{Bijective:}  Every \(e^{i\theta}\in S^{1}\) is the image
              of \(e^{i\theta}\,\mathbb{R}_{>0}\), and the map is clearly
              injective.
        \item \emph{Homomorphism:}  Because the operation on the quotient and
              on \(S^{1}\) is multiplication, $\psi$ preserves it.
      \end{itemize}
      Thus \(\psi\) is an isomorphism, giving
      \[
         \mathbb{C}^{\times}/\mathbb{R}_{>0} \;\cong\; S^{1}.
      \]
      
      \smallskip
      \textbf{3.\;Compose isomorphisms.}
      Combining the isomorphisms from Steps 1 and 2 yields
      \[
         \operatorname{GL}_{2}(\mathbb{C})/H
         \;\cong\;
         \mathbb{C}^{\times}/\mathbb{R}_{>0}
         \;\cong\;
         S^{1}=B.
      \]
      Explicitly, the composite map
      \[
         AH
         \;\longmapsto\;
         \det(A)\,\mathbb{R}_{>0}
         \;\longmapsto\;
         \frac{\det(A)}{|\det(A)|}\;\in\;B
      \]
      is the desired isomorphism.
      
      \end{proof}b
\end{document}
