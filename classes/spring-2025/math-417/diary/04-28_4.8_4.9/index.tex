\documentclass[12pt]{article}

% Packages
\usepackage[margin=1in]{geometry}
\usepackage{amsmath,amssymb,amsthm}
\usepackage{enumitem}
\usepackage{hyperref}
\usepackage{xcolor}
\usepackage{import}
\usepackage{xifthen}
\usepackage{pdfpages}
\usepackage{transparent}
\usepackage{listings}
\usepackage{tikz}
\usepackage{physics}
\usepackage{siunitx}
  \usetikzlibrary{calc,patterns,arrows.meta,decorations.markings}


\DeclareMathOperator{\Log}{Log}
\DeclareMathOperator{\Arg}{Arg}

\lstset{
    breaklines=true,         % Enable line wrapping
    breakatwhitespace=false, % Wrap lines even if there's no whitespace
    basicstyle=\ttfamily,    % Use monospaced font
    frame=single,            % Add a frame around the code
    columns=fullflexible,    % Better handling of variable-width fonts
}

\newcommand{\incfig}[1]{%
    \def\svgwidth{\columnwidth}
    \import{./Figures/}{#1.pdf_tex}
}
\theoremstyle{definition} % This style uses normal (non-italicized) text
\newtheorem{solution}{Solution}
\newtheorem{proposition}{Proposition}
\newtheorem{problem}{Problem}
\newtheorem{lemma}{Lemma}
\newtheorem{theorem}{Theorem}
\newtheorem{remark}{Remark}
\newtheorem{note}{Note}
\newtheorem{definition}{Definition}
\newtheorem{example}{Example}
\newtheorem{corollary}{Corollary}
\theoremstyle{plain} % Restore the default style for other theorem environments
%

% Theorem-like environments
% Title information
\title{}
\author{Jerich Lee}
\date{\today}

\begin{document}

\maketitle

% ---------------------------------------------------------------------------
% Principal, Prime, and Maximal Ideals – Definitions and Key Proposition
% ---------------------------------------------------------------------------
\subsection*{Proper, Principal, and Prime Ideals}

\begin{definition}[Proper ideal]
  Let \(R\) be a ring.  
  An ideal \(I\subset R\) is called \emph{proper} if \(I\neq R\).
  Equivalently, \(I\subset R\) is proper  \(\Longleftrightarrow\)
  the factor ring \(R/I\) is \emph{non-trivial} (i.e.\ \(0\neq1\) in \(R/I\)).
\end{definition}

\begin{definition}[Principal ideal]
  Let \(R\) be a \emph{commutative} ring.  
  An ideal \(I\subset R\) is \emph{principal} if there exists an element \(a\in R\) such that  
  \[
    I \;=\; (a)\;=\;\{\,ra \mid r\in R\,\}.
  \]
\end{definition}

\begin{definition}[Prime ideal]
  Let \(R\) be a \emph{commutative} ring.  
  A proper ideal \(I\subset R\) is called \emph{prime} if for every \(a,b\in R\)
  \[
    ab\in I \quad\Longrightarrow\quad a\in I \;\text{ or }\; b\in I .
  \]
\end{definition}

\subsection*{Prime Ideals via Integral Domains}

\begin{proposition}
  Let \(R\) be a commutative ring and \(I\subset R\) an ideal.  
  Then \(I\) is prime \textbf{iff} the quotient ring \(R/I\) is an \emph{integral domain}.
\end{proposition}

\begin{proof}
  We prove the two implications separately.

  \paragraph{(\(\Longrightarrow\)) If \(I\) is prime, then \(R/I\) is an integral domain.}
  \begin{enumerate}
    \item Because \(I\) is proper, \(R/I\) is non-trivial, so \(0\neq 1\) in \(R/I\).
    \item Take any cosets \((a+I),(b+I)\in R/I\) with \((a+I)(b+I)=0+I\).  
          That is, \((ab)+I = 0+I\), so \(ab\in I\).
    \item Since \(I\) is \emph{prime} and \(ab\in I\), we must have \(a\in I\) or \(b\in I\).
          Equivalently, \(a+I=0+I\) or \(b+I=0+I\).
    \item Hence \(R/I\) has \emph{no zero divisors}.  
          Together with commutativity, \(R/I\) is an integral domain.
  \end{enumerate}

  \paragraph{(\(\Longleftarrow\)) If \(R/I\) is an integral domain, then \(I\) is prime.}
  \begin{enumerate}
    \item Suppose \(I\) were \emph{not} prime.  
          Then there exist \(a,b\in R\) such that \(ab\in I\) but \(a\notin I\) and \(b\notin I\).
    \item In the quotient ring this means
          \[
            (a+I)\neq 0+I,\qquad (b+I)\neq 0+I,\qquad
            (a+I)(b+I)=ab+I = 0+I .
          \]
          Thus \((a+I)\) and \((b+I)\) are non-zero elements whose product is zero.
    \item That contradicts the fact that \(R/I\) is an integral domain (which has no zero divisors).
          Therefore such \(a,b\) cannot exist, and \(I\) \emph{must} be prime.
  \end{enumerate}
\end{proof}
% ---------------------------------------------------------------------------
% Maximal Ideals – Definition, Characterisation, and Corollary
% ---------------------------------------------------------------------------
\subsection*{Maximal Ideals}

\begin{definition}[Maximal ideal]
  Let \(R\) be a \emph{commutative} ring and let \(I\subset R\) be a proper ideal.
  We say that \(I\) is \emph{maximal} if it is maximal in the poset of proper ideals
  under inclusion; that is,
  \[
     I\subset J\subset R\quad\Longrightarrow\quad J=I\ \text{or}\ J=R.
  \]
\end{definition}

\begin{proposition}
  Let \(R\) be a commutative ring and \(I\subset R\) an ideal.
  Then \(I\) is maximal \emph{iff} the quotient ring \(R/I\) is a \emph{field}.
\end{proposition}

\begin{proof}
  The ring \(R/I\) is automatically commutative, so we only need to verify the field
  axioms concerning multiplicative inverses of non-zero elements.

  \paragraph{(\(\Longrightarrow\)) If \(I\) is maximal, then \(R/I\) is a field.}
  \begin{enumerate}
    \item Let \(a+I\in R/I\) be non-zero; equivalently, \(a\notin I\).
    \item Consider the ideal generated by \(a\) together with \(I\):
          \[
             (a)+I \;=\; \{\,ra+b\mid r\in R,\ b\in I\,\}.
          \]
          Clearly \(I\subset (a)+I\) and \(a\in (a)+I\).
    \item Because \(a\notin I\), the containment is strict: \(I\subsetneq (a)+I\).
          Since \(I\) is \emph{maximal}, we must have \((a)+I = R\).
    \item Hence there exist \(r\in R\) and \(b\in I\) such that
          \[
             ra + b = 1 .
          \]
          Passing to the quotient gives \((r+I)(a+I)=1+I\), so \(r+I\) is the
          multiplicative inverse of \(a+I\).
          % ---------------------------------------------------------------------------
% Expanded justification of Step 4 in the proof “maximal ⇒ field”
% ---------------------------------------------------------------------------
\paragraph*{Detailed justification of Step 4.}

From Step 3 we have shown that 
\[
  (a)+I \;=\; R .
\]
Recall the meaning of ideal equality:

\[
  J = R 
  \;\Longleftrightarrow\; 
  1 \in J .
\]

Because \((a)+I = R\), the unit \(1\in R\) \emph{must} lie inside the ideal \((a)+I\).
By definition of \((a)+I\),
\[
  (a)+I 
  \;=\; 
  \{\, ra + b \mid r\in R,\; b\in I \,\}.
\]
Hence there exist elements \(r\in R\) and \(b\in I\) such that
\[
  1 \;=\; ra + b .
\]

Intuitively, the equality \((a)+I = R\) tells us that every element of \(R\)—in particular the multiplicative identity \(1\)—can be written as an \(R\!\)-linear combination of \(a\) and an element of \(I\).  This explicit representation of \(1\) is precisely what we need in the next step to show that \(r+I\) is the inverse of \(a+I\) in the quotient ring \(R/I\).

(If you prefer to see it in a single line: \(1\in(a)+I\) \(\Longrightarrow\) \(1=ra+b\) for some \(r\in R,\; b\in I\).)
\begin{lemma}
  Let \(R\) be a ring and \(J\subset R\) an ideal.  
  Then
  \[
      J = R \quad\Longleftrightarrow\quad 1\in J .
  \]
\end{lemma}

\begin{proof}
  \begin{description}
    \item[\(\boldsymbol{\Rightarrow}\)]  
      If \(J = R\) then every element of \(R\) lies in \(J\); in particular \(1\in J\).

    \item[\(\boldsymbol{\Leftarrow}\)]  
      Conversely, suppose \(1\in J\).  
      Because \(J\) is an \emph{ideal}, it is closed under multiplication by arbitrary ring
      elements: for every \(r\in R\),
      \[
        r = r\cdot 1 \in J .
      \]
      Hence every element of \(R\) belongs to \(J\), i.e.\ \(R\subset J\).
      But we already have \(J\subset R\), so \(J = R\).
  \end{description}
\end{proof}
% ---------------------------------------------------------------------------
% Why does the element \(1\) (a multiplicative identity) exist in \(R\;\)?
% ---------------------------------------------------------------------------
\paragraph*{Why \(1\) is available in our discussion.}

\begin{itemize}
  \item \textbf{Convention in these notes.}  
        Throughout the foregoing definitions and proofs we have been using
        standard terminology such as “\emph{field}’’ and “\emph{integral
        domain}.’’  
        Both objects are \emph{by definition} commutative rings \emph{with}
        a multiplicative identity~\(1\).  
        Consequently, whenever we say
        \[
             \text{``Let }R\text{ be a ring.''}
        \]
        in this context, we are implicitly assuming \emph{unital} (a.k.a.\ \emph{unitary})
        rings; i.e.\ rings that possess a distinguished element
        \(1\in R\) satisfying \(1\cdot r = r\cdot 1 = r\) for all \(r\in R\).

  \item \textbf{Why the identity matters in the proof.}  
        Step~4 of the proof required us to write
        \[
           1 \;=\; ra + b
           \quad\text{with}\quad
           r\in R,\; b\in I.
        \]
        That manipulation only makes sense if an element called ``\(1\)’’
        already exists in \(R\).  Because we are working with a unital ring,
        such an element is guaranteed.

  \item \textbf{If you adopt the ``rings need not have \(1\)’’ convention.}  
        Some algebra texts define a ring merely as an abelian group under
        addition equipped with an associative multiplication (no requirement
        of a multiplicative identity).  
        Under that convention, one must \emph{add} the hypothesis that
        \(R\) has an identity to state the proposition “\(I\) maximal 
        \(\Longleftrightarrow R/I\) is a field,’’ because a field (or
        integral domain) \emph{necessarily} contains a unit element.

  \item \textbf{Bottom line.}  
        In these notes we are using the common modern definition:
        \[
           \text{Ring} \;=\; \text{``associative\, +\, distributive\, +\, has \(1\)’’}.
        \]
        Hence the multiplicative identity \(1\) is part of the starting
        data of \(R\), so Step~4 (and the lemma \(J=R\iff 1\in J\)) is
        perfectly valid.
\end{itemize}
    \item Thus every non-zero element of \(R/I\) is invertible; hence \(R/I\) is a field.
  \end{enumerate}

  \paragraph{(\(\Longleftarrow\)) If \(R/I\) is a field, then \(I\) is maximal.}
  \begin{enumerate}
    \item Suppose \(J\) is an ideal with \(I\subsetneq J\subsetneq R\).
    \item Choose \(a\in J\setminus I\).  
          Then \(a+I\neq 0+I\) in \(R/I\), so \(a+I\) is \emph{invertible}:
          there exists \(b\in R\) with \((a+I)(b+I)=1+I\), i.e.\ \(ab-1\in I\subset J\).
    \item Since \(a\in J\) and \(ab-1\in J\), we have
          \[
             1 = ab-(ab-1)\in J .
          \]
          \textbf{Expanded justification of Step 3.}

Recall the situation so far:

* \(J\) is an ideal with \(I\subsetneq J\subsetneq R\).
* We picked \(a\in J\setminus I\).
* Because \(R/I\) is a field, \(a+I\) is invertible, so there exists \(b\in R\) such that
  \[
    (a+I)(b+I)=1+I
    \;\;\Longrightarrow\;\;
    ab-1\in I\subset J .
  \]

Now we show \(1\in J\).

\begin{enumerate}
  \item Since \(a\in J\) and \(b\in R\) while \(J\) is an \emph{ideal}, the product
        \(ab\) lies in \(J\);
        \[
          a\in J,\; b\in R 
          \quad\Longrightarrow\quad
          ab\in J 
          \quad
          (\text{closure of }J\text{ under }R\text{-multiplication}).
        \]
  \item We also have \(ab-1\in J\) from the previous step.
  \item Ideals are additive subgroups, hence closed under subtraction:
        \[
          ab\in J,\; ab-1\in J
          \;\Longrightarrow\;
          \bigl(ab\bigr) \;-\; \bigl(ab-1\bigr)
          \;=\;
          1 
          \;\in\; J .
        \]
        % ---------------------------------------------------------------------------
% Why “\(\,ab\in J\) and \(ab-1\in J\)  ⇒  \(1\in J\)” forces \(J=R\)
% ---------------------------------------------------------------------------
\paragraph*{Additive–subgroup property of an ideal.}
By definition an ideal \(J\subset R\) satisfies
\begin{enumerate}
  \item \(J\) is an \emph{additive subgroup} of the underlying abelian
        group \((R,+)\).  
        Concretely, for all \(x,y\in J\)
        \[
            x+y\in J,
            \qquad
            -x\in J
            \quad\Longrightarrow\quad
            x-y = x + (-y)\in J .
        \]
        \emph{Hence ideals are closed under subtraction.}

  \item (Absorption) For all \(r\in R\) and \(x\in J\), we have 
        \(rx\in J\) (and \(xr\in J\) if \(R\) is not assumed commutative).
\end{enumerate}

\paragraph*{Applying these facts to the proof.}
We were given two elements already in \(J\):
\[
    ab\;\in\;J
    \quad\text{and}\quad
    ab-1\;\in\;J .
\]

\begin{align*}
  \text{Since }J\text{ is additive } &\Rightarrow
    \bigl(ab\bigr) - \bigl(ab-1\bigr)\;\in\;J .
    \\[4pt]
  \text{But } 
    \bigl(ab\bigr) - \bigl(ab-1\bigr) 
    &= ab - ab + 1
      \;=\;1 .
\end{align*}

Therefore
\[
    1\;\in\;J .
\]

\paragraph*{Why \(1\in J\) implies \(J=R\).}
Assume \(1\in J\).  
For \emph{every} element \(r\in R\) we have
\[
   r 
   \;=\; r\cdot 1
   \;\in\; J
   \qquad\text{(by the absorption property).}
\]
Hence \(R\subset J\).  As the reverse inclusion \(J\subset R\) is automatic,
we conclude \(J = R\).

\bigskip
Because this contradicted the standing assumption \(J\subsetneq R\),
no such strictly larger ideal \(J\) can exist; consequently the original
ideal \(I\) must be maximal.
\end{enumerate}

Thus \(1\in J\), which forces \(J=R\) (because any ideal containing \(1\) is the whole ring).  
This contradicts the assumption \(J\subsetneq R\), so no such \(J\) exists and \(I\) is maximal.
    \item But \(1\in J\) implies \(J=R\), contradicting \(J\subsetneq R\).
          Therefore no such \(J\) exists and \(I\) is maximal.
  \end{enumerate}
\end{proof}

\begin{corollary}
  If \(I\) is a maximal ideal in a commutative ring \(R\), then \(I\) is prime.
\end{corollary}

\begin{proof}
  From the proposition, maximal \(\Longrightarrow\) \(R/I\) is a \emph{field};
  every field is an \emph{integral domain}.  
  By the earlier characterisation of prime ideals
  (\(I\) prime \(\Longleftrightarrow R/I\) integral domain),
  it follows that \(I\) is prime.
\end{proof}
% ---------------------------------------------------------------------------
% 4.9  Factorisation in Integral Domains
% ---------------------------------------------------------------------------
\subsection*{Divisibility, Association, and a Characterisation}

Throughout this section \(R\) denotes an \emph{integral domain}  
(commutative, \(1\neq0\), and with no zero–divisors).

\paragraph{Divisibility.}
For \(a,b\in R\) we write
\[
   a\mid b
   \quad\Longleftrightarrow\quad
   \exists\,c\in R\;\text{such that}\; b = ac .
\]

\begin{definition}[Associated elements]
  Two \emph{non-zero} elements \(a,b\in R\) are called \emph{associated},
  written \(a\sim b\), if
  \[
      a\mid b
      \quad\text{and}\quad
      b\mid a .
  \]
  Equivalently, there exist \(c,d\in R\) with
  \(
     b = ac
     \text{ and }
     a = bd .
  \)
\end{definition}

\begin{theorem}
  Let \(R\) be an integral domain and \(a,b\in R\setminus\{0\}\).  
  Then
  \[
     a \sim b
     \;\Longleftrightarrow\;
     a = bu
     \quad\text{for some unit }u\in R^{\times}.
  \]
\end{theorem}

\begin{proof}
  We prove both implications.

  %---------------------------------------------
  \paragraph*{\((\Rightarrow)\)}
  Assume \(a\sim b\).  
  Then \(a\mid b\) and \(b\mid a\); hence there exist \(c,d\in R\) such that
  \[
      b = ac,
      \qquad
      a = bd .
  \]
  Substituting \(b=ac\) into \(a=bd\) gives
  \[
      a
      \;=\;
      (ac)d
      \;=\;
      a(cd).
  \]
  Since \(a\neq0\) and \(R\) is an integral domain, we may cancel \(a\):
  \[
      cd = 1 .
  \]
  Thus \(c\) and \(d\) are \emph{mutual inverses}; in particular
  \(c\) is a \emph{unit}.  
  But \(b = ac\) now reads \(a = b\,c^{-1}\), so
  \[
      a = bu
      \quad\text{with}\quad
      u := c^{-1}\in R^{\times}.
  \]

  %---------------------------------------------
  \paragraph*{\((\Leftarrow)\)}
  Conversely, suppose \(a = bu\) for some unit \(u\in R^{\times}\).
  Let \(v:=u^{-1}\) be its inverse.
  \[
      \underbrace{a}_{=bu}\;=\;bu 
      \quad\Longrightarrow\quad
      b\;\mid\; a ,
      \qquad
      \underbrace{b}_{=av}\;=\;av 
      \quad\Longrightarrow\quad
      a\;\mid\; b .
  \]
  Hence \(a\) and \(b\) divide each other, so \(a\sim b\).

  %---------------------------------------------
  Both directions have been established; the theorem follows.
\end{proof}

\paragraph{Remark.}
In an integral domain, “associated’’ is an equivalence relation whose
equivalence classes correspond to non-zero elements up to multiplication by units.
This notion is fundamental in studying unique factorisation:
prime/irreducible factors are only unique up to association.
% ---------------------------------------------------------------------------
%  Principal Ideals and Associated Elements in an Integral Domain
% ---------------------------------------------------------------------------
\begin{theorem}
  Let \(R\) be an integral domain and let \(a,b\in R\) be \emph{non-zero} elements.  
  Then
  \[
     (a) \subseteq (b)
     \;\Longleftrightarrow\;
     b \mid a .
  \]
  Consequently,
  \[
     a \text{ and } b \text{ are associated}
     \;\Longleftrightarrow\;
     (a) = (b).
  \]
  In particular, for \(m,n\in\mathbb Z\setminus\{0\}\) we have
  \[
     m \sim n
     \;\Longleftrightarrow\;
     n=\pm m .
  \]
\end{theorem}

\begin{proof}
  Recall the notation \((x)=\{rx \mid r\in R\}\) for the principal ideal generated by \(x\).

  %-----------------------------------------------------------------
  \paragraph{\(\boldsymbol{(a)\subseteq(b)\ \Rightarrow\ b\mid a}\)}
  If \((a)\subseteq(b)\), then in particular \(a\in(a)\subseteq(b)\).  
  Therefore there exists an element \(r\in R\) such that
  \[
      a = br ,
  \]
  i.e.\ \(b\mid a\).

  %-----------------------------------------------------------------
  \paragraph{\(\boldsymbol{b\mid a\ \Rightarrow\ (a)\subseteq(b)}\)}
  Suppose \(b\mid a\); so \(a = br\) for some \(r\in R\).  
  Take an arbitrary element \(x\in(a)\); by definition \(x = sa\) for some \(s\in R\).  
  Substitute \(a=br\):
  \[
      x = s(br) = (sr)\,b \in (b) .
  \]
  Hence every element of \((a)\) lies in \((b)\), so \((a)\subseteq(b)\).

  %-----------------------------------------------------------------
  \paragraph{Association \(\Longleftrightarrow\) Ideal Equality.}
  By definition \(a\) and \(b\) are \emph{associated} if \(a\mid b\) and \(b\mid a\).  
  Using the equivalence already proved,
  \[
     a\mid b \Longleftrightarrow (b)\subseteq(a),
     \qquad
     b\mid a \Longleftrightarrow (a)\subseteq(b).
  \]
  Both inclusions hold simultaneously \(\Longleftrightarrow\) the two ideals
  coincide, i.e.\ \((a)=(b)\).

  %-----------------------------------------------------------------
  \paragraph{Specialisation to \(\mathbb Z\).}
  In \(\mathbb Z\) the \emph{units} are exactly \(\pm1\).  
  Two integers \(m,n\neq0\) generate the same principal ideal precisely when
  \(n=\pm m\).  Thus \(m\sim n\) \(\iff\) \(n=\pm m\).
\end{proof}
% ---------------------------------------------------------------------------
%  What does the notation  (a)  mean?
% ---------------------------------------------------------------------------
\paragraph{Principal ideal generated by \(a\).}
For an element \(a\) of a ring \(R\) we write
\[
  (a)\;=\;\{\,r\,a \mid r\in R\,\}.
\]

\begin{itemize}
  \item \textbf{Terminology.}  
        The set \((a)\) is called the \emph{principal ideal generated by \(a\)}.
        In a \emph{commutative} ring it is automatically a (two-sided) ideal.
        In a non-commutative ring, \((a)=Ra\) is the \emph{left} ideal generated by \(a\);
        the corresponding right ideal is \(aR=\{\,a\,r \mid r\in R\,\}\).

  \item \textbf{Why is \((a)\) an ideal?}  
        \begin{enumerate}
          \item \emph{Additive subgroup.}  
                If \(r_1a,r_2a\in (a)\) then
                \(
                   (r_1a)-(r_2a) = (r_1-r_2)a \in (a),
                \)
                so \((a)\) is closed under addition and additive inverses.
          \item \emph{Absorption.}  
                For any \(s\in R\) and any \(ra\in(a)\),
                \(
                   s\,(ra) = (sr)\,a \in (a).
                \)
                Hence multiplying by any element of \(R\) (on the left) keeps you inside \((a)\).
        \end{enumerate}

  \item \textbf{Concrete examples.}
        \begin{itemize}
          \item In the integers \(\Bbb Z\) we have
                \[
                  (6) = \{\,\dots,-12,-6,0,6,12,18,\dots\} 
                       = 6\Bbb Z .
                \]
          \item In the polynomial ring \(k[x]\) (over a field \(k\)),
                \(
                  (x^2) = \{\,p(x)\,x^2 \mid p(x)\in k[x]\}
                \)
                consists of all polynomials with at least a double root at \(x=0\).
        \end{itemize}

  \item \textbf{Key property in integral domains.}  
        For non-zero \(a,b\) in an \emph{integral domain},
        \[
           (a)\subseteq(b) \;\Longleftrightarrow\; b\mid a,
           \quad\text{and}\quad
           (a)=(b) \;\Longleftrightarrow\; a \text{ and } b \text{ are associated}.
        \]
        These equivalences underlie the factorisation theory you are studying.
\end{itemize}
% ---------------------------------------------------------------------------
%  Ideal  vs.  Principal Ideal –  What is the difference?
% ---------------------------------------------------------------------------
\begin{definition}[Ideal]
  Let \(R\) be a ring (associative, commutative, with \(1\)).  
  A subset \(I\subseteq R\) is an \emph{ideal} if
  \[
    \begin{aligned}
      &\text{(i) } I \text{ is an additive subgroup of } (R,+), \\[-2pt]
      &\text{(ii) } r\,x \in I \quad\text{for all } r\in R,\;x\in I .
    \end{aligned}
  \]
  In words, ideals are the additive subgroups of \(R\) that are
  ``absorbing’’ under multiplication by arbitrary ring elements.
\end{definition}

\begin{definition}[Principal ideal]
  An ideal \(I\subseteq R\) is \emph{principal} if it can be generated by a
  \emph{single} element; i.e.\ there exists \(a\in R\) with
  \[
      I \;=\; (a)
      \;:=\;
      \{\,r\,a \mid r\in R\,\}.
  \]
  We call \(a\) a \emph{generator} of the ideal and write \(I=(a)\).
\end{definition}

\paragraph{Key distinction.}
\[
  \boxed{\text{Every principal ideal is an ideal,\; but not every ideal is principal.}}
\]

\begin{itemize}
  \item An \textbf{ideal} may need many generators.
        For example, in the polynomial ring \(\Bbb Q[x,y]\) the set
        \(
          J = (x,y) = \{\,f(x,y)\,x + g(x,y)\,y \mid f,g\in\Bbb Q[x,y]\}
        \)
        is an ideal generated by \emph{two} elements; \(J\) is \emph{not} principal.
  \item A \textbf{principal ideal} is the ``simplest’’ kind of ideal:
        it is completely determined by one element.
        In a \emph{principal ideal domain} (PID) \emph{every} ideal happens to be principal
        (e.g.\ \(\Bbb Z\) or \(k[x]\) for a field \(k\)), but this is a special property of the
        underlying ring, not of the notion ``ideal’’ itself.
\end{itemize}

\paragraph{Concrete examples.}

\[
\begin{array}{|l|l|l|}\hline
   \text{Ring }R              & \text{Ideal }I                              & \text{Principal?} \\ \hline
   \Bbb Z                     & (12)=12\Bbb Z                               & \text{Yes (generator }12) \\ 
   \Bbb Z[\sqrt{-5}]         & (2,\,1+\sqrt{-5})                            & \text{No (needs both)} \\ 
   k[x]                       & (x^3)                                       & \text{Yes (generator }x^3) \\
   k[x,y]                     & (x,y)                                       & \text{No (at least two)} \\ \hline
\end{array}
\]

\paragraph{Summary.}
\[
  \text{ideal } = \text{``absorbent additive subgroup''},\quad
  \text{principal ideal } = \text{ideal generated by one element.}
\]
Thinking of ideals as “generalised multiples’’ of a set of elements,
a principal ideal is the case where the entire set of multiples of a
\emph{single} element already captures all the absorption properties we need.
% ---------------------------------------------------------------------------
%  Why the ideal  \(\;(2,\;1+\sqrt{-5})\;\subset\;\Bbb Z[\sqrt{-5}]\)  is \emph{not} principal
% ---------------------------------------------------------------------------

\subsection*{The ring and the ideal}

Throughout this note let
\[
    R \;=\; \Bbb Z[\sqrt{-5}]
    \;=\;
    \{\,a+b\sqrt{-5}\mid a,b\in\Bbb Z\,\}.
\]

Inside \(R\) consider the ideal
\[
    \mathfrak a \;:=\; (\,2,\;1+\sqrt{-5}\,)
    \;=\;
    \bigl\{\,2x + (1+\sqrt{-5})y \mid x,y\in R\,\bigr\}.
\]

We will prove that \(\mathfrak a\) cannot be generated by a single element of \(R\),
i.e.\ \(\mathfrak a\) is \emph{not principal}.

\subsection*{Step 1 – The norm map}

Define the (multiplicative) \emph{norm}
\[
    N\colon R \longrightarrow \Bbb Z_{\ge 0},
    \qquad
    N(a+b\sqrt{-5}) \;=\; a^2 + 5b^2 .
\]

Key properties we will use:

\begin{enumerate}
  \item \(N(\alpha\beta)=N(\alpha)\,N(\beta)\) \quad for all \(\alpha,\beta\in R\);
  \item \(N(\alpha)=1\) \(\Longleftrightarrow\) \(\alpha\) is a \emph{unit}
        (indeed the units of \(R\) are \(\pm1\) only).
\end{enumerate}

\subsection*{Step 2 – Consequences if \(\mathfrak a\) were principal}

Assume, for contradiction, that \(\mathfrak a\) is principal:  
there exists \(\alpha\in R\) such that \(\mathfrak a=(\alpha)\).

Because \(\alpha\) generates \(\mathfrak a\), it must divide \emph{each} generator
of \(\mathfrak a\):

\[
      \alpha \mid 2,
      \qquad
      \alpha \mid 1+\sqrt{-5}.
\]

Write \(2=\alpha\beta\) and \(1+\sqrt{-5}=\alpha\gamma\) for some
\(\beta,\gamma\in R\).
Apply the norm map:

\[
    \underbrace{N(2)}_{=4}=N(\alpha)\,N(\beta),
    \qquad
    \underbrace{N(1+\sqrt{-5})}_{=1^2+5\cdot1^2=6}=N(\alpha)\,N(\gamma).
\]

Hence \(N(\alpha)\) divides both \(4\) and \(6\),
so \(N(\alpha)\) divides \(\gcd(4,6)=2\).
Therefore
\[
    N(\alpha)\in\{1,\,2\}.
\]

\subsection*{Step 3 – Eliminate \(N(\alpha)=1\)}

If \(N(\alpha)=1\) then \(\alpha\) is a \emph{unit}.  
But the ideal generated by a unit is the whole ring:
\(
   (\alpha)=R \neq\mathfrak a
\).
Therefore \(N(\alpha)\ne1\); we must have \(N(\alpha)=2\).

\subsection*{Step 4 – No element of norm \(2\) exists in \(R\)}

Solve \(a^2 + 5b^2 = 2\) with \(a,b\in\Bbb Z\).

\[
\begin{array}{@{}rcll@{}}
b = 0      &\Longrightarrow& a^{2}=2      &\text{(impossible in }\mathbb Z)\\[4pt]
|b| = 1    &\Longrightarrow& a^{2}=2-5=-3 &\text{(impossible)}\\[4pt]
|b|\ge 2   &\Longrightarrow& 5b^{2}\ge 20>2 &\text{(impossible)}
\end{array}
\]

Thus \emph{no} element of \(R\) has norm \(2\).
Consequently \(N(\alpha)=2\) is impossible.

\subsection*{Step 5 – Contradiction}

We deduced that \(N(\alpha)\) must be \(2\), but such an \(\alpha\) does not exist.
Hence our original assumption that \(\mathfrak a\) is principal is false.

\[
    \boxed{\;\mathfrak a = (2,\;1+\sqrt{-5}) \text{ is \emph{not} principal in } \Bbb Z[\sqrt{-5}].\;}
\]

\paragraph{Why this example matters.}
\(\Bbb Z[\sqrt{-5}]\) fails to be a \emph{principal ideal domain} (PID);
indeed, the lack of unique factorisation of \(6\)
\[
   6 = 2\cdot3 = (1+\sqrt{-5})(1-\sqrt{-5})
\]
is intimately connected to the existence of non-principal ideals like \(\mathfrak a\).
In Dedekind–domain language, the ideal class of \(\mathfrak a\) is non-trivial,
witnessing that the class group of \(\Bbb Z[\sqrt{-5}]\) is \(\cong\Bbb Z/2\Bbb Z\).
% ---------------------------------------------------------------------------
%  Definition of a Unit in a Ring
% ---------------------------------------------------------------------------
\begin{definition}[Unit]
  Let \(R\) be a ring with multiplicative identity \(1\neq 0\).
  An element \(u\in R\) is called a \emph{unit} if there exists
  \(v\in R\) such that
  \[
      uv = vu = 1 .
  \]
  In this situation \(v\) is uniquely determined and is denoted \(u^{-1}\).
  The set of all units in \(R\) is written
  \[
      R^{\times}
      \;=\;
      \{\,u\in R \mid \exists\,v\in R,\; uv=1\,\},
  \]
  and forms a group under multiplication, called the \emph{group of units}.
\end{definition}

% ---------------------------------------------------------------------------
%  Illustrative examples
% ---------------------------------------------------------------------------
\begin{itemize}
  \item \(\mathbb Z\):  the units are \(\pm 1\).
  \item \(\mathbb Q,\;\mathbb R,\;\mathbb C\):  
        every non-zero element is a unit, so \(R^{\times}=R\setminus\{0\}\).
  \item Polynomial ring \(k[x]\) over a field \(k\):
        the units are the non-zero constants from \(k\).
  \item Matrix ring \(M_n(k)\):  
        the units are exactly the invertible matrices, i.e.\ those with
        non-zero determinant.
\end{itemize}

% ---------------------------------------------------------------------------
%  Useful facts
% ---------------------------------------------------------------------------
\begin{itemize}
  \item Units are never zero divisors.
  \item In an integral domain, units are the elements with multiplicative
        inverses that stay inside the domain.
  \item A principal ideal \((u)\) generated by a unit \(u\) is the whole
        ring: \((u)=R\).
\end{itemize}
% ---------------------------------------------------------------------------
%  Irreducible Elements, Prime Elements, and Unique Factorisation Domains
% ---------------------------------------------------------------------------
\subsection*{1.  Irreducible vs.\ Prime}

\begin{definition}[Irreducible element]
  Let \(R\) be an integral domain.  
  A non–unit \(a\in R\) is \emph{irreducible} if, whenever
  \(
      a = bc
  \)
  with \(b,c\in R\), then \emph{one of} \(b\) or \(c\) is a unit.
\end{definition}

\begin{definition}[Prime element]
  A non–unit \(p\in R\) is \emph{prime} if
  \[
     p \mid bc \;\Longrightarrow\; p\mid b \text{ or } p\mid c
     \qquad
     (\forall\,b,c\in R).
  \]
\end{definition}

\textbf{Fact.}  
In every integral domain: \(\text{prime}\;\;\Longrightarrow\;\;\text{irreducible}\).  
The converse holds in PIDs (and hence in UFDs), but not in general.

\medskip

\subsection*{2.  Association}

\begin{definition}[Associated elements]
  Two non–zero elements \(a,b\in R\) are \emph{associated},
  written \(a\sim b\), if \(a\mid b\) and \(b\mid a\).
  Equivalently, \(a=bu\) for some unit \(u\in R^{\times}\).
\end{definition}

\textbf{Example in \(\mathbb Z\).}  
\(10\) has the factorisations
\(
   10 = 2\cdot 5 = (-2)\cdot(-5),
\)
but \(\pm2\) are associated, as are \(\pm5\); so “essentially’’ there is
only one list \(\{2,5\}\) of irreducible factors.

\medskip

\subsection*{3.  Unique Factorisation Domains (UFDs)}

\begin{definition}[UFD]
  An integral domain \(R\) is a \emph{unique factorisation domain}
  (UFD) if
  \begin{enumerate}
    \item every non–zero, non–unit \(x\in R\) can be written as
          a finite product of irreducibles:
          \(
             x = a_{1}\cdots a_{n},
          \)
          and
    \item given two such factorizations
          \(
             x = a_{1}\cdots a_{n} = b_{1}\cdots b_{m},
          \)
          we have \(m=n\) and, after re–ordering, each \(a_{i}\) is
          \emph{associated} to the corresponding \(b_{i}\).
  \end{enumerate}
\end{definition}

\paragraph{Key consequences.}
\begin{itemize}
  \item \textbf{Fundamental Theorem of Arithmetic.}  
        \(\mathbb Z\) is a UFD; its irreducibles are the (positive) primes.
  \item \textbf{Every PID is a UFD.}  
        Examples: \(\mathbb Z\), \(k[x]\) for a field \(k\).
  \item \textbf{Not every integral domain is a UFD.}  
        A classical counter-example is
        \(\mathbb Z[\sqrt{-5}]\), where
        \(
           6 = 2\cdot3 = (1+\sqrt{-5})(1-\sqrt{-5})
        \)
        gives two non-associate, irreducible factorizations.
\end{itemize}

\medskip

\subsection*{4.  Summary table}

\[
\begin{array}{|c|c|c|c|}\hline
  \text{Ring} & \text{PID?} & \text{UFD?} & \text{Remarks}\\ \hline
  \mathbb Z                      & \checkmark & \checkmark & \text{classic FTA}\\
  k[x]\,(\text{field }k)         & \checkmark & \checkmark & \deg \text{ gives Euclidean algo.}\\
  \mathbb Z[\sqrt{-5}]          & \times     & \times     & 6=2\cdot3=(1+\sqrt{-5})(1-\sqrt{-5})\\ \hline
\end{array}
\]

\bigskip
\noindent
These notions—unit, irreducible, prime, UFD—form the backbone of
modern commutative algebra and algebraic number theory, generalising
the unique prime factorisation familiar from the integers.
% ---------------------------------------------------------------------------
%  What is a \emph{non–unit}?  —  Definition, intuition, and concrete examples
% ---------------------------------------------------------------------------

\subsection*{1.  Formal definition}

Let $R$ be a (unital) ring.  
An element $x\in R$ is called a \textbf{unit} if there exists
$y\in R$ such that $xy=yx=1$.  
A \textbf{non–unit} is simply an element of $R$ that \emph{fails} to have such
a two–sided multiplicative inverse.

\vspace{4pt}
\[
  x\text{ non–unit }\;\Longleftrightarrow\;
  \nexists\,y\in R\text{ with }xy=yx=1 .
\]

\subsection*{2.  Intuitive picture}

* Units are the “invertible” elements; they behave like $\pm1$ in
  $\Bbb Z$ or non–zero scalars in a field.
* A non–unit lacks a multiplicative inverse \emph{inside the ring}.
  Sometimes that is because $x$ is a \emph{zero–divisor}
  (e.g.\ $\bar 2$ in $\Bbb Z/4\Bbb Z$);  
  sometimes because it merely has the “wrong size’’ (e.g.\ $2$ in $\Bbb Z$,
  or $x$ in $k[x]$).

\medskip

\subsection*{3.  Canonical examples of non–units}

\begin{enumerate}
  \item \textbf{Integers \boldmath$\Bbb Z$.}  
        The units are $\{\pm1\}$.  
        Hence \fbox{$2,\;3,\;6,\dots$ are all non–units}.

  \item \textbf{Polynomial ring \boldmath$k[x]$ over a field $k$.}  
        The units are the non–zero \emph{constants} $k^{\times}$.  
        Any \emph{non–constant} polynomial, say \fbox{$x^2+1$}, is a non–unit.

  \item \textbf{Residue ring \boldmath$\Bbb Z/4\Bbb Z$.}  
        The units are the classes whose representatives are coprime to $4$,
        namely $\{\bar1,\bar3\}$.  
        The class \fbox{$\bar2$} is a non–unit (and also a zero–divisor,
        since $\bar2\cdot\bar2=\bar0$).

  \item \textbf{Matrix ring $M_2(\Bbb R)$.}  
        Units are the invertible matrices ($\det\neq0$).
        A singular matrix such as
        \[
           \begin{pmatrix}1&0\\0&0\end{pmatrix}
        \]
        has determinant $0$ and is therefore a \fbox{non–unit}.
\end{enumerate}

\subsection*{4.  Key facts and common misconceptions}

\begin{itemize}
  \item \emph{All units are non–zero and never zero–divisors.}  
        The converse fails: you can have a non–unit that is \emph{not} a
        zero–divisor (e.g.\ $2$ in $\Bbb Z$).
  \item In an \emph{integral domain}, the group of units is always a proper
        subset of $R\setminus\{0\}$ (unless $R$ happens to be a field).
  \item If $u$ is a unit and $x$ is \emph{any} element, then $ux$ and $xu$
        are units $\Longleftrightarrow$ $x$ is a unit.  
        Thus “multiplying by a unit’’ does not change unit–status; it does,
        however, move you among \emph{associates}.
\end{itemize}

\bigskip
\noindent
\textbf{Take-away.}  
A \emph{non–unit} is simply an element that cannot be “undone’’
multiplicatively \emph{within the ring you are working in}.  
Recognising non–units is essential for factorisation theory, since
\emph{only} non–units qualify as potential factors in decompositions
into irreducibles or primes.
% ---------------------------------------------------------------------------
%  Highest Common Factor (HCF) and Lowest Common Multiple (LCM) in UFDs
% ---------------------------------------------------------------------------
\subsection*{1.  Definitions in an arbitrary integral domain \(\,R\)}

\begin{definition}[Highest common factor]
  For non–zero elements \(a,b\in R\), an element \(d\in R\) is a
  \emph{highest common factor} (HCF) of \(a\) and \(b\) if
  \begin{enumerate}
    \item \(d\mid a\) and \(d\mid b\);
    \item for any \(d'\in R\) with \(d'\mid a\) and \(d'\mid b\), we have \(d'\mid d\).
  \end{enumerate}
\end{definition}

\begin{definition}[Lowest common multiple]
  For non–zero \(a,b\in R\), an element \(c\in R\) is a
  \emph{lowest common multiple} (LCM) of \(a\) and \(b\) if
  \begin{enumerate}
    \item \(a\mid c\) and \(b\mid c\);
    \item for any \(c'\in R\) with \(a\mid c'\) and \(b\mid c'\), we have \(c\mid c'\).
  \end{enumerate}
\end{definition}

\paragraph{Remarks.}
\begin{enumerate}
  \item In a general integral domain an HCF or LCM \emph{need not exist}.
  \item If \(d\) is an HCF, then so is every element \(d'\) \emph{associated} to \(d\)
        (i.e.\ \(d'=du\) for a unit \(u\)).  
        Thus HCF/LCM are well–defined only \emph{up to association}.
\end{enumerate}

% ---------------------------------------------------------------------------
%  2.  Existence & explicit form of an HCF in a UFD
% ---------------------------------------------------------------------------
\subsection*{2.  Existence of HCF in a UFD}

\begin{theorem}
  Let \(R\) be a unique factorisation domain (UFD).
  Then every pair of non–zero elements \(a,b\in R\) has an HCF.
  Moreover, if
  \[
      a \;=\; u\,p_{1}^{\alpha_{1}}\cdots p_{r}^{\alpha_{r}},
      \qquad
      b \;=\; v\,p_{1}^{\beta_{1}}\cdots p_{r}^{\beta_{r}},
  \]
  with \(u,v\) units, the \(p_{i}\) pairwise non–associate irreducibles,
  and exponents \(\alpha_{i},\beta_{i}\ge0\),
  then an HCF is given by
  \[
      \operatorname{HCF}(a,b)
      \;=\;
      p_{1}^{\gamma_{1}}\cdots p_{r}^{\gamma_{r}},
      \qquad
      \gamma_{i} \;=\; \min(\alpha_{i},\beta_{i}).
  \]
\end{theorem}

\begin{proof}
  \textbf{Step 1 – Factorise \(a\) and \(b\).}  
  Because \(R\) is a UFD, each non–zero element admits a factorisation
  into irreducibles that is unique up to units and ordering.
  We therefore write \(a,b\) as above, padding exponents with zeros so the
  same list of pairwise non–associate irreducibles \(\{p_{1},\dots,p_{r}\}\)
  appears in both products.

  \textbf{Step 2 – Candidate for the HCF.}  
  Define
  \[
      d \;:=\; p_{1}^{\gamma_{1}}\cdots p_{r}^{\gamma_{r}}
      \quad\text{with}\quad
      \gamma_{i}=\min(\alpha_{i},\beta_{i}).
  \]

  \textbf{Step 3 – \(d\) divides \(a\) and \(b\).}  
  Since \(\gamma_{i}\le\alpha_{i}\) and \(\gamma_{i}\le\beta_{i}\) for all \(i\),
  each exponent in \(d\) is not larger than the corresponding exponent in
  \(a\) or \(b\), hence \(d\mid a\) and \(d\mid b\).

  \textbf{Step 4 – Maximality property.}  
  Let \(d'\) be any common factor of \(a\) and \(b\).
  Factor \(d'\) (again unique up to units):
  \[
      d' \;=\; w\,p_{1}^{\delta_{1}}\cdots p_{r}^{\delta_{r}} .
  \]
  The divisibility \(d'\mid a\) forces \(\delta_{i}\le\alpha_{i}\), and
  \(d'\mid b\) forces \(\delta_{i}\le\beta_{i}\); therefore
  \(\delta_{i}\le\gamma_{i}\) for every \(i\).
  Consequently \(d'\mid d\).

  \textbf{Step 5 – Conclusion.}  
  The two properties above show that \(d\) satisfies the defining
  universal property of an HCF.  
  Any other HCF differs from \(d\) by a unit, i.e.\ is \emph{associated} to \(d\).
\end{proof}

% ---------------------------------------------------------------------------
%  3.  Worked example in \(\mathbb Z\)
% ---------------------------------------------------------------------------
\subsection*{3.  Example in \(\mathbb Z\)}

Let \(a = 2^{3}\,3^{2}\,5 = 360\) and \(b = 2^{4}\,3\,7 = 336\).
\[
  \begin{aligned}
     \operatorname{HCF}(a,b)
     &= 2^{\min(3,4)}\,3^{\min(2,1)}\,5^{\min(1,0)}\,7^{\min(0,1)}
      = 2^{3}\,3^{1}\,5^{0}\,7^{0} \\
     &= 2^{3}\,3 = 24 .
  \end{aligned}
\]
Indeed \(24\mid360\) and \(24\mid336\); any larger common divisor would
necessarily contain a higher power of either \(2\) or \(3\), contradicting
minimality.

% ---------------------------------------------------------------------------
%  4.  Non–existence in a non–UFD
% ---------------------------------------------------------------------------
\subsection*{4.  When an HCF can fail to exist}

In \(R=\Bbb Z[\sqrt{-5}]\) (not a UFD) take
\[
    a = 2,\qquad b = 1+\sqrt{-5}.
\]
The only common divisors of \(a\) and \(b\) are the \emph{units}
\(\pm1\), so there is no \emph{non–unit} element satisfying the
maximality property.  Thus HCF\((a,b)\) does \emph{not} exist in this
ring, illustrating the necessity of the UFD hypothesis in the theorem.

% ---------------------------------------------------------------------------
%  Least Common Multiple in a UFD  +  Euclidean Domains
% ---------------------------------------------------------------------------
\subsection*{1.  Least Common Multiple (LCM) in a UFD}

\begin{proposition}
  Let \(R\) be a \emph{unique factorisation domain} (UFD) and let
  \(a,b\in R\setminus\{0\}\).  
  Then \(a\) and \(b\) possess a least common multiple.
  Writing the unique factorisations
  \[
     a \;=\; u\,p_{1}^{\alpha_{1}}\cdots p_{r}^{\alpha_{r}},
     \qquad
     b \;=\; v\,p_{1}^{\beta_{1}}\cdots p_{r}^{\beta_{r}},
  \]
  where \(u,v\) are units, the \(p_{i}\) are pairwise non–associate irreducibles
  and \(\alpha_{i},\beta_{i}\ge0\), an LCM is given by
  \[
       \operatorname{LCM}(a,b)
       \;=\;
       p_{1}^{\gamma_{1}}\cdots p_{r}^{\gamma_{r}},
       \qquad
       \gamma_{i} \;=\; \max(\alpha_{i},\beta_{i}).
  \]
\end{proposition}

\begin{proof}[Step–by–step outline]
  \textbf{1.~Factorise \(a\) and \(b\).}\;
  Uniqueness of factorisation in a UFD lets us expand both elements using the
  same irreducible list \(\{p_{1},\dots,p_{r}\}\), padding missing exponents
  with zeros as needed.\\[-2pt]

  \textbf{2.~Candidate \(c\) for the LCM.}\;
  Define
  \(
       c := p_{1}^{\gamma_{1}}\cdots p_{r}^{\gamma_{r}}
  \)
  with \(\gamma_{i}=\max(\alpha_{i},\beta_{i})\).

  \textbf{3.~Show \(a,b\mid c\).}\;
  Because \(\alpha_{i}\le\gamma_{i}\) and \(\beta_{i}\le\gamma_{i}\),
  each prime power in \(a\) or \(b\) divides the corresponding power in \(c\);
  hence \(a\mid c\) and \(b\mid c\).

  \textbf{4.~Minimality.}\;
  Let \(c'\) be any common multiple of \(a\) and \(b\).
  Factor
  \(c' = w\,p_{1}^{\delta_{1}}\cdots p_{r}^{\delta_{r}}\).
  The divisibilities \(a\mid c'\) and \(b\mid c'\) imply
  \(\alpha_{i}\le\delta_{i}\) and \(\beta_{i}\le\delta_{i}\) for all \(i\),
  whence \(\gamma_{i}\le\delta_{i}\).
  Thus \(c\mid c'\).  \\[-2pt]

  \textbf{5.~Conclusion.}\;
  The element \(c\) satisfies the universal property of an LCM, and any other
  LCM is associated to \(c\).
\end{proof}

\paragraph{Quick corollary.}
If \(a\) is a \emph{unit}, then \(\operatorname{HCF}(a,b)=1\) and
\(\operatorname{LCM}(a,b)=b\) for all \(b\neq0\), because units neither
affect \(\min\) nor \(\max\) of the exponent lists.

\bigskip
% ---------------------------------------------------------------------------
%  2.  Euclidean Domains
% ---------------------------------------------------------------------------
\subsection*{2.  Euclidean domains}

\begin{definition}[Euclidean domain]
  An integral domain \(R\) is \emph{Euclidean} if there exists a function
  \(\varphi:R\setminus\{0\}\to\Bbb N\cup\{0\}\) (called a \emph{Euclidean
  valuation}) such that
  \begin{enumerate}
    \item \(\varphi(ab)\;\ge\;\varphi(a)\) for all non–zero \(a,b\in R\);
    \item For any \(a,b\in R\) with \(b\neq0\) there exist \(q,r\in R\) satisfying
          \[
             a = bq + r,
             \qquad
             \text{either } r = 0 \text{ or } \varphi(r)<\varphi(b).
          \]
  \end{enumerate}
\end{definition}

\paragraph{Intuition.}
A Euclidean valuation allows a \emph{division algorithm} analogous to that for
integers, enabling a gcd–style algorithm that rapidly computes HCFs (and
hence LCMs) without full factorisation.

\begin{itemize}
  \item \(\Bbb Z\): \(\varphi(n)=|n|\).
  \item Polynomial ring \(k[x]\) over a field \(k\):
        \(\varphi(f)=\deg f\).
  \item Every Euclidean domain is a PID \(\Longrightarrow\) a UFD,
        but the converses fail in general (\(\Bbb Z[\tfrac{1+\sqrt{-19}}{2}]\)
        is a PID but not Euclidean).
\end{itemize}

\bigskip
% ---------------------------------------------------------------------------
%  3.  Worked example  (Euclidean algorithm in \(\Bbb Z\))
% ---------------------------------------------------------------------------
\subsection*{3.  Example: computing \(\gcd(360,336)\) in \(\Bbb Z\)}

\[
\begin{aligned}
360 &= 1\cdot336 + 24\\
336 &= 14\cdot24 + 0
\end{aligned}
\quad\Longrightarrow\quad
\gcd(360,336)=24.
\]

The quotient–remainder steps terminate because each remainder
has strictly smaller absolute value than its divisor, precisely condition~(2)
of the Euclidean definition.

% ---------------------------------------------------------------------------
%  End
% ---------------------------------------------------------------------------
% ---------------------------------------------------------------------------
%  Lemma –  Recasting the Euclidean “division” axiom
% ---------------------------------------------------------------------------
\subsection*{Lemma}
Let \(R\) be an integral domain equipped with a Euclidean valuation
\(\varphi:R\setminus\{0\}\to\Bbb N\cup\{0\}\).
The standard \emph{division axiom}

\[
\tag{2}\label{E2}
\forall\,a,b\in R\;(b\neq0)\;\;\exists\,q,r\in R
\quad
a = bq + r,
\quad
\text{with }r=0\text{ or }\varphi(r)<\varphi(b)
\]

is equivalent to the following alternative formulation:

\[
\tag{2'}\label{E2p}
\forall\,a,b\in R\setminus\{0\},\;
\varphi(a)\ge\varphi(b)
\;\Longrightarrow\;
\exists\,c\in R
\text{ such that either } a = bc
\text{ or }\varphi(a-bc)<\varphi(a).
\]

\bigskip
\begin{proof}
\textbf{\eqref{E2} \(\Longrightarrow\) \eqref{E2p}.}\;
Assume \eqref{E2}.  
Take \(a,b\in R\setminus\{0\}\) with \(\varphi(a)\ge\varphi(b)\).
Apply \eqref{E2} to obtain \(q,r\) with \(a=bq+r\) and either \(r=0\) or
\(\varphi(r)<\varphi(b)\le\varphi(a)\).
\begin{itemize}
  \item If \(r=0\) set \(c:=q\); then \(a=bc\).
  \item Otherwise set \(c:=q\) again; we have
        \(
           \varphi(a-bc)
           =\varphi(r)
           <\varphi(a).
        \)
\end{itemize}
Thus \eqref{E2p} holds.

\medskip
\textbf{\eqref{E2p} \(\Longrightarrow\) \eqref{E2}.}\;
Fix \(a,b\in R\) with \(b\neq0\).
If \(b\mid a\) we are done with \(q=\tfrac{a}{b}\) and \(r=0\).
Otherwise \(b\nmid a\).

\smallskip
Choose \(q\in R\) such that 
\[
    \varphi(a-bq) \;\text{ is minimal among }\;
    \{\varphi(a-bc)\mid c\in R\}.
\]
By minimality we have \(b\nmid (a-bq)\) (else remainder would be \(0\)).
If \(\varphi(a-bq)\ge\varphi(b)\) then hypothesis \eqref{E2p}
(with \(c=q\)) would give some \(c'\) such that
\(\varphi\bigl(a-bc'\bigr)<\varphi(a-bq)\),
contradicting minimality.
Hence \(\varphi(r)<\varphi(b)\) for \(r:=a-bq\).

Thus we have written \(a=bq+r\) with either \(r=0\)
(if \(b\mid a\)) or \(\varphi(r)<\varphi(b)\), which is exactly \eqref{E2}.
\end{proof} 
% ---------------------------------------------------------------------------
%  Explaining the “Remarks” after the definition of a Euclidean domain
% ---------------------------------------------------------------------------
\paragraph*{Remark 1 — Why \(\varphi(a)=|a|\) on \(\mathbb Z\) is the prototype.}
\begin{itemize}
  \item For each non–zero integer \(a\) set \(\varphi(a)=|a|\in\mathbb N\cup\{0\}\).
        \begin{enumerate}
          \item \emph{Monotonicity.}\;
                \(|ab|\ge|a|\) for all \(a,b\neq0\), so condition (1) of a
                Euclidean valuation holds.
          \item \emph{Division algorithm.}\;
                Given integers \(a,b\neq0\) the usual quotient–remainder
                theorem produces \(q,r\) with \(a=bq+r\) and
                \(0\le r<|b|\).
                Translating, \(r=0\) or \(\varphi(r)<\varphi(b)\),
                which is condition (2).
        \end{enumerate}
  \item Hence \(\mathbb Z\) satisfies both axioms with this \(\varphi\); it is a
        \textbf{Euclidean domain}.  
        All other Euclidean valuations generalise the same “size–
        decreases-when-you-take-remainders’’ idea.
\end{itemize}

\bigskip
\paragraph*{Remark 2 — Why allow the value \(0\) in the codomain.}
\begin{itemize}
  \item If we restricted \(\varphi\) to land in \(\mathbb N=\{1,2,3,\dots\}\),
        some perfectly serviceable valuations would be ruled out.
        \[
           \varphi:R\setminus\{0\}\;\to\;\mathbb N\cup\{0\}
        \]
        is therefore more flexible.
  \item Two common situations where the value \(0\) is indispensable:
        \begin{enumerate}
          \item \textbf{Polynomial rings.}\;
                For \(k[x]\) the Euclidean valuation \(\varphi(f)=\deg f\)
                gives \(\varphi(f)=0\) precisely for \emph{non–zero constants}.
          \item \textbf{Fields.}\;
                If \(R\) is already a field, the
                “trivial’’ valuation \(\varphi\equiv0\) fulfils both axioms,
                making every field a (degenerate) Euclidean domain.
        \end{enumerate}
        Allowing \(0\) therefore \emph{enlarges} the class of rings that fit
        the definition without harming the classical examples.
\end{itemize}
% ------------------------------------------------------------
%  Theorem:  F field  ⇒  F[X] Euclidean
% ------------------------------------------------------------
\begin{theorem}
  Let \(F\) be a field.  
  Then the polynomial ring \(F[X]\) is a \emph{Euclidean domain}.
  \end{theorem}
  
  \begin{proof}
  We must exhibit a function
  \[
    \varphi : F[X]\setminus\{0\}\;\longrightarrow\;\mathbb N\cup\{0\}
  \]
  satisfying the two Euclidean axioms.
  
  \medskip
  \textbf{Step 1.  Define the valuation.}  
  For a non-zero polynomial \(f\in F[X]\) set
  \[
     \boxed{\; \varphi(f) := \deg(f). \;}
  \]
  
  \medskip
  \textbf{Step 2.  Verify Axiom (1)  (\(\varphi(fg) \ge \varphi(f)\)).}
  
  Because \(F\) is a field, \(F[X]\) is an integral domain, so
  \[
     \deg(fg) = \deg(f) + \deg(g) \;\ge\; \deg(f)
     \quad\Longrightarrow\quad
     \varphi(fg)\;\ge\;\varphi(f)
     \quad\forall\,f,g\in F[X]\setminus\{0\}.
  \]
  
  \medskip
  \textbf{Step 3.  Verify Axiom (($2^\prime$  ))}  
  (the equivalent “remainder-drop” version of the division axiom):
  
  \[
     \forall\,f,g\in F[X]\setminus\{0\},\;
     \varphi(f)\ge\varphi(g)
     \;\Longrightarrow\;
     \exists\,c\in F[X]\;
     \text{with either } f = gc \text{ or } \varphi(f-gc)<\varphi(f).
  \]
  
  \emph{Proof of ($2^\prime$  ).}  
  Write
  \[
    f = a_0 + a_1X + \dots + a_nX^{\,n},
    \qquad
    g = b_0 + b_1X + \dots + b_mX^{\,m},
  \]
  where \(n=\deg(f),\;m=\deg(g)\) and \(a_n,b_m\neq0\).
  Assume \(\varphi(f)=n\ge m=\varphi(g)\).
  Then \(n-m\ge 0\) and the monomial \(X^{\,n-m}\in F[X]\).
  
  Consider
  \[
    c \;:=\; a_n\,b_m^{-1}\,X^{\,n-m}\in F[X],
  \]
  which is well-defined because \(b_m\) is invertible in the field \(F\).
  The leading term of \(gc\) is
  \(
      b_m\,X^{\,m}\cdot a_n\,b_m^{-1}\,X^{\,n-m} = a_n\,X^{\,n},
  \)
  exactly matching the leading term of \(f\).
  Hence
  \[
     \deg\!\bigl(f-gc\bigr)\;<\;\deg(f).
  \]
  Consequently
  \(
     \varphi(f-gc) = \deg(f-gc) < \deg(f) = \varphi(f),
  \)
  so property ($2^\prime$  ) is satisfied.
    %%%%%%%%%%%%%%%%%%%%%%%%%%%%%%%%%%%%%%%%%%%%%%%%%%%%%%%%%%%%%%%%%%%%%%%%%%%%
%  Where does the line
%     “$X^{\,n-m}b_m^{-1}a_n g$ has leading term $a_nX^n$”
%  come from?
%%%%%%%%%%%%%%%%%%%%%%%%%%%%%%%%%%%%%%%%%%%%%%%%%%%%%%%%%%%%%%%%%%%%%%%%%%%%

\textbf{Setup.}  
Write the polynomials with their leading coefficients:

\[
   f = a_0 + a_1X + \dots + a_nX^{\,n},
   \qquad
   g = b_0 + b_1X + \dots + b_mX^{\,m},
\]
where \(a_n\neq0,\;b_m\neq0\) and \(n=\deg f,\;m=\deg g\).

Assume \(\varphi(f)=\deg(f)\ge\varphi(g)=\deg(g)\); hence \(n\ge m\).

\bigskip
\textbf{1.  Shift $g$ up to degree $n$.}

Because \(n-m\ge0\), the monomial \(X^{\,n-m}\) lies in \(F[X]\) and

\[
   X^{\,n-m}g
   \quad\text{has degree}\quad
   m + (n-m) = n.
\]

So \(X^{\,n-m}g\) is “high enough’’ to compete with \(f\)’s leading term.

\bigskip
\textbf{2.  Match the leading coefficient.}

The leading term of \(X^{\,n-m}g\) is \(b_mX^{\,n}\).
Since \(F\) is a \emph{field}, \(b_m\) is \underline{invertible}.  
Multiplying by \(b_m^{-1}a_n\) rescales that leading term to
\(a_nX^{\,n}\):

\[
   \underbrace{\bigl(b_m^{-1}a_n\bigr)}_{\text{scalar in }F}
   \times
   b_mX^{\,n}
   \;=\;
   a_nX^{\,n}.
\]

\bigskip
\textbf{3.  Define $c$ and check the claim.}

Set
\[
     c \;:=\; b_m^{-1}a_n\,X^{\,n-m}\in F[X].
\]
Then
\[
     cg
     \;=\;
     \bigl(b_m^{-1}a_n\,X^{\,n-m}\bigr)g
\]
indeed has leading term \(a_nX^{\,n}\).

\bigskip
\textbf{4.  Consequence for the degree drop.}

Subtracting this multiple from \(f\) cancels the highest-degree term:

\[
   f - cg
   \quad\text{has no }X^{\,n}\text{ term}
   \;\Longrightarrow\;
   \deg(f-cg) < \deg(f).
\]

This is exactly the inequality
\(
   \varphi(f-cg) < \varphi(f)
\)
used to verify Euclidean property $(2^\prime )$.

\bigskip
\textbf{Mnemonic.}  
“\emph{Shift \(g\) to match the degree, then scale by the ratio of leading
coefficients.}’’
Because \(F\) is a field, that ratio \(a_n/b_m\) lives inside \(F\),
so the construction stays within \(F[X]\).
  \medskip
  \textbf{Step 4.  Conclude Axiom (2).}  
  Since ($2^\prime$  ) is equivalent to the ordinary division axiom (lemma proved
  earlier), \(F[X]\) admits quotient-remainder division with the degree
  strictly dropping at each step.
  
  \medskip
  \textbf{Step 5.  Finish.}  
  Both Euclidean axioms hold for \(\varphi(f)=\deg(f)\); therefore \(F[X]\)
  is Euclidean.
  \end{proof}
  
  \begin{remark}
  The argument uses that \(b_m\neq0\) has a multiplicative inverse
  \(b_m^{-1}\) in \(F\).  If \(F\) were merely a commutative ring without
  every non-zero element invertible, the construction of \(c\) would fail
  and \(F[X]\) need not be Euclidean.
  \end{remark}
  % ---------------------------------------------------------------------------
%  Euclidean Domains  ▸  Existence of the Highest Common Factor (HCF)
%                      and the Bézout Representation
% ---------------------------------------------------------------------------
\begin{theorem}
  Let \(R\) be a Euclidean domain with Euclidean valuation
  \(\varphi : R\setminus\{0\}\to\mathbb N\cup\{0\}\).
  For any non-zero \(a,b\in R\) there exists a highest common factor
  \[
     (a,b)\;=\;\operatorname{HCF}(a,b)\;\in R,
  \]
  and it can be written in \emph{Bézout form}
  \[
     (a,b)\;=\;au + bv
     \quad\text{for some }u,v\in R .
  \]
  \end{theorem}
  
  \begin{proof}
  \textbf{1.  Start the Euclidean algorithm.}  
  Without loss of generality assume \(\varphi(a)\ge\varphi(b)\).
  Apply the division axiom (property~(2)):
  
  \[
     a \;=\; bq_{1} + r_{1},
     \qquad
     \text{where either } r_{1}=0 \text{ or } \varphi(r_{1})<\varphi(b).
  \]
  
  \smallskip
  \textbf{2.  If \(r_{1}=0\) we are done.}  
  Then \(b\mid a\) so \((a,b)=b\), and setting \(u=0,\;v=1\) gives the
  Bézout expression.
  
  \smallskip
  \textbf{3.  Otherwise continue recursively.}  
  Repeat the division step:
  
  \[
     b \;=\; q_{2}r_{1} + r_{2},
     \qquad
     r_{2}=0 \text{ or } \varphi(r_{2})<\varphi(r_{1}),
  \]
  \[
     r_{1} \;=\; q_{3}r_{2} + r_{3}, \quad \dots
  \]
  
  At each stage we obtain a strictly decreasing chain
  \[
     \varphi(b) \;>\; \varphi(r_{1}) \;>\; \varphi(r_{2}) \;>\; \dots
  \]
  in the well-ordered set \(\mathbb N\cup\{0\}\), so the process
  \emph{must} terminate after finitely many steps:
  \[
     r_{n-2} \;=\; q_{n}r_{n-1} + r_{n},
     \qquad
     r_{n}=0.
  \]
  
  \smallskip
  \textbf{4.  The last non-zero remainder \(r_{n-1}\) divides both \(a\) and \(b\).}  
  Inductively from the recursion one shows \(r_{k}\mid r_{k-1}\) for
  all \(k\).  In particular \(r_{n-1}\mid r_{n-2}\mid\cdots\mid r_{1}\mid b\)
  and \(r_{n-1}\mid r_{n-2}\mid\cdots\mid a\); hence \(r_{n-1}\) is a
  \emph{common} divisor of \(a\) and \(b\).
  % ---------------------------------------------------------------------------
%  Expanded explanation of the “divisibility–chain’’ step
%  inside the Euclidean–algorithm proof of  HCF$(a,b)$
% ---------------------------------------------------------------------------

We keep the notation from the proof:

\[
\begin{aligned}
   a      &= bq_{1}+r_{1}, &          &0\le\varphi(r_{1})<\varphi(b),\\
   b      &= q_{2}r_{1}+r_{2}, &      &0\le\varphi(r_{2})<\varphi(r_{1}),\\
          &\vdots& &\\
   r_{n-3}&= q_{n-1}r_{n-2}+r_{n-1}, & &0\le\varphi(r_{n-1})<\varphi(r_{n-2}),\\
   r_{n-2}&= q_{n}\,r_{n-1}+r_{n},   & &0\le\varphi(r_{n})  <\varphi(r_{n-1}),\\
   r_{n-1}&= q_{n+1}r_{n}+0 .
\end{aligned}
\]

Hence \(r_{n}\neq0\) is the \emph{last non–zero remainder}.

\bigskip
\subsection*{1.\;Why \(r_{n}\mid r_{n-1}\)}

The final line is \(r_{n-1}=q_{n+1}r_{n}\).  
This is literally the definition of divisibility, so
\[
   r_{n}\;\mid\;r_{n-1}.
\]

\bigskip
\subsection*{2.\;Backward induction: \(r_{n}\mid r_{k}\) for all \(k\le n-1\)}

\textbf{Inductive claim.}  
If \(r_{n}\mid r_{k}\) and \(r_{n}\mid r_{k+1}\), then \(r_{n}\mid r_{k-1}\).

\emph{Proof of the claim.}  
From the recurrence
\(
   r_{k-1}=q_{k}r_{k}+r_{k+1},
\)
any common divisor of \(r_{k}\) and \(r_{k+1}\) also divides \(r_{k-1}\).

\medskip
\textbf{Base step.}  
We already have \(r_{n}\mid r_{n}\) (trivial) and \(r_{n}\mid r_{n-1}\).

\textbf{Inductive step.}  
Assume \(r_{n}\mid r_{k}\) and \(r_{n}\mid r_{k+1}\).
Apply the claim to deduce \(r_{n}\mid r_{k-1}\).
By descending induction this holds for every remainder \(r_{i}\) and in
particular
\[
   r_{n}\mid r_{1},
   \qquad
   r_{n}\mid b.
\]
% ---------------------------------------------------------------------------
%  Detailed expansion of the \emph{inductive step} in the divisibility chain
%  (showing $\,r_n\mid r_k\,$ for every $k\le n-1$)
% ---------------------------------------------------------------------------

\subsubsection*{Recap of the recurrence}
For each $i\ge1$ the Euclidean algorithm furnishes
\[
   r_{i-1}=q_{i+1}r_{i}+r_{i+1},
   \qquad
   0\le \varphi(r_{i+1})<\varphi(r_{i}),
\]
with $r_{n}$ the \emph{last non–zero} remainder.

\bigskip
\textbf{Inductive claim.}\;
If $r_{n}\mid r_{k}$ \emph{and} $r_{n}\mid r_{k+1}$, then $r_{n}\mid r_{k-1}$.

\begin{proof}[Proof of the claim]
From the recurrence for index $k$ we have
\[
   r_{k-1}=q_{k+1}r_{k}+r_{k+1}.
\]
Any element that divides \emph{both} $r_{k}$ and $r_{k+1}$ must also
divide their $R$–linear combination $q_{k+1}r_{k}+r_{k+1}=r_{k-1}$.
Hence $r_{n}\mid r_{k-1}$.
\end{proof}

\bigskip
\textbf{Establishing the induction}

\begin{enumerate}
  \item \emph{Base cases.}
        \[
          r_{n}\mid r_{n}\quad\text{(trivial)},\qquad
          r_{n}\mid r_{n-1}\quad\text{(because }r_{n-1}=q_{n+1}r_{n}).
        \]
  \item \emph{Inductive step.}\;
        Assume $r_{n}\mid r_{k}$ and $r_{n}\mid r_{k+1}$ for some
        $2\le k\le n-1$.  
        The claim then gives $r_{n}\mid r_{k-1}$.
\end{enumerate}

By descending induction this property propagates all the way down the
chain of remainders:

\[
   r_{n}\mid r_{n-1}\;\Rightarrow\;
   r_{n}\mid r_{n-2}\;\Rightarrow\;\dots\;\Rightarrow\;
   r_{n}\mid r_{1}.
\]

Since $r_{1}$ divides $b$ and the first division equation
$a=bq_{1}+r_{1}$ shows that $r_{n}\mid a$ as well, we conclude

\[
   r_{n}\;\mid\; a
   \quad\text{and}\quad
   r_{n}\;\mid\; b.
\]

Thus $r_{n}$ is a common divisor of $a$ and $b$; the subsequent Bézout
argument completes the proof that it is the \emph{highest} common
factor.
% ---------------------------------------------------------------------------
%  Why the Euclidean Algorithm Must Terminate
% ---------------------------------------------------------------------------
\section*{Setup}

Let \(R\) be a Euclidean domain with valuation  
\[
   \varphi : R\setminus\{0\}\longrightarrow \mathbb N\cup\{0\},
   \qquad
   \text{and let } a,b\in R\;(b\ne0).
\]
At each step the algorithm produces
\[
   a_0 = a,\quad a_1 = b,\quad 
   a_{k-1} \;=\; q_{k}a_k + a_{k+1},
   \qquad
   0\;\le\;\varphi(a_{k+1})\;<\;\varphi(a_{k}),
\]
with \(a_{n+1}=0\) signalling termination.

\medskip
\section*{Key observation}

\[
   \boxed{\;
      \varphi(a_{0}) > \varphi(a_{1}) > \varphi(a_{2}) > \dots
   \;}
\]

\begin{align}
\text{Why?}\quad
  &\varphi(a_{k+1}) < \varphi(a_{k})
    &&\text{by Euclidean axiom (division with \(\varphi\)-drop).}
\end{align}

\medskip
\section*{Termination proof}

\begin{enumerate}
  \item The image of \(\varphi\) lies in \(\mathbb N\cup\{0\}\),  
        a \emph{well–ordered} set:  
        every non-empty subset has a least element, and
        there is no infinite strictly decreasing sequence.
  \item The chain  
        \(\varphi(a_{0})>\varphi(a_{1})>\varphi(a_{2})>\dots\)  
        is strictly decreasing inside \(\mathbb N\cup\{0\}\).
  \item Well-ordering therefore forces the chain to be \emph{finite}.  
        Equivalently, some remainder must be \(0\) after finitely many steps.
\end{enumerate}

\medskip
\section*{Concrete integer illustration}

For \(\mathbb Z\) take \(\varphi(n)=|n|\).

\[
   \begin{aligned}
   1071 &= 2\cdot 462 + 147 \quad &\Longrightarrow& \;|147|<|462|,\\
    462 &= 3\cdot 147 +  21 \quad &\Longrightarrow& \;| 21|<|147|,\\
    147 &= 7\cdot  21 +   0 \quad &\Longrightarrow& \; 0 < |21|.
   \end{aligned}
\]

Absolute values drop from \(462\) to \(147\) to \(21\) and cannot keep
decreasing indefinitely inside \(\mathbb N\cup\{0\}\); the algorithm
must stop—here after three steps.

\medskip
\section*{Take-away}

The Euclidean algorithm terminates because its controlling
measure \(\varphi\) produces a strictly decreasing sequence of
\emph{natural numbers}.  
Since the naturals cannot sustain an infinite descent, the process
reaches remainder \(0\) in finitely many steps.
\bigskip
\subsection*{3.\;Why \(r_{n}\mid a\)}

The first division equation is \(a=bq_{1}+r_{1}\).
Since \(r_{n}\mid b\) and \(r_{n}\mid r_{1}\), we get \(r_{n}\mid a\).

Thus \(r_{n}\) divides \emph{both} \(a\) and \(b\); i.e.\ it is a
\emph{common divisor}.

\bigskip
\subsection*{4.\;Why \(r_{n}\) is the \textbf{highest} common factor}

Let \(d\) be \emph{any} common divisor of \(a\) and \(b\).
Because \(d\mid a\) and \(d\mid b\), every Bézout combination
\(ua+vb\) is also divisible by \(d\).
From the earlier part of the proof we already expressed
\[
   r_{n}=ua+vb.
\]
Hence \(d\mid r_{n}\).

Therefore \(r_{n}\) is divisible by every common divisor and is itself a
common divisor; it satisfies the universal property of an \emph{HCF}:

\[
     \boxed{\;\operatorname{HCF}(a,b)=r_{n}=ua+vb\;}
\]

\bigskip
\paragraph{Key take-aways.}
\begin{enumerate}
  \item The step \(r_{n}\mid r_{n-1}\) comes directly from the \emph{last}
        division equation.
  \item Successive backward substitution propagates this divisibility
        all the way to \(b\) and \(r_{1}\), and hence to \(a\).
  \item Bézout’s identity finishes the “maximality’’ argument.
\end{enumerate}
  
  \smallskip
  \textbf{5.  Maximality $\implies$   \(r_{n-1}\) is the HCF.}  
  Let \(d\in R\) be any common divisor of \(a\) and \(b\).
  Tracing the remainder equations backwards shows \(d\mid r_{1},d\mid r_{2},
  \dots,d\mid r_{n-1}\).  Thus \(d\mid r_{n-1}\).
  Therefore \(r_{n-1}\) satisfies the universal property of an HCF:
  \[
     (a,b)=r_{n-1}.
  \]
  
  \smallskip
  \textbf{6.  Bézout representation \(r_{n-1}=au+bv\).}  
  Write the remainder equations explicitly:
  
  \[
  \begin{aligned}
     r_{1} &= a - bq_{1},\\
     r_{2} &= b - q_{2}r_{1},\\
     r_{3} &= r_{1} - q_{3}r_{2},\\[-2pt]
           &\ \vdots
  \end{aligned}
  \]
  
  An induction on the index shows that every remainder \(r_{k}\) is an
  \(R\)-linear combination of \(a\) and \(b\).
  In particular, for \(k=n-1\) there exist \(u,v\in R\) such that
  \(
     r_{n-1}=au+bv.
  \)
  Hence
  \[
     \boxed{\;(a,b)=au+bv\;}
  \]
  as required.
  \end{proof}
  % ---------------------------------------------------------------------------
%  Concrete integer example of the Euclidean algorithm
%  (gcd and explicit Bézout coefficients)
% ---------------------------------------------------------------------------
\[
  \boxed{\;\gcd(1071,\;462)\;}
  \]
  
  \begin{align}
  1071 &=  2\cdot 462 + 147    &\qquad&(1) \\[2pt]
  462  &=  3\cdot 147 +  21    &      &(2) \\[2pt]
  147  &=  7\cdot  21 +   0    &      &(3)
  \end{align}
  
  \smallskip
  Because the last non–zero remainder is \(21\), we have
  \[
  \gcd(1071,462)=21.
  \]
  
  \bigskip
  % ---------------------------------------------------------------------------
  %  Extended Euclidean algorithm  –  back–substitution
  % ---------------------------------------------------------------------------
  From (2):\quad \(21 = 462 - 3\cdot 147\).
  
  From (1):\quad \(147 = 1071 - 2\cdot 462\).
  
  Substitute the expression for \(147\) into the first line:
  \begin{align*}
  21
    &= 462 - 3\bigl(1071 - 2\cdot 462\bigr) \\[2pt]
    &= 462 - 3\cdot1071 + 6\cdot462 \\[2pt]
    &= -3\cdot1071 + 7\cdot462 .
  \end{align*}
  
  \[
  \boxed{\;21 = (-3)\,1071 \;+\; 7\,462\;}
  \]
  
  Thus Bézout coefficients are \(u=-3,\;v=7\):
  \[
  1071(-3)+462(7)=21.
  \]
  
  % ---------------------------------------------------------------------------
  %  Sanity check
  % ---------------------------------------------------------------------------
  \begin{align*}
  -3\cdot1071 &= -3213,\\
  7\cdot462   &=  3234,\\
              &\underline{\smash{+}}\ \phantom{21}\\[-4pt]
              &\; 21\;\checkmark
  \end{align*}
  
  \bigskip
  \textbf{Summary.}\;
  The Euclidean algorithm produced the sequence of remainders
  \(147,21,0\) and terminated because each remainder’s absolute value
  is \emph{strictly} smaller than its divisor.  
  Back-substitution then expressed the gcd as a linear
  combination \(1071u+462v\), confirming Bézout’s theorem.
  \begin{remark}
  The constructive part of the proof \emph{is} the classical
  \textbf{Euclidean (extended) algorithm}.  
  The sequence of quotients \(q_{i}\) yields the coefficients
  \(u,v\) by back-substitution.
  \end{remark}
  % ---------------------------------------------------------------------------
%  Why every remainder \(r_i\) is a Bézout combination of \(a\) and \(b\)
%  (expanded induction inside the Euclidean–algorithm proof)
% ---------------------------------------------------------------------------
\subsection*{Goal}
Show that for every index \(i\) produced by the Euclidean algorithm  
there exist coefficients \(u_i,v_i\in R\) such that
\[
      r_i \;=\; u_i\,a \;+\; v_i\,b .
\]
In particular, the \emph{last} non–zero remainder \(r_n\) satisfies  
\(r_n = ua+vb\) and hence is the Bézout representation of
\(\operatorname{HCF}(a,b)\).

\bigskip
\subsection*{Euclidean recursion recalled}
\[
\begin{aligned}
   r_0 &= a,\\[2pt]
   r_1 &= b,\\[4pt]
   r_{i-1} &= q_{i+1}\,r_{i} \;+\; r_{i+1}, 
   \qquad (0\le i\le n-1),
\end{aligned}\tag{$\star$}
\]
with the convention \(r_{\,n+1}=0\).

\bigskip
\subsection*{Induction on the index \(i\)}

\paragraph{Base cases.}
\[
    r_0 = 1\cdot a + 0\cdot b,
    \qquad
    r_1 = 0\cdot a + 1\cdot b ,
\]
so the desired representation holds for \(i=0,1\).

\paragraph{Inductive hypothesis.}
Assume for some \(i\ge2\) we already have
\[
   r_{i-1}=u_{i-1}a+v_{i-1}b,
   \qquad
   r_{i-2}=u_{i-2}a+v_{i-2}b .
\]

\paragraph{Inductive step.}
From the recurrence $(\star)$ we know
\[
   r_i = -\,q_i\,r_{i-1} + r_{i-2}.
\]
Insert the inductive expressions for \(r_{i-1}\) and \(r_{i-2}\):
\[
\begin{aligned}
   r_i
   &= -q_i\bigl(u_{i-1}a+v_{i-1}b\bigr) 
      \;+\;
       \bigl(u_{i-2}a+v_{i-2}b\bigr)\\[4pt]
   &= \underbrace{\bigl(-q_i u_{i-1}+u_{i-2}\bigr)}_{=:u_i}\;a
      \;+\;
      \underbrace{\bigl(-q_i v_{i-1}+v_{i-2}\bigr)}_{=:v_i}\;b .
\end{aligned}
\]

Both \(u_i,v_i\) lie in \(R\), so \(r_i= u_i a+ v_i b\).  
Thus the property holds for index \(i\).

\paragraph{Conclusion.}
By mathematical induction the representation
\(r_i=u_i a+v_i b\) is valid for \emph{every} remainder \(r_i\).
In particular, for the last non–zero remainder \(r_n\) we have
\[
      r_n = u_n a + v_n b .
\]
Because \(r_n\) divides both \(a\) and \(b\) (shown earlier in the proof),
and every other common divisor divides \(r_n\),
we conclude
\[
      \operatorname{HCF}(a,b)=r_n = u_n a + v_n b .
\]
This completes the expanded justification of the inductive step.
  % ------------------------------------------------------------

% ------------------------------------------------------------
% ---------------------------------------------------------------------------
%  Corollary:  In a Euclidean domain, every pair has an LCM
% ---------------------------------------------------------------------------
\begin{corollary}
  Let \(R\) be a Euclidean domain and let \(a,b\in R\setminus\{0\}\).
  Then \(a\) and \(b\) possess a \emph{least common multiple}
  \[
        \operatorname{LCM}(a,b)\in R,
  \]
  and it can be chosen in the explicit form
  \[
        m \;=\; \frac{ab}{\operatorname{HCF}(a,b)}.
  \]
  \end{corollary}
  
  \begin{proof}
  Denote \(d := \operatorname{HCF}(a,b)\).
  From the Euclidean algorithm (previous theorem) we already know that
  \(d = au + bv\) for some \(u,v\in R\) and that \(d\mid a,\;d\mid b\).
  
  \smallskip
  \textbf{1.\;Definition of the candidate.}
  Because \(d\mid a\) (and \(R\) is an integral domain), the fraction
  \(
     m := \dfrac{ab}{d}
  \)
  lies in \(R\).
  
  \smallskip
  \textbf{2.\;\(a\mid m\) and \(b\mid m\).}
  Write \(a = da'\) and \(b = db'\) with \(a',b'\in R\).  Then
  \[
      m = \tfrac{ab}{d} = \tfrac{da' \, db'}{d} = a \, b'
      \quad\Longrightarrow\quad
      a\mid m ,
  \]
  and symmetrically \(m = b\,a'\) gives \(b\mid m\).
  
  \smallskip
  \textbf{3.\;Minimality (universal property).}
  Let \(m'\) be \emph{any} common multiple: \(a\mid m'\) and \(b\mid m'\).
  Hence \(ab\mid ab\,m'\).
  
  Because \(d=\operatorname{HCF}(a,b)\) divides both \(a\) and \(b\),
  it divides every common multiple, so \(d\mid m'\).
  Consequently,
  \[
     m = \frac{ab}{d} \;\mid\; \frac{ab\,m'}{d} = m'
  \]
  in the integral domain \(R\).
  Thus \(m\) divides every common multiple and is itself a common multiple,
  so \(m\) satisfies the defining property of \(\operatorname{LCM}(a,b)\).
  
  \smallskip
  All conditions are met; therefore \(m\) is an LCM of \(a\) and \(b\).
  \end{proof}
  
  \begin{remark}
  The construction shows:
  \[
     \operatorname{HCF}(a,b)\cdot \operatorname{LCM}(a,b)
     = a\,b
     \qquad(\text{up to multiplication by a unit}).
  \]
  This familiar identity from elementary number theory extends to every
  Euclidean domain.
  \end{remark}
% ---------------------------------------------------------------------------
%  Theorem: Every Euclidean Domain is a Principal Ideal Domain (PID)
% ---------------------------------------------------------------------------
\subsection*{Statement}

\begin{theorem}
  Let \(R\) be a Euclidean domain with Euclidean valuation
  \(\varphi:R\setminus\{0\}\to\mathbb N\cup\{0\}\).
  Then \(R\) is a \emph{principal ideal domain} (PID);
  i.e.\ every ideal \(I\subseteq R\) is generated by a single element.
\end{theorem}

\subsection*{Expanded proof}

\paragraph{Step 0.  Handle the zero ideal.}
If \(I=\{0\}\) we are done, since \(I=(0)\).  
Henceforth assume \(I\neq(0)\).

\medskip
\paragraph{Step 1.  Pick an element of \emph{minimal valuation}.}
Because \(I\setminus\{0\}\) is non–empty and
\(\varphi\bigl(I\setminus\{0\}\bigr)\subset\mathbb N\cup\{0\}\)
is a finite, non–empty subset, it has a least element.
Choose \(a\in I\setminus\{0\}\) with
\[
      \boxed{\;
        \varphi(a)=\min\!\bigl\{\varphi(b)\mid b\in I\setminus\{0\}\bigr\}.
      \;}
\]
Our goal is to show \(I=(a)\).

\medskip
\paragraph{Step 2.  Show \((a)\subseteq I\).}
This is immediate because \(a\in I\) and \(I\) is an \emph{ideal},
so every multiple \(ra\;(r\in R)\) lies in \(I\).

\medskip
\paragraph{Step 3.  Prove the reverse inclusion \(I\subseteq(a)\).}
Suppose, for contradiction, that there exists an element
\[
      r\;\in\; I\setminus(a).
\]
Then \(a\nmid r\).

\smallskip
\textbf{Apply the Euclidean division:}  
Because \(a\neq0\), there exist \(q,s\in R\) with
\[
   r = qa + s,
   \quad
   \text{where either } s=0 \text{ or } \varphi(s)<\varphi(a).
\]

\smallskip
\textbf{Case \(s=0\).}\;
Then \(r=qa\in(a)\), contradicting the assumption \(r\notin(a)\).

\smallskip
\textbf{Case \(s\neq0\).}\;
Observe:
\[
  s = r - qa\;\in\;I
  \quad (\text{since }I\text{ is an ideal and }r,\,qa\in I).
\]
But then \(s\in I\setminus\{0\}\) and
\(\varphi(s)<\varphi(a)\), contradicting the \emph{minimality} of
\(\varphi(a)\).

\smallskip
Hence no such \(r\) exists and \(I\subseteq(a)\).

\medskip
\paragraph{Step 4.  Conclude principalness.}
Combining Step 2 and Step 3 gives \(I=(a)\).  
Since \(I\) was arbitrary, every ideal of \(R\) is principal;  
thus \(R\) is a PID. \(\square\)

\bigskip
\subsection*{Key take-aways}

\begin{itemize}
  \item The proof uses \emph{well-ordering} of the image of \(\varphi\)
        to pick an element of minimal “size’’ inside an ideal.
  \item The Euclidean division inequality
        \(\varphi(s)<\varphi(a)\) enables a contradiction unless
        the remainder \(s\) vanishes, forcing divisibility.
  \item Minimal valuation \(a\) \(\Longrightarrow\) \(I=(a)\) —
        this is the essence of why Euclidean domains are principal.
\end{itemize}
\end{document}
