\documentclass[12pt]{article}

% Packages
\usepackage[margin=1in]{geometry}
\usepackage{amsmath,amssymb,amsthm}
\usepackage{enumitem}
\usepackage{hyperref}
\usepackage{xcolor}
\usepackage{import}
\usepackage{xifthen}
\usepackage{pdfpages}
\usepackage{transparent}
\usepackage{listings}
\DeclareMathOperator{\Log}{Log}
\DeclareMathOperator{\Arg}{Arg}

\lstset{
    breaklines=true,         % Enable line wrapping
    breakatwhitespace=false, % Wrap lines even if there's no whitespace
    basicstyle=\ttfamily,    % Use monospaced font
    frame=single,            % Add a frame around the code
    columns=fullflexible,    % Better handling of variable-width fonts
}

\newcommand{\incfig}[1]{%
    \def\svgwidth{\columnwidth}
    \import{./Figures/}{#1.pdf_tex}
}
\theoremstyle{definition} % This style uses normal (non-italicized) text
\newtheorem{solution}{Solution}
\newtheorem{proposition}{Proposition}
\newtheorem{problem}{Problem}
\newtheorem{lemma}{Lemma}
\newtheorem{theorem}{Theorem}
\newtheorem{remark}{Remark}
\newtheorem{note}{Note}
\newtheorem{definition}{Definition}
\newtheorem{example}{Example}
\newtheorem{corollary}{Corollary}
\theoremstyle{plain} % Restore the default style for other theorem environments
%

% Theorem-like environments
% Title information
\title{MATH 417: HW 7}
\author{Jerich Lee}
\date{\today}

\begin{document}

\maketitle
\begin{problem}[1]
    \textbf{Claim.} If $N$ is a normal subgroup of $G$ and $H$ is any subgroup of $G$, then
\[
  NH \;=\; \{\,nh : n\in N,\; h\in H\}\;=\;\{\,hn : h\in H,\; n\in N\}\;=\;HN.
\]
\end{problem}
\begin{solution}
    \noindent


\begin{proof}[Step-by-Step Proof]
\textbf{Step 1. Show $NH \subseteq HN$.}

\noindent
Take any element $x \in NH$. Then by definition
\[
  x \;=\; n h
  \quad\text{for some }n \in N,\; h \in H.
\]
Because $N$ is normal, for every $g \in G$ we have $gNg^{-1} = N$.  
In particular, for our $h \in H \subseteq G$, we get $h^{-1} n\,h \in N$.  
Set $n' = h^{-1} n\,h$.  Then $n' \in N$ and
\[
  x \;=\; n\,h 
        \;=\; h\bigl(h^{-1}n\,h\bigr)
        \;=\; h\,n'.
\]
So $x$ can be written as $h n'$ with $h \in H$ and $n' \in N$, which is exactly an element of $HN$.  
Hence $x \in HN$, and thus $NH \subseteq HN$.

\medskip

\textbf{Step 2. Show $HN \subseteq NH$.}

\noindent
Take any element $y \in HN$. Then by definition
\[
  y \;=\; h\,n
  \quad\text{for some }h \in H,\; n \in N.
\]
Again, using normality of $N$, set $n' = h\,n\,h^{-1} \in N$.  
Then
\[
  y \;=\; h\,n
        \;=\;\bigl(h\,n\,h^{-1}\bigr)\,h
        \;=\; n'\,h.
\]
So $y$ can be written as $n' h$ with $n' \in N$ and $h \in H$, i.e.\ $y \in NH$.  
Thus $HN \subseteq NH$.

\medskip

\textbf{Conclusion:} Combining the two inclusions $NH \subseteq HN$ and $HN \subseteq NH$, we conclude
\[
NH \;=\; HN,
\]
as desired.
\end{proof}
\end{solution}
\begin{problem}[2]

\end{problem}
\begin{solution}

% \subsection*{(a) Order $45 = 3^2 \cdot 5$}

By Sylow's theorems, let $n_3$ be the number of Sylow $3$-subgroups, and $n_5$ be the number of Sylow $5$-subgroups. Then
\[
n_3 \,\mid\, 5, \quad n_3 \equiv 1 \pmod 3, 
\quad
n_5 \,\mid\, 9, \quad n_5 \equiv 1 \pmod 5.
\]
These conditions force $n_3 = 1$ and $n_5 = 1$. Hence the Sylow $3$-subgroup (of order $9$) and the Sylow $5$-subgroup (of order $5$) are both normal in $G$. Therefore
\[
G \,\cong\, \mathrm{Sylow}_3 \,\times\, \mathrm{Sylow}_5.
\]
A group of order $9$ must be abelian (either $\mathbb{Z}_9$ or $\mathbb{Z}_3\times\mathbb{Z}_3$), and $\mathrm{Sylow}_5 \cong \mathbb{Z}_5$. Moreover, no nontrivial semidirect product arises, so
\[
G \,\cong\, 
\begin{cases}
\mathbb{Z}_9 \,\times\, \mathbb{Z}_5 \;\;(\cong \mathbb{Z}_{45}),\\
(\mathbb{Z}_3\times \mathbb{Z}_3)\,\times\, \mathbb{Z}_5.
\end{cases}
\]
Either way, $G$ is abelian. Hence there are exactly two abelian groups of order $45$.

% \subsection*{Order $51 = 3 \cdot 17$}

By Sylow's theorems, let $n_3$ be the number of Sylow $3$-subgroups, and $n_{17}$ be the number of Sylow $17$-subgroups. Then
\[
n_3 \,\mid\, 17,\quad n_3 \equiv 1 \pmod 3,
\quad
n_{17} \,\mid\, 3,\quad n_{17} \equiv 1 \pmod{17}.
\]
Thus $n_3 = 1$ and $n_{17} = 1$. This forces both Sylow subgroups to be normal, giving
\[
G \,\cong\, \mathrm{Sylow}_3 \,\times\, \mathrm{Sylow}_{17}
\;\cong\;
\mathbb{Z}_3 \,\times\, \mathbb{Z}_{17}
\;\cong\;
\mathbb{Z}_{51}.
\]
No nontrivial semidirect product can occur, so there is exactly one group of order $51$, namely the cyclic group.

% \subsection*{(c) Order $85 = 5 \cdot 17$}

By Sylow's theorems, let $n_5$ be the number of Sylow $5$-subgroups, and $n_{17}$ the number of Sylow $17$-subgroups. Then
\[
n_5 \,\mid\, 17,\quad n_5 \equiv 1 \pmod 5,
\quad
n_{17} \,\mid\, 5,\quad n_{17} \equiv 1 \pmod{17}.
\]
Hence $n_5=1$ and $n_{17}=1$, so both Sylow subgroups are normal. Therefore
\[
G \,\cong\, \mathbb{Z}_5 \,\times\, \mathbb{Z}_{17}
\;\cong\;
\mathbb{Z}_{85}.
\]
Again, no nontrivial semidirect product arises. So there is exactly one group of order $85$, namely $\mathbb{Z}_{85}$.
\end{solution}
\begin{problem}[3]
    \textbf{Claim.} If a finite group $G$ has exactly two distinct conjugacy classes, then $\lvert G\rvert = 2$, and hence $G \cong \mathbb{Z}_2$.

\end{problem}
\begin{solution}

\medskip

\noindent
\textbf{Proof.}
One of the two classes must be $\{\,e\}$, where $e$ is the identity element. Hence the other conjugacy class is $G \setminus \{e\}$, consisting of all non‐identity elements of $G$.

\begin{enumerate}
\item[(1)] If $G$ has more than two elements, let $x$ be any non‐identity element. Then its conjugacy class is $\mathrm{cl}(x) = G \setminus \{e\}$. Thus
\[
\lvert \mathrm{cl}(x)\rvert \;=\; \lvert G \rvert \;-\; 1.
\]

\item[(2)] On the other hand, by the orbit‐stabilizer theorem (or standard centralizer formula),
\[
\lvert \mathrm{cl}(x)\rvert \;=\; \bigl[G : C_G(x)\bigr]
\;=\;
\frac{\lvert G\rvert}{\lvert C_G(x)\rvert},
\]
where $C_G(x)$ is the centralizer of $x$ in $G$.

\item[(3)] Equating these two expressions for $\lvert \mathrm{cl}(x)\rvert$ gives
\[
\lvert G\rvert - 1
\;=\;
\frac{\lvert G\rvert}{\lvert C_G(x)\rvert}.
\]
Rearranging,
\[
\lvert G\rvert
\;=\;
(\lvert G\rvert - 1)\,\lvert C_G(x)\rvert.
\]
Since $x$ lies in its own centralizer, we have $\lvert C_G(x)\rvert \ge 2$. If $\lvert G\rvert > 2$, then
\[
(\lvert G\rvert -1)\,\lvert C_G(x)\rvert
\;>\;
\lvert G\rvert,
\]
a contradiction.

\item[(4)] Hence it must be that $\lvert G\rvert = 2$. The only group of order 2 is $\mathbb{Z}_2$.

\end{enumerate}

\noindent
\textbf{Conclusion.} Thus if a finite group $G$ has exactly two conjugacy classes, $G$ is isomorphic to the cyclic group of order $2$, i.e.\ $\mathbb{Z}_2$.

\end{solution}
\begin{problem}[4]
    
\end{problem}
\begin{solution}
    \subsection*{(a) The group $S_3$}

Recall that $S_3$ (the symmetric group on 3 letters) has 6 elements, which naturally partition into cycle types:
\[
\begin{aligned}
&\text{Identity: } e,\\
&\text{Transpositions: } (12), (13), (23),\\
&\text{3-cycles: } (123), (132).
\end{aligned}
\]
In $S_3,$ elements are conjugate precisely when they have the same cycle structure.  Hence the conjugacy classes are:
\[
\{\,e\}, 
\quad
\{\,(12),(13),(23)\},
\quad
\{\,(123),(132)\}.
\]

\subsection*{(b) The quaternion group $Q_8$}

Recall $Q_8 = \{\pm 1,\pm i,\pm j,\pm k\}$ with the usual quaternion relations $i^2=j^2=k^2=ijk=-1$. The center of $Q_8$ is $\{\,\pm 1\}$, so both $1$ and $-1$ are central, each forming its own (singleton) conjugacy class:
\[
\{\,1\}, 
\quad
\{\,-1\}.
\]
The remaining 6 elements $\{\,\pm i,\pm j,\pm k\}$ all lie in a single conjugacy class.  Concretely, one checks (for example) that 
\[
k\,i\,k^{-1} = j,\quad k\,j\,k^{-1} = -i,\quad
\text{etc.}
\]
so they are all conjugate under the group.  Hence the full set of classes is:
\[
\{\,1\},\quad \{\,-1\}, \quad \{\pm i,\pm j,\pm k\}.
\]

\subsection*{(c) The group $\mathbb{Z}_{12}^*$}

This is the group of units modulo $12$, i.e.\ the set of integers relatively prime to $12$, under multiplication mod $12$.  
\[
\mathbb{Z}_{12}^* \;=\;\{\,1,5,7,11\}.
\]
A quick check shows $5^2 \equiv 7^2 \equiv 11^2 \equiv 1\pmod{12}$, so every non‐identity element has order $2.$ 
Hence $\mathbb{Z}_{12}^*$ is an abelian group of order $4$ (in fact isomorphic to the Klein $4$‐group).  
\emph{In any abelian group, each element forms its own conjugacy class}, since $g x g^{-1} = x$.  
Therefore the conjugacy classes in $\mathbb{Z}_{12}^*$ are simply
\[
\{\,1\},\quad \{\,5\},\quad \{\,7\},\quad \{\,11\}.
\]
\end{solution}
\begin{problem}[5]
    
\end{problem}
\begin{solution}
    Recall that 
    \[
    G \;=\;\mathbb{Z}_2 \oplus \mathbb{Z}_2
    \]
    is the Klein four‐group $V_4$, with elements
    \[
    \{\, (0,0), (1,0), (0,1), (1,1)\,\}.
    \]
    An automorphism of $G$ must permute the three non‐identity elements, since
    \begin{itemize}
    \item they all have the same order (namely $2$),
    \item no nontrivial combination of them equals $(0,0)$, and
    \item $(0,0)$ must map to itself.
    \end{itemize}
    Hence \emph{every} automorphism is uniquely determined by its action on these three elements (which form the set $G\setminus\{\,(0,0)\}$).
    
    \medskip
    
    \noindent
    \textbf{Key observation:}  In fact, \emph{any} permutation of the three non‐identity elements extends to an automorphism, because once you prescribe images of $(1,0)$ and $(0,1)$ (which generate $G$), the image of $(1,1)$ is forced, but in a consistent way.  
    
    Thus the automorphism group $\mathrm{Aut}(G)$ is isomorphic to the full symmetric group on three letters.  Concretely,
    \[
    \mathrm{Aut}(\mathbb{Z}_2 \oplus \mathbb{Z}_2)
    \;\cong\; S_3.
    \] 
\end{solution}
\begin{problem}[6]
    
\end{problem}
\begin{solution}
    Recall that $\lvert S_6\rvert = 6! = 720 = 2^4 \cdot 3^2 \cdot 5.$ Hence a Sylow $3$-subgroup of $S_6$ has order $3^2 = 9.$

\medskip

\noindent
\textbf{Example.} Consider the two disjoint $3$-cycles
\[
(1\,2\,3)
\quad\text{and}\quad
(4\,5\,6)
\]
in $S_6.$ Each has order $3,$ and they commute (since they act on disjoint sets of points). The subgroup
\[
\langle\,(1\,2\,3),\,(4\,5\,6)\rangle
\]
then has order $9,$ because it is isomorphic to the direct product $\mathbb{Z}_3 \times \mathbb{Z}_3.$ This subgroup is therefore a \emph{Sylow $3$-subgroup} of $S_6$.

\end{solution}
\end{document}
