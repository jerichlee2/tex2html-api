\documentclass[12pt]{article}

% Packages
\usepackage[margin=1in]{geometry}
\usepackage{amsmath,amssymb,amsthm}
\usepackage{enumitem}
\usepackage{hyperref}
\usepackage{xcolor}
\usepackage{import}
\usepackage{xifthen}
\usepackage{pdfpages}
\usepackage{transparent}
\usepackage{listings}
\DeclareMathOperator{\Log}{Log}
\DeclareMathOperator{\Arg}{Arg}

\lstset{
    breaklines=true,         % Enable line wrapping
    breakatwhitespace=false, % Wrap lines even if there's no whitespace
    basicstyle=\ttfamily,    % Use monospaced font
    frame=single,            % Add a frame around the code
    columns=fullflexible,    % Better handling of variable-width fonts
}

\newcommand{\incfig}[1]{%
    \def\svgwidth{\columnwidth}
    \import{./Figures/}{#1.pdf_tex}
}
\theoremstyle{definition} % This style uses normal (non-italicized) text
\newtheorem{solution}{Solution}
\newtheorem{proposition}{Proposition}
\newtheorem{problem}{Problem}
\newtheorem{lemma}{Lemma}
\newtheorem{theorem}{Theorem}
\newtheorem{remark}{Remark}
\newtheorem{note}{Note}
\newtheorem{definition}{Definition}
\newtheorem{example}{Example}
\newtheorem{corollary}{Corollary}
\theoremstyle{plain} % Restore the default style for other theorem environments
%

% Theorem-like environments
% Title information
\title{}
\author{Jerich Lee}
\date{\today}

\begin{document}

\maketitle

\section*{Difference Between a Stabilizer and a Kernel: Step-by-Step}

In a group action, we have two important concepts:

\subsection*{1. Definitions}
\begin{itemize}
    \item \textbf{Stabilizer:} For a group \(G\) acting on a set \(X\) and for a specific element \(x\in X\), the stabilizer is defined by
    \[
    \operatorname{Stab}_G(x) = \{\,g\in G \mid g\cdot x = x\,\}.
    \]
    It consists of all elements that fix the element \(x\).

    \item \textbf{Kernel:} The kernel of the action is the set of all elements in \(G\) that fix every element of \(X\):
    \[
    \operatorname{Ker}(G\curvearrowright X) = \{\,g\in G \mid g\cdot x = x \text{ for all } x\in X\,\}.
    \]
    Equivalently, it is the intersection of all stabilizers:
    \[
    \operatorname{Ker}(G\curvearrowright X) = \bigcap_{x\in X} \operatorname{Stab}_G(x).
    \]
\end{itemize}

\subsection*{2. Example 1: Permutation Action on a Set}
Let \(G = S_3\) (the symmetric group on 3 elements) and \(X = \{1,2,3\}\), with \(S_3\) acting on \(X\) by permutation.

\begin{enumerate}[label=\textbf{Step \arabic*:}, leftmargin=*]
    \item \textbf{Determine the Stabilizer:}
    \begin{itemize}
        \item Choose an element, say \(x=1\).
        \item The stabilizer of \(1\) is
        \[
        \operatorname{Stab}_{S_3}(1) = \{g \in S_3 \mid g(1)=1\}.
        \]
        \item In \(S_3\), the permutations that fix \(1\) form a subgroup isomorphic to \(S_2\), which has 2 elements.
    \end{itemize}
    
    \item \textbf{Determine the Kernel:}
    \begin{itemize}
        \item The kernel of the action is
        \[
        \operatorname{Ker}(S_3\curvearrowright \{1,2,3\}) = \{g\in S_3 \mid g(i)=i \text{ for all } i\in \{1,2,3\}\}.
        \]
        \item Since the permutation action is faithful (only the identity fixes every element), we have
        \[
        \operatorname{Ker}(S_3\curvearrowright \{1,2,3\}) = \{e\}.
        \]
    \end{itemize}
\end{enumerate}

\subsection*{3. Example 2: Conjugation Action}
Let \(G = S_3\) and consider \(G\) acting on itself by conjugation:
\[
g \cdot x = gxg^{-1}.
\]

\begin{enumerate}[label=\textbf{Step \arabic*:}, leftmargin=*]
    \item \textbf{Stabilizer under Conjugation (Centralizer):}
    \begin{itemize}
        \item Pick an element, say \(x = (12)\).
        \item The stabilizer (or centralizer) of \((12)\) is defined as
        \[
        \operatorname{Stab}_{S_3}((12)) = \{g\in S_3 \mid g(12)g^{-1} = (12)\} = C_{S_3}((12)).
        \]
        \item Using the orbit-stabilizer theorem and knowing that the conjugacy class of \((12)\) in \(S_3\) has 3 elements, we have
        \[
        |C_{S_3}((12))| = \frac{|S_3|}{|\text{conjugacy class of } (12)|} = \frac{6}{3} = 2.
        \]
    \end{itemize}
    
    \item \textbf{Kernel of the Conjugation Action:}
    \begin{itemize}
        \item The kernel of the conjugation action is given by
        \[
        \operatorname{Ker}(S_3\curvearrowright S_3) = \{g\in S_3 \mid gxg^{-1} = x \text{ for all } x\in S_3\}.
        \]
        \item This kernel is exactly the center of \(S_3\):
        \[
        \operatorname{Ker}(S_3\curvearrowright S_3) = Z(S_3).
        \]
        \item It is known that for \(S_3\), the center is trivial:
        \[
        Z(S_3) = \{e\}.
        \]
    \end{itemize}
\end{enumerate}

\subsection*{4. Summary}
\begin{itemize}
    \item \textbf{Stabilizer:}
    \begin{itemize}
        \item In Example 1, \(\operatorname{Stab}_{S_3}(1) \cong S_2\) (order 2).
        \item In Example 2, \(\operatorname{Stab}_{S_3}((12)) = C_{S_3}((12))\) (order 2).
    \end{itemize}
    \item \textbf{Kernel:}
    \begin{itemize}
        \item In Example 1, \(\operatorname{Ker}(S_3\curvearrowright \{1,2,3\}) = \{e\}\).
        \item In Example 2, \(\operatorname{Ker}(S_3\curvearrowright S_3) = Z(S_3) = \{e\}\).
    \end{itemize}
\end{itemize}

This step-by-step example demonstrates that while the stabilizer is specific to a single element (or its conjugacy, in the case of the conjugation action), the kernel reflects the overall faithfulness of the action by being the intersection of all stabilizers.

\section*{Faithful Group Actions}

A group action \( G \curvearrowright X \) is said to be \textbf{faithful} if distinct elements of \( G \) induce distinct permutations of \( X \). In other words, the only group element that acts as the identity on every element of \( X \) is the identity element of \( G \).

\subsection*{Formal Definition}
Consider the homomorphism associated with the action:
\[
\phi : G \to \operatorname{Sym}(X), \quad \phi(g)(x) = g \cdot x.
\]
The action is \textbf{faithful} if and only if \(\phi\) is injective. Equivalently, the kernel of the action is trivial:
\[
\operatorname{Ker}(G \curvearrowright X) = \{ g \in G \mid g\cdot x = x \quad \forall\, x \in X \} = \{e\}.
\]

\subsection*{Step-by-Step Explanation}
\begin{enumerate}[label=\textbf{Step \arabic*:}]
    \item \textbf{Identify the Kernel:} \\
    Determine the set of elements in \(G\) that fix every element in \(X\):
    \[
    \operatorname{Ker}(G \curvearrowright X) = \{ g \in G \mid g \cdot x = x \quad \forall\, x \in X \}.
    \]
    
    \item \textbf{Check for Triviality:} \\
    The action is faithful if the kernel is trivial, meaning:
    \[
    \operatorname{Ker}(G \curvearrowright X) = \{e\}.
    \]
    
    \item \textbf{Interpretation:} \\
    This condition implies that no non-identity element of \(G\) acts as the identity on \(X\). Hence, different elements of \(G\) produce different permutations on \(X\), which means the action reflects the structure of \(G\) accurately.
    
    \item \textbf{Concrete Example:} \\
    Consider the symmetric group \( S_n \) acting on the set \( X = \{1,2,\dots,n\} \) by permutation. If a permutation \( g \in S_n \) fixes every element in \( X \) (i.e., \( g(i)=i \) for all \( i \)), then \( g \) must be the identity permutation. Thus,
    \[
    \operatorname{Ker}(S_n \curvearrowright \{1,2,\dots,n\}) = \{e\},
    \]
    and the action is faithful.
\end{enumerate}

\section*{Relationship Between the Kernel of a Group Action and the Kernel of a Group Homomorphism}

Let a group \( G \) act on a set \( X \). This action induces a homomorphism
\[
\phi: G \to \operatorname{Sym}(X)
\]
defined by
\[
\phi(g)(x) = g \cdot x \quad \text{for all } x \in X.
\]

\subsection*{Kernel of the Homomorphism}
The kernel of the homomorphism \(\phi\) is
\[
\operatorname{Ker}(\phi) = \{ g \in G \mid \phi(g) = \operatorname{id}_X \},
\]
where \(\operatorname{id}_X\) is the identity permutation on \(X\).

\subsection*{Kernel of the Group Action}
By definition, the kernel of the group action is
\[
\operatorname{Ker}(G \curvearrowright X) = \{ g \in G \mid g\cdot x = x \text{ for all } x \in X \}.
\]
This means that every \(g\) in the kernel of the action leaves every element of \(X\) unchanged.

\subsection*{Conclusion}
Since \(\phi(g) = \operatorname{id}_X\) if and only if \(g \cdot x = x\) for all \(x \in X\), we have
\[
\operatorname{Ker}(\phi) = \operatorname{Ker}(G \curvearrowright X).
\]
Thus, the kernel of the group action is exactly the kernel of the induced group homomorphism.

\section*{Examples of Kernels of Group Homomorphisms}

Recall that if \(\varphi: G \to H\) is a group homomorphism, then its kernel is defined by
\[
\ker(\varphi) = \{ g \in G \mid \varphi(g) = e_H \},
\]
where \(e_H\) is the identity element of \(H\).

\subsection*{Example 1: The Trivial Homomorphism}
Let \( \varphi: G \to H \) be defined by
\[
\varphi(g) = e_H \quad \text{for all } g \in G.
\]
Since every element of \(G\) is mapped to \(e_H\), we have:
\[
\ker(\varphi) = G.
\]

\subsection*{Example 2: Projection from \(\mathbb{Z}\) to \(\mathbb{Z}_n\)}
Define the homomorphism
\[
\pi: \mathbb{Z} \to \mathbb{Z}_n, \quad \pi(k) = k \mod n.
\]
The identity in \(\mathbb{Z}_n\) is \([0]\). The kernel consists of all integers \(k\) such that
\[
k \equiv 0 \pmod{n},
\]
so
\[
\ker(\pi) = n\mathbb{Z} = \{ nk \mid k \in \mathbb{Z} \}.
\]

\subsection*{Example 3: The Sign Homomorphism on \(S_n\)}
Consider the sign homomorphism
\[
\operatorname{sgn}: S_n \to \{1, -1\},
\]
which assigns to each permutation its sign (even or odd). The kernel is the set of all even permutations, which forms the alternating group:
\[
\ker(\operatorname{sgn}) = A_n.
\]

\subsection*{Example 4: The Determinant Homomorphism}
The determinant map
\[
\det: GL(n, \mathbb{R}) \to \mathbb{R}^*
\]
sends an invertible \(n \times n\) matrix to its determinant (a nonzero real number). The kernel is the set of matrices with determinant 1:
\[
\ker(\det) = SL(n, \mathbb{R}).
\]

\subsection*{Example 5: The Exponential Map}
Consider the exponential map
\[
\exp: (\mathbb{R}, +) \to S^1, \quad \exp(x) = e^{2\pi i x}.
\]
Here, the identity in \(S^1\) is \(1\) (or \(e^{0}\)). The kernel is the set of all \(x \in \mathbb{R}\) such that
\[
e^{2\pi i x} = 1.
\]
Since \(e^{2\pi i x} = 1\) if and only if \(x\) is an integer, we have:
\[
\ker(\exp) = \mathbb{Z}.
\]

\section*{Kernel of the Differentiation Operator}

Consider the differentiation operator
\[
D: C^1(I) \to C(I),
\]
where \(I \subseteq \mathbb{R}\) is an interval, \(C^1(I)\) is the space of continuously differentiable functions on \(I\), and \(C(I)\) is the space of continuous functions on \(I\).

\subsection*{Definition of the Kernel}
The kernel of \(D\) is given by
\[
\ker(D) = \{ f \in C^1(I) \mid D(f) = f' = 0 \text{ for all } x \in I \}.
\]

\subsection*{Why the Kernel Consists of Constant Functions}
By the Mean Value Theorem, if a function \(f\) satisfies \(f'(x) = 0\) for all \(x \in I\) (and \(I\) is connected), then \(f\) must be constant on \(I\). Therefore,
\[
\ker(D) = \{ f \in C^1(I) \mid f(x) = c \text{ for some constant } c \in \mathbb{R} \}.
\]

Thus, in this setting, the constants are precisely the kernel of the differentiation operator.

\section*{The Chinese Remainder Theorem}

The \textbf{Chinese Remainder Theorem} (CRT) is a result about solving systems of simultaneous congruences with pairwise coprime moduli. It states that if
\[
\begin{aligned}
x &\equiv a_1 \pmod{n_1}, \\
x &\equiv a_2 \pmod{n_2}, \\
&\;\;\vdots \\
x &\equiv a_k \pmod{n_k},
\end{aligned}
\]
where the moduli \(n_1, n_2, \dots, n_k\) are pairwise relatively prime (i.e., \(\gcd(n_i, n_j)=1\) for \(i\neq j\)), then there exists a unique solution modulo 
\[
N = n_1 n_2 \cdots n_k.
\]

\subsection*{Constructive Proof Idea}

Define
\[
N_i = \frac{N}{n_i} \quad \text{for } i=1,2,\dots,k.
\]
Since \(\gcd(N_i, n_i)=1\), there exists an integer \(M_i\) such that
\[
N_i M_i \equiv 1 \pmod{n_i}.
\]
Then the solution can be constructed as
\[
x \equiv \sum_{i=1}^{k} a_i N_i M_i \pmod{N}.
\]

\subsection*{Example 1}

Solve the system
\[
\begin{aligned}
x &\equiv 2 \pmod{3}, \\
x &\equiv 3 \pmod{5}.
\end{aligned}
\]

\noindent\textbf{Step 1:} Compute \(N\) and \(N_i\). \\
Here, \(n_1 = 3\) and \(n_2 = 5\) so
\[
N = 3 \times 5 = 15, \quad N_1 = \frac{15}{3} = 5, \quad N_2 = \frac{15}{5} = 3.
\]

\noindent\textbf{Step 2:} Find \(M_1\) and \(M_2\) such that:
\[
5M_1 \equiv 1 \pmod{3} \quad \text{and} \quad 3M_2 \equiv 1 \pmod{5}.
\]
Since \(5 \equiv 2 \pmod{3}\), we need:
\[
2M_1 \equiv 1 \pmod{3} \quad \Longrightarrow \quad M_1 = 2,
\]
because \(2 \times 2 = 4 \equiv 1 \pmod{3}\).

Similarly, \(3 \times 2 = 6 \equiv 1 \pmod{5}\), so \(M_2 = 2\).

\noindent\textbf{Step 3:} Construct the solution:
\[
x \equiv 2 \cdot 5 \cdot 2 + 3 \cdot 3 \cdot 2 = 20 + 18 = 38 \pmod{15}.
\]
Since \(38 \mod 15 = 8\), the unique solution is:
\[
x \equiv 8 \pmod{15}.
\]

\subsection*{Example 2}

Solve the system
\[
\begin{aligned}
x &\equiv 1 \pmod{4}, \\
x &\equiv 2 \pmod{5}, \\
x &\equiv 3 \pmod{7}.
\end{aligned}
\]

\noindent\textbf{Step 1:} Compute \(N\) and \(N_i\). \\
Here, \(n_1 = 4\), \(n_2 = 5\), \(n_3 = 7\) so
\[
N = 4 \times 5 \times 7 = 140,
\]
\[
N_1 = \frac{140}{4} = 35, \quad N_2 = \frac{140}{5} = 28, \quad N_3 = \frac{140}{7} = 20.
\]

\noindent\textbf{Step 2:} Find \(M_i\) such that:
\[
35M_1 \equiv 1 \pmod{4}, \quad 28M_2 \equiv 1 \pmod{5}, \quad 20M_3 \equiv 1 \pmod{7}.
\]
For \(n_1=4\): Since \(35 \equiv 3 \pmod{4}\), we need
\[
3M_1 \equiv 1 \pmod{4} \quad \Longrightarrow \quad M_1 = 3 \quad (\text{since } 3 \times 3 = 9 \equiv 1 \pmod{4}).
\]

For \(n_2=5\): Since \(28 \equiv 3 \pmod{5}\), we have:
\[
3M_2 \equiv 1 \pmod{5} \quad \Longrightarrow \quad M_2 = 2 \quad (\text{since } 3 \times 2 = 6 \equiv 1 \pmod{5}).
\]

For \(n_3=7\): Since \(20 \equiv 6 \pmod{7}\) (or equivalently \(-1 \pmod{7}\)), we need:
\[
6M_3 \equiv 1 \pmod{7} \quad \Longrightarrow \quad M_3 \equiv 6 \pmod{7} \quad (\text{since } 6 \times 6 = 36 \equiv 1 \pmod{7}).
\]

\noindent\textbf{Step 3:} Construct the solution:
\[
x \equiv 1 \cdot 35 \cdot 3 + 2 \cdot 28 \cdot 2 + 3 \cdot 20 \cdot 6 \pmod{140}.
\]
Calculating, we get:
\[
x \equiv 105 + 112 + 360 = 577 \pmod{140}.
\]
Since \(577 \mod 140 = 577 - 4 \times 140 = 577 - 560 = 17\), the unique solution is:
\[
x \equiv 17 \pmod{140}.
\]
\section*{Relationship Between the Classical CRT and Its Algebraic Versions}

The \textbf{classical Chinese Remainder Theorem (CRT)} states that if we have a system of congruences
\[
\begin{aligned}
x &\equiv a_1 \pmod{n_1}, \\
x &\equiv a_2 \pmod{n_2}, \\
&\;\;\vdots \\
x &\equiv a_r \pmod{n_r},
\end{aligned}
\]
with the moduli \(n_1, n_2, \dots, n_r\) being pairwise coprime, then there exists a unique solution modulo 
\[
n = n_1 n_2 \cdots n_r.
\]
In essence, knowing the remainders \(x \bmod n_i\) for each \(i\) completely determines the residue of \(x\) modulo \(n\).

\section*{Additive Version (Theorem 1.19)}

The additive version of the CRT states that when the moduli \(n_1, n_2, \dots, n_r\) are pairwise relatively prime, there is a ring isomorphism:
\[
\mathbb{Z}_n \;\cong\; \mathbb{Z}_{n_1} \oplus \mathbb{Z}_{n_2} \oplus \cdots \oplus \mathbb{Z}_{n_r}.
\]
This means that every element in the ring \(\mathbb{Z}_n\) corresponds uniquely to a tuple
\[
\bigl(x \bmod n_1,\, x \bmod n_2,\, \dots,\, x \bmod n_r\bigr)
\]
in the direct sum of the rings \(\mathbb{Z}_{n_i}\). This is precisely the statement of the classical CRT rephrased in the language of ring theory.

\section*{Multiplicative Version (Theorem 1.20)}

The multiplicative version focuses on the units (invertible elements) in these rings. It states that:
\[
(\mathbb{Z}_n)^\times \;\cong\; (\mathbb{Z}_{n_1})^\times \times (\mathbb{Z}_{n_2})^\times \times \cdots \times (\mathbb{Z}_{n_r})^\times.
\]
Here, \((\mathbb{Z}_n)^\times\) denotes the multiplicative group of units modulo \(n\), and each \((\mathbb{Z}_{n_i})^\times\) is the group of units modulo \(n_i\). This theorem tells us that an invertible element modulo \(n\) corresponds uniquely to a tuple of invertible elements modulo each \(n_i\). The process of "breaking apart" the multiplicative structure mirrors the classical CRT, but now for the multiplicative groups.

\section*{How They Relate to the Classical CRT}

Both the additive and multiplicative versions are algebraic restatements of the classical Chinese Remainder Theorem:

\begin{itemize}
    \item \textbf{Classical CRT:} Guarantees a unique solution to a system of congruences with pairwise coprime moduli.
    \item \textbf{Additive Version:} Reinterprets this unique correspondence as an isomorphism between the ring \(\mathbb{Z}_n\) and the direct sum \(\mathbb{Z}_{n_1} \oplus \cdots \oplus \mathbb{Z}_{n_r}\).
    \item \textbf{Multiplicative Version:} Further refines this idea by showing that the multiplicative structure (units) in \(\mathbb{Z}_n\) splits as a direct product of the groups \((\mathbb{Z}_{n_i})^\times\).
\end{itemize}

In both cases, the key is the fact that the moduli are pairwise coprime. This condition ensures that the mapping
\[
x \mapsto \bigl(x \bmod n_1,\, x \bmod n_2,\, \dots,\, x \bmod n_r\bigr)
\]
is bijective and structure-preserving (either as a ring homomorphism or as a group homomorphism). Thus, these algebraic versions are simply different perspectives on the same underlying principle articulated by the classical CRT.

\section*{Example: Solving a System of Congruences with the Chinese Remainder Theorem}

Consider the system of congruences:
\[
\begin{aligned}
x &\equiv 2 \pmod{3}, \\
x &\equiv 3 \pmod{5}.
\end{aligned}
\]
Since \(3\) and \(5\) are relatively prime, the Chinese Remainder Theorem guarantees a unique solution modulo 
\[
N = 3 \times 5 = 15.
\]

\subsection*{Step 1: Understanding the Isomorphism}

According to the additive version of the CRT, we have a ring isomorphism:
\[
\mathbb{Z}_{15} \cong \mathbb{Z}_3 \oplus \mathbb{Z}_5.
\]
This means every element \(x \in \mathbb{Z}_{15}\) corresponds uniquely to the pair 
\[
\bigl( x \bmod 3,\, x \bmod 5 \bigr).
\]

\subsection*{Step 2: Constructing the Solution}

Define:
\[
N_1 = \frac{N}{3} = \frac{15}{3} = 5, \quad N_2 = \frac{N}{5} = \frac{15}{5} = 3.
\]
We need to find integers \(M_1\) and \(M_2\) such that:
\[
N_1 M_1 \equiv 1 \pmod{3} \quad \text{and} \quad N_2 M_2 \equiv 1 \pmod{5}.
\]

\paragraph{Finding \(M_1\):}  
Since \(N_1 = 5\) and \(5 \equiv 2 \pmod{3}\), we need:
\[
2M_1 \equiv 1 \pmod{3}.
\]
Choosing \(M_1 = 2\) works because \(2 \times 2 = 4 \equiv 1 \pmod{3}\).

\paragraph{Finding \(M_2\):}  
Since \(N_2 = 3\) and \(3 \equiv 3 \pmod{5}\), we need:
\[
3M_2 \equiv 1 \pmod{5}.
\]
Choosing \(M_2 = 2\) works as well since \(3 \times 2 = 6 \equiv 1 \pmod{5}\).

\paragraph{Constructing \(x\):}  
The solution is given by:
\[
x \equiv a_1 N_1 M_1 + a_2 N_2 M_2 \pmod{15},
\]
where \(a_1 = 2\) and \(a_2 = 3\). Thus,
\[
x \equiv 2 \cdot 5 \cdot 2 + 3 \cdot 3 \cdot 2 = 20 + 18 = 38 \pmod{15}.
\]
Since \(38 \mod 15 = 8\), we have:
\[
x \equiv 8 \pmod{15}.
\]

\subsection*{Step 3: Verification}

To check the solution:
\[
8 \bmod 3 = 2 \quad \text{and} \quad 8 \bmod 5 = 3.
\]
Thus, the solution \(x = 8\) satisfies both congruences.

\end{document}
