\documentclass[12pt]{article}

% Packages
\usepackage[margin=1in]{geometry}
\usepackage{amsmath,amssymb,amsthm}
\usepackage{enumitem}
\usepackage{hyperref}
\usepackage{xcolor}
\usepackage{import}
\usepackage{xifthen}
\usepackage{pdfpages}
\usepackage{transparent}
\usepackage{listings}
\DeclareMathOperator{\Log}{Log}
\DeclareMathOperator{\Arg}{Arg}

\lstset{
    breaklines=true,         % Enable line wrapping
    breakatwhitespace=false, % Wrap lines even if there's no whitespace
    basicstyle=\ttfamily,    % Use monospaced font
    frame=single,            % Add a frame around the code
    columns=fullflexible,    % Better handling of variable-width fonts
}

\newcommand{\incfig}[1]{%
    \def\svgwidth{\columnwidth}
    \import{./Figures/}{#1.pdf_tex}
}
\theoremstyle{definition} % This style uses normal (non-italicized) text
\newtheorem{solution}{Solution}
\newtheorem{proposition}{Proposition}
\newtheorem{problem}{Problem}
\newtheorem{lemma}{Lemma}
\newtheorem{theorem}{Theorem}
\newtheorem{remark}{Remark}
\newtheorem{note}{Note}
\newtheorem{definition}{Definition}
\newtheorem{example}{Example}
\newtheorem{corollary}{Corollary}
\theoremstyle{plain} % Restore the default style for other theorem environments
%

% Theorem-like environments
% Title information
\title{}
\author{Jerich Lee}
\date{\today}

\begin{document}

\maketitle
\section*{A Non-abelian Group of Order 55}

\textbf{Claim.} There are exactly two (isomorphism classes of) groups of order $55$, one abelian and one non-abelian.

\medskip

\noindent
\textbf{Step 1: Sylow subgroups.} 
By Sylow's theorems, any group $G$ of order $55 = 5 \cdot 11$ has:
\[
n_5 \;\bigm|\; 11,\quad n_5 \equiv 1 \pmod{5},
\quad\text{and}\quad
n_{11} \;\bigm|\; 5,\quad n_{11} \equiv 1 \pmod{11}.
\]
Hence $n_5 = 1$ and $n_{11} = 1$. This means there is exactly one Sylow $5$-subgroup $G_5$ (of order $5$) and exactly one Sylow $11$-subgroup $G_{11}$ (of order $11$), both of which are normal in $G$.

\medskip

\noindent
\textbf{Step 2: Semidirect product structure.} 
Because $G_5$ and $G_{11}$ are both normal and cyclic, $G$ must be a semidirect product:
\[
G \;\cong\; G_5 \rtimes G_{11}
\quad\text{or}\quad
G \;\cong\; G_{11} \rtimes G_5.
\]
Being cyclic of prime order, $G_5 \cong C_5$ and $G_{11} \cong C_{11}$.

\medskip

\noindent
\textbf{Step 3: Automorphism groups.}
\[
\mathrm{Aut}(C_{5}) \;\cong\; C_{4}, 
\quad\quad
\mathrm{Aut}(C_{11}) \;\cong\; C_{10}.
\]
Thus:
\begin{itemize}
\item Any homomorphism $C_{11} \to \mathrm{Aut}(C_{5}) \cong C_{4}$ must be trivial, because $11$ does not divide $4$. This yields the \emph{direct product} $C_5 \times C_{11}$ (abelian).
\item A homomorphism $C_5 \to \mathrm{Aut}(C_{11}) \cong C_{10}$ can be non-trivial if it sends a generator of $C_5$ to an element of order $5$ in $C_{10}$. Such an action produces a \emph{non-abelian} semidirect product $C_{11} \rtimes C_5$.
\end{itemize}

\medskip

\noindent
\textbf{Step 4: Concretely defining the non-trivial action.} 
Let $a$ be a generator of $C_5$ and $b$ be a generator of $C_{11}$. Pick $k$ such that $k \in (\mathbb{Z}/11\mathbb{Z})^\times$ has order $5$; i.e.\ $k^5 \equiv 1 \pmod{11}$ but $k \not\equiv 1$. Then define
\[
a : b \;\mapsto\; b^k.
\]
Equivalently, the homomorphism $\alpha: C_5 \to \mathrm{Aut}(C_{11})$ sends $a$ to the automorphism $b \mapsto b^k$. This action is non-trivial and gives us a non-abelian group structure:
\[
C_{11} \rtimes_{\alpha} C_{5}.
\]

\medskip

\noindent
\textbf{Conclusion.} 
There are exactly two groups of order $55$ up to isomorphism:
\[
C_5 \times C_{11}
\quad (\text{abelian}), 
\quad\text{and}\quad
C_{11} \rtimes C_{5}
\quad (\text{non-abelian}).
\]
This completes the classification.


\section*{Why Any Homomorphism \( C_{11} \to \mathrm{Aut}(C_{5}) \cong C_{4} \) Must be Trivial}

\subsection*{1. Groups in Play}
\begin{itemize}
    \item \(C_{11}\) is a cyclic group of order \(11\), generated by an element \(x\).
    \item \(\mathrm{Aut}(C_{5})\) is isomorphic to \(C_{4}\) since 
    \[
    |\mathrm{Aut}(C_{5})| = \varphi(5) = 4.
    \]
\end{itemize}
Thus, any group homomorphism 
\[
\varphi: C_{11} \to C_{4}
\]
maps a cyclic group of order \(11\) to a cyclic group of order \(4\).

\subsection*{2. Homomorphisms and Element Orders}
A key property of group homomorphisms is that for any \(g \in C_{11}\),
\[
\mathrm{ord}(\varphi(g)) \quad \text{divides} \quad \mathrm{ord}(g).
\]
In other words, if \(x\) is a generator of \(C_{11}\) with \(\mathrm{ord}(x) = 11\), then the order of \(\varphi(x)\) in \(C_{4}\) must divide \(11\).

\subsection*{3. Applying This to Our Groups}
Let \(x\) be a generator of \(C_{11}\). Suppose \(\varphi(x) \neq e\) in \(C_{4}\), and let \(d = \mathrm{ord}(\varphi(x))\). Then:
\begin{itemize}
    \item \(d\) divides \(11\) because \(\varphi(x)\) comes from an element of order \(11\).
    \item \(d\) also divides \(4\) since \(C_{4}\) is of order \(4\).
\end{itemize}
The only positive integer that divides both \(11\) and \(4\) is \(1\). Thus,
\[
d = 1,
\]
meaning \(\varphi(x)\) is the identity element in \(C_{4}\).

\subsection*{4. Conclusion: The Homomorphism is Trivial}
Since \(x\) generates \(C_{11}\), if \(\varphi(x) = e\), then for every \(x^k \in C_{11}\) we have:
\[
\varphi(x^k) = \varphi(x)^k = e^k = e.
\]
Hence, the homomorphism \(\varphi\) is trivial:
\[
\boxed{\varphi(x) = e \text{ for all } x \in C_{11}.}
\]

\subsection*{5. Implication in the Semidirect Product Context}
When forming a semidirect product \(C_5 \rtimes C_{11}\), we require a homomorphism
\[
\alpha: C_{11} \to \mathrm{Aut}(C_5) \cong C_4
\]
to describe how \(C_{11}\) acts on \(C_5\). Since any such homomorphism is forced to be trivial, the action is trivial, and the semidirect product becomes a direct product:
\[
C_5 \times C_{11}.
\]
This is why a non-abelian semidirect product of the form \(C_{5} \rtimes C_{11}\) cannot occur.


\section*{Defining the Non-Trivial Action in \(\boldsymbol{C_{11} \rtimes C_{5}}\)}

\noindent
\textbf{Setting.} 
We have two cyclic groups:
\[
C_{5} = \langle a \mid a^5 = 1 \rangle,
\quad
C_{11} = \langle b \mid b^{11} = 1 \rangle.
\]
Their product order is \(5 \cdot 11 = 55\).

\subsection*{1. Automorphism Group of \(\boldsymbol{C_{11}}\)}
Recall that \(\mathrm{Aut}(C_{11}) \cong C_{10}\). Concretely, 
\(\mathrm{Aut}(C_{11})\) can be identified with 
\((\mathbb{Z}/11\mathbb{Z})^\times\), a cyclic group of order \(10\).

\subsection*{2. Picking an Element of Order 5}
We seek a non-trivial action of \(C_{5}\) on \(C_{11}\). 
Hence we need a group homomorphism
\[
\alpha : C_{5} \;\to\; \mathrm{Aut}(C_{11}) \cong C_{10}
\]
that is non-trivial. Since \(C_{10}\) has an element of order \(5\), we choose 
\(k \in (\mathbb{Z}/11\mathbb{Z})^\times\) such that
\[
k^5 \equiv 1 \pmod{11}, 
\quad
k \not\equiv 1 \pmod{11}.
\]
Such a \(k\) has order \(5\) in the multiplicative group modulo \(11\).

\subsection*{3. Defining the Action}
Let \(a\) be a generator of \(C_{5}\). We define \(\alpha\) by sending 
\(\displaystyle a\) to the automorphism \(\displaystyle (b \mapsto b^k)\) of \(C_{11}\). 
Equivalently,
\[
a : b \;\mapsto\; b^k,
\]
which implies the conjugation relation
\[
a b \, a^{-1} \;=\; b^k.
\]

\subsection*{4. The Semidirect Product}
Given this action, we form the semidirect product
\[
C_{11} \rtimes_{\alpha} C_{5}
\;=\;
\langle a, b \mid a^5 = 1,\; b^{11} = 1,\; a b\, a^{-1} = b^k \rangle.
\]
Since \(k \neq 1\), this group is \emph{non-abelian} (i.e.\ \(a\) and \(b\) do not commute).

\subsection*{5. Why is it Non-Abelian?}
If \(k\) were \(1\), the action would be trivial, yielding the direct product 
\(C_{11} \times C_{5}\), which is abelian. 
By choosing \(k \neq 1\), we ensure a non-trivial action, and thus the resulting 
semidirect product is non-abelian.

\bigskip
\noindent
\textbf{Conclusion.} This construction yields a non-abelian group of order 55, 
often denoted \(C_{11} \rtimes C_{5}\).


\section*{Concrete Example: The Non-Trivial Action in \(C_{11} \rtimes C_5\)}

\noindent
\textbf{Setting.} \\
We start with two cyclic groups:
\[
C_{5} = \langle a \mid a^5 = 1 \rangle, \quad
C_{11} = \langle b \mid b^{11} = 1 \rangle.
\]
Their orders are \(5\) and \(11\), respectively, so the total order of the product is \(5 \times 11 = 55\).

\medskip

\noindent
\textbf{1. Automorphism Group of \(C_{11}\)} \\
Recall that 
\[
\mathrm{Aut}(C_{11}) \cong (\mathbb{Z}/11\mathbb{Z})^\times,
\]
which is a cyclic group of order \(10\). This group is isomorphic to \(C_{10}\).

\medskip

\noindent
\textbf{2. Choosing a Concrete Element of Order 5} \\
We want a non-trivial homomorphism 
\[
\alpha : C_{5} \to \mathrm{Aut}(C_{11}) \cong C_{10}.
\]
To have a non-trivial action, we need an element of order 5 in \(C_{10}\). In \((\mathbb{Z}/11\mathbb{Z})^\times\), consider the element \(k = 4\). We verify its order:
\[
4^5 = 1024.
\]
Since 
\[
1024 \div 11 = 93 \text{ with a remainder of } 1, 
\]
we have
\[
4^5 \equiv 1 \pmod{11}.
\]
Also, \(4 \not\equiv 1 \pmod{11}\), so \(4\) indeed has order 5 in \((\mathbb{Z}/11\mathbb{Z})^\times\).

\medskip

\noindent
\textbf{3. Defining the Action} \\
Let \(a\) be the generator of \(C_{5}\) and \(b\) be the generator of \(C_{11}\). Define the homomorphism \(\alpha\) by assigning to the generator \(a\) the automorphism of \(C_{11}\) given by
\[
\alpha(a) : b \mapsto b^4.
\]
This means that when \(a\) acts on any element \(b \in C_{11}\), it sends \(b\) to \(b^4\). In terms of conjugation in the semidirect product, this is written as:
\[
a\, b\, a^{-1} = b^4.
\]

\medskip

\noindent
\textbf{4. Forming the Semidirect Product} \\
With this action, the semidirect product is defined as:
\[
C_{11} \rtimes_{\alpha} C_{5} = \langle a, b \mid a^5 = 1,\; b^{11} = 1,\; a\, b\, a^{-1} = b^4 \rangle.
\]
Since \(4 \neq 1\) modulo 11, the relation \(a\, b\, a^{-1} = b^4\) is non-trivial, ensuring that \(a\) and \(b\) do not commute. Therefore, the group is non-abelian.

\medskip

\noindent
\textbf{5. Why is the Group Non-Abelian?} \\
If the chosen \(k\) were 1, the action would be trivial, and the semidirect product would reduce to the direct product:
\[
C_{11} \times C_{5},
\]
which is abelian. By choosing \(k = 4\) (with \(4^5 \equiv 1\) but \(4 \not\equiv 1\)), we force a non-trivial action, and hence the resulting group
\[
C_{11} \rtimes_{\alpha} C_{5}
\]
is non-abelian.

\medskip

\noindent
\textbf{Conclusion.} \\
This construction gives us a non-abelian group of order 55, explicitly defined by the presentation:
\[
\langle a, b \mid a^5 = 1,\; b^{11} = 1,\; a\, b\, a^{-1} = b^4 \rangle.
\]
\section*{Why Conjugation?}

In a semidirect product $N \rtimes_\alpha H$, each element of $H$ acts on $N$
by an automorphism given by the homomorphism
\[
  \alpha : H \to \mathrm{Aut}(N).
\]
Concretely, if $(n,h)$ and $(n',h')$ are elements of $N \rtimes_\alpha H$, 
their product is defined by
\[
  (n,h)\,(n',h')
  \;=\;
  \bigl(n\,\alpha(h)(n'),\,h h'\bigr).
\]
This implies that \emph{conjugation} by $(1,h)$ applies the automorphism 
$\alpha(h)$ to any $n \in N$:
\[
  (1,h)\,(n,1)\,(1,h)^{-1} 
  \;=\;
  \bigl(\alpha(h)(n),\,1\bigr).
\]

\subsection*{Example: \(C_{11} \rtimes C_{5}\)}
Let $N = \langle b \mid b^{11} = 1\rangle \cong C_{11}$ 
and $H = \langle a \mid a^5 = 1\rangle \cong C_{5}$. 
If $\alpha(a)$ is the automorphism of $C_{11}$ given by $b \mapsto b^k$ for 
some $k \neq 1$, then in the semidirect product we get the relation
\[
  a\,b\,a^{-1} = b^k.
\]
Hence we say ``$a$ acts on $b$ by conjugation'' because in the group $C_{11} \rtimes C_{5}$, 
the effect of $a$ on $b$ is realized through the conjugation relation 
\(\,a b a^{-1} = b^k.\)

\section*{Why We Send \(b\) to \(b^k\)}

\subsection*{1. Automorphisms of \(C_{11}\)}
The group \(\mathrm{Aut}(C_{11})\) is isomorphic to \((\mathbb{Z}/11\mathbb{Z})^\times\), 
which is cyclic of order 10. Every automorphism of \(C_{11}\) is determined by its 
effect on the generator \(b\):
\[
b \mapsto b^m
\quad (\text{where } m \in \{1,2,\dots,10\}).
\]

\subsection*{2. Defining a Non-Trivial Action}
We want a homomorphism 
\[
\alpha: C_{5} \to \mathrm{Aut}(C_{11}),
\]
where \(C_5 = \langle a \rangle\). Hence \(\alpha\) is determined by \(\alpha(a)\). 
To avoid the trivial (identity) automorphism, pick \(k \neq 1\) so that
\[
\alpha(a): b \mapsto b^k.
\]

\subsection*{3. Ensuring the Action Has Order 5}
Since \(a^5 = 1\) in \(C_5\), we want \(\alpha(a)^5\) to be the identity automorphism in 
\(\mathrm{Aut}(C_{11})\). This requires 
\[
(b \mapsto b^k)^5 = (b \mapsto b),
\]
which is equivalent to \(k^5 \equiv 1 \pmod{11}\). 
By choosing such a \(k\), we get a non-trivial action of order 5.

\subsection*{4. Conjugation Relation}
In the semidirect product \(C_{11} \rtimes_\alpha C_{5}\), this action is encoded by 
\[
a\,b\,a^{-1} = b^k.
\]
Since \(k \neq 1\), \(a\) and \(b\) do not commute, so the group is non-abelian.

\begin{theorem}
    The dihedral group $D_3$ is isomorphic to the symmetric group $S_3$.
    \end{theorem}
    
    \begin{proof}
    Label the vertices of an equilateral triangle by $1, 2, 3$. The group $D_3$ consists of:
    \[
    \text{(1) the identity, (2) two nontrivial rotations, and (3) three reflections,}
    \]
    making a total of 6 symmetries.
    
    Define a map
    \[
    \phi: D_3 \longrightarrow S_3
    \]
    by letting $\phi(\sigma)$ be the permutation of $\{1,2,3\}$ given by
    \[
    \phi(\sigma)(i) = \sigma \cdot i,
    \]
    that is, each symmetry $\sigma$ in $D_3$ is sent to the permutation of vertices it induces.
    
    \textbf{Homomorphism:}
    If $\sigma, \tau \in D_3$, then by definition
    \[
    \phi(\sigma \circ \tau)(i) = (\sigma \circ \tau)(i) = \sigma\bigl(\tau(i)\bigr).
    \]
    On the other hand,
    \[
    \phi(\sigma)\bigl(\phi(\tau)(i)\bigr)
    = \phi(\sigma)\bigl(\tau(i)\bigr)
    = \sigma\bigl(\tau(i)\bigr).
    \]
    Hence $\phi(\sigma \circ \tau) = \phi(\sigma)\phi(\tau)$, so $\phi$ is a group homomorphism.
    
    \textbf{Injectivity:}
    If $\phi(\sigma) = \phi(\tau)$, then $\sigma$ and $\tau$ act identically on the vertices $1, 2, 3$. A symmetry of a triangle is completely determined by its action on these vertices, so $\sigma = \tau$. Therefore, $\phi$ is injective.
    
    \textbf{Surjectivity:}
    Since $|D_3| = 6$ and $|S_3| = 6$, and $\phi$ is injective, it follows that $\phi$ is also surjective (an injective map between two finite sets of the same cardinality is necessarily onto).
    
    Because $\phi$ is a bijective homomorphism, it is an isomorphism. Therefore,
    \[
    D_3 \cong S_3.
    \]
    \end{proof}

    \noindent
    \begin{enumerate}
        \item for all $n \in \mathbb{{N}}$ there is a non-abelian group of order $n$
        \item for all finite groups G, Aut(G) is finite  
        \item for all infinite groups G, Aut(G) is infinite  
        \item $Q_8$ is the semi-direct product of $\left\{ -1, 1, -i, i \right\} and$
        \item Is $(12)(345)$ and $(12345)$ conjugate to each other? 
        \item What are the conjugacy classes of the klein-4 group?
        \item what are the conjugacy classes of $D_4$?
        \item Prove that a group of order $35$ is abelian. classify all groups of order 35 up to isomorphism 
    \end{enumerate}
\end{document}
