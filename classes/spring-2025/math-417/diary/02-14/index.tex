\documentclass[12pt]{article}

% Packages
\usepackage[margin=1in]{geometry}
\usepackage{amsmath,amssymb,amsthm}
\usepackage{enumitem}
\usepackage{hyperref}
\usepackage{xcolor}
\usepackage{import}
\usepackage{xifthen}
\usepackage{pdfpages}
\usepackage{transparent}
\usepackage{listings}


\lstset{
    breaklines=true,         % Enable line wrapping
    breakatwhitespace=false, % Wrap lines even if there's no whitespace
    basicstyle=\ttfamily,    % Use monospaced font
    frame=single,            % Add a frame around the code
    columns=fullflexible,    % Better handling of variable-width fonts
}

\newcommand{\incfig}[1]{%
    \def\svgwidth{\columnwidth}
    \import{./Figures/}{#1.pdf_tex}
}
\theoremstyle{definition} % This style uses normal (non-italicized) text
\newtheorem{solution}{Solution}
\newtheorem{proposition}{Proposition}
\newtheorem{problem}{Problem}
\newtheorem{lemma}{Lemma}
\newtheorem{theorem}{Theorem}
\newtheorem{remark}{Remark}
\newtheorem{note}{Note}
\theoremstyle{plain} % Restore the default style for other theorem environments
%

% Theorem-like environments
% Title information
\title{hw 3}
\author{Jerich Lee}
\date{\today}

\begin{document}

\maketitle
\section*{Solution}

We have the following groups to compare:
\[
\mathbb{Z}_{12}, \quad \mathbb{Z}_{3} \oplus \mathbb{Z}_{2} \oplus \mathbb{Z}_{2}, \quad 
\mathbb{Z}_{2} \oplus \mathbb{Z}_{3} \oplus \mathbb{Z}_{4}, \quad
\mathbb{Z}_{3} \oplus \mathbb{Z}_{4}, \quad
C_{12}, \quad
C_{4} \times C_{3}.
\]

We use two basic invariants for abelian groups:
\begin{enumerate}
    \item \textbf{Order:} Two groups of different orders cannot be isomorphic.
    \item \textbf{Cyclicity:} Among groups of the same order, check if they are cyclic 
    (i.e., generated by a single element). A group is cyclic if and only if it 
    can be written as $\mathbb{Z}_n$ for some $n$, or equivalently, if it has 
    an element of the largest possible order (equal to the group’s cardinality).
\end{enumerate}

\noindent \textbf{Step by step analysis:}
\begin{itemize}
    \item \(\mathbb{Z}_{12}\) and \(C_{12}\) both have order 12 and are clearly cyclic groups of order 12.
    \item \(\mathbb{Z}_{3} \oplus \mathbb{Z}_{4}\) has order \(3 \times 4 = 12\). Since 
    \(\gcd(3,4) = 1\), this group is also cyclic and is in fact isomorphic to 
    \(\mathbb{Z}_{12}\). Concretely, \(\mathbb{Z}_{3}\oplus \mathbb{Z}_{4} \cong \mathbb{Z}_{12}\).
    \item \(C_{4} \times C_{3}\) is just another notation for \(\mathbb{Z}_{4} \times \mathbb{Z}_{3}\).
    Its order is \(4 \times 3 = 12\), and again \(\gcd(4,3) = 1\). Thus this group 
    is also cyclic of order 12 and is isomorphic to \(\mathbb{Z}_{12}\).
    \[
      C_{4} \times C_{3} \cong \mathbb{Z}_{4} \times \mathbb{Z}_{3} \cong \mathbb{Z}_{12}.
    \]

    \item \(\mathbb{Z}_{3} \oplus \mathbb{Z}_{2} \oplus \mathbb{Z}_{2}\) also has order 
    \(3 \times 2 \times 2 = 12\). However, the largest possible order of any 
    element in this group is the least common multiple of \(\{3,2,2\}\), which 
    is 6. Hence no element has order 12, so this group is \emph{not} cyclic. 
    Consequently, it is \emph{not} isomorphic to any of the groups above (which \emph{are} cyclic).

    \item \(\mathbb{Z}_{2} \oplus \mathbb{Z}_{3} \oplus \mathbb{Z}_{4}\) has order 
    \(2 \times 3 \times 4 = 24\), which differs from all the preceding groups 
    (they all have order 12). So it cannot be isomorphic to any of them.
\end{itemize}

\noindent \textbf{Conclusion:}
\begin{itemize}
    \item All of 
      \(\mathbb{Z}_{12},\, C_{12},\, \mathbb{Z}_{3}\oplus \mathbb{Z}_{4},\, C_{4}\times C_{3}\)
      are isomorphic to one another (the cyclic group of order 12).
    \item \(\mathbb{Z}_{3}\oplus \mathbb{Z}_{2}\oplus \mathbb{Z}_{2}\) is a distinct, 
      noncyclic group of order 12 and is not isomorphic to the cyclic ones.
    \item \(\mathbb{Z}_{2}\oplus \mathbb{Z}_{3}\oplus \mathbb{Z}_{4}\) has order 24, so it 
      is not isomorphic to any 12-element group.
\end{itemize}

\section*{Determining the order of a direct sum of groups}

For \emph{finite} abelian groups, the order of the direct sum (or direct product) is just the product of the orders of the component groups. Symbolically, if
\[
  G \;=\; G_1 \oplus G_2 \oplus \cdots \oplus G_k,
\]
and each \(G_i\) is finite with \(\lvert G_i \rvert = n_i\), then the order of \(G\) is
\[
  \lvert G \rvert 
  \;=\; 
  n_1 \times n_2 \times \cdots \times n_k.
\]
Below are some concrete examples:

\begin{enumerate}
  \item \(\mathbb{Z}_2 \oplus \mathbb{Z}_3\):
  \[
    \lvert \mathbb{Z}_2 \rvert = 2, 
    \quad 
    \lvert \mathbb{Z}_3 \rvert = 3, 
    \quad 
    \text{so} 
    \quad 
    \lvert \mathbb{Z}_2 \oplus \mathbb{Z}_3 \rvert = 2 \times 3 = 6.
  \]

  \item \(\mathbb{Z}_4 \oplus \mathbb{Z}_8\):
  \[
    \lvert \mathbb{Z}_4 \rvert = 4,
    \quad
    \lvert \mathbb{Z}_8 \rvert = 8,
    \quad
    \text{so}
    \quad
    \lvert \mathbb{Z}_4 \oplus \mathbb{Z}_8 \rvert = 4 \times 8 = 32.
  \]

  \item \(\mathbb{Z}_2 \oplus \mathbb{Z}_2 \oplus \mathbb{Z}_2\):
  \[
    \lvert \mathbb{Z}_2 \rvert = 2,
    \quad
    \text{so for three factors we have}
    \quad
    2 \times 2 \times 2 = 8.
  \]
  Hence \(\lvert \mathbb{Z}_2 \oplus \mathbb{Z}_2 \oplus \mathbb{Z}_2 \rvert = 8.\)

  \item \(\mathbb{Z}_3 \oplus \mathbb{Z}_4 \oplus \mathbb{Z}_5\):
  \[
    \lvert \mathbb{Z}_3 \rvert = 3, 
    \quad
    \lvert \mathbb{Z}_4 \rvert = 4,
    \quad
    \lvert \mathbb{Z}_5 \rvert = 5,
    \quad
    \text{so}
    \quad
    \lvert \mathbb{Z}_3 \oplus \mathbb{Z}_4 \oplus \mathbb{Z}_5 \rvert
    = 3 \times 4 \times 5 
    = 60.
  \]

\end{enumerate}

\section*{Examples of Direct Sums of Groups}

Below are a few examples illustrating the notion of direct sums of (abelian) groups. 
Recall that for abelian groups \(A\) and \(B\), the direct sum \(A \oplus B\) is simply 
the cartesian product \(A \times B\) with componentwise operation.

\begin{enumerate}
  \item \(\mathbb{Z}_2 \oplus \mathbb{Z}_2\).\\
    This is the \emph{Klein four-group}, often denoted \(V_4\). It consists of 
    all ordered pairs \((a,b)\) with \(a,b \in \{0,1\}\) under componentwise addition mod 2:
    \[
      (a,b) + (x,y) = (a+x \bmod 2,\; b+y \bmod 2).
    \]
    It has four elements in total and is \emph{not} cyclic.

  \item \(\mathbb{Z}_2 \oplus \mathbb{Z}_3\).\\
    This group has six elements and is isomorphic to \(\mathbb{Z}_6\) because 
    \(\gcd(2,3) = 1\). In other words, it is actually a \emph{cyclic} group of order 6.

  \item \(\mathbb{Z}_4 \oplus \mathbb{Z}_4\).\\
    Here, each factor is of order 4, so the direct sum has \(4 \times 4 = 16\) elements. 
    The group operation is componentwise addition mod 4. Unlike \(\mathbb{Z}_8 \oplus \mathbb{Z}_2\) (which also has 16 elements), 
    \(\mathbb{Z}_4 \oplus \mathbb{Z}_4\) is not cyclic because the largest element order 
    in \(\mathbb{Z}_4 \oplus \mathbb{Z}_4\) is 4. 

  \item \(\mathbb{Z}_6 \oplus \mathbb{Z}_4\).\\
    This is a direct sum of order \(6 \times 4 = 24\). Since \(\gcd(6,4) = 2\), no element can have order 
    24, so this group is \emph{not} cyclic. The maximum order of any element is 
    \(\mathrm{lcm}(6,4) = 12\).

  \item \(\mathbb{Z}_2 \oplus \mathbb{Z}_2 \oplus \mathbb{Z}_3\).\\
    This direct sum has \(2 \times 2 \times 3 = 12\) elements. Elements look like 
    \((a,b,c)\) with \(a,b \in \{0,1\}\) and \(c \in \{0,1,2\}\). Since \(\gcd(2,2,3)\neq 1\),
    there is no single element of order 12, so again it is not cyclic.

  \item \(\mathbb{Z} \oplus \mathbb{Z}\).\\
    Here is an \emph{infinite} example. The group \(\mathbb{Z} \oplus \mathbb{Z}\) is 
    just the set of all ordered pairs of integers with coordinatewise addition. 
    It is often denoted by \(\mathbb{Z}^2\). Unlike the finite examples above, 
    this group has infinitely many elements. 
\end{enumerate}

\noindent
\textbf{Note:} For finite abelian groups, the direct sum \(G_1 \oplus G_2 \oplus \cdots \oplus G_k\) 
has order equal to the product of the orders of each \(G_i\). Also, two direct sums 
\(\mathbb{Z}_m \oplus \mathbb{Z}_n\) and \(\mathbb{Z}_a \oplus \mathbb{Z}_b\) can be 
isomorphic only if they share the same invariant factors or elementary divisors 
(e.g., checking whether \(\gcd(m,n) = 1\) helps determine if the group is cyclic).


\section*{The Klein-4 Group}

The \emph{Klein-4 group}, often denoted by \(V_4\) or simply \(V\), is a group with four elements. It is the unique (up to isomorphism) non-cyclic group of order 4.

\subsection*{Definition and Properties}
One common presentation of \(V_4\) is:
\[
V_4 = \{e, a, b, c\},
\]
where:
\begin{itemize}
  \item \(e\) is the identity element.
  \item The elements satisfy:
    \[
      a^2 = b^2 = c^2 = e.
    \]
  \item The product of any two distinct non-identity elements gives the third:
    \[
      a \cdot b = c,\quad b \cdot c = a,\quad c \cdot a = b.
    \]
  \item The group is abelian; that is, for any \(x,y \in V_4\), we have \(xy = yx\).
\end{itemize}

\subsection*{Alternative Representation}
The Klein-4 group is also isomorphic to the direct sum of two copies of \(\mathbb{Z}_2\):
\[
V_4 \cong \mathbb{Z}_2 \oplus \mathbb{Z}_2.
\]
Here, each element can be represented as an ordered pair \((x,y)\) with \(x,y \in \{0,1\}\), and the group operation is defined by componentwise addition modulo 2:
\[
(x,y) + (x',y') = (x+x' \bmod 2,\; y+y' \bmod 2).
\]
\end{document}
