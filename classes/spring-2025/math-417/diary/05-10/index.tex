\documentclass[12pt]{article}

% Packages
\usepackage[margin=1in]{geometry}
\usepackage{amsmath,amssymb,amsthm}
\usepackage{enumitem}
\usepackage{hyperref}
\usepackage{xcolor}
\usepackage{import}
\usepackage{xifthen}
\usepackage{pdfpages}
\usepackage{transparent}
\usepackage{listings}
\usepackage{tikz}
\usepackage{physics}
\usepackage{siunitx}
\usepackage{booktabs}
\usepackage{cancel}
  \usetikzlibrary{calc,patterns,arrows.meta,decorations.markings}


\DeclareMathOperator{\Log}{Log}
\DeclareMathOperator{\Arg}{Arg}
\DeclareMathOperator{\Aut}{Aut}
\DeclareMathOperator{\Inn}{Inn}
\DeclareMathOperator{\Out}{Out}
\DeclareMathOperator{\F}{F}


\lstset{
    breaklines=true,         % Enable line wrapping
    breakatwhitespace=false, % Wrap lines even if there's no whitespace
    basicstyle=\ttfamily,    % Use monospaced font
    frame=single,            % Add a frame around the code
    columns=fullflexible,    % Better handling of variable-width fonts
}

\newcommand{\incfig}[1]{%
    \def\svgwidth{\columnwidth}
    \import{./Figures/}{#1.pdf_tex}
}
\theoremstyle{definition} % This style uses normal (non-italicized) text
\newtheorem{solution}{Solution}
\newtheorem{proposition}{Proposition}
\newtheorem{problem}{Problem}
\newtheorem{lemma}{Lemma}
\newtheorem{theorem}{Theorem}
\newtheorem{remark}{Remark}
\newtheorem{note}{Note}
\newtheorem{definition}{Definition}
\newtheorem{example}{Example}
\newtheorem{corollary}{Corollary}
\theoremstyle{plain} % Restore the default style for other theorem environments
%

% Theorem-like environments
% Title information
\title{MATH-417}
\author{Jerich Lee}
\date{\today}

\begin{document}

\maketitle
\begin{align}
  y^n 
    &= (g x g^{-1})^{\,n} \\[4pt]
    &= \underbrace{(g x g^{-1})(g x g^{-1})\dotsm(g x g^{-1})}_{n\ \text{copies}} \\[4pt]
    &= g\,x\,(g^{-1}g)\,x\,(g^{-1}g)\,\dotsm\,x\,g^{-1} \\[4pt]
    &= g\,x^{\,n}\,g^{-1}.
  \end{align}
  \begin{definition}[Centralizer]
    Let \(G\) be a group and \(S \subseteq G\) a (non-empty) subset.
    The \emph{centralizer} of \(S\) in \(G\) is
    \[
        C_G(S) \;=\; \{\,g \in G \mid gs = sg \text{ for every } s \in S \}.
    \]
    When \(S = \{x\}\) is a single element, we write
    \[
        C_G(x) \;=\; \{\,g \in G \mid gx = xg \}.
    \]
    The \emph{center} of \(G\) is the centralizer of all of \(G\):
    \[
        Z(G) \;=\; C_G(G) \;=\; \{\,g \in G \mid gh = hg \text{ for every } h \in G \}.
    \]
  \end{definition}
  \begin{proof}
    Let $G$ be a finite group with exactly two conjugacy classes.
    Since the identity element $e$ commutes with every element of $G$, its conjugacy class is the singleton $\{e\}$.
    Consequently the other class must be the set $G\setminus\{e\}$.
    
    Pick an element $g\in G\setminus\{e\}$.
    Its conjugacy class has size
    \[
        |g^G| \;=\; |G\setminus\{e\}| \;=\; |G|-1 .
    \]
    By the orbit–stabiliser theorem,
    \[
        |g^G| \;=\; [\,G:C_G(g)\,] 
        \;=\; \frac{|G|}{|C_G(g)|},
    \]
    so $|g^G|$ divides $|G|$.
    Thus $|G|-1\mid |G|$.
    
    Let $|G| = n$.  
    If $n-1 \mid n$ then $n-1 \mid n-(n-1)=1$, whence $n-1 = 1$ and $n=2$.
    
    Therefore $|G|=2$ and every group of order 2 is cyclic:
    \[
        G \;\cong\; \Bbb Z_2.\qedhere
    \]
    \end{proof}
    \begin{definition}[Stabilizer {\&} centralizer]
      Let $G$ act on a set $X$.  
      For $x\in X$ the \emph{stabilizer} of $x$ is
      \[
          G_x \;=\; \{\,g\in G \mid g\!\cdot\!x = x\}.
      \]
      
      Now let $X=G$ with the conjugation action $g\!\cdot\!x = gxg^{-1}$.
      Then
      \[
          G_x \;=\; \{\,g\in G \mid gxg^{-1}=x\}
                 \;=\; C_G(x),
      \]
      the \emph{centralizer} of $x$.
      \end{definition}
      \begin{remark}
        For $x\in G$ let
        \[
           C_G(x)=\{\,g\in G\mid gx=xg\},\qquad
           x^{G}=\{\,g x g^{-1}\mid g\in G\}.
        \]
        Then
        \[
           |x^{G}| = [G:C_G(x)] = \frac{|G|}{|C_G(x)|}.
        \]
        Since $C_G(x)$ is a subgroup and $x^{G}$ is an orbit, they coincide
        only when $|G|=1$ (the trivial group).  In every non-trivial group
        they are distinct: the centralizer contains the stabiliser information,
        while the conjugacy class contains the orbit information.
        \end{remark}
        Let $|G| = n \in \mathbb{N}$.
Assume $n-1 \mid n$.
Then there exists $k\in\mathbb{N}$ such that
\[
    n = k\,(n-1).
\]
But $n = (n-1)+1$, so
\[
    k\,(n-1) \;=\; (n-1)+1
    \;\;\Longrightarrow\;\;
    (k-1)\,(n-1) = 1.
\]
Both factors on the left are positive integers, hence each must equal $1$:
\[
    k-1 = 1
    \quad\text{and}\quad
    n-1 = 1.
\]
Consequently $n = 2$.  Therefore the only finite group with exactly two
conjugacy classes has order $2$, i.e.\ $G \cong \mathbb{Z}_2$.
\begin{definition}
  Let $G$ be a group.
  \begin{enumerate}
    \item For $g\in G$, the map
          \[
              \varphi_g : G \longrightarrow G, \quad
              h \mapsto g h g^{-1},
          \]
          is an \emph{inner automorphism}.
          The set of all such maps is denoted $\Inn(G)$.
    \item The full automorphism group is $\Aut(G)$.
          The \emph{outer automorphism group} is
          \[
              \Out(G) \;=\; \Aut(G)\big/ \Inn(G).
          \]
          Its non-trivial elements are called \emph{outer automorphisms}.
  \end{enumerate}
  \end{definition}
  \begin{problem}
    Show that any finite group of order $255$ is abelian.
    \end{problem}
    
    \begin{proof}[Solution]
    Write the order in prime factor form:
    \[
        |G| \;=\; 255 \;=\; 3\cdot 5\cdot 17 ,
    \qquad
        3<5<17.
    \]
    
    \paragraph{Step 1 — The Sylow $17$–subgroup is unique.}
    Let $n_{17}$ be the number of Sylow $17$–subgroups.  
    Sylow’s theorem gives
    \[
        n_{17}\mid 3\cdot 5=15,
        \qquad
        n_{17}\equiv 1 \;(\mathrm{mod}\; 17).
    \]
    The only divisor of $15$ congruent to $1$ modulo $17$ is $1$.  
    Hence the Sylow $17$–subgroup $P_{17}$ is **unique and therefore normal**.
    
    \paragraph{Step 2 — Bounding the numbers of Sylow $5$– and $3$–subgroups.}
    Similarly,
    \[
        n_{5}\mid 3\cdot 17 = 51, \quad n_{5}\equiv 1\pmod{5},
    \qquad
        n_{3}\mid 5\cdot 17 = 85, \quad n_{3}\equiv 1\pmod{3}.
    \]
    Thus
    \[
        n_{5}\in\{1,51\},
    \qquad
        n_{3}\in\{1,85\}.
    \]
    
    \paragraph{Step 3 — Eliminating the large possibilities.}
    Suppose $n_{5}=51$.  
    Each Sylow $5$–subgroup has $4$ non-identity elements, and distinct Sylow subgroups intersect trivially, so this would contribute
    \(
        51\cdot 4 = 204
    \)
    elements of order $5$.
    
    If simultaneously $n_{3}=85$, the Sylow $3$–subgroups would contribute 
    \(85\cdot 2 = 170\) non-identity elements of order $3$.  
    Counting also the identity, we would have
    \(
        204 + 170 + 1 = 375 > 255,
    \)
    a contradiction.  
    Hence \emph{at least one} of $n_{5},n_{3}$ equals $1$.
    
    \paragraph{Step 4 — Forcing both $n_{5}$ and $n_{3}$ to be $1$.}
    Assume $n_{5}=51$ but $n_{3}=1$ (unique Sylow $3$–subgroup $P_{3}$).  
    Then $P_{3}\trianglelefteq G$, so the quotient
    \(
        G/P_{3}
    \)
    has order $85 = 5\cdot 17$.  
    Because $5\nmid 17-1=16$, a standard Sylow argument shows that any group of order $85$ is cyclic.  
    A cyclic group has a \emph{unique} subgroup of order $5$, which pulls back to a unique Sylow $5$–subgroup of $G$—contradicting $n_{5}=51$.  
    Thus $n_{5}$ cannot be $51$; hence $n_{5}=1$.  
    An identical argument (swapping the roles of $3$ and $5$) forces $n_{3}=1$.
    
    \paragraph{Step 5 — All Sylow subgroups are normal and the group is cyclic.}
    We have shown
    \[
        n_{17}=n_{5}=n_{3}=1.
    \]
    Therefore each Sylow subgroup
    \(
        P_{17},\,P_{5},\,P_{3}
    \)
    is normal in $G$.  
    Their orders are pairwise coprime, so
    \[
        G\;=\;P_{3}\,P_{5}\,P_{17}\;\cong\;P_{3}\times P_{5}\times P_{17}.
    \]
    Each $P_{p}$ is cyclic (all groups of prime order are), and a direct product of cyclic groups of pairwise coprime orders is itself cyclic:
    \[
        P_{3}\times P_{5}\times P_{17}\;\cong\;\mathbb Z_{3}\times\mathbb Z_{5}\times\mathbb Z_{17}
                                            \;\cong\;\mathbb Z_{255}.
    \]
    A cyclic group is abelian, so $G$ is abelian.
    \end{proof}
    \textbf{Claim.} If $P_{3}\trianglelefteq G$ and $G/P_{3}$ is cyclic of
order $85$, then $n_{5}=1$.

\emph{Proof.}
Let $\pi:G\to G/P_{3}$ be the quotient map.
Because $\ker\pi=P_{3}$ has order $3$, $\pi$ is injective on every
Sylow $5$–subgroup.
The cyclic group $G/P_{3}$ has a unique subgroup
$H=\langle \pi(g)^{17}\rangle$ of order $5$.
Hence for each Sylow $5$–subgroup $S\le G$ one has
$\pi(S)=H\;(=|H|=5)$, so $S\subseteq\pi^{-1}(H)$.
But $\lvert\pi^{-1}(H)\rvert=|P_{3}||H|=15$ and it can contain
\emph{at most} one subgroup of order $5$.
Thus $n_{5}=1$.
\qed
\begin{proof}[Detailed expansion of Step 4]
  Recall that $|G| = 3\cdot 5 \cdot 17 = 255$ and
  \[
  n_{17}=1,\qquad 
  n_{5}\in\{1,51\},\qquad 
  n_{3}\in\{1,85\}.
  \]
  
  \begin{enumerate}
  %-----------------------------------------------------------
  \item \textbf{Assume by contradiction that $n_{5}=51$ and $n_{3}=1$.}
  
        Thus the unique Sylow $3$–subgroup 
        $P_{3}\le G$ is normal, $P_{3}\trianglelefteq G$.
        \paragraph{Why does \(n_{3}=1\) make \(P_{3}\) normal?}

Let \(P_{3}\le G\) be the (only) Sylow \(3\)-subgroup.

\begin{enumerate}
    \item Sylow’s theorem guarantees that for every \(g\in G\) the conjugate
          \(gP_{3}g^{-1}\) is again a Sylow \(3\)-subgroup because
          \[
              \lvert gP_{3}g^{-1}\rvert 
              = \lvert P_{3}\rvert
              = 3 .
          \]
    \item If \(n_{3}=1\) there is \emph{exactly one} Sylow \(3\)-subgroup in
          the whole group, namely \(P_{3}\) itself.
    \item Hence for every \(g\in G\)
          \[
              gP_{3}g^{-1}\;=\;P_{3},
          \]
          i.e.\ \(P_{3}\) is fixed under conjugation by every element of
          \(G\).
\end{enumerate}

Therefore \(P_{3}\) is invariant under all inner automorphisms and is
normal:
\[
    P_{3}\trianglelefteq G.
\]
        
  %-----------------------------------------------------------
  \item \textbf{Pass to the quotient by $P_{3}$.}
  
        Because $|P_{3}|=3$ and $P_{3}\trianglelefteq G$, the quotient
        $$
            \bar{G}\;=\;G/P_{3}
        $$
        is a group of order
        $$
            |\bar{G}| \;=\;\frac{|G|}{|P_{3}|}\;=\;\frac{255}{3}\;=\;85
            \;=\;5\cdot17.
        $$
  
  %-----------------------------------------------------------
  \item \textbf{A group of order $85$ is cyclic.}
  
        Let $q=17>p=5$.  
        Sylow’s theorem in $\bar{G}$ gives
        \begin{align}
            n_{17}\mid 5
            \ \text{and}\ 
            n_{17}\equiv 1\pmod{17}
            &\;\Longrightarrow\;
            n_{17}=1, \\
            n_{5}\mid 17
            \ \text{and}\ 
            n_{5}\equiv 1\pmod{5}
            &\;\Longrightarrow\;
            n_{5}=1.
        \end{align}
        Hence $\bar{G}$ possesses \emph{unique} Sylow subgroups
        $\bar{P}_{17}$ and $\bar{P}_{5}$, both normal.  
        Their orders are coprime, so
        $$
            \bar{G}\;=\;\bar{P}_{5}\,\bar{P}_{17}
            \;\cong\;
            \bar{P}_{5}\times\bar{P}_{17}
            \;\cong\;
            \mathbb{Z}_{5}\times\mathbb{Z}_{17}
            \;\cong\;
            \mathbb{Z}_{85},
        $$
        i.e.\ $\bar{G}$ is cyclic.
  
  %-----------------------------------------------------------
  \item \textbf{Cyclic $\bar{G}$ has a \emph{unique} subgroup of order $5$.}
  
        Let $\bar{H}$ denote this subgroup of order $5$ in $\bar{G}$.
        Because $\bar{G}$ is cyclic, $\bar{H}=\langle \bar{g}^{17}\rangle$
        for any generator $\bar{g}$ of $\bar{G}$.
  
  %-----------------------------------------------------------
  \item \textbf{Lift $\bar{H}$ back to $G$.}
  
        Let $\pi:G\twoheadrightarrow\bar{G}=G/P_{3}$ be the quotient map.
        Set $H:=\pi^{-1}(\bar{H})$, so
        $|H| = |\bar{H}|\cdot|P_{3}| = 5\cdot 3 = 15$
        and $P_{3}\le H\le G$.
  
        Each Sylow $5$–subgroup $S\le G$ satisfies $\pi(S)=\bar{H}$:
        since $|S|=5$ and $\ker\pi=P_{3}$ has order $3$ coprime to $5$,
        we have $\ker(\pi|_S)=\{e\}$, so $\pi$ is injective on $S$ and
        $\pi(S)$ is a subgroup of $\bar{G}$ of order $5$, hence $\bar{H}$.
  
        Therefore \emph{every} Sylow $5$–subgroup of $G$ lies inside $H$.
  %%%%%%%%%%%%%%%%%%%%%%%%%%%%%%%%%%%%%%%%%%%%%%%%%%%%%%%%%%%%%%%%%%%%%%%%
%  Expanded explanation of “Step 5 — Lift $\bar{H}$ back to $G$’’
%%%%%%%%%%%%%%%%%%%%%%%%%%%%%%%%%%%%%%%%%%%%%%%%%%%%%%%%%%%%%%%%%%%%%%%%

\paragraph{Step 5 — Lifting the subgroup \(\bar{H}\) of order \(5\) from \(\bar{G}=G/P_{3}\) back to \(G\).}

\begin{enumerate}
%......................................................................
\item \textbf{Set–up and basic facts.}

      Let
      \[
          \pi : G \;\twoheadrightarrow\; \bar{G}=G/P_{3}
      \]
      be the canonical quotient map.  
      By construction, \(\ker\pi = P_{3}\) has order \(3\).

      Inside \(\bar{G}\cong\Bbb Z_{85}\) choose its \emph{unique}
      subgroup of order \(5\),
      \[
          \bar{H}\;=\;\langle\bar{g}^{17}\rangle
          \qquad(\lvert\bar{H}\rvert = 5).
      \]

%......................................................................
\item \textbf{Define the pre-image of \(\bar{H}\).}

      \[
          H \;:=\; \pi^{-1}(\bar{H})
          \;=\;
          \bigl\{\,g\in G \mid \pi(g)\in\bar{H}\bigr\}.
      \]
      The subgroup \(H\) contains \(P_{3}\) (because \(P_{3}=\ker\pi\))
      and satisfies
      \[
          \lvert H\rvert
          \;=\;
          \lvert\bar{H}\rvert\;\lvert P_{3}\rvert
          \;=\;
          5\cdot 3
          \;=\;
          15.
      \]
      Thus
      \[
          P_{3}\;\le\;H\;\le\;G,
          \quad
          \text{with}\;
          \lvert H\rvert = 3\cdot5 .
      \]

%......................................................................
\item \textbf{Image of every Sylow \(5\)-subgroup lands in \(\bar{H}\).}

      Let \(S\le G\) be \emph{any} Sylow \(5\)-subgroup (\(\lvert S\rvert =5\)).
      Because \(\ker\pi = P_{3}\) has order \(3\) coprime to \(5\),
      we have
      \[
          \ker\bigl(\pi{\restriction_{S}}\bigr)
          \;=\;
          P_{3}\cap S
          \;=\;\{e\}.
      \]
      Hence \(\pi{\restriction_{S}}\) is \emph{injective}, so
      \[
          \lvert\pi(S)\rvert = \lvert S\rvert = 5.
      \]
      But in \(\bar{G}\) there is \emph{only one} subgroup of order \(5\),
      namely \(\bar{H}\).  Therefore
      \[
          \pi(S)=\bar{H}.
      \]

%......................................................................
\item \textbf{Consequences for containment.}

      Since \(\pi(S)=\bar{H}\)
      and \(H=\pi^{-1}(\bar{H})\),
      it follows immediately that
      \[
          S \;\subseteq\; H.
      \]
      Because \(S\) was an \emph{arbitrary} Sylow \(5\)-subgroup of \(G\),
      we have shown:
      \[
          \boxed{\;
            \text{Every Sylow }5\text{-subgroup of }G
            \text{ is contained in }H.
          \;}
      \]
\end{enumerate}

\noindent
This containment is the key to the contradiction:  
\(H\) has order \(15\) and can house \emph{at most one} subgroup of
order \(5\) (their pairwise intersections are trivial, so two distinct
ones would already contribute \(5+5-1=9>15-3\) non-identity elements).
Hence all Sylow \(5\)-subgroups coincide, implying \(n_{5}=1\),
contradicting the earlier assumption \(n_{5}=51\).
  %-----------------------------------------------------------
  \item \textbf{$H$ contains exactly \underline{one} Sylow $5$–subgroup.}
  
        A subgroup of order $15$ can contain at most one subgroup of
        order $5$ because the intersection of any two distinct such
        subgroups would be trivial, giving $5+5-1>15$.  Concretely,
        $15 = 3\cdot5$ and $\gcd(3,5)=1$ force uniqueness.
  
        Hence $H$ has a \emph{single} subgroup of order $5$, which means
        $$
            n_{5}=1,
        $$
        contradicting the assumption $n_{5}=51$.
  
        Thus our initial assumption is impossible; we conclude
        $$
            n_{5}=1.
        $$
  %%%%%%%%%%%%%%%%%%%%%%%%%%%%%%%%%%%%%%%%%%%%%%%%%%%%%%%%%%%%%%%%%%%%%%%%
%  Why a subgroup $H$ of order $15$ has \emph{exactly one} subgroup of
%  order $5$  (expanded explanation for Step 6)
%%%%%%%%%%%%%%%%%%%%%%%%%%%%%%%%%%%%%%%%%%%%%%%%%%%%%%%%%%%%%%%%%%%%%%%%

\begin{lemma}\label{lem:15-cyclic}
  Every group of order $15$ is cyclic.
  \end{lemma}
  
  \begin{proof}
  Write $|H|=15 = 3\cdot 5$ with primes $p=3<q=5$ and note that
  $p\nmid(q-1)$ because $3\nmid 4$.
  
  \begin{enumerate}
  \item \textbf{The Sylow--$5$ subgroup is unique and normal.}
        Sylow’s theorem gives
        \[
            n_{5}\mid 3
            \quad\text{and}\quad
            n_{5}\equiv 1 \pmod{5}.
        \]
        The only divisor of $3$ congruent to $1 \pmod{5}$ is $1$,
        so the Sylow--$5$ subgroup $Q$ is unique and therefore normal.
  
  \item \textbf{The conjugation homomorphism into $\Aut(Q)$ is trivial.}
        Because $Q\cong\Bbb Z_{5}$, its automorphism group is
        $\Aut(Q)\cong\Bbb Z_{4}$ of order $4$.
        Conjugation induces a homomorphism
        \[
            \varphi:H \longrightarrow \Aut(Q),
            \qquad
            h\mapsto (x\mapsto hxh^{-1}).
        \]
        Its image has order dividing both $|H|$ and $|\Aut(Q)|$, hence
        divides $\gcd(15,4)=1$; thus $\varphi$ is trivial and $Q$
        lies in the centre of $H$.
  
  \item \textbf{Direct product decomposition and cyclicity.}
        Let $P$ be a Sylow--$3$ subgroup.  Because $|P|=3$ is coprime to
        $|Q|=5$ and $Q\le Z(H)$,
        \[
            H \;=\; P Q
            \;\cong\; P\times Q
            \;\cong\; \Bbb Z_{3}\times\Bbb Z_{5}
            \;\cong\; \Bbb Z_{15},
        \]
        a cyclic group.
  \end{enumerate}
  \end{proof}
  
  \medskip
  \noindent
  \textbf{Consequences for Sylow--$5$ subgroups inside $H$.}
  
  Since $H$ is cyclic of order $15$, the subgroups of $H$ are exactly
  the unique subgroups of each order dividing $15$:
  \[
      \langle g^{3}\rangle \;(\text{order }5), 
      \quad
      \langle g^{5}\rangle \;(\text{order }3),
      \quad
      \langle g^{15}\rangle = \{e\}.
  \]
  Hence:
  
  \[
  \boxed{\;H\text{ contains a \emph{single} subgroup of order }5.\;}
  \]
  
  \bigskip
  \noindent
  \textbf{Alternative counting argument (if one prefers not to invoke Lemma~\ref{lem:15-cyclic}).}
  
  Suppose, to the contrary, that $H$ contains two distinct subgroups
  $S_{1},S_{2}$ of order $5$.
  
  \begin{itemize}
  \item Groups of prime order are cyclic, so
        $S_{1}\cap S_{2}$ is a common subgroup of order dividing $5$.
        If $S_{1}\neq S_{2}$, the intersection must be $\{e\}$.
  \item Then $|S_{1}\cup S_{2}| = |S_{1}|+|S_{2}|-|\{e\}| = 5+5-1 = 9$.
  \item $H$ also contains the Sylow--$3$ subgroup $P_{3}$ of order $3$.
        Aside from the identity, $P_{3}$ contributes $2$ further elements.
  \end{itemize}
  
  Consequently $H$ would have at least $9+2 = 11$ distinct non-identity
  elements, plus the identity, totalling at least $12$, which forces
  $|H|\ge12$.  However $|H|=15$, and \emph{any} additional element of
  order $5$ or $3$ would push the count past $15$, contradicting the fixed
  order.  Therefore only \emph{one} subgroup of order $5$ can exist in $H$.
  
  \bigskip
  Either route yields the desired uniqueness; in particular
  \[
      n_{5}=1,
  \]
  contradicting the earlier assumption $n_{5}=51$ and completing Step 6.
  %-----------------------------------------------------------
  \item \textbf{Swap the roles of $3$ and $5$.}
  
        An identical argument—starting with $n_{3}=85$ and $n_{5}=1$,
        quotienting by the normal Sylow $5$–subgroup—forces $n_{3}=1$.
        Therefore
        $$
            n_{3}=n_{5}=1.
        $$
  
  \end{enumerate}
  
  Consequently \emph{all} Sylow subgroups ($P_{3},P_{5},P_{17}$) are
  normal in $G$, whence
  $
      G\cong P_{3}\times P_{5}\times P_{17}
            \cong\mathbb{Z}_{3}\times\mathbb{Z}_{5}\times\mathbb{Z}_{17}
            \cong\mathbb{Z}_{255},
  $
  which is abelian.
  \end{proof}
  \paragraph{Why the “extra element’’ remark is needed—and why it leads to a contradiction.}

Assume, for contradiction, that \(H\) (with \(|H|=15\)) contains
\emph{two} distinct subgroups of order \(5\),
\(S_{1}\) and \(S_{2}\).

\begin{enumerate}
\item[\textbf{(i)}]  Because groups of prime order are cyclic,
      \(S_{1}\cap S_{2}=\{e\}\).
      Hence
      \[
          |S_{1}\cup S_{2}| \;=\;
          |S_{1}|+|S_{2}|-|\{e\}|
          \;=\; 5+5-1
          \;=\; 9
      \]
      non-identity elements
      (together with \(e\), that makes \(10\) elements counted so far).

\item[\textbf{(ii)}]  \(H\) \emph{also} contains \emph{its} Sylow-\(3\)
      subgroup \(P_{3}\) of order \(3\),
      which contributes exactly two further non-identity elements
      (they are not in \(S_{1}\cup S_{2}\) because
      \(\gcd(3,5)=1\)).
      Now we have already identified
      \[
          9 + 2 = 11
      \]
      non-identity elements,
      plus the identity,
      so at least \(12\) elements in total.

\item[\textbf{(iii)}]  But \(|H|=15\), so \emph{exactly three} elements
      are still unaccounted for.
      What could their orders be?

      \begin{itemize}
      \item If an uncounted element had order \(5\),
            it would generate a \emph{third} subgroup of order \(5\),
            necessarily disjoint from \(S_{1}\cup S_{2}\)
            (apart from \(e\)),
            adding at least \(4\) new non-identity elements
            and pushing the total \(\ge 16>15\).

      \item If an uncounted element had order \(3\),
            it would either lie in \(P_{3}\) (already counted)
            or generate a \emph{second} subgroup of order \(3\),
            disjoint from \(P_{3}\) except for \(e\),
            adding \(2\) new non-identity elements
            and taking the total to \(14\);
            a second such element (necessarily present,
            because a subgroup of order \(3\) has \(2\) generators)
            would push the total to \(16>15\).
      \end{itemize}

      Therefore \emph{no new element of order \(5\) or \(3\) can exist},
      but those are the only possible prime orders dividing \(15\).
      Hence no uncounted elements can exist at all.
\end{enumerate}

We have reached a contradiction:
we cannot fill the remaining three “slots’’ in a way compatible with
the assumption \(S_{1}\neq S_{2}\).
Consequently \(H\) can contain \emph{only one} subgroup of order \(5\).

\[
\boxed{\;n_{5}=1\;}
\]

This contradicts the temporary assumption \(n_{5}=51\) and completes
the counting argument.
\paragraph{Corrected counting argument.}

Assume $H$ (with $\lvert H\rvert = 15$) contains two \emph{distinct}
subgroups of order $5$, say $S_1$ and $S_2$.

\begin{itemize}
    \item Each subgroup of order $5$ is cyclic, so
          $S_1 \cap S_2 = \{e\}$.

          Hence
          \[
              \lvert S_1 \cup S_2\rvert
              = \lvert S_1\rvert + \lvert S_2\rvert - \lvert\{e\}\rvert
              = 5 + 5 - 1
              = 9
          \]
          elements, of which $8$ are non-identity.

    \item $H$ also contains its Sylow–$3$ subgroup
          $P_3$ (order $3$).  Aside from $e$, $P_3$ contributes
          $2$ more non-identity elements that
          cannot lie in $S_1 \cup S_2$ because $\gcd(3,5)=1$.

          We have therefore accounted for
          \[
              8 \;+\; 2 \;=\; 10 \text{ non-identity elements},
          \]
          plus $e$, totalling $11$ elements.

    \item Exactly $4$ elements of $H$ remain unaccounted for.
          Any such element must have order $3$ or $5$ (the only prime
          divisors of $15$):

          \begin{enumerate}
              \item If one of the remaining elements has order $5$,
                    it generates a \emph{third} subgroup of order $5$,
                    disjoint from $S_1\cup S_2$ except at $e$.
                    This contributes \emph{at least} $4$ new
                    non-identity elements, pushing the total
                    non-identity count to $14$ and the group size to
                    $\ge 15 + (4-1) = 18 > 15$ — impossible.

              \item If one of the remaining elements has order $3$ and
                    is \emph{not} in $P_3$, it generates another
                    subgroup of order $3$ disjoint from $P_3$ except at
                    $e$.  That adds $2$ new non-identity elements; a
                    second generator of that same subgroup adds one more,
                    so we exceed $15$ again.
          \end{enumerate}
\end{itemize}

Thus it is impossible for $H$ to contain two distinct subgroups of
order $5$.  Therefore \(H\) has \emph{exactly one} Sylow–$5$ subgroup,
implying \(n_5 = 1\).

\begin{problem}
  Show that every group of order $255$ is abelian.
\end{problem}

\begin{proof}
  \begin{enumerate}
      \item \textbf{Prime factorisation and Sylow counts.}\\
            Since $255 = 3\cdot 5\cdot 17$, let $n_{p}$ denote the number of Sylow $p$–subgroups of a group $G$ with $\lvert G\rvert =255$.
            For $p = 17$ we have
            \begin{align}
                n_{17} \equiv 1 \pmod{17},
                \qquad
                n_{17} \mid 3\cdot 5 = 15 .
            \end{align}
            The only divisor of $15$ that is congruent to $1$ modulo $17$ is $1$, so $n_{17}=1$.
            Hence the unique Sylow $17$–subgroup
            \begin{align}
                G_{17}\le G,\qquad \lvert G_{17}\rvert = 17
            \end{align}
            is \emph{normal}.

      \item \textbf{The conjugation action on $G_{17}$ is trivial.}\\
            Consider the conjugation homomorphism
            \begin{align}
                \alpha\colon G \longrightarrow \operatorname{Aut}(G_{17}),
                \qquad
                \alpha(g)(h) = ghg^{-1}\quad (g\in G,\;h\in G_{17}).
            \end{align}
            Because $G_{17}\cong \mathbb{Z}_{17}$ is cyclic, its automorphism group is
            \begin{align}
                \operatorname{Aut}(G_{17})\cong \mathbb{Z}_{17}^{\times},
                \qquad
                \lvert \operatorname{Aut}(G_{17}) \rvert = 16 .
            \end{align}
            Fix any $g\in G$.  Its order divides $255$, so
            \begin{align}
                \operatorname{ord}_{\operatorname{Aut}(G_{17})}\!\bigl(\alpha(g)\bigr)
                \mid 255
                \quad\text{and}\quad
                \operatorname{ord}_{\operatorname{Aut}(G_{17})}\!\bigl(\alpha(g)\bigr)
                \mid 16 .
            \end{align}
            Because $\gcd(255,16)=1$, the only positive integer dividing \emph{both} $255$ and $16$ is $1$.
            Thus $\alpha(g)=\operatorname{id}_{G_{17}}$ for every $g\in G$; equivalently,
            \begin{align}
                ghg^{-1}=h \qquad\text{for all } g\in G,\;h\in G_{17}.
            \end{align}
            Hence $G_{17}\le Z(G)$, the center of $G$.

      \item \textbf{The quotient $G/G_{17}$ is cyclic of order $15$.}\\
            Because $G_{17}$ is normal, the quotient has order
            \begin{align}
                \lvert G/G_{17}\rvert = \frac{255}{17}=15 = 3\cdot 5 .
            \end{align}
            A standard result (groups of order $pq$ with $p<q$ and $p\nmid(q-1)$ are cyclic) shows that every group of order $15$ is cyclic.  Thus $G/G_{17}\cong\mathbb{Z}_{15}$.

      \item \textbf{Every element of $G$ splits uniquely into a ``$\mathbb{Z}_{15}$–part'' and a ``$G_{17}$–part''.}\\
            Choose $a\in G$ whose coset $aG_{17}$ generates $G/G_{17}$.
            Then each $g\in G$ can be written uniquely as
            \begin{align}
                g = a^{k}z, \qquad 0\le k\le 14,\;z\in G_{17}.
            \end{align}

      \item \textbf{Elements commute, so $G$ is abelian.}\\
            Take two arbitrary elements
            \begin{align}
                g  = a^{k}z,\qquad
                g' = a^{\ell}z',
                \qquad
                z,z'\in G_{17}.
            \end{align}
            Because $G_{17}\le Z(G)$, we have $zz' = z'z$ and $a^{k}z = za^{k}$, etc.\  Hence
            \begin{align}
                gg'
                &= a^{k}z\,a^{\ell}z'
                 = a^{k+\ell}zz' ,\\
                g'g
                &= a^{\ell}z'\,a^{k}z
                 = a^{\ell+k}z'z
                 = a^{k+\ell}zz' .
            \end{align}
            Thus $gg' = g'g$ for all $g,g'\in G$; $G$ is abelian.

      \item \textbf{$G$ is cyclic.}\\
            Any abelian group of square–free order is cyclic, so
            \begin{align}
                G \cong \mathbb{Z}_{255}.
            \end{align}
  \end{enumerate}
\end{proof}
% Explanation of the “+\,\tfrac12” trick for rounding to the nearest integer
Let \(\lfloor\cdot\rfloor\) denote the ordinary **floor** function, which always  
rounds a real number \emph{down} to the greatest integer \(\le\) that number.  
To transform “round down’’ into “round to the nearest integer’’ we first shift
the real line by \(\tfrac12\) and then apply the floor:

\begin{align}
r \;:=\; \Bigl\lfloor x + \tfrac12 \Bigr\rfloor \;\in\; \mathbb{Z}.
\end{align}

---

 Why does this work?

Write \(x = n + \theta\) with \(n\in\mathbb{Z}\) and \(0 \le \theta < 1\).
Adding \(\tfrac12\) gives  
\(x + \tfrac12 = n + \bigl(\theta + \tfrac12\bigr)\).

\[
\begin{array}{c|c}
0 \le \theta < \tfrac12 & \tfrac12 \le \theta < 1 \\ \hline
n \le x < n+\tfrac12     & n+\tfrac12 \le x < n+1 \\
n \le x+\tfrac12 < n+1   & n+1 \le x+\tfrac12 < n+2
\end{array}
\]

Hence
\[
\lfloor x + \tfrac12 \rfloor
=
\begin{cases}
n,   & 0 \le \theta < \tfrac12,\\[4pt]
n+1, & \tfrac12 \le \theta < 1,
\end{cases}
\]
exactly the integer \emph{nearest} to \(x\) (ties at
\(\theta=\tfrac12\) are broken toward \(n+1\), but any fixed
tie-breaking rule suffices).

Consequently
\[
|x-r|\;\le\;\tfrac12,
\qquad
r = \lfloor x+\tfrac12\rfloor \text{ is the unique integer with this property.}
\]

---

Application to Gaussian integers

For \(z = x + iy \in \mathbb{C}\) set
\[
q \;=\;
\bigl\lfloor x + \tfrac12\bigr\rfloor
\;+\;
\bigl\lfloor y + \tfrac12\bigr\rfloor\,i
\;\in\; \mathbb{Z}[i].
\]
Since both real and imaginary parts are within \(\tfrac12\) of those of
\(z\), we have
\[
|z-q|
=\sqrt{(x-\operatorname{Re}q)^2 + (y-\operatorname{Im}q)^2}
\;\le\;
\sqrt{\bigl(\tfrac12\bigr)^2 + \bigl(\tfrac12\bigr)^2}
\;=\;
\frac1{\sqrt2},
\]
so \(q\) is a “nearest’’ Gaussian integer to \(z\).

---

 Why not simply \(\lfloor x\rfloor\)?

Using \(q = \lfloor x\rfloor + i\lfloor y\rfloor\) would always
round \emph{down}, not to the nearest lattice point, giving
\(|x-\lfloor x\rfloor|<1\) but potentially as large as \(1\),
while the \(\tfrac12\)-shift guarantees a maximal error of only
\(\tfrac12\) in each coordinate (and \(\tfrac1{\sqrt2}\) overall).
% Why the homomorphism  \psi : \mathbb Z \to \mathbb Z[i]/I \; (n\mapsto n+I) \; is introduced

\bigskip
\textbf{Goal}.  In part~(b) we must \emph{identify} the quotient ring
\(\mathbb Z[i]/I\) where \(I=(2+i)\).
A slick way to recognise an unfamiliar quotient is to produce a
\emph{surjective ring homomorphism} onto it and then invoke the
First Isomorphism Theorem.  
That is exactly the rôle of
\[
\psi:\;\mathbb Z \;\longrightarrow\; \mathbb Z[i]/I,
\qquad n\;\longmapsto\; n+I.
\]

\bigskip
\textbf{Why does \(\psi\) help?}

\begin{enumerate}
    \item \emph{It is automatically a ring homomorphism.}  
          Addition and multiplication of cosets are defined component-wise,
          so sending \(n\) to \(n+I\) plainly respects \(+\) and \(\cdot\).

    \item \emph{Its image is everything.}  
          Every coset \(z+I\) contains an \emph{integer}:
          write \(z=x+iy\) and subtract \(\lfloor x+\tfrac12\rfloor
          +i\lfloor y+\tfrac12\rfloor\in I\) as in Step~2 of the notes.
          Hence \(\psi\) is surjective.

    \item \emph{The kernel is easy to compute.}  
          An integer \(n\) lies in \(\ker\psi\) precisely when
          \(n\in I\), i.e.\ when \(2+i\) divides \(n\) in
          \(\mathbb Z[i]\).
          Write \(n=(2+i)u\) with \(u=a+bi\in\mathbb Z[i]\) and take norms:
          \[
               n^{2}=N(n)=N\bigl((2+i)u\bigr)=N(2+i)\,N(u)=5\bigl(a^{2}+b^{2}\bigr).
          \]
          Because \(5\mid n^{2}\) and \(5\) is prime in \(\mathbb Z\),
          we get \(5\mid n\).  Thus
          \[
               \ker\psi = 5\mathbb Z.
          \]

    \item \emph{Apply the First Isomorphism Theorem.}  
          We obtain the canonical isomorphism
          \[
                \mathbb Z[i]/I 
                \;\cong\; 
                \mathbb Z/\ker\psi
                \;=\;
                \mathbb Z/5\mathbb Z
                \;=\;
                \mathbb Z_{5}.
          \]
\end{enumerate}

\medskip
\textbf{Summary}.  Introducing \(\psi\) packages the proof of
\(\mathbb Z[i]/(2+i)\cong\mathbb Z_{5}\) into a
single clean argument:
\[
\boxed{\;
    \text{construct a surjective homomorphism }\psi,\;
    \text{find }\ker\psi,\;
    \text{invoke the First Isomorphism Theorem.}
\;}
\]
Without \(\psi\) one would have to manipulate cosets by hand and guess
the isomorphism, a longer and less systematic route.
The implication  
\[
\bigl(\,p \text{ prime and } p \mid n^{2}\bigr)\; \Longrightarrow\; p \mid n
\]
is \emph{always} true in \(\mathbb{Z}\) (it is Euclid’s lemma).  
Consequently **there is no counter-example when \(p\) is prime.**

\medskip
What \emph{can} fail is the same implication with a **composite** divisor.
For instance
\[
4 \;\mid\; 6^{2}=36
\qquad\text{but}\qquad
4 \nmid 6.
\]
Here \(4\) is not prime, so Euclid’s lemma does not apply.  
This example shows that the primality condition is indispensable.
% Euclid’s Lemma in $\Bbb Z$ and its proof
\begin{theorem}[Euclid’s Lemma]
  Let $p$ be a prime number and let $a,b\in\Bbb Z$.  
  If $p\mid ab$, then $p\mid a$ or $p\mid b$.
\end{theorem}

\begin{proof}
  \begin{enumerate}
      \item Because $p$ is prime, its only positive divisors are $1$ and $p$.  
            Hence $\gcd(p,a)$ is either $1$ or $p$.
      \item Set $d=\gcd(p,a)$.  Apply Bézout’s identity: there exist integers $u,v$
            such that
            \begin{align}
                up+va=d.
            \end{align}
      \item Multiply both sides by $b$:
            \begin{align}
                upb+vab=db.
            \end{align}
            By hypothesis $p\mid ab$, so $p\mid vab$.  Clearly $p\mid upb$.
            Therefore $p\mid (upb+vab)=db$.
      \item Since $p\mid db$ and $d=\gcd(p,a)$ divides $p$, there are two cases:
            \begin{enumerate}
                \item $d=p$.  Then $p\mid a$ (because $d=\gcd(p,a)$ divides $a$).
                \item $d=1$.  Then the previous line gives $p\mid b$.
            \end{enumerate}
      \item In either case we have shown $p\mid a$ or $p\mid b$, completing the proof.
  \end{enumerate}
\end{proof}

\bigskip
\textbf{Corollary (Square version).}
If $p$ is prime and $p\mid n^{2}$, then $p\mid n$.  
\emph{Proof.}  Apply Euclid’s lemma to $ab = n\cdot n$.
% Name and interpretation of the identity
\[
(1-x)\bigl(1+x+\dots+x^{k-1}\bigr)=1-x^{k}\qquad(k\in\Bbb N,\;k\ge 1).
\]

This is the \textbf{finite geometric-series identity}.  
Equivalently, it is the factorisation of the \emph{difference of $k$-th powers}:

\[
1-x^{k}=(1-x)\bigl(1+x+x^{2}+\dots+x^{k-1}\bigr).
\]

\bigskip
\textbf{Why it holds.}  Expand the left-hand side:

\[
(1-x)(1+x+\dots+x^{k-1})
=1+x+\dots+x^{k-1}-\bigl(x+x^{2}+\dots+x^{k}\bigr),
\]
and notice that every term except the first \(1\) and the last \(-x^{k}\)
cancels, giving \(1-x^{k}\).

\bigskip
\textbf{Standard rearrangement.}  For \(x\neq1\) one often writes
\[
1+x+\dots+x^{k-1}
=\frac{1-x^{k}}{1-x},
\]
so the displayed identity is just that equation multiplied through by
\((1-x)\).
% Why $\;\Bbb Z/5\Bbb Z\;$ and $\;\Bbb Z_{5}\;$ are the same object

By definition
\[
\Bbb Z/5\Bbb Z
=\bigl\{\,\overline{n}\;:\;n\in\Bbb Z\bigr\},
\qquad
\overline{n}:=\{\,n+5k : k\in\Bbb Z\,\},
\]
with addition and multiplication performed
\emph{component-wise} and then reduced modulo \(5\).

The symbol \(\Bbb Z_{5}\) (or sometimes \(\mathbf{F}_{5}\))
is simply a \emph{shorthand name} for this very ring:
\[
\Bbb Z_{5}
\;=\;
\{\,0,1,2,3,4\,\},
\quad
\text{operations taken modulo }5.
\]

The map
\[
\begin{aligned}
\phi:\;\Bbb Z/5\Bbb Z &\;\longrightarrow\; \Bbb Z_{5},\\
\overline{n}&\;\longmapsto\;n\pmod{5}\;\in\{0,1,2,3,4\},
\end{aligned}
\]
is a well-defined bijective ring homomorphism, hence an \emph{isomorphism}.
Because there is a unique ring of order \(5\) up to isomorphism,
it is customary (and harmless) to \emph{identify}
\(\Bbb Z/5\Bbb Z\) with \(\Bbb Z_{5}\) and write
\[
\boxed{\;\Bbb Z/5\Bbb Z\;=\;\Bbb Z_{5}\;}
\]
whenever convenient.
\bigskip
\textbf{Correction of Step 3 (surjectivity).}

When we divide $z=a+bi\in\Bbb Z[i]$ by $2+i$ we obtain
\[
z=(2+i)q+r, \qquad q,r\in\Bbb Z[i],\qquad N(r)<5.
\]
Because $N(r)=x^{2}+y^{2}$ for some $x,y\in\Bbb Z$, the
possible values of $N(r)$ under the bound $N(r)<5$ are
\[
N(r)\in\{0,1,2,4\}.
\]
(The value $3$ cannot occur since $x^{2}+y^{2}\equiv0,1,2\pmod4$ but
never $3$.)  A complete set of representatives is

\[
\begin{array}{c|c|c}
N(r) & r & \text{integer congruent to }r\pmod{I=(2+i)}\\\hline
0 & 0                     & 0\\
1 & 1,\,-1,\,i,\,-i       & 1\ (\,\text{or }4\text{ if }r=-1)\\
2 & 1+i,\,-1+i,\;1-i,\,-1-i & 2\ (\,\text{or }3,4)\\
4 & 2,\,-2,\,2i,\,-2i     & 0
\end{array}
\]

\emph{How to see the “integer” column.}
Write $r=x+yi$.  Reduce modulo $I=(2+i)$ by subtracting
multiples of $2+i$ until only the integer part remains:

\[
\begin{aligned}
1+i &\equiv (1+i)-(2+i)=-1\equiv 4\pmod{I},\\
-1+i&\equiv(-1+i)+(2-i)=1\pmod{I},\\
1-i &\equiv(1-i)-(2+i)=-1-2i\equiv2\pmod{I},
\end{aligned}
\]
etc.  In every case $r$ is congruent to \emph{some} integer
in $\{0,1,2,3,4\}$.

Hence \textbf{every} coset of $I$ contains an integer
representative, so the homomorphism
\[
\psi:\Bbb Z\longrightarrow\Bbb Z[i]/I,\qquad n\longmapsto n+I
\]
is \textbf{surjective}.

\bigskip
\textbf{Conclusion (unchanged).}
Since $\ker\psi=5\Bbb Z$, the First Isomorphism Theorem gives
\[
\Bbb Z[i]/I\;\cong\;\Bbb Z/\ker\psi
          \;=\;\Bbb Z/5\Bbb Z
          \;=\;\Bbb Z_{5}.
\]

\medskip
\emph{Thus the only missing norm in the original typed notes was $2$;
including it does not affect the proof, only completes the list.}
When an \emph{integer} \(n\in\Bbb Z\subset\Bbb Z[i]\) lies in the ideal
\(I=(2+i)\), we can write it as a product
\[
n \;=\;(a+bi)(2+i), \qquad a,b\in\Bbb Z.
\]

Multiply out the right–hand side and separate real and imaginary parts:

\[
(a+bi)(2+i)
\;=\;\underbrace{(2a-b)}_{\text{real part}}
\;+\;
\underbrace{(a+2b)}_{\text{imaginary part}}\,i.
\]

Because \(n\) is a *real* integer, its imaginary part must be zero.  
Hence
\[
a+2b \;=\;0 
\quad\Longrightarrow\quad
a \;=\;-\,2b.
\]

(The equality \(a=-2b\) is nothing more than the condition that the
imaginary component vanishes.)
Substituting \(a=-2b\) back into the real part \(2a-b\) gives
\[
n \;=\;2a-b \;=\;2(-2b)-b \;=\;-5b,
\]
confirming that \(n\) is an integer multiple of \(5\).
% Detailed justification that \ker\psi\;=\;(5)

Recall the homomorphism
\[
\psi:\;\mathbb Z \longrightarrow \mathbb Z[i]/I, 
\qquad n \longmapsto n+I,
\qquad I=(2+i)\mathbb Z[i].
\]

\bigskip
\textbf{1.  Show that \(\ker\psi \subseteq (5)\).}

Assume \(n\in\ker\psi\).
Then \(n\in I\); i.e.\ there exist \(a,b\in\mathbb Z\) with
\[
n=(a+bi)(2+i).
\]
Expanding and separating real/imaginary parts gives
\[
n=(2a-b) +(a+2b)i.
\]
Because \(n\) is an \emph{integer}, its imaginary part must vanish, so
\(a+2b=0\) and hence \(a=-2b\).
Substituting this into the real part:
\[
n \;=\; 2a-b \;=\; 2(-2b)-b \;=\; -5b.
\]
Thus \(5\mid n\), i.e.\ \(n\in(5)=5\mathbb Z\).  
Hence \(\boxed{\ker\psi\;\subseteq\;(5)}\).

\bigskip
\textbf{2.  Show that \((5) \subseteq \ker\psi\).}

Observe that
\[
5 \;=\; (2+i)(2-i) \;\in\; I
\quad\Longrightarrow\quad
5+I=0+I\in\mathbb Z[i]/I,
\]
so \(5\in\ker\psi\).
Because \(\ker\psi\) is an ideal of \(\mathbb Z\) (the kernel of any ring
homomorphism is an ideal) and \(5\in\ker\psi\),
it follows that \emph{every} multiple of \(5\) lies in the kernel:
\[
(5)=5\mathbb Z \;\subseteq\; \ker\psi.
\]
Hence \(\boxed{(5)\;\subseteq\;\ker\psi}\).

\bigskip
\textbf{3.  Conclude equality.}

Combining the two inclusions we have
\[
(5)\;\subseteq\;\ker\psi\;\subseteq\;(5)
\quad\Longrightarrow\quad
\boxed{\;\ker\psi = (5)=5\mathbb Z\;}.
\]

This completes the expanded justification of the line  
“\(\ker\psi\subseteq(5)\) and, since \(5=(2+i)(2-i)\in I\), we obtain 
\((5)\subseteq\ker\psi\).”
% Why $5\in I$ implies $5+I = 0+I$ in the quotient ring $\Bbb Z[i]/I$

\begin{enumerate}
  \item \textbf{Recall the ideal.}  
        The notation
        \[
             I \;=\; (\,2+i\,)
        \]
        means
        \[
             I \;=\;\bigl\{\, (2+i)\,w \;\bigm|\; w\in\Bbb Z[i] \bigr\},
        \]
        i.e.\ all Gaussian integers that are \emph{multiples} of the
        generator $2+i$.

  \item \textbf{Show that $5\in I$.}  
        Factor $5$ in $\Bbb Z[i]$:
        \[
             (2+i)(2-i) \;=\; 4+1 \;=\; 5.
        \]
        Since $2-i\in\Bbb Z[i]$, we have  
        $5=(2+i)(2-i)\in I$ by definition of $I$.

  \item \textbf{Translate membership in $I$ to the quotient ring.}  
        In \emph{any} quotient ring $R/J$, two elements
        $a,b\in R$ represent the \emph{same} coset iff their difference
        lies in $J$:
        \[
            a-b\in J \;\Longrightarrow\; a+J \;=\; b+J.
        \]
        Here $R=\Bbb Z[i]$ and $J=I$.
        Because $5\in I$, the difference $5-0$ lies in $I$, so
        \[
            5+I \;=\; 0+I \quad\text{in }\Bbb Z[i]/I.
        \]

  \item \textbf{Interpretation.}  
        The coset $0+I$ is the \emph{zero element} of the quotient
        ring.  Thus $5$ maps to $0$ in $\Bbb Z[i]/I$, which is exactly
        the statement that $5$ lies in the kernel of the homomorphism
        \(\psi:\Bbb Z\to\Bbb Z[i]/I,\; n\mapsto n+I\).
\end{enumerate}

\[
\boxed{\;
   5=(2+i)(2-i)\in I
   \;\Longrightarrow\;
   5+I = 0+I \text{ in } \Bbb Z[i]/I
\;}
\]
\begin{problem}
  Show that
  \begin{enumerate}
    \item[\textnormal{(a)}] $\Bbb Z_{2}[x]/(x^{2}+1)$ is \emph{not} a field;
    \item[\textnormal{(b)}] $\Bbb Z_{2}[x]/(x^{2}+x+1)$ is a finite field of order $4$;
    \item[\textnormal{(c)}] $\Bbb Z_{2}[x]/(x^{3}+x+1)$ is a finite field of order $8$.
  \end{enumerate}
\end{problem}

\begin{solution}
Throughout let $F=\Bbb Z_{2}=\{0,1\}$ and write $I\bigl(f(x)\bigr)$ for the
principal ideal generated by $f(x)\in F[x]$.

\bigskip
\textbf{(a) $\boldsymbol{F[x]/(x^{2}+1)}$ is not a field.}

\begin{enumerate}
  \item[(i)]  In characteristic~$2$ one has
              \[
                  (x+1)^{2}=x^{2}+2x+1=x^{2}+1,
              \]
              because $2x=0$ in $F[x]$.  Hence $x^{2}+1=(x+1)^{2}$ is
              \emph{reducible} in $F[x]$.

  \item[(ii)] Because $(x^{2}+1)$ is not generated by an irreducible
              polynomial, the ideal $I(x^{2}+1)$ is not maximal.
              Therefore the quotient ring
              $F[x]/I(x^{2}+1)$ cannot be a field.

  \item[(iii)]  Concretely, set
              \[
                 \alpha = x+1 + I(x^{2}+1)\;\in\;F[x]/I(x^{2}+1).
              \]
              Then
              $\alpha\neq0$ but
              \[
                 \alpha^{2}
                 =(x+1)^{2}+I(x^{2}+1)
                 =(x^{2}+1)+I(x^{2}+1)
                 =0+I(x^{2}+1),
              \]
              so $\alpha$ is a \emph{non-zero nilpotent}, and the quotient
              ring has zero divisors—another immediate proof that it is
              not a field.
\end{enumerate}

\bigskip
\textbf{(b) $\boldsymbol{F[x]/(x^{2}+x+1)}$ is a field of order $4$.}

\begin{enumerate}
  \item[(i)]  Check irreducibility of $f(x)=x^{2}+x+1$ over $F$.
              The only possible linear factors are $x$ and $x+1$, so it
              suffices to evaluate $f$ at $0$ and $1$:
              \[
                 f(0)=1\neq0, 
                 \qquad
                 f(1)=1+1+1=1\neq0
                 \quad(\text{in }F).
              \]
              Thus $f(x)$ has no root in $F$, hence is irreducible
              (all degree-$2$ polynomials over a field either factor
              linearly or are irreducible).

  \item[(ii)]  Because $f(x)$ is irreducible, the ideal $I(f)$ is
              \emph{maximal}, so the quotient
              $F[x]/I(f)$ is a field.

  \item[(iii)]  As a vector space over $F$ the quotient is spanned by
              $\{1,\bar{x}\}$, where $\bar{x}=x+I(f)$.
              Therefore its dimension is $2$ and its cardinality is
              $|F|^{2}=2^{2}=4$.
\end{enumerate}

\bigskip
\textbf{(c) $\boldsymbol{F[x]/(x^{3}+x+1)}$ is a field of order $8$.}

\begin{enumerate}
  \item[(i)]  Let $g(x)=x^{3}+x+1$.  
              \emph{Irreducibility test.}
              A degree-$3$ polynomial over a field is irreducible
              $\Longleftrightarrow$ it has no root in the field.
              Again check the two elements of $F$:
              \[
                 g(0)=1\neq0,\qquad g(1)=1+1+1=1\neq0.
              \]
              Hence $g(x)$ has no linear factor and is irreducible.

  \item[(ii)]  The ideal $I\bigl(g(x)\bigr)$ is maximal, so
              $F[x]/I(g)$ is a field.

  \item[(iii)]  The cosets $\{1,\bar{x},\bar{x}^{2}\}$ form an $F$-basis,
              so the field has dimension~$3$ over $F$ and therefore
              $2^{3}=8$ elements.
\end{enumerate}

\bigskip
\textbf{Conclusion.}  Part~(a) fails because the modulus is reducible,
producing zero divisors, while parts~(b) and~(c) succeed because the
respective moduli are irreducible, giving fields whose orders are
$2^{\deg f}$.
\end{solution}
% Why $x^{2}+1$ factors over $\Bbb Z_{2}[x]$

Let $F=\Bbb Z_{2}$, the field with two elements $\{0,1\}$.
In $F$ we have $1+1=0$, hence \emph{characteristic} $2$.

\bigskip
\textbf{Step 1.  Expand $(x+1)^{2}$ in characteristic $2$.}

\[
(x+1)^{2}=x^{2}+2x+1.
\]

Because $2=1+1=0$ in $F$, the middle term vanishes:

\[
x^{2}+2x+1 \;=\; x^{2}+0\cdot x+1 \;=\; x^{2}+1
\quad\text{in }F[x].
\]

\bigskip
\textbf{Step 2.  Observe the factorisation.}

Thus
\[
x^{2}+1 \;=\; (x+1)^{2}
           \;=\; (x+1)(x+1)
           \quad\text{in }F[x].
\]

\bigskip
\textbf{Step 3.  Apply the definition of reducible.}

A non-constant polynomial $f(x)\in F[x]$ is called
\emph{reducible} if it can be written as a product
$f(x)=g(x)h(x)$ with $\deg g,\deg h\ge1$.
Since $x^{2}+1$ has been expressed as the product of the
degree-$1$ polynomial $x+1$ with itself, it meets that criterion.

\[
\boxed{\;x^{2}+1\text{ is reducible in }F[x]\;}
\]

(Over fields of characteristic $\neq2$ the cross-term $2x$ would
survive, so $x^{2}+1$ would \emph{not} factor as a square in the same
way.)
% Why every element of \(\Bbb Z_{2}[x]/(f)\) can be written as \(ax+b+(f)\)
% and why this shows the quotient has \(4\) elements
\[
f(x)=x^{2}+x+1\in\Bbb Z_{2}[x], 
\qquad (f):=(\,f(x)\,)\subset\Bbb Z_{2}[x].
\]

\bigskip
\textbf{1.  Division algorithm in \(\Bbb Z_{2}[x]\).}

Because \(f\) has degree \(2\), the polynomial–division algorithm over the
field \(\Bbb Z_{2}\) says that for \emph{every} \(g(x)\in\Bbb Z_{2}[x]\)
there exist unique polynomials \(q(x),r(x)\in\Bbb Z_{2}[x]\) such that
\[
g(x)=q(x)f(x)+r(x),
\qquad
\deg r<\deg f=2.
\]
Thus \(r(x)\) is either \(0\), a linear polynomial, or a constant.

\bigskip
\textbf{2.  Representatives of cosets.}

In the quotient ring
\(\Bbb Z_{2}[x]/(f)\) the polynomials \(g(x)\) and \(r(x)\)
represent the \emph{same} coset because their difference \(q(x)f(x)\)
lies in the ideal \((f)\):
\[
g(x)+(f)=r(x)+(f).
\]
Hence \emph{every} coset has a representative of the form
\[
r(x)=ax+b,
\qquad a,b\in\Bbb Z_{2}.
\]

\bigskip
\textbf{3.  How many such representatives exist?}

The coefficient field \(\Bbb Z_{2}\) has exactly two elements,
\(0\) and \(1\).  
Therefore the ordered pair \((a,b)\) can take \(2\times2=4\) distinct
values:
\[
(0,0),\;(1,0),\;(0,1),\;(1,1).
\]
Each pair gives a distinct polynomial \(ax+b\) and hence a distinct coset
because the representation \(r(x)=ax+b\) is \emph{unique}.  

\bigskip
\textbf{4.  Conclusion.}

Thus the entire quotient ring is
\[
\Bbb Z_{2}[x]/(f)
=\bigl\{\,0+(f),\; 1+(f),\; x+(f),\; x+1+(f)\bigr\},
\]
and it contains exactly \(4\) elements:

\[
\boxed{\; \bigl|\Bbb Z_{2}[x]/(x^{2}+x+1)\bigr| \;=\;4\; }.
\]

(The same reasoning works for any degree-\(n\) irreducible polynomial:
the quotient will have \(2^{n}\) elements over \(\Bbb Z_{2}\).)
% Why a subfield \(K\subseteq\F_{p^{\,n}}\) must have order \(p^{m}\) with \(m\mid n\)

Let \(\F_{p^{\,n}}\) be a finite field of characteristic \(p\) (so \(p\) is prime)  
and suppose \(K\subseteq\F_{p^{\,n}}\) is a subfield.  
We want to justify the statement

\[
\lvert K\rvert \;=\; p^{m}\quad\text{for some integer }m\text{ dividing }n,
\]
because
\[
[\F_{p^{\,n}}:K] \;=\; \frac{n}{m}
\]
is a field–extension degree and therefore \emph{must} be an integer.

\bigskip
\textbf{Step 1.  Both fields contain the prime field \(\F_{p}\).}

Every finite field of characteristic \(p\) contains a copy of the prime
field \(\F_{p}\).  Thus
\[
\F_{p}\;\subseteq\;K\;\subseteq\;\F_{p^{\,n}}.
\]

\bigskip
\textbf{Step 2.  View \(\F_{p^{\,n}}\) as a vector space over \(K\).}

Because \(K\) is a subfield, \(\F_{p^{\,n}}\) is automatically a
\emph{vector space} over \(K\).
Write its dimension as
\[
[\F_{p^{\,n}}:K] \;=\; d \;\in\;\Bbb N.
\]

\bigskip
\textbf{Step 3.  Relate cardinalities via vector–space dimension.}

For finite fields the size of a vector space is the base–field size
raised to the dimension:

\[
\lvert\F_{p^{\,n}}\rvert 
    \;=\;\lvert K\rvert^{\;d}.
\]

But \(\lvert\F_{p^{\,n}}\rvert = p^{n}\).  Put \(\lvert K\rvert=p^{m}\)
for some \(m\).  Then

\[
p^{n} \;=\; \bigl(p^{m}\bigr)^{d}
          \;=\; p^{\,md}.
\]

\bigskip
\textbf{Step 4.  Compare exponents.}

Equality of prime powers forces equality of exponents:

\[
n = m d,
\qquad\text{i.e.}\qquad
d = \frac{n}{m}.
\]

Because \(d=[\F_{p^{\,n}}:K]\) is a \emph{dimension}, it is an integer,
so \(n/m\in\Bbb N\).  
Equivalently, \(m\mid n\).

\bigskip
\textbf{Conclusion.}

\[
\boxed{\;
       K\text{ finite }\Longrightarrow\lvert K\rvert=p^{m}
       \text{ with }m\mid n
      \;}
\]
and
\[
\boxed{\;
       [\F_{p^{\,n}}:K] = \frac{n}{m}\in\Bbb N
      \;}
\]

This is precisely the assertion quoted:  
\emph{the degree of the field extension must be an integer, so \(m\) divides \(n\).}
% Concrete illustration of the “\(m\mid n\)” rule for subfields of \(\F_{p^{\,n}}\)

\section*{Example:  A field of order \(3^{4}=81\)}

\subsection*{1.  Construct the field \(\F_{81}\)}

Over the prime field \(\F_{3}=\{0,1,2\}\) consider the polynomial
\[
f(t)=t^{4}+t+2\in\F_{3}[t].
\]
\begin{itemize}
    \item  Evaluate at \(0,1,2\):
           \(
                f(0)=2,\;
                f(1)=1,\;
                f(2)=2\pmod3;
           \)
           none of these are \(0\), so \(f\) has no linear factor.
    \item  Therefore \(f(t)\) is irreducible (degree \(4\) with no root).
\end{itemize}

Define
\[
\F_{81}\;:=\;\F_{3}[t]/(f)\quad\text{and}\quad\alpha:=t+(f)\in\F_{81}.
\]
Then \(\deg f=4\) gives \(|\F_{81}|=3^{4}=81\), and every element of
\(\F_{81}\) is uniquely expressible as
\[
c_{0}+c_{1}\alpha+c_{2}\alpha^{2}+c_{3}\alpha^{3},
\qquad c_{i}\in\F_{3}.
\]

\subsection*{2.  The subfield of order \(3^{2}=9\)}

The Frobenius map \(\varphi:x\mapsto x^{3}\) is an automorphism of
\(\F_{81}\) whose fixed field is
\[
\F_{9}\;=\;\bigl\{\,x\in\F_{81}\;:\;x^{9}=x\,\bigr\}.
\]
Because \(9=3^{2}\) divides \(81=3^{4}\), this subfield \emph{must}
exist by the theory, and indeed
\[
[\F_{81}:\F_{9}] \;=\;\frac{4}{2}=2.
\]

A concrete generator:  set
\[
\gamma:=\alpha^{10}\in\F_{81}.
\]
One checks (using \(\alpha^{4}= -\alpha-2\) from \(f(\alpha)=0\)) that
\(\gamma^{9}=\gamma\) and \(\gamma\notin\F_{3}\); hence
\(\F_{9}=\F_{3}(\gamma)\).

\subsection*{3.  Why no subfield of order \(3^{3}=27\)}

If a subfield \(K\subseteq\F_{81}\) had size \(|K|=27=3^{3}\) it would
correspond to \(m=3\) in the formula
\[
|K|=3^{m},\qquad m\mid 4.
\]
But \(3\nmid 4\), so \emph{no integer degree}
\(
[\F_{81}:K]=\dfrac{4}{3}
\)
exists.  
Hence such a subfield cannot be embedded in \(\F_{81}\).

\subsection*{4.  Summary of the example}

\[
\boxed{
   \F_{81}
   \supset
   \F_{9}\,(\text{order }3^{2})
   \supset
   \F_{3}\,(\text{order }3^{1}),
   \quad
   \text{no subfield of order }3^{3}=27.
}
\]

This concretely realises the general fact:

\begin{center}
If \(K\subseteq\F_{p^{\,n}}\), then \(|K|=p^{m}\) with \(m\mid n\)
because \( [\F_{p^{\,n}}:K]=n/m \) must be an integer.
\end{center}
\end{document}
