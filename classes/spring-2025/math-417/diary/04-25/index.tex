\documentclass[12pt]{article}

% Packages
\usepackage[margin=1in]{geometry}
\usepackage{amsmath,amssymb,amsthm}
\usepackage{enumitem}
\usepackage{hyperref}
\usepackage{xcolor}
\usepackage{import}
\usepackage{xifthen}
\usepackage{pdfpages}
\usepackage{transparent}
\usepackage{listings}
\usepackage{tikz}
\usepackage{physics}
\usepackage{siunitx}
  \usetikzlibrary{calc,patterns,arrows.meta,decorations.markings}


\DeclareMathOperator{\Log}{Log}
\DeclareMathOperator{\Arg}{Arg}

\lstset{
    breaklines=true,         % Enable line wrapping
    breakatwhitespace=false, % Wrap lines even if there's no whitespace
    basicstyle=\ttfamily,    % Use monospaced font
    frame=single,            % Add a frame around the code
    columns=fullflexible,    % Better handling of variable-width fonts
}

\newcommand{\incfig}[1]{%
    \def\svgwidth{\columnwidth}
    \import{./Figures/}{#1.pdf_tex}
}
\theoremstyle{definition} % This style uses normal (non-italicized) text
\newtheorem{solution}{Solution}
\newtheorem{proposition}{Proposition}
\newtheorem{problem}{Problem}
\newtheorem{lemma}{Lemma}
\newtheorem{theorem}{Theorem}
\newtheorem{remark}{Remark}
\newtheorem{note}{Note}
\newtheorem{definition}{Definition}
\newtheorem{example}{Example}
\newtheorem{corollary}{Corollary}
\theoremstyle{plain} % Restore the default style for other theorem environments
%

% Theorem-like environments
% Title information
\title{}
\author{Jerich Lee}
\date{\today}

\begin{document}

\maketitle
\begin{definition}[Principal Ideal]
  Let \(R\) be a ring (not necessarily commutative and not necessarily with identity).
  \begin{enumerate}
      \item For any element \(a\in R\), the \emph{left ideal generated by \(a\)} is
      \[
          (a)_\ell \;=\; R\,a \;=\; \{\,r a \mid r\in R \,\}.
      \]
      \item The \emph{right ideal generated by \(a\)} is
      \[
          (a)_r \;=\; a\,R \;=\; \{\,a r \mid r\in R \,\}.
      \]
      \item If \(R\) is \emph{commutative} (or if we specify both left‐ and right‐multiples), we simply write
      \[
          (a) \;=\; \langle a\rangle \;=\; R a R \;=\; \{\,r_1 a r_2 \mid r_1,r_2\in R \,\}.
      \]
      In a commutative ring this simplifies to \( (a)=aR = Ra=\{\,a r \mid r\in R \,\}. \)
  \end{enumerate}
  Any ideal of the form \((a)_\ell\), \((a)_r\), or \((a)\) for some single element \(a\in R\) is called a \emph{principal ideal}, and the element \(a\) is said to \emph{generate} the ideal.
  
  \smallskip
  \textbf{Notation.}  Common notations for the principal ideal generated by \(a\) include
  \[
      (a),\quad \langle a\rangle,\quad aR,\quad Ra.
  \]
  
  \smallskip
  \textbf{Examples.}
  \begin{enumerate}
      \item In the ring of integers \(\mathbb{Z}\), every ideal is principal: for \(n\in\mathbb{Z}\),
            \[
                (n)=n\mathbb{Z}=\{\;kn : k\in\mathbb{Z}\;\}.
            \]
      \item In the polynomial ring \(k[x]\) over a field \(k\), the ideal generated by a polynomial \(f(x)\) is
            \[
                (f)=f\,k[x]=\{\,f(x)g(x)\mid g(x)\in k[x]\,\}.
            \]
      \item In the matrix ring \(M_n(k)\) for \(n\ge 2\), the left ideal generated by a single matrix \(A\) need not equal the right ideal generated by \(A\); hence one distinguishes \((A)_\ell\) and \((A)_r\).
  \end{enumerate}
  \end{definition}
  \begin{center}
    \Large\textbf{Examples of Ideals That \emph{Are Not} Principal}
    \end{center}
    
    Throughout, an ideal is called \emph{principal} if it can be generated by a single element of the ring.  
    Each example below is accompanied by a short proof of non-principalness.
    
    \bigskip
    %%%%%%%%%%%%%%%%%%%%%%%%%%%%%%%%%%%%%%%%%%%%%%%%%%%%%%%%%%%%%%%%%%%%%
    \begin{example}
    Let $k$ be any field and consider the polynomial ring in two variables
    \[
    R \;=\; k[x,y],\qquad 
    I \;=\; (x,y)\;=\; \{\,f(x,y)\in k[x,y] \mid f(0,0)=0\,\}.
    \]
    \emph{Claim:} $I$ is not principal.
    \emph{Proof.}
    Assume $I=(f)$ for some $f\in k[x,y]$.
    Since $x\in I$, we must have $x=f\cdot g_1$ for some $g_1\in R$, so $f$ divides $x$.
    Likewise $y=f\cdot g_2$, hence $f$ divides $y$.
    The only common divisors of $x$ and $y$ (up to units) are the units of $k$, so $f$ must be a unit.
    But then $(f)=R\neq I$, a contradiction.  \qed
    \end{example}
    
    \bigskip
    %%%%%%%%%%%%%%%%%%%%%%%%%%%%%%%%%%%%%%%%%%%%%%%%%%%%%%%%%%%%%%%%%%%%%
    \begin{example}
    Let
    \[
    R \;=\; \mathbb Z[x],\qquad
    J \;=\; (2,x)\;=\;\{\,2a(x)+x b(x)\mid a,b\in\mathbb Z[x]\,\}.
    \]
    \emph{Claim:} $J$ is not principal.
    
    \emph{Proof.}
    Suppose $J=(f)$ for some $f\in\mathbb Z[x]$.
    Since $2\in J$ we have $f\mid 2$, so $f\in\{\pm1,\pm2\}$.
    If $f=\pm1$ then $(f)=R\neq J$.
    If $f=\pm2$ then $(f)$ consists only of even polynomials, yet $x\in J$ is not even at $x=1$.
    Hence no single $f$ generates $J$. \qed
    \end{example}
    
    \bigskip
    %%%%%%%%%%%%%%%%%%%%%%%%%%%%%%%%%%%%%%%%%%%%%%%%%%%%%%%%%%%%%%%%%%%%%
    \begin{example}[A non-principal ideal in a quadratic integer ring]
    Let
    \[
    R \;=\; \mathbb Z[\sqrt{-5}],\qquad
    K \;=\; (2,\;1+\sqrt{-5}).
    \]
    \emph{Claim:} $K$ is not principal.
    
    \emph{Sketch of proof (norm argument).}
    For $\alpha=a+b\sqrt{-5}\in R$ define the field norm $N(\alpha)=a^{2}+5b^{2}$.
    If $K=(\beta)$ were principal, $N(K)=|N(\beta)|$ would be generated by a single integer.
    Yet it is easy to check
    \[
    N(2)=4,\qquad N(1+\sqrt{-5})=6\quad\Longrightarrow\quad 
    (4) \subseteq N(K)\subseteq (2).
    \]
    Because $(4)\neq(2)$ in $\mathbb Z$, no such single $\beta$ can exist.
    (A full proof shows $N(K)=(2)$ while $N(\beta)$ would be a divisor of $2$.) \qed
    \end{example}
    
    \bigskip
    %%%%%%%%%%%%%%%%%%%%%%%%%%%%%%%%%%%%%%%%%%%%%%%%%%%%%%%%%%%%%%%%%%%%%
    \begin{example}[A non-principal \textit{left} ideal in a matrix ring]
    Let $n\ge 2$ and $R=M_n(k)$, the $n\times n$ matrices over a field $k$.
    Let
    \[
    L \;=\; \Bigl\{\,A\in R \,\Big|\, \text{the first column of $A$ is }0\,\Bigr\}.
    \]
    \emph{Claim:} $L$ cannot be generated by a single matrix as a left ideal.
    
    \emph{Proof.}
    If $L=R\cdot A$ for some fixed matrix $A$, every matrix whose first column is $0$
    would have to be expressible as $BA$ for some $B\in R$.
    However, $BA$ always has image contained in the column‐space of $A$, which is at most $n$-dimensional,
    yet $L$ contains matrices whose column-spaces span any chosen $(n-1)$-dimensional subspace of $k^{n}$.
    Hence one generator cannot suffice. \qed
    \end{example}
    
    \bigskip
    %%%%%%%%%%%%%%%%%%%%%%%%%%%%%%%%%%%%%%%%%%%%%%%%%%%%%%%%%%%%%%%%%%%%%
    \begin{example}[An ideal of continuous functions vanishing on a set]
    Let $X=[0,1]$ and $R=C(X,\mathbb R)$, the ring of real‐valued continuous functions on $X$.
    For the closed subset $E=\{0,\tfrac12\}$ define
    \[
    I_E \;=\; \{\,f\in R \mid f(0)=f(\tfrac12)=0\,\}.
    \]
    \emph{Claim:} $I_E$ is not principal.
    
    \emph{Proof.}
    Assume $I_E=(g)$ for some $g\in R$.
    Because $g$ must vanish at both $0$ and $\tfrac12$, the zero set $Z(g)$ contains $E$.
    If $g$ had any additional zero $x_0\not\in E$, every multiple of $g$ would vanish at $x_0$, so $I_E$ would force vanishing at $x_0$—contradicting the definition of $I_E$.
    Hence $Z(g)=E$.  
    But a continuous function whose zero set is exactly two isolated points cannot serve:
    for any $\varepsilon>0$ we can construct $f\in I_E$ with $|f(x)|<\varepsilon$ everywhere except near $x=0$, contradicting $f=g\cdot h$ unless $g=0$.  \qed
    \end{example}
    
    \bigskip
    \hrule
    \medskip
    \textbf{Takeaway.}  
    Whenever a ring fails to be a \emph{principal ideal domain} (PID), one can usually exhibit a small, concrete ideal—often generated by two “obvious” elements—that cannot be compressed into a single generator.
\end{document}
