\documentclass[12pt]{article}

% Packages
\usepackage[margin=1in]{geometry}
\usepackage{amsmath,amssymb,amsthm}
\usepackage{enumitem}
\usepackage{hyperref}
\usepackage{xcolor}
\usepackage{import}
\usepackage{xifthen}
\usepackage{pdfpages}
\usepackage{transparent}
\usepackage{listings}
\usepackage{tikz}
\usepackage{physics}
\usepackage{siunitx}
\usepackage{cancel}
  \usetikzlibrary{calc,patterns,arrows.meta,decorations.markings}


\DeclareMathOperator{\Log}{Log}
\DeclareMathOperator{\Arg}{Arg}

\lstset{
    breaklines=true,         % Enable line wrapping
    breakatwhitespace=false, % Wrap lines even if there's no whitespace
    basicstyle=\ttfamily,    % Use monospaced font
    frame=single,            % Add a frame around the code
    columns=fullflexible,    % Better handling of variable-width fonts
}

\newcommand{\incfig}[1]{%
    \def\svgwidth{\columnwidth}
    \import{./Figures/}{#1.pdf_tex}
}
\theoremstyle{definition} % This style uses normal (non-italicized) text
\newtheorem{solution}{Solution}
\newtheorem{proposition}{Proposition}
\newtheorem{problem}{Problem}
\newtheorem{lemma}{Lemma}
\newtheorem{theorem}{Theorem}
\newtheorem{remark}{Remark}
\newtheorem{note}{Note}
\newtheorem{definition}{Definition}
\newtheorem{example}{Example}
\newtheorem{corollary}{Corollary}
\theoremstyle{plain} % Restore the default style for other theorem environments
%

% Theorem-like environments
% Title information
\title{}
\author{Jerich Lee}
\date{\today}

\begin{document}

\maketitle
\begin{theorem}[Gauss--D’Alembert lemma, ``only–if’’ direction]%
  \label{thm:UFD_to_fraction_field}
  Let \(R\) be a unique–factorisation domain (UFD) and let
  \(F=\operatorname{Frac}(R)\) be its field of fractions.
  For any non–zero polynomial \(f\in R[X]\subseteq F[X]\) we have
  \[
     \text{if }f\text{ is irreducible in }R[X]
     \;\Longrightarrow\;
     f\text{ is irreducible in }F[X].
  \]
  \end{theorem}
  
  \begin{proof}
  We prove the contrapositive:
  \[
     \text{if }f\text{ is \emph{reducible} in }F[X]
     \;\Longrightarrow\;
     f\text{ is \emph{reducible} in }R[X].
  \]
  
  \medskip
  \textbf{1.  A reducible factorisation in \(F[X]\).}
  Assume \(f\in R[X]\) and that there exist \(g,h\in F[X]\) such that
  \[
     f = gh, \qquad 
     \deg g,\deg h > 0, \qquad
     g,h\notin F[X]^{\!*}=F^{\!*}.
  \]
  (The degree assumptions guarantee that \(g,h\) are \emph{not} units.)
  
  \medskip
  \textbf{2.  Clearing denominators.}
  Because \(g,h\) have coefficients in \(F\), we can choose a single
  non–zero element \(d\in R\) such that
  \(d\,g,\,d\,h\in R[X]\).
  Writing \(d\,g=:g_0\) and \(d\,h=:h_0\) we get
  \[
     d\,f = g_0h_0, \qquad g_0,h_0\in R[X].
  \]
  
  \medskip
  \textbf{3.  Extracting contents.}
  Recall that every polynomial \(p\in R[X]\) can be written uniquely (up to
  units in \(R\)) as \(p=\operatorname{cont}(p)\,p^{\mathrm{pr}}\) where
  
  * \(\operatorname{cont}(p)\in R\) is the \emph{content} (the gcd of its
    coefficients), and  
  * \(p^{\mathrm{pr}}\) is \emph{primitive} (its coefficients have gcd \(1\)).
  
  Write
  \[
     g_0 = c_g\,g_1,\qquad
     h_0 = c_h\,h_1,\qquad
     f   = \gamma\,f_1
  \]
  with \(g_1,h_1,f_1\in R[X]\) primitive and
  \(c_g,c_h,\gamma\in R\).
  Substituting into \(d\,f=g_0h_0\) yields
  \[
     d\,\gamma\,f_1 = c_g\,c_h\,g_1h_1.
  \]
  
  \medskip
  \textbf{4.  Comparing contents.}
  Both sides are polynomials in \(R[X]\).
  By uniqueness of the content–primitive decomposition,
  their \emph{contents} must coincide up to a unit in \(R\):
  \[
     d\,\gamma \;\sim\; c_g\,c_h
     \quad\Longrightarrow\quad
     \exists\,u\in R^{\!*}\text{ with }d\,\gamma = u\,c_g\,c_h.
  \]
  Absorbing the unit \(u\) into (say) \(c_g\) we may replace \(c_g\) by an
  associate and assume
  \[
     d\,\gamma = c_g\,c_h.
  \]
  
  \medskip
  \textbf{5.  Cancelling the (now equal) contents.}
  Divide both sides of the equality in Step 4 by \(d\,\gamma=c_g\,c_h\):
  \[
     f_1 = g_1h_1.
  \]
  
  \medskip
  \textbf{6.  Returning to \(f\).}
  Because \(f=\gamma f_1\) we obtain
  \[
     f \;=\; \gamma\,g_1h_1.
  \]
  Here \(\gamma\in R^{\!*}\cup(\text{non–units})\), but
  \emph{units in \(R[X]\) are precisely the units in \(R\)}:
  \(R[X]^{\!*}=R^{\!*}\).
  Hence:
  
  * \(\gamma\in R[X]^{\!*}\) (it is a constant), and  
  * \(g_1,h_1\) are primitive of positive degree, so
    \(g_1,h_1\notin R[X]^{\!*}\).
  
  Thus \(f\) is expressed as a product of two \emph{non–units} in \(R[X]\),
  i.e.\ \(f\) is reducible in \(R[X]\).
  
  \medskip
  \textbf{7.  Contrapositive completed.}
  We have shown
  \[
     f\text{ reducible in }F[X] \;\Longrightarrow\;
     f\text{ reducible in }R[X].
  \]
  Taking the contrapositive gives the stated theorem.
  \end{proof}
\end{document}
