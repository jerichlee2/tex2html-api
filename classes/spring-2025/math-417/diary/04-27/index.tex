\documentclass[12pt]{article}

% Packages
\usepackage[margin=1in]{geometry}
\usepackage{amsmath,amssymb,amsthm}
\usepackage{enumitem}
\usepackage{hyperref}
\usepackage{xcolor}
\usepackage{import}
\usepackage{xifthen}
\usepackage{pdfpages}
\usepackage{transparent}
\usepackage{listings}
\usepackage{tikz}
\usepackage{physics}
\usepackage{siunitx}
  \usetikzlibrary{calc,patterns,arrows.meta,decorations.markings}


\DeclareMathOperator{\Log}{Log}
\DeclareMathOperator{\Arg}{Arg}

\lstset{
    breaklines=true,         % Enable line wrapping
    breakatwhitespace=false, % Wrap lines even if there's no whitespace
    basicstyle=\ttfamily,    % Use monospaced font
    frame=single,            % Add a frame around the code
    columns=fullflexible,    % Better handling of variable-width fonts
}

\newcommand{\incfig}[1]{%
    \def\svgwidth{\columnwidth}
    \import{./Figures/}{#1.pdf_tex}
}
\theoremstyle{definition} % This style uses normal (non-italicized) text
\newtheorem{solution}{Solution}
\newtheorem{proposition}{Proposition}
\newtheorem{problem}{Problem}
\newtheorem{lemma}{Lemma}
\newtheorem{theorem}{Theorem}
\newtheorem{remark}{Remark}
\newtheorem{note}{Note}
\newtheorem{definition}{Definition}
\newtheorem{example}{Example}
\newtheorem{corollary}{Corollary}
\theoremstyle{plain} % Restore the default style for other theorem environments
%

% Theorem-like environments
% Title information
\title{MATH-417: HW 10}
\author{Jerich Lee}
\date{\today}

\begin{document}

\maketitle
\begin{solution}
  \textbf{Step 1. Choose a suitable ring.}  
  Take the quotient ring
  \[
  R \;=\; \mathbb{Z}_8 \;=\; \mathbb{Z}/8\mathbb{Z},
  \]
  which is a commutative ring with identity $\overline{1}$.
  
  \medskip
  \textbf{Step 2. Define the polynomial.}  
  Let
  \[
  f(x) \;=\; x^{2}-1\;\in R[x].
  \]
  
  \medskip
  \textbf{Step 3. Compute all squares in $\mathbb{Z}_8$.}  
  The elements of $\mathbb{Z}_8$ are
  \[
  \overline{0},\overline{1},\overline{2},\overline{3},\overline{4},\overline{5},\overline{6},\overline{7},
  \]
  and a quick calculation (working modulo $8$) gives
  \[
  \begin{aligned}
  \overline{0}^{\,2}&=\overline{0}, &\qquad
  \overline{1}^{\,2}&=\overline{1}, &\qquad
  \overline{2}^{\,2}&=\overline{4}, &\qquad
  \overline{3}^{\,2}&=\overline{9}=\overline{1},\\[4pt]
  \overline{4}^{\,2}&=\overline{16}=\overline{0}, &\qquad
  \overline{5}^{\,2}&=\overline{25}=\overline{1}, &\qquad
  \overline{6}^{\,2}&=\overline{36}=\overline{4}, &\qquad
  \overline{7}^{\,2}&=\overline{49}=\overline{1}.
  \end{aligned}
  \]
  
  \medskip
  \textbf{Step 4. Identify the roots of $f(x)$.}  
  Since
  \[
  f(\overline{1})=\overline{1}-\overline{1}=\overline{0},\quad
  f(\overline{3})=\overline{3}^{\,2}-\overline{1}=\overline{9}-\overline{1}=\overline{0},
  \]
  \[
  f(\overline{5})=\overline{5}^{\,2}-\overline{1}=\overline{25}-\overline{1}=\overline{0},\quad
  f(\overline{7})=\overline{7}^{\,2}-\overline{1}=\overline{49}-\overline{1}=\overline{0},
  \]
  the elements
  \[
  \overline{1},\;\overline{3},\;\overline{5},\;\overline{7}
  \]
  are distinct roots of $x^{2}-1$ in $R$.
  
  \medskip
  \textbf{Step 5. Count the roots.}  
  We have found \emph{four} roots, which is strictly greater than the degree $2$ of the polynomial.
  
  \medskip
  \textbf{Conclusion.}  
  The ring $R=\mathbb{Z}/8\mathbb{Z}$ supplies the desired example: in this ring the polynomial $x^{2}-1$ has more than two roots (indeed, four).
  
  \medskip
  \textbf{Remark (optional).}  
  Over a field $K$, the usual proof that a non-zero polynomial of degree $d$ has at most $d$ roots relies on the fact that a non-zero polynomial cannot vanish at more than $d$ points without forcing all its non-leading coefficients to be zero \emph{and} on the ability to divide by the leading coefficient (which is guaranteed to be a unit in a field).  When $K$ is merely a commutative ring, the leading coefficient may be a zero divisor or fail to be invertible, and the argument breaks down.
  \end{solution}
  \begin{solution}
    Let $R$ be an integral domain and put
    \[
      X \;=\; R \times \bigl(R\setminus\{0\}\bigr), 
      \qquad
      (p,q)\sim(p_1,q_1) \;\Longleftrightarrow\; pq_1=p_1q.
    \]
    
    Throughout, write the $\sim$-class of $(p,q)$ as the \emph{formal fraction}
    \[
      \frac{p}{q}\;:=\;[(p,q)]_{\sim},
      \qquad
      X/\!\sim \;=\;\bigl\{\tfrac{p}{q}\mid(p,q)\in X\bigr\}.
    \]
    
    %%%%%%%%%%%%%%%%%%%%%%%%%%%%%%%%%%%%%%%%%%%%%%%%%%
    \subsection*{1.  \,$\sim$ is an equivalence relation}
    \begin{enumerate}[label=\textbf{(\alph*)}]
      \item \emph{Reflexive:} $(p,q)\sim(p,q)$ because $pq=pq$.
      \item \emph{Symmetric:} If $pq_{1}=p_{1}q$, then $p_{1}q=pq_{1}$, so
            $(p_{1},q_{1})\sim(p,q)$.
      \item \emph{Transitive:}  Suppose $(p,q)\sim(p_{1},q_{1})$ and
            $(p_{1},q_{1})\sim(p_{2},q_{2})$.  Then
            \[
              pq_{1}=p_{1}q
              \quad\text{and}\quad
              p_{1}q_{2}=p_{2}q_{1}.
            \]
            Multiplying the first equation by $q_{2}$ and the second by $q$
            gives $pq_{1}q_{2}=p_{1}qq_{2}=p_{2}qq_{1}$.
            Because $q_{1}\neq0$ in the integral domain $R$, we may cancel
            $q_{1}$ to obtain $pq_{2}=p_{2}q$, i.e.\ $(p,q)\sim(p_{2},q_{2})$.
    \end{enumerate}
    
    %%%%%%%%%%%%%%%%%%%%%%%%%%%%%%%%%%%%%%%%%%%%%%%%%%
    \subsection*{2.  Addition and multiplication are well-defined}
    
    Define for $\tfrac{p_{1}}{q_{1}},\tfrac{p_{2}}{q_{2}}\in X/\!\sim$:
    \[
      \frac{p_{1}}{q_{1}}+\frac{p_{2}}{q_{2}}
      \;:=\;\frac{p_{1}q_{2}+p_{2}q_{1}}{q_{1}q_{2}},
      \qquad
      \frac{p_{1}}{q_{1}}\;\cdot\;\frac{p_{2}}{q_{2}}
      \;:=\;\frac{p_{1}p_{2}}{q_{1}q_{2}}.
    \]
    
    \begin{proof}[Well-definedness of $+$]
      Let $(p_{1},q_{1})\sim(p_{1}',q_{1}')$ and
      $(p_{2},q_{2})\sim(p_{2}',q_{2}')$.  Then
      $p_{1}q_{1}'=p_{1}'q_{1}$ and $p_{2}q_{2}'=p_{2}'q_{2}$.
      Compute
      \[
        (p_{1}q_{2}+p_{2}q_{1})q_{1}'q_{2}'
        \;=\;
        p_{1}q_{2}q_{1}'q_{2}' + p_{2}q_{1}q_{1}'q_{2}'
        \;=\;
        p_{1}'q_{1}q_{2}q_{2}' + p_{2}'q_{2}q_{1}q_{1}'
        \;=\;
        (p_{1}'q_{2}'+p_{2}'q_{1}')q_{1}q_{2}.
      \]
      Hence
      \[
          (p_{1}q_{2}+p_{2}q_{1},\,q_{1}q_{2})
          \;\sim\;
          (p_{1}'q_{2}'+p_{2}'q_{1}',\,q_{1}'q_{2}'),
      \]
      so the sum does not depend on the chosen representatives.  
      The proof for multiplication is similar (and simpler):
      $(p_{1}p_{2})q_{1}'q_{2}' = p_{1}'p_{2}'q_{1}q_{2}$.
    \end{proof}
    
    %%%%%%%%%%%%%%%%%%%%%%%%%%%%%%%%%%%%%%%%%%%%%%%%%%
    \subsection*{3.  $X/\!\sim$ is a field}
    
    We verify the field axioms using the above operations.
    
    \begin{enumerate}[label=\textbf{(\alph*)}]
      \item \emph{Commutativity and associativity of $+$ and $\cdot$.}  
            These follow from the corresponding properties in $R$ and the
            formulas for $+$ and $\cdot$.
      \item \emph{Additive and multiplicative identities.}
            \[
              0 \;:=\;\frac{0}{1}, 
              \qquad
              1 \;:=\;\frac{1}{1}.
            \]
            One checks directly that $\tfrac{p}{q}+0=\tfrac{p}{q}$ and
            $\tfrac{p}{q}\cdot1=\tfrac{p}{q}$.
      \item \emph{Additive inverses.}  
            For $\tfrac{p}{q}\in X/\!\sim$ set
            \[
                -\frac{p}{q}\;:=\;\frac{-p}{q}.
            \]
            Then
            $\tfrac{p}{q}+(-\tfrac{p}{q})=\tfrac{p-q+p}{q}=\tfrac{0}{1}=0$.
      \item \emph{Multiplicative inverses.}  
            If $\tfrac{p}{q}\neq 0$, then $p\neq0$.  
            Define
            \[
              \bigl(\tfrac{p}{q}\bigr)^{-1}\;:=\;\tfrac{q}{p}.
            \]
            Since $R$ is an integral domain (hence $p\neq0$ implies $p$ is not
            a zero divisor), $q/p$ is a legitimate class and
            $\tfrac{p}{q}\cdot\tfrac{q}{p}=\tfrac{pq}{qp}=\tfrac{1}{1}=1$.
      \item \emph{Distributivity.}  Using the explicit formulas:
            \[
              \frac{p_{1}}{q_{1}}\bigl(\frac{p_{2}}{q_{2}}+\frac{p_{3}}{q_{3}}\bigr)
              \;=\;
              \frac{p_{1}(p_{2}q_{3}+p_{3}q_{2})}{q_{1}q_{2}q_{3}}
              \;=\;
              \frac{p_{1}p_{2}}{q_{1}q_{2}}
              \;+\;
              \frac{p_{1}p_{3}}{q_{1}q_{3}},
            \]
            and similarly on the right.
    \end{enumerate}
    All field axioms are satisfied; therefore $X/\!\sim$ is a field, often
    called \emph{the field of fractions of $R$}.
    \end{solution}
    \begin{solution}
      Let $\Bbb Z_{2}=\{0,1\}$ be the field of two elements and recall the basic fact:
      
      \begin{quote}
      For a field $F$ and $f\in F[x]$, the quotient ring $F[x]/(f)$ is a \emph{field}  
      iff $f$ is \emph{irreducible} over $F$.
      \end{quote}
      
      We treat parts (a)–(c) one by one.
      
      %%%%%%%%%%%%%%%%%%%%%%%%%%%%%%%%%%%%%%%%%%%%%%%%%%%%%%%%%%%%%%%%%%%%%%%%
      \subsection*{(a)  $\Bbb Z_{2}[x]\bigl/\bigl(x^{2}+1\bigr)$ is \emph{not} a field}
      
      \begin{enumerate}[label=\textbf{Step \arabic*:}, leftmargin=*]
        \item \textbf{Check for linear factors.}  
              In $\Bbb Z_{2}$ we have only $x=0,1$.  
              \[
                f(0)=1\neq0,
                \qquad
                f(1)=1^{2}+1=0.
              \]
              Hence $x=1$ is a root, so $x+1$ divides $x^{2}+1$.
        \item \textbf{Factorisation.}  
              Over $\Bbb Z_{2}$ we actually have
              \[
                x^{2}+1=(x+1)^{2}.
              \]
              Thus $x^{2}+1$ is \emph{reducible}.
        \item \textbf{Conclusion.}  
              Because the modulus is reducible, the quotient ring contains
              zero–divisors and therefore cannot be a field.  
              For instance
              \[
                (x+1)^{2}=0\quad\text{in}\quad
                \Bbb Z_{2}[x]\bigl/\bigl(x^{2}+1\bigr).
              \]
      \end{enumerate}
      
      %%%%%%%%%%%%%%%%%%%%%%%%%%%%%%%%%%%%%%%%%%%%%%%%%%%%%%%%%%%%%%%%%%%%%%%%
      \subsection*{(b)  $\Bbb Z_{2}[x]\bigl/\bigl(x^{2}+x+1\bigr)$ is a finite field of order $4$}
      
      \begin{enumerate}[label=\textbf{Step \arabic*:}, leftmargin=*]
        \item \textbf{Root test.}  
              Evaluate $g(x)=x^{2}+x+1$ at $0$ and $1$:
              \[
                g(0)=1,\qquad g(1)=1+1+1=1\quad(\text{mod }2).
              \]
              No roots in $\Bbb Z_{2}$ $\Longrightarrow$ no linear factors.
        \item \textbf{Irreducibility.}  
              A monic quadratic over a field with no linear factor is irreducible.
        \item \textbf{Field property.}  
              Hence $\Bbb Z_{2}[x]/(g)$ \emph{is} a field.
        \item \textbf{Order count.}  
              The cosets are represented by
              \[
                \{\,0,\;1,\;x,\;x+1\,\},
              \]
              so the field has $2^{\deg g}=2^{2}=4$ elements.
      \end{enumerate}
      
      %%%%%%%%%%%%%%%%%%%%%%%%%%%%%%%%%%%%%%%%%%%%%%%%%%%%%%%%%%%%%%%%%%%%%%%%
      \subsection*{(c)  $\Bbb Z_{2}[x]\bigl/\bigl(x^{3}+x+1\bigr)$ is a finite field of order $8$}
      
      \begin{enumerate}[label=\textbf{Step \arabic*:}, leftmargin=*]
        \item \textbf{Root test again.}  
              For $h(x)=x^{3}+x+1$,
              \[
                h(0)=1,\qquad h(1)=1+1+1=1\quad(\text{mod }2).
              \]
              No roots $\Longrightarrow$ no linear factors.
        \item \textbf{Irreducibility.}  
              A cubic over a field is irreducible iff it has no root, so $h$ is
              irreducible.
        \item \textbf{Field property.}  
              Therefore $\Bbb Z_{2}[x]/(h)$ is a field.
        \item \textbf{Order count.}  
              The field has $2^{\deg h}=2^{3}=8$ elements.
      \end{enumerate}
      
      \paragraph{Summary.}
      \[
        \boxed{\;
          \begin{aligned}
            &\Bbb Z_{2}[x]/(x^{2}+1)\text{ is \emph{not} a field},\\[4pt]
            &\Bbb Z_{2}[x]/(x^{2}+x+1)\text{ is }\mathbb{F}_{4},\\[4pt]
            &\Bbb Z_{2}[x]/(x^{3}+x+1)\text{ is }\mathbb{F}_{8}.
          \end{aligned}
        \;}
      \]
      \end{solution}
      \begin{solution}
        \textbf{Goal.}  Count the primitive roots (i.e.\ generators) of the multiplicative group
        \[
          \Bbb Z_{101}^{\times}.
        \]
        
        \begin{enumerate}[label=\textbf{Step \arabic*:}, leftmargin=*]
          \item \textbf{Structure of \(\Bbb Z_{p}^{\times}\).}  
                For every prime \(p\), the group of units
                \(\Bbb Z_{p}^{\times}\) is cyclic of order \(p-1\).
        
          \item \textbf{Primitive roots \(\Longleftrightarrow\) generators.}  
                An element \(g\in\Bbb Z_{p}^{\times}\) is a \emph{primitive root}
                exactly when it generates the entire cyclic group—
                equivalently, when its order is \(p-1\).
        
          \item \textbf{Counting generators of a cyclic group.}  
                A cyclic group of order \(n\) has precisely
                \(\varphi(n)\) generators, where \(\varphi\) is Euler’s
                totient function.
        
          \item \textbf{Compute \(\varphi(100)\).}  
                Since \(101-1=100=2^{2}\cdot5^{2}\),
                \[
                  \varphi(100)
                  \;=\;100\!\bigl(1-\frac12\bigr)\!\bigl(1-\frac15\bigr)
                  \;=\;100\cdot\frac12\cdot\frac45
                  \;=\;40.
                \]
        
          \item \textbf{Conclusion.}  
                The field \(\Bbb Z_{101}\) has
                \[
                  \boxed{40}
                \]
                primitive roots.
        \end{enumerate}
        \end{solution}
        \begin{solution}
          \textbf{Claim.} A field of order \(81\) \emph{cannot} contain a subfield of order \(27\).
          
          \medskip
          \begin{enumerate}[label=\textbf{Step \arabic*:}, leftmargin=*]
          
            \item \textbf{Structure of finite fields.}  
                  Every finite field has order \(p^{n}\) for a unique prime \(p\) (its
                  characteristic) and a positive integer \(n\) (its degree over
                  \(\Bbb F_{p}\)).  Conversely, for each pair \((p,n)\) there is exactly
                  one field of order \(p^{n}\) up to isomorphism.
          
            \item \textbf{Possible subfield orders.}  
                  If \(K\) is a subfield of \(F_{p^{n}}\), then
                  \[
                    |K|=p^{m}\quad\text{for some }m\mid n,
                  \]
                  because \([F_{p^{n}}:K]=n/m\) is the degree of the field extension and
                  must be an integer.
          
            \item \textbf{Apply to the case \(|F|=81\).}  
                  Since \(81=3^{4}\), any field of order \(81\) is
                  \(\Bbb F_{3^{4}}\).  Its subfields therefore have orders
                  \[
                    3^{m}
                    \quad\text{with}\quad
                    m \mid 4 \;\;\Longrightarrow\;\; m\in\{1,2,4\}.
                  \]
                  The only possible subfield orders are \(3^{1}=3\), \(3^{2}=9\), and
                  \(3^{4}=81\) (the field itself).
          
            \item \textbf{Check \(|K|=27\).}  
                  The order \(27=3^{3}\) would correspond to \(m=3\), but
                  \(3\nmid4\).  Hence no subfield of order \(27\) can exist.
          
          \end{enumerate}
          
          \medskip
          \textbf{Conclusion.}  A field of order \(81\) has \emph{no} subfield of
          order \(27\).
          \end{solution}
          \begin{solution}
            Let
            \[
              K \;=\; \Bbb Q\!\bigl(\sqrt{3},\sqrt{2}\bigr),
              \qquad
              \Bbb Q \subseteq \Bbb Q\!\bigl(\sqrt{2}\bigr)\subseteq K.
            \]
            
            %%%%%%%%%%%%%%%%%%%%%%%%%%%%%%%%%%%%%%%%%%%%%%%%%%%%%%%%%%%%%%%%%%%%%%%
            \subsection*{Step 1.  The tower to be used}
            
            Write the tower of fields  
            \[
               \Bbb Q \;\subseteq\; \Bbb Q(\sqrt{2}) \;\subseteq\; K .
            \]
            By the Tower Theorem,
            \[
              [K:\Bbb Q] \;=\; [K:\Bbb Q(\sqrt{2})]\,[\Bbb Q(\sqrt{2}):\Bbb Q].
            \]
            
            %%%%%%%%%%%%%%%%%%%%%%%%%%%%%%%%%%%%%%%%%%%%%%%%%%%%%%%%%%%%%%%%%%%%%%%
            \subsection*{Step 2.  Compute \([\Bbb Q(\sqrt{2}):\Bbb Q]\)}
            
            Because \(x^{2}-2\) is irreducible over \(\Bbb Q\),  
            \[
              [\Bbb Q(\sqrt{2}):\Bbb Q]=2.
            \]
            
            %%%%%%%%%%%%%%%%%%%%%%%%%%%%%%%%%%%%%%%%%%%%%%%%%%%%%%%%%%%%%%%%%%%%%%%
            \subsection*{Step 3.  Show that \(\sqrt{3}\notin\Bbb Q(\sqrt{2})\)}
            
            Assume for contradiction that \(\sqrt{3}\in\Bbb Q(\sqrt{2})\).
            Then there exist \(a,b\in\Bbb Q\) such that
            \[
              \sqrt{3}=a+b\sqrt{2}.
            \]
            Squaring both sides gives
            \[
              3 \;=\; a^{2}+2b^{2}+2ab\sqrt{2}.
            \]
            Comparing the rational and the \(\sqrt{2}\)-parts yields
            \(2ab=0\).  
            
            \begin{itemize}
              \item If \(a=0\), then \(3=2b^{2}\), impossible in \(\Bbb Q\).
              \item If \(b=0\), then \(3=a^{2}\), again impossible in \(\Bbb Q\).
            \end{itemize}
            Thus no such \(a,b\) exist, so \(\sqrt{3}\notin\Bbb Q(\sqrt{2})\).
            
            %%%%%%%%%%%%%%%%%%%%%%%%%%%%%%%%%%%%%%%%%%%%%%%%%%%%%%%%%%%%%%%%%%%%%%%
            \subsection*{Step 4.  Degree of the top extension}
            
            Since \(\sqrt{3}\) is not in the intermediate field,
            its minimal polynomial over \(\Bbb Q(\sqrt{2})\) is \(x^{2}-3\),
            which is irreducible (it has no root in \(\Bbb Q(\sqrt{2})\)).
            Therefore
            \[
              [K:\Bbb Q(\sqrt{2})]\;=\;2.
            \]
            
            %%%%%%%%%%%%%%%%%%%%%%%%%%%%%%%%%%%%%%%%%%%%%%%%%%%%%%%%%%%%%%%%%%%%%%%
            \subsection*{Step 5.  Combine via the tower law}
            
            \[
              [K:\Bbb Q] \;=\; 2\times 2 \;=\; 4.
            \]
            
            \paragraph{Conclusion.}
            \[
              \boxed{\, [\,\Bbb Q(\sqrt{3},\sqrt{2}):\Bbb Q\,]=4 \, }.
            \]
            \end{solution}
\end{document}
