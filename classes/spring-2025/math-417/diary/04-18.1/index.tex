\documentclass[12pt]{article}

% Packages
\usepackage[margin=.5in]{geometry}
\usepackage{amsmath,amssymb,amsthm}
\usepackage{enumitem}
\usepackage{hyperref}
\usepackage{xcolor}
\usepackage{import}
\usepackage{xifthen}
\usepackage{pdfpages}
\usepackage{transparent}
\usepackage{listings}
\usepackage{tikz}
  \usetikzlibrary{calc,patterns,arrows.meta,decorations.markings}


\DeclareMathOperator{\Log}{Log}
\DeclareMathOperator{\Arg}{Arg}

\lstset{
    breaklines=true,         % Enable line wrapping
    breakatwhitespace=false, % Wrap lines even if there's no whitespace
    basicstyle=\ttfamily,    % Use monospaced font
    frame=single,            % Add a frame around the code
    columns=fullflexible,    % Better handling of variable-width fonts
}

\newcommand{\incfig}[1]{%
    \def\svgwidth{\columnwidth}
    \import{./Figures/}{#1.pdf_tex}
}
\theoremstyle{definition} % This style uses normal (non-italicized) text
\newtheorem{solution}{Solution}
\newtheorem{proposition}{Proposition}
\newtheorem{problem}{Problem}
\newtheorem{lemma}{Lemma}
\newtheorem{theorem}{Theorem}
\newtheorem{remark}{Remark}
\newtheorem{note}{Note}
\newtheorem{definition}{Definition}
\newtheorem{example}{Example}
\newtheorem{corollary}{Corollary}
\theoremstyle{plain} % Restore the default style for other theorem environments
%

% Theorem-like environments
% Title information
\title{MATH 417: HW 8}
\author{Jerich Lee}
\date{\today}

\begin{document}

\maketitle
\begin{problem}[]
  
\end{problem}

\begin{solution}
  \textbf{Goal.}  Determine all ring homomorphisms
  \[
  \psi\colon \mathbb Z \longrightarrow \mathbb Z .
  \]
  
  \medskip
  \textbf{Step 1.  Recall the definition of a ring homomorphism.}  
  A (not necessarily unital) ring homomorphism between rings $(R,+,\cdot)$ and $(S,+,\cdot)$ is a map
  $\varphi\colon R\to S$ satisfying
  \[
  \varphi(a+b)=\varphi(a)+\varphi(b),\qquad
  \varphi(ab)=\varphi(a)\varphi(b),\qquad
  \varphi(0_R)=0_S .
  \]
  If one further requires $\varphi(1_R)=1_S$, one speaks of a \emph{unital} ring homomorphism.
  
  \medskip
  \textbf{Step 2.  The image of $1$ must be idempotent.}  
  For any ring homomorphism $\psi$,
  \[
  \psi(1)^2 \;=\; \psi(1\cdot 1)\;=\;\psi(1).
  \]
  Hence $\psi(1)$ is an \emph{idempotent} element of the target ring $\mathbb Z$.
  
  \medskip
  \textbf{Step 3.  Classify idempotents in $\mathbb Z$.}  
  An integer $e\in\mathbb Z$ is idempotent if $e^2=e$, i.e.\ $e(e-1)=0$.  
  Because $\mathbb Z$ is an integral domain, this forces $e=0$ or $e=1$.  
  Thus $\psi(1)$ can only be $0$ or $1$.
  
  \medskip
  \textbf{Step 4.  Case $\psi(1)=0$: the zero homomorphism.}  
  For any $n\in\mathbb Z$ we can write
  \[
  n=\underbrace{1+\cdots+1}_{n\text{ times}} \quad\text{(if }n>0\text{)},
  \qquad
  -n=\underbrace{(-1)+\cdots+(-1)}_{n\text{ times}}.
  \]
  Using additivity,
  \[
  \psi(n)=\psi(\underbrace{1+\cdots+1}_{n})=n\,\psi(1)=n\cdot 0=0,
  \]
  and similarly $\psi(-n)=0$.  Hence $\psi(n)=0$ for \emph{all} $n\in\mathbb Z$, giving the
  \[
  \boxed{\psi_0(n)=0\ \text{ for every }n\in\mathbb Z.}
  \]
  
  \medskip
  \textbf{Step 5.  Case $\psi(1)=1$: the identity homomorphism.}  
  Proceeding as above,
  \[
  \psi(n)=n\,\psi(1)=n\cdot 1=n,
  \]
  and likewise for negative integers.  Thus
  \[
  \boxed{\psi_{\mathrm{id}}(n)=n\ \text{ for every }n\in\mathbb Z.}
  \]
  
  \medskip
  \textbf{Step 6.  Conclusion.}  
  There are exactly two ring homomorphisms $\mathbb Z\to\mathbb Z$, namely
  \[
  \psi_0 \quad\text{and}\quad \psi_{\mathrm{id}} .
  \]
  If one insists that a ring homomorphism preserve the multiplicative identity ($\psi(1)=1$), then \emph{only} the identity map $\psi_{\mathrm{id}}$ remains.
  
  \end{solution}
%----------------------------------------------------
%  Problem 2 — Units form a group
%----------------------------------------------------
\begin{problem}[]
  Let $R$ be a ring with identity $1_R$.
  Show that the set of \emph{units}
  \[
      R^{\times} \;=\; \{\,a\in R \mid \exists\,b\in R \text{ such that } ab = ba = 1_R\,\}
  \]
  is a group under the multiplication inherited from $R$.
  \begin{enumerate}
      \item[\textbf{Parameters.}]
            $R$: a ring with $1_R\neq 0_R$.
  \end{enumerate}
\end{problem}
\begin{solution}
 \begin{proof}
  We verify the group axioms step by step.

  \begin{enumerate}
      \item[\textbf{1. Non–emptiness (Identity).}]
            The identity element $1_R$ satisfies $1_R \cdot 1_R = 1_R$,
            hence $1_R\in R^{\times}$.
            Thus $R^{\times}\neq\varnothing$.

      \item[\textbf{2. Closure under multiplication.}]
            Let $a,b\in R^{\times}$.
            Then there exist $a^{-1},b^{-1}\in R$ with
            \begin{align}
                a\,a^{-1} = a^{-1}a = 1_R,
                \qquad
                b\,b^{-1} = b^{-1}b = 1_R.
            \end{align}
            Consider $ab\in R$.  Its (two–sided) inverse is $b^{-1}a^{-1}$ because
            \begin{align}
                (ab)(b^{-1}a^{-1}) = a(bb^{-1})a^{-1} = a\,1_R\,a^{-1} = aa^{-1} = 1_R, \\[2pt]
                (b^{-1}a^{-1})(ab) = b^{-1}(a^{-1}a)b = b^{-1}\,1_R\,b = b^{-1}b = 1_R.
            \end{align}
            Therefore $ab\in R^{\times}$ and $R^{\times}$ is closed under multiplication.

      \item[\textbf{3. Associativity.}]
            Multiplication in $R$ is associative, and $R^{\times}$ is a subset of $R$,
            so associativity holds automatically:
            \begin{align}
                (ab)c = a(bc)
                \quad\text{for all } a,b,c\in R^{\times}.
            \end{align}

      \item[\textbf{4. Existence of inverses.}]
            By definition, every $a\in R^{\times}$ comes equipped with a
            two–sided inverse $a^{-1}\in R$.
            We must check $a^{-1}\in R^{\times}$ as well.
            Indeed, $a^{-1}$ is invertible with inverse $a$:
            \begin{align}
                a^{-1}a = aa^{-1} = 1_R,
            \end{align}
            so $a^{-1}\in R^{\times}$.

      \item[\textbf{5. Identity element lies in $R^{\times}$.}]
            Already shown in Step 1, and
            \begin{align}
                1_R\cdot a = a\cdot 1_R = a
                \quad\text{for all } a\in R^{\times}.
            \end{align}
  \end{enumerate}
  Having satisfied closure, associativity, identity, and inverses,
  $R^{\times}$ is a group under multiplication.
\end{proof}
\ 
\end{solution}
\begin{problem}[]
  Show that every finite integral domain is a field.
  \begin{enumerate}
      \item[\textbf{Parameters.}]
            $R$: a finite integral domain, i.e.\ a finite, commutative ring
            with identity $1_R\neq 0_R$ and no zero--divisors.
  \end{enumerate}
\end{problem}
\begin{solution}
 %----------------------------------------------------
%  Problem 3 — A finite integral domain is a field
%----------------------------------------------------


\begin{proof}
  We must prove that every non‑zero element of $R$ is invertible.
  \begin{enumerate}
      \item[\textbf{1.  Fix a non‑zero element.}]
            Let $a\in R$ with $a\neq 0_R$.

      \item[\textbf{2.  Define a map by left multiplication.}]
            Consider the function
            \[
                \varphi_a : R \longrightarrow R,
                \qquad
                \varphi_a(x)=ax \quad (x\in R).
            \]

      \item[\textbf{3.  $\varphi_a$ is injective (one‑to‑one).}]
            Suppose $\varphi_a(x)=\varphi_a(y)$, i.e.\ $ax=ay$.
            Then $a(x-y)=0_R$.
            Since $R$ has no zero‑divisors and $a\neq 0_R$,
            we must have $x-y=0_R$, hence $x=y$.

      \item[\textbf{4.  Finite injective $\Longrightarrow$ bijective.}]
            Because $R$ is \emph{finite}, any injective map
            $R\to R$ is automatically surjective.
            Thus $\varphi_a$ is bijective.

      \item[\textbf{5.  Existence of an inverse element.}]
            Surjectivity means there exists $y\in R$ such that
            \[
                \varphi_a(y)=ay = 1_R .
            \]
            Likewise, because multiplication is commutative,
            \[
                ya = 1_R .
            \]
            Hence $y$ is a (two‑sided) multiplicative inverse of $a$.

      \item[\textbf{6.  All non‑zeros are units.}]
            The argument applies to every non‑zero $a\in R$,
            so every non‑zero element of $R$ is invertible.

      \item[\textbf{7.  Conclusion.}]
            A commutative ring in which every non‑zero element is a unit
            is, by definition, a \emph{field}.
            Therefore the finite integral domain $R$ is a field.
  \end{enumerate}
\end{proof} 
\end{solution}
\begin{problem}[]
  Let $R$ be a commutative ring.  
  An element $x\in R$ is called \emph{nilpotent} if $x^{k}=0$ for some integer $k\ge 1$.  
  Show that the set
  \[
      \mathcal{N}(R) \;=\; \{\,x\in R \mid \exists\,k\ge 1\text{ such that }x^{k}=0\,\}
  \]
  of all nilpotent elements of $R$ forms an ideal.
  \begin{enumerate}
      \item[\textbf{Parameters.}]
            $R$: a commutative ring with identity $1_R$ (not required for the argument).
  \end{enumerate}
\end{problem}
\begin{solution}
 %----------------------------------------------------
%  Problem 4 — Nilpotent elements form an ideal
%----------------------------------------------------


\begin{proof}
  We verify the two ideal conditions for $\mathcal{N}(R)$.

  \begin{enumerate}
      \item[\textbf{1.  Non--emptiness and additive subgroup.}]
            \begin{enumerate}
                \item[(a)] \emph{Contains $0_R$.}  
                          Since $0_R^{\,1}=0_R$, we have $0_R\in\mathcal{N}(R)$.
                \item[(b)] \emph{Closed under addition.}  
                          Let $x,y\in\mathcal{N}(R)$ with $x^{m}=0$ and $y^{n}=0$.  
                          Commutativity allows the binomial theorem:
                          \[
                              (x+y)^{\,m+n}
                              =\sum_{i=0}^{m+n}\binom{m+n}{i}\,x^{\,m+n-i}\,y^{\,i}.
                          \]
                          If $i<n$ then $m+n-i>m$ and $x^{\,m+n-i}=0$;   
                          if $i\ge n$ then $y^{\,i}=0$.  
                          Hence every term is $0$, so $(x+y)^{\,m+n}=0$ and $x+y\in\mathcal{N}(R)$.
                \item[(c)] \emph{Closed under additive inverses.}  
                          For $x^{m}=0$,
                          \[
                              (-x)^{m}=(-1)^{m}x^{m}=0,
                          \]
                          so $-x\in\mathcal{N}(R)$.
            \end{enumerate}
            Thus $\mathcal{N}(R)$ is an additive subgroup of $R$.

      \item[\textbf{2.  Absorption under multiplication.}]
            Let $r\in R$ and $x\in\mathcal{N}(R)$ with $x^{m}=0$.  
            Because $R$ is commutative,
            \[
                (rx)^{m} \;=\; r^{m}\,x^{m} \;=\; r^{m}\,0 \;=\; 0,
            \]
            so $rx\in\mathcal{N}(R)$.  
            The same calculation shows $xr\in\mathcal{N}(R)$ (identical in a commutative ring).

  \end{enumerate}
  Having shown that $\mathcal{N}(R)$ is an additive subgroup closed under multiplication by arbitrary elements of $R$, we conclude that $\mathcal{N}(R)$ is indeed an ideal of $R$.
\end{proof} 
\end{solution}
\end{document}
