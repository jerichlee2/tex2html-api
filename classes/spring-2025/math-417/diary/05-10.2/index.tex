\documentclass[12pt]{article}

% Packages
\usepackage[margin=1in]{geometry}
\usepackage{amsmath,amssymb,amsthm}
\usepackage{enumitem}
\usepackage{hyperref}
\usepackage{xcolor}
\usepackage{import}
\usepackage{xifthen}
\usepackage{pdfpages}
\usepackage{transparent}
\usepackage{listings}
\usepackage{tikz}
\usepackage{physics}
\usepackage{siunitx}
\usepackage{booktabs}
\usepackage{cancel}
  \usetikzlibrary{calc,patterns,arrows.meta,decorations.markings}


\DeclareMathOperator{\Log}{Log}
\DeclareMathOperator{\Arg}{Arg}

\lstset{
    breaklines=true,         % Enable line wrapping
    breakatwhitespace=false, % Wrap lines even if there's no whitespace
    basicstyle=\ttfamily,    % Use monospaced font
    frame=single,            % Add a frame around the code
    columns=fullflexible,    % Better handling of variable-width fonts
}

\newcommand{\incfig}[1]{%
    \def\svgwidth{\columnwidth}
    \import{./Figures/}{#1.pdf_tex}
}
\theoremstyle{definition} % This style uses normal (non-italicized) text
\newtheorem{solution}{Solution}
\newtheorem{proposition}{Proposition}
\newtheorem{problem}{Problem}
\newtheorem{lemma}{Lemma}
\newtheorem{theorem}{Theorem}
\newtheorem{remark}{Remark}
\newtheorem{note}{Note}
\newtheorem{definition}{Definition}
\newtheorem{example}{Example}
\newtheorem{corollary}{Corollary}
\theoremstyle{plain} % Restore the default style for other theorem environments
%

% Theorem-like environments
% Title information
\title{MATH 417 Practice Final Exam 2}
\author{Jerich Lee}
\date{\today}

\begin{document}

\maketitle
%------------------------------------------------------------
%  Practice Final Exam #2 – MATH 417 (Comprehensive)
%  13 questions • no calculators • justify all answers unless told otherwise
%------------------------------------------------------------
\newcommand{\Z}{\mathbb Z}
\newcommand{\Q}{\mathbb Q}
\newcommand{\R}{\mathbb R}

\bigskip
\begin{problem}
  Let
  \[
     U_{8}\;=\;\{\,1,3,5,7\}
     \subset\Z_{8}^{\times},
     \qquad
     a\ast b:=ab\pmod{8}.
  \]
  \begin{enumerate}
      \item[(a)] Complete the Cayley table of \((U_{8},\ast)\).
      \item[(b)] Identify the identity element.
      \item[(c)] Determine the inverse of each element.
  \end{enumerate}
\end{problem}

\bigskip
\begin{problem}
  Let \(p\) be prime and let \(|G|=p^{2}\).
  \begin{enumerate}
      \item[(a)] Prove that the center \(Z(G)\) is non-trivial.
      \item[(b)] Deduce that \(G\) is abelian, and hence
                \(G\cong\Z_{p^{2}}\) or \(G\cong\Z_{p}\oplus\Z_{p}\).
  \end{enumerate}
\end{problem}

\bigskip
\begin{problem}
  \begin{enumerate}
      \item[(a)] List \emph{all} subgroups of \(\Z_{24}\) and their indices.
      \item[(b)] Draw the subgroup lattice of \(\Z_{24}\).
  \end{enumerate}
\end{problem}

\bigskip
\begin{problem}
  Compute each quantity:
  \begin{enumerate}
      \item[(a)] \(7^{222}\pmod{13};\)
      \item[(b)] \(9^{1000}\pmod{31};\)
      \item[(c)] the last two digits of \(3^{123}\).
  \end{enumerate}
\end{problem}

\bigskip
\begin{problem}
  \begin{enumerate}
      \item[(a)] List all elements of \(\Z_{10}^{\times}\) and
                compute the multiplicative inverse of each.
      \item[(b)] Which elements satisfy \(a^{-1}=a\)?
  \end{enumerate}
\end{problem}

\bigskip
\begin{problem}
  Let \(\varphi:G\to H\) be a group homomorphism.
  Prove that \(\operatorname{ord}_{H}\!\bigl(\varphi(a)\bigr)\)
  divides \(\operatorname{ord}_{G}(a)\) for every \(a\in G\).
\end{problem}

\bigskip
\begin{problem}
  Show that the multiplicative group
  \[
      C_{n}
      \;=\;
      \bigl\{e^{2\pi i k/n}:k=0,1,\dots,n-1\bigr\}
      \subset\mathbb{C}^{\times}
  \]
  is cyclic and that \(C_{n}\cong\Z_{n}\).
\end{problem}

\bigskip
\begin{problem}
  Classify, up to isomorphism, all abelian groups of order \(72\).
\end{problem}

\bigskip
\begin{problem}
  Let \(p\) be an odd prime and let \(|G|=2p\).
  Prove that \(G\) is isomorphic either to the cyclic group
  \(\Z_{2p}\) or to the dihedral group \(D_{2p}\).
\end{problem}

\bigskip
\begin{problem}
  \begin{enumerate}
      \item[(a)]  Verify that
                  \(F=\Z_{2}[x]/(x^{3}+x+1)\) is a field of order \(8\).
      \item[(b)] Let \(\alpha=x+(x^{3}+x+1)\in F\).
                Determine the multiplicative order of \(\alpha\).
  \end{enumerate}
\end{problem}

\bigskip
\begin{problem}
  Compute a greatest common divisor of
  \(15+11i\) and \(4+9i\) in \(\Z[i]\),
  and express it as
  \(u(15+11i)+(4+9i)v\) with \(u,v\in\Z[i]\).
\end{problem}

\bigskip
\begin{problem}
  \begin{enumerate}
      \item[(a)] Prove that exactly half of the permutations in \(S_{n}\)
                are even; that is, \(|A_{n}|=n!/2\).
      \item[(b)] Decompose
                \(\sigma=(12)(13)(1456)\in S_{6}\)
                into disjoint cycles, find \(\operatorname{ord}(\sigma)\),
                and state whether \(\sigma\) is even or odd.
  \end{enumerate}
\end{problem}

\bigskip
\begin{problem}
  Let \(Q_{8}=\{\pm1,\pm i,\pm j,\pm k\}\).
  \begin{enumerate}
      \item[(a)] List all conjugacy classes of \(Q_{8}\).
      \item[(b)] Determine the center \(Z(Q_{8})\) and verify the
                class equation of \(Q_{8}\).
      \item[(c)] Compute \(\operatorname{Aut}(Q_{8})\) and its order.
  \end{enumerate}
\end{problem}
%------------------------------------------------------------
\end{document}
