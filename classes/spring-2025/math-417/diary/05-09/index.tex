\documentclass[12pt]{article}

% Packages
\usepackage[margin=1in]{geometry}
\usepackage{amsmath,amssymb,amsthm}
\usepackage{enumitem}
\usepackage{hyperref}
\usepackage{xcolor}
\usepackage{import}
\usepackage{xifthen}
\usepackage{pdfpages}
\usepackage{transparent}
\usepackage{listings}
\usepackage{tikz}
\usepackage{physics}
\usepackage{siunitx}
\usepackage{booktabs}
\usepackage{cancel}
  \usetikzlibrary{calc,patterns,arrows.meta,decorations.markings}


\DeclareMathOperator{\Log}{Log}
\DeclareMathOperator{\Arg}{Arg}
\DeclareMathOperator{\lcm}{lcm}
\DeclareMathOperator{\ord}{ord}

\lstset{
    breaklines=true,         % Enable line wrapping
    breakatwhitespace=false, % Wrap lines even if there's no whitespace
    basicstyle=\ttfamily,    % Use monospaced font
    frame=single,            % Add a frame around the code
    columns=fullflexible,    % Better handling of variable-width fonts
}

\newcommand{\incfig}[1]{%
    \def\svgwidth{\columnwidth}
    \import{./Figures/}{#1.pdf_tex}
}
\theoremstyle{definition} % This style uses normal (non-italicized) text
\newtheorem{solution}{Solution}
\newtheorem{proposition}{Proposition}
\newtheorem{problem}{Problem}
\newtheorem{lemma}{Lemma}
\newtheorem{theorem}{Theorem}
\newtheorem{remark}{Remark}
\newtheorem{note}{Note}
\newtheorem{definition}{Definition}
\newtheorem{example}{Example}
\newtheorem{corollary}{Corollary}
\newtheorem{explanation}{Explanation}

\theoremstyle{plain} % Restore the default style for other theorem environments
%

% Theorem-like environments
% Title information
\title{}
\author{Jerich Lee}
\date{\today}

\begin{document}

\maketitle
%-----------------------------------------------------------
%  Requires: amsmath, amssymb, booktabs, geometry (optional)
%-----------------------------------------------------------
\section{Euler’s Totient Function \texorpdfstring{$\varphi(n)$}{phi(n)}}

\subsection{Definition}
Euler’s totient function counts the positive integers not exceeding~$n$
that are coprime to~$n$:
\[
  \varphi(n)=\bigl|\{\,k\in\{1,2,\dots ,n\}\mid\gcd(k,n)=1\,\}\bigr|.
\]

\subsection{Basic Values}
\begin{table}[h]
  \centering
  \small
  \setlength\tabcolsep{10pt}
  \renewcommand{\arraystretch}{1.1}
  \begin{tabular}{ccc}
    \toprule
    $n$ & Numbers $\le n$ coprime to $n$              & $\varphi(n)$ \\ \midrule
    1   & 1                                           & 1 \\
    2   & 1                                           & 1 \\
    3   & 1,\,2                                       & 2 \\
    4   & 1,\,3                                       & 2 \\
    5   & 1,\,2,\,3,\,4                               & 4 \\
    6   & 1,\,5                                       & 2 \\
    8   & 1,\,3,\,5,\,7                               & 4 \\
    9   & 1,\,2,\,4,\,5,\,7,\,8                       & 6 \\
    \bottomrule
  \end{tabular}
  \caption{Illustrative values of $\varphi(n)$.}
\end{table}

\subsection{Prime‑Factor Formula}
If 
\(
  n=\displaystyle\prod_{i=1}^r p_i^{k_i},
\)
then
\[
  \boxed{\;
    \varphi(n)\;=\;
      n\prod_{i=1}^r\!\left(1-\frac1{p_i}\right)
      \;=\;
      \prod_{i=1}^r p_i^{\,k_i-1}(p_i-1)
  \;}
\]

\subsection{Example}
For \(n=60=2^{2}\!\cdot 3 \cdot 5\),
\[
  \varphi(60)=60
     \Bigl(1-\tfrac12\Bigr)
     \Bigl(1-\tfrac13\Bigr)
     \Bigl(1-\tfrac15\Bigr)
  =60\cdot\tfrac12\cdot\tfrac23\cdot\tfrac45
  =16.
\]

\subsection{Key Properties}
\begin{enumerate}[label=\arabic*.]
  \item \textbf{Multiplicativity.} If \(\gcd(m,n)=1\) then
        \(\varphi(mn)=\varphi(m)\varphi(n)\).
  \item \textbf{Prime and Prime Powers.}
        \(\varphi(p)=p-1\) for prime~\(p\);\;
        \(\varphi(p^{k})=p^{\,k-1}(p-1)\).
  \item \textbf{Divisor‑Sum Identity.}\;
        \(\displaystyle\sum_{d\mid n}\varphi(d)=n\).
  \item \textbf{Euler’s Theorem.} If \(\gcd(a,n)=1\) then
        \(a^{\varphi(n)}\equiv 1\pmod n\).
\end{enumerate}

\subsection{Remarks}
The factors \((1-1/p_i)\) show how each distinct prime divisor of~\(n\)
“thins out’’ the integers coprime to~\(n\).
Because these factors multiply, \(\varphi\) enjoys the strong
multiplicativity property that underlies its central role in
number theory and cryptography (e.g.\ RSA).
%------------------------------------------------------------
%  Solution to Problem 5:  Isomorphism classes of the groups
%------------------------------------------------------------
\begin{solution}
  Let us label the six groups for easy reference:
  \[
  \begin{aligned}
  G_1 &= \Bbb Z_{12}, &
  G_2 &= \Bbb Z_3 \oplus \Bbb Z_2 \oplus \Bbb Z_2, &
  G_3 &= \Bbb Z_2 \oplus \Bbb Z_3 \oplus \Bbb Z_4,\\
  G_4 &= \Bbb Z_3 \oplus \Bbb Z_4, &
  G_5 &= C_{12}, &
  G_6 &= C_4 \times C_3.
  \end{aligned}
  \]
  
  \subsection*{1.\;  Invariants provided by the Fundamental Theorem of Finite Abelian Groups}
  The fundamental theorem states that every finite abelian group is (up to
  isomorphism) a direct sum of cyclic \(p\)-power groups.  Useful
  \emph{isomorphism invariants} are:
  \[
  \lvert G\rvert,\qquad
  \text{exponent}(G)=\max\{\lvert g\rvert:g\in G\},\qquad
  \text{the multiset of invariant factors.}
  \]
  
  \begin{center}
  \renewcommand{\arraystretch}{1.15}
  \begin{tabular}{c|c c c c}
   & Order & Primary decomposition & Exponent & Cyclic?\\\hline
  \(G_1\) & \(12\) & \(\Bbb Z_4\oplus\Bbb Z_3\) & \(12\) & yes\\
  \(G_2\) & \(12\) & \(\Bbb Z_2\oplus\Bbb Z_2\oplus\Bbb Z_3\) & \(6\)  & no\\
  \(G_3\) & \(24\) & \(\Bbb Z_4\oplus\Bbb Z_3\oplus\Bbb Z_2\) & \(12\) & no\\
  \(G_4\) & \(12\) & \(\Bbb Z_4\oplus\Bbb Z_3\) & \(12\) & yes\\
  \(G_5\) & \(12\) & \(\Bbb Z_{12}\) (i.e.\ \(\Bbb Z_4\oplus\Bbb Z_3\)) & \(12\) & yes\\
  \(G_6\) & \(12\) & \(\Bbb Z_4\oplus\Bbb Z_3\) & \(12\) & yes\\
  \end{tabular}
  \end{center}
  
  \subsection*{2.\;  Classification}
  \begin{itemize}
    \item \textbf{Order~24.}\; Only \(G_3\) has order 24, so it is
          \emph{not} isomorphic to any other group on the list.
  
    \item \textbf{Order~12, non‑cyclic.}\;
          \(G_2\) is the unique non‑cyclic abelian group of order 12
          (its exponent is~6, not~12), hence it is
          \emph{not} isomorphic to the cyclic groups below.
  
    \item \textbf{Order~12, cyclic.}\;
          When two integers are coprime, the direct product of the
          corresponding cyclic groups is cyclic of the product order:
          \(
            C_m\times C_n\cong C_{mn}\;(\gcd(m,n)=1).
          \)
          Therefore
          \[
            G_1\cong G_4\cong G_5\cong G_6\cong C_{12}.
          \]
  \end{itemize}
  
  \subsection*{3.\;  Pair‑by‑pair verdict}
  Write “\,\(\cong\)” for “isomorphic” and “\(\not\cong\)” otherwise.
  \[
  \begin{array}{c|cccccc}
          & G_1 & G_2 & G_3 & G_4 & G_5 & G_6\\\hline
    G_1   &      & \not\cong & \not\cong & \cong & \cong & \cong\\
    G_2   &      &           & \not\cong & \not\cong & \not\cong & \not\cong\\
    G_3   &      &           &           & \not\cong & \not\cong & \not\cong\\
    G_4   &      &           &           &          & \cong & \cong\\
    G_5   &      &           &           &          &        & \cong
  \end{array}
  \]
  
  \paragraph{Summary.}
  \begin{enumerate}[label=\alph*)]
    \item \(G_1,G_4,G_5,G_6\) form a single isomorphism class (all are cyclic of order 12).
    \item \(G_2\) is the unique non‑cyclic group of order 12 on the list.
    \item \(G_3\) is the only group of order 24 and hence stands alone.
  \end{enumerate}
  \end{solution}%------------------------------------------------------------
%  Solution to Problem 5:  Isomorphism classes of the groups
%------------------------------------------------------------
\begin{solution}
Let us label the six groups for easy reference:
\[
\begin{aligned}
G_1 &= \Bbb Z_{12}, &
G_2 &= \Bbb Z_3 \oplus \Bbb Z_2 \oplus \Bbb Z_2, &
G_3 &= \Bbb Z_2 \oplus \Bbb Z_3 \oplus \Bbb Z_4,\\
G_4 &= \Bbb Z_3 \oplus \Bbb Z_4, &
G_5 &= C_{12}, &
G_6 &= C_4 \times C_3.
\end{aligned}
\]

\subsection*{1.\;  Invariants provided by the Fundamental Theorem of Finite Abelian Groups}
The fundamental theorem states that every finite abelian group is (up to
isomorphism) a direct sum of cyclic \(p\)-power groups.  Useful
\emph{isomorphism invariants} are:
\[
\lvert G\rvert,\qquad
\text{exponent}(G)=\max\{\lvert g\rvert:g\in G\},\qquad
\text{the multiset of invariant factors.}
\]

\begin{center}
\renewcommand{\arraystretch}{1.15}
\begin{tabular}{c|c c c c}
 & Order & Primary decomposition & Exponent & Cyclic?\\\hline
\(G_1\) & \(12\) & \(\Bbb Z_4\oplus\Bbb Z_3\) & \(12\) & yes\\
\(G_2\) & \(12\) & \(\Bbb Z_2\oplus\Bbb Z_2\oplus\Bbb Z_3\) & \(6\)  & no\\
\(G_3\) & \(24\) & \(\Bbb Z_4\oplus\Bbb Z_3\oplus\Bbb Z_2\) & \(12\) & no\\
\(G_4\) & \(12\) & \(\Bbb Z_4\oplus\Bbb Z_3\) & \(12\) & yes\\
\(G_5\) & \(12\) & \(\Bbb Z_{12}\) (i.e.\ \(\Bbb Z_4\oplus\Bbb Z_3\)) & \(12\) & yes\\
\(G_6\) & \(12\) & \(\Bbb Z_4\oplus\Bbb Z_3\) & \(12\) & yes\\
\end{tabular}
\end{center}

\subsection*{2.\;  Classification}
\begin{itemize}
  \item \textbf{Order~24.}\; Only \(G_3\) has order 24, so it is
        \emph{not} isomorphic to any other group on the list.

  \item \textbf{Order~12, non‑cyclic.}\;
        \(G_2\) is the unique non‑cyclic abelian group of order 12
        (its exponent is~6, not~12), hence it is
        \emph{not} isomorphic to the cyclic groups below.

  \item \textbf{Order~12, cyclic.}\;
        When two integers are coprime, the direct product of the
        corresponding cyclic groups is cyclic of the product order:
        \(
          C_m\times C_n\cong C_{mn}\;(\gcd(m,n)=1).
        \)
        Therefore
        \[
          G_1\cong G_4\cong G_5\cong G_6\cong C_{12}.
        \]
\end{itemize}

\subsection*{3.\;  Pair‑by‑pair verdict}
Write “\,\(\cong\)” for “isomorphic” and “\(\not\cong\)” otherwise.
\[
\begin{array}{c|cccccc}
        & G_1 & G_2 & G_3 & G_4 & G_5 & G_6\\\hline
  G_1   &      & \not\cong & \not\cong & \cong & \cong & \cong\\
  G_2   &      &           & \not\cong & \not\cong & \not\cong & \not\cong\\
  G_3   &      &           &           & \not\cong & \not\cong & \not\cong\\
  G_4   &      &           &           &          & \cong & \cong\\
  G_5   &      &           &           &          &        & \cong
\end{array}
\]

\paragraph{Summary.}
\begin{enumerate}[label=\alph*)]
  \item \(G_1,G_4,G_5,G_6\) form a single isomorphism class (all are cyclic of order 12).
  \item \(G_2\) is the unique non‑cyclic group of order 12 on the list.
  \item \(G_3\) is the only group of order 24 and hence stands alone.
\end{enumerate}
\end{solution}
%----------------------------------------------------------------------
%  Alternative write‑up mirroring the instructor’s solution in the scan
%  (Problem 5: deciding pair‑wise isomorphisms)
%----------------------------------------------------------------------

\begin{solution}
  Let the six groups be
  \[
  \Bbb Z_{12},\quad
  \Bbb Z_3\oplus\Bbb Z_2\oplus\Bbb Z_2,\quad
  \Bbb Z_2\oplus\Bbb Z_3\oplus\Bbb Z_4,\quad
  \Bbb Z_3\oplus\Bbb Z_4,\quad
  C_{12},\quad
  C_4\times C_3 .
  \]
  
  \subsection*{Step 0.  Transitivity of isomorphism}
  If \(A\cong B\) and \(B\cong C\) then \(A\cong C\):
  given isomorphisms \(\varphi:A\to B\) and \(\psi:B\to C\),
  the composite \(\psi\circ\varphi\) is a bijective homomorphism, hence an
  isomorphism.  We may therefore group the list into isomorphism classes
  and then check only representatives.
  
  \subsection*{Step 1.  A block of four mutually isomorphic groups}
  \[
  \boxed{\;
    \Bbb Z_{12}\;\cong\;
    \Bbb Z_3\oplus\Bbb Z_4\;\cong\;
    C_{12}\;\cong\;
    C_4\times C_3
  \;}
  \]
  
  \begin{itemize}
    \item \textbf{Chinese Remainder Theorem (CRT).}\;
          Because \(\gcd(3,4)=1\),
          \(
            \Bbb Z_{12}\cong\Bbb Z_{3}\oplus\Bbb Z_{4}.
          \)
    \item \textbf{Notation.}\;
          \(C_{12}\) is simply another name for \(\Bbb Z_{12}\).
    \item \textbf{Product of coprime cyclics.}\;
          Again by the CRT,
          \(
            \Bbb Z_4\oplus\Bbb Z_3\cong\Bbb Z_{12},
          \)
          so \(C_4\times C_3\) is cyclic of order 12 and
          lands in the same class.
  \end{itemize}
  
  Consequently any two of these four groups are isomorphic by
  transitivity.
  
  \subsection*{Step 2.  Why \(\Bbb Z_3\oplus\Bbb Z_2\oplus\Bbb Z_2\) is not in the class}
  
  In a cyclic group of order~12 there exists an element of order~4.
  In \(\Bbb Z_3\oplus\Bbb Z_2\oplus\Bbb Z_2\) we have \(6x=0\) for every
  \(x\) (check on generators), so the only element orders are
  \(1,2,3,6\).
  Since no element reaches order 4, the group cannot be cyclic of
  order 12 and hence is not isomorphic to \(\Bbb Z_{12}\) or to any of
  the three groups from Step 1.
  
  \subsection*{Step 3.  Why \(\Bbb Z_2\oplus\Bbb Z_3\oplus\Bbb Z_4\) is isolated}
  
  Its order is \(2\cdot3\cdot4 = 24\).  
  All other five groups have order~12.
  Different orders preclude isomorphism, so
  \[
     \Bbb Z_2\oplus\Bbb Z_3\oplus\Bbb Z_4
     \quad\text{is not isomorphic to any of the remaining five.}
  \]
  
  \subsection*{Conclusion}
  
  \[
  \begin{array}{@{}l}
  \text{Isomorphism class 1: }
  \Bbb Z_{12}\;\simeq\;
  \Bbb Z_3\oplus\Bbb Z_4\;\simeq\;
  C_{12}\;\simeq\;
  C_4\times C_3.\\[6pt]
  \text{Isomorphism class 2: }
  \Bbb Z_3\oplus\Bbb Z_2\oplus\Bbb Z_2
  \quad(\text{order }12,\;\text{non‑cyclic}).\\[6pt]
  \text{Isomorphism class 3: }
  \Bbb Z_2\oplus\Bbb Z_3\oplus\Bbb Z_4
  \quad(\text{order }24).
  \end{array}
  \]
  \end{solution}
  %-----------------------------------------------------------------
%  Step 2 (fully expanded):  why  \Bbb Z_3 \oplus \Bbb Z_2 \oplus \Bbb Z_2
%  cannot lie in the cyclic‑12 isomorphism class
%-----------------------------------------------------------------
\subsubsection*{Step 2.\;  Why \( \Bbb Z_3\oplus\Bbb Z_2\oplus\Bbb Z_2 \) is \emph{not} cyclic of order 12}

\paragraph{1.\;  The exponent of \(G=\Bbb Z_3\oplus\Bbb Z_2\oplus\Bbb Z_2\).}
Write elements additively:
\[
  x=(a,b,c)\quad 
  \bigl(a\in\Bbb Z_3,\; b\in\Bbb Z_2,\; c\in\Bbb Z_2\bigr).
\]
The order of \(x\) is
\[
  \lvert x\rvert=\operatorname{lcm}\bigl(\lvert a\rvert,\lvert b\rvert,\lvert c\rvert\bigr),
\]
with \(\lvert a\rvert\mid 3\) and \(\lvert b\rvert,\lvert c\rvert\mid 2\).
Hence every element order divides
\[
  \operatorname{lcm}(3,2,2)=6.
\]
\emph{Definition.}  
The least common multiple of the component moduli is called the
\emph{exponent} of the group; here  
\[
  \exp(G)=6.
\]

\paragraph{2.\;  The relation \(6x=0\) for all \(x\in G\).}
Because \(\exp(G)=6\),
\[
   6x = x+x+\dots+x = 0
   \quad\text{for every }x\in G.
\]
(This can be checked directly on the generators
\((1,0,0),\,(0,1,0),\,(0,0,1)\).)

\paragraph{3.\;  Possible element orders in \(G\).}
An element order must divide \(6\), so the only possibilities are
\[
  \boxed{\;1,\;2,\;3,\;6\;}.
\]
\emph{Crucial observation: there is \textbf{no} element of order 4.}

\paragraph{4.\;  Cyclic groups of order 12 \(\bigl(\Bbb Z_{12}\bigr)\).}
A cyclic group of order~12 necessarily contains \emph{all} divisors of~12
as element orders (because the powers of a generator realise each
divisor).  
In particular such a group has an element of order \(4\).

\paragraph{5.\;  Contradiction.}
Since \(G\) lacks an element of order~4,
\(G\) \emph{cannot} be cyclic of order 12.
Consequently
\[
  G\;\not\cong\;\Bbb Z_{12},
  \quad
  \text{nor to any of the other three cyclic groups from Step 1.}
\]

\paragraph{6.\;  Alternate quick test.}
\[
   \exp(G)=6 \;<\; \lvert G\rvert=12
   \;\Longrightarrow\;
   G\text{ is not cyclic.}
\]
(For a finite group, cyclicity \(\Leftrightarrow\exp(G)=\lvert G\rvert\).)

\bigskip
\noindent
\emph{Therefore} \(
  \Bbb Z_3\oplus\Bbb Z_2\oplus\Bbb Z_2
\) forms its own isomorphism class, distinct from the cyclic‑12 class.
\begin{problem}
  \begin{enumerate}[label=\textbf{(\alph*)}]
    \item Let $n\in\mathbb{N}$ and $k\in\mathbb{Z}_n$.  Find $\operatorname{ord}_{\mathbb{Z}_n}(k)$.
    \item Let $n,d\in\mathbb{N}$ with $d\mid n$.  Show that
          \[
            \mathbb{Z}_n\big/\{0,d,2d,\dots,(n/d-1)d\}\;\cong\;\mathbb{Z}_d .
          \]
  \end{enumerate}
\end{problem}

\begin{solution}
  \begin{enumerate}[label=\textbf{(\alph*)}]
    \item
      The \emph{order} of $k\in\mathbb{Z}_n$ is the least $m\in\mathbb{N}$ such that
      $mk \equiv 0 \pmod{n}$.
      Write $d=\gcd(n,k)$ and set $n'=\tfrac{n}{d}$ and $k'=\tfrac{k}{d}$,
      so $(n',k')=1$.
      \[
        mk\equiv 0 \pmod{n}
        \;\Longleftrightarrow\;
        n \mid mk
        \;\Longleftrightarrow\;
        \frac{n}{d}\;\Bigm|\; m\,\frac{k}{d}
        \;\Longleftrightarrow\;
        n' \mid m k'.
      \]
      Because $n'$ and $k'$ are coprime, $n'\mid m$.
      The smallest such $m$ is therefore $n'$,
      giving
      \[
        \boxed{\operatorname{ord}_{\mathbb{Z}_n}(k)=\frac{n}{\gcd(n,k)}}.
      \]

    \item
      Define
      \[
        \psi:\mathbb{Z}_n \longrightarrow \mathbb{Z}_d, \qquad
        \psi\bigl([a]_n\bigr)=[a]_d.
      \]
      \begin{description}
        \item[Well‑definedness.]
          If $[a]_n=[b]_n$ then $n\mid a-b$.  Since $d\mid n$, we also have
          $d\mid a-b$, so $[a]_d=[b]_d$.
        \item[Homomorphism.]
          \(
            \psi([a]_n+[b]_n)=\psi([a+b]_n)=[a+b]_d=[a]_d+[b]_d
          \).
        \item[Surjectivity.]
          For any $[r]_d\in\mathbb{Z}_d$, $\psi([r]_n)=[r]_d$.
        \item[Kernel.]
          \[
            \ker\psi
              =\{[a]_n : d\mid a\}
              =\{0,d,2d,\dots,(n/d-1)d\}.
          \]
      \end{description}

      The First Isomorphism Theorem yields
      \[
        \mathbb{Z}_n/\ker\psi \;\cong\; \operatorname{im}\psi = \mathbb{Z}_d,
      \]
      which is exactly the claimed isomorphism.
      \qedhere
  \end{enumerate}
\end{solution}
\begin{example}
  We illustrate Problem 2 with the concrete choice
  \[
    n=12,\quad k=8,\quad d=4\;(=\!4\mid 12).
  \]

  \begin{enumerate}[label=\textbf{(\alph*)}]
    %-----------------------------------------------------------------------
    \item \textbf{Order of $k$ in $\mathbf Z_{12}$.}

          First compute the greatest common divisor:
          \[
            \gcd(12,8)=4.
          \]
          The theorem predicts
          \[
            \operatorname{ord}_{\mathbf Z_{12}}(8)=\frac{12}{4}=3.
          \]
          Check directly by taking successive multiples of $8$ modulo $12$:
          \[
            8\equiv 8,\quad
            2\cdot 8=16\equiv 4,\quad
            3\cdot 8=24\equiv 0\pmod{12}.
          \]
          No smaller positive multiple hits $0$, so the order really is~$3$.

    %-----------------------------------------------------------------------
    \item \textbf{Quotient $\mathbf Z_{12}\big/\{0,4,8\}\cong \mathbf Z_{4}$.}

          Let
          \[
            H=\{0,4,8\}\subseteq\mathbf Z_{12},
          \]
          the set of all classes in $\mathbf Z_{12}$ that are divisible by $d=4$.
          The cosets of $H$ are
          \[
            \begin{aligned}
              [0] &= \{0,4,8\}, &
              [1] &= \{1,5,9\},\\
              [2] &= \{2,6,10\}, &
              [3] &= \{3,7,11\}.
            \end{aligned}
          \]
          Thus the quotient group $\mathbf Z_{12}/H$ has exactly \(\lvert\mathbf Z_{12}\rvert/\lvert H\rvert=12/3=4\) elements.

          Define the map
          \[
            \psi : \mathbf Z_{12}\longrightarrow \mathbf Z_{4},
            \qquad
            \psi([a]_{12})=[a]_{4}.
          \]
          \begin{itemize}
            \item \emph{Homomorphism:}\;
                  $\psi([a+b]_{12})=[a+b]_{4}=[a]_{4}+[b]_{4}=\psi([a]_{12})+\psi([b]_{12})$.
            \item \emph{Kernel:}\;
                  $\ker\psi = \{[a]_{12}:4\mid a\}=H$.
            \item \emph{Surjective:}\;
                  Every class in $\mathbf Z_{4}$ is hit, e.g.\ $\psi([r]_{12})=[r]_{4}$.
          \end{itemize}
          By the First Isomorphism Theorem,
          \[
            \mathbf Z_{12}/H \;\cong\; \mathbf Z_{4}.
          \]

          Concretely, the isomorphism sends each coset to the corresponding
          residue modulo~\(4\):
          \[
            [0]\!\mapsto\!0,\;
            [1]\!\mapsto\!1,\;
            [2]\!\mapsto\!2,\;
            [3]\!\mapsto\!3.
          \]
  \end{enumerate}
\end{example}
\begin{problem}
  Let $G_1,\dots,G_s$ be groups and let $G=G_1\times\dots\times G_s$ be
  their direct product.
  \begin{enumerate}[label=\textbf{(\alph*)}]
    \item For finite–order elements $a_i\in G_i$ show that
          \[
            \ord_G\bigl((a_1,\dots,a_s)\bigr)
            \;=\;
            \lcm\bigl(\ord_{G_1}(a_1),\dots,\ord_{G_s}(a_s)\bigr).
          \]
    \item Show that
          \(
            \mathbb Z_4\oplus\mathbb Z_4
            \not\cong
            \mathbb Z_4\oplus\mathbb Z_2\oplus\mathbb Z_2.
          \)
          (Hint in the text: compare how many elements of order \(2\) each
          group has.)
  \end{enumerate}
\end{problem}

\begin{solution}
  \begin{enumerate}[label=\textbf{(\alph*)}]
    %---------------------------------------------------------------------
    \item
      Let
      \[
        n_i=\ord_{G_i}(a_i)\qquad(i=1,\dots,s),\qquad
        n=\lcm(n_1,\dots,n_s).
      \]
      Then \(n_i\mid n\) for every~\(i\), so
      \[
        (a_1,\dots,a_s)^{\!n}
        =\bigl(a_1^{\;n},\dots,a_s^{\;n}\bigr)
        =(e_{G_1},\dots,e_{G_s})=e_G,
      \]
      which shows \(\ord_G(a_1,\dots,a_s)\le n\).

      Conversely, write \(m=\ord_G(a_1,\dots,a_s)\).
      Because \((a_1,\dots,a_s)^{\!m}=e_G\),
      we have \(a_i^{\,m}=e_{G_i}\) for each \(i\),
      hence \(n_i\mid m\).
      Therefore \(n=\lcm(n_1,\dots,n_s)\mid m\),
      so \(m\ge n\).
      Combining the two inequalities gives \(m=n\), i.e.
      \[
        \boxed{\ord_G(a_1,\dots,a_s)=\lcm\bigl(\ord(a_1),\dots,\ord(a_s)\bigr)}.
      \]

      \paragraph{Concrete illustration.}
      Take
      \(G_1=\mathbb Z_4\), \(G_2=\mathbb Z_6\), \(G_3=\mathbb Z_8\) and
      \[
        a_1=2\pmod 4,\qquad
        a_2=3\pmod 6,\qquad
        a_3=4\pmod 8.
      \]
      Their individual orders are
      \(\ord(a_1)=2\), \(\ord(a_2)=2\), \(\ord(a_3)=2\),
      so the formula predicts
      \[
        \ord\bigl((2,3,4)\bigr)=\lcm(2,2,2)=2.
      \]
      Indeed
      \((2,3,4)^2=(4,6,8)\equiv(0,0,0)\), and no smaller positive power
      annihilates the element.

    %---------------------------------------------------------------------
    \item
      By part (a) (in its additive language) and the formula from Problem 2,
      an element \((a,b)\in\mathbb Z_4\oplus\mathbb Z_4\) has order
      \[
        \ord\bigl((a,b)\bigr)=\lcm\!\left(
          \frac{4}{\gcd(a,4)},\;
          \frac{4}{\gcd(b,4)}
        \right).
      \]
      This equals \(2\) **iff**
      \(\frac{4}{\gcd(a,4)}=2=\frac{4}{\gcd(b,4)}\), i.e.
      \(\gcd(a,4)=\gcd(b,4)=2\).
      The only residues modulo \(4\) with \(\gcd(\,\cdot\, ,4)=2\)
      are \(2\) and \(2\) itself, so \(a,b\in\{2\}\).
      But \(a\) and \(b\) cannot *both* be \(0\) or \(2\) simultaneously
      (that would make the lcm equal to \(1\) or \(2\) incorrectly),
      hence the order‑\(2\) elements are
      \[
        (2,0),\;(0,2),\;(2,2).
      \]
      Thus \(\mathbb Z_4\oplus\mathbb Z_4\) has **exactly 3**
      elements of order \(2\).

      Now let \((a,b,c)\in\mathbb Z_4\oplus\mathbb Z_2\oplus\mathbb Z_2\).
      Again using part (a),
      \[
        \ord(a,b,c)=\lcm\!\left(
          \frac{4}{\gcd(a,4)},\;
          \frac{2}{\gcd(b,2)},\;
          \frac{2}{\gcd(c,2)}
        \right).
      \]
      To obtain order \(2\) we need
      \[
        \frac{4}{\gcd(a,4)},\;
        \frac{2}{\gcd(b,2)},\;
        \frac{2}{\gcd(c,2)}
        \;\in\;\{1,2\}
        \quad\text{and not all equal to }1.
      \]
      Concretely, \(a\) must be \(0\) or \(2\), while
      \((b,c)\neq(0,0)\).
      That gives
      \[
        2\ \text{choices for }a\;\times\;(2^2-1)=3\ \text{choices for }(b,c)
        =6
      \]
      elements—but we must exclude the one case where \((a,b,c)=(0,0,0)\),
      so altogether there are **7** elements of order \(2\).

      Because isomorphisms preserve element orders, the two groups cannot be
      isomorphic:
      \[
        \boxed{
          \mathbb Z_4\oplus\mathbb Z_4
          \ \not\cong\
          \mathbb Z_4\oplus\mathbb Z_2\oplus\mathbb Z_2
        }.
      \]
      (One has \(3\) elements of order \(2\), the other \(7\).)
  \end{enumerate}
\end{solution}
\begin{theorem}[Chinese Remainder Theorem – integer version]
  \label{thm:CRT-integers}
  Let $n_{1},\dots,n_{k}\in\mathbb Z_{>0}$ be pairwise coprime
  ($\gcd(n_{i},n_{j})=1$ for $i\neq j$).
  Set $N=n_{1}n_{2}\dots n_{k}$.
  \begin{enumerate}[label=\textup{(\roman*)}]
    \item For any integers $a_{1},\dots,a_{k}$ the system of congruences
    \[
      x\equiv a_{1}\pmod{n_{1}},\;
      \dots,\;
      x\equiv a_{k}\pmod{n_{k}}
    \]
    has a solution $x\in\mathbb Z$.
    \item The solution is unique modulo $N$:
          if $x$ and $y$ both satisfy the system, then $x\equiv y\pmod{N}$.
  \end{enumerate}
  Equivalently,
  \[
    \boxed{\;
      \mathbb Z/N\mathbb Z
      \;\;\cong\;\;
      \mathbb Z/n_{1}\mathbb Z\;\times\;\dots\;\times\;
      \mathbb Z/n_{k}\mathbb Z
    \;}
  \]
  via the map $[x]_{N}\mapsto\bigl([x]_{n_{1}},\dots,[x]_{n_{k}}\bigr)$.
  \end{theorem}
  
  \begin{theorem}[Chinese Remainder Theorem – ring version]
  \label{thm:CRT-rings}
  Let $R$ be a commutative ring with $1$, and let $I_{1},\dots,I_{k}$ be
  pairwise comaximal ideals
  ($I_{i}+I_{j}=R$ whenever $i\neq j$).
  Then
  \[
    \boxed{\;
      R\Bigl/\bigl(\,I_{1}\cap\dots\cap I_{k}\bigr)
      \;\;\cong\;\;
      R/I_{1}\;\times\;\dots\;\times\;R/I_{k}
    \;}
  \]
  by the canonical homomorphism
  \(
    r \;\mapsto\; (\,r+I_{1},\,\dots,\,r+I_{k}\,).
  \)
  \end{theorem}
  %%%%%%%%%%%%%%%%%%%%%%%%%%%%%%%%%%%%%%%%%%%%%%%%%%%%%%%%%%%%%%%%%%%%%%%%
%  All elements of A_4 and their orders
%%%%%%%%%%%%%%%%%%%%%%%%%%%%%%%%%%%%%%%%%%%%%%%%%%%%%%%%%%%%%%%%%%%%%%%%
\begin{solution}
  Recall that the alternating group
  \(
    A_{4}\subset S_{4}
  \)
  consists of the \emph{even} permutations of four letters.
  A permutation is even iff its cycle decomposition contains an \emph{even}
  number of cycles of even length.
  
  \subsection*{1.\;Classification of even permutations in $S_{4}$}
  \begin{itemize}
    \item \textbf{Identity.} One element \(e\) (order 1).
    \item \textbf{3–cycles.}  
          A cycle of length 3 is even:
          its disjoint‑cycle form has one cycle of length 3 and
          one fixed point (cycle of length 1), so the total number of even‑length
          cycles is \(0\) (even).
          Counting: each 3‑element subset of \(\{1,2,3,4\}\) gives two
          orientations, hence
          \(4\choose 3\)\(\times 2 = 8\) such cycles.
    \item \textbf{Double transpositions.}  
          A product of two disjoint transpositions
          \(\bigl((ab)(cd)\bigr)\) is even because it has \emph{two}
          cycles of even length.
          There are exactly three of them:
          \((12)(34),\,(13)(24),\,(14)(23)\).
  \end{itemize}
  
  Thus
  \[
    A_{4}\;=\;
    \Bigl\{
      e,\,
      (12)(34),\,(13)(24),\,(14)(23),\,
      (123),\,(132),\,(124),\,(142),\,(234),\,(243),\,(134),\,(143)
    \Bigr\},
  \]
  and \(|A_{4}|=1+3+8=12\).
  
  \subsection*{2.\;Orders of the elements}
  
  \begin{center}
  \renewcommand{\arraystretch}{1.3}
  \begin{tabular}{|c|c|}
  \hline
  \textbf{Type} & \textbf{Order} \\ \hline
  Identity \(e\)                         & \(1\) \\ \hline
  3‑cycles \((123),(132),(124),(142),(234),(243),(134),(143)\) & \(3\) \\ \hline
  Double transpositions
  \((12)(34),(13)(24),(14)(23)\)         & \(2\) \\ \hline
  \end{tabular}
  \end{center}
  
  \medskip\noindent
  In summary:
  \[
    \boxed{\;
      \text{identity (1 element) of order }1,\;
      \text{three double transpositions of order }2,\;
      \text{eight 3‑cycles of order }3
    \;}
  \]
  which exhausts the 12 elements of \(A_{4}\).
  \end{solution}
  %%%%%%%%%%%%%%%%%%%%%%%%%%%%%%%%%%%%%%%%%%%%%%%%%%%%%%%%%%%%%%%%%%%%%%%%
%  Chinese Remainder Theorem applied to 3 and 5
%%%%%%%%%%%%%%%%%%%%%%%%%%%%%%%%%%%%%%%%%%%%%%%%%%%%%%%%%%%%%%%%%%%%%%%%
\begin{example}[CRT for coprime cyclic groups]
  We show explicitly that 
  \[
    \boxed{\;\mathbb Z_{3}\times\mathbb Z_{5}\;\cong\;\mathbb Z_{15}\;}
  \]
  using the Chinese Remainder Theorem (CRT).
  
  \subsection*{1.  CRT statement (two moduli)}
  For coprime integers $m,n$ the map
  \[
    \varphi:\mathbb Z_{mn}\longrightarrow 
            \mathbb Z_{m}\times\mathbb Z_{n},
    \qquad
    \varphi\bigl([x]_{mn}\bigr)=\bigl([x]_{m},[x]_{n}\bigr)
  \]
  is a group isomorphism (additive notation).
  
  \subsection*{2.  Specialise to $m=3$, $n=5$}
  Set $N=mn=15$.  Then
  \[
    \varphi:\mathbb Z_{15}\;\longrightarrow\;
            \mathbb Z_{3}\times\mathbb Z_{5},
    \qquad
    [x]_{15}\;\longmapsto\;\bigl([x]_{3},[x]_{5}\bigr).
  \]
  
  \begin{enumerate}[label=\textup{(\roman*)}]
    \item \textbf{Homomorphism:} 
          $\varphi([x+y]_{15})=(\,[x+y]_{3},[x+y]_{5}\,)
          =( \,[x]_{3},[x]_{5}\,)+( \,[y]_{3},[y]_{5}\,)
          =\varphi([x]_{15})+\varphi([y]_{15})$.
  
    \item \textbf{Kernel:} 
          \[
            \ker\varphi
            =\{[x]_{15}\mid x\equiv0\pmod3\text{ and }x\equiv0\pmod5\}
            =[0]_{15},
          \]
          because $0$ is the only residue $\le14$ divisible by both $3$ and $5$.
  
    \item \textbf{Surjectivity:} 
          Given any pair $\bigl([a]_{3},[b]_{5}\bigr)$,
          the CRT guarantees a unique $x\pmod{15}$ solving
          $x\equiv a\pmod3$, $x\equiv b\pmod5$.
          Hence every element of $\mathbb Z_{3}\times\mathbb Z_{5}$ is hit.
  \end{enumerate}
  
  Since $\varphi$ is a bijective homomorphism, it is an isomorphism, proving
  \[
    \mathbb Z_{15}\;\cong\;\mathbb Z_{3}\times\mathbb Z_{5}.
  \]
  
  \subsection*{3.  Cyclic reinterpretation}
  \begin{itemize}
    \item $\mathbb Z_{15}$ is cyclic, generated by $[1]_{15}$.
    \item Under $\varphi$, this generator maps to $(1,1)$.
          Its order is $\operatorname{lcm}(3,5)=15$, so 
          $(1,1)$ itself generates $\mathbb Z_{3}\times\mathbb Z_{5}$.
  \end{itemize}
  Thus the direct product of two \emph{coprime} cyclic groups is again cyclic,
  and the CRT provides the explicit bridge between the two descriptions.
  \end{example}
  \paragraph{Why every \(3\)-cycle is an \emph{even} permutation.}

There are two common ways to see this.

\medskip
\textbf{(1)  Transposition count.}
Any cycle of length \(r\) can be expressed as a product of \(r-1\)
transpositions:
\[
  (a_{1}\,a_{2}\,\dots\,a_{r})
  \;=\;
  (a_{1}\,a_{r})\,(a_{1}\,a_{r-1})\cdots(a_{1}\,a_{2}).
\]
For \(r=3\) that is  
\[
  (a\,b\,c)=(a\,c)\,(a\,b),
\]
which involves \emph{exactly two} transpositions.
Because the parity of a permutation equals the parity of any transposition
decomposition, a \(3\)-cycle is even (two is an even number).

\medskip
\textbf{(2)  Even–cycle criterion.}
Another characterization says a permutation is even
iff its disjoint‑cycle decomposition contains an \emph{even number of
even‑length cycles}.
In \(S_{4}\) a \(3\)-cycle always appears as
\[
  (a\,b\,c)\;(d),
\]
namely one cycle of length \(3\) and one fixed point \((d)\), which is a cycle
of length \(1\).
Here the count of \emph{even‑length} cycles is
\[
  \underbrace{0}_{\text{from the 3‑cycle}}
  \;+\;
  \underbrace{0}_{\text{from the 1‑cycle}}
  \;=\;0,
\]
and \(0\) is itself even.
Hence the permutation is even.

\medskip
Either viewpoint confirms that every \(3\)-cycle in \(S_{4}\)
belongs to the alternating group \(A_{4}\).
\[
\boxed{\;
  D_{n}\;=\;
  \langle\, r,s \;\bigm|\; r^{\,n}=e,\; s^{2}=e,\; srs = r^{-1}\,\rangle
\;}
\]

\begin{itemize}
  \item $r$ represents a counter‑clockwise rotation of a regular $n$‑gon,
        so $r^{\,n}=e$.
  \item $s$ represents any one reflection (a flip across an axis),
        so $s^{2}=e$.
  \item The relation $srs=r^{-1}$ captures the geometric fact that a
        reflection reverses the orientation of a rotation.
\end{itemize}

A convenient normal form for the $2n$ elements is
\[
  D_{n}
  =\bigl\{\,r^{k}\;(0\le k\le n-1),\;\;
           s\,r^{k}\;(0\le k\le n-1)\bigr\}.
\]
Rotations $r^{k}$ have order $\tfrac{n}{\gcd(n,k)}$,
while each reflection $sr^{k}$ has order $2$.
\begin{solution}
  Let 
  \[
    Q_{8}= \{\pm 1,\pm i,\pm j,\pm k \mid i^{2}=j^{2}=k^{2}=-1,\; ij=k,\; jk=i,\; ki=j\}
  \]
  act on the set of its subgroups by conjugation:
  \[
    g\cdot H \;=\; gHg^{-1}\quad (g\in Q_{8},\, H\le Q_{8}).
  \]

  \begin{enumerate}
    \item \textbf{Identify the subgroup.}\\
          Set 
          \[
            H=\{1,-1,i,-i\}= \langle i\rangle \cong C_{4}.
          \]

    \item \textbf{Check normality.}\\
          $Q_{8}$ is a \emph{Hamiltonian} group, so \emph{every} subgroup is normal.  
          Equivalently, for any $g\in Q_{8}$,
          \[
            g\,i\,g^{-1}\in\{\pm i\}\subseteq H
            \quad\Longrightarrow\quad
            gHg^{-1}=H.
          \]

    \item \textbf{Determine the stabilizer.}\\
          The stabilizer (normalizer) of $H$ in $Q_{8}$ is
          \[
            N_{Q_{8}}(H)=\{g\in Q_{8}\mid gHg^{-1}=H\}=Q_{8}.
          \]
          Hence the entire group fixes $H$ setwise.
  \end{enumerate}

  \vspace{0.5em}
  \textbf{Result.}\;
  The stabilizer of the subgroup $\{1,-1,i,-i\}$ under conjugation by $Q_{8}$ is the whole quaternion group $Q_{8}$ itself.
\end{solution}
\begin{explanation}
  A (non-abelian) group in which \emph{every} subgroup is normal is called
  a \textbf{Hamiltonian group}.
  To see why the quaternion group
  \[
    Q_8=\{\pm1,\pm i,\pm j,\pm k\mid i^2=j^2=k^2=-1,\;
           ij=k,\; jk=i,\; ki=j\}
  \]
  is Hamiltonian—and hence why \emph{every} subgroup, including
  $H=\{1,-1,i,-i\}$, is normal—we can argue in two complementary ways.

  %--------------------------------------------------------------------
  \subsection*{1.\  Direct element-by-element verification}
  The centre of $Q_8$ is $Z(Q_8)=\{\pm1\}$.  
  For any $g\in Q_8\smallsetminus Z(Q_8)$ we have
  $g^2=-1\in Z(Q_8)$, so
  \[
      g h g^{-1}=g h\,(-g)=-(ghg)\qquad(h\in Q_8).
  \]
  In particular, conjugation by \emph{any} element merely flips the sign
  of the element being conjugated:
  \[
     g\,(\pm i)\,g^{-1}=\pm i,\qquad
     g\,(\pm j)\,g^{-1}=\pm j,\qquad
     g\,(\pm k)\,g^{-1}=\pm k.
  \]
  Hence the sets
  $\langle i\rangle=\{1,-1,i,-i\}$,
  $\langle j\rangle=\{1,-1,j,-j\}$,
  $\langle k\rangle=\{1,-1,k,-k\}$
  are all invariant under conjugation, i.e.\ \emph{normal}.
  Because \emph{every} subgroup of $Q_8$ is one of
  \[
    \{1\},\;
    \{\pm1\},\;
    \langle i\rangle,\;
    \langle j\rangle,\;
    \langle k\rangle,\;
    Q_8,
  \]
  all subgroups are normal.

  %--------------------------------------------------------------------
  \subsection*{2.\  Structural classification}
  A classical theorem (Baer\,–\,Hall) says that the only finite
  non-abelian groups in which \emph{all} subgroups are normal are the
  direct products
  \[
     Q_8\times C_2^{\,r}\times A,
  \]
  where $r\ge 0$ and $A$ is an abelian group of odd order.
  In particular, $Q_8$ itself is the \emph{smallest} non-abelian
  Hamiltonian group, so by definition every subgroup is normal.

  %--------------------------------------------------------------------
  \paragraph{Conclusion.}
  Since $Q_8$ is Hamiltonian, the subgroup
  $H=\{1,-1,i,-i\}$ is normal, i.e.
  $gHg^{-1}=H$ for all $g\in Q_8$.
  Therefore its stabiliser (normaliser) under the conjugation action is
  the whole group:
  \[
     N_{Q_8}(H)=Q_8.\qedhere
  \]
\end{explanation}
\begin{explanation}
  The statement  
  \[
     Q_{8} \text{ is the semidirect product of }
     H=\{1,-1\}\cong C_{2} 
     \text{ and }
     K=\{1,-1,j,-j\}\cong C_{4}
  \]
  is \textbf{false}, because \(H\) and \(K\) fail the defining
  conditions for an \emph{internal semidirect product}.

  \medskip
  Recall that, for subgroups \(H,K\le G\), we have
  \[
     G = H\rtimes K
     \quad\Longleftrightarrow\quad
     \begin{cases}
        \text{(i)} & H\unlhd G,\\
        \text{(ii)}& H\cap K=\{1\},\\
        \text{(iii)}& HK=G.
     \end{cases}
  \]

  \paragraph{Failure of condition (ii).}
  For \(H=\{1,-1\}\) and \(K=\{1,-1,j,-j\}\subseteq Q_{8}\),
  \[
     H\cap K = \{1,-1\}= H \neq \{1\},
  \]
  so the intersection is \emph{not} trivial.

  \paragraph{Failure of condition (iii).}
  Because \(H\subseteq K\), we obtain
  \[
     HK = K = \{1,-1,j,-j\},
  \]
  whose order is \(4\), whereas \(|Q_{8}|=8\).
  Hence \(HK\neq Q_{8}\).

  \medskip
  Since conditions (ii) and (iii) both fail, \(Q_{8}\) cannot be
  expressed as the semidirect product \(H\rtimes K\).
\end{explanation}
\end{document}
