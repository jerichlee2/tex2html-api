\documentclass[12pt]{article}

% Packages
\usepackage[margin=1in]{geometry}
\usepackage{amsmath,amssymb,amsthm}
\usepackage{enumitem}
\usepackage{hyperref}
\usepackage{xcolor}
\usepackage{import}
\usepackage{xifthen}
\usepackage{pdfpages}
\usepackage{transparent}
\usepackage{listings}
\usepackage{tikz}
\usepackage{physics}
\usepackage{siunitx}
\usepackage{booktabs}
\usepackage{cancel}
  \usetikzlibrary{calc,patterns,arrows.meta,decorations.markings}


\DeclareMathOperator{\Log}{Log}
\DeclareMathOperator{\Arg}{Arg}
\DeclareMathOperator{\Aut}{Aut}


\lstset{
    breaklines=true,         % Enable line wrapping
    breakatwhitespace=false, % Wrap lines even if there's no whitespace
    basicstyle=\ttfamily,    % Use monospaced font
    frame=single,            % Add a frame around the code
    columns=fullflexible,    % Better handling of variable-width fonts
}

\newcommand{\incfig}[1]{%
    \def\svgwidth{\columnwidth}
    \import{./Figures/}{#1.pdf_tex}
}
\theoremstyle{definition} % This style uses normal (non-italicized) text
\newtheorem{solution}{Solution}
\newtheorem{proposition}{Proposition}
\newtheorem{problem}{Problem}
\newtheorem{lemma}{Lemma}
\newtheorem{theorem}{Theorem}
\newtheorem{remark}{Remark}
\newtheorem{note}{Note}
\newtheorem{definition}{Definition}
\newtheorem{example}{Example}
\newtheorem{corollary}{Corollary}
\newtheorem{explanation}{Explanation}
\theoremstyle{plain} % Restore the default style for other theorem environments
%

% Theorem-like environments
% Title information
\title{}
\author{Jerich Lee}
\date{\today}

\begin{document}

\maketitle
\begin{problem}
  Let $G$ be a group of order $63 = 3^{2}\cdot7$.
  \begin{enumerate}[]
      \item Use Sylow’s theorems to show that $G$ has a \emph{unique} Sylow–$7$ subgroup, which is therefore normal in $G$.
      \item Is $G$ necessarily abelian?  Prove your claim or give a counter-example.
  \end{enumerate}
  \end{problem}
  
  \begin{solution}
  \textbf{(a) A unique Sylow--$7$ subgroup.}
  
  \begin{enumerate}[]
      \item \emph{Apply Sylow’s divisibility and congruence conditions.}\\
            Let $n_{7}$ be the number of Sylow--$7$ subgroups of $G$.  
            By Sylow’s theorems
            \[
                n_{7}\mid 3^{2}=9
                \quad\text{and}\quad 
                n_{7}\equiv 1 \pmod{7}.
            \]
      \item \emph{List the possibilities.}\\
            The divisors of $9$ are $\{1,3,9\}$; among these, only
            \[
                1\equiv 1\pmod{7},\qquad
                3\equiv 3\pmod{7},\qquad
                9\equiv 2\pmod{7}.
            \]
            Hence the \emph{only} value satisfying both conditions is $n_{7}=1$.
      \item \emph{Conclude normality.}\\
            A Sylow subgroup that is the \emph{only} one of its order is fixed under conjugation by every element of $G$; therefore it is normal.  Thus $G$ contains a unique normal subgroup
            \[
                P_{7}\;\trianglelefteq\;G,
                \qquad |P_{7}|=7.
            \]
  \end{enumerate}
  
  \bigskip
  \textbf{(b)  $G$ is \emph{not} necessarily abelian.}
  
  \begin{enumerate}[]
      \item \emph{Reduce the problem via the normal Sylow--$7$ subgroup.}\\
            Because $P_{7}\lhd G$, the quotient group $G/P_{7}$ has order
            $|G|/|P_{7}| = 63/7 = 9$.  Hence
            \[
                G/P_{7}\cong
                \begin{cases}
                     \;\Bbb Z_{9}, &\text{cyclic of order }9,\\
                     \;\Bbb Z_{3}\times\Bbb Z_{3}, &\text{elementary abelian of order }9.
                \end{cases}
            \]
            In either case $G$ is a \emph{semidirect} product
            \[
                G \;\cong\; P_{7}\rtimes_{\!\varphi}H,
                \qquad H\;\le G,\; |H|=9,
            \]
            where $\varphi:H\to\operatorname{Aut}(P_{7})$ describes the action of~$H$ on~$P_{7}$ by conjugation.
      \item \emph{Determine the possible actions.}\\
            Since $P_{7}\cong\Bbb Z_{7}$ is cyclic,
            \[
                \operatorname{Aut}(P_{7})\;\cong\;\Bbb Z_{6}.
            \]
            Any homomorphism $\varphi:H\to\Bbb Z_{6}$ must have image whose order divides both $|H|=9$ and $|\Bbb Z_{6}|=6$; hence $|\operatorname{im}\varphi|\mid 3$.
            There are two possibilities:
            \begin{itemize}
                \item $\operatorname{im}\varphi = \{1\}$ (the trivial action) $\;\Longrightarrow\;$ the semidirect product is actually the \emph{direct} product
                      \[
                          G\;\cong\;P_{7}\times H\;\cong\;\Bbb Z_{7}\times H,
                      \]
                      which is abelian.
                \item $\operatorname{im}\varphi \cong \Bbb Z_{3}$ (a non-trivial action) $\;\Longrightarrow\;$ the semidirect product is \emph{non-abelian}.
            \end{itemize}
      \item \emph{An explicit non-abelian example.}\\
            Let $H=\Bbb Z_{9}=\langle b\rangle$ and let
            \(
               \alpha\in\operatorname{Aut}(\Bbb Z_{7})\cong\Bbb Z_{6}
            \)
            be the automorphism $\alpha(a)=a^{2}$, which has order~$3$ because
            $2^{3}=8\equiv 1\pmod{7}$.
            Define $\varphi(b)=\alpha$.  Then
            \[
                G \;=\; \bigl\langle
                       a,b \;\bigm|\;
                       a^{7}=1,\; b^{9}=1,\;
                       b\,a\,b^{-1}=a^{2}
                     \bigr\rangle
            \]
            has order $7\cdot9=63$ and is \emph{non-abelian} (the relation
            $b\,a = a^{2}b$ forces $ab\ne ba$).
  \end{enumerate}
  
  \paragraph{Conclusion.}
  Every group of order $63$ contains a normal Sylow–$7$ subgroup, but it
  need \emph{not} be abelian; the semidirect product above furnishes a
  counter-example.
  \end{solution}
  \begin{explanation}
    \textbf{Definition (external semidirect product).}
    Let $N$ and $H$ be groups and let $\varphi:H\to\Aut(N)$ be a
    homomorphism.  
    The \emph{(external) semidirect product} of $N$ by $H$ with respect to
    $\varphi$ is the set
    \[
       N\rtimes_{\!\varphi} H \;=\; \{(n,h)\mid n\in N,\;h\in H\},
    \]
    equipped with the multiplication
    \[
       (n_{1},h_{1})\,(n_{2},h_{2})
         \;=\;
       \bigl(n_{1}\,\varphi(h_{1})(n_{2}),\;h_{1}h_{2}\bigr).
    \]
    Under this operation
    \[
       N\;\cong\;\{(n,1_{H})\}\;\trianglelefteq\;N\rtimes_{\!\varphi}H,
       \qquad
       H\;\cong\;\{(1_{N},h)\}\;\le\;N\rtimes_{\!\varphi}H,
    \]
    and $N\rtimes_{\!\varphi}H$ has order $|N|\,|H|$.
    
    \bigskip
    \textbf{Applying the definition to $|G|=63$.}
    
    \begin{enumerate}[]
       \item \emph{Choose $N$ and $H$.}\\
             Let 
             \[
                N=P_{7}=\langle a\rangle\cong\Bbb Z_{7},
                \qquad
                H=\langle b\rangle\cong\Bbb Z_{9}.
             \]
       \item \emph{Describe $\Aut(N)$.}\\
             Because $N$ is cyclic of prime order, 
             \[
                \Aut(N)\;\cong\;\Bbb Z_{7}^{\times}
                          \;=\;\{1,2,3,4,5,6\}\;\cong\;\Bbb Z_{6}.
             \]
       \item \emph{Define a homomorphism $\varphi:H\to\Aut(N)$.}\\
             Pick the automorphism
             \(
                 \alpha\in\Aut(N),\; \alpha(a)=a^{2}.
             \)
             It has order~$3$ because $2^{3}=8\equiv1\pmod{7}$.  
             Send the generator of $H$ to~$\alpha$:
             \[
                \varphi(b)=\alpha,
                \qquad\text{so}\qquad
                \varphi\bigl(b^{k}\bigr)=\alpha^{k}.
             \]
             Since $\alpha^{9}=1$ (order~$3$ divides $9$), $\varphi$ is well
             defined and a homomorphism.
       \item \emph{Form the semidirect product.}\\
             Set 
             \[
                G \;=\; N\rtimes_{\!\varphi} H.
             \]
             Elements of $G$ are pairs $(a^{i},b^{j})$
             with $0\le i<7,\;0\le j<9$ and multiplication
             \[
                (a^{i},b^{j})\,(a^{u},b^{v})
                  \;=\;
                \bigl(a^{\,i}\,\alpha^{\,j}(a^{u}),\;b^{\,j+v}\bigr)
                  \;=\;
                \bigl(a^{\,i+2^{\,j}u},\;b^{\,j+v}\bigr).
             \]
             In particular,
             \[
                (1,b)\,(a,1)\,(1,b)^{-1}
                  \;=\;
                \bigl(\alpha(a),1\bigr)
                  \;=\;
                (a^{2},1),
             \]
             which translates to the single–relation presentation
             \[
               G\;=\;
               \Bigl\langle
                      a,b
                      \;\Bigm|\;
                      a^{7}=1,\;b^{9}=1,\;
                      b\,a\,b^{-1}=a^{2}
               \Bigr\rangle.
             \]
    \end{enumerate}
    
    \bigskip
    \textbf{Why this \emph{is} a semidirect product.}
    
    \begin{itemize}
       \item $N=\langle a\rangle\cong\Bbb Z_{7}$
             sits inside $G$ as the normal subgroup
             $\{(a^{i},1)\}$.
       \item $H=\langle b\rangle\cong\Bbb Z_{9}$
             sits inside $G$ as the subgroup $\{(1,b^{j})\}$.
       \item Every element of $G$ factors uniquely as $(a^{i},1)(1,b^{j})$,
             so $|G|=7\cdot9=63$.
       \item The conjugation rule $b\,a\,b^{-1}=a^{2}\ne a$
             shows $G$ is \emph{not} abelian; hence we genuinely obtain the
             non-trivial semidirect product prescribed by~$\varphi$.
    \end{itemize}
    \end{explanation}
    \begin{theorem}
      For every prime $p$, the multiplicative group of units
      \[
         \bigl(\Bbb Z/p\Bbb Z\bigr)^{\times}
           \;=\;
         \{\,\overline{1},\overline{2},\ldots,\overline{p-1}\}
      \]
      is (i) cyclic of order $p-1$ and hence
      (ii) isomorphic to the additive cyclic group
      $\Bbb Z_{p-1}=\Bbb Z/(p-1)\Bbb Z$.
      \end{theorem}
      
      \begin{proof}[Proof in four steps]
      \textbf{1.\;Order of the group.}
      Because the non–zero residues modulo $p$ are exactly
      $\{\overline{1},\dots,\overline{p-1}\}$,
      \[
         \bigl|(\Bbb Z/p\Bbb Z)^{\times}\bigr|=p-1.
      \]
      
      \smallskip
      \textbf{2.\;Choose an element of maximal order.}
      Since the group is finite, there exists some element
      $g\in(\Bbb Z/p\Bbb Z)^{\times}$ whose order
      \[
          m=\operatorname{ord}(g)
      \]
      is \emph{maximal} among all element orders in the group.
      Necessarily $m\mid(p-1)$ by Lagrange’s theorem.
      
      \smallskip
      \textbf{3.\;Show the maximal order is $p-1$ (i.e.\ $g$ is a primitive root).}
      
      \begin{enumerate}[label=(\alph*),leftmargin=*]
         \item
         Every element $a\in(\Bbb Z/p\Bbb Z)^{\times}$ satisfies
         $a^{p-1}=1$ (Fermat’s little theorem), so
         $\operatorname{ord}(a)\mid p-1$.
         \item
         Write $p-1=m\cdot r$.
         Suppose, for contradiction, that $r>1$.
         Then every element $a$ satisfies $a^{m}=1$
         because $a^{m}$ raised to the power $r$
         is $(a^{p-1})=1$.
         Therefore \emph{every} element’s order
         divides $m$, contradicting the maximality of $m$
         unless $m=p-1$.
      \end{enumerate}
      
      Thus $g$ has order $p-1$ and the group is cyclic:
      \[
         (\Bbb Z/p\Bbb Z)^{\times}=\langle g\rangle.
      \]
      
      \smallskip
      \textbf{4.\;Explicit isomorphism to $\Bbb Z_{p-1}$.}
      Define
      \[
         \Phi:\Bbb Z_{p-1}\;\longrightarrow\;(\Bbb Z/p\Bbb Z)^{\times},
         \qquad
         \Phi\bigl(\,\overline{k}\,\bigr)=g^{\,k}.
      \]
      \begin{itemize}
         \item $\Phi$ is well defined because $g^{\,k+p-1}=g^{\,k}$.
         \item $\Phi$ is a homomorphism:
               $\Phi(\overline{k_1}+\overline{k_2})=g^{k_1+k_2}
               =g^{k_1}\,g^{k_2}
               =\Phi(\overline{k_1})\,\Phi(\overline{k_2})$.
         \item $\Phi$ is injective (kernel is $\{\overline{0}\}$ since
               $g$ has full order $p-1$) and therefore, by cardinality,
               bijective.
      \end{itemize}
      Hence $\Phi$ is an isomorphism, establishing
      \[
         (\Bbb Z/p\Bbb Z)^{\times}\;\cong\;\Bbb Z_{p-1}.
      \]
      \end{proof}
      
      \begin{remark}
      An element $g\in(\Bbb Z/p\Bbb Z)^{\times}$ that generates the whole
      group is called a \emph{primitive root modulo $p$}.
      The existence of such a root for every prime~$p$ is precisely
      the statement proved above.
      \end{remark}
      \begin{align*}
        \bigl(a^{i},b^{j}\bigr)\bigl(a^{u},b^{v}\bigr)
            &=\Bigl(a^{i}\,\varphi\bigl(b^{j}\bigr)\!\bigl(a^{u}\bigr),\,b^{j+v}\Bigr) \\[4pt]
            &=\Bigl(a^{i}\,\alpha^{\,j}\!\bigl(a^{u}\bigr),\,b^{j+v}\Bigr)
               \qquad\bigl(\text{because }\varphi(b)=\alpha\bigr) \\[4pt]
            &=\Bigl(a^{i}\,a^{\,2^{\,j}u},\,b^{j+v}\Bigr)     % \alpha^j raises exponents by 2^j
               =\bigl(a^{\,i+2^{\,j}u},\,b^{\,j+v}\bigr).
        \end{align*}
        
        Here $\alpha$ is the automorphism of $N=\langle a\rangle\cong\Bbb Z_{7}$
        defined by $\alpha(a)=a^{2}$, so
        $\alpha^{\,j}(a^{u})=a^{\,2^{\,j}u}$.
        
        \bigskip
        \textbf{Conjugating $(a,1)$ by $(1,b)$.}
        
        \[
           (1,b)(a,1)(1,b)^{-1}
           =\bigl(1,b\bigr)\bigl(a,1\bigr)\bigl(1,b^{-1}\bigr),
        \]
        because the inverse of $(n,h)$ in a semidirect product is
        \(
           (n,h)^{-1}=\bigl(\varphi(h^{-1})(n^{-1}),h^{-1}\bigr).
        \)
        
        \begin{align*}
        (1,b)(a,1)
           &=\Bigl(1\;\varphi(b)(a),\,b\cdot1\Bigr)
             =\bigl(\alpha(a),\,b\bigr), \\[6pt]
        \bigl(\alpha(a),b\bigr)(1,b^{-1})
           &=\Bigl(\alpha(a)\;\varphi(b)(1),\,b\,b^{-1}\Bigr)
             =\bigl(\alpha(a),1\bigr)
             =\bigl(a^{2},1\bigr).
        \end{align*}
        
        Thus the conjugation relation in $G=N\rtimes_{\!\varphi}H$ is
        \[
           (1,b)(a,1)(1,b)^{-1}=(a^{2},1),
        \]
        which is exactly the relation $\,bab^{-1}=a^{2}\,$ in the usual
        presentation
        \(
           G=\langle a,b\mid a^{7}=b^{9}=1,\;bab^{-1}=a^{2}\rangle.
        \)
        \begin{explanation}
          \textbf{Why pick $\alpha(a)=a^{2}$ and why does it have order $3$?}
          
          \smallskip
          \emph{1.  The ambient group of automorphisms.}
          For a cyclic group $N=\langle a\rangle\cong\Bbb Z_{7}$ every automorphism
          is determined by where it sends the generator~$a$.
          Because
          \[
             \Aut(N)\;\cong\;\Bbb Z_{7}^{\times}
               =\{1,2,3,4,5,6\}
          \]
          under the correspondence
          \[
             k\;\longleftrightarrow\;
             \bigl(a\longmapsto a^{k}\bigr),
          \]
          the \emph{order} of such an automorphism is the multiplicative order of
          $k$ modulo~$7$.
          
          \smallskip
          \emph{2.  We need an automorphism of order \(\mathbf{3}\).}
          The map $\varphi:H\to\Aut(N)$ must have image whose order divides BOTH
          $|H|=9$ and $|\Aut(N)|=6$.
          Hence the image can only be $\{1\}$ or a subgroup of order~$3$.
          To obtain a \textbf{non-trivial} semidirect product we want the latter,
          so we must pick a $k\in\{1,\dots,6\}$ whose multiplicative order
          modulo~$7$ is exactly~$3$.
          
          \smallskip
          \emph{3.  Which $k$ work?}
          Compute powers $\bmod 7$:
          \[
             2^{1}=2,\;2^{2}=4,\;2^{3}=8\equiv1,
          \qquad
             4^{1}=4,\;4^{2}=16\equiv2,\;4^{3}=8\equiv1.
          \]
          Thus \(k=2\) and \(k=4\) have order~$3$ in $\Bbb Z_{7}^{\times}$.
          We arbitrarily choose \(k=2\), giving the automorphism
          \[
             \alpha:N\longrightarrow N,
             \qquad
             \alpha(a)=a^{2}.
          \]
          
          \smallskip
          \emph{4.  Clarifying the ``$2^{3}=8$’’ versus ``$(a^{2})^{3}=a^{6}$’’ issue.}
          
          \begin{itemize}
             \item The relation $2^{3}=8\equiv1\pmod7$ is used
                   to verify \emph{the \underline{order of the automorphism}}, i.e.
                   \[
                       \alpha^{3}(a)
                         \;=\;
                       a^{\,2^{3}}
                         \;=\;
                       a^{\,8}
                         =a.
                   \]
                   Hence \(\alpha^{3}=\operatorname{id}_{N}\), so $\alpha$ indeed has
                   order~$3$.
          
             \item When you compute \((a^{2})^{3}=a^{6}\), you are instead raising
                   \emph{the \underline{group element} $a^{2}$} to the $3$rd power
                   \emph{inside~$N$}.
                   That has nothing to do with the order of the automorphism
                   $\alpha$; it simply uses the rule \( (a^{r})^{s}=a^{rs}\).
          
             \item In contrast, applying the automorphism repeatedly multiplies
                   the \emph{exponent} by successive powers of~$2$:
                   \[
                      \alpha^{2}(a)=\alpha\!\bigl(\alpha(a)\bigr)=\alpha\!\bigl(a^{2}\bigr)
                         =(a^{2})^{2}=a^{4},\quad
                      \alpha^{3}(a)=\alpha\!\bigl(a^{4}\bigr)=a^{8}=a.
                   \]
                   The key point is that the exponents \emph{compose
                   multiplicatively} (\(2^{2},2^{3},\ldots\)) rather than additively.
          \end{itemize}
          
          \smallskip
          \emph{5.  Any $k$ of order $3$ would work.}
          Choosing $k=4$ would give another non-abelian group
          \(
             G=\langle a,b\mid a^{7}=b^{9}=1,\;bab^{-1}=a^{4}\rangle,
          \)
          isomorphic to the one with $k=2$ (indeed, $\,4\equiv2^{-1}\pmod7$), but
          $k=1$ (the identity automorphism) would make the semidirect product a
          direct product and hence abelian.
          
          \bigskip
          \textbf{Summary.}  
          We pick $\alpha(a)=a^{2}$ because \(2\) has order~$3$ in
          $\Bbb Z_{7}^{\times}$, giving an automorphism of order~$3$.  
          Applying $\alpha$ three times multiplies the exponent of~$a$ by
          \(2^{3}=8\equiv1\), so $\alpha^{3}=\mathrm{id}$,
          whereas \((a^{2})^{3}=a^{6}\) is simply the cube of a \emph{group
          element} inside~$N$.
          \end{explanation}
          \begin{problem}
            Let $F$ be the (unique) field with $16$ elements.
            \begin{enumerate}[]
               \item Show that $F$ contains \emph{exactly three} subfields and
                     describe each one explicitly.
               \item How many elements of order $15$ does $F^{\times}$ have?
            \end{enumerate}
            \end{problem}
            
            \begin{solution}
            Throughout we view $F$ as the finite field $\Bbb F_{16}$ of
            characteristic $2$.
            
            \bigskip
            \noindent\textbf{(a)  The subfields of $\Bbb F_{16}$.}
            
            \begin{enumerate}[]
               \item \emph{Possible sizes.}\\
                     Any subfield $K$ of $\Bbb F_{16}$ must itself be a finite
                     field of characteristic $2$, hence has order
                     $|K| = 2^{\,m}$ for some $m\mid4$.
                     The divisors of $4$ are $1,2,4$, so the only candidate sizes
                     are
                     \[
                        2^{1}=2,\qquad
                        2^{2}=4,\qquad
                        2^{4}=16.
                     \]
               \item \emph{Existence and uniqueness for each size.}\\
                     For each $m\mid4$ there is \emph{exactly one}
                     subfield of order $2^{m}$:
                     \[
                        \Bbb F_{2}\subset\Bbb F_{4}\subset\Bbb F_{16}.
                     \]
                     This follows from the Galois–correspondence
                     for the (Galois) extension $\Bbb F_{16}/\Bbb F_{2}$ and the fact
                     that there is a \emph{unique} finite field of a given order.
               \item \emph{Explicit descriptions.}\\[-3pt]
                     \begin{itemize}
                        \item $\displaystyle\Bbb F_{2}
                               =\{\,0,1\,\}.$
                        \item Choose an element
                              $\beta\in\Bbb F_{16}$ satisfying
                              $\beta^{2}+\beta+1=0$.
                              Then
                              \[
                                 \Bbb F_{4}=\Bbb F_{2}(\beta)
                                   =\{\,0,1,\beta,\beta+1\,\},
                                   \qquad
                                 \beta^{3}=1.
                              \]
                        \item $\displaystyle\Bbb F_{16}$ itself is obtained by adjoining
                              a root $\alpha$ of some irreducible quartic,
                              e.g.\ $\alpha^{4}+\alpha+1=0$:
                              \[
                                 \Bbb F_{16}=\Bbb F_{2}(\alpha)
                                 \quad\bigl(|\Bbb F_{16}|=16\bigr).
                              \]
                     \end{itemize}
            \end{enumerate}
            
            Hence $\Bbb F_{16}$ has \emph{exactly three} subfields:
            $\Bbb F_{2}$, $\Bbb F_{4}$ and $\Bbb F_{16}$.
            
            \bigskip
            \noindent\textbf{(b)  Elements of order $15$ in $F^{\times}$.}
            
            The multiplicative group
            \[
               F^{\times}=\Bbb F_{16}^{\times}
            \]
            is cyclic of order $16-1=15$.  
            In a finite cyclic group of order $n$, the number of elements of exact
            order $n$ equals Euler’s totient
            \(
               \varphi(n).
            \)
            Therefore
            \[
               \#\{\,x\in F^{\times}\mid\operatorname{ord}(x)=15\,\}
                  =\varphi(15)
                  =\varphi(3)\,\varphi(5)
                  =(3-1)(5-1)
                  =2\cdot4
                  =8.
            \]
            
            \paragraph{Answer.}
            There are $\boxed{8}$ elements of order $15$ in $F^{\times}$.
            \end{solution}
\end{document}
