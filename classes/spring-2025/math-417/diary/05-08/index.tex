\documentclass[12pt]{article}

\usepackage[margin=1in]{geometry}
\usepackage{amsmath,amssymb,amsthm}
\usepackage{enumitem}
\usepackage{xcolor}
\usepackage{graphicx}
\usepackage{tikz}
\usepackage{listings}
\usepackage{physics}
\usepackage{siunitx}
\usepackage{booktabs}
\usepackage{cancel}
\usepackage{setspace}
\usepackage{titlesec}
\usepackage{tabularx}
\usepackage[utf8]{inputenc}
\usepackage[T1]{fontenc}
\usepackage{textcomp}
\linespread{1.5}


% hyperref should be LAST
\usepackage[colorlinks=true, linkcolor=blue, urlcolor=blue, citecolor=blue]{hyperref}

  \usetikzlibrary{calc,patterns,arrows.meta,decorations.markings}


\DeclareMathOperator{\Log}{Log}
\DeclareMathOperator{\Arg}{Arg}

\lstset{
    breaklines=true,         % Enable line wrapping
    breakatwhitespace=false, % Wrap lines even if there's no whitespace
    basicstyle=\ttfamily,    % Use monospaced font
    frame=single,            % Add a frame around the code
    columns=fullflexible,    % Better handling of variable-width fonts
}

\newcommand{\incfig}[1]{%
    \def\svgwidth{\columnwidth}
    \import{./Figures/}{#1.pdf_tex}
}
\theoremstyle{definition} % This style uses normal (non-italicized) text
\newtheorem{solution}{Solution}
\newtheorem{proposition}{Proposition}
\newtheorem{problem}{Problem}
\newtheorem{lemma}{Lemma}
\newtheorem{theorem}{Theorem}
\newtheorem{remark}{Remark}
\newtheorem{note}{Note}
\newtheorem{definition}{Definition}
\newtheorem{example}{Example}
\newtheorem{corollary}{Corollary}
\theoremstyle{plain} % Restore the default style for other theorem environments
%

% Theorem-like environments
% Title information
\title{}
% !TEX program = pdflatex

% --- packages you’ll need ---

\geometry{paperwidth=8.5in,paperheight=11in,margin=1in}

\begin{document}
\begin{titlepage}
  \centering
  \vspace*{-3cm}                                   % push image up a bit
  \includegraphics[width=0.75\textwidth]{tailmate_render.png}\par
  \vspace{1cm}

  {\Huge\bfseries Tail Mate\par}
  \vspace{0.4cm}
  {\Large Dog Potty Collector\par}

  \vfill                                            % stretch to vertical center

  {\large\scshape ME170 Design Team\par}
  \vspace{0.2cm}
  {\large AB5-2\par}

  \vspace{0.8cm}

  {\Large Team Members\par}
  \vspace{0.3cm}
  \begin{tabular}{c}
    Trayvonn Foster\\
    Jennifer Ren\\
    Navya Nair\\
    Sophie Li
  \end{tabular}

  \vfill

  {\large Date: 05/09/2025\par}
\end{titlepage}

\tableofcontents
\pagebreak
\section{Introduction / Product Description}
\vspace{.5cm}
Our project began with a focus on college students and young adults living in dorms
or shared apartments who own dogs. These individuals often face unique challenges
in maintaining consistent pet care due to limited time, space, and budget. To
understand their needs, we followed the Human-Centered Design (HCD) process
and conducted interviews aimed at uncovering specific pain points in their daily
routines.

\vspace{.5cm}
We asked questions such as:
\vspace{.5cm} 
\noindent
\begin{enumerate}
  \item "Walk me through a typical weekday with your pet."
\item "Which parts of your pet-care routine feel most time-consuming or tricky?"
\item "How does your living space influence how you store pet gear or set up
potty/play areas?"

\item
"When shopping for pet products, what do you prioritize—price, size,
features?"

\item
"If you could improve one thing about your routine, what would it be and
why?"
\end{enumerate}
\vspace{.5cm} 
From our interviews, we identified recurring issues:
\vspace{.5cm} 
\noindent
\begin{enumerate}
  \item  Rushing down stairs for every potty break disrupted schedules.
 \item Small or shared living spaces made it hard to store bulky pet equipment
\item Budget constraints limited access to helpful tools.

We translated these insights into “How Might We” (HMW) questions to drive ideation:


\end{enumerate}
\noindent
\begin{enumerate}
  \item How might we reduce the time and effort needed for potty breaks?

\item How might we provide hygienic solutions that fit small spaces and budgets?


\end{enumerate}
\vspace{.5cm} 
These questions led us to create \textbf{TailMate} — a small, lightweight clip that attaches to
a dog’s tail and holds a standard poop bag to catch waste right away. This
hands-free solution removes the need for pet owners to bend down or use their
hands to clean up, making the process much faster and more hygienic. Users just
clip it on, insert a bag, and let the product do the rest.

Though we originally designed TailMate for busy students, we soon realized it could
help many other groups as well — especially elderly pet owners or those with limited
mobility. For them, picking up waste the traditional way can be painful or even
impossible. TailMate allows these users to care for their pets more independently,
with less physical strain and discomfort.

TailMate is also \textbf{affordable and flexible}. Unlike similar products that require users to
buy expensive custom bags, our design works with any standard poop bag — the
kind most pet owners already buy. This not only keeps costs low but also saves
users from needing to reorder special supplies. The compact clip is easy to store in
small living spaces like dorms or apartments, and it's quick to use on the go.

What makes TailMate stand out is its \textbf{simplicity, compatibility, and inclusive
design.} There’s no need for bulky tools or complicated gadgets — just a small,
comfortable clip that helps all kinds of pet owners handle an everyday task more
easily. Whether you're a student rushing to class, a senior with back pain, or
someone who just wants a cleaner way to clean up, TailMate makes pet care less of
a hassle.
\pagebreak

\section{Concept Sketches Initial to Finalized:}
The development of our product began with a series of conceptual sketches
exploring different mechanisms to solve a simple yet specific problem: how to hold
open a dog waste bag securely and hands-free during cleanup. Our goal was to
create a lightweight, non-invasive attachment that would improve convenience for
dog owners without compromising the dog’s comfort or safety.

In the early stages, our design ideas varied significantly. Some sketches
focused on rigid clamp-style mechanisms that would grip the bag from the outside,
while others explored wearable devices attached to the leash or the user’s wrist. We
used ideation cards in order to get different variations in designs and create the most
efficient part to complete the task. As we evaluated these concepts, we prioritized
simplicity, manufacturability, and ease of use in real-world scenarios. 
Comfort for the dog was also a top concern, which influenced every design decision moving forward.

Our design process began with individual ideation and sketching to encourage
a wide range of creative solutions. Each of the four team members was responsible
for generating five unique sketches of the same product concept, using ideation
cards to alter the design approach and stimulate divergent thinking. These cards
introduced constraints or prompts—such as modifying materials, altering user
interaction, or reimagining form—that pushed us to think beyond initial assumptions.

Once all twenty sketches (five per team member) were completed, each
individual reviewed their own concepts and selected the most promising design from
their set. These four selected concepts were then brought forward as finalists and
evaluated using a Pugh matrix, where we compared them against weighted criteria
such as functionality, comfort, manufacturability, and feasibility.

Below are all of the initial concept sketches, each accompanied by a brief
description. The designs that were selected to move forward into the final
design phase are outlined in red:
\pagebreak
\subsubsection*{Person \#1 Initial Sketches:}
Basically a belt around the body of the dog. And we
use buckle to open and close the belt. And there are three sub-belts from different
sides of the body, making sure the whole belt sticks to the dog. We add three hooks
at the end of the strings, and we attach the bag using the hooks.
\begin{figure}[htbp]
  \centering
  \includegraphics[width=0.3\textwidth]{2.1.png}
  \caption{}
  \label{fig:}
\end{figure}
\subsubsection*{Person \#2 Initial Sketches:}
This innovative device is designed for easy attachment
to a dog's tail to streamline the cleanup of waste. It features a soft, bendable tail clip
that ensures comfort for the dog. This clip connects to a disposable bag equipped
with a liner, making the entire setup completely disposable. The product is
particularly beneficial for elderly users or dogs that cannot produce solid waste,
simplifying waste management and keeping backyards clean. It's an ideal solution
for maintaining hygiene with minimal effort.
\begin{figure}[htbp]
  \centering
  \includegraphics[width=0.3\textwidth]{2.2.png}
  \caption{}
  \label{fig:}
\end{figure}
\pagebreak
\subsubsection*{Person \#3 Initial Sketches:}
A plastic clip is clipped around the dogs' tail and
fastened by a buckle. The buckle can be tightened more or less depending on the
dog's tail size (think of a fanny pack belt buckle). Inside the clip is a roll of plastic
bags that can be pulled out of the clip when needed and hang directly out of the clip
and under the dog's bottom. We need to consider how bags will be refilled into the
clip and how to put the bag in if pulled out too much.
\begin{figure}[htbp]
  \centering
  \includegraphics[width=0.4\textwidth]{2.3.png}
  \caption{}
  \label{fig:}
\end{figure}

\subsubsection*{Person \#4 Initial Sketches:}
An adjustable belt that attaches to a plastic bag for
collecting your pet's waste, which is perfect for indoors. First, clip the belt around the
torso of the dog, attach the ends of the three straps to the ends of the waste bag,
and tighten the adjustable straps to bring the bag to the rear of the dog. After that,
you're ready to go!
\begin{figure}[htbp]
  \centering
  \includegraphics[width=0.4\textwidth]{2.4.png}
  \caption{}
  \label{fig:}
\end{figure}

\subsubsection*{Transition to the Final Design}
After reviewing our initial concepts, we identified a recurring design
opportunity: leveraging the dog’s tail as a passive support structure. This inspired a
shift toward tail-mounted mechanisms, which would allow the product to hold the
waste bag open at the source while remaining unobtrusive.

From this insight, we iteratively refined our sketches toward a clip-based
system that could securely fasten to a dog’s tail without causing discomfort. The
solution took shape as a putty clip—a two-part injection-molded component using a
slotted spring pin hinge, with a flexible interface to gently clamp onto the tail. The clip
features a secure press fit for the pin on one side and a rotating clearance fit on the
other, allowing for smooth opening and closing of the bag holder arms.

This final design balances durability, comfort, and usability. It emerged
through deliberate exploration of form factors, analysis of mechanical constraints,
and a deep consideration of user and animal experience. The progression from early
rough concepts to a purpose-built clip demonstrates our iterative approach and
commitment to functional, user-centered design.

\subsubsection*{This was also systematically decided using this Pugh Matrix:}

We chose this as our datum because it's
a product that already exists and has
multiple good reviews. This product has
already been used by some customers
and has worked for some of them so we
know this is an idea that is both feasible
and functional.

\begin{table}[h]
  \centering
  \small                       % (optional) slightly smaller text
  \setlength\tabcolsep{4pt}    % default is 6 pt – tighten it a bit
  \begin{tabular*}{\textwidth}{@{\extracolsep{\fill}} lccccc}
      \toprule
      \textbf{Criteria} & \textbf{Datum} &
      \textbf{Concept 1} & \textbf{Concept 2} &
      \textbf{Concept 3} & \textbf{Concept 4}\\
      \midrule
      Ergonomics                  &        & $-$ & $+$ & $-$ & $-$\\
      Safety                      &        & $-$ & $+$ & $-$ & $-$\\
      Comfort                     & D      & $-$ & $-$ & $-$ & $-$\\
      Materials                   & A      & $+$ & $-$ & $+$ & $+$\\
      Product Life Span           & T      & S   & $-$ & $+$ & $-$\\
      Performance                 & U      & $-$ & $+$ & S   & $-$\\
      Size \& Weight              & M      & $+$ & $+$ & $-$ & $+$\\
      Environmental Consideration &        & S   & S   & S   & $-$\\
      Target Product Cost         &        & S   & S   & $-$ & $+$\\
      Maintenance                 &        & S   & $+$ & $-$ & $-$\\
      \midrule
      \textbf{Total “+”}          &        & 2   & 5   & 2   & 3\\
      \textbf{Total “–”}          &        & 4   & 4   & 6   & 7\\
      \textbf{Net Score}          &        & $-2$& $+1$& $-4$& $-4$\\
      \bottomrule
  \end{tabular*}
  \caption{Decision matrix comparison of dog-waste collection concepts
           (S = same as datum, “+” = better, “–” = worse).}
\end{table}

After evaluating all four selected concepts using a weighted decision matrix, we
decided to move forward with Concept 2. This concept achieved the highest overall
score based on key criteria such as functionality, ease of use, comfort for the dog,
and manufacturability. Its strong performance across multiple categories made it the
most well-rounded and feasible option for further development.

\pagebreak
\section{Product Design Specifications}

\subsubsection*{Description}

This product features a clip that attaches to a dog's tail, designed to streamline the
waste collection process. When the dog relieves itself, the waste is directly captured
into an attached bag, eliminating the need for manual cleanup. This is particularly
useful for dealing with non-solid waste or for owners who prefer a more hands-off
approach to waste management. Simply attach the clip before letting your dog
outside, and upon their return, remove and dispose of the bag. This system provides
a convenient and hygienic solution for managing pet waste.

\subsubsection*{29 Primary Elements}
\noindent
\begin{enumerate}
  \item Performance: The clip must securely attach to a dog's tail and hold the waste
  bag open during use.
  \item Environment: Designed for outdoor use in various weather conditions,
  should be waterproof.
  \item Service Life: Should last for at least 5 years of regular use without breaking
  or losing function.
  \item Maintenance: Minimal; users may rinse the product with water if it becomes
  dirty.
  \item Targets Costs: $10 to $15 dollar selling price and $1 to $5 dollar production
  cost
  \item Competition: Competes with hands-free dog waste tools and traditional bag
  dispensers by allowing users to attach their own bags.
  \item Shipping: Lightweight and compact enough for low-cost shipping in bulk.
  \item Product Volume(quantity): Initially targeted for large-scale production and
  global-scale testing with 1 million products being made initially.
  \item Packing: Simple plastic or cardboard packaging with basic instructions.
  \item Manufacturing Facility: The clip must be produced in a facility with
  cleanroom manufacturing standards; quality control inspections should be
  conducted.
  \item Size: Designed for plastic injection molding at small to medium manufacturing
  facilities.
  \item Weight: Compact—should not exceed 100 mm in any dimension.
  \item Aesthetics and Finish: Should be visibly appealing and have rounded edges
  for a sleek design look
  \item Materials: Should be hard plastic for structure and soft for interaction with
  dog.
  \item Product Life Span: Should last for at least 5 years of regular use without
  breaking or losing function.
  \item Standards, Specifications and Legal Aspects: Must comply with basic
  consumer safety and pet product guidelines.
  \item Ergonomics: Must fit the dog well and be very user-friendly with the human
  interaction (taking the bag off and putting on the clip)
  \item Customer: Pet owners seeking a cleaner, hands-free waste collection
  method. Older users who can’t reach down and collect poop. Old dogs who
  have loose stool.
  \item Quality and Reliability: Should function consistently and withstand regular
  use without failure.
  \item Shelf Life: At least 2 years in storage without material degradation.
  \item Processes: Designed for injection molding, simple pin assembly for hinge.
  \item Timescales: All designs and reports should be done before Friday May 9th
  \item Testing: No testing required
  \item Safety: Must not harm the dog’s tail or cause discomfort; edges should be
  smooth.
  \item Company Constraints: Designed within the scope of a first-year university
  design project.
  \item Market Constraints: Limited initial market awareness; targeted toward niche
  or novelty pet owners.
  \item Patents, Literature and Product Data: Reviewed similar products to ensure
  originality; no known patent conflicts.
  \item Political and Social Implications: Promotes responsible pet ownership and
  cleaner public spaces.
  \item Disposal: Recyclable plastic components where local facilities allow. 
\end{enumerate}
\pagebreak

\section{CAD Assembly Models}
\subsubsection*{Full Assembly Model}
This is the complete Tail Mate assembly, displaying all components in their
assembled configuration.
\begin{figure}[htbp]
  \centering
  \includegraphics[width=0.5\textwidth]{170-4.1.png}
  \caption{}
  \label{fig:}
\end{figure}
\subsubsection*{Cross Section of Shaft with Torsion Spring}
This view demonstrates the
connection between the torsion spring
dowel and the top and bottom clips,
showing how they assemble into a
functional hinge.
\begin{figure}[htbp]
  \centering
  \includegraphics[width=0.5\textwidth]{170-4.2.png}
  \caption{}
  \label{fig:}
\end{figure}
\pagebreak
\subsubsection*{Bag Connector Section View}
This view illustrates how the bag connector interfaces with the bottom clip.
\begin{figure}[htbp]
  \centering
  \includegraphics[width=0.4\textwidth]{170-4.3.png}
  \caption{}
  \label{fig:}
\end{figure}
\subsubsection*{Spring Locking Mechanism Section View}
This view shows the interface between the spring and the bag connector,
illustrating how the locking mechanism functions with both springs attached.
\begin{figure}[htbp]
  \centering
  \includegraphics[width=0.4\textwidth]{170-4.4.png}
  \caption{}
  \label{fig:}
\end{figure}
\pagebreak
\subsubsection*{Exploded View of all components}
\begin{figure}[htbp]
  \centering
  \includegraphics[width=0.3\textwidth]{170-4.5.png}
  \caption{}
  \label{fig:}
\end{figure}
\pagebreak

\section{Exploded Assembly with BOM}
\subsubsection*{Exploded Assembly View}
\begin{figure}[htbp]
  \centering
  \includegraphics[width=0.8\textwidth]{170-5.1.png}
  \caption{}
  \label{fig:}
\end{figure}
\pagebreak
\section{Cross-section assembly drawing}
\begin{figure}[htbp]
  \centering
  \includegraphics[width=0.8\textwidth]{170-6.1.png}
  \caption{}
  \label{fig:}
\end{figure}
\pagebreak

\section{Detailed Engineering Drawings}
\subsubsection*{Bottom Clip Drawings}
\begin{figure}[htbp]
  \centering
  \includegraphics[width=0.5\textwidth]{7.1.png}
  \caption{Page 1}
  \label{fig:}
\end{figure}
\begin{figure}[htbp]
  \centering
  \includegraphics[width=0.5\textwidth]{7.2.png}
  \caption{Page 2}
  \label{fig:}
\end{figure}
\pagebreak
\subsubsection*{Top Clip Drawings}
\begin{figure}[htbp]
  \centering
  \includegraphics[width=0.5\textwidth]{7.3.png}
  \caption{Page 1}
  \label{fig:}
\end{figure}

\begin{figure}[htbp]
  \centering
  \includegraphics[width=0.5\textwidth]{7.4.png}
  \caption{Page 2}
  \label{fig:}
\end{figure}

\pagebreak
\subsubsection*{Spring Case Drawing}
\begin{figure}[htbp]
  \centering
  \includegraphics[width=0.8\textwidth]{7.5.png}
  \caption{}
  \label{fig:}
\end{figure}

\pagebreak
\subsubsection*{Spring Pin Drawing}
\begin{figure}[htbp]
  \centering
  \includegraphics[width=0.8\textwidth]{7.6.png}
  \caption{}
  \label{fig:}
\end{figure}

\pagebreak
\subsubsection*{Flex Insert Drawing}
\begin{figure}[htbp]
  \centering
  \includegraphics[width=0.8\textwidth]{7.7.png}
  \caption{}
  \label{fig:}
\end{figure}

\pagebreak
\subsubsection*{Baggie Sealer Drawing}
\begin{figure}[htbp]
  \centering
  \includegraphics[width=0.8\textwidth]{7.8.png}
  \caption{}
  \label{fig:}
\end{figure}

\pagebreak
\subsubsection*{Baggie Connector Drawing}
\begin{figure}[htbp]
  \centering
  \includegraphics[width=0.8\textwidth]{7.9.png}
  \caption{}
  \label{fig:}
\end{figure}
\pagebreak

\section{Radial and Axial Fit Tolerance Analysis – Slotted Spring Pin in Top
and Bottom Clip System}

This section presents a tolerance analysis of the fits used in a slotted spring pin
assembly within an injection-molded clip mechanism. The pin is press-fit into one
component, preventing axial movement, and allows rotation within another
component. The analysis classifies fits based on their functional constraints—radial
(rotation) and axial (linear retention)—and uses worst-case tolerancing to validate fit
performance under manufacturing variation.

\subsubsection*{Component Overview}
\noindent
\begin{enumerate}
  \item \textbf{Material}: Injection-molded Polypropylene
  (medium impact copolymer)
  \begin{figure}[htbp]
    \centering
    \includegraphics[width=0.5\textwidth]{81.png}
    \caption{}
    \label{fig:}
  \end{figure}
  \item \textbf{Slotted Spring Pin}: $3/32$" nominal diameter
  ($2.38125$ mm), per ASME B$18.8.2$ 

  \item \textbf{Axial Fit: Fixed Hole on Bottom Clip
  (Interference fit/press fit)}: $\varnothing 2.3368 \pm$ mm
  \begin{figure}[htbp]
    \centering
    \includegraphics[width=0.3\textwidth]{82.png}
    \caption{}
    \label{fig:}
  \end{figure}
  \pagebreak
  \item \textbf{Radial Fit: Rotating Hole on Top Clip
  (clearance fit)}: $\varnothing 2.4892 \pm 0.0254$ mm  
  \begin{figure}[htbp]
    \centering
    \includegraphics[width=0.5\textwidth]{83.png}
    \caption{}
    \label{fig:}
  \end{figure}
\end{enumerate}
\subsubsection*{Axial Fit Info}:
\textbf{Interference Fit - Fixed Hole (Top Clip)}:
\begin{figure}[htbp]
  \centering
  \includegraphics[width=0.5\textwidth]{84.png}
  \caption{}
  \label{fig:}
\end{figure}
The fixed hole provides axial retention for the spring pin. The selected tolerance
results in an interference fit based on the nominal pin diameter.
\noindent
\begin{enumerate}
  \item \textbf{Hole size range:}  2.3114 mm(minimum) to 2.3622mm (maximum)
\item
\textbf{Pin diameter (nominal)} : 2.38125 mm
\end{enumerate}
\textbf{Worst-Case Interference Calculations:} 
\vspace{.5cm} 
\noindent
\begin{enumerate}
  \item \textbf{Maximum Interference}  = 2.38125 - 2.3114 = 0.06985 mm
  \item
  \textbf{Minimum Interference}  = 2.38125 - 2.3622 = 0.01905 mm
\end{enumerate}
\vspace{.5cm} 
\textbf{Fit Type Equivalent} : ISO \textbf{H7/s6}  (locational interference fit)
The interference is designed to be manageable by hand during assembly due to the
spring pin's compressibility and the compliance of polypropylene. This ensures that
the pin stays fixed axially and provides consistent alignment between components.
\pagebreak
\subsubsection*{Radial Fit Info}:
\textbf{Clearance Fit – Rotating
Hole(Bottom Clip)}: The rotating hole allows the second
component to pivot around the spring
pin. A clearance fit is applied to ensure
smooth motion.
\begin{figure}[htbp]
  \centering
  \includegraphics[width=0.5\textwidth]{85.png}
  \caption{}
  \label{fig:}
\end{figure}
\noindent
\begin{enumerate}
  \item \textbf{Hole size range} : 2.4638 mm (minimum) to 2.5146 mm (maximum)
  \item \textbf{Pin diameter}  (nominal): 2.38125 mm
\end{enumerate}

\textbf{Worst-Case Clearance Calculations}:
\vspace{.5cm} 
\noindent
\begin{enumerate}
  \item \textbf{Maximum Clearance}  = 2.5146 - 2.38125 = 0.13335 mm
  
 \item \textbf{Minimum Clearance}  =2.4638 - 2.38125 = 0.08255 mm
\end{enumerate} 
\textbf{Fit Type Equivalent} : ISO \textbf{H7/g6} (running fit)
This clearance range ensures reliable rotation of the component without jamming,
even when accounting for molding variation or slight warping. The fit is sufficiently
tight to maintain positional stability while allowing smooth, repeatable rotation.

\textbf{Conclusion}:
The selected tolerances for the fixed and rotating holes produce appropriate
interference and clearance fits for the slotted spring pin. The worst-case analysis
confirms that:
\noindent
\begin{enumerate}
  \item The \textbf{fixed hole} retains the pin with up to \textbf{0.06985 mm}  interference, ensuring
  stable axial positioning.
  \item 
  The \textbf{rotating hole}  provides \textbf{0.08255-0.13335 mm} clearance, enabling free
  rotation.
\end{enumerate}
Together, these fits support the intended mechanical function of the clip system by
balancing secure retention with smooth rotational movement. During repeated
use—such as opening and closing the Tail Mate on a dog's tail—it is critical that the
hinge mechanism maintains durability without loosening or detaching over time. By
selecting these specific tolerances, we ensure the product remains structurally sound
and functions reliably under normal and worst-case conditions. The analysis also
confirms that the tolerances are suitable for injection molding, allow for hand
assembly, and provide consistent mechanical performance.
\pagebreak
\section{Materials and Manufacturing With Bill of Materials}

\begin{table}[h]
  \centering
  \scriptsize
  \renewcommand{\arraystretch}{1.1}
  \setlength\tabcolsep{2pt}

  \begin{tabular}{l l p{4cm} c c c r}
    \toprule
    Part \# & Description &
    \shortstack[l]{Material and Manufacturing Method (or order details if an\\off-the-shelf item)} &
    Part Cost & Quantity Needed & Total Part Cost & Investment Costs \\
    \midrule
    Part 1 & Bottom Bag Liner & Plastic Molding - Polypropylene Med Impact Copol: Injection Molding & \$0.01 & 1 & \$0.02 & \$5,553.90 \\
    Part 2 & Bottom Clip & Plastic Molding - Polypropylene Med Impact Copol: Injection Molding & \$0.16 & 1 & \$0.16 & \$9,585.82 \\
    Part 3 & Top Clip & Plastic Molding - Polypropylene Med Impact Copol: Injection Molding & \$0.14 & 1 & \$0.14 & \$7,860.77 \\
    Part 4 & Flexible & Plastic Molding - Closed-Cell Resin Croslite & \$0.11 & 2 & \$0.22 & \$4,843.44 \\
    Part 5 & Spring Case & Plastic Molding - Polypropylene Med Impact Copol: Injection Molding & \$0.04 & 1 & \$0.04 & \$4,349.47 \\
    Part 6 & Spring Pin & Plastic Molding - Polypropylene Med Impact Copol: Injection Molding & \$0.24 & 1 & \$0.24 & \$4,367.95 \\
    Part 7 & Spring Dowel & 18-8 Stainless Steel Slotted Spring Pin 3/32" Diameter, 3/4" Long & \$0.21 & 1 & \$0.21 & \$0 \\
    Part 8 & Compression Spring & PC012-088-6630-MW-0190-C-N-IN & \$0.22 & 1 & \$0.22 & \$0 \\
    Part 9 & Torsion Spring & 0.0006 Inch-Pounds per Degree Rate Music Wire Torsion Spring & \$0.10 & 2 & \$0.10 & \$0 \\
    Part 10 & Top Bag Liner & Plastic Molding - Polypropylene Med Impact Copol: Injection Molding & \$0.04 & 1 & \$0.04 & \$0 \\
    \midrule
    \multicolumn{5}{l}{Totals} & \$1.25 & \$42,526.02 \\
    \bottomrule
  \end{tabular}
  \caption{Cost BOM and Total Cost Calculations.}
\end{table}

The table above outlines the Bill of Materials (BOM) for our final product: It includes
all components required for a single assembled unit, along with material specifications,
manufacturing methods, individual part costs, quantities needed, and associated investment
costs.

Each part is identified with a Part Number and a Description, followed by the material
used and the manufacturing process (primarily injection molding using polypropylene, a
cost-effective and flexible thermoplastic suitable for pet products). Off-the-shelf components
such as springs and dowels are also listed with sourcing details.

The Part Cost reflects the cost to produce or purchase a single unit of each item. The
Quantity Needed shows how many of that item are required in one complete product.
Multiplying these two gives the Total Part Cost for that item in one unit.

The Investment Cost column refers to estimated tooling or setup costs associated
with manufacturing that specific part, primarily related to injection molding.

In total:
\noindent
\begin{enumerate}
  \item The material and part cost per product is \textbf{\$1.25} , making the product cost-effective to
  manufacture in bulk.
\item The total estimated investment cost for mold tooling and setup across all plastic
components is \textbf{\$42,526.02}.
\end{enumerate}

We plan to sell the product at a retail price of \textbf{\$10.99} , which offers a strong profit margin.
After subtracting the \$1.25 unit cost, the gross margin per unit is \$9.74, or roughly 89\%. This
margin not only covers packaging, marketing, distribution, and labor but also allows for
reinvestment into future product development and scaling.
In terms of market positioning, \$10.99 is competitive:

\noindent
\begin{enumerate}
  \item It aligns with prices for other pet accessories and hands-free waste solutions.
  \item It is affordable for the average pet owner, yet perceived as a premium, specialized
  product due to its innovation and convenience.
\end{enumerate}
\vspace{.5cm} 
Overall, the pricing strategy balances affordability for consumers, profitability for the
company, and scalability in production, making it a sustainable and competitive product in
the pet accessory market.
\end{document}
