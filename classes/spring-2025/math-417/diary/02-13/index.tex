\documentclass[12pt]{article}

% Packages
\usepackage[margin=1in]{geometry}
\usepackage{amsmath,amssymb,amsthm}
\usepackage{enumitem}
\usepackage{hyperref}
\usepackage{xcolor}
\usepackage{import}
\usepackage{xifthen}
\usepackage{pdfpages}
\usepackage{transparent}
\usepackage{listings}


\lstset{
    breaklines=true,         % Enable line wrapping
    breakatwhitespace=false, % Wrap lines even if there's no whitespace
    basicstyle=\ttfamily,    % Use monospaced font
    frame=single,            % Add a frame around the code
    columns=fullflexible,    % Better handling of variable-width fonts
}

\newcommand{\incfig}[1]{%
    \def\svgwidth{\columnwidth}
    \import{./Figures/}{#1.pdf_tex}
}
\theoremstyle{definition} % This style uses normal (non-italicized) text
\newtheorem{solution}{Solution}
\newtheorem{proposition}{Proposition}
\newtheorem{problem}{Problem}
\newtheorem{lemma}{Lemma}
\newtheorem{theorem}{Theorem}
\newtheorem{remark}{Remark}
\newtheorem{note}{Note}
\theoremstyle{plain} % Restore the default style for other theorem environments
%

% Theorem-like environments
% Title information
\title{}
\author{Jerich Lee}
\date{\today}

\begin{document}

\maketitle
We want to show why
\[
\lvert G\rvert \;=\; \lvert \operatorname{Im}(f)\rvert\;\lvert \ker(f)\rvert
\]
holds for a finite group $G$ and a group homomorphism $f: G \to K$.

\subsection*{Step 1: First Isomorphism Theorem}
The First Isomorphism Theorem tells us
\[
G \big/ \ker(f) \;\cong\; \operatorname{Im}(f).
\]
Since isomorphic groups have the same number of elements, we immediately get
\[
\lvert G/\ker(f)\rvert \;=\; \lvert \operatorname{Im}(f)\rvert.
\]

\subsection*{Step 2: The size of a quotient group}
From basic group theory, if $H$ is a subgroup of $G$ of finite index, then the index of $H$ in $G$ is defined as
\[
\lvert G : H\rvert \;=\; \frac{\lvert G\rvert}{\lvert H\rvert}.
\]
Moreover, when $H \trianglelefteq G$ is a normal subgroup, the quotient group $G/H$ is well-defined and it has
\[
\lvert G/H\rvert \;=\; \lvert G : H\rvert \;=\; \frac{\lvert G\rvert}{\lvert H\rvert}.
\]
Applying this to $H = \ker(f)$ (which is necessarily normal) gives
\[
\lvert G/\ker(f)\rvert \;=\; \frac{\lvert G\rvert}{\lvert \ker(f)\rvert}.
\]

\subsection*{Combining both facts}
Putting these two observations together:
\[
\lvert G/\ker(f)\rvert
= \lvert \operatorname{Im}(f)\rvert
\quad\text{and}\quad
\lvert G/\ker(f)\rvert 
= \frac{\lvert G\rvert}{\lvert \ker(f)\rvert},
\]
immediately gives us
\[
\frac{\lvert G\rvert}{\lvert \ker(f)\rvert}
\;=\;
\lvert \operatorname{Im}(f)\rvert
\quad\Longrightarrow\quad
\lvert G\rvert 
\;=\; 
\lvert \operatorname{Im}(f)\rvert 
\;\lvert \ker(f)\rvert.
\]

\noindent
\textbf{Claim.} $\operatorname{GL}_n(\mathbb{R}) \big/ \operatorname{SL}_n(\mathbb{R}) \;\cong\; \mathbb{R}^{\times}.$

\subsection*{Step 1: Define the map}
Consider the determinant map
\[
\det:\; \operatorname{GL}_n(\mathbb{R})\;\longrightarrow\;\mathbb{R}^\times,
\]
where $\mathbb{R}^\times$ is the group of all nonzero real numbers under multiplication. Since every matrix in $\operatorname{GL}_n(\mathbb{R})$ is invertible, its determinant is a nonzero real number, so the map is well-defined.

\subsection*{Step 2: Show that $\operatorname{Im}(\det) = \mathbb{R}^\times$}
For any $\alpha \in \mathbb{R}^\times$, we can construct a diagonal matrix 
\[
D =
\begin{pmatrix}
\alpha & 0 & \cdots & 0 \\
0 & 1 & \cdots & 0 \\
\vdots & \vdots & \ddots & \vdots \\
0 & 0 & \cdots & 1
\end{pmatrix}
\]
which clearly lies in $\operatorname{GL}_n(\mathbb{R})$ and has determinant $\alpha$. Hence the determinant map is \emph{onto} (surjective).

\subsection*{Step 3: Identify the kernel of $\det$}
The kernel of the determinant map is
\[
\ker(\det) 
\;=\; 
\{\, A \in \operatorname{GL}_n(\mathbb{R}) \;\mid\; \det(A) = 1 \}
\;=\;
\operatorname{SL}_n(\mathbb{R}),
\]
the special linear group of all $n \times n$ real matrices with determinant $1$.

\subsection*{Step 4: Apply the First Isomorphism Theorem}
The First Isomorphism Theorem tells us that
\[
\operatorname{GL}_n(\mathbb{R}) \,\big/\, \ker(\det)
\;\cong\;
\operatorname{Im}(\det).
\]
From Steps 2 and 3, we know
\[
\ker(\det) = \operatorname{SL}_n(\mathbb{R})
\quad\text{and}\quad
\operatorname{Im}(\det) = \mathbb{R}^\times.
\]
Hence,
\[
\operatorname{GL}_n(\mathbb{R}) \,\big/\, \operatorname{SL}_n(\mathbb{R})
\;\cong\;
\mathbb{R}^\times.
\]

\subsection*{Conclusion}
We have established a natural isomorphism between the quotient group $\operatorname{GL}_n(\mathbb{R})/\operatorname{SL}_n(\mathbb{R})$ and the multiplicative group of nonzero real numbers $\mathbb{R}^\times$ by way of the determinant map.


\section*{Solution to Part (b)}

\noindent
\textbf{Claim.} $\displaystyle \mathbb{C}^{\times} \big/ C_n \;\cong\; \mathbb{C}^{\times},$ 
where 
\[
C_n \;=\; \bigl\{\, e^{2\pi i k/n}\;:\; k=0,1,\dots,n-1\bigr\}
\]
is the group of $n$th roots of unity under multiplication, and $\mathbb{C}^\times$ is the multiplicative group of all nonzero complex numbers.

\subsection*{Step 1: Define the map}
Consider the map
\[
\varphi:\;\mathbb{C}^\times \;\longrightarrow\;\mathbb{C}^\times,
\quad
z \;\longmapsto\; z^n.
\]
Since $z \neq 0$ in $\mathbb{C}^\times$, $z^n$ is also a nonzero complex number, so $\varphi$ is well-defined.

\subsection*{Step 2: Surjectivity of $\varphi$}
Let $w \in \mathbb{C}^\times$ be arbitrary. We want to solve $z^n = w$ for $z \in \mathbb{C}^\times$. By choosing any branch of the complex $n$th root of $w$, we see there is at least one solution $z$ with $z^n = w$. Indeed, if $w = re^{i\theta}$ in polar form, then a principal $n$th root for $w$ can be chosen as
\[
z = r^{1/n}e^{i \theta/n}.
\]
Hence, $\varphi$ is onto: $\operatorname{Im}(\varphi) = \mathbb{C}^\times$.

\subsection*{Step 3: Kernel of $\varphi$}
The kernel of $\varphi$ is precisely the set of $z \in \mathbb{C}^\times$ such that $z^n = 1$. These are precisely the $n$th roots of unity, i.e.\
\[
\ker(\varphi)
\;=\;
\bigl\{\,z \in \mathbb{C}^\times : z^n = 1 \bigr\}
\;=\;
C_n.
\]

\subsection*{Step 4: Apply the First Isomorphism Theorem}
Since $\mathbb{C}^\times$ is abelian, $C_n$ is a normal subgroup. By the First Isomorphism Theorem,
\[
\mathbb{C}^\times \,\big/\, \ker(\varphi) 
\;\cong\;
\operatorname{Im}(\varphi).
\]
From Steps 2 and 3, we know that
\[
\ker(\varphi) = C_n
\quad\text{and}\quad
\operatorname{Im}(\varphi) = \mathbb{C}^\times.
\]
Hence,
\[
\mathbb{C}^\times \,\big/\, C_n 
\;\cong\; 
\mathbb{C}^\times,
\]
as desired.

\subsection*{Conclusion}
We have shown that the quotient of $\mathbb{C}^\times$ by the subgroup of $n$th roots of unity $C_n$ is isomorphic to $\mathbb{C}^\times$, using the natural map $z \mapsto z^n$.

\end{document}
