\documentclass[12pt]{article}

% Packages
\usepackage[margin=1in]{geometry}
\usepackage{amsmath,amssymb,amsthm}
\usepackage{enumitem}
\usepackage{hyperref}
\usepackage{xcolor}
\usepackage{import}
\usepackage{xifthen}
\usepackage{pdfpages}
\usepackage{transparent}
\usepackage{listings}


\lstset{
    breaklines=true,         % Enable line wrapping
    breakatwhitespace=false, % Wrap lines even if there's no whitespace
    basicstyle=\ttfamily,    % Use monospaced font
    frame=single,            % Add a frame around the code
    columns=fullflexible,    % Better handling of variable-width fonts
}

\newcommand{\incfig}[1]{%
    \def\svgwidth{\columnwidth}
    \import{./Figures/}{#1.pdf_tex}
}
\theoremstyle{definition} % This style uses normal (non-italicized) text
\newtheorem{solution}{Solution}
\newtheorem{proposition}{Proposition}
\newtheorem{problem}{Problem}
\newtheorem{lemma}{Lemma}
\newtheorem{theorem}{Theorem}
\newtheorem{remark}{Remark}
\newtheorem{note}{Note}
\theoremstyle{plain} % Restore the default style for other theorem environments
%

% Theorem-like environments
% Title information
\title{}
\author{Jerich Lee}
\date{\today}

\begin{document}

\maketitle
\[
\textbf{Compute } 400^{60} \pmod{61}:
\]

\textbf{Step 1: Reduce the base modulo 61.}  
\[
400 \equiv 400 - 6 \cdot 61 
     = 400 - 366
     = 34 
     \pmod{61}.
\]
Hence,
\[
400^{60} \equiv 34^{60} \pmod{61}.
\]

\textbf{Step 2: Apply Fermat's Little Theorem.}  

Since \(61\) is prime, Fermat's Little Theorem says:
\[
a^{p-1} \equiv 1 \pmod{p}
\quad
\text{for all } a \text{ with } \gcd(a, p) = 1.
\]
Here \(p = 61\) and \(p-1 = 60\). Also, \(\gcd(34, 61) = 1\). Therefore,
\[
34^{60} \equiv 1 \pmod{61}.
\]

\textbf{Step 3: Conclusion.}  

Combining everything,
\[
400^{60} 
\equiv 34^{60} 
\equiv 1
\pmod{61}.
\]

Hence the final answer is
\[
\boxed{400^{60} \equiv 1 \pmod{61}}.
\]
\[
\textbf{Compute the last digit of } 17^{50}.
\]

We want the remainder of \(17^{50}\) when divided by 10, i.e.,
\[
17^{50} \pmod{10}.
\]

\textbf{Step 1: Reduce the base modulo 10.}
\[
17 \equiv 7 \pmod{10}.
\]
Hence,
\[
17^{50} \equiv 7^{50} \pmod{10}.
\]

\textbf{Step 2: Find the pattern of powers of 7 modulo 10.}
\[
\begin{aligned}
7^1 &\equiv 7 \pmod{10},\\
7^2 &\equiv 49 \equiv 9 \pmod{10},\\
7^3 &\equiv 9 \cdot 7 = 63 \equiv 3 \pmod{10},\\
7^4 &\equiv 3 \cdot 7 = 21 \equiv 1 \pmod{10},\\
7^5 &\equiv 1 \cdot 7 = 7 \pmod{10}.
\end{aligned}
\]
Thus, the cycle of \(7^n \pmod{10}\) is \(\{7,9,3,1\}\) and repeats every 4 powers.

\textbf{Step 3: Determine the position in the cycle.}

We take \(50 \mod 4\):
\[
50 \equiv 2 \pmod{4}.
\]
Since the remainder is 2, we pick the second element in the cycle \(\{7,9,3,1\}\), which is \(9\).

\textbf{Conclusion:}
\[
7^{50} \equiv 9 \pmod{10} 
\quad\Longrightarrow\quad
17^{50} \equiv 9 \pmod{10}.
\]

Therefore, the last digit of \(17^{50}\) in base-10 notation is
\[
\boxed{9}.
\]
\[
\textbf{Why does the sequence of } 7^k \text{ (mod 10) repeat?}
\]

\textbf{1. Finite number of remainders:}

When working \(\bmod\; 10\), any integer is congruent to one of the 10 possible remainders 
\[
\{0,1,2,\dots,9\}.
\]
Consider the sequence
\[
7^1 \bmod 10,\quad 7^2 \bmod 10,\quad 7^3 \bmod 10,\quad \dots
\]
Since there are only 10 possible remainders, by the \textit{pigeonhole principle}, eventually two of these powers must give the same remainder. Once a remainder repeats, the sequence will cycle from that point onward.

\textbf{2. Coprimality and Euler's Theorem:}

We also note that \(\gcd(7,10) = 1\). By Euler's theorem (a special case of Fermat's little theorem for general \(n\)), if \(\gcd(a,n)=1\), then
\[
a^{\varphi(n)} \equiv 1 \pmod{n},
\]
where \(\varphi(n)\) is Euler's totient function. For \(n=10\), we have \(\varphi(10) = 4\). Therefore,
\[
7^4 \equiv 1 \pmod{10}.
\]
This directly implies that
\[
7^5 \equiv 7 \pmod{10}, \quad 7^6 \equiv 7^2 \pmod{10}, \quad \ldots
\]
so the pattern of remainders \(\bmod\,10\) repeats every 4 powers.

Hence, \emph{there must be a repeating cycle} in the powers of 7 modulo 10.
\end{document}
