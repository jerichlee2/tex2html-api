\documentclass[12pt]{article}

% Packages
\usepackage[margin=.5in]{geometry}
\usepackage{amsmath,amssymb,amsthm}
\usepackage{enumitem}
\usepackage{hyperref}
\usepackage{xcolor}
\usepackage{import}
\usepackage{xifthen}
\usepackage{pdfpages}
\usepackage{transparent}
\usepackage{listings}


\lstset{
    breaklines=true,         % Enable line wrapping
    breakatwhitespace=false, % Wrap lines even if there's no whitespace
    basicstyle=\ttfamily,    % Use monospaced font
    frame=single,            % Add a frame around the code
    columns=fullflexible,    % Better handling of variable-width fonts
}

\newcommand{\incfig}[1]{%
    \def\svgwidth{\columnwidth}
    \import{./Figures/}{#1.pdf_tex}
}
\theoremstyle{definition} % This style uses normal (non-italicized) text
\newtheorem{solution}{Solution}
\newtheorem{proposition}{Proposition}
\newtheorem{problem}{Problem}
\newtheorem{lemma}{Lemma}
\newtheorem{theorem}{Theorem}
\newtheorem{remark}{Remark}
\newtheorem{note}{Note}
\newtheorem{definition}{Definition}
\newtheorem{example}{Example}
\theoremstyle{plain} % Restore the default style for other theorem environments
%

% Theorem-like environments
% Title information
\title{}
\author{Jerich Lee}
\date{\today}

\begin{document}

\maketitle
\noindent
\textbf{Problem 1.} Calculate:
\[
\text{(a)}\; 3^{100} \pmod{30}, 
\quad
\text{(b)}\; 5^{63} \pmod{14},
\quad
\text{(c)}\; 10^{70} \pmod{142},
\quad
\text{(d)}\; \text{the last two digits of } 17^{41}.
\]

\bigskip
\textbf{(a) Compute } $3^{100} \pmod{30}$

\begin{enumerate}
\item First, observe that powers of 3 modulo 30 form a repeating cycle. We check the first few powers:
\[
\begin{aligned}
3^1 &\equiv 3 \pmod{30},\\
3^2 &\equiv 9 \pmod{30},\\
3^3 &\equiv 27 \pmod{30},\\
3^4 &\equiv 81 \equiv 21 \pmod{30}, \\
3^5 &\equiv 3 \times 21 \equiv 63 \equiv 3 \pmod{30}.
\end{aligned}
\]
\item Hence, the residues repeat every 4 powers: the cycle is $(3,\,9,\,27,\,21)$ and then back to $3$.
\item Since $100$ is a multiple of 4 (i.e.\ $100 \equiv 0 \pmod{4}$), we look at the 4th term in the cycle.
\[
3^{100} \equiv 21 \pmod{30}.
\]
\end{enumerate}

\bigskip
\textbf{(b) Compute } $5^{63} \pmod{14}$

\begin{enumerate}
\item Since $\gcd(5,14) = 1$, we can apply Euler's theorem (or Carmichael's theorem). 
\item Recall $\varphi(14) = \varphi(2 \times 7) = (2-1)\times(7-1) = 6$. 
\item By Euler's theorem:
\[
5^6 \equiv 1 \pmod{14}.
\]
\item Note that $63 \equiv 3 \pmod{6}$, so
\[
5^{63} \;=\; 5^{6 \cdot 10 + 3} \;=\; (5^6)^{10} \cdot 5^3 \;\equiv\; 1^{10} \cdot 125 \equiv 125 \pmod{14}.
\]
\item Finally, $125 \equiv 125 - 112 \equiv 13 \pmod{14}$. Thus,
\[
5^{63} \equiv 13 \pmod{14}.
\]
\end{enumerate}

\bigskip
\textbf{(c) Compute } $10^{70} \pmod{142}$

\begin{enumerate}
\item Factor $142 = 2 \times 71$. We will use the Chinese Remainder Theorem (CRT).
\item \textbf{Modulo 2:} Clearly, $10^{70}$ is even, so
\[
10^{70} \equiv 0 \pmod{2}.
\]
\item \textbf{Modulo 71:} Note that $\gcd(10,71) = 1$ and $71$ is prime, so by Fermat's Little Theorem:
\[
10^{70} \equiv 1 \pmod{71}.
\]
\item We want $x \equiv 10^{70}$ such that 
\[
\begin{cases}
x \equiv 0 \pmod{2},\\
x \equiv 1 \pmod{71}.
\end{cases}
\]
\item Write $x = 1 + 71k$. For $x$ to be even, we need:
\[
1 + 71k \equiv 0 \pmod{2}.
\]
Since $71 \equiv 1 \pmod{2}$, we get
\[
1 + k \equiv 0 \pmod{2} \quad\Longrightarrow\quad k \equiv 1 \pmod{2}.
\]
\item Taking the smallest positive solution $k=1$ gives $x = 1 + 71 \cdot 1 = 72$, which is even. Therefore,
\[
10^{70} \equiv 72 \pmod{142}.
\]
\end{enumerate}

\bigskip
\textbf{(d) Find the last two digits of } $17^{41}$

\begin{enumerate}
\item Finding the last two digits of a number is the same as finding it modulo 100.
\item Since $\gcd(17,100) = 1$, we can apply Euler's theorem with $\varphi(100) = 40$.
\item Thus,
\[
17^{40} \equiv 1 \pmod{100}.
\]
\item Hence,
\[
17^{41} = 17^{40} \cdot 17 \equiv 1 \cdot 17 \equiv 17 \pmod{100}.
\]
\item Therefore, the last two digits of $17^{41}$ are $\boxed{17}$.
\end{enumerate}

\bigskip
\textbf{Answers:}
\[
\begin{aligned}
(a)\;&3^{100} \equiv 21 \pmod{30},\\
(b)\;&5^{63} \equiv 13 \pmod{14},\\
(c)\;&10^{70} \equiv 72 \pmod{142},\\
(d)\;&\text{the last two digits of }17^{41}\text{ are }17.
\end{aligned}
\]

\begin{theorem}[Euler's Theorem]
    Let $n$ be a positive integer and $a$ be an integer coprime to $n$ 
    (i.e.\ $\gcd(a,n) = 1$). Then
    \[
      a^{\varphi(n)} \equiv 1 \pmod{n},
    \]
    where $\varphi(n)$ is Euler's totient function (the number of integers 
    less than or equal to $n$ that are coprime to $n$).
  \end{theorem}
  
  \begin{theorem}[Chinese Remainder Theorem]
    Let $n_1, n_2, \ldots, n_k$ be pairwise coprime positive integers,
    and let $N = n_1\,n_2 \cdots n_k$. Suppose $a_1, a_2, \ldots, a_k$ 
    are any integers. Then there exists a unique integer $x$ modulo $N$ 
    such that
    \[
      x \equiv a_i \pmod{n_i} 
      \quad\text{for each}\quad i = 1, 2, \ldots, k.
    \]
    Moreover, all solutions $x$ to this system are congruent modulo $N$.
  \end{theorem} 

  \begin{definition}[Euler's Totient Function]
    Let $n$ be a positive integer. The \emph{Euler totient function},
    denoted by $\varphi(n)$, is defined as the number of positive integers 
    less than or equal to $n$ that are \emph{coprime} to $n$. That is,
    \[
      \varphi(n) = \bigl|\{\,1 \le k \le n : \gcd(k,n) = 1\}\bigr|.
    \]
    Equivalently, if $n$ has the prime factorization
    \[
      n = p_1^{\alpha_1}\,p_2^{\alpha_2}\cdots p_m^{\alpha_m},
    \]
    then
    \[
      \varphi(n) \;=\; n \,\Bigl(1 - \frac{1}{p_1}\Bigr)
                       \,\Bigl(1 - \frac{1}{p_2}\Bigr)\cdots
                       \Bigl(1 - \frac{1}{p_m}\Bigr).
    \]
  \end{definition}


\begin{example}
    Let us compute $\varphi(48)$, where $\varphi(\cdot)$ denotes the
    Euler totient function.
  
    \medskip
    First, factorize $48$ into primes:
    \[
      48 \;=\; 2^4 \,\times\, 3^1.
    \]
    Using the product formula for the totient function,
    \[
      \varphi(n) \;=\; n \biggl(1 - \frac{1}{p_1}\biggr)
                      \biggl(1 - \frac{1}{p_2}\biggr)
                      \cdots
                      \biggl(1 - \frac{1}{p_m}\biggr),
    \]
    we get
    \[
      \varphi(48)
      \;=\;
      48
      \bigl(1 - \tfrac{1}{2}\bigr)
      \bigl(1 - \tfrac{1}{3}\bigr).
    \]
    Simplify step by step:
    \[
      1 - \tfrac{1}{2} \;=\; \tfrac{1}{2},
      \quad
      1 - \tfrac{1}{3} \;=\; \tfrac{2}{3}.
    \]
    Hence,
    \[
      \varphi(48)
      \;=\;
      48 \times \tfrac{1}{2} \times \tfrac{2}{3}
      \;=\;
      48 \times \tfrac{1}{3}
      \;=\;
      16.
    \]
    Therefore, $\varphi(48) = 16$.
  \end{example}


\begin{theorem}[Fermat's Little Theorem]
    If $p$ is a prime number, then for any integer $a$,
    \[
      a^p \equiv a \pmod{p}.
    \]
    Consequently, if $\gcd(a,p) = 1$, then
    \[
      a^{p-1} \equiv 1 \pmod{p}.
    \]
  \end{theorem}
  
  \begin{example}
    \textbf{Applying Fermat's Little Theorem:} 
  
    Suppose we wish to compute $2^{101} \pmod{5}$. Observe that $\gcd(2, 5)=1$ and $5$ is prime. Hence by Fermat's Little Theorem,
    \[
      2^{4} \equiv 1 \pmod{5}.
    \]
    Since $101 = 4 \times 25 + 1$, we can write:
    \[
      2^{101} = 2^{4 \times 25 + 1} = (2^4)^{25} \times 2^1 
               \equiv 1^{25} \times 2 
               \equiv 2 \pmod{5}.
    \]
    Therefore,
    \[
      2^{101} \equiv 2 \pmod{5}.
    \]
  \end{example}

  \begin{theorem}[Chinese Remainder Theorem]
    Let $n_1, n_2, \ldots, n_k$ be positive integers that are pairwise coprime; 
    that is, $\gcd(n_i, n_j) = 1$ for all $i \neq j$. 
    Suppose $a_1, a_2, \ldots, a_k$ are any integers. 
    Then the system of simultaneous congruences
    \[
      \begin{cases}
        x \equiv a_1 \pmod{n_1},\\
        x \equiv a_2 \pmod{n_2},\\
        \quad \vdots \\
        x \equiv a_k \pmod{n_k},
      \end{cases}
    \]
    has a unique solution $x$ modulo $N = n_1\,n_2 \cdots n_k$. 
    In other words, there exists exactly one residue class $x \bmod N$ 
    satisfying all of these congruences simultaneously.
  \end{theorem}
  
  \begin{example}
    \textbf{Applying the Chinese Remainder Theorem:}
  
    Consider the system:
    \[
      \begin{cases}
        x \equiv 2 \pmod{3},\\
        x \equiv 3 \pmod{5}.
      \end{cases}
    \]
    Here, $n_1 = 3$, $n_2 = 5$, and they are coprime. 
    We have $N = 3 \cdot 5 = 15$.
  
    \medskip
    \textbf{Step 1:} Write the system as:
    \[
      x = 2 + 3k_1, \quad\text{for some integer } k_1.
    \]
    We want $x$ to satisfy $x \equiv 3 \pmod{5}$. Substituting $x = 2 + 3k_1$ into this congruence:
    \[
      2 + 3k_1 \equiv 3 \pmod{5}.
    \]
    Simplify:
    \[
      3k_1 \equiv 1 \pmod{5}.
    \]
    Since $3 \cdot 2 \equiv 6 \equiv 1 \pmod{5}$, we see $k_1 \equiv 2 \pmod{5}$. 
    Thus the smallest nonnegative choice is $k_1 = 2$, giving
    \[
      x = 2 + 3 \cdot 2 = 2 + 6 = 8.
    \]
    Therefore, $x \equiv 8 \pmod{15}$ is a solution that satisfies both 
    congruences simultaneously.
  
    \medskip
    \textbf{Step 2:} Verify:
    \[
      8 \equiv 2 \pmod{3} 
      \quad\text{and}\quad
      8 \equiv 3 \pmod{5}.
    \]
    Indeed, both statements check out. By the CRT, \emph{every} solution 
    to the system is congruent to $8$ modulo $15$.
  
    \medskip
    Thus the unique solution modulo $15$ is $\boxed{x \equiv 8 \pmod{15}}$.
  \end{example}

  \section*{Solution to Problem 2}

\subsection*{(a) Find $\mathrm{ord}_{\mathbb{Z}_n}(k)$.}

\textbf{Problem statement.} Let $n \in \mathbb{N}$ and $k \in \mathbb{Z}_n$. Recall that
\[
 \mathrm{ord}_{\mathbb{Z}_n}(k)
\]
is defined as the smallest positive integer $m$ such that $mk = 0$ in $\mathbb{Z}_n$, i.e.\ $mk \equiv 0 \pmod{n}$.
 
\textbf{Solution.}

\begin{enumerate}
  \item Since $k \in \mathbb{Z}_n$, we can represent $k$ by an integer in $\{0,1,\dots,n-1\}$. Let us still denote that representative by $k$.
  \item We want the smallest positive integer $m$ for which
  \[
    mk \equiv 0 \pmod{n}.
  \]
  \item The condition $mk \equiv 0 \pmod{n}$ is equivalent to saying $n \mid mk$, i.e.\ $mk$ is a multiple of $n$.
  \item By number theory, the smallest such positive $m$ for which $n \mid mk$ is given by
  \[
    m = \frac{n}{\gcd(n,k)}.
  \]
  (Indeed, the product $mk$ is a multiple of $n$ precisely when $m \gcd(n,k)$ is a multiple of $\gcd(n,k)$ times $\frac{n}{\gcd(n,k)}$, which leads to $m$ being a multiple of $\frac{n}{\gcd(n,k)}$. The least positive such $m$ is exactly $\frac{n}{\gcd(n,k)}$.)
\end{enumerate}

Therefore,
\[
  \mathrm{ord}_{\mathbb{Z}_n}(k) \;=\; \frac{n}{\gcd(n,k)}.
\]

\subsection*{(b) Show that $\mathbb{Z}_n / \{0,\, d,\, \ldots,\, \bigl(\tfrac{n}{d}-1\bigr)d\} \simeq \mathbb{Z}_d$ when $d \mid n$.}

\textbf{Problem statement.} Let $n,d \in \mathbb{N}$ such that $d \mid n$. We consider the subset
\[
  H = \{\,0,\, d,\, 2d,\, \ldots,\,(\tfrac{n}{d}-1)d \} \;\subset\; \mathbb{Z}_n.
\]
We wish to show that the quotient ring (or quotient group under addition)
\[
  \mathbb{Z}_n / H
\]
is isomorphic to $\mathbb{Z}_d$.

\textbf{Solution.}

\begin{enumerate}
  \item \textbf{Interpretation of $H$ as a subgroup.} Note that $H$ is precisely the set of multiples of $d$ in $\mathbb{Z}_n$. Concretely, if we identify $\mathbb{Z}_n$ with the set $\{\,0,1,2,\ldots,n-1\}$ under addition mod $n$, then
  \[
    H = d\mathbb{Z}_n = \{\,d \cdot 0,\, d \cdot 1,\, \ldots,\, d \cdot (\tfrac{n}{d}-1)\}.
  \]
  This is an additive subgroup of $\mathbb{Z}_n$, often denoted $d\mathbb{Z}_n$.
  
  \item \textbf{Well-known quotient fact.} A standard result from elementary number theory (or group theory) is that
  \[
    \mathbb{Z}_n / d\mathbb{Z}_n \; \simeq \; \mathbb{Z}_d
  \]
  whenever $d\mid n$. Intuitively, in $\mathbb{Z}_n$ we are ``collapsing'' all elements that differ by a multiple of $d$ into the same equivalence class.

  \item \textbf{Explicit isomorphism.}
  We can give a direct construction of the isomorphism
  \[
    \varphi \colon \mathbb{Z}_n / H \;\to\; \mathbb{Z}_d
  \]
  as follows: take the coset of $x$ in $\mathbb{Z}_n / H$ (denoted $[x]$) and map it to $x \bmod d$ in $\mathbb{Z}_d$:
  \[
    \varphi([x]) \;=\; x \mod d.
  \]
  This is well-defined because if $x$ and $y$ are in the same coset of $H$, then $x - y$ is a multiple of $d$, so $x \equiv y \pmod{d}$. 

  \item \textbf{Homomorphism properties.} One checks that $\varphi$ is a homomorphism of (additive) groups and that it is bijective:
    \begin{itemize}
      \item \emph{Surjectivity}: every element $r \in \{0,1,\dots,d-1\}$ has a preimage in $\mathbb{Z}_n$ (namely, the coset of $r$ itself). 
      \item \emph{Injectivity}: if $[x]$ and $[y]$ map to the same element in $\mathbb{Z}_d$, then $x \equiv y \pmod{d}$, so $x-y$ is in $d\mathbb{Z}_n$, thus $[x] = [y]$ in the quotient. 
    \end{itemize}
  
  \item \textbf{Conclusion.} Hence, 
  \[
    \mathbb{Z}_n / H \;\simeq\; \mathbb{Z}_d,
  \]
  completing the proof.
\end{enumerate}

\section*{First Isomorphism Theorem}

\begin{theorem}[First Isomorphism Theorem for Groups]
Let $G$ and $H$ be groups and let 
\[
  \varphi : G \longrightarrow H
\]
be a group homomorphism. Then the quotient group
\[
  G/\ker(\varphi)
\]
is isomorphic to the image of $\varphi$, i.e.\ $\operatorname{im}(\varphi) \subseteq H$. 
In symbols:
\[
  G/\ker(\varphi) \;\simeq\; \operatorname{im}(\varphi).
\]
\end{theorem}

\begin{example}
Consider the homomorphism 
\[
  f : \mathbb{Z} \longrightarrow \mathbb{Z}_n
\]
defined by 
\[
  f(k) \;=\; [\,k\,]_n,
\]
where $[\,k\,]_n$ denotes the residue class of $k$ modulo $n$. 

\begin{itemize}
  \item The kernel of $f$ is 
  \[
    \ker(f) \;=\; \{\,k \in \mathbb{Z} : f(k) = [0]_n\,\}
    \;=\; \{\,k \in \mathbb{Z} : k \equiv 0 \pmod{n}\}
    \;=\; n\mathbb{Z}.
  \]

  \item The image of $f$ is all of $\mathbb{Z}_n$, because for every $[\,r\,]_n \in \mathbb{Z}_n$, we can take any integer representative $r \in \mathbb{Z}$ and see that 
  $f(r) = [\,r\,]_n$.

\end{itemize}

Thus, by the First Isomorphism Theorem,
\[
  \mathbb{Z}/\,n\mathbb{Z} \;\simeq\; \operatorname{im}(f) = \mathbb{Z}_n.
\]
\end{example}

\section*{Problem 3}

\noindent\textbf{(a)} Find subgroups $G_1,\, G_2$ of $\mathbb{Z}_6$ such that 
\[
  G_1 \simeq \mathbb{Z}_2,\quad
  G_2 \simeq \mathbb{Z}_3
  \quad\text{and}\quad
  \mathbb{Z}_6 \simeq G_1 \oplus G_2.
\]

\subsubsection*{Solution to (a)}

\begin{itemize}
  \item We regard $\mathbb{Z}_6$ as an additive cyclic group of order $6$, generated by the element $1 \pmod{6}$.
  \item A subgroup of order 2 in $\mathbb{Z}_6$ is generated by $3 \pmod{6}$, because $2 \cdot 3 = 6 \equiv 0 \,(\bmod\,6)$. Hence we can set 
  \[
    G_1 \;=\; \langle 3 \rangle \;=\; \{0,\,3\}.
  \]
  Clearly, $|G_1|=2$ and $G_1 \simeq \mathbb{Z}_2$.

  \item A subgroup of order 3 in $\mathbb{Z}_6$ is generated by $2 \pmod{6}$, since $3\cdot 2 = 6 \equiv 0\pmod{6}$. Hence 
  \[
    G_2 \;=\; \langle 2 \rangle \;=\; \{0,\,2,\,4\}.
  \]
  We have $|G_2|=3$ and $G_2 \simeq \mathbb{Z}_3$.

  \item Observe that every element of $\mathbb{Z}_6$ can be uniquely written as a sum of an element in $G_1$ and an element in $G_2$.  Concretely:
  \[
    \{0,1,2,3,4,5\}
    \;=\;
    \{ (0+0),\,(3+4),\,(0+2),\,(3+0),\,(0+4),\,(3+2)\},
  \]
  and one checks this covers $\mathbb{Z}_6$ without overlap.
  
  Thus, $\mathbb{Z}_6 \simeq G_1 \oplus G_2$, as required.
\end{itemize}

\bigskip

\noindent\textbf{(b)} Find subgroups $G_1,\,G_2,\,G_3$ of $\mathbb{Z}_{30}$ such that 
\[
  G_1 \simeq \mathbb{Z}_2,\quad
  G_2 \simeq \mathbb{Z}_3,\quad
  G_3 \simeq \mathbb{Z}_5
  \quad\text{and}\quad
  \mathbb{Z}_{30} \simeq G_1 \oplus G_2 \oplus G_3.
\]

\subsubsection*{Solution to (b)}

\begin{itemize}
  \item The group $\mathbb{Z}_{30}$ is also cyclic (additive) of order 30. By the \emph{structure theorem for finite cyclic groups} (or the Chinese Remainder Theorem viewpoint), $\mathbb{Z}_{30}$ is isomorphic to $\mathbb{Z}_2 \oplus \mathbb{Z}_3 \oplus \mathbb{Z}_5$ because $2 \cdot 3 \cdot 5 = 30$.
  \item We exhibit explicit subgroups of $\mathbb{Z}_{30}$ of the required orders:
    \begin{itemize}
      \item A subgroup of order 2 is generated by $15 \pmod{30}$:
      \[
        G_1 \;=\; \langle 15\rangle \;=\; \{0,\,15\}.
      \]
      \item A subgroup of order 3 is generated by $10 \pmod{30}$:
      \[
        G_2 \;=\; \langle 10\rangle \;=\; \{0,\,10,\,20\}.
      \]
      \item A subgroup of order 5 is generated by $6 \pmod{30}$:
      \[
        G_3 \;=\; \langle 6\rangle \;=\; \{0,\,6,\,12,\,18,\,24\}.
      \]
    \end{itemize}
  \item One checks that $|G_1|=2$, $|G_2|=3$, $|G_3|=5$, and that every element of $\mathbb{Z}_{30}$ can be written uniquely as a sum of elements from $G_1, G_2, G_3$.  Thus
  \[
    \mathbb{Z}_{30} \;\simeq\; G_1 \oplus G_2 \oplus G_3.
  \]
\end{itemize}

\bigskip

\noindent\textbf{(c)} Find subgroups $G_1,\,G_2$ of $\mathbb{Z}_{15}^*$ (the multiplicative group of units mod 15) such that 
\[
  G_1 \simeq \mathbb{Z}_3^*, 
  \quad
  G_2 \simeq \mathbb{Z}_5^*,
  \quad\text{and}\quad
  \mathbb{Z}_{15}^* \simeq G_1 \times G_2.
\]

\subsubsection*{Solution to (c)}

\begin{itemize}
  \item First recall that $\mathbb{Z}_{15}^*$ is the group (under multiplication mod 15) of all integers relatively prime to 15.  Since $15=3\cdot 5$, the Chinese Remainder Theorem tells us
  \[
    \mathbb{Z}_{15}^* \;\simeq\; \mathbb{Z}_3^* \times \mathbb{Z}_5^*.
  \]
  Also, $\mathbb{Z}_3^* \simeq \mathbb{Z}_2$ (it has elements $\{[1],[2]\}$) and $\mathbb{Z}_5^*$ is cyclic of order 4 (isomorphic to $\mathbb{Z}_4$).
  \item Concretely, the elements of $\mathbb{Z}_{15}^*$ are 
  \[
    \{\,1,2,4,7,8,11,13,14\}
    \quad(\text{mod }15).
  \]
  This is a group of order $8$.
  \item We need to exhibit:
    \[
      G_1 \;\simeq\; \mathbb{Z}_3^*\quad (\text{order }2),
      \quad
      G_2 \;\simeq\; \mathbb{Z}_5^*\quad (\text{order }4),
      \quad
      \mathbb{Z}_{15}^* \;\simeq\; G_1 \times G_2.
    \]
  \item A convenient choice:
    \begin{itemize}
      \item \textbf{Subgroup of order 2:}  pick the element $14 \equiv -1 \pmod{15}$, whose square is $14^2=196 \equiv 1 \,(\bmod\,15)$. Then 
      \[
        G_1 = \{\,1,\,14\}.
      \]
      \item \textbf{Subgroup of order 4:} one can pick the element $2\pmod{15}$, which has order 4 since
      \[
        2^2 = 4 \neq 1,\quad 2^3 = 8 \neq 1,\quad 2^4 = 16 \equiv 1\pmod{15}.
      \]
      Hence
      \[
        G_2 = \langle 2 \rangle = \{\,1,\,2,\,4,\,8\}.
      \]
    \end{itemize}
  \item Notice that $G_1 \cap G_2 = \{1\}$ and $|G_1|\cdot |G_2| = 2\cdot 4 = 8$, which matches the total number of units mod 15.  Consequently,
  \[
    \mathbb{Z}_{15}^* \;=\; G_1 \,G_2 \;\simeq\; G_1 \times G_2,
  \]
  as required.
\end{itemize}

\section*{Lagrange's Theorem for Finite Groups}

\begin{theorem}[Lagrange's Theorem]
Let $G$ be a finite group and $H$ be a subgroup of $G$. Then the order of $H$ divides the order of $G$, i.e.
\[
  |H|\; \Big|\; |G|.
\]
\end{theorem}

\begin{example}
Consider the group $G = \mathbb{Z}_{12}$ under addition modulo 12 (so $|G|=12$), and let $H$ be the subgroup generated by $4 \in \mathbb{Z}_{12}$. That is,
\[
  H \;=\; \langle 4 \rangle \;=\; \{0,4,8\} \quad (\text{mod }12).
\]
We see that $H$ has 3 elements, while $G$ has 12. Lagrange's Theorem tells us that $|H|$ must divide $|G|$. Indeed, $3 \,\big|\, 12$. 
\end{example}
\section*{Problem 4}

\subsection*{(a) Order of an element in a direct product}

\textbf{Statement.} Let $G_1, G_2, \dots, G_s$ be groups, and let 
\[
  G \;=\; G_1 \,\times\, G_2 \,\times\, \dots \,\times\, G_s
\]
be their direct product. Suppose each $G_i$ contains an element $a_i$ of finite order (denoted $\mathrm{ord}_{G_i}(a_i) < \infty$). Show that for the element
\[
  (a_1, a_2, \dots, a_s) \;\in\; G,
\]
its order in $G$ is given by 
\[
  \mathrm{ord}_G \bigl(a_1, a_2, \dots, a_s\bigr)
  \;=\; \mathrm{lcm}\!\bigl(\mathrm{ord}_{G_1}(a_1),\,\mathrm{ord}_{G_2}(a_2),\,\dots,\,\mathrm{ord}_{G_s}(a_s)\bigr).
\]

\paragraph{Step-by-Step Proof:}

\begin{enumerate}
  \item \textbf{Interpretation of the order in the product.}
  By definition, 
  \[
    \mathrm{ord}_G(a_1, \dots, a_s)
  \]
  is the smallest positive integer $n$ such that 
  \[
    n\cdot(a_1, \dots, a_s) \;=\; \bigl(e_{G_1}, e_{G_2}, \dots, e_{G_s}\bigr),
  \]
  where $e_{G_i}$ is the identity in $G_i$ (and $n\cdot(a_1,\dots,a_s)$ denotes the $n$-fold group operation in $G$, e.g.\ repeated addition if the groups are written additively, or repeated multiplication otherwise).

  \item \textbf{Condition for each component to be the identity.}
  The element $n\cdot(a_1,\dots,a_s)$ equals 
  \[
    \bigl(n\cdot a_1,\; n\cdot a_2,\;\dots,\; n\cdot a_s\bigr).
  \]
  (Here each $n\cdot a_i$ is computed inside $G_i$.)  For this product to be the identity $(e_{G_1},\dots,e_{G_s})$, we need 
  \[
    n\cdot a_i \;=\; e_{G_i} 
    \quad \text{for each } i=1,\dots,s.
  \]
  In other words, $\mathrm{ord}_{G_i}(a_i)$ must divide $n$ for \emph{every} $i$.

  \item \textbf{Largest required divisor.}
  Therefore, any $n$ which annihilates $\bigl(a_1,\dots,a_s)$ to the identity must be a common multiple of 
  $\mathrm{ord}_{G_1}(a_1),\;\mathrm{ord}_{G_2}(a_2),\;\dots,\;\mathrm{ord}_{G_s}(a_s)$.

  \item \textbf{Minimal such $n$ is the LCM.}
  The smallest positive integer satisfying $n \cdot a_i = e_{G_i}$ for each $i$ is precisely
  \[
    \mathrm{lcm}\bigl(\mathrm{ord}_{G_1}(a_1),\;\mathrm{ord}_{G_2}(a_2),\;\dots,\;\mathrm{ord}_{G_s}(a_s)\bigr).
  \]
  Hence
  \[
    \mathrm{ord}_G\bigl(a_1,\dots,a_s\bigr)
    \;=\;
    \mathrm{lcm}\!\bigl(\mathrm{ord}_{G_1}(a_1),\,\dots,\,\mathrm{ord}_{G_s}(a_s)\bigr).
  \]

\end{enumerate}

\bigskip

\subsection*{(b) \(\mathbb{Z}_4 \oplus \mathbb{Z}_4\) is not isomorphic to \(\mathbb{Z}_4 \oplus \mathbb{Z}_2 \oplus \mathbb{Z}_2\)}

\textbf{Claim.} The abelian groups $\mathbb{Z}_4 \oplus \mathbb{Z}_4$ and $\mathbb{Z}_4 \oplus \mathbb{Z}_2 \oplus \mathbb{Z}_2$ are not isomorphic.

\paragraph{Step-by-Step Reasoning:}

\begin{enumerate}
  \item \textbf{Both groups have the same cardinality.}
  \[
    |\mathbb{Z}_4 \oplus \mathbb{Z}_4| \;=\; 4 \times 4 \;=\; 16,
    \quad
    |\mathbb{Z}_4 \oplus \mathbb{Z}_2 \oplus \mathbb{Z}_2| \;=\; 4 \times 2 \times 2 \;=\; 16.
  \]
  So they agree in size.

  \item \textbf{Count elements of order 2 in \(\mathbb{Z}_4 \oplus \mathbb{Z}_4\).}
    \begin{itemize}
      \item An element in $\mathbb{Z}_4 \oplus \mathbb{Z}_4$ can be written as $(\overline{x}, \overline{y})$ where $x,y \in \{0,1,2,3\}$ and $\overline{x}, \overline{y}$ denote residue classes modulo 4.
      \item $(\overline{x},\overline{y})$ has order 2 exactly when 
      \[
        2\cdot(\overline{x}, \overline{y}) \;=\; (\overline{0}, \overline{0})
        \quad \text{but} \quad (\overline{x}, \overline{y}) \neq (\overline{0}, \overline{0}).
      \]
      \item Since $2\cdot(\overline{x}, \overline{y}) = (\overline{2x}, \overline{2y})$, we need $\overline{2x} = \overline{0}$ and $\overline{2y} = \overline{0}$ in $\mathbb{Z}_4$. This implies $x,y \in \{0,2\}$. 
      \item The candidates are therefore $(\overline{0},\overline{0}),\,(\overline{2},\overline{0}),\,(\overline{0},\overline{2}),\,(\overline{2},\overline{2})$. Out of these four:
        \[
          \text{the identity }(\overline{0},\overline{0})\text{ is \emph{not} of order 2.}
        \]
      \item \emph{Hence there are exactly 3 elements of order 2} in $\mathbb{Z}_4 \oplus \mathbb{Z}_4$.
    \end{itemize}

  \item \textbf{Count elements of order 2 in \(\mathbb{Z}_4 \oplus \mathbb{Z}_2 \oplus \mathbb{Z}_2\).}
    \begin{itemize}
      \item An element is $(\overline{x}, \overline{y}, \overline{z})$ with $x \in \{0,1,2,3\}$ (mod 4) and $y,z \in \{0,1\}$ (mod 2).
      \item Its order is 2 when
      \[
        2 \cdot (\overline{x}, \overline{y}, \overline{z}) 
        \;=\; (\overline{0}, \overline{0}, \overline{0})
        \quad \text{but} \quad
        (\overline{x},\overline{y},\overline{z}) \neq (\overline{0},\overline{0},\overline{0}).
      \]
      \item Now, $2\cdot(\overline{x}, \overline{y}, \overline{z}) 
      = (\overline{2x},\,\overline{2y},\,\overline{2z})$. 
      \begin{itemize}
        \item In $\mathbb{Z}_2$, we always have $\overline{2y} = \overline{0}$ regardless of $y\in\{0,1\}$. Hence $y,z$ are unrestricted by the doubling condition (they do not affect whether the element squares to the identity).
        \item The condition $\overline{2x} = \overline{0}$ in $\mathbb{Z}_4$ implies $x \in \{0,2\}$.
      \end{itemize}
      \item Therefore the possible $x$ values are 0 or 2, while $y,z$ can each be 0 or 1 independently. That gives $2 \times 2 \times 2 = 8$ possible tuples $(\overline{x},\overline{y},\overline{z})$. However, one of them is the identity $(\overline{0},\overline{0},\overline{0})$ and does \emph{not} have order 2.
      \item \emph{Hence there are $8 - 1 = 7$ elements of order 2} in $\mathbb{Z}_4 \oplus \mathbb{Z}_2 \oplus \mathbb{Z}_2$.
    \end{itemize}

  \item \textbf{Conclusion:} Since $\mathbb{Z}_4 \oplus \mathbb{Z}_4$ has only 3 elements of order 2 but $\mathbb{Z}_4 \oplus \mathbb{Z}_2 \oplus \mathbb{Z}_2$ has 7 elements of order 2, they cannot be isomorphic (an isomorphism must preserve the number of elements of each order). Thus,
  \[
    \mathbb{Z}_4 \oplus \mathbb{Z}_4 
    \;\not\simeq\; 
    \mathbb{Z}_4 \oplus \mathbb{Z}_2 \oplus \mathbb{Z}_2.
  \]
\end{enumerate}

\section*{Theorem}
\begin{theorem}
Let $G_1$ and $G_2$ be groups, and consider their direct product
\[
  G_1 \times G_2.
\]
This direct product is abelian (commutative) if and only if both $G_1$ and $G_2$ are abelian.
\end{theorem}

\begin{proof}
We prove both directions.

\bigskip
\noindent\textbf{(\(\Longrightarrow\)) Suppose $G_1\times G_2$ is abelian. Show that $G_1$ and $G_2$ are abelian.}

\begin{enumerate}
  \item \textbf{Setup:} We know $G_1\times G_2$ is commutative, meaning for any 
  \[
    (a_1, a_2), (b_1, b_2) \;\in\; G_1\times G_2,
    \quad
    (a_1,a_2)\,(b_1,b_2) \;=\; (b_1,b_2)\,(a_1,a_2).
  \]
  
  \item \textbf{Show $G_1$ is abelian:}
    \begin{itemize}
      \item Take arbitrary $g_1, h_1 \in G_1$. We need to show $g_1 h_1 = h_1 g_1$ in $G_1$.
      \item Consider $(g_1,e_{G_2})$ and $(h_1, e_{G_2})$ in $G_1\times G_2$. By commutativity of the product:
      \[
        (g_1, e_{G_2}) \,(h_1, e_{G_2}) \;=\; (h_1, e_{G_2})\,(g_1, e_{G_2}).
      \]
      \item Using the definition of multiplication in the direct product,
      \[
        (g_1 h_1,\, e_{G_2} e_{G_2}) = (h_1 g_1,\, e_{G_2} e_{G_2}).
      \]
      \item This equality implies $g_1 h_1 = h_1 g_1$ in $G_1$, so $G_1$ is abelian.
    \end{itemize}

  \item \textbf{Show $G_2$ is abelian:}
    \begin{itemize}
      \item Take arbitrary $g_2, h_2 \in G_2$. We need $g_2 h_2 = h_2 g_2$ in $G_2$.
      \item Similar reasoning: consider $(e_{G_1},g_2)$ and $(e_{G_1},h_2)$ in $G_1\times G_2$. By commutativity,
      \[
        (e_{G_1}, g_2)\,(e_{G_1}, h_2) \;=\; (e_{G_1}, h_2)\,(e_{G_1}, g_2).
      \]
      \item This yields $(e_{G_1} e_{G_1},\, g_2 h_2) = (e_{G_1} e_{G_1},\, h_2 g_2)$, forcing $g_2 h_2 = h_2 g_2$.
    \end{itemize}

  \item \textbf{Conclusion for this direction:} Since both $G_1$ and $G_2$ are abelian, we have shown that $G_1 \times G_2$ being abelian implies $G_1$ and $G_2$ must each be abelian.
\end{enumerate}

\bigskip
\noindent\textbf{(\(\Longleftarrow\)) Suppose $G_1$ and $G_2$ are abelian. Show that $G_1\times G_2$ is abelian.}

\begin{enumerate}
  \item \textbf{Setup:} Since $G_1$ is abelian, $g_1 h_1 = h_1 g_1$ for all $g_1, h_1 \in G_1$. Similarly, $G_2$ is abelian means $g_2 h_2 = h_2 g_2$ for all $g_2, h_2 \in G_2$.

  \item \textbf{Goal:} We need to show that every pair of elements in $G_1\times G_2$ commutes. That is, for any $(g_1, g_2), (h_1, h_2)\in G_1\times G_2$,
  \[
    (g_1,g_2)\,(h_1,h_2) \;=\; (h_1,h_2)\,(g_1,g_2).
  \]

  \item \textbf{Compute product in each order:}
    \[
      (g_1,g_2)\,(h_1,h_2) 
      \;=\; 
      \bigl(g_1 h_1,\; g_2 h_2\bigr),
    \]
    \[
      (h_1,h_2)\,(g_1,g_2) 
      \;=\; 
      \bigl(h_1 g_1,\; h_2 g_2\bigr).
    \]

  \item \textbf{Use commutativity within each group:}
    \[
      g_1 h_1 = h_1 g_1 
      \quad \text{(since $G_1$ is abelian)},
    \]
    \[
      g_2 h_2 = h_2 g_2
      \quad \text{(since $G_2$ is abelian)}.
    \]

  \item \textbf{Conclude equality:}
    \[
      \bigl(g_1 h_1,\; g_2 h_2\bigr)
      =
      \bigl(h_1 g_1,\; h_2 g_2\bigr).
    \]
    Thus $(g_1,g_2)(h_1,h_2) = (h_1,h_2)(g_1,g_2)$.

  \item \textbf{Conclusion for this direction:} We've shown every pair of elements in the product commutes, so $G_1\times G_2$ is abelian.
\end{enumerate}

By combining both directions, we conclude that $G_1 \times G_2$ is abelian if and only if $G_1$ and $G_2$ are each abelian.
\end{proof}

\end{document}
