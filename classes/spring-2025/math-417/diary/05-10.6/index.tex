\documentclass[12pt]{article}

% Packages
\usepackage[margin=1in]{geometry}
\usepackage{amsmath,amssymb,amsthm}
\usepackage{enumitem}
\usepackage{hyperref}
\usepackage{xcolor}
\usepackage{import}
\usepackage{xifthen}
\usepackage{pdfpages}
\usepackage{transparent}
\usepackage{listings}
\usepackage{tikz}
\usepackage{physics}
\usepackage{siunitx}
\usepackage{booktabs}
\usepackage{cancel}
  \usetikzlibrary{calc,patterns,arrows.meta,decorations.markings}


\DeclareMathOperator{\Log}{Log}
\DeclareMathOperator{\Arg}{Arg}

\lstset{
    breaklines=true,         % Enable line wrapping
    breakatwhitespace=false, % Wrap lines even if there's no whitespace
    basicstyle=\ttfamily,    % Use monospaced font
    frame=single,            % Add a frame around the code
    columns=fullflexible,    % Better handling of variable-width fonts
}

\newcommand{\incfig}[1]{%
    \def\svgwidth{\columnwidth}
    \import{./Figures/}{#1.pdf_tex}
}
\theoremstyle{definition} % This style uses normal (non-italicized) text
\newtheorem{solution}{Solution}
\newtheorem{proposition}{Proposition}
\newtheorem{problem}{Problem}
\newtheorem{lemma}{Lemma}
\newtheorem{theorem}{Theorem}
\newtheorem{remark}{Remark}
\newtheorem{note}{Note}
\newtheorem{definition}{Definition}
\newtheorem{example}{Example}
\newtheorem{corollary}{Corollary}
\theoremstyle{plain} % Restore the default style for other theorem environments
%

% Theorem-like environments
% Title information
\title{MATH 417}
\author{Jerich Lee}
\date{\today}

\begin{document}

\maketitle
% Intuitive explanation that  \operatorname{Aut}(P)\cong\bigl(\mathbb{Z}/p\mathbb{Z}\bigr)^{\times}

\newcommand{\Z}{\mathbb{Z}}
\newcommand{\Aut}{\operatorname{Aut}}

\begin{center}
\textbf{Why $\Aut(P)\;\cong\;\bigl(\Z/p\Z\bigr)^{\times}$ when $P$ is cyclic of order $p$}
\end{center}

\medskip
Let 
\[
   P \;=\; \bigl(\, \Z/p\Z,\,+\,\bigr)
\]
be the \emph{additive} cyclic group of prime order~$p$.

\begin{enumerate}
   \item \textbf{Every automorphism is determined by where it sends a generator.}\\
         The element~$1\in P$ generates the whole group.  
         An automorphism $\alpha\in\Aut(P)$ must send $1$ to some element 
         $\alpha(1)=a\in P$ that is \emph{also} a generator.
         \[
            \alpha(k)\;=\;\alpha\!\bigl(1+\dots+1\bigr)
            \;=\;k\,\alpha(1)
            \;=\;k\,a\quad(\text{$k$ times}).
         \]
         Thus $\alpha$ is nothing more than ``multiplication by~$a$''.

   \item \textbf{Which $a$ work?  Exactly the units $\pmod{p}$.}\\
         An element~$a\in P$ generates $P$ iff $\gcd(a,p)=1$,
         i.e.\ $a\not\equiv 0\pmod{p}$.  
         So the $p-1$ valid choices for $a$ are precisely the
         \emph{units} in the ring $\Z/p\Z$, denoted
         $\bigl(\Z/p\Z\bigr)^{\times}$.

   \item \textbf{Define the mapping}
         \[
            \Phi : \bigl(\Z/p\Z\bigr)^{\times}\;\longrightarrow\;\Aut(P),
            \qquad
            a \;\longmapsto\;(x\mapsto ax).
         \]
         \begin{itemize}
            \item \emph{Homomorphism:}
                  $\Phi(ab)(x)=abx=\Phi(a)\!\bigl(\Phi(b)(x)\bigr)$,  
                  so $\Phi(ab)=\Phi(a)\circ\Phi(b)$.
            \item \emph{Injective:}
                  $\Phi(a)=\Phi(b)\;\Rightarrow\;ax=bx$ for all $x$,
                  hence $a=b$.
            \item \emph{Surjective:}
                  Every automorphism is ``multiplication by some generator’’,
                  so each $\alpha\in\Aut(P)$ equals $\Phi(a)$
                  for the unique $a=\alpha(1)\in(\Z/p\Z)^{\times}$.
         \end{itemize}

   \item \textbf{Conclusion.}\;
         $\Phi$ is a bijective homomorphism, hence an isomorphism:
         \[
            \boxed{\;\Aut(P)\;\cong\;\bigl(\Z/p\Z\bigr)^{\times}\,. }
         \]
\end{enumerate}

\medskip
\textit{Intuition in one sentence:}  
an automorphism of a cyclic group just “rescales’’ its generator,
and the permissible rescalings are precisely the non-zero
integers modulo~$p$, which multiply exactly like
the unit group~$\bigl(\Z/p\Z\bigr)^{\times}$.
% The multiplicative group modulo 11

\newcommand{\units}[1]{\bigl(\Z/#1\Z\bigr)^{\times}}

\[
  \units{11}
  \;=\;
  \{\,1,2,3,4,5,6,7,8,9,10\,\},
\]
with the group operation given by multiplication modulo 11.

\begin{itemize}
  \item \textbf{Order.}  
        Because \(11\) is prime, every non-zero residue mod 11 is invertible,  
        so \(|\units{11}| = 11-1 = 10\).

  \item \textbf{Structure.}  
        A classical result says that \(\units{p}\) is \emph{cyclic} for any prime \(p\);  
        hence
        \[
            \units{11}\;\cong\;C_{10},
        \]
        the cyclic group of order 10.

  \item \textbf{Primitive roots (generators).}  
        Any element whose order is 10 generates the whole group.
        For example, \(2\) is a primitive root modulo 11:
        \[
           2^1= 2,\;
           2^2= 4,\;
           2^3= 8,\;
           2^4= 5,\;
           2^5=10,\;
           2^6= 9,\;
           2^7= 7,\;
           2^8= 3,\;
           2^9= 6,\;
           2^{10}\equiv 1\pmod{11}.
        \]
        Thus
        \[
           \langle 2\rangle
           \;=\;
           \{\,1,2,4,8,5,10,9,7,3,6\,\}
           \;=\;\units{11}.
        \]
        Other generators (primitive roots) modulo 11 are \(2, 6, 7, 8\).
\end{itemize}
\begin{solution}
  \textbf{Problem.}\;
  Let \(G\) be a finite group of order
  \[
     |G| \;=\; 3^{2}\cdot 11 \;=\; 99 .
  \]
  
  \medskip
  \textbf{(a) A unique Sylow \(11\)-subgroup.}
  
  Write \(n_{11}\) for the number of Sylow \(11\)-subgroups of \(G\).
  Sylow’s theorems give the two conditions  
  \[
     n_{11}\;\bigm|\;3^{2},
     \qquad
     n_{11}\,\equiv\,1\pmod{11}.
  \]
  Hence \(n_{11}\in\{1,9\}\) but also \(n_{11}\equiv1\pmod{11}\),
  so the only possibility is
  \[
     n_{11}=1.
  \]
  Therefore \(G\) contains \emph{exactly one} subgroup of order \(11\).
  Because conjugation permutes the Sylow \(11\)-subgroups,
  this unique subgroup is fixed under conjugation and is hence
  \emph{normal} in \(G\).
  
  Let us denote it by
  \[
     N \;:=\; \text{Sylow}_{11}(G)
     \;\cong\;C_{11}\;\trianglelefteq\;G.
  \]
  
  \medskip
  \textbf{(b) Must \(G\) be abelian?}
  
  Let \(P\le G\) be any Sylow \(3\)-subgroup; then \(|P|=3^{2}=9\).
  
  \smallskip
  \emph{Step 1.\; All groups of order \(\mathbf{9}\) are abelian.}  
  There are only two isomorphism types:
  \(C_{9}\) and \(C_{3}\times C_{3}\), both abelian.
  
  \smallskip
  \emph{Step 2.\; Conjugation gives a homomorphism into
  \(\operatorname{Aut}(N)\).}  
  Because \(N\trianglelefteq G\), conjugation by an element
  \(g\in G\) restricts to an automorphism of \(N\).
  Thus the action of \(P\) on \(N\) defines a homomorphism
  \[
     \varphi : P \;\longrightarrow\; \operatorname{Aut}(N).
  \]
  Since \(N\cong C_{11}\) is cyclic of prime order,
  \[
     \operatorname{Aut}(N)\;\cong\;(\mathbb Z/11\mathbb Z)^{\times}
     \;\cong\;C_{10},
  \]
  a cyclic group of order \(10\).
  
  \smallskip
  \emph{Step 3.\; The image of \(\varphi\) is trivial.}  
  The order of \(\operatorname{Im}\varphi\) divides both \(|P|=9\)
  and \(|\operatorname{Aut}(N)|=10\), hence divides
  \(\gcd(9,10)=1\).  Therefore \(\operatorname{Im}\varphi=\{1\}\) and
  \[
     \varphi \ \text{ is the trivial homomorphism}.
  \]
  
  \smallskip
  \emph{Step 4.\; \(P\) centralises \(N\).}  
  A trivial action means every element of \(P\) commutes
  with every element of \(N\); equivalently,
  \(P\le C_{G}(N)\).
  
  \smallskip
  \emph{Step 5.\; \(G\) is a direct product of abelian groups.}  
  Because \(N\cap P=\{1\}\) and \(NP=G\)
  (\(|N|\cdot|P|=11\cdot9=99=|G|\)),
  we have
  \[
     G \;=\; N\cdot P \;=\; N \times P .
  \]
  Both \(N\) and \(P\) are abelian,
  so their direct product \(G\) is abelian.
  
  \medskip
  \textbf{Conclusion.}\;
  Every group of order \(99=3^{2}\cdot11\) is necessarily abelian.  
  (No non-abelian example exists.)
  \end{solution}
  \begin{problem}
    Let \(G\) be a group of order
    \[
       45 \;=\; 3^{2}\cdot5.
    \]
    \begin{enumerate}[]
       \item
          Using Sylow’s theorems, show that \(G\) contains a \emph{unique}
          Sylow \(5\)-subgroup.  Deduce that this subgroup is normal in \(G\).
       \item
          Must \(G\) be abelian?  
          Either prove that it is, or construct an explicit non-abelian
          group of order \(45\) and justify your example.
    \end{enumerate}
    \end{problem}
    \begin{remark}
      \textbf{Why a \emph{unique} Sylow \(p\)-subgroup is automatically normal.}
      
      Let \(G\) be a finite group and let \(P\le G\) be its (unique) Sylow
      \(p\)-subgroup.
      \begin{enumerate}
         \item[\(\triangleright\)]
               \emph{Conjugates are Sylow.}  
               For any \(g\in G\), the subgroup
               \(gPg^{-1}\) has the same order as \(P\)
               (conjugation preserves order) and is therefore also a Sylow
               \(p\)-subgroup.
         \item[\(\triangleright\)]
               \emph{Uniqueness forces equality.}  
               Because \(P\) is the \emph{only} Sylow \(p\)-subgroup,
               every such conjugate must coincide with \(P\) itself:
               \[
                  gPg^{-1}=P
                  \quad\text{for all }g\in G.
               \]
         \item[\(\triangleright\)]
               \emph{Definition of normality.}  
               The condition \(gPg^{-1}=P\) for all \(g\in G\) is
               exactly the statement that \(P\) is a \emph{normal} subgroup
               (\(P\trianglelefteq G\)).
      \end{enumerate}
      Hence a Sylow subgroup is normal whenever it is unique.
      \end{remark}
      %--------------------------------------------------------------------
%  Expanding the “trivial versus non-trivial” action for |G| = 63
%--------------------------------------------------------------------


\begin{center}
\large Groups of order \(63 = 3^{2}\cdot7\) via semidirect products
\end{center}

\bigskip
\textbf{Step 0.  The two Sylow subgroups}

\[
   |G| \;=\; 3^{2}\cdot7
   \quad\Longrightarrow\quad
   \begin{cases}
      N := \text{Sylow}_{7}(G), & |N| = 7,\\[3pt]
      Q := \text{Sylow}_{3}(G), & |Q| = 9.
   \end{cases}
\]
By Sylow’s theorems \(N\trianglelefteq G\) (it is the \emph{only}
Sylow \(7\)-subgroup), so every group of order~63 is an extension
\[
   1\;\longrightarrow\;N\;\longrightarrow\;G
     \;\xrightarrow{\;\;\pi\;\;}\;Q\;\longrightarrow\;1.
\]

\bigskip
\textbf{Step 1.  Encoding the extension by an action}

Because \(N\cong C_{7}\) is cyclic, its automorphism group is
\[
   \Aut(N)\;\cong\;(\Z/7\Z)^{\times}\;\cong\;C_{6}
   \;=\;\langle\;\alpha\;|\;\alpha^{6}=1\rangle,
   \quad
   \alpha:\;a\longmapsto a^{2},
\]
where \(a\) is a fixed generator of \(N\).

Conjugation in \(G\) gives a homomorphism
\[
   \varphi : Q \;\longrightarrow\; \Aut(N),\qquad
   x \;\longmapsto\;(a\mapsto xax^{-1}).
\]
The structure of \(G\) is completely determined (up to isomorphism)
by the pair \(\bigl(Q,\varphi\bigr)\).

\bigskip
\textbf{Step 2.  Possible images of \(\varphi\)}

\[
   |\Aut(N)| = 6,\quad |Q| = 9
   \;\;\Longrightarrow\;\;
   |\varphi(Q)| \,\bigm|\,\gcd(9,6) = 3.
\]
Hence \(\varphi(Q)\) has order \(1\) or \(3\).

\smallskip
\begin{enumerate}
   \item[\(\boldsymbol{|\,\varphi(Q)| = 1}\)] (\emph{trivial action})\\
         \(\varphi\) is the zero homomorphism, so \(Q\) centralises \(N\).
         The extension splits as a \emph{direct} product
         \[
            G \;\cong\; N\times Q
            \;\cong\; C_{7}\times Q,
         \]
         which is abelian.  Since every group of order \(9\) is abelian,
         the two possibilities are
         \[
            C_{7}\times C_{9}
            \quad\text{or}\quad
            C_{7}\times (C_{3}\times C_{3}).
         \]
   \item[\(\boldsymbol{|\,\varphi(Q)| = 3}\)] (\emph{non-trivial action})\\
         The image is the unique subgroup of order \(3\) inside \(C_{6}\),
         namely \(\langle\alpha^{2}\rangle\) or \(\langle\alpha^{4}\rangle\).
         Any such action is equivalent (via automorphisms of \(N\))
         to the one determined by
         \[
            \alpha^{2}:\; a\longmapsto a^{2}\quad
            \bigl(\text{note }2^{3} \equiv 1\pmod{7}\bigr).
         \]
         \emph{Cyclic \(Q\).}  
         For the map \(\varphi\) to have cyclic image of order 3, one needs
         a cyclic \(Q\) of order 9 with a kernel of index 3.
         Concretely, take
         \[
            Q = \langle\,b \mid b^{9}=1\,\rangle,
            \qquad
            \varphi(b)\;=\;\alpha^{2}.
         \]
         Then \(K := \ker\varphi = \langle b^{3}\rangle \cong C_{3}\).

         \smallskip
         The resulting group is the \emph{semidirect} product
         \[
            G \;\cong\;
            C_{7}\rtimes_{\varphi} C_{9}
            \;=\;
            \bigl\langle\,a,b
              \;\bigm|\;
                 a^{7}=1,\;
                 b^{9}=1,\;
                 b^{-1}ab = a^{2}\bigr\rangle.
         \]
         Because \(b\) does not commute with \(a\) (\(a^{2}\neq a\)),
         this \(G\) is \emph{non-abelian}.

         \emph{Elementary-abelian \(Q\).}  
         If \(Q\cong C_{3}\times C_{3}\), every homomorphism from \(Q\)
         to a cyclic group is trivial (all elements of \(Q\) have order 3),
         so \(|\varphi(Q)|=3\) is \emph{impossible} in that case.
\end{enumerate}

\bigskip
\textbf{Step 3.  Classification summary}

Every group of order \(63\) is isomorphic to \emph{exactly one} of the
following:
\[
\begin{array}{ll}
\text{(i)} & C_{7}\times C_{9}\quad (\text{abelian}),\\[4pt]
\text{(ii)} & C_{7}\times(C_{3}\times C_{3})\quad (\text{abelian}),\\[4pt]
\text{(iii)} & C_{7}\rtimes C_{9}
             = \langle a,b \mid a^{7}=b^{9}=1,\;
               b^{-1}ab=a^{2}\rangle\quad (\text{non-abelian}).
\end{array}
\]
Thus the “\(|\varphi(Q)| = 1\)” line in the note leads to the two
abelian direct products, while the “\(|\varphi(Q)| = 3\)” line
produces the unique non-abelian semidirect product of order \(63\).
\end{document}
