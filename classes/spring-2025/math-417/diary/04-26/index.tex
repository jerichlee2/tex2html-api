\documentclass[12pt]{article}

% Packages
\usepackage[margin=1in]{geometry}
\usepackage{amsmath,amssymb,amsthm}
\usepackage{enumitem}
\usepackage{hyperref}
\usepackage{xcolor}
\usepackage{import}
\usepackage{xifthen}
\usepackage{pdfpages}
\usepackage{transparent}
\usepackage{listings}
\usepackage{tikz}
\usepackage{physics}
\usepackage{siunitx}
  \usetikzlibrary{calc,patterns,arrows.meta,decorations.markings}


\DeclareMathOperator{\Log}{Log}
\DeclareMathOperator{\Arg}{Arg}

\lstset{
    breaklines=true,         % Enable line wrapping
    breakatwhitespace=false, % Wrap lines even if there's no whitespace
    basicstyle=\ttfamily,    % Use monospaced font
    frame=single,            % Add a frame around the code
    columns=fullflexible,    % Better handling of variable-width fonts
}

\newcommand{\incfig}[1]{%
    \def\svgwidth{\columnwidth}
    \import{./Figures/}{#1.pdf_tex}
}
\theoremstyle{definition} % This style uses normal (non-italicized) text
\newtheorem{solution}{Solution}
\newtheorem{proposition}{Proposition}
\newtheorem{problem}{Problem}
\newtheorem{lemma}{Lemma}
\newtheorem{theorem}{Theorem}
\newtheorem{remark}{Remark}
\newtheorem{note}{Note}
\newtheorem{definition}{Definition}
\newtheorem{example}{Example}
\newtheorem{corollary}{Corollary}
\theoremstyle{plain} % Restore the default style for other theorem environments
%

% Theorem-like environments
% Title information
\title{MATH-417: HW 9}
\author{Jerich Lee}
\date{\today}

\begin{document}

\maketitle
\begin{problem}[]
  
\end{problem}
\begin{solution}

    Show that the quotient ring
    \[
      \mathbb{R}[x]\big/\bigl((x-1)(x-2)\bigr)
      \;\cong\;
      \mathbb{R}\;\oplus\;\mathbb{R}.
    \]

    
    \begin{proof}[Step–by–step solution]
    \textbf{Step 1:  Define an evaluation homomorphism.}\newline
    Let
    \[
      \varphi:\mathbb{R}[x]\;\longrightarrow\;\mathbb{R}\oplus\mathbb{R},
      \qquad
      f(x)\;\longmapsto\;\bigl(f(1),\,f(2)\bigr).
    \]
    Clearly $\varphi$ preserves addition and multiplication, so it is a ring
    homomorphism.
    
    \bigskip
    \textbf{Step 2:  Show that $\varphi$ is surjective.}\newline
    Given any pair $(a,b)\in\mathbb{R}\oplus\mathbb{R}$, construct the linear
    polynomial
    \[
      g(x)\;=\;a\,\frac{x-2}{1-2}\;+\;b\,\frac{x-1}{2-1}
             \;=\;a\,(x-2)\;-\;b\,(x-1).
    \]
    Then $g(1)=a$ and $g(2)=b$, so $\varphi(g)=(a,b)$.  Thus $\operatorname{Im}\varphi
    =\mathbb{R}\oplus\mathbb{R}$.
    
    \bigskip
    \textbf{Step 3:  Identify $\ker\varphi$.}\newline
    A polynomial $f(x)\in\mathbb{R}[x]$ lies in $\ker\varphi$ iff
    \[
      f(1)=0\quad\text{and}\quad f(2)=0,
    \]
    which holds precisely when $f(x)$ is divisible by both $(x-1)$ and $(x-2)$.
    Because these two linear factors are coprime,
    \[
      \ker\varphi
      \;=\;\bigl((x-1)\bigr)\cap\bigl((x-2)\bigr)
      \;=\;\bigl((x-1)(x-2)\bigr).
    \]
    
    \bigskip
    \textbf{Step 4:  Apply the First Isomorphism Theorem.}\newline
    The theorem states that
    \[
      \mathbb{R}[x]/\ker\varphi
      \;\cong\;
      \operatorname{Im}\varphi.
    \]
    Substituting the results of Steps 2–3 yields
    \[
      \boxed{\;
        \mathbb{R}[x]\big/\bigl((x-1)(x-2)\bigr)
        \;\cong\;
        \mathbb{R}\;\oplus\;\mathbb{R}\;} .
    \]
    
    \bigskip
    \textbf{Alternative viewpoint (Chinese Remainder Theorem).}\newline
    Since the ideals $(x-1)$ and $(x-2)$ are coprime,
    the Chinese Remainder Theorem gives
    \[
      \mathbb{R}[x]\big/\bigl((x-1)(x-2)\bigr)
      \;\cong\;
      \mathbb{R}[x]/(x-1)\;\oplus\;\mathbb{R}[x]/(x-2)
      \;\cong\;
      \mathbb{R}\;\oplus\;\mathbb{R},
    \]
    confirming the result.
    \end{proof} 

\end{solution}
\begin{problem}[]
  
\end{problem}
\begin{solution}
    Show that the ring of Gaussian integers
    \[
      \mathbb{Z}[i]\;=\;\{\,a+bi \mid a,b\in\mathbb{Z}\,\}
    \]
    is a \emph{Euclidean domain}.
    
    \begin{proof}[Step–by–step solution]
    Recall that an integral domain $R$ is \emph{Euclidean} if there exists a
    function $N:R\setminus\{0\}\to\mathbb{Z}_{\ge 0}$ (a \textbf{Euclidean norm})
    such that for every $a,b\in R$ with $b\neq 0$ there are $q,r\in R$ satisfying
    \[
      a \;=\; bq + r
      \quad\text{with}\quad
      r=0\;\text{ or }\;N(r)<N(b).
    \]
    
    \bigskip
    \textbf{Step 1:  Choose the norm.}\newline
    For $z=a+bi\in\mathbb{Z}[i]\setminus\{0\}$ define
    \[
      N(z)\;=\;a^{2}+b^{2}\;=\;|z|^{2}.
    \]
    \emph{Why this works:}
    \begin{itemize}
      \item $N(z)\in\mathbb{Z}_{\ge 0}$ for every $z\in\mathbb{Z}[i]\setminus\{0\}$.
      \item $N$ is \emph{multiplicative}: $N(zw)=N(z)N(w)$, because
            $|zw|=|z||w|$ in~$\mathbb{C}$.
    \end{itemize}
    
    \bigskip
    \textbf{Step 2:  Find a “nearest” Gaussian integer.}\newline
    Let $z\in\mathbb{C}$ be arbitrary.  Write $z=x+iy$ with $x,y\in\mathbb{R}$ and set
    \[
      q=\bigl(\!\lfloor x+\tfrac12 \rfloor\bigr)
         +
         \bigl(\!\lfloor y+\tfrac12 \rfloor\bigr)i
         \;\in\;\mathbb{Z}[i],
    \]
    i.e.\ round each coordinate to the nearest integer.
    Then
    \[
      |x-\Re(q)|\le \tfrac12,
      \quad
      |y-\Im(q)|\le \tfrac12,
      \quad\Longrightarrow\quad
      |z-q|
      \;=\;
      \sqrt{(x-\Re q)^{2}+(y-\Im q)^{2}}
      \;\le\;
      \sqrt{\tfrac12}
      \;=\;
      \frac{1}{\sqrt{2}}.
    \]
    
    \bigskip
    \textbf{Step 3:  Perform Euclidean division.}\newline
    Given $a,b\in\mathbb{Z}[i]$ with $b\neq 0$, put
    \[
      z \;=\; \frac{a}{b}\in\mathbb{C},
      \qquad
      q \text{ as in Step 2},
      \qquad
      r \;=\; a-bq.
    \]
    By construction $q\in\mathbb{Z}[i]$ and therefore $r\in\mathbb{Z}[i]$.
    Moreover
    \[
      |r|
      \;=\;
      |a-bq|
      \;=\;
      |b|\cdot\bigl|z-q\bigr|
      \;\le\;
      |b|\cdot\frac1{\sqrt{2}},
      \qquad\Longrightarrow\qquad
      N(r)=|r|^{2}\;\le\;\frac12\,|b|^{2}\;=\;\frac12\,N(b)\;<\;N(b).
    \]
    If $|z-q|=0$ then $r=0$, so the remainder condition is satisfied as well.
    
    \bigskip
    \textbf{Step 4:  Conclude.}\newline
    Because every pair $(a,b)$ with $b\ne 0$ admits such $q,r$, the function
    $N(a+bi)=a^{2}+b^{2}$ is a Euclidean norm on $\mathbb{Z}[i]$.
    Hence $\boxed{\;\mathbb{Z}[i]\text{ is a Euclidean domain.}\;}$
    \end{proof}
\end{solution}
\begin{problem}[]
  
\end{problem}
\begin{solution}
    Let $I=(2+i)\subseteq\mathbb{Z}[i]$ be the principal ideal generated by the
    Gaussian integer $2+i$.  Prove
    
    \begin{enumerate}[label=\textup{(\alph*)}]
      \item $1\notin I$;
      \item $\displaystyle\mathbb{Z}[i]\big/I\;\cong\;\mathbb{Z}_5$.
    \end{enumerate}
    
    \emph{Hint:}  Consider the homomorphism
    $\psi:\mathbb{Z}\longrightarrow\mathbb{Z}[i]/I,\;n\mapsto n+I$.
    
    \begin{proof}[Step--by--step solution]
    Throughout, recall the \emph{norm} on $\mathbb{Z}[i]$,
    \(
      N(a+bi)=a^{2}+b^{2},
    \)
    which is multiplicative: $N(zw)=N(z)\,N(w)$.
    
    \bigskip
    \textbf{(a) Show that $1\notin I$.}
    
    \begin{enumerate}[label=\arabic*.]
      \item If $1\in I$ then there exists $u\in\mathbb{Z}[i]$ such that
            \(
              1=(2+i)\,u.
            \)
      \item Taking norms gives
            \(
              1
              \;=\;
              N(1)
              \;=\;
              N\bigl((2+i)u\bigr)
              \;=\;
              N(2+i)\;N(u).
            \)
      \item Compute $N(2+i)=2^{2}+1^{2}=5$.
            Hence we would have $1=5\,N(u)$, an impossibility because
            $5\nmid 1$ in~$\mathbb{Z}$.
    \end{enumerate}
    Therefore \(1\notin I\).
    
    \bigskip
    \textbf{(b) Show that \(\mathbb{Z}[i]/I\cong\mathbb{Z}_5\).}
    
    \smallskip
    \textit{Step 1:  Define a convenient homomorphism.}
    
    \[
      \psi:\mathbb{Z}\;\longrightarrow\;\mathbb{Z}[i]/I,
      \qquad
      n\;\longmapsto\;n+I.
    \]
    
    \smallskip
    \textit{Step 2:  Determine \(\ker\psi\).}
    
    \begin{enumerate}[label=\alph*.]
      \item An integer $n$ lies in $\ker\psi$ iff $n\in I$, i.e.\ $n=(2+i)(a+bi)$
            for some $a,b\in\mathbb{Z}$.
      \item Taking norms again,
            \(
              n^{2}=N(n)=N(2+i)\,N(a+bi)
                     =5\bigl(a^{2}+b^{2}\bigr).
            \)
            Hence $5\mid n^{2}$, so $5\mid n$.
      \item Conversely, if $n=5k$ then
            \(
              5k=(2+i)\bigl(k(2-i)\bigr)\in I,
            \)
            so $5k\in\ker\psi$.
    \end{enumerate}
    Thus
    \[
      \ker\psi = 5\mathbb{Z}.
    \]
    
    \smallskip
    \textit{Step 3:  Show that \(\psi\) is surjective.}
    
    Given any $z=a+bi\in\mathbb{Z}[i]$, perform Euclidean division by $2+i$:
    there exist $q,r\in\mathbb{Z}[i]$ with
    \(
      z=(2+i)q+r,\;N(r)<5.
    \)
    Because $N(r)<5$, the only possible norms are $0,1,2,4$; hence
    $N(r)\in\{0,1,4\}$ and in each case $r$ is congruent modulo~$I$
    to some integer in $\{0,1,2,3,4\}$.  Consequently every coset of $I$ has a
    \emph{integer} representative, so $\psi$ hits them all.
    
    \smallskip
    \textit{Step 4:  Apply the First Isomorphism Theorem.}
    
    Since $\psi$ is a surjective ring homomorphism with kernel $5\mathbb{Z}$,
    \[
      \mathbb{Z}[i]/I
      \;\cong\;
      \mathbb{Z}\big/\ker\psi
      \;=\;
      \mathbb{Z}\big/5\mathbb{Z}
      \;=\;
      \boxed{\mathbb{Z}_5}.
    \]
    \end{proof}
\end{solution}
\begin{problem}[]
  
\end{problem}
\begin{solution}
    Let $x$ be a \emph{nilpotent} element in a ring $R$ with identity~$1$; that is,
    there exists some integer $n\ge 1$ such that $x^{\,n}=0$.
    Show that $1-x$ is invertible in~$R$.
    
    \begin{proof}[Step–by–step solution]
    \textbf{Step 1:  Use the definition of nilpotence.}\newline
    Because $x$ is nilpotent, choose the least positive integer $n$ such that
    \[
      x^{\,n}=0.
    \]
    
    \bigskip
    \textbf{Step 2:  Construct a candidate for the inverse.}\newline
    Define
    \[
      y \;=\; 1 + x + x^{\,2} + \dots + x^{\,n-1}\;\in R.
    \]
    
    \bigskip
    \textbf{Step 3:  Verify the left inverse property.}\newline
    Multiply:
    \[
      (1 - x)\,y
      \;=\;
      (1 - x)\bigl(1 + x + x^{\,2} + \dots + x^{\,n-1}\bigr)
      \;=\;
      1 - x^{\,n}
      \;=\;
      1,
    \]
    because every intermediate term cancels in pairs and $x^{\,n}=0$.
    
    \bigskip
    \textbf{Step 4:  Verify the right inverse property.}\newline
    By associativity in~$R$,
    \[
      y\,(1 - x)
      \;=\;
      \bigl(1 + x + x^{\,2} + \dots + x^{\,n-1}\bigr)(1 - x)
      \;=\;
      1 - x^{\,n}
      \;=\;
      1.
    \]
    
    \bigskip
    \textbf{Step 5:  Conclude invertibility.}\newline
    The element $y$ satisfies
    \(
      (1-x)\,y = y\,(1-x) = 1,
    \)
    hence $y=(1-x)^{-1}$.
    Therefore
    \[
      \boxed{\,1 - x\ \text{is invertible in}\ R.}
    \]
    \end{proof}
\end{solution}
\end{document}
