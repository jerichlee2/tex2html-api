\documentclass[12pt]{article}

% Packages
\usepackage[margin=1in]{geometry}
\usepackage{amsmath,amssymb,amsthm}
\usepackage{enumitem}
\usepackage{hyperref}
\usepackage{xcolor}
\usepackage{import}
\usepackage{xifthen}
\usepackage{pdfpages}
\usepackage{transparent}
\usepackage{listings}
\usepackage{tikz}
\usepackage{physics}
\usepackage{siunitx}
\usepackage{booktabs}
\usepackage{cancel}
  \usetikzlibrary{calc,patterns,arrows.meta,decorations.markings}


\DeclareMathOperator{\Log}{Log}
\DeclareMathOperator{\Arg}{Arg}
\DeclareMathOperator{\ord}{Ord}
\DeclareMathOperator{\Aut}{Aut}


\lstset{
    breaklines=true,         % Enable line wrapping
    breakatwhitespace=false, % Wrap lines even if there's no whitespace
    basicstyle=\ttfamily,    % Use monospaced font
    frame=single,            % Add a frame around the code
    columns=fullflexible,    % Better handling of variable-width fonts
}

\newcommand{\incfig}[1]{%
    \def\svgwidth{\columnwidth}
    \import{./Figures/}{#1.pdf_tex}
}
\theoremstyle{definition} % This style uses normal (non-italicized) text
\newtheorem{solution}{Solution}
\newtheorem{proposition}{Proposition}
\newtheorem{problem}{Problem}
\newtheorem{lemma}{Lemma}
\newtheorem{theorem}{Theorem}
\newtheorem{remark}{Remark}
\newtheorem{note}{Note}
\newtheorem{definition}{Definition}
\newtheorem{example}{Example}
\newtheorem{corollary}{Corollary}
\theoremstyle{plain} % Restore the default style for other theorem environments
%

% Theorem-like environments
% Title information
\title{MATH 417 Practice Exam 2 Solution}
\author{Jerich Lee}
\date{\today}

\begin{document}

\maketitle
\begin{solution}
  Let 
  \[
     U_{8}\;=\;\{\,1,3,5,7\,\}
     \subset\bigl(\Bbb Z/8\Bbb Z\bigr)^{\times},
     \qquad 
     a\*b\;:=\;ab\pmod{8}.
  \]
  Because each element of $U_{8}$ is relatively prime to~$8$, the product
  of any two elements again lies in~$U_{8}$, so $(U_{8},\*)$ is a group.
  
  \bigskip
  \begin{enumerate}[]
  \item \textbf{Cayley table of $(U_{8},\*)$.}\;
        Compute every product modulo~$8$:
  
        \[
        \renewcommand{\arraystretch}{1.2}
        \begin{array}{c|cccc}
            \* & 1 & 3 & 5 & 7\\\hline
            1  & 1 & 3 & 5 & 7\\
            3  & 3 & 1 & 7 & 5\\
            5  & 5 & 7 & 1 & 3\\
            7  & 7 & 5 & 3 & 1
        \end{array}
        \]
  
  \item \textbf{Identity element.}\;
        From the first row (or column) we see that 
        $\boxed{\,1\,}$ satisfies $1\*a=a\*1=a$ for all $a\in U_{8}$, 
        so $1$ is the identity.
  
  \item \textbf{Inverses.}\;
        Inspecting the diagonal of the Cayley table (or using $a^{2}\equiv1\pmod8$ for every $a\in U_{8}$) gives
        \[
           1^{-1}=1,\qquad
           3^{-1}=3,\qquad
           5^{-1}=5,\qquad
           7^{-1}=7.
        \]
        Thus every non-identity element is its own inverse; the group
        $(U_{8},\*)$ is isomorphic to the Klein four group
        $C_{2}\times C_{2}$.
  \end{enumerate}
  \end{solution}
  \begin{problem}
    Let $p$ be prime and suppose $|G| = p^{2}$.
    \begin{enumerate}[]
       \item Prove that the center $Z(G)$ is non-trivial.
       \item Deduce that $G$ is abelian, and hence
             \[
                G \;\cong\; \Bbb Z_{p^{2}}
                \quad\text{or}\quad
                G \;\cong\; \Bbb Z_{p}\;\oplus\;\Bbb Z_{p}.
             \]
    \end{enumerate}
    \end{problem}
    
    \begin{solution}
    Recall the \emph{class equation}
    \[
       |G| \;=\; |Z(G)| \;+\;
                \sum_{i=1}^{k} [G : C_{G}(g_{i})],
    \]
    where $g_{1},\dots,g_{k}$ run through one representative from each
    non-central conjugacy class
    and $C_{G}(g_{i})=\{\,x\in G \mid xg_{i}=g_{i}x\,\}$ is the
    centraliser of~$g_{i}$.
    
    \bigskip
    \textbf{(a)  $Z(G)$ is non-trivial.}
    
    Since $|G|=p^{2}$, every index $[G : C_{G}(g_{i})]$ divides $|G|$ and is
    strictly larger than $1$ for non-central $g_{i}$.
    Hence each such index equals~$p$.
    
    \[
       \Longrightarrow\quad
       p^{2}=|G|
       \;=\;
       |Z(G)| + m\,p,
       \qquad m\in\Bbb Z_{\ge0}.
    \]
    Taking both sides modulo~$p$ gives $|Z(G)|\equiv0\pmod{p}$, so
    \[
       |Z(G)| \;=\; p \quad\text{or}\quad |Z(G)|=p^{2}.
    \]
    Either way $|Z(G)|>1$, hence $Z(G)\neq\{1\}$ is \emph{non-trivial}.
    
    \bigskip
    \textbf{(b)  $G$ is abelian and its classification.}
    
    \smallskip
    \emph{Case 1: $|Z(G)|=p^{2}$.}\;
    Then $Z(G)=G$ and $G$ is abelian.
    
    \smallskip
    \emph{Case 2: $|Z(G)|=p$.}\;
    Consider the quotient
    \[
       G/Z(G).
    \]
    Its order is
    \(
       |G|/|Z(G)| = p^{2}/p = p,
    \)
    so $G/Z(G)$ is \emph{cyclic}.
    Whenever a group is modded out by its centre and the quotient is
    cyclic, the group itself must be abelian:
    pick $xZ(G)$ as a generator;
    every element of $G$ can be written $x^{k}z$ with $z\in Z(G)$, and such
    elements commute.
    
    \medskip
    Consequently $G$ is abelian in \emph{all} cases.
    
    \paragraph{Structure of abelian groups of order $p^{2}$.}
    The classification theorem for finite abelian groups yields exactly two
    possibilities:
    \[
       G \;\cong\; \Bbb Z_{p^{2}}
       \quad\text{or}\quad
       G \;\cong\; \Bbb Z_{p}\;\oplus\;\Bbb Z_{p}.
    \]
    
    \smallskip
    \[
       \boxed{\;Z(G)\text{ is non-trivial and }G\text{ is abelian of one of
       the two above types.}\;}
    \]
    \end{solution}
    \begin{solution}
      \textbf{Notation.}  
      Let $\Bbb Z_{24}=\langle\overline{1}\rangle=\{\overline{0},\overline{1},
      \ldots,\overline{23}\}$ (bars denote residue classes modulo $24$).
      Because $\Bbb Z_{24}$ is cyclic, each divisor $d\mid24$ corresponds to
      a \emph{unique} subgroup of order $d$, namely
      \[
              H_{d}\;=\;\bigl\langle\,\overline{24/d}\,\bigr\rangle.
      \]
      The index of $H_{d}$ in $\Bbb Z_{24}$ is $[\Bbb Z_{24}:H_{d}]=24/d$.
      
      \bigskip
      \noindent
      \textbf{(a)  All subgroups of $\Bbb Z_{24}$ and their indices}
      
      \[
      \renewcommand{\arraystretch}{1.25}
      \begin{array}{c|c|c}
      \text{generator(s)} & |H_{d}| & [\Bbb Z_{24}:H_{d}]\\\hline
      \{\overline{0}\} & 1 & 24\\
      \langle\overline{12}\rangle & 2 & 12\\
      \langle\overline{8}\rangle  & 3 & 8\\
      \langle\overline{6}\rangle  & 4 & 6\\
      \langle\overline{4}\rangle  & 6 & 4\\
      \langle\overline{3}\rangle  & 8 & 3\\
      \langle\overline{2}\rangle  & 12 & 2\\
      \langle\overline{1}\rangle=\Bbb Z_{24} & 24 & 1
      \end{array}
      \]
      
      Each row gives the subgroup, its order, and its index in $\Bbb Z_{24}$.
      
      \bigskip
      \noindent
      \textbf{(b)  Subgroup lattice of $\Bbb Z_{24}$}
      
      Below is the Hasse diagram (cover–relation diagram) of the subgroup
      lattice.  An edge joins $H_{d}$ to $H_{e}$ precisely when
      $H_{e}\subset H_{d}$ and there is no subgroup strictly between them
      (\,equivalently, $e\mid d$ but $e\!\nmid d'$ for any $e<d'<d$\,).
      
      

      \emph{Reading the diagram.}  
      The top node represents the whole group (order $24$);
      moving downward corresponds to taking proper subgroups whose orders are
      divisors of the one above.  
      Indices can be read by comparing levels  
      (e.g.\ the subgroup of order $6$ has index $4$ in $\Bbb Z_{24}$).
      \end{solution}
      \begin{solution}
        \begin{enumerate}[]
        %---------------------------------------------------------------
        \item \textbf{Compute $7^{222}\pmod{13}$.}
        
        \[
        \varphi(13)=12
        \quad\Longrightarrow\quad
        7^{12}\equiv1\pmod{13}.
        \]
        \[
        222 = 12\cdot18 + 6
        \quad\Longrightarrow\quad
        7^{222}\equiv 7^{6}\pmod{13}.
        \]
        Now
        \[
        7^{2}=49\equiv10,\qquad
        7^{4}\equiv 10^{2}=100\equiv 9,\qquad
        7^{6}\equiv 7^{4}\!\cdot7^{2}\equiv9\cdot10=90\equiv12.
        \]
        Hence
        \[
        \boxed{\,7^{222}\equiv12\pmod{13}\;(-1\text{ mod }13)\, }.
        \]
        
        %---------------------------------------------------------------
        \item \textbf{Compute $9^{1000}\pmod{31}$.}
        
        \[
        \varphi(31)=30
        \quad\Longrightarrow\quad
        9^{30}\equiv1\pmod{31}.
        \]
        \[
        1000\equiv 10\pmod{30}
        \quad\Longrightarrow\quad
        9^{1000}\equiv 9^{10}\pmod{31}.
        \]
        Step-by-step powers:
        \[
        9^{2}=81\equiv19,\qquad
        9^{4}\equiv19^{2}=361\equiv20,\qquad
        9^{8}\equiv20^{2}=400\equiv28.
        \]
        \[
        9^{10}=9^{8}\cdot9^{2}\equiv28\cdot19=532\equiv5.
        \]
        Therefore
        \[
        \boxed{\,9^{1000}\equiv5\pmod{31}\, }.
        \]
        
        %---------------------------------------------------------------
        \item \textbf{Last two digits of $3^{123}$ (i.e.\ $3^{123}\pmod{100}$).}
        
        Since $\gcd(3,100)=1$,
        \[
        \varphi(100)=40
        \quad\Longrightarrow\quad
        3^{40}\equiv1\pmod{100}.
        \]
        (Indeed, already $3^{20}\equiv1\pmod{100}$, so $3^{40}\equiv1$ as well.)
        
        \[
        123 = 40\cdot3 + 3
        \quad\Longrightarrow\quad
        3^{123}\equiv 3^{3}=27\pmod{100}.
        \]
        
        Hence the last two digits are
        \[
        \boxed{\,27\, }.
        \]
        \end{enumerate}
        \end{solution}
        \begin{solution}
          Recall that \(\Bbb Z_{10}^{\times}\) is the set of residue classes
          \(\overline{a}\) modulo \(10\) with \(\gcd(a,10)=1\).
          
          \bigskip
          \textbf{(a)  All units of \(\Bbb Z_{10}\) and their inverses.}
          
          \[
          \renewcommand{\arraystretch}{1.3}
          \begin{array}{c|c}
          a\;(\text{mod }10) & a^{-1}\;(\text{mod }10) \\\hline
          1 & 1 \\ 
          3 & 7 \\ 
          7 & 3 \\ 
          9 & 9
          \end{array}
          \]
          
          \bigskip
          \textbf{(b)  Elements satisfying \(a^{-1}=a\).}
          
          From the table we see
          \[
             a^{-1}=a \quad\Longleftrightarrow\quad a\in\{\,1,\;9\,\}.
          \]
          
          Hence the only self–inverse units modulo \(10\) are
          \[
          \boxed{\;\overline{1}\text{ and }\overline{9}\;}.
          \]
          \end{solution}
          \begin{solution}
            Let $\varphi:G\longrightarrow H$ be a group homomorphism and
            fix $a\in G$.
            
            \medskip
            \noindent
            \textbf{Step 1.\;}  Write $n=\ord_{G}(a)$.  
            By definition, $n$ is the \emph{least positive} integer with
            $a^{\,n}=e_{G}$ (the identity in $G$).
            
            \medskip
            \noindent
            \textbf{Step 2.\;}  Apply $\varphi$ to the relation $a^{\,n}=e_{G}$:
            \[
               \varphi\!\bigl(a^{\,n}\bigr)
               =\varphi(e_{G})
               =e_{H}.
            \]
            Because $\varphi$ is a homomorphism,
            \[
               \varphi\!\bigl(a^{\,n}\bigr)
                  =\bigl(\varphi(a)\bigr)^{n}.
            \]
            Hence
            \[
               \bigl(\varphi(a)\bigr)^{n}=e_{H}.
            \]
            
            \medskip
            \noindent
            \textbf{Step 3.\;}  Let $m=\ord_{H}\!\bigl(\varphi(a)\bigr)$.  
            By definition, $m$ is the \emph{least positive} integer satisfying
            \(
               \bigl(\varphi(a)\bigr)^{m}=e_{H}.
            \)
            Since $\bigl(\varphi(a)\bigr)^{n}=e_{H}$, the integer $n$ is a positive
            multiple of~$m$.  
            Formally, the least–positive–integer property gives
            \[
               m \;\mid\; n
               \quad\Longleftrightarrow\quad
               \ord_{H}\!\bigl(\varphi(a)\bigr)
                 \;\mid\;
               \ord_{G}(a).
            \]
            
            \medskip
            \[
               \boxed{\;
                  \ord_{H}\!\bigl(\varphi(a)\bigr)
                  \,\mid\,
                  \ord_{G}(a)
                  \quad\text{for every }a\in G.
               \;}
            \]
            \end{solution}
            \begin{solution}
              \textbf{Goal.}\;
              Let
              \[
                 C_{n}
                    \;=\;
                 \Bigl\{\,e^{2\pi i k/n}\;\Bigm|\;k=0,1,\dots,n-1\Bigr\}
                 \;\subset\;\Bbb C^{\times}.
              \]
              We will show
              \begin{enumerate}[]
                 \item $C_{n}$ is a \emph{cyclic} subgroup of $(\Bbb C^{\times},\cdot)$;
                 \item $C_{n}\cong\Bbb Z_{n}$.
              \end{enumerate}
              
              \bigskip
              \textbf{1.\;Closure, identity, inverses.}
              For $k,\ell\in\{0,\dots,n-1\}$,
              \[
                 e^{2\pi i k/n}\,e^{2\pi i \ell/n}
                    =e^{2\pi i (k+\ell)/n}
                    \in C_{n},
              \]
              so $C_{n}$ is closed under multiplication.
              
              \[
                 e^{2\pi i\cdot 0/n}=1
                 \quad\text{acts as the identity,}\qquad
                 \bigl(e^{2\pi i k/n}\bigr)^{-1}=e^{-2\pi i k/n}=e^{2\pi i (n-k)/n}\in C_{n}.
              \]
              
              Hence $C_{n}$ is a finite subgroup of $\Bbb C^{\times}$ of size $n$.
              
              \bigskip
              \textbf{2.\;Cyclicity.}
              Set
              \[
                 \zeta=e^{2\pi i/n}\qquad(\text{a \emph{primitive} $n$th root of unity}).
              \]
              Its successive powers are
              \[
                 \zeta^{k}=e^{2\pi i k/n},\quad k=0,\dots,n-1,
              \]
              which are precisely the $n$ distinct elements of $C_{n}$.
              Thus
              \[
                 C_{n}=\langle\zeta\rangle
              \]
              and the group is cyclic, generated by $\zeta$.
              
              \bigskip
              \textbf{3.\;Explicit isomorphism with \(\Bbb Z_{n}\).}
              
              Define
              \[
                 \Phi:\Bbb Z\;\longrightarrow\;C_{n},
                 \qquad
                 \Phi(k)=\zeta^{\,k}=e^{2\pi i k/n}.
              \]
              \begin{itemize}
                 \item $\Phi$ is a homomorphism:
                       \(
                         \Phi(k+\ell)=\zeta^{k+\ell}=\zeta^{k}\zeta^{\ell}
                                    =\Phi(k)\Phi(\ell).
                       \)
                 \item $\ker\Phi=\{\,k\in\Bbb Z\mid\zeta^{\,k}=1\}
                                  =n\Bbb Z.
                       $
              \end{itemize}
              By the First Isomorphism Theorem,
              \[
                 \Bbb Z/\ker\Phi
                    \;\cong\;
                 \operatorname{im}\Phi
                    =C_{n}.
              \]
              Since $\ker\Phi=n\Bbb Z$, the quotient is
              \(
                 \Bbb Z_{n}=\Bbb Z/n\Bbb Z.
              \)
              
              \[
                 \boxed{\;
                   C_{n}\;\cong\;\Bbb Z_{n}
                 \;}
              \]
              and this bijection is concretely given by
              \(
                 \overline{k}\longmapsto e^{2\pi i k/n}.
              \)
              
              \end{solution}
              \begin{solution}
                \textbf{Step 1.\;Prime–power decomposition.}  
                \[
                   72 \;=\; 2^{3}\,3^{2}.
                \]
                
                \textbf{Step 2.\;Classify the $p$–primary parts.}  
                For a fixed prime power $p^{k}$, finite abelian $p$–groups correspond
                bijectively to \emph{partitions} of~$k$.
                
                \[
                \renewcommand{\arraystretch}{1.2}
                \begin{array}{c|c|c}
                p & k & \text{partitions of }k \Longrightarrow \text{$p$-groups}\\\hline
                2 & 3 & (3)\;\Longrightarrow\;\Bbb Z_{8}\\
                  &   & (2+1)\;\Longrightarrow\;\Bbb Z_{4}\oplus\Bbb Z_{2}\\
                  &   & (1+1+1)\;\Longrightarrow\;\Bbb Z_{2}\oplus\Bbb Z_{2}\oplus\Bbb Z_{2}\\\hline
                3 & 2 & (2)\;\Longrightarrow\;\Bbb Z_{9}\\
                  &   & (1+1)\;\Longrightarrow\;\Bbb Z_{3}\oplus\Bbb Z_{3}
                \end{array}
                \]
                
                \textbf{Step 3.\;Form all direct products.}  
                An abelian group of order $72=8\cdot9$
                is the direct product of one $2$–primary group (order $8$)
                and one $3$–primary group (order $9$).
                \[
                  \renewcommand{\arraystretch}{1.3}
                  \begin{array}{l|l|l}   % ← three columns now
                  \text{$2$-part} & \text{$3$-part} & \text{direct product}\\\hline
                  \Bbb Z_{8} & \Bbb Z_{9} & \Bbb Z_{8}\oplus\Bbb Z_{9}\;\cong\;\Bbb Z_{72}\\
                  \Bbb Z_{8} & \Bbb Z_{3}\oplus\Bbb Z_{3} 
                             & \Bbb Z_{8}\oplus\Bbb Z_{3}\oplus\Bbb Z_{3}
                               \;\cong\;\Bbb Z_{24}\oplus\Bbb Z_{3}\\\hline
                  \Bbb Z_{4}\oplus\Bbb Z_{2} & \Bbb Z_{9} 
                             & (\Bbb Z_{4}\oplus\Bbb Z_{9})\oplus\Bbb Z_{2}
                               \;\cong\;\Bbb Z_{36}\oplus\Bbb Z_{2}\\
                  \Bbb Z_{4}\oplus\Bbb Z_{2} & \Bbb Z_{3}\oplus\Bbb Z_{3}
                             & \Bbb Z_{4}\oplus\Bbb Z_{2}\oplus\Bbb Z_{3}\oplus\Bbb Z_{3}
                               \;\cong\;\Bbb Z_{6}\oplus\Bbb Z_{12}\\\hline
                  \Bbb Z_{2}\oplus\Bbb Z_{2}\oplus\Bbb Z_{2} & \Bbb Z_{9}
                             & (\Bbb Z_{2}\oplus\Bbb Z_{9})\oplus\Bbb Z_{2}\oplus\Bbb Z_{2}
                               \;\cong\;\Bbb Z_{18}\oplus\Bbb Z_{2}\oplus\Bbb Z_{2}\\
                  \Bbb Z_{2}\oplus\Bbb Z_{2}\oplus\Bbb Z_{2} & \Bbb Z_{3}\oplus\Bbb Z_{3}
                             & \Bbb Z_{2}\oplus\Bbb Z_{2}\oplus\Bbb Z_{2}\oplus
                               \Bbb Z_{3}\oplus\Bbb Z_{3}
                               \;\cong\;\Bbb Z_{2}\oplus\Bbb Z_{6}\oplus\Bbb Z_{6}
                  \end{array}
                  \] 
                
                \textbf{Step 4.\;Invariant–factor list (each divisor of the next).}
                \[
                \boxed{
                \begin{aligned}
                1.\;&\Bbb Z_{72}\\
                2.\;&\Bbb Z_{24}\;\oplus\;\Bbb Z_{3}\\
                3.\;&\Bbb Z_{36}\;\oplus\;\Bbb Z_{2}\\
                4.\;&\Bbb Z_{12}\;\oplus\;\Bbb Z_{6}\\
                5.\;&\Bbb Z_{18}\;\oplus\;\Bbb Z_{2}\;\oplus\;\Bbb Z_{2}\\
                6.\;&\Bbb Z_{6}\;\oplus\;\Bbb Z_{6}\;\oplus\;\Bbb Z_{2}
                \end{aligned}}
                \]
                
                \paragraph{Conclusion.}
                There are exactly \(\mathbf{6}\) abelian groups (up to isomorphism) of
                order \(72\), listed above in invariant-factor form.
                \end{solution}
                \begin{solution}
                  Let $p$ be an \emph{odd} prime and suppose $\lvert G\rvert = 2p$.
                  
                  \bigskip
                  \textbf{1.\;A unique Sylow–$p$ subgroup is normal.}
                  
                  Let $n_{p}$ be the number of Sylow–$p$ subgroups.  
                  Sylow’s theorems say
                  \[
                     n_{p}\;\mid\;2
                     \quad\text{and}\quad
                     n_{p}\equiv 1 \pmod{p}.
                  \]
                  Because $p$ is odd, the divisors of $2$ are $1,2$ and only
                  $n_{p}=1$ satisfies the congruence.  
                  Hence the Sylow–$p$ subgroup
                  \[
                     P=\;\text{the unique subgroup of order }p
                  \]
                  is \emph{normal} in $G$:
                  \[
                     P\;\trianglelefteq\;G.
                  \]
                  
                  \bigskip
                  \textbf{2.\;Write $G$ as a semidirect product $P\rtimes C_{2}$.}
                  
                  Pick an element $r\in G\smallsetminus P$; then $r^{2}=e$ by Lagrange,
                  so $\langle r\rangle\cong C_{2}$ and
                  \[
                     G = P\langle r\rangle,\qquad P\cap\langle r\rangle=\{e\},
                  \]
                  i.e.\ $G \cong P\rtimes_{\!\varphi} C_{2}$,
                  where the action is given by conjugation:
                  \[
                     \varphi:C_{2}\longrightarrow\Aut(P),
                     \qquad
                     \varphi(r)(x)=rxr^{-1}.
                  \]
                  Because $P\cong\Bbb Z_{p}$ is cyclic, 
                  $\Aut(P)\cong\Bbb Z_{p-1}$ is cyclic of order $p-1$.
                  
                  \bigskip
                  \textbf{3.\;Only two possible actions.}
                  
                  The image of $\varphi$ is a subgroup of $\Aut(P)$ whose order divides
                  both $\lvert C_{2}\rvert=2$ and $\lvert\Aut(P)\rvert=p-1$.
                  Hence
                  \[
                     \lvert\operatorname{im}\varphi\rvert = 1
                     \quad\text{or}\quad
                     \lvert\operatorname{im}\varphi\rvert = 2.
                  \]
                  
                  \smallskip
                  \emph{(i) Trivial action $\bigl(\operatorname{im}\varphi=\{1\}\bigr)$.}\;
                  Then $r$ commutes with $P$ and
                  \[
                     G \;\cong\; P\times\langle r\rangle
                               \;\cong\; \Bbb Z_{p}\times\Bbb Z_{2}
                               \;\cong\; \Bbb Z_{2p},
                  \]
                  because $\gcd(p,2)=1$ implies the direct product is cyclic.
                  
                  \smallskip
                  \emph{(ii) Non-trivial action of order $2$.}\;
                  Since $\Aut(\Bbb Z_{p})$ is cyclic, it has a \emph{unique} element of
                  order $2$, namely $x\mapsto x^{-1}$.  
                  Thus
                  \[
                     \varphi(r)(x)=x^{-1},\qquad x\in P,
                  \]
                  and $G$ has the presentation
                  \[
                     G
                       \;=\;
                     \Bigl\langle\,a,r\;\Bigm|\;
                                a^{p}=r^{2}=e,\; rar^{-1}=a^{-1}\Bigr\rangle
                       \;=\; D_{2p},
                  \]
                  the dihedral group of order $2p$.
                  
                  \bigskip
                  \textbf{4.\;Conclusion.}
                  Exactly two (and only two) isomorphism types occur:
                  \[
                     \boxed{%
                        G\;\cong\;\Bbb Z_{2p}
                        \quad\text{or}\quad
                        G\;\cong\;D_{2p}.}
                  \]
                  \end{solution}
                  \begin{solution}
                    \textbf{(a)  \(F=\Bbb Z_{2}[x]/(x^{3}+x+1)\) is a field of order \(8\).}
                    
                    \medskip
                    \emph{Irreducibility of \(x^{3}+x+1\) over \(\Bbb Z_{2}\).}  
                    A cubic over a field is irreducible iff it has no root in that field.
                    \[
                    f(x)=x^{3}+x+1\in\Bbb Z_{2}[x], 
                    \qquad
                    f(0)=1\neq0,\; f(1)=1+1+1=1\neq0,
                    \]
                    so \(f\) has no root in \(\Bbb Z_{2}\) and is therefore irreducible.
                    
                    \medskip
                    \emph{Consequences.}  
                    Because \(f\) is irreducible, the quotient ring
                    \(\Bbb Z_{2}[x]/(f)\) is a field.  
                    Its elements are the residue classes of polynomials of degree
                    \(\le2\), i.e.\ the \(2^{3}=8\) vectors
                    \[
                    \{\,0,1,x,x+1,x^{2},x^{2}+1,x^{2}+x,x^{2}+x+1\,\},
                    \]
                    so \(\lvert F\rvert = 8\).
                    
                    \bigskip
                    \textbf{(b)  Multiplicative order of \(\alpha=x+(x^{3}+x+1)\in F\).}
                    
                    Write simply \(\alpha=x\).  
                    Inside \(F\) we have the relation
                    \[
                       x^{3} = x+1
                       \quad\Longrightarrow\quad
                       x^{3}+x = 1.
                    \]
                    
                    \smallskip
                    \emph{Powers of \(\alpha\):}
                    \[
                    \begin{aligned}
                    \alpha^{1} &= x,\\
                    \alpha^{2} &= x^{2},\\
                    \alpha^{3} &= x^{3}=x+1,\\
                    \alpha^{4} &= \alpha^{3}\alpha = (x+1)x = x^{2}+x,\\
                    \alpha^{5} &= \alpha^{4}\alpha = (x^{2}+x)x = x^{3}+x^{2}= (x+1)+x^{2}=x^{2}+x+1,\\
                    \alpha^{6} &= \alpha^{5}\alpha = (x^{2}+x+1)x = x^{3}+x^{2}+x = (x+1)+x^{2}+x = x^{2}+1,\\
                    \alpha^{7} &= \alpha^{6}\alpha = (x^{2}+1)x = x^{3}+x = 1.
                    \end{aligned}
                    \]
                    
                    \emph{Minimality.}  
                    None of \(\alpha,\alpha^{2},\dots,\alpha^{6}\) equals \(1\); hence
                    \(\operatorname{ord}_{F^{\times}}(\alpha)=7\).
                    
                    \smallskip
                    \emph{Observation.}  
                    The multiplicative group \(F^{\times}\) has order \(8-1=7\) and is
                    cyclic. Because \(7\) is prime, \emph{every} non-zero element has order
                    \(7\).  Thus \(\alpha\) is a generator of \(F^{\times}\).
                    
                    \[
                    \boxed{\;\operatorname{ord}(\alpha)=7\;}
                    \]
                    \end{solution}
                    \begin{solution}
                      We work in the Euclidean domain $\Bbb Z[i]$ with norm
                      $N(z)=z\overline z=x^{2}+y^{2}$ for $z=x+iy$.
                      
                      \bigskip
                      \textbf{Euclidean algorithm.}
                      
                      \[
                      \renewcommand{\arraystretch}{1.2}
                      \begin{array}{r|c|c|c}
                      k & r_{k} & q_{k} & r_{k-1}=q_{k}r_{k}+r_{k+1}\\\hline
                      0 & a_{0}=15+11i &      & \\[2pt]
                      1 & a_{1}=4+9i   & q_{1}=2-i
                            & 15+11i=(2-i)(4+9i)+\boxed{(-2-3i)}\\[4pt]
                      2 & r_{1}=-2-3i  & q_{2}=-3
                            & 4+9i=(-3)(-2-3i)+\boxed{(-2)}\\[4pt]
                      3 & r_{2}=-2     & q_{3}=1+2i
                            & -2-3i=(1+2i)(-2)+\boxed{i}\\[4pt]
                      4 & r_{3}=i      &           & (-2)=(-i)\,i+0
                      \end{array}
                      \]
                      
                      The last non–zero remainder is a \emph{unit} $r_{3}=i$, so
                      \[
                      \gcd\bigl(15+11i,\;4+9i\bigr)\;=\;\pm1,\;\pm i.
                      \]
                      
                      \bigskip
                      \textbf{Bézout coefficients.}
                      
                      Back-substitute to express $i$ (and hence $1$) as a $\Bbb Z[i]$–linear
                      combination of the two inputs.
                      
                      \[
                      \begin{aligned}
                      i
                        &\;=\;(-2-3i) - (1+2i)(-2)\\[4pt]
                        &\;=\; \bigl(15+11i-(2-i)(4+9i)\bigr) - (1+2i)\bigl(4+9i+3(-2-3i)\bigr)\\[4pt]
                        &\;=\;(-2-6i)\,(15+11i) \;+\; (9+8i)\,(4+9i).
                      \end{aligned}
                      \]
                      
                      Multiply both sides by $-i$ to obtain the Bézout identity with right-hand
                      side $1$:
                      
                      \[
                      \boxed{\;
                      1 \;=\;(-6+2i)\,(15+11i) \;+\; (8-9i)\,(4+9i).
                      \;}
                      \]
                      
                      Thus $u=-6+2i$ and $v=8-9i$ satisfy
                      \[
                      u\,(15+11i) + v\,(4+9i)=1,
                      \]
                      confirming that the two Gaussian integers are coprime.
                      \end{solution}
                      \begin{solution}
                        \textbf{(a)  Exactly half of the permutations are even: \(\lvert A_{n}\rvert=\dfrac{n!}{2}\).}
                        
                        \smallskip
                        \emph{Sign homomorphism.}  
                        Define
                        \[
                        \varepsilon:S_{n}\;\longrightarrow\;\{\pm1\},\qquad
                        \varepsilon(\sigma)=
                          \begin{cases}
                             +1 &\text{if }\sigma\text{ is even},\\[-2pt]
                             -1 &\text{if }\sigma\text{ is odd}.
                          \end{cases}
                        \]
                        Because the parity of a product of permutations is the product of their
                        parities, \(\varepsilon\) is a group homomorphism.
                        
                        \smallskip
                        \emph{Kernel and image.}  
                        The kernel of \(\varepsilon\) is the alternating group
                        \(A_{n}\) (all even permutations).  
                        Transpositions show that \(\varepsilon\) is \emph{surjective}, so
                        \(\operatorname{im}\varepsilon=\{\pm1\}\) has order \(2\).
                        
                        \smallskip
                        \emph{Counting.}  
                        By the First Isomorphism Theorem,
                        \[
                        \lvert A_{n}\rvert
                           =\lvert\ker\varepsilon\rvert
                           =\frac{\lvert S_{n}\rvert}{\lvert\operatorname{im}\varepsilon\rvert}
                           =\frac{n!}{2}.
                        \]
                        
                        \[
                        \boxed{\;\lvert A_{n}\rvert=\dfrac{n!}{2}\;}
                        \]
                        
                        %--------------------------------------------------------------------
                        \bigskip
                        \textbf{(b)  The permutation \(\sigma=(12)(13)(1456)\) in \(S_{6}\).}
                        
                        \emph{Disjoint–cycle decomposition}  
                        (write cycles right-to-left):
                        
                        \[
                        \begin{aligned}
                        1&\mapsto4\mapsto5\mapsto6\mapsto3\mapsto2\mapsto1,\\
                        \text{all other points are included above}.
                        \end{aligned}
                        \qquad\Longrightarrow\qquad
                        \sigma=(1\,4\,5\,6\,3\,2).
                        \]
                        
                        \emph{Order of \(\sigma\).}  
                        A \(k\)-cycle has order \(k\); here \(k=6\), so
                        \[
                        \operatorname{ord}(\sigma)=6.
                        \]
                        
                        \emph{Parity of \(\sigma\).}  
                        A \(k\)-cycle is even iff \(k-1\) is even.  
                        Since \(6-1=5\) is odd, the \(6\)-cycle above is \emph{odd}.  
                        (Equivalently, \((12)\) and \((13)\) are odd, \((1456)\) is odd
                        \((4-1=3)\); the product of three odd permutations is odd.)
                        
                        \[
                        \boxed{\;
                           \sigma=(1\,4\,5\,6\,3\,2),\quad
                           \operatorname{ord}(\sigma)=6,\quad
                           \sigma\text{ is odd}.
                        \;}
                        \]
                        \end{solution}
                        \begin{solution}
                          Recall the quaternion group
                          \[
                             Q_{8}= \{\pm1,\;\pm i,\;\pm j,\;\pm k\},
                             \qquad
                             i^{2}=j^{2}=k^{2}=ijk=-1 .
                          \]
                          
                          %-----------------------------------------------------------------
                          \bigskip
                          \textbf{(a) Conjugacy classes in \(Q_{8}\).}
                          
                          Because
                          \(\langle -1\rangle\subset Z(Q_{8})\), multiplication by \(-1\)
                          takes each element to a central one, so \(\pm x\) always belong to the
                          \emph{same} class.  Direct calculation (or symmetry) gives
                          \[
                             \boxed{\;
                               \{1\},\;
                               \{-1\},\;
                               \{\,i,-i\,\},\;
                               \{\,j,-j\,\},\;
                               \{\,k,-k\,\}.
                             \;}
                          \]
                          
                          %-----------------------------------------------------------------
                          \bigskip
                          \textbf{(b) The centre and the class equation.}
                          
                          Every element commutes with \(\pm1\), while
                          \(i,j,k\) fail to commute with each other.
                          Hence
                          \[
                             Z(Q_{8})=\{\,\pm1\,\},
                             \quad\text{so }|Z(Q_{8})|=2.
                          \]
                          Using the conjugacy–class sizes found in~(a):
                          \[
                             1 + 1 + 2 + 2 + 2 \;=\; 8
                             \quad\Longrightarrow\quad
                             |Z(Q_{8})|+\sum\limits_{\text{non-central classes}}|C|
                             =|Q_{8}|,
                          \]
                          verifying the class equation.
                          
                          %-----------------------------------------------------------------
                          \bigskip
                          \textbf{(c) The automorphism group \(\Aut(Q_{8})\).}
                          
                          \begin{enumerate}[label=\arabic*., leftmargin=*]
                          \item An automorphism is completely determined by the images of \(i\)
                                and \(j\), provided those images \emph{generate} \(Q_{8}\) and
                                satisfy the relation
                                \(\varphi(j)\,\varphi(i)=\varphi(k)=-\varphi(i)\,\varphi(j)\).
                          
                          \item The set
                                \(\{\pm i,\pm j,\pm k\}\) contains \(6\) elements of order \(4\).
                                To generate \(Q_{8}\) we may send
                                \[
                                    i \longmapsto \epsilon_{1}x, \qquad
                                    j \longmapsto \epsilon_{2}y,
                                \]
                                where
                                \(x\neq \pm y\) are chosen from \(\{i,j,k\}\)
                                and \(\epsilon_{1},\epsilon_{2}\in\{\pm1\}\).
                                \begin{itemize}
                                   \item There are \(3\cdot2=6\) ordered pairs
                                         \((x,y)\) with \(x\neq y\).
                                   \item For each such pair there are \(2\cdot2=4\) sign choices.
                                \end{itemize}
                                Hence \(|\Aut(Q_{8})| = 6\cdot4 = 24\).
                          
                          \item Composition with the natural action on the set
                                \(\{\pm i,\pm j,\pm k\}/\{\pm1\}=\{\,\{\pm i\},\{\pm j\},\{\pm k\}\}\)
                                gives a faithful homomorphism
                                \[
                                    \Aut(Q_{8})\;\hookrightarrow\;S_{4},
                                \]
                                whose image has order \(24\); therefore
                                \[
                                    \boxed{\;
                                       \Aut(Q_{8})\;\cong\;S_{4},\qquad
                                       |\Aut(Q_{8})|=24 .
                                    \;}
                                \]
                          \end{enumerate}
                          \end{solution}
\end{document}
