\documentclass[hidelinks,11pt]{article}
\usepackage{geometry}
\geometry{letterpaper}

\usepackage{amssymb,amsmath,cancel}
\numberwithin{equation}{subsection}
\usepackage[labelsep=period]{caption}
\usepackage{subcaption}

\usepackage{tikz} 
\usepackage{afterpage} 
\usepackage{titlesec}

\usepackage{amssymb,amsmath} %Standard libraries that add lots of equation features and math symbols
%\numberwithin{equation}{subsection} %This makes equations numbered by section rather than sequentially overall. 
\usepackage[labelsep=period]{caption}
\usepackage{subcaption}
\usepackage{hyperref} %Adds some useful tricks with references
\hypersetup{ %Makes all sections in ToC, figure and table numbers, and references clickable
    colorlinks=false, %set true if you want colored links
    linktoc=all,     %set to all if you want both sections and subsections linked
}
\usepackage{setspace}
\usepackage{fullpage}
\usepackage{lipsum} % generates filler text for template
\usepackage{tikz} %Package for making diagrams
\usepackage{afterpage} 
\usepackage{paralist,multirow}
\usetikzlibrary{decorations.pathreplacing, calc} 
\usetikzlibrary{snakes,shapes,decorations.text}

\newcommand{\degrees}{\ensuremath{^\circ}} %use this command to insert a degree symbol in text mode

\setlength{\parindent}{2em}
\setlength{\parskip}{0em}
\renewcommand{\baselinestretch}{1}\normalsize

 %setup spacing between sections and sizes of sections and subsection
\titlespacing*{\section}
{0pt}{2.5ex plus 1ex minus .2ex}{1.0ex plus .2ex}
\titleformat{\section}
  {\normalfont\fontsize{14}{14}\bfseries}{\thesection}{1em}{}
\titleformat{\subsection}
  {\normalfont\fontsize{11}{11}\bfseries}{\thesubsection}{1em}{}

\begin{document}
\noindent {\LARGE \bfseries Status Report} \\[0.2cm]
\begin{tabular}{l l}
\textbf{To:} & Sponsor Name \\
\textbf{Date:} & Date Here \\
\textbf{From:} & University of Illinois at Urbana-Champaign ME470 Team \# N:\\
& Student A, Student B, Student C, Student D, Student E
\end{tabular}\\[0.2cm]
\noindent {\Large \bfseries Long Title of Project} 


\section*{Purpose} %Ensure all section/subsections include asterisks like this so that the sections don't get numbered. 

This is a draft of a~\LaTeX ~template for use for the status report in ME470 at UIUC. If you find any bugs or have any suggestions, please email Christopher Marry at \href{mailto:cmarry2@illinois.edu}{cmarry2@illinois.edu}. This template is a modified version of my proposal report from when I took senior design in Spring 2016, and was originally created by James Buckland. Future versions of this document will be intended to serve as both a template and a~\LaTeX~tutorial for a undergraduate student. Some comments have been added throughout to provide explanation of how the template works, but they are not yet comprehensive. I recommend that you use this template only if someone on your team already has some~\LaTeX~knowledge since the learning curve is somewhat steep.

\section*{Introduction}

\lipsum[1-2]

\section*{Discussion}

\subsection*{Topic 1}

Reference a figure like this: Fig.~\ref{fig:theta}. Another figure is Fig. \ref{fig:MELmap}. The last figure is Fig.~\ref{fig:cat}. Reference a table like this: Table~\ref{table:deliverables}. Another table is Table \ref{table:budget}. Make sure every figure/table is referenced in text somewhere.
Include a citation like this:~\cite{me470site}. Cite multiple sources at once like this: \cite{Nguyen2015,Hovakimyan2010}

\subsection*{Topic 2}

\lipsum[1]

\subsection*{Topic 3}

\lipsum[1-2]

\section*{Conclusion}

\lipsum[1-2]
\pagebreak
%Place all your figure and table setup text after your body text to encourage LaTeX to put them after all your text. Use the [p] location specifier to force figures/tables onto special float pages

\begin{figure}[p]
%This is an example of creating subfigures. For more details and more complex examples, see the package documentation:
%http://mirror.las.iastate.edu/tex-archive/obsolete/macros/latex/contrib/subfigure/subfigure.pdf
  \centering
  \begin{subfigure}[b]{0.7\textwidth}
    \centering
		\includegraphics[width=1\textwidth]{fig/plot-zplots.png}
    \caption{$z$}
    \label{fig:plot-z}
  \end{subfigure}
  \\[1cm]
  \begin{subfigure}[b]{0.7\textwidth}
    \centering
		\includegraphics[width=1\textwidth]{fig/plot-thetafittedvsoriginal.png}
    \caption{$\theta$}
    \label{fig:plot-theta}
  \end{subfigure}
  \caption{Comparison of Original and Fitted $z$ and $\theta$}
  \label{fig:theta}
\end{figure}


\begin{figure}[h] 
  \centering %usually you'll want your figure to be centered. 
	\includegraphics[width=0.6\textwidth]{fig/MELmap.png} % Replace fig/MELmap.png with the local path to your image file. Change the number in front of \textwidth to change the size of the figure as appropriate
    \caption{Map of the first floor of the Mechanical Engineering Laboratory at UIUC, and also a demonstration of how to create a basic figure}
    \label{fig:MELmap} %this is the name that you use when referencing the figure
\end{figure}

\begin{figure}[p]
  \centering
		\includegraphics[width=0.9\textheight,angle=90]{fig/cat.jpg} 
    \caption{This is my cat, Vincent, who is a very good boy. This is also a demonstration of how to have a rotated figure, such as for a Gantt Chart.}
    \label{fig:cat}
\end{figure}

\begin{table}[p]
    \centering
    \caption{Deliverables}
		\begin{tabular} { r | r l l}
			\textbf{Deliverable} & \textbf{Lead} & \textbf{Start Date} & \textbf{Completion Date} \\ \hline
			Constructing Half-Scale Machine & Student A & 2/4 & 2/18 \\
			Refining Half-Scale Machine & Student B & 3/5 & 3/9 \\
			Status Report & Student C,D & 3/5 & 3/15 \\
			First Iteration of Full-Scale Machine & Student E & 2/23 & 3/21 \\
			Refinements to Full-Scale Design & Student A & 3/21 & 4/6 \\
			Final Report & All & 4/6 & 5/6 \\
		\end{tabular}
		    \label{table:deliverables}

\end{table}


\begin{table}[p]
  \centering
    \caption{Budget}
		\begin{tabular}{r | l l l}
		\multirow{2}{*}{\textbf{Item}} & \textbf{Unit Price} & \textbf{Quantity} & \textbf{Total} \\
		 & \$/Unit & Unit & \$ \\ \hline
		 Item1  	& 0.11/ft   & 100ft & 11.00 \\
		 Item2 		& 2.10		& 4		& 8.40 \\
		 Item3		& 29.98		& 1		& 29.98 \\
		 Item4		& 299.98	& 1		& 299.68 \\
		 Item5 		& 12.97		& 4		& 51.88 \\
		 Item6		& 22.50		& 3		& 22.50 \\
		 Item7	    & 24.95		& 1		& 24.95 \\
		 Item8 		& 8.49		& 1		& 8.49 \\
		 Item9		& 10.00		& 1		& 10.00 \\
		 \hline
		 \textbf{Machining Cost}			&59.00/hr& 3 hr		&\$177.00 \\
		 \textbf{Total Cost}			&		& 	& \$643.88
		\end{tabular}
		\label{table:budget}

\end{table}

\bibliographystyle{IEEEtran}
\bibliography{status} 

\end{document}
