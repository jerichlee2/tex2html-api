\documentclass[9pt]{article}

% Packages
\usepackage[margin=.5in]{geometry}
\usepackage{amsmath,amssymb,amsthm}
\usepackage{enumitem}
\usepackage{hyperref}
\usepackage{xcolor}
\usepackage{import}
\usepackage{xifthen}
\usepackage{pdfpages}
\usepackage{transparent}
\usepackage{listings}


\lstset{
    breaklines=true,         % Enable line wrapping
    breakatwhitespace=false, % Wrap lines even if there's no whitespace
    basicstyle=\ttfamily,    % Use monospaced font
    frame=single,            % Add a frame around the code
    columns=fullflexible,    % Better handling of variable-width fonts
}

\newcommand{\incfig}[1]{%
    \def\svgwidth{\columnwidth}
    \import{./Figures/}{#1.pdf_tex}
}
\theoremstyle{definition} % This style uses normal (non-italicized) text
\newtheorem{solution}{Solution}
\newtheorem{proposition}{Proposition}
\newtheorem{problem}{Problem}
\newtheorem{lemma}{Lemma}
\newtheorem{theorem}{Theorem}
\newtheorem{remark}{Remark}
\newtheorem{note}{Note}
\newtheorem{definition}{Definition}
\newtheorem{example}{Example}
\theoremstyle{plain} % Restore the default style for other theorem environments
%

% Theorem-like environments
% Title information
\title{}
\author{Jerich Lee}
\date{\today}

\begin{document}
\section*{ME 470 Spindle Vibration Progress Report Outline}

\begin{enumerate}
    \item \textbf{Title page}
    \item \textbf{Overview}
    \begin{enumerate}
        \item Team Introduction
        \item Project Background
        \item Solution Procedure
        \item Ideation
        \item Testing
        \item Simulation
        \item Summary
        \item Questions
    \end{enumerate}

    \item \textbf{Team Introduction}
    \item \textbf{Meet the Team} 
    \begin{itemize}
        \item Justin Frank
        \item Eron Mahmudi
        \item Jerich Lee
        \item Liam Wydra
        \item Ryan Vandewiele
        \item Zachary Balakrishnan
    \end{itemize}

    \item \textbf{Project Background}

    \item \textbf{Who is Peddinghaus, Inc.?}
    \begin{itemize}
        \item Structural steel machinery
        \item PeddiSubX-1120 milling machine
    \end{itemize}

    \item \textbf{PeddiSubX-1120 Layout}

    \item \textbf{What is the Problem?}
    \begin{itemize}
        \item Excess vibration during asymmetric loading
        \item Impact: Damage, reduced quality
    \end{itemize}

    \item \textbf{Solution Procedure}

    \item \textbf{Project Goal}
    \begin{itemize}
        \item Easy to implement, low maintenance, budget: \$1500
    \end{itemize}

    \item \textbf{Principal Tasks \& Deliverables}
    \begin{itemize}
        \item Root cause analysis
        \item Research and testing
        \item CAD modeling and simulation
    \end{itemize}

    \item \textbf{Project Timeline}
    \begin{itemize}
        \item Project Proposal: 2/17/2025
        \item Mid-Semester Status Presentations: 4/1/2025
        \item Trade Show: 4/22/2025
        \item Final Presentations: 5/6/2025
    \end{itemize}

    \item \textbf{Peddinghaus, Inc. Site Visit}
    \item \textbf{Vibration Research}
    \item \textbf{Test Plan Development} 
    \item \textbf{Simulation}
    \item \textbf{Success Criterion}  
    \item \textbf{Ideation}  
    \item \textbf{Bearing Layout Change}
    \item \textbf{Symmetrical Spindle Slide}
    \item \textbf{Motor Repositioning}
    \item \textbf{Spindle Housing Material Change}
    \item \textbf{Spindle Material Change}
    \item \textbf{Concept Selection}
    \item \textbf{Testing Round 1}
    \item \textbf{Budget}
    \item \textbf{Budget}
    \item \textbf{Summary}
    \item \textbf{Summary}
    \item \textbf{Questions}
    \item \textbf{Contact info}
    \item \textbf{References}
    
    
    \item \textbf{Activities Completed}
    \begin{itemize}
        \item Torsion bar manufacturing and accelerometer acquisition
        \item Impact and operational tests
        \item Qualitative simulation
        \item Literature review
    \end{itemize}

    \item \textbf{Detailed Test Data}
    \begin{itemize}
        \item Time Series Data: No Tool, X-axis
        \item Time Series Data: 10-inch Tool, X-axis
        \item Time Series Data: No Tool, Y-axis
        \item Time Series Data: 10-inch Tool, Y-axis
        \item Time Series Data: Torsion Bar
        \item Spindle Unloaded Operation
        \item Spindle Loaded Operation
        \item Wavelet Analysis: Long Tool
        \item Wavelet Analysis: Short Tool
        \item Wavelet-Time Series Overlay
    \end{itemize}

    \item \textbf{Simulation Overview}
    \begin{itemize}
        \item Simplified models
        \item Boundary conditions
        \item Modal and frequency analysis
    \end{itemize}

    \item \textbf{Mesh Refinement}
    \begin{itemize}
        \item Goal: skewness $<$ 0.33
    \end{itemize}

    \item \textbf{Current Activities}
    \begin{itemize}
        \item CAD model creation and simulation improvements
        \item Experimental frequency and wavelet analysis
        \item Boundary condition revision
    \end{itemize}

    \item \textbf{Planned Activities}
    \begin{itemize}
        \item Data analysis
        \item Simulation refinement
        \item Begin CAD designs for solutions
        \item Test Session \#2 (tentative)
    \end{itemize}

    \item \textbf{References}

    \item \textbf{Backup Slides}

    \item \textbf{Machine Axes System}

    \item \textbf{Impact Test Details}
    \begin{itemize}
        \item Impact Test: 10-inch Tool, X-Direction
        \item Impact Test: 10-inch Tool, Y-Direction
        \item Impact Test: Torsional Mode
        \item Impact Test: No Tool, X-Direction
        \item Impact Test: No Tool, Y-Direction
    \end{itemize}

    \item \textbf{ISO Standards}
    \begin{itemize}
        \item ISO 20816-3 Success Criteria
        \item ISO 2954 Requirements: Mounting and vibration criteria
    \end{itemize}

    \item \textbf{References}

\end{enumerate}
\end{document}