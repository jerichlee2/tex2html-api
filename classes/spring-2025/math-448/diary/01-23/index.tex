\documentclass[12pt]{article}

% Packages
\usepackage[margin=1in]{geometry}
\usepackage{amsmath,amssymb,amsthm}
\usepackage{enumitem}
\usepackage{hyperref}
\usepackage{xcolor}
\usepackage{import}
\usepackage{xifthen}
\usepackage{pdfpages}
\usepackage{transparent}
\usepackage{listings}


\lstset{
    breaklines=true,         % Enable line wrapping
    breakatwhitespace=false, % Wrap lines even if there's no whitespace
    basicstyle=\ttfamily,    % Use monospaced font
    frame=single,            % Add a frame around the code
    columns=fullflexible,    % Better handling of variable-width fonts
}

\newcommand{\incfig}[1]{%
    \def\svgwidth{\columnwidth}
    \import{./Figures/}{#1.pdf_tex}
}
\theoremstyle{definition} % This style uses normal (non-italicized) text
\newtheorem{solution}{Solution}
\newtheorem{proposition}{Proposition}
\newtheorem{problem}{Problem}
\newtheorem{lemma}{Lemma}
\newtheorem{theorem}{Theorem}
\newtheorem{remark}{Remark}
\newtheorem{note}{Note}
\theoremstyle{plain} % Restore the default style for other theorem environments
%

% Theorem-like environments
% Title information
\title{}
\author{Jerich Lee}
\date{\today}

\begin{document}

\maketitle
\begin{align}
    2\cos\left[ \sum_{k=1}^{n} kx^{k} +\sum_{k=1}^{n} x^{k} \right] - \cos\theta \left[ \sum_{k=1}^{n} 2k+\sum_{k=1}^{n} 1 \right]  
\end{align}

    \[
\boxed{
2 \cos\!\biggl(
\frac{2\,x \;-\; x^2 \;-\; (n+2)\,x^{n+1} \;+\; (n+1)\,x^{n+2}}
{(1-x)^2}
\biggr)
\;-\;
\cos(\theta)\,\bigl[n(n+2)\bigr].
}
\]
\[
\tan^{-1}\Bigl(\tfrac{1}{\sqrt{3}}\Bigr) \;=\; \frac{\pi}{6}\quad\text{(radians)} \;=\; 30^\circ.
\]

\section*{Explanation of the Step}

We want to see how
\[
4\Bigl\{\lvert z\rvert^2 - 2\,\mathrm{Re}\bigl(z\overline{i}\bigr) + \lvert i\rvert^2 \Bigr\}
\;=\;
\lvert z\rvert^2 - 2\,\mathrm{Re}(z) + \lvert 1\rvert^2
\]
simplifies to
\[
3 \lvert z\rvert^2 \;-\; 8y \;+\; 2x \;=\;-3.
\]

\begin{enumerate}
\item \textbf{Write $z$ in terms of $x$ and $y$.}

Let
\[
z \;=\; x + i\,y, 
\quad 
\lvert z\rvert^2 = x^2 + y^2,
\quad
\mathrm{Re}(z) = x.
\]

\item \textbf{Compute $\mathrm{Re}\bigl(z\overline{i}\bigr)$.}

Since $i$ has the complex conjugate $\overline{i} = -\,i$, we get:
\[
z\overline{i} = (x + i\,y)\cdot(-\,i) = -\,i\,x \;-\; y.
\]
Hence
\[
\mathrm{Re}\bigl(z\overline{i}\bigr) \;=\; \mathrm{Re}\bigl(-\,i\,x \;-\; y \bigr)
\;=\;
-\,y.
\]

\item \textbf{Expand the left-hand side.}

\[
4\Bigl\{\lvert z\rvert^2 - 2\,\mathrm{Re}\bigl(z\overline{i}\bigr) + \lvert i\rvert^2\Bigr\}
=
4\Bigl\{(x^2 + y^2) \;-\; 2(-\,y) \;+\; 1 \Bigr\}
=
4\bigl(x^2 + y^2 + 2y + 1\bigr).
\]
So
\[
= 4x^2 + 4y^2 + 8y + 4.
\]

\item \textbf{Expand the right-hand side.}

\[
\lvert z\rvert^2 - 2\,\mathrm{Re}(z) + \lvert 1\rvert^2
=
(x^2 + y^2) \;-\; 2x + 1.
\]

\item \textbf{Set them equal and simplify.}

Equating both sides,
\[
4x^2 + 4y^2 + 8y + 4
\;=\;
x^2 + y^2 - 2x + 1.
\]
Rearrange:
\[
4x^2 - x^2 + 4y^2 - y^2 + 8y + 4 - 1 + 2x 
= 0
\quad\Longrightarrow\quad
3x^2 + 3y^2 + 2x + 8y + 3 = 0.
\]
Since $\lvert z\rvert^2 = x^2 + y^2$, we rewrite:
\[
3(x^2 + y^2) + 2x + 8y + 3 = 0
\;\;\Longrightarrow\;\;
3\,\lvert z\rvert^2 + 2x + 8y = -3.
\]
Or equivalently,
\[
3 \lvert z\rvert^2 \;-\; 8y \;+\; 2x 
\;=\;
-3.
\]

\end{enumerate}

\section*{Derivation in Complex Notation}

We wish to see why
\[
4\Bigl(\lvert z\rvert^{2} \;-\; 2\,\mathrm{Re}[\,z\,\overline{i}\,] \;+\; \lvert i\rvert^{2}\Bigr)
\;=\;
\lvert z\rvert^{2} \;-\; 2\,\mathrm{Re}(z) \;+\; 1
\]
simplifies to
\[
3\,\lvert z\rvert^{2} \;-\;8\,\mathrm{Im}(z) \;+\;2\,\mathrm{Re}(z)
\;=\;
-3.
\]

\begin{enumerate}
\item
Recall that 
\[
\lvert z\rvert^{2} \;=\; z\,\overline{z}, 
\quad
\lvert i\rvert^{2} \;=\; i\,\overline{i} \;=\; i\,(-\,i) \;=\;1,
\]
\[
\mathrm{Re}(z) \;=\;\frac{z + \overline{z}}{2},
\quad
\mathrm{Im}(z) \;=\;\frac{z - \overline{z}}{2\,i}.
\]

\item
Next, observe that 
\[
\overline{i} = -\,i,
\quad
\text{so}
\quad
\mathrm{Re}\bigl(z\,\overline{i}\bigr) 
\;=\;
\mathrm{Re}\bigl(z\,(-\,i)\bigr)
\;=\;
\mathrm{Re}(-\,i\,z).
\]
A useful identity is
\[
\mathrm{Re}(w) \;=\;\frac{w + \overline{w}}{2}.
\]
Applying this to $w = -\,i\,z$ shows 
\(\mathrm{Re}(z\,\overline{i}) = -\,\mathrm{Im}(z).\)
(Indeed, multiplying by $-\,i$ rotates $z$ by 90 degrees in the complex plane, turning the imaginary part of $z$ into a negative real part.)

\item
\textbf{Left-Hand Side (LHS).}
\[
4\Bigl(\lvert z\rvert^{2} \;-\; 2\,\mathrm{Re}[z\,\overline{i}] \;+\;\lvert i\rvert^{2}\Bigr)
\;=\;
4\Bigl(z\,\overline{z} \;-\;2\bigl(-\,\mathrm{Im}(z)\bigr)\;+\;1\Bigr).
\]
Hence
\[
= 4\Bigl(z\,\overline{z} + 2\,\mathrm{Im}(z) + 1\Bigr)
\;=\;
4\,z\,\overline{z} \;+\; 8\,\mathrm{Im}(z) \;+\;4.
\]

\item
\textbf{Right-Hand Side (RHS).}
\[
\lvert z\rvert^{2} \;-\;2\,\mathrm{Re}(z) \;+\;\lvert 1\rvert^{2}
\;=\;
z\,\overline{z} \;-\;2\,\mathrm{Re}(z) \;+\;1.
\]

\item
\textbf{Equate and rearrange.}

Setting LHS $=$ RHS:
\[
4\,z\,\overline{z} \;+\;8\,\mathrm{Im}(z) \;+\;4
\;=\;
z\,\overline{z} \;-\;2\,\mathrm{Re}(z) \;+\;1.
\]
Bring everything to one side:
\[
4\,z\,\overline{z} - z\,\overline{z}
\;+\;
8\,\mathrm{Im}(z)
\;+\;
4 - 1
\;+\;
2\,\mathrm{Re}(z)
\;=\;0.
\]
That is,
\[
3\,z\,\overline{z} \;+\; 2\,\mathrm{Re}(z) \;+\; 8\,\mathrm{Im}(z) \;+\;3 \;=\;0.
\]
Rewriting $z\,\overline{z}$ as $\lvert z\rvert^{2}$, we get
\[
3\,\lvert z\rvert^{2} \;+\;2\,\mathrm{Re}(z) \;+\;8\,\mathrm{Im}(z)
\;=\;-3.
\]
Often one writes
\[
3\,\lvert z\rvert^{2} 
\;-\;8\,\mathrm{Im}(z) 
\;+\;2\,\mathrm{Re}(z) 
\;=\;-3,
\]
depending on sign conventions for $\mathrm{Im}(z)$.
\end{enumerate}

\end{document}
