\documentclass[12pt]{article}

% Packages
\usepackage[margin=1in]{geometry}
\usepackage{amsmath,amssymb,amsthm}
\usepackage{enumitem}
\usepackage{hyperref}
\usepackage{xcolor}
\usepackage{import}
\usepackage{xifthen}
\usepackage{pdfpages}
\usepackage{transparent}
\usepackage{listings}


\lstset{
    breaklines=true,         % Enable line wrapping
    breakatwhitespace=false, % Wrap lines even if there's no whitespace
    basicstyle=\ttfamily,    % Use monospaced font
    frame=single,            % Add a frame around the code
    columns=fullflexible,    % Better handling of variable-width fonts
}

\newcommand{\incfig}[1]{%
    \def\svgwidth{\columnwidth}
    \import{./Figures/}{#1.pdf_tex}
}
\theoremstyle{definition} % This style uses normal (non-italicized) text
\newtheorem{solution}{Solution}
\newtheorem{proposition}{Proposition}
\newtheorem{problem}{Problem}
\newtheorem{lemma}{Lemma}
\newtheorem{theorem}{Theorem}
\newtheorem{remark}{Remark}
\newtheorem{note}{Note}
\newtheorem{definition}{Definition}
\newtheorem{example}{Example}
\newtheorem{corollary}{Corollary}
\theoremstyle{plain} % Restore the default style for other theorem environments
%

% Theorem-like environments
% Title information
\title{Exam 2 MATH 448 Practice Questions}
\author{Jerich Lee}
\date{\today}

\begin{document}

\maketitle
\section*{Exercises for Section 1.6}

Compute the following line integrals.
\begin{enumerate}
    \item $\int_\gamma z \, dz$, where $\gamma$ is the semicircle from $i$ to $-i$, which passes through $-1$.
    \item $\int_\gamma e^z \, dz$, where $\gamma$ is the line segment from $0$ to $z_0$.
    \item $\int_\gamma |z|^2 \, dz$, where $\gamma$ is the line segment from $2$ to $3 + i$.
    \item $\int_\gamma 1/(z + 4) \, dz$, where $\gamma$ is the circle of radius $1$ centered at $-4$, oriented counterclockwise.
    \item $\int_\gamma \text{Re}(z) \, dz$, where $\gamma$ is the line segment from $1$ to $i$.
    \item $\int_\gamma (z^2 + 3z + 4) \, dz$, where $\gamma$ is the circle $|z| = 2$ oriented counterclockwise.
    \item Let $\gamma_1$ be the semicircle from $1$ to $-1$ through $i$ and $\gamma_2$ the semicircle from $1$ to $-1$ through $-i$ (Fig.\ 1.36).
\end{enumerate}


Compute $\int_{\gamma_1} z^2 \, dz$ and $\int_{\gamma_2} z^2 \, dz$. Can you account for the fact that they are equal?

Now compute $\int_{\gamma_1} \bar{z} \, dz$ and $\int_{\gamma_2} \bar{z} \, dz$. Can you account for the fact that they are not equal?

\begin{enumerate}
    \setcounter{enumi}{8}
    \item Explicitly verify the conclusion of Example 11 by computing $\int_\gamma z \, dz$ and $\int_\gamma z^2 \, dz$ when $\gamma$ is the square with vertices at $\pm 1 \pm i$.
    \item 
    \begin{enumerate}
        \item Show that 
        \[
        \frac{1}{2\pi} \int_0^{2\pi} e^{i k \theta} \, d\theta = 
        \begin{cases}
        1, & k = 0 \\
        0, & k \ne 0
        \end{cases}
        \]
        \item Explicitly verify the conclusion of Example 13 for $\gamma$, the circle of radius $1$ centered at $0$. (Hint:
        \[
        \frac{1}{z - p} = \sum_{k = 0}^\infty \frac{p^k}{z^{k + 1}} \quad \text{if } 0 \leq |p| < 1 \text{ and } |z| = 1
        \]
        \[
        \frac{1}{z - p} = -\sum_{k = 0}^\infty \frac{z^k}{p^{k + 1}} \quad \text{if } |z| = 1 \text{ and } 1 < |p| < \infty
        \]
        )
    \end{enumerate}
    \item Interchange summation and integration; then use (a).)
    \item Let $f = u + iv$ be a continuous function and $\gamma(t) = x(t) + i y(t)$ be a piecewise smooth curve. Show that
    \[
    \text{Re} \left( \int_\gamma f(z) \, dz \right) = \int_\gamma (u \, dx - v \, dy)
    \]
    and
    \[
    \text{Im} \left( \int_\gamma f(z) \, dz \right) = \int_\gamma (v \, dx + u \, dy).
    \]
    Here, $dx = x'(t) \, dt$, $dy = y'(t) \, dt$.
    \item Use Exercise 10 to show the equivalence of the two formulations of Green’s Theorem: (7) and (8).
    \item Use the result of (the extension of) Example 11 to compute the following integrals:
    \begin{enumerate}
        \item $\int_\gamma (z^3 - 6z^2 + 4) \, dz$, where $\gamma$ is any curve joining $-1 + i$ to $1$.
        \item $\int_\gamma (z^2 + z^3) \, dz$, where $\gamma$ is any curve joining $-1$ to $2 + i$.
    \end{enumerate}
    \item Use Green’s Theorem to derive Green’s Identity:
    \[
    \iint_\Omega \left( \frac{\partial u}{\partial x} \frac{\partial v}{\partial x} + \frac{\partial u}{\partial y} \frac{\partial v}{\partial y} \right) dx \, dy = \int_{\partial \Omega} v \frac{\partial u}{\partial n} \, ds - \iint_\Omega v \Delta u \, dx \, dy.
    \]
    \item Use Exercise 13 to derive Green’s Formula (9):
    \[
    \int_{\partial \Omega} \left( g \frac{\partial f}{\partial n} - f \frac{\partial g}{\partial n} \right) ds = \iint_\Omega (g \Delta f - f \Delta g) \, dx \, dy.
    \]
    \item Let $u$ be a continuous function on the complex plane, which is bounded: $|u(z)| \leq C$ for all $z$. Let $\gamma_R$ be the circle $|z| = R$. Show that
    \[
    \lim_{R \to \infty} \int_{\gamma_R} \frac{u(z)}{(z - z_0)^m} \, dz = 0,
    \]
    for each $z_0$. (Hint: Use (3).)
    \item Let $z_0$ be outside a piecewise smooth simple closed curve $\gamma$. Extend Example 13 by showing that
    \[
    \int_\gamma \frac{dz}{(z - z_0)^m} = 0, \quad m = 2, 3, 4, \ldots
    \]
\end{enumerate}

\section*{Exercises for Section 2.1}

\begin{enumerate}
    \item Establish the following differentiation formulas:
    \begin{enumerate}
        \item $(\sin z)' = \cos z$
        \item $(\cos z)' = -\sin z$
        \item $(\sinh z)' = \cosh z$
        \item $(\cosh z)' = \sinh z$
        \item $(\tan z)' = \sec^2 z$
        \item $(\tanh z)' = 1 - \tanh^2 z$
        \item $(\arctan z)' = \frac{1}{1 + z^2}$
        \item $(\arcsin z)' = \frac{1}{(1 - z^2)^{1/2}}$
    \end{enumerate}
    Refer to Section 5 of Chapter 1 for the definitions of these trigonometric functions.
    
    \item Use the rules for differentiation, formulas (2) to (6), to find the derivative of each of the functions in Exercises 2 to 7.
    
    \item $z^2 + 10z$
    
    \item $\exp(z^2 - 2)$
    
    \item $[\cos(z^2)]^3$
    
    \item $(z^2 + 100)^{-4}$
    
    \item $(\log z)^3$ on the plane minus the negative reals
    
    \item $\sinh(e^z)$
    
    \item For each function $f$ listed in Exercises 8 to 11, find an analytic function $F$ with $F' = f$:
    \begin{enumerate}
        \item $f(z) = z^{-2}$
        \item $f(z) = \frac{z + 1}{z^2 + 1}$
        \item $f(z) = \sin z \cos z$
    \end{enumerate}
    
    \item Let $f$ and $g$ be analytic on a domain. Show that $(a)$ if $g$ is analytic on $D$ and $f = g'$, then $f$ is analytic on $D$, and $(b)$ if $f = \phi' + i\psi$, i.e., $f/g$ is analytic at all points $z_0$, where $g(z_0) \ne 0$, and
    \[
    \left( \frac{f}{g} \right)' = \frac{f'g - fg'}{g^2}.
    \]
    
    \item Show that if $f$ is analytic on a domain $D$ and $g$ is analytic on a domain containing the range of $f$, then $g(f(z))$ is analytic on $D$, and the chain rule holds:
    \[
    (g(f(z)))' = g'(f(z)) f'(z).
    \]
    
    \item Let $P(z) = (z - z_1) \cdots (z - z_n)$ where $A$ and $z_1, \dots, z_n$ are complex numbers and $A \ne 0$. Show that
    \[
    \frac{P'(z)}{P(z)} = \sum_{k=1}^n \frac{1}{z - z_k}, \quad z \ne z_1, \dots, z_n.
    \]
    
    \item Let $f$ be analytic on a domain $D$ and suppose that $f'(z) = 0$ for all $z \in D$. Show that $f$ is constant on $D$.
    
    \item Find the derivative of the linear fractional transformation $T(z) = \frac{az + b}{cz + d}, \, ad - bc \ne 0$. In what way does the condition $ad - bc \ne 0$ enter? Conclude that if $T(z)$ is never zero, $z^n = d/c$.
    
    \item Suppose that $f$ is analytic on a domain $D$ and $f'(z) = e^{\psi(z)}$, where $\psi$ is a constant. Show that $f(z) = C \exp(\psi z)$, $C$ a constant. (Hint: Consider $g(z) = e^{-\psi z} f(z)$ and use Exercise 15 or 19.)
    
    \item Show that $h(z) = \bar{z}$ is not analytic on any domain. (Hint: Check the Cauchy–Riemann equations.)
    
    \item Fill in the details to make the following argument correct. Let $w = g(z) = \log z$ and $w_0 = \log z_0$. Then
    \[
    g'(z_0) = \lim_{z \to z_0} \frac{\log z - \log z_0}{z - z_0} = \lim_{w \to w_0} \frac{w - w_0}{e^w - e^{w_0}} = \left( \lim_{w \to w_0} \frac{1}{e^w} \right) = \frac{1}{z_0}.
    \]
    
    \item Let $f = u + iv$ be analytic. In each of the following, find $v$ given $u$.
    \begin{enumerate}
        \item $u = x^2 - y^2$
        \item $u = \frac{x}{x^2 + y^2}$
        \item $u = x^2 + 2x + 1 - y^2$
        \item $u = \cosh y \sin x$
        \item $u = \sinh x \cos y$
    \end{enumerate}
    
    \item Let $\gamma$ be a piecewise smooth simple closed curve, and suppose that $F$ is analytic on some domain containing $\gamma$. Show that
    \[
    \int_\gamma F(z) \, dz = 0.
    \]
    
    \item Use the conclusion of Exercise 21 and Example 13 of Section 6, Chapter 1, to prove that $f(z) = \log z$ cannot be analytic on any domain $D$ that contains a piecewise smooth simple closed curve $\gamma$ that surrounds the origin. (Hint: What is the value of $\int_\gamma f(z) \, dz$?)
    
    \item Show that if $f$ is analytic on a domain $D$ and if the range of $f$ lies in either a straight line or a circle, then $f$ is a constant.
    
    \item If $f(z)$ is a differentiable function of the real variable $t$, $a \le t \le b$, and if $g$ is an analytic function on a domain $D$ that contains the range of $z(t)$, $a \le t \le b$, show that $g(z(t)) = w(t)$ is a differentiable function of $t$ and
    \[
    w'(t) = g'(z(t)) z'(t).
    \]
\end{enumerate}

\subsection*{Cauchy–Riemann Equations in Polar Coordinates}
\begin{enumerate}
    \setcounter{enumi}{24}
    \item Suppose $f = u + iv$ is analytic in a domain $D$. Show that the Cauchy–Riemann equations in polar coordinates are
    \[
    \frac{\partial u}{\partial r} = \frac{1}{r} \frac{\partial v}{\partial \theta}, \quad 
    \frac{\partial v}{\partial r} = -\frac{1}{r} \frac{\partial u}{\partial \theta}.
    \]

    \item Suppose that $\gamma$ is a piecewise smooth simple closed curve and $u$ is a continuous function on $\gamma$. Let $D$ be a domain disjoint from $\gamma$, and define a function $h$ on $D$ by the rule
    \[
    h(z) = \frac{1}{2\pi i} \int_\gamma \frac{u(\zeta)}{\zeta - z} \, d\zeta, \quad z \in D.
    \]
    Show that $h$ is analytic on $D$.
\end{enumerate}

\section*{Exercises for Section 2.2}

In Exercises 1 to 6, use Theorem 2 or Example 4 to find the radius of convergence of the given power series.
\begin{enumerate}
    \item $\sum_{k=1}^\infty k(z - 1)^k$
    \item $\sum_{k=0}^\infty \frac{(k!)^2}{(2k)!} z^k$
    \item $\sum_{n=0}^\infty \frac{z^j}{3^j}$
    \item $\sum_{k=0}^\infty (2k - 1)^k z^k$
    \item $\sum_{k=1}^\infty 5^{-1/2^k} z^k$
    \item $\sum_{k=1}^\infty \frac{(2k)(2k - 2) \cdots 4 \cdot 2}{(2k - 1)(2k - 3) \cdots 3 \cdot 1} z^k$
\end{enumerate}

In Exercises 7 to 13, find the power series about the origin for the given function.
\begin{enumerate}
    \setcounter{enumi}{6}
    \item $e^{-z}$
    \item $z^2 \cos z$
    \item $\frac{z^3}{1 - z^3}, \quad |z| < 1$
    \item $\frac{1}{1 + \frac{1}{z}}, \quad |z| > 1$
    \item $\frac{1}{(4 - z^2)^2}, \quad |z| < 4$ \hfill (Hint: $\frac{1}{(a - z)^2} = \frac{d}{dz} \left( \frac{1}{a - z} \right)$)
    \item $\frac{1}{(a - z)^2}$
    \item $\sinh(z^2)$
\end{enumerate}

In Exercises 14 to 18, find a closed form (i.e., a simple expression) for each of the given power series.
\begin{enumerate}
    \setcounter{enumi}{13}
    \item $\sum_{n=0}^\infty \frac{z^n}{n!}$
    \item $\sum_{n=1}^\infty n(2 - 1)^n z^n$ \hfill (Hint: Divide by $z^2$)
    \item $\sum_{n=0}^\infty (-1)^n z^{2n}$
    \item $\sum_{n=1}^\infty n(n - 1) z^{n - 2}$
    \item $\sum_{n=1}^\infty (-1)^n (z - 2\pi i)^n$
\end{enumerate}
\noindent
\begin{enumerate}
    \item (a) Suppose that the radius of convergence of the power series $f(z) = \sum_{n=0}^\infty a_n (z - z_0)^n$ is $R$, $R > 0$. Show that the radius of convergence of 
    \[
    F(z) = \sum_{n=0}^\infty \frac{a_n}{n+1} (z - z_0)^{n+1}
    \]
    is at least $R$.
    
    (b) Show that $F'(z) = f(z)$ for all $z$ with $|z - z_0| < R$.
    
    (c) Use parts (a) and (b) to show that 
    \[
    \log(1 - z) = -z - \frac{z^2}{2} - \frac{z^3}{3} - \frac{z^4}{4} - \cdots, \quad \text{if } |z| < 1.
    \]
    
    \item Establish the binomial formula for any pair $(a, b)$ of complex numbers:
    \[
    (a + b)^n = \sum_{j=0}^n \binom{n}{j} a^{n - j} b^j, \quad \binom{n}{j} = \frac{n!}{j!(n-j)!}.
    \]
    (Hint: One technique is to prove the formula by induction on $n$.)
    
    \item Show that
    \begin{enumerate}
        \item $\lim_{n \to \infty} \sqrt[n]{n} = 1$
        \item $\lim_{n \to \infty} n^p r^n = 0$ if $|r| < 1$
        \item $\lim_{n \to \infty} \sqrt[n]{M} = 1$ if $M$ is positive.
    \end{enumerate}
    (Hint: For (a), write $\sqrt[n]{n} = 1 + \epsilon_n$, $\epsilon_n > 0$; then $\ln(1 + \epsilon_n) > \frac{1}{n}$ and $\frac{n(1 + \epsilon_n)^2}{2} > \epsilon_n$ by Exercise 20.)
    
    \item (a) If $f(z) = \sum_{n=0}^\infty a_n (z - z_0)^n$ has radius of convergence $R > 0$ and if $f(z) = 0$ for all $z$, $|z - z_0| < r \leq R$, show that $a_0 = a_1 = \cdots = 0$.
    
    (b) If $F(z) = \sum_{n=0}^\infty a_n (z - z_0)^n$ and $G(z) = \sum_{n=0}^\infty b_n (z - z_0)^n$ are equal on some disc $|z - z_0| < r$, show that $a_n = b_n$ for all $n$.
    
    \item Let $a_0, a_1, a_2, \dots$ be complex numbers with either
    \[
    \lim_{n \to \infty} \sqrt[n]{|a_n|} = L \quad \text{or} \quad \lim_{n \to \infty} \left| \frac{a_{n+1}}{a_n} \right| = L.
    \]
    Show that the series 
    \[
    \sum_{n=0}^\infty \frac{a_n}{(z - z_0)^n}
    \]
    converges absolutely for all $z$ with $|z - z_0| > L$ and diverges for all $z$ with $|z - z_0| < L$.
    
    \item Let the radius of convergence of $f(z) = \sum_{n=0}^\infty a_n (z - z_0)^n$ be $R$, $0 < R \le \infty$, and let $r < R$. Set
    \[
    f_n(z) = \sum_{k=1}^n a_k (z - z_0)^k, \quad n = 1, 2, \ldots.
    \]
    Given $\epsilon > 0$, show that there is an $N$ such that $|f_n(z) - f(z)| < \epsilon$ for all $z$ with $|z - z_0| \le r$ and $n \ge N$.
    
    (Hint: The series $\sum_{j=0}^\infty |a_j| r^j$ converges, so there is an $N$ such that $\sum_{j=N+1}^\infty |a_j| r^j < \epsilon$. Hence, for $n \ge N$ and $|z - z_0| \le r$, we have:
    \[
    |f_n(z) - f(z)| = \left| \sum_{j=n+1}^\infty a_j (z - z_0)^j \right| \le \sum_{j=n+1}^\infty |a_j| r^j < \epsilon.
    \])
    
    \item Suppose that $\alpha$ is a real number but not a nonnegative integer. Define $\binom{\alpha}{j}$ for each nonnegative integer $j$ by:
    \[
    \binom{\alpha}{0} = 1, \quad \binom{\alpha}{1} = \alpha,
    \]
    \[
    \binom{\alpha}{j} = \frac{\alpha(\alpha - 1) \cdots (\alpha - j + 1)}{j!}, \quad j = 1, 2, \ldots.
    \]
    Show that 
    \[
    \left( \binom{\alpha}{j+1} \middle/ \binom{\alpha}{j} \right) \to 1 \quad \text{as } j \to \infty.
    \]
    
    \item Show that the radius of convergence of the series
    \[
    F(z) = \sum_{k=0}^\infty \binom{\alpha}{k} z^k
    \]
    is 1.
    
    \item (a) Show that the function $F$ defined in Exercise 26 satisfies the differential equation:
    \[
    (1 + z) F'(z) = \alpha F(z).
    \]
    (Hint: Compute $F'$, then multiply it by $(1 + z)$ and add terms with equal powers of $z$.)
    
    (b) Conclude from (a) that $F(z) = (1 + z)^\alpha$, that is,
    \[
    (1 + z)^\alpha = \sum_{k=0}^\infty \binom{\alpha}{k} z^k, \quad |z| < 1.
    \]
    
    \subsection*{General Formula for the Radius of Convergence of a Power Series}
    Let $\{x_n\}$ be a bounded sequence of nonnegative real numbers. Define
    \[
    \limsup_{n \to \infty} x_n = \rho
    \]
    to be the largest number $\rho$ such that each interval $(\rho - \epsilon, \rho + \epsilon)$ contains $x_n$ for infinitely many indices $n$.
    
    \item (a) Show that there are integers $\{n_j\}_{j=1}^\infty$ increasing to $\infty$ such that
    \[
    \lim x_{n_j} = \rho.
    \]
    
    (b) If $\{n_j\}_{j=1}^\infty$ is a sequence of integers increasing to $\infty$ and if $\lim_{j \to \infty} x_{n_j} = \sigma$, show that $\sigma \le \rho$.
    
    \item Let $\sum a_n (z - z_0)^n$ be a power series with radius of convergence $R$. Show that
    \[
    \frac{1}{R} = \limsup_{n \to \infty} \sqrt[n]{|a_n|}.
    \]
\end{enumerate}
\section*{Exercises for Section 2.3}

In Exercises 1 to 4, evaluate the given integral using Cauchy’s Formula or Theorem.
\begin{enumerate}
    \item $\int_{|z|=1} \frac{z}{(z - 2)^2} \, dz$
    \item $\int_{|z-3|=2} \frac{e^z}{z(z - 3)} \, dz$
    \item $\int_{|z+1|=2} \frac{z^2}{4 - z^2} \, dz$
    \item $\int_{|z|=1} \frac{\sin z}{z} \, dz$
\end{enumerate}

In Exercises 5 to 8, evaluate the definite trigonometric integral making use of the technique of Examples 6 and 7 in this section.
\begin{enumerate}
    \setcounter{enumi}{4}
    \item $\int_0^{2\pi} \frac{d\theta}{2 + \cos \theta}$
    \item $\int_0^{2\pi} \frac{d\theta}{3 + \sin \theta + \cos \theta}$
    \item $\int_0^{2\pi} \frac{d\theta}{a + b \cos \theta}, \quad a > b > 0$
    \item $\int_0^{2\pi} \frac{d\theta}{1 + \sin^2 \theta}$
\end{enumerate}

In Exercises 9 to 12, evaluate the given integral using the technique of Example 10; indicate which theorem or device you used to obtain your answer.
\begin{enumerate}
    \setcounter{enumi}{8}
    \item $\int_\gamma \frac{dz}{z^2}$, where $\gamma$ is any curve in $\text{Re}\, z > 0$ joining $1 - i$ to $1 + i$.
    \item $\int_\gamma \left( z + \frac{1}{z} \right) dz$, where $\gamma$ is any curve in $\text{Im}\, z > 0$ joining $-4 + i$ to $6 + 2i$.
    \item $\int_\gamma e^z \, dz$, where $\gamma$ is the semicircle from $-1$ to $1$ passing through $i$.
    \item $\int_\gamma \sin z \, dz$, where $\gamma$ is any curve joining $i$ to $\pi$.
\end{enumerate}
\noindent
\begin{enumerate}
    \item Integrate $e^{iz^2}$ around the contour $\gamma$ shown in Figure 2.8 to obtain the Fresnel integrals:
    \[
    \int_0^\infty \cos(x^2) \, dx = \int_0^\infty \sin(x^2) \, dx = \sqrt{\frac{\pi}{8}}.
    \]
    
    \item Specialize Theorem 4 to the case when $z$ is the center of the circle and show that
    \[
    f(z) = \frac{1}{2\pi} \int_0^{2\pi} f(z + re^{it}) \, dt.
    \]
    
    \item 
    \begin{enumerate}
        \item Use the estimate (3), Section 6, Chapter 1, and (5) above to conclude that
        \[
        |f(z)| \le \max_{0 \le t \le 2\pi} |f(z + re^{it})|
        \]
        for all sufficiently small $r$.
        
        \item Conclude from (a) that $|f|$ cannot have a strict local maximum within its domain of analyticity. That is, the graph in three-dimensional space of $(x, y, |f(x + iy)|)$ has no “peaks.”
    \end{enumerate}
    
    \item Use the conclusion of Exercise 15 to establish the following result. If $f$ is analytic and never zero on a domain $D$, then $|f(z)|$ has no local minima in $D$. That is, the graph $(x, y, |f(x + iy)|)$ has no “pits.”
    
    \item Let $f(z) = \sum_{j=0}^\infty a_j(z - z_0)^j$ have a radius of convergence of $R > 0$. Show that for each $r \in (0, R)$,
    \[
    f^{(j)}(z_0) = j! a_j = \frac{1}{2\pi i} \int_{|z - z_0| = r} \frac{f(z)}{(z - z_0)^{j+1}} \, dz, \quad j = 0, 1, 2, \ldots.
    \]
    (Hint: Substitute the series into the integral and interchange summation and integration.)
    
    \item Use the result of Exercise 17 to establish these formulas:
    \begin{enumerate}
        \item $\frac{1}{2\pi} \int_0^{2\pi} \sum_{n=0}^\infty \sin(e^{i\theta}) \, d\theta = 
            \begin{cases}
            0, & n = 0, 2, 4, \ldots \\
            \frac{(-1)^{(n - 1)/2}}{n!}, & n = 1, 3, 5, \ldots
            \end{cases}$
        \item $\frac{1}{2\pi} \int_0^{2\pi} z^n e^{-i\theta} e^{i\theta} d\theta = \frac{1}{n!}, \quad n = 0, 1, 2, \ldots$
        \item $\frac{1}{2\pi} \int_0^{2\pi} e^{im(2e^{i\theta} - e^{i(1+\theta)})^{-1}} d\theta = 
            \begin{cases}
            0, & n < m \\
            \frac{1}{2^{m - n}}, & n \ge m
            \end{cases}$
    \end{enumerate}
    
    \item Let $A$ be the annulus $A = \{ z : 0 < r \le |z| < R \}$. Suppose that $g$ is an analytic function on $A$ (and that $g'$ is continuous) with the property that
    \[
    \int_\gamma g(z) \, dz = 0
    \]
    where $\gamma$ is the positively oriented circle of radius $s$ centered at the origin, and $s$ is some positive number in the open interval $(r, R)$. Show that there is an analytic function $G$ on $A$ with $G' = g$ throughout $A$.
    
    \item Let $I = \int_{-\infty}^\infty e^{-x^2} \, dx$. Then:
    \[
    I^2 = \int_{-\infty}^\infty \int_{-\infty}^\infty e^{-(x^2 + y^2)} \, dx \, dy = \int_0^{2\pi} \int_0^\infty e^{-r^2} r \, dr \, d\theta = \pi.
    \]
    Hence, $I = \sqrt{\pi}$.
    
    \subsection*{The Poisson Kernel}
    
    \item Verify each of the statements (a) through (e) below for $z = re^{i\theta}$ and $r < 1$:
\end{enumerate}

\begin{enumerate}
    \item $\text{Re} \left( \frac{e^{it} + z}{e^{it} - z} \right) = \frac{1 - r^2}{1 - 2r \cos(\theta - t) + r^2}$
    \item $\text{Re} \left( \frac{e^{it} + z}{e^{it} - z} \right) = \text{Re} \left( \frac{e^{it}}{e^{it} - z} + \frac{z e^{it}}{1 - z e^{it}} \right)$
    \item $\frac{1}{2\pi} \int_0^{2\pi} \frac{f(e^{it})}{1 - z e^{it}} dt = 0$, if $f(w)$ is analytic for $|w| < 1 + \epsilon$
    \item $\frac{1}{2\pi} \int_0^{2\pi} \frac{f(e^{it}) e^{it}}{e^{it} - z} \, dt = f(z)$
    \item Add (c) and (d); then use (a) and (b) to conclude that
    \[
    f(z) = \frac{1}{2\pi} \int_0^{2\pi} f(e^{it}) \cdot \frac{1 - r^2}{1 - 2r \cos(\theta - t) + r^2} \, dt, \quad z = re^{i\theta}.
    \]
\end{enumerate}
\section*{Exercises for Section 2.4}

In Exercises 1 to 8, give the order of each of the zeros of the given function.
\begin{enumerate}
    \item $\sin z$
    \item $(e^z - 1)^3$
    \item $(z^2 + 1 - 2^z)^3$
    \item $(z - 4z + 4)^3$
    \item $z(\cos z - 1)$
    \item $\log(1 - z^3)$, \quad $|z| < 1$
    \item $\frac{z}{e^z - 1}$
    \item $\frac{z^2}{z^2 + 1}$
\end{enumerate}

In Exercises 9 to 16, find the power-series expansion about the given point for each of the functions, the first four digits of which are visible.

\begin{enumerate}
    \setcounter{enumi}{8}
    \item $e^{z^{-1}}$ about $z = \infty$
    \item $\frac{1}{z - 1}$ about $z = 0$
    \item $\frac{1}{z^2 + 1}$ about $z_0 = i$
    \item $\frac{1}{z^2 + 1}$ about $z_0 = 1$
    \item $\frac{2z}{z^2 + 1}$ about $z_0 = -1$
    \item $\frac{2z}{z^2 + 1}$ about $z_0 = 1$
    \item $\log(1 + z)$ about $z_0 = 0$ (first four terms)
    \item $\sin z$ about $z_0 = \pi$
\end{enumerate}

\noindent
\begin{enumerate}
    \item Suppose that $f$ is analytic on a domain $D$ and has a zero of order $m$ at $z_0$ in $D$. Show that (a) $f'(z)$ has a zero of order $m - 1$ at $z_0$; (b) $f^2$ has a zero of order $2m$ at $z_0$.

\item Suppose that $f$ is analytic on $D$ and satisfies the Cauchy estimate:
\[
|f^{(n)}(z_0)| \le \frac{n!}{r^n} \max_{|z - z_0| = r} |f(z)|, \quad n = 0, 1, 2, \dots
\]
whenever $f$ is analytic on a domain containing the set $\{ z : |z - z_0| \le r \}$.

\item Use the Cauchy estimates (Exercise 18) or Theorem 18.1 to give another proof of Liouville’s Theorem by showing that the derivative of a bounded entire function is identically zero.

\item Suppose that $f$ is an entire function and $\text{Re}(f(z)) < e^{-|z|}$ for all $z$. Show that $f$ is constant. (Hint: Consider $\exp(f(z))$)

\item Suppose that $f$ is an entire function and that there are positive constants $A$ and $m$ with $|f(z)| \le A|z|^m$ if $|z| \ge R_0$. Show that $f$ is a polynomial of degree at most $m$. (Hint: Use the Cauchy estimates (Exercise 18) for $r > R$ and let $r \to \infty$.)

\item Let $D$ be a simply-connected domain and $f$ an analytic function on $D$ that has no zeros in $D$. Let $\phi$ be a complex number. $\gamma \ne 0$. Show that there is an analytic function on $D$ with $e^{\phi} = f$. (Hint: Use Application 2.)

\item Suppose that $f$ is analytic in the region $|z| > R$, including at $\infty$ (that is, $G(z) = f(1/z)$ is analytic for $|z| < 1/R$). Show that $f$ can be expressed as a power series in $1/z$:
\[
f(z) = \sum_{k=1}^\infty \frac{c_k}{z^k}, \quad |z| > R,
\]
and derive a formula for $c_k$, similar to that given in Theorem 1.

\item Use Morera’s Theorem and an interchange of the order of integration to show that each of the following functions is analytic on the indicated domain; find a power-series expansion for each function by using the known power series for the integrand and interchanging the summation and integration.
\begin{enumerate}
    \item $\int_0^z \frac{1 - t^2}{1 + t^2} dt$, all $z$
    \item $\int_0^z \frac{dt}{1 + t^2}$, all $z$
    \item $\int_0^z \log(1 - tz) \, dt$, on $|z| < 2$
    \item $\int_0^z \log(1 + t) dt$, all $z$
    \item $\int_0^z \sin(t^2 + 3t) dt$, all $z$
\end{enumerate}

\subsection*{Differential Equations in the Complex Plane}

\item Find all solutions to the differential equation:
\[
f^{(4)}(z) + \beta^4 f(z) = 0, \quad \text{$f$ is an entire function}.
\]
(Hint: Write $f(z) = \sum a_j z^j$ and solve for the coefficients $a_3, a_7, \ldots$ in terms of $a_0, a_1$, and $\beta$.)

\item Use the technique of Exercise 25 to give the solutions of these differential equations:
\begin{enumerate}
    \item $f^{(2)}(z) - \frac{3}{z} f'(z) + 2f(z) = 0$, \quad $a_0 = 1$, $a_1 = 2$
    \item $f^{(2)}(z) - \frac{2}{z} f'(z) = 0$, \quad $a_0 = 1$, $f(0) = 1$, $f'(0) = 0$
    \item $f^{(2)}(z) + f'(z) = 0$
\end{enumerate}

\item If $f$ and $A$ are analytic in a simply-connected domain $D$ and
\[
f'(z) = A(z) f(z),
\]
show that $f(z) = C \exp\left(\int_\gamma A(w) dw\right)$ for a constant $C$, where the integral is taken over any piecewise smooth curve joining a fixed point $z_0$ to $z$. (Hint: Let
\[
g(z) = \exp\left(-\int_\gamma A(w) dw\right)
\]
and show that $(f(z) g(z))' = 0$ in $D$.)

\subsection*{Bessel Functions}

\item Let $\nu$ be an integer, $\nu > 0$. Show that one solution of the differential equation
\[
z^2 f^{(2)}(z) + z f'(z) + (z^2 - \nu^2) f(z) = 0
\]
is
\[
J_\nu(z) = \sum_{m=0}^\infty \frac{(-1)^m}{m! \, \Gamma(m + \nu + 1)} \left( \frac{z}{2} \right)^{2m + \nu}.
\]
$J_\nu$ is the Bessel function of the first kind of order $\nu$.

\item Show that the Bessel functions satisfy the recurrence relation:
\[
J_{\nu - 1}(z) - J_{\nu + 1}(z) = 2J_\nu'(z).
\]
\end{enumerate}
\section*{Exercises for Section 2.5}

In Exercises 1 to 6, locate each of the isolated singularities of the given function and tell whether it is a removable singularity, a pole, or an essential singularity. If the singularity is removable, give the value of the function at the point; if the singularity is a pole, give the order of the pole.
\begin{enumerate}
    \item $\frac{e^z - 1}{z}$
    \item $\frac{z^2}{\sin z}$
    \item $\frac{z^4 - 2z^2 + 1}{(z - 1)^2}$
    \item $\cot \pi z$
    \item $\frac{2z + 1}{z + 2}$
    \item $\frac{e^z}{z^2 - 1}$
\end{enumerate}

In Exercises 7 to 13, find the Laurent series for the given function about the indicated point. Also, give the residue of the function at the point.
\begin{enumerate}
    \setcounter{enumi}{6}
    \item $\frac{e^z - 1}{z^2}, \quad z_0 = 0$
    \item $\frac{z^2}{z^3 - 1}, \quad z_0 = 1$
    \item $\frac{\sin z}{(z - \pi)^2}, \quad z_0 = \pi$
    \item $\frac{\sin z}{z^2}, \quad z_0 = 0$ (four terms of the Laurent series)
    \item $\frac{az + b}{cz + d}, \quad c \ne 0$
    \item $\frac{1}{z \ln(1 + z)}, \quad z_0 = 0$ (four terms of the Laurent series)
    \item $\frac{1}{1 - \cos z}, \quad z_0 = 0$ (four terms of the Laurent series)
\end{enumerate}

\noindent
\begin{enumerate}
    \item If $f$ is analytic in $|z - z_0| < R$ and has a zero of order $m$ at $z_0$, show that
\[
\text{Res}\left( \frac{f'(z)}{f(z)}; z_0 \right) = m.
\]

\item If $f$ is analytic in $0 < |z - z_0| < R$ and has a pole of order 1 at $z_0$, show that
\[
\text{Res}\left( \frac{f'(z)}{f(z)}; z_0 \right) = -1.
\]
(Hint: Write $f(z) = \frac{g(z)}{(z - z_0)^l}$, where $g$ is analytic in $|z - z_0| < R$ and $g(z_0) \ne 0$.)

\item Suppose $f$ is analytic in $0 < |z - z_0| < R$ and has a zero of order $l \ge m$ at $z_0$. If $g$ is analytic in $0 < |z - z_0| < R$ and has a pole of order $l, l \le m$, at $z_0$, show that $fg$ has a removable singularity at $z_0$.

\item Let $f$ be analytic in $0 < |z - z_0| < r$ and suppose that $f$ has an essential singularity at $z_0$. Let $w$ be any complex number. Show that
\[
g(z) = \frac{1}{f(z) - w}, \quad z \ne z_0
\]
is not bounded in any punctured disc $0 < |z - z_0| < \epsilon$. Conclude that the range of $f$ gets arbitrarily close to all points in the complex plane.

\item Here is an alternate proof that
\[
\frac{1}{2\pi i} \int_{|z - z_0| = r} f(z) \, dz = \text{Res}(f; z_0)
\]
is independent of $r$. Assume $z_0 = 0$ with no loss of generality.
\begin{enumerate}
    \item Show that
    \[
    \frac{\partial}{\partial t} (f(se^{it})e^{it}) = ie^{it} f'(se^{it}) + is e^{2it} f'(se^{it}).
    \]
    \item Show that
    \[
    \frac{d}{ds} \left( \frac{1}{2\pi i} \int_0^{2\pi} f(se^{it}) e^{it} \, dt \right) = \frac{1}{2\pi i} \int_0^{2\pi} \frac{\partial}{\partial s} (f(se^{it}) e^{it}) \, dt.
    \]
    \item Conclude from (a) and (b) that
    \[
    \frac{d}{ds} \left( \frac{1}{2\pi i} \int_{|z| = s} f(z) \, dz \right) = \frac{1}{2\pi i} \int_0^{2\pi} \frac{\partial}{\partial s} (f(se^{it}) e^{it}) \, dt.
    \]
\end{enumerate}

\item Suppose that the Laurent series $\sum_{n=-\infty}^{\infty} a_n (z - z_0)^n$ converges for $0 < |z - z_0| < R$ and
\[
\sum_{n=-\infty}^{\infty} a_n (z - z_0)^n = 0, \quad 0 < |z - z_0| < r.
\]
Show that $a_n = 0$ for all $n = 0, \pm1, \pm2, \ldots$

\item If $f$ is analytic in $0 < r < |z - z_0| < R$, show that its Laurent series is uniquely determined.

\item If $f$ is analytic on $0 < |z - z_0| < R$ and there is an analytic function $G$ on $0 < |z - z_0| < R$ with $G' = f$, then
\[
\text{Res}(f; z_0) = 0.
\]

\item Find the Laurent series about $z_0 = 0$ for the following functions, valid in the indicated regions:
\begin{enumerate}
    \item $e^{1/z}$ in $0 < |z| < \infty$
    \item $z^4 \sin\left( \frac{1}{z} \right)$ in $0 < |z| < \infty$
    \item $\frac{1}{z - 1} - \frac{1}{z + 1}$ in $0 < |z| < \infty$
    \item $\exp\left( z + \frac{1}{z} \right)$ in $0 < |z| < \infty$
    \item $z \cos\left( \frac{1}{z} \right)$ in $0 < |z| < \infty$
\end{enumerate}

\item Use equations (12) and (13) to find the Laurent expansions of the following rational functions in powers of $z$ and $1/z$ in the indicated region(s):
\begin{enumerate}
    \item $f(z) = \frac{z + 2}{z^2 + 1}$ in $1 < |z| < 2$ and then in $2 < |z| < \infty$
    \item $f(z) = \frac{z^2 - 4}{(z - 2)(z^2 + 1)}$ in $|z| < 1$ and then in $1 < |z| < 3$
    \item $f(z) = \frac{z}{(z - 2)(z - 3)(z - 5)}$ in $2 < |z| < 3$ and then in $5 < |z| < \infty$
\end{enumerate}

\item Let $f$ be analytic in $0 < |z - z_0| < r$, and let
\[
f(z) = \sum_{n=-\infty}^{\infty} a_n (z - z_0)^n
\]
be its Laurent series. Show that
\begin{enumerate}
    \item $z_0$ is a removable singularity for $f$ if and only if $a_n = 0$ for all $n = -1, -2, \ldots$
    \item $z_0$ is a pole of order $m \ge 1$ if and only if $a_{-m} \ne 0$ but $a_{-n} = 0$ for all $n > m + 1$
    \item $z_0$ is an essential singularity for $f$ if and only if there are infinitely many $a_{-n}$, $n > 0$, that are not zero
\end{enumerate}

\item Show that $f$ has a removable singularity at $\infty$ if and only if $f(z)$ is bounded for $|z| > R_0$.

\item Classify the nature of the singularity at $\infty$ for each of the following functions. If the singularity is removable, give the value at $\infty$. If the singularity is a pole, give its order; in each case find the first few terms in the Laurent series about $\infty$.
\begin{enumerate}
    \item $\frac{3z^2 + 4}{z - 1}$
    \item $\frac{1}{(1 - z)(z - 4)}$
    \item $\frac{z^2}{e^z - 4}$
    \item $\frac{(1 - z^2 - z^3)^2}{e^z}$
    \item $\left( \frac{z + 1}{z} \right)^2$
    \item $\frac{\sin 1/z}{z}$
    \item $\sum_{n=0}^\infty \frac{(-1)^n z^{2n}}{(2n)!}$
\end{enumerate}

\subsection*{Bessel Functions Continued}

\item Let $\nu$ be a complex number and let
\[
G(z; \nu) = \exp\left( \left( \frac{z}{2} \right) \left( t - \frac{1}{t} \right) \right), \quad z \ne 0.
\]
Show that $G(z; \nu)$ is an analytic function of $z$ for $z \ne 0$ and has an essential singularity at $z = 0$ (unless $\nu = 0$).

\item Let
\[
G(z; \nu) = \sum_{n=0}^\infty J_\nu(z) t^n
\]
be the Laurent series of $G(z; \nu)$ about the origin, where the coefficients $J_\nu(z)$ are given by:
\[
J_\nu(z) = \frac{1}{2\pi} \int_0^{2\pi} \cos(\nu \theta - z \sin \theta) \, d\theta.
\]
Prove this by using the real integral representation and then series expansion.

\item Multiply out the series for $\exp(z/2)$ and the series for $\exp(-z/2)$ and then collect equal powers of $z$. Conclude that
\[
J_\nu(z) = \sum_{k=0}^\infty \frac{(-1)^k}{k! \Gamma(k + \nu + 1)} \left( \frac{z}{2} \right)^{2k + \nu}, \quad \text{for } n > 0.
\]
Compare this to the conclusion of Exercise 29, Section 4.

\item Show that $J_{-\nu}(z) = (-1)^\nu J_\nu(z)$. (Hint: Replace $\nu$ with $-\nu$ in the Laurent series; then change the summation index $n$ to $-n$ and compare series.)
\end{enumerate}
\begin{enumerate}
    \item Evaluate the line integral
    \[
    \int_2^i \bar{z} \, dz
    \]
    along the straight line from $2$ to $i$.

    \item Estimate the line integral
    \[
    \int_2^{2i} \frac{dz}{\bar{z}^2 - 1}
    \]
    along the circular arc with center at $0$ running from $2$ to $2i$ counterclockwise.

    \item Evaluate the line integral
    \[
    \int_{-i}^{3i} \frac{dz}{\bar{z}^2}
    \]
    along the semicircle with center at $i$ in $\text{Re}\, z > 0$.

    \item Evaluate the line integral
    \[
    \int_{1 - 2i}^{-3 + i} \frac{dz}{z}
    \]
    along the straight line. (Use the branch on $\log z$ with $0 < \arg z < 2\pi$.)
\end{enumerate}
\begin{enumerate}
    \item Let $f(z) = z^{-1} \tan^2 z$. Characterize all singularities of $f$ as removable, pole, or essential. If removable, then find the value; if pole, then find the order. Find the radius of convergence of the Taylor series at $z = i$.
    
    \setcounter{enumi}{2}
    
    \item Find the Laurent series of $f(z) = (z + 1)^{-2} + (z - 2)^{-3}$ into powers of $z$ in the annulus $1 < |z| < 2$. Express the coefficient of $z^n$ in terms of $n$.

    \item Let $f$ be an analytic function in $\mathbb{C} \setminus \{0\}$. Suppose $|f(z)| \le |z|^{10} + |z|^{-10}$ for all $z \ne 0$. Prove that $f$ is rational. (Hint: Use Cauchy estimates for Laurent series.)
\end{enumerate}
\begin{enumerate}
    \item Find 4 terms of the Laurent series of the function
    \[
    f(z) = \frac{1}{1 - e^z}
    \]
    in a punctured neighborhood of $0$. (Use division algorithm.)

    \item Find 3 \textit{nonzero} terms of the Laurent series of the function
    \[
    f(z) = \frac{z}{\sin^2 z}
    \]
    in a punctured neighborhood of $0$. (Find several terms of the Taylor series of $\sin^2 z$ by multiplying the series of $\sin z$ by itself and then use division algorithm.)

    \item Evaluate
    \[
    \int_{|z| = 1} \frac{dz}{e^z - e^{-z}} \quad \text{and} \quad \int_{|z| = 4} \frac{dz}{e^z - e^{-z}}.
    \]
    (Use the formula of residue in a pole. Note the second integral has more than one singularity inside the contour!)

    \item Evaluate
    \[
    \int_{|z| = 1} \frac{dz}{z^2 (e^z - 1)}.
    \]
    (Use the Laurent series.)
\end{enumerate}

\begin{enumerate}
    \item (10 points) Evaluate
    \[
    \int_{0}^{2} z \, dz
    \]
    along the upper semi-circle with center at 1.

    \item (10 points) Evaluate
    \[
    \int_{-2}^{-i} \frac{dz}{z}
    \]
    along the straight line.
\end{enumerate}

\begin{enumerate}
    \item (10 points) Evaluate
    \[
    I = \int_{|z+2|=2} \frac{dz}{(z^3 + 1)^2}
    \]
    using the Cauchy integral formula for derivatives.

    \item (10 points) Find all zeros and their orders of the function
    \[
    f(z) = (z^2 + z - 2)^2 \,\bigl(e^{\pi i z} - 1\bigr).
    \]
\end{enumerate}
\begin{enumerate}
    \item (10 points) Find all singularities of the function
    \[
    f(z) = z^{-2} \,(z+1)^{-1} \,(z-2)^{-3}\,\bigl(e^{4\pi i z} - 1\bigr)
    \]
    and determine their types (removable, essential, or a pole of a certain order). 
    \emph{Caution: While one factor can have a singular point, another factor may vanish at that point.}

    \item (10 points) Find the Laurent series
    \[
    \sum_{n=-\infty}^{\infty} a_n (z+1)^n
    \]
    of the function 
    \[
    f(z) = \frac{1}{z-1}
    \]
    in the domain 
    \[
    \{\,z : |z+1| > 2\}.
    \]
    Express $a_n$ in terms of $n$.
\end{enumerate}
\end{document}
