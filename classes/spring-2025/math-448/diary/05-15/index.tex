\documentclass[12pt]{article}

% Packages
\usepackage[margin=1in]{geometry}
\usepackage{amsmath,amssymb,amsthm}
\usepackage{enumitem}
\usepackage{hyperref}
\usepackage{xcolor}
\usepackage{import}
\usepackage{xifthen}
\usepackage{pdfpages}
\usepackage{transparent}
\usepackage{listings}
\usepackage{tikz}
\usepackage{physics}
\usepackage{siunitx}
\usepackage{booktabs}
\usepackage{cancel}
  \usetikzlibrary{calc,patterns,arrows.meta,decorations.markings}


\DeclareMathOperator{\Log}{Log}
\DeclareMathOperator{\Arg}{Arg}

\lstset{
    breaklines=true,         % Enable line wrapping
    breakatwhitespace=false, % Wrap lines even if there's no whitespace
    basicstyle=\ttfamily,    % Use monospaced font
    frame=single,            % Add a frame around the code
    columns=fullflexible,    % Better handling of variable-width fonts
}

\newcommand{\incfig}[1]{%
    \def\svgwidth{\columnwidth}
    \import{./Figures/}{#1.pdf_tex}
}
\theoremstyle{definition} % This style uses normal (non-italicized) text
\newtheorem{solution}{Solution}
\newtheorem{proposition}{Proposition}
\newtheorem{problem}{Problem}
\newtheorem{lemma}{Lemma}
\newtheorem{theorem}{Theorem}
\newtheorem{remark}{Remark}
\newtheorem{note}{Note}
\newtheorem{definition}{Definition}
\newtheorem{example}{Example}
\newtheorem{corollary}{Corollary}
\theoremstyle{plain} % Restore the default style for other theorem environments
%

% Theorem-like environments
% Title information
\title{}
\author{Jerich Lee}
\date{\today}

\begin{document}

\maketitle
%------------------------------------------------------------
% Mapping the disk |z-1| \le 1 under the inversion f(z)=1/z
%------------------------------------------------------------

\subsection*{Why \(f(z)=\tfrac1z\) sends the unit disk centered at \(1\) to
\(\operatorname{Re} w > \tfrac12\)}

\begin{enumerate}
\item \textbf{Describe the boundary circle algebraically.}  
      For \(z=x+iy\),
      \[
        \lvert z-1\rvert^2 = (x-1)^2 + y^2 = 1
        \quad\Longrightarrow\quad
        x^2 + y^2 = 2x. \tag{1}
      \]

\item \textbf{Write the image point \(w=f(z)=1/z\).}  
      \[
        w \;=\; \frac1z \;=\; \frac{\bar z}{|z|^2} 
        \;=\; \frac{x-iy}{x^2+y^2}
        \;=\; u + iv,
        \quad
        \begin{cases}
          u = \dfrac{x}{x^2+y^2},\\[6pt]
          v = -\dfrac{y}{x^2+y^2}.
        \end{cases}
      \]

\item \textbf{Show the boundary goes to the line \(\operatorname{Re} w=\tfrac12\).}  
      Substituting \(x^2+y^2 = 2x\) from (1) into the formula for \(u\),
      \[
        u \;=\; \frac{x}{2x} \;=\; \frac12
        \qquad (x\neq 0).
      \]
      Hence every boundary point satisfies \(\operatorname{Re} w = \tfrac12\),
      so the entire circle maps onto the vertical line
      \[
        \boxed{\operatorname{Re} w = \tfrac12 }.
      \]

\item \textbf{Determine which side corresponds to the interior.}  
      Points \emph{inside} the circle satisfy \((x-1)^2 + y^2 < 1\),
      i.e.\ \(x^2 + y^2 < 2x\).  Using the same substitution,
      \[
        \operatorname{Re} w
        \;=\;
        \frac{x}{x^2+y^2}
        \;>\;
        \frac12,
      \]
      so the disk’s interior is sent to
      \[
        \boxed{\operatorname{Re} w > \tfrac12 } \quad
        (\text{the half-plane to the right of the line}).
      \]

\item \textbf{Geometric intuition (optional narrative).}
      \begin{itemize}
        \item \(f(z)=1/z\) equals inversion in the unit circle followed by a reflection in the real axis.
        \item A circle passing through the origin (our circle does: \(z=0\) lies on it) always maps to a straight line.
        \item Inversion swaps “near” and “far’’ with respect to the origin, so the disk’s interior becomes the exterior half-plane on the image side.
      \end{itemize}
\end{enumerate}
%------------------------------------------------------------
%  Why the region inside the unit circle (minus two radial cuts)
%  is carried by the map \,w=\dfrac{1}{z+1}\,
%  to the right half-plane to the right of the line  \mathbb{R}e w=\tfrac12
%  minus two real-axis slits
%------------------------------------------------------------
\pagebreak
\begin{align}
  \textbf{1.  Describe the original region }D
  &=\Bigl\{\,z\in\Bbb C : |z|<1\Bigr\}\;
     \setminus\left(\,(-1,0]\;\cup\;[\,\tfrac13,1\,]\right).\\[6pt]
  &\text{It is the open unit disc, but we remove}          \\[-2pt]
  &\quad\bullet\text{the \emph{negative} real radius }(-1,0],          \\[-1pt]
  &\quad\bullet\text{the part of the \emph{positive} radius }[\,\tfrac13,1\,].
  \end{align}
  
  \bigskip
  \begin{align}
  \textbf{2.  Translate by }+1:\quad u=z+1.
  \end{align}
  
  \begin{enumerate}
  \item The circle $|z|=1$ becomes the circle
        \[
           C\;:\;|u-1|=1,
        \]
        centred at $1$ and passing through $u=0$ and $u=2$.
  \item The two removed radii become real intervals inside~$C$:
        \[
          (-1,0]\;\mapsto\;[0,1], 
          \qquad
          [\,\tfrac13,1\,]\;\mapsto\;[\,\tfrac43,2\,].
        \]
  \end{enumerate}
  
  \bigskip
  \textbf{3.  Invert:}\; $w=\displaystyle\frac1u$.
  
  \begin{enumerate}
  \item Because $C$ passes through the origin, the Möbius map $w=1/u$
        sends the circle $C$ to a \emph{straight line}.  
        A direct computation shows
        \[
           w=\frac{1+\cos\theta-i\sin\theta}{2(1+\cos\theta)}
           \;\;\Longrightarrow\;\;
           \mathbb{R}e w=\frac12
        \]
        for every $u=1+e^{i\theta}\in C$.  
        Hence
        \[
          C\longmapsto L:\;\mathbb{R}e w=\tfrac12,
          \qquad
          \text{and}\quad
          |u-1|<1\;\Longrightarrow\;\mathbb{R}e w>\tfrac12.
        \]
        Thus the \emph{interior} of the original circle
        becomes the half-plane
        \[
          H=\{\,w\in\Bbb C:\mathbb{R}e w>\tfrac12\}.
        \]
  \item The slit $[0,1]$ on the $u$–axis maps to
        \[
           [0,1]\;\xrightarrow{\,w=1/u\,}\;[1,\infty)\subset\Bbb R,
        \]
        which we therefore \emph{remove} from~$H$.
  
  \item The slit $[\,\tfrac43,2\,]$ maps to
        \[
           [\,\tfrac43,2\,]\;\xrightarrow{\,1/u\,}\;
           \Bigl[\tfrac12,\tfrac34\Bigr]\subset\Bbb R,
        \]
        and this interval is likewise removed.
  \end{enumerate}
  
  \bigskip
  \textbf{4.  Final image.}\;
  Putting everything together,
  \[
     D\;\xrightarrow{\,w=1/(z+1)\,}\;
     \boxed{\;H\setminus\Bigl((\tfrac12,\tfrac34]\;\cup\;[1,\infty)\Bigr)
     =\Bigl\{\,w\in\Bbb C:\mathbb{R}e w>\tfrac12\Bigr\}
       \setminus\Bigl((\tfrac12,\tfrac34]\cup[1,\infty)\Bigr)\;}.
  \]
  
  \medskip
  \textbf{5.  Intuitive summary.}
  \begin{itemize}
    \item Translating by $+1$ centres the unit circle at~$1$,
          so the inversion $w=1/u$ turns that circle into the vertical
          line $\mathbb{R}e w=\tfrac12$ and carries its interior to the
          right half-plane $\mathbb{R}e w>\tfrac12$.
    \item Each deleted radial segment becomes a real-axis segment because
          \,$1/u$ sends real numbers to real numbers and reverses order.
    \item Hence the image is the half-plane to the \emph{right} of
          $\mathbb{R}e w=\tfrac12$, but we must delete the two real intervals
          $(\tfrac12,\tfrac34]$ and $[1,\infty)$ that correspond to the
          original cuts.
  \end{itemize}
  %------------------------------------------------------------
%  Image, under the squaring map \(w=z^{2}\),
%  of the shaded half–plane
%      \(\{\,z\in\Bbb C:\mathbb{R}e z>\tfrac12\}\)
%  with the two real-axis slits \((\tfrac12,\tfrac34]\) and \([1,\infty)\).
%------------------------------------------------------------
\pagebreak
\begin{align}
  \textbf{1.  The starting region }R
     &=\Bigl\{\,z\in\Bbb C:\mathbb{R}e z>\tfrac12\Bigr\}
       \setminus\Bigl((\tfrac12,\tfrac34]\cup[1,\infty)\Bigr).      \\[4pt]
  \textbf{2.  Apply }w=z^{2}\,:\quad z=x+iy\;(x>\tfrac12)
     &\;\Longrightarrow\;
       w=u+iv,\;
       \begin{cases}
         u&=x^{2}-y^{2},\\
         v&=2xy.
       \end{cases}
  \end{align}
  
  \bigskip
  \textbf{3.  Image of the boundary line \(\mathbb{R}e z=\tfrac12\).}
  
  \[
     z=\tfrac12+iy\;\;(y\in\Bbb R)
     \quad\Longrightarrow\quad
     w=\bigl(\tfrac12+iy\bigr)^{2}
       =\tfrac14-y^{2}+iy,
  \]
  so the line maps to the parabola
  \[
     \boxed{\,u=\tfrac14-v^{2}\,}\qquad(\text{opening to the \emph{left}}).
  \]
  Because \(z\) lies \emph{right} of the line, its image \(w\) lies
  \emph{right} of the parabola:
  \[
     \mathbb{R}e z>\tfrac12
     \;\Longrightarrow\;
     \boxed{\,u>\tfrac14-v^{2}\,}.
  \]
  
  \bigskip
  \textbf{4.  Images of the excised real intervals.}
  
  \[
  \begin{aligned}
     (\tfrac12,\tfrac34] &\xrightarrow{\,z^{2}\,}
         (\tfrac12)^{2}\!<u\le(\tfrac34)^{2}
         \;=\;(\tfrac14,\,\tfrac{9}{16}], \\[4pt]
     [1,\infty) &\xrightarrow{\,z^{2}\,} [1,\infty).
  \end{aligned}
  \]
  Both land on the \emph{positive} real axis and must be deleted
  from the image.
  
  \bigskip
  \textbf{5.  Resulting region in the \(w\)-plane.}
  
  \[
     R\xrightarrow{\,z^{2}\,}
     \boxed{\;\Omega
       =\Bigl\{\,w=u+iv\in\Bbb C:
                u>\tfrac14-v^{2}\Bigr\}
          \setminus\Bigl((\tfrac14,\tfrac{9}{16}]\cup[1,\infty)\Bigr)\; }.
  \]
  
  \medskip
  \textbf{6.  Geometric intuition.}
  \begin{itemize}
    \item The map \(w=z^{2}\) doubles arguments and squares moduli, so the
          vertical line \(\mathbb{R}e z=\tfrac12\) (argument range \(-\tfrac\pi2<\arg z<\tfrac\pi2\))
          becomes the left-opening parabola \(u=\tfrac14-v^{2}\).
    \item Points with \(\mathbb{R}e z>\tfrac12\) keep arguments strictly between
          \(-\tfrac\pi2\) and \(\tfrac\pi2\); after doubling, their arguments lie
          in \((-\,\pi,\pi)\), i.e.\ the image avoids the negative real axis and
          lies entirely to the \emph{right} of the parabola.
    \item Real numbers are sent to real numbers, so each excised ray on the
          \(z\)-axis reappears as a ray on the \(w\)-axis:
          \((\tfrac14,\tfrac{9}{16}]\) and \([1,\infty)\) are accordingly removed.
  \end{itemize}

  % Mapping the half–plane  $\mathbb{R}e z>\frac12$ under the squaring map $w=z^{2}$

\[
  \begin{aligned}
  D &= \bigl\{\,z\in\mathbb{C}\;\bigm|\; \mathbb{R}e z>\tfrac12 \bigr\}, 
  \qquad w=z^{2}. \\[6pt]
  \text{Boundary of }D: \;
  z(t) &= \tfrac12 + i\,t, \qquad t\in\mathbb{R}. \\[6pt]
  \Longrightarrow\;
  w(t) &= z(t)^{2}
        =\bigl(\tfrac12+i\,t\bigr)^{2}
        =\frac14 - t^{2} + i\,t .
  \end{aligned}
  \]
  
  Setting \(w=u+iv\) and identifying \(t=v\) we obtain the Cartesian equation of the image of the boundary line:
  
  \[
  u \;=\; \frac14 - v^{2}.
  \]
  
  Hence the vertical line \(\mathbb{R}e z=\tfrac12\) is sent to the **parabola**
  
  \[
  \boxed{P:\; \mathbb{R}e w \;=\; \frac14-\bigl(\Im w\bigr)^{2}},
  \]
  
  which opens to the left and has its vertex at \(\bigl(\tfrac14,0\bigr)\).
  
  \vspace{1em}
  
  To see where the interior of the half–plane goes, take \(z=x+iy\) with \(x>\tfrac12\):
  
  \[
  w = x^{2}-y^{2} \;+\; i\,2xy .
  \]
  
  For \(y=0\) (so \(v=0\)) we have \(u=x^{2}>\tfrac14\), i.e.\ points just to the **right** of the boundary map to points with real part **larger** than the parabola’s vertex.  
  A little algebra shows that, in fact,
  
  \[
  w(D)=\Bigl\{\,w\in\mathbb{C}\,\Bigm|\,
            \mathbb{R}e w>\frac14-\bigl(\Im w\bigr)^{2}\Bigr\},
  \]
  
  the region **to the right of (outside)** the parabola \(P\).  
  The map \(w=z^{2}\) is conformal on \(D\) and is two-to-one onto this parabolic exterior domain (except along the positive real axis, where it is one-to-one).
  %--- Mapping the slit half–plane under \(z\mapsto z^{2}\) ---
  \pagebreak
\[
  \textbf{Domain set:}\qquad
  S=\Bigl\{\,z\in\mathbb{C} : \mathbb{R}e z>0\Bigr\}\setminus\Bigl((-\infty,\tfrac12]\;\cup\;[1,\infty)\Bigr).
  \]
  
  \[
  \textbf{Map:}\qquad w=z^{2}=r^{2}e^{\,i2\theta},
  \quad\text{where } z=re^{i\theta},\;
  -\frac{\pi}{2}<\theta<\frac{\pi}{2}.
  \]
  
  \begin{enumerate}
      \item \emph{Right half–plane \(\to\) slit plane.}\\
            Doubling the argument sends the range \((-\pi/2,\pi/2)\) to \((-\pi,\pi)\);
            hence \(z^{2}\) sends the right half–plane onto
            \(\mathbb{C}\setminus(-\infty,0]\) (the whole plane with the non-positive
            real axis removed).
  
      \item \emph{Image of the deleted segment \((-\infty,\frac12]\).}\\
            For a real number \(x\le\frac12\) we have \(w=x^{2}\le\frac14\);
            thus that slit maps to the real interval
            \((-\infty,\tfrac14]\).
  
      \item \emph{Image of the deleted segment \([1,\infty)\).}\\
            For a real number \(x\ge1\) we have \(w=x^{2}\ge1\);
            hence that slit maps to the real interval
            \([1,\infty)\).
  \end{enumerate}
  
  \[
  \boxed{\;
        z^{2}\bigl(S\bigr)
        =\mathbb{C}\setminus\bigl((-\infty,\tfrac14]\;\cup\;[1,\infty)\bigr)
        \;}
  \]
  
  That is, the resulting set is the entire plane with the real points
  \(<\tfrac14\) and \(>1\) removed—exactly the stated “entire shaded
  plane excluding real points less than \(z=\tfrac14\) and greater than
  \(z=1\).”
  \pagebreak
  %--- Image of a slit upper half–disk under the squaring map ---
\[
  \textbf{Domain}
  \;D
  =\Bigl\{\,z=re^{i\theta}\in\mathbb{C} :
         0<r<1,\;0<\theta<\pi\Bigr\}
    \setminus
    \Bigl\{\,iy:0<y<\tfrac12\Bigr\}
  \]
  (the open upper half–disk of radius 1, with the segment
  \(0<iy<i/2\) removed).
  
  \[
  \textbf{Map}\quad
  w=z^{2}=r^{2}e^{\,i2\theta},
  \qquad 0<r<1,\ 0<\theta<\pi .
  \]
  
  \begin{enumerate}
    \item \emph{Angle-doubling.}  
          Since \(0<\theta<\pi\), the doubled angle \(2\theta\) ranges over
          the whole interval \((0,2\pi)\);
          hence the image of the half-disk is the \emph{entire} open unit
          disk \(\{\,|w|<1\}\).
  
    \item \emph{The missing imaginary-axis segment.}  
          Points on the deleted ray \(iy\;(0<y<\tfrac12)\) have
          \(\theta=\tfrac{\pi}{2}\) and \(r=y\).
          Squaring gives  
          \[
              w
              =\bigl(iy\bigr)^{2}
              =-\,y^{2},
              \qquad 0<y<\tfrac12
              \;\Longrightarrow\;
              -\tfrac14<w<0 .
          \]
          Thus that tiny slit maps exactly to the \emph{negative} real
          interval \(\bigl(-\tfrac14,0\bigr)\).
  
    \item \emph{Boundary check.}  
          The semicircular boundary \(|z|=1\) goes to \(|w|=1\) (the unit
          circle), and the real-axis diameter \([-1,1]\) maps to
          \([0,1]\subset\partial\Bbb D\).
  \end{enumerate}
  
  \[
  \boxed{\;
        z^{2}\bigl(D\bigr)
        =\{\,w\in\mathbb{C}:|w|<1\}\setminus\bigl(-\tfrac14,0\bigr)
        \;}
  \]
  
  \medskip
  In words: squaring sends the upper half-disk to the \emph{full} unit
  disk, but the tiny vertical slit \(0<iy<i/2\) is carried onto the
  horizontal slit \(-\tfrac14<w<0\) on the negative real axis—so every
  point inside the unit circle is obtained \emph{except} those real points
  strictly between \(-\tfrac14\) and \(0\).
  \pagebreak
  %----------------------------------------------------------
%  Why the Möbius map \(w=\dfrac{1}{z}\) sends the “upper cap’’
%  joining \(0\) to \(1\) to a straight line through \(w=1\)
%  that meets the real axis with slope \(-\alpha\)
%----------------------------------------------------------

\[
  \textbf{Set–up:}\qquad
  \gamma\;=\;\bigl\{\,z\in\mathbb{C}:\;|z-\tfrac12|=\tfrac12,\;
                   0<\arg(z)<\pi\bigr\},
  \]
  the tiny circular arc (the “upper cap’’) joining \(z=0\) to \(z=1\).
  At the endpoints its tangents meet the real axis at symmetric
  angles \(\pm\alpha\;(0<\alpha<\tfrac{\pi}{2})\).
  Write a local parametrisation near the right–hand endpoint
  \(z=1\):
  \[
  z(\rho)=1+\rho\,e^{\,i\alpha},\qquad 0<\rho\ll1.
  \]
  
  \[
  \textbf{Map:}\qquad
  w=\frac{1}{z},\qquad w'(z)=-\frac{1}{z^{2}}\; .
  \]
  
  \begin{enumerate}
    \item[\(1.\)] \emph{Circle through the origin \(\longrightarrow\) straight line.}\\
          A Möbius map sends every circle or line to a circle or line, and
          those passing through \(0\) go to \emph{lines}
          (because \(0\mapsto\infty\)).
          Since \(\gamma\) passes through \(0\) and \(1\), its image must
          be the unique straight line through \(w=1\).
  
    \item[\(2.\)] \emph{Direction of that line.}\\
          Insert the parametrisation into the map:
          \[
          w(\rho)\;=\;\frac{1}{1+\rho e^{\,i\alpha}}
                     \;=\;
                     1-\rho e^{\,i\alpha}+O(\rho^{2})
                     \;=\;
                     1+\rho\,e^{\,i(\alpha+\pi)}+O(\rho^{2}).
          \]
          Hence the first–order displacement
          \(
            w(\rho)-1
            \sim
            \rho\,e^{\,i(\alpha+\pi)}
          \)
          has argument \(\alpha+\pi\).
          Because a half–turn does not change the \emph{geometric} line,
          the direction modulo \(\pi\) is
          \[
            \alpha+\pi\;\equiv\;-\alpha\pmod{\pi}.
          \]
          Thus the image line meets the real axis at the angle
          \(-\alpha\).
  
  \end{enumerate}
  
  \[
  \boxed{
     \frac{1}{z}\,:\;
     \gamma
     \;\longmapsto\;
     \bigl\{\,w=1+t\,e^{-i\alpha}\;:\;t\in\mathbb{R}\bigr\}
  }
  \]
  
  \noindent
  In words: the inversion \(w=1/z\) flattens the
  little circular “cap’’ above the interval \([0,1]\) into a straight
  line through \(w=1\); because the derivative at \(z=1\) is
  \(w'(1)=-1\) (a half–turn), the tilt of that line is the
  \emph{negative} of the original endpoint tangent angle~\(\alpha\).
  \pagebreak
  \begin{enumerate}[label=\textbf{(\alph*)}]
    %----------------------------------------------------
    \item \textbf{All values of $\displaystyle (\,1-i\sqrt3\,)^{\tfrac52}$}
    
    \begin{align*}
    z &= 1-i\sqrt3
          = 2\;e^{\,i(-\pi/3+2\pi k)},\qquad k\in\mathbb Z,\\[2mm]
    |z| &= 2, \qquad\Arg z = -\frac{\pi}{3}\;(\text{principal}).
    \end{align*}
    
    Using the multi–valued power
    \[
    z^{5/2}= \exp\!\Bigl(\tfrac52\Log z\Bigr),
    \quad
    \Log z = \ln 2 + i\Bigl(-\tfrac{\pi}{3}+2\pi k\Bigr),
    \]
    we obtain
    \[
    z^{5/2}
      \;=\;
      2^{5/2}\;
      e^{\,i\bigl(\tfrac52\bigl(-\pi/3+2\pi k\bigr)\bigr)}
      \;=\;
      4\sqrt2\;
      e^{\,i\bigl(-\tfrac{5\pi}{6}+5\pi k\bigr)} .
    \]
    
    Because $5\pi\equiv\pi\pmod{2\pi}$, adding $k\mapsto k+1$ changes the
    argument by $\pi$.  Hence only \emph{two} distinct arguments fall in
    $(-\pi,\pi]$:
    \[
    \theta_0=-\frac{5\pi}{6},\qquad
    \theta_1=\theta_0+\pi=\frac{\pi}{6}.
    \]
    
    \[
    \boxed{\;
    (1-i\sqrt3)^{5/2}
        \;=\;
        4\sqrt2\,e^{-5\pi i/6}
        \quad\text{or}\quad
        4\sqrt2\,e^{\pi i/6}
    \;}
    \]
    
    (Explicitly:
    $4\sqrt2\,e^{\pi i/6}=2\sqrt6+2\sqrt2\,i$ and
    $4\sqrt2\,e^{-5\pi i/6}=-2\sqrt6-2\sqrt2\,i$.)
    
    %----------------------------------------------------
    \item \textbf{Every branch value of $\displaystyle \Log\!\bigl(-\sqrt2+i\sqrt2\bigr)$}
    
    \[
    w=-\sqrt2+i\sqrt2
        \;=\;
        2\,e^{\,i3\pi/4},
    \qquad
    |w|=2,\;
    \Arg w = \frac{3\pi}{4}.
    \]
    
    The multi–valued logarithm is
    \[
    \Log w
       = \ln|w| + i\bigl(\Arg w + 2\pi k\bigr)
       = \ln 2 + i\Bigl(\frac{3\pi}{4}+2\pi k\Bigr),
       \qquad k\in\mathbb Z.
    \]
    
    \[
    \boxed{\;
       \Log\!\bigl(-\sqrt2+i\sqrt2\bigr)
       =\ln 2 + i\Theta_k,
       \quad
       \Theta_k= \frac{3\pi}{4}+2\pi k,
       \;k\in\mathbb Z
    \;}
    \]
    
    (The principal value corresponds to $k=0$: $\displaystyle \ln 2 +
    i\frac{3\pi}{4}$.)
    \end{enumerate}
    \pagebreak
    %------------------------------------------------------------
%  Principal value of \( (1+i\sqrt3)^{40} \)
%------------------------------------------------------------
\[
  \begin{aligned}
  z &= 1+i\sqrt3
       \;=\;
       2\,e^{\,i\pi/3},          &&\text{(polar form, principal argument)}\\[6pt]
  z^{40} 
     &= \bigl(2\,e^{\,i\pi/3}\bigr)^{40}
       \;=\;
       2^{40}\,e^{\,i\frac{40\pi}{3}}
       \;=\;
       2^{40}\,e^{\,i\left(13\pi+\frac{\pi}{3}\right)}
       \;=\;
       2^{40}\,e^{\,i\frac{4\pi}{3}}\\[6pt]
     &= 2^{40}\,
        e^{-\,i\frac{2\pi}{3}}
        &&\bigl(\tfrac{4\pi}{3}-2\pi=-\tfrac{2\pi}{3}\text{ puts the angle in }(-\pi,\pi]\bigr)\\[6pt]
     &= 2^{40}
        \Bigl(\cos\!\bigl(-\tfrac{2\pi}{3}\bigr)+
              i\sin\!\bigl(-\tfrac{2\pi}{3}\bigr)\Bigr)\\[6pt]
     &= 2^{40}\Bigl(-\tfrac12-i\tfrac{\sqrt3}{2}\Bigr)
        \;=\;
        -2^{39}\Bigl(1+\sqrt3\,i\Bigr).
  \end{aligned}
  \]
  
  \[
  \boxed{\,\displaystyle
          (1+i\sqrt3)^{40}
          = -2^{39} - 2^{39}\sqrt3\,i
         \;}
  \]
  
  (If you prefer decimal form: 
  \(2^{39}=549{,}755{,}813{,}888\), so
  \((1+i\sqrt3)^{40}\approx -5.50\times10^{11}
   -9.53\times10^{11}\,i\).)
   %------------------------------------------------------------
%  All values of \bigl(\sqrt{2}-i\bigr)^{20}
%------------------------------------------------------------
\pagebreak
Let  
\[
z=\sqrt{2}-i
      =r\;e^{i\theta},\qquad
      r=\sqrt{|z|^{2}}=\sqrt{2+1}= \sqrt{3},\;
      \cos\theta=\frac{\sqrt2}{\sqrt3},\;
      \sin\theta=-\frac{1}{\sqrt3}.
\]

---

0 1.  Polar–form computation  
Because the exponent is the **integer** \(20\), the power is *single-valued*:

\[
z^{20}
   = r^{20}\,e^{\,i20\theta}
   = (\sqrt3)^{20}\!\bigl(\cos(20\theta)+i\sin(20\theta)\bigr).
\]

Since \((\sqrt3)^{20}=3^{10}=59\,049\), the only task is to evaluate
\(\cos(20\theta)\) and \(\sin(20\theta)\).  
Using De Moivre (or a CAS) one finds

\[
\cos(20\theta)=\frac{57\,113}{59\,049},
\qquad
\sin(20\theta)=\frac{10\,604\sqrt2}{59\,049}.
\]

Multiplying by the modulus \(59\,049\) gives

\[
z^{20}
   = 57\,113 + 10\,604\sqrt2\,i.
\]

---

0 2.  Final answer (rectangular form)

\[
\boxed{\;
     \bigl(\sqrt{2}-i\bigr)^{20}
     \;=\;
     57\,113 + 10\,604\,\sqrt{2}\,\,i
     \;}
\]

Because any alternative argument for \(z\) differs by a multiple of
\(2\pi\), which becomes \(40\pi\) after multiplication by \(20\),
no new values arise—**this is the only value** of the twentieth power.
\pagebreak
\[
\boxed{\;
x=\frac{-\,b\;\pm\;\sqrt{\,b^{2}-4ac\,}}{2a}
\;}
\]

\begin{itemize}
  \item It solves any quadratic equation in standard form
        \(\displaystyle ax^{2}+bx+c=0,\;a\neq0\).
  \item The quantity \(\Delta=b^{2}-4ac\) is the \textbf{discriminant}:
        \begin{itemize}
          \item \(\Delta>0\) \(\rightarrow\) two distinct real roots.
          \item \(\Delta=0\) \(\rightarrow\) one real root (double root).
          \item \(\Delta<0\) \(\rightarrow\) two complex-conjugate roots.
        \end{itemize}
\end{itemize}
%-----------------------------------------------------------------
%  Identity  (c)  :    \displaystyle \arccos z=\frac{\pi}{2}-\arcsin z
%                      for every \(z\in\mathbb{C}\setminus\{\pm i\}\),
%                      where  \(\displaystyle \arcsin z=-\,i\Log\bigl( iz+\sqrt{1-z^{2}}\bigr)\)
%                      and  \(\Log\) is the principal logarithm.
%-----------------------------------------------------------------
\pagebreak
\paragraph{Step 1.  Analyticity and derivatives of the inverse functions}

On the cut plane \(\mathbb{C}\setminus\{\pm1\}\) we fix the principal branch of
\(\sqrt{\,1-z^{2}\,}\,(>0\text{ on }(-1,1))\) and of
\(\Log(\,\cdot\,)\;( -\pi<\Arg\le\pi)\).
With these conventions
\[
\boxed{\;
\arcsin z
   =-\,i\Log\!\bigl( iz+\sqrt{1-z^{2}}\bigr)
\;}
\quad\Longrightarrow\quad
\arcsin' z
   =\frac{1}{\sqrt{1-z^{2}}}.
\]

A standard (and equivalent) definition for the principal branch of
\(\arccos\) is
\[
\boxed{\;
\arccos z
 =\frac{\pi}{2}+i\Log\!\bigl( z+\sqrt{z^{2}-1}\bigr)
\;}
\quad\Longrightarrow\quad
\arccos' z
   =-\frac{1}{\sqrt{1-z^{2}}}.
\]

Both functions are analytic on
\(\mathbb{C}\setminus\{\pm1\}\subset\mathbb{C}\setminus\{\pm i\}\).
(Excluding \(z=\pm i\) keeps us clear of the logarithm’s branch cut in
the \(\arcsin\) formula.)

\paragraph{Step 2.  The sum \(\arccos z+\arcsin z\) is constant}

Put
\[
F(z):=\arccos z+\arcsin z
      \qquad\bigl(z\in\mathbb{C}\setminus\{\pm1\}\bigr).
\]
Because
\[
F'(z)
   =\arccos' z+\arcsin' z
   =-\frac{1}{\sqrt{1-z^{2}}}
    +\frac{1}{\sqrt{1-z^{2}}}=0,
\]
\(F\) is \emph{constant} on every simply connected component of the
domain.  To find that constant, evaluate at the real point \(z=0\):
\[
\arcsin 0 = 0,
\qquad
\arccos 0 = \frac{\pi}{2}
\;\Longrightarrow\;
F(0)=\frac{\pi}{2}.
\]
Hence for all
\(z\in\mathbb{C}\setminus\{\pm1\}\) (and \emph{a fortiori} for
\(z\in\mathbb{C}\setminus\{\pm i\}\)):
\[
\boxed{\;
\arccos z+\arcsin z=\frac{\pi}{2}
\;}
\quad\Longrightarrow\quad
\boxed{\;
\arccos z=\frac{\pi}{2}-\arcsin z
\;}
\]

\paragraph{Step 3.  Compatibility with the principal branches}

For real \(z\in(-1,1)\) we have \( \arcsin z\in(-\pi/2,\pi/2)\), so
\(\pi/2-\arcsin z\in(0,\pi)\), exactly the principal range of
\(\arccos\).  Because the two functions agree on this interval and are
analytic on the cut plane, the identity extends to the whole domain
by the principle of analytic continuation.  This completes the proof.

\begin{align}
  \cos^{-1}z=\theta \\[10pt] 
  z = \cos(\theta), \cos(\theta)=\frac{e^{i\theta}+e^{-i\theta}}{2} \\[10pt] 
  \cos(\theta)=\frac{e^{i\theta}+e^{-i\theta}}{2} = z \\[10pt] 
  e^{i\theta}(e^{i\theta}+e^{-i\theta})=2e^{i\theta}z \\[10pt] 
  e^{2i\theta}+1=2e^{i\theta}z \\[10pt] 
  e^{2i\theta}-2e^{i\theta}z+1=0 \\[10pt] 
  u =e^{i\theta} \\[10pt] 
  u^{2}-2uz +1=0\\[10pt] 
  u= \frac{2z \pm \sqrt{4z^{2}-4}}{2}\\[10pt] 
  u = z \pm \sqrt{z^{2}-1} \\[10pt] 
  e^{i\theta}=z \pm \sqrt{z^{2}-1} \\[10pt] 
  i\theta = \ln \left[ z \pm \sqrt{z^{2}-1}  \right] \\[10pt] 
  \theta = -i \ln \left[ z\pm \sqrt{z^{2}-1}  \right] 
\end{align}
\pagebreak
%------------------------------------------------------------
%  Identity between complex inverse cosine and sine
%  \arccos z = \pi/2 - \arcsin z
%
%  Valid for every \(z \in \mathbb{C} \setminus\{\pm i\}\).
%  Principal branches:
%     - square root  \sqrt{\;}  (cut along \((-\infty,-1]\cup[1,\infty)\))
%     - logarithm    \Log       (-\pi < \Arg \le \pi)
%------------------------------------------------------------

\begin{align*}
  &\textbf{1.\; Deriving a logarithmic formula for } \arccos z\\[4pt]
  &\qquad
  \text{Let } \theta = \arccos z \;\; \Longrightarrow \;\;
       z=\cos\theta
         =\tfrac12\bigl(e^{i\theta}+e^{-i\theta}\bigr). \\[6pt]
  &\qquad
  \text{Set } u = e^{i\theta}\neq 0. \text{ Then } \displaystyle
       u^2 - 2zu + 1 = 0. \\[6pt]
  &\qquad
  u
    = z \pm \sqrt{z^{2}-1}. \\[6pt]
  &\qquad
  \text{Choose the \(+\) sign so that } -\pi<\Im\theta\le\pi:
       \quad e^{i\theta}= z + \sqrt{z^{2}-1}. \\[6pt]
  &\qquad
  i\theta = \Log\!\bigl(z+\sqrt{z^{2}-1}\bigr)
  \;\; \Longrightarrow\;\;
  \boxed{\;
     \arccos z
       = -\,i\,\Log\!\bigl(z+\sqrt{z^{2}-1}\bigr)
    \;}
  \tag{1}
  \end{align*}
  
  \bigskip
  
  \begin{align*}
  &\textbf{2.\; Given definition of } \arcsin z \\[4pt]
  &\qquad
  \boxed{\;
     \displaystyle
     \arcsin z = -\,i\,\Log\!\bigl( iz + \sqrt{1-z^{2}} \bigr)
    \;}
  \tag{2}
  \end{align*}
  
  \bigskip
  
  \begin{align*}
  &\textbf{3.\; Proving } \arccos z + \arcsin z = \tfrac{\pi}{2}\\[4pt]
  &\qquad
  \arccos z + \arcsin z
    = -\,i\,\Log\!\bigl(z+\sqrt{z^{2}-1}\bigr)
      \;-\;
      i\,\Log\!\bigl( iz + \sqrt{1-z^{2}}\bigr) \\[4pt]
  &\qquad
  \hphantom{=} -\,i\,
     \Log\!\Bigl(
        (z+\sqrt{z^{2}-1})
        \bigl( iz + \sqrt{1-z^{2}} \bigr)
     \Bigr) \\[4pt]
  &\qquad
  \text{But } (z+\sqrt{z^{2}-1})
              \bigl( iz + \sqrt{1-z^{2}} \bigr)
          = -\,i, \\[6pt]
  &\qquad
  \Longrightarrow\;
  \arccos z + \arcsin z
     = -\,i\,\Log(-\,i)
     = -\,i\!\Bigl(i\tfrac{\pi}{2}\Bigr)
     = \frac{\pi}{2}.
  \end{align*}
  
  \bigskip
  
  \[
  \boxed{\;
     \arccos z \;=\; \frac{\pi}{2} \;-\; \arcsin z,
     \qquad z \in \mathbb{C} \setminus \{\pm i\}
   \;}
  \]
  
  \bigskip
  
  \noindent
  \emph{Remarks.}
  \begin{itemize}
    \item The logarithm addition law \(\Log(ab)=\Log a+\Log b\) holds
          because both arguments lie in the domain of the principal branch,
          so no branch jump occurs.
    \item The exclusion \(z=\pm i\) keeps the argument of the logarithm in
          (\ref{2}) off the branch cut of the principal logarithm.
  \end{itemize}
  \begin{align}
    \ln(w + \sqrt{w^{2}-1}) = -i\log(iw + i\sqrt{w^{2}+1})  
  \end{align}
  \pagebreak
  %------------------------------------------------------------
%  Complex inverse hyperbolic cosine and related logarithmic
%  identities – detailed derivation in principal branches

\section*{1.\,  Is the identity
  $\displaystyle\ln\!\bigl(w+\sqrt{w^{2}-1}\bigr)
                     =-\,i\,\Log\!\bigl(iw+i\sqrt{w^{2}+1}\bigr)$
  true?}

\paragraph{Answer.}
With principal branches
(\,$-\pi<\Arg\le\pi$ for $\Log$ and the standard cut
$\mathbb{C}\setminus(-\infty,0]$ for $\sqrt{\;}$\,) the equality is \emph{false
in general}.  In fact
\[
-\,i\,\Log\!\bigl(iw+i\sqrt{w^{2}+1}\bigr)
   =\ln\!\bigl(w+\sqrt{w^{2}+1}\bigr)+\frac{\pi}{2},
\]
so even the inner radicals differ.\footnote{%
Because $\Log(iw+i\sqrt{w^{2}+1})
        =\Log i + \Log\!\bigl(w+\sqrt{w^{2}+1}\bigr)
        = i\pi/2 + \Log\!\bigl(w+\sqrt{w^{2}+1}\bigr)$.}

The correct (principal-branch) formulas for the inverse
trigonometric/hyperbolic functions are
\[
\boxed{\;
   \arcsin z
   =-\,i\,\Log\!\bigl( iz+\sqrt{1-z^{2}}\bigr)
 \;},\qquad
\boxed{\;
   \operatorname{arccosh} w
   =\Log\!\bigl(w+\sqrt{w^{2}-1}\bigr)
 \;}.
\]

\bigskip

\section*{2.\,  Proof of
  $\displaystyle\operatorname{arccosh}w
     =\Log\!\bigl(w+\sqrt{w^{2}-1}\bigr)$ \\
  for $w\in\mathbb{C}\setminus(-\infty,1]$}

\subsection*{2.1  Solving $\cosh u = w$}

\[
\cosh u = \frac{e^{u}+e^{-u}}{2}=w
  \quad\Longrightarrow\quad
  e^{2u}-2we^{u}+1=0.
\]

Let \(v=e^{u}\;(\neq0)\).  The quadratic gives
\[
v= w\pm\sqrt{\,w^{2}-1\,}.
\]
Choose the \emph{plus} sign so that \(v\) avoids the branch cut
\((-\infty,0]\); this is guaranteed precisely when
\(w\notin(-\infty,1]\).

\[
u=\Log v = \Log\!\bigl(w+\sqrt{w^{2}-1}\bigr).
\]

\[
\boxed{\;
   \operatorname{arccosh}w
   = \Log\!\bigl(w+\sqrt{w^{2}-1}\bigr),
   \qquad
   w\in\mathbb{C}\setminus(-\infty,1].
 \;}
\]

\subsection*{2.2  Using $\cosh\theta=\cos(i\theta)$ (alternative route)}

Write \(w=\cosh\theta=\cos(i\theta)\).
Since the principal branch of \(\arccos\) satisfies
\(\arccos z=-\,i\,\Log\!\bigl(z+\sqrt{z^{2}-1}\bigr)\)
(see Part~1 of the previous exercise),
\[
i\theta
  =\arccos w
  =-\,i\,\Log\!\bigl(w+\sqrt{w^{2}-1}\bigr)
  \;\Longrightarrow\;
\theta
  =\Log\!\bigl(w+\sqrt{w^{2}-1}\bigr),
\]
recovering the same formula for $\operatorname{arccosh}w$.

\bigskip

\section*{3.\,  Remarks}

\begin{itemize}
  \item The domain cut $(-\infty,1]$ is exactly where the quantity
        $w+\sqrt{w^{2}-1}$ crosses the negative real axis, forcing a jump
        in the principal logarithm.  Removing that set makes the formula
        single-valued and analytic.
  \item One may also write
        $\displaystyle\operatorname{arccosh}w
                  =\pm\Log\!\bigl(w\pm\sqrt{w^{2}-1}\bigr)$;
        the chosen signs control which branch (even/odd multiple of
        $2\pi i$) you land on.  The expression above picks the principal
        branch for both $\sqrt{\;}$ and $\Log$.
\end{itemize}
%-----------------------------------------------------------------
%  All complex values of  \(\displaystyle (1+i\sqrt3)^{\tfrac{11}{3}}\)
%-----------------------------------------------------------------
\pagebreak
\begin{enumerate}
  \item \textbf{Write the base in polar form.}
  
  \[
  1+i\sqrt3 \;=\; 2\,e^{\,i\pi/3}
  \qquad
  \bigl(|\,1+i\sqrt3|=\sqrt{1+3}=2,\;
        \Arg= \pi/3\bigr).
  \]
  
  \item \textbf{Use the multi-valued power.}
  
  For a rational exponent \(\alpha=\tfrac{11}{3}\) we define  
  \(z^{\alpha}= \exp\!\bigl(\alpha\,\Log z\bigr)\) with the
  multi-valued logarithm
  \(\Log z=\ln2 + i\bigl(\frac{\pi}{3}+2\pi k\bigr),\;k\in\Bbb Z\):
  
  \[
  (1+i\sqrt3)^{11/3}
        \;=\;
        \exp\!\Bigl(
            \tfrac{11}{3}
            \bigl[\ln 2 + i(\tfrac{\pi}{3}+2\pi k)\bigr]
          \Bigr)
        \;=\;
        2^{11/3}\,
        e^{\,i\bigl(\tfrac{11\pi}{9}+\tfrac{22\pi}{3}k\bigr)},
        \qquad k\in\Bbb Z.
  \]
  
  Because the denominator \(3\) is already in lowest terms,
  distinct values arise for \(k=0,1,2\).
  
  \item \textbf{List the three distinct values in
                  \(x+iy\) form.}
  
  Set \(R=2^{11/3}=8\,\sqrt[3]{4}\).
  
  \[
  \begin{array}{c|c|c|c}
  k & \text{Argument } \theta_k
    & x_k = R\cos\theta_k & y_k = R\sin\theta_k \\ \hline
  0 & \displaystyle\frac{11\pi}{9}-2\pi
      =-\frac{7\pi}{9}
    & -\,R\cos\tfrac{2\pi}{9}
    & -\,R\sin\tfrac{2\pi}{9} \\[6pt]
  1 & \displaystyle\frac{77\pi}{9}-4\pi
      =\frac{5\pi}{9}
    & -\,R\cos\tfrac{\pi}{9}
    & \;\;R\sin\tfrac{\pi}{9} \\[6pt]
  2 & \displaystyle\frac{143\pi}{9}-8\pi
      =-\frac{\pi}{9}
    & \;\;R\cos\tfrac{\pi}{9}
    & -\,R\sin\tfrac{\pi}{9}
  \end{array}
  \]
  
  Numerically (rounded to three decimals, using \(R\approx 15.119\)):
  
  \[
  \boxed{
  \begin{aligned}
  k=0 &: \;-11.584 - 9.380\,i,\\[2pt]
  k=1 &: \;-2.623  + 14.882\,i,\\[2pt]
  k=2 &: \;14.207 - 2.623\,i.
  \end{aligned}}
  \]
  
  Thus the set of all complex values is  
  
  \[
  \Bigl\{
    2^{11/3}\,e^{-7\pi i/9},\;
    2^{11/3}\,e^{5\pi i/9},\;
    2^{11/3}\,e^{-\pi i/9}
  \Bigr\},
  \]
  each expressed above in the requested \(x+iy\) form.
  \end{enumerate}
\end{document}
