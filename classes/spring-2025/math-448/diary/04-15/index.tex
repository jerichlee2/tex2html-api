\documentclass[12pt]{article}

% Packages
\usepackage[margin=1in]{geometry}
\usepackage{amsmath,amssymb,amsthm}
\usepackage{enumitem}
\usepackage{hyperref}
\usepackage{xcolor}
\usepackage{import}
\usepackage{xifthen}
\usepackage{pdfpages}
\usepackage{transparent}
\usepackage{listings}
\usepackage{tikz}

\DeclareMathOperator{\Log}{Log}
\DeclareMathOperator{\Arg}{Arg}

\lstset{
    breaklines=true,         % Enable line wrapping
    breakatwhitespace=false, % Wrap lines even if there's no whitespace
    basicstyle=\ttfamily,    % Use monospaced font
    frame=single,            % Add a frame around the code
    columns=fullflexible,    % Better handling of variable-width fonts
}

\newcommand{\incfig}[1]{%
    \def\svgwidth{\columnwidth}
    \import{./Figures/}{#1.pdf_tex}
}
\theoremstyle{definition} % This style uses normal (non-italicized) text
\newtheorem{solution}{Solution}
\newtheorem{proposition}{Proposition}
\newtheorem{problem}{Problem}
\newtheorem{lemma}{Lemma}
\newtheorem{theorem}{Theorem}
\newtheorem{remark}{Remark}
\newtheorem{note}{Note}
\newtheorem{definition}{Definition}
\newtheorem{example}{Example}
\newtheorem{corollary}{Corollary}
\theoremstyle{plain} % Restore the default style for other theorem environments
%

% Theorem-like environments
% Title information
\title{}
\author{Jerich Lee}
\date{\today}

\begin{document}

\maketitle


\section*{Maximum Modulus and Mean Value: A Step-by-Step Walkthrough}

\noindent
\textbf{Setup.} Suppose that $f$ is an analytic function on a domain $D$. From earlier results,
we know that if the range of $f$ lies in a circle or on a straight line, then $f$ must be constant.
We now show a stronger statement: \emph{either} $f$ is constant on $D$ \emph{or} the range of $f$
is an open set.

\medskip

\noindent
\textbf{Step 1. Assume $f$ is nonconstant and fix a point in its range.}\\
Let $f$ be not identically constant, and choose some $z_0$ in $D$. Denote $w_0 = f(z_0)$. This $w_0$
is thus in the range of $f$.

\medskip

\noindent
\textbf{Step 2. The function $f(z) - w_0$ has an isolated zero of order $m \ge 1$.}\\
Consider the function $g(z) = f(z) - w_0$. Since $f$ is not constant, $g(z)$ is not the zero function.
In fact, $g$ has a zero of order $m \ge 1$ at $z = z_0$. 

\medskip

\noindent
\textbf{Step 3. Choose $r>0$ so that $g(z)$ has no other zeros on $0 < |z - z_0| \le r$.}\\
Because the zeros of a nonconstant analytic function are isolated, we can choose $r > 0$ small enough
so that $g(z) = f(z) - w_0$ does not vanish anywhere on $0 < |z - z_0| \le r$ (i.e.\ $z_0$ is the only
zero inside the disk of radius $r$).

\medskip

\noindent
\textbf{Step 4. Define $\delta$ as the minimum value of $|g(z)|$ on the boundary $|z - z_0| = r$.}\\
Set
\[
  \delta = \min_{\substack{|z - z_0| = r}} \bigl|f(z) - w_0\bigr|.
\]
Since $f(z) - w_0$ is continuous and we are looking at a compact set (the circle $|z - z_0| = r$),
this minimum is well-defined and positive (because $g(z)$ has no zeros on that circle).

\medskip

\noindent
\textbf{Step 5. Take any $w$ with $|w - w_0| < \delta$, and compare $f(z) - w$ vs.\ $f(z) - w_0$.}\\
For $|z - z_0| = r$, we have
\[
  \bigl|\bigl(f(z) - w\bigr) - \bigl(f(z) - w_0\bigr)\bigr|
  = |w - w_0| < \delta \le |f(z) - w_0|.
\]
This is the crucial inequality (sometimes labeled as equation (1) in the text).

\medskip

\noindent
\textbf{Step 6. Apply Rouch\'e's theorem.}\\
By Rouch\'e's theorem, on the circle $|z - z_0| = r$, the two functions $f(z) - w$ and $f(z) - w_0$
have the \emph{same} number of zeros inside that circle.  Since $f(z) - w_0$ has exactly $m$ zeros
(with multiplicity) at $z_0$, it follows that $f(z) - w$ must also have exactly $m$ zeros in
$|z - z_0| < r$.

\medskip

\noindent
\textbf{Step 7. Conclude that every $w$ near $w_0$ is in the range of $f$.}\\
Because $f(z) - w$ has a zero in $|z - z_0| < r$, it means there is some $z$ in that disk for which
$f(z) = w$.  In other words, for \emph{all} $w$ with $|w - w_0| < \delta$, there exists a $z$ such that
$f(z) = w$.  Hence, each point $w_0$ in the range is the center of a small disk (of radius $\delta$)
lying entirely in the range of $f$.  Therefore, the range of $f$ is open.

\begin{theorem}[Open Mapping Theorem for Analytic Functions]
Suppose that $f$ is a nonconstant analytic function on a domain $D$. Then the set 
$\{\,f(z) : z \in D\}$ is an open set.
\end{theorem}
\section*{Why $f(z) - w_0$ Has an Isolated Zero of Order $m \ge 1$}

\noindent
\textbf{Setup.} Let $f$ be an analytic function on a domain $D$ and $z_0 \in D$ such that
$w_0 = f(z_0)$.  Define 
\[
  g(z) \;=\; f(z) \;-\; w_0.
\]

\medskip

\noindent
\textbf{Step 1. $g$ is not the zero function.}\\
If $f$ is \emph{not} identically constant, there exists at least one point $z_1 \in D$ with
$f(z_1) \neq w_0$, hence $g(z_1) = f(z_1) - w_0 \neq 0$.  Therefore, $g$ itself cannot be the zero
function (which would vanish everywhere).

\medskip

\noindent
\textbf{Step 2. $g(z_0) = 0$.}\\
By definition, $g(z_0) = f(z_0) - w_0 = w_0 - w_0 = 0$.  Thus $z_0$ is a zero of $g$.

\medskip

\noindent
\textbf{Step 3. Zeros of nonconstant analytic functions are isolated.}\\
Since $g$ is analytic but \emph{not} identically zero, all its zeros must be isolated unless $g$
were the zero function.  Therefore, there exists some radius $r > 0$ such that $z_0$ is the
only zero of $g$ in the disk $\{\,z : |z - z_0| < r\}$.

\medskip

\noindent
\textbf{Step 4. The order of the zero $m \ge 1$.}\\
By analytic theory, we can write the Taylor expansion of $g(z)$ around $z_0$.  In particular,
there is a smallest integer $m \ge 1$ such that
\[
  g(z) \;=\; (z - z_0)^m \, h(z),
\]
where $h(z)$ is an analytic function with $h(z_0) \neq 0$.  The integer $m$ is called the
\emph{order} of the zero of $g$ at $z_0$.

\medskip

\noindent
\textbf{Conclusion.}\\
Putting it all together: $g(z) = f(z) - w_0$ has an isolated zero at $z = z_0$ (by Step 3), and
the zero is of finite order $m \ge 1$ (by Step 4).  Hence we say
``\,$f(z) - w_0$ has an isolated zero of order $m \ge 1$ at $z = z_0$\,'' as required.

\section*{Local $m$-to-1 Behavior and Non-Invertibility}

\noindent
\textbf{Statement.} 
Suppose $f$ is a nonconstant analytic function on a domain $D$ and that
$f - f(z_0)$ has a zero of order $m \ge 1$ at $z_0$. Then in a neighborhood of $z_0$,
the map $f$ behaves like an $m$-to-1 covering. In particular, if $f'(z_0) = 0$ (so $m \ge 2$),
then $f$ cannot be injective in any disc containing $z_0$.

\bigskip

\noindent
\textbf{Step 1. Express $f$ in a local factorization around $z_0$.}\\
Since $f - f(z_0)$ has a zero of order $m$ at $z_0$, we can write
\[
  f(z) \;=\; f(z_0) \;+\; (z - z_0)^{m}\,h(z),
\]
where $h$ is analytic in a neighborhood of $z_0$ and $h(z_0) \neq 0$.

\medskip

\noindent
\textbf{Step 2. Local invertibility and the $m$-to-1 property.}\\
Define a local coordinate $w$ around $z_0$ by $w = z - z_0$. Then near $z_0$, we have
\[
  f(z) \;=\; f(z_0) + w^m \,h(z_0 + w).
\]
Because $h(z_0) \neq 0$, the function $z \mapsto w^m$ shows that, for any value close to $f(z_0)$,
there are $m$ distinct solutions for $z$ (provided $w$ is small and $m \ge 1$). Hence in a small
enough disc around $z_0$, each value of $f$ has exactly $m$ \emph{local} preimages.

\medskip

\noindent
\textbf{Step 3. Consequence when $m \ge 2$.}\\
If $m \ge 2$, in particular if $f'(z_0) = 0$, then there are at least two distinct solutions for $z$
that map to the same value of $f(z_0 + w)$ near $f(z_0)$. Thus $f$ is \emph{not} injective on any
disc containing $z_0$. 

\medskip

\noindent
\textbf{Conclusion.}\\
We conclude that $f$ is locally $m$-to-1 in a small neighborhood of $z_0$. Specifically, if $m=1$
(that is, if $f'(z_0) \neq 0$), then $f$ is locally invertible near $z_0$. If $m \ge 2$, $f$ is
\emph{not} one-to-one in any disc around $z_0$.
\begin{theorem}[Schwarz's Lemma]
    Suppose that $f$ is analytic on the open unit disk $\{\, z : |z| < 1\}$, with $f(0) = 0$ 
    and $|f(z)| \le 1$ for all $|z| < 1$. Then for each $z$ with $|z|<1$,
    \[
       |f(z)| \;\le\; |z|.
    \]
    Moreover, if there exists some $z_0 \neq 0$ with $|f(z_0)| = |z_0|$, then $f$ must be of the form
    $f(z) = \lambda \, z$ for some constant $\lambda$ with $|\lambda| = 1$.
    \end{theorem}
    
    \begin{proof}
    \textbf{Step 1. Define $g(z)$.}\\
    Since $f(0)=0$, we can define 
    \[
       g(z) \;=\;
       \begin{cases}
         \dfrac{f(z)}{z}, & z \neq 0, \\[6pt]
         f'(0), & z = 0.
       \end{cases}
    \]
    Because $f$ is analytic and $f(0) = 0$, one can check by applying l'H\^opital's rule or by looking
    at the power series expansion that $g$ is continuous on $|z|<1$, and in fact is analytic there (the
    limit defining $g(0) = f'(0)$ matches the analytic continuation).
    
    \medskip
    
    \noindent
    \textbf{Step 2. Estimate $|g(z)|$ on the circle $|z| = r$.}\\
    Fix $0 < r < 1$ and consider $|z|=r$.  We then have
    \[
       |g(z)| \;=\; \biggl|\frac{f(z)}{z}\biggr|
                \;\le\;\frac{|f(z)|}{|z|}
                \;\le\;\frac{1}{r},
    \]
    because $|f(z)| \le 1$ for $|z|<1$.  Hence on the circle $|z| = r$, we see that
    \[
      |g(z)| \;\le\;\frac{1}{r}.
    \]
    
    \medskip
    
    \noindent
    \textbf{Step 3. Apply the Maximum Modulus Principle.}\\
    Since $g$ is analytic on $|z|<1$ and continuous on the closed disk $|z|\le r$, the maximum‐modulus
    principle implies that the maximum of $|g(z)|$ on the disk $|z|\le r$ occurs on the boundary
    $|z|=r$.  We have already established $|g(z)| \le \frac{1}{r}$ for $|z|=r$.  Therefore,
    \[
      |g(z)| \;\le\;\frac{1}{r}
      \quad\text{for all } |z| \le r.
    \]
    
    \medskip
    
    \noindent
    \textbf{Step 4. Let $r$ approach 1.}\\
    Since $r$ can be chosen arbitrarily close to 1 (but still $r<1$), we conclude that for any
    $|z|<1$,
    \[
       |g(z)| \;\le\; 1.
    \]
    Equivalently, $|f(z)| = |z|\cdot|g(z)| \le |z|\cdot 1 = |z|$. 
    Hence
    \[
      |f(z)| \;\le\; |z|
      \quad\text{for all } |z|<1.
    \]
    
    \medskip
    
    \noindent
    \textbf{Step 5. Characterize the equality case.}\\
    Suppose there is some $z_0 \neq 0$ with $|f(z_0)| = |z_0|$.  Then 
    \[
       |g(z_0)| \;=\; \frac{|f(z_0)|}{|z_0|} \;=\; 1.
    \]
    But from Step~4, we have $|g(z)| \le 1$ for \emph{all} $z$ with $|z|<1$.  Hence $|g(z)|$ attains its
    maximum value $1$ at an interior point $z_0$ with $|z_0|<1$.  By the \emph{strong form} of the
    maximum‐modulus principle, $g$ must be a constant function on $|z|<1$.  Denote this constant by
    $\lambda$ with $|\lambda|=1$.  Thus
    \[
       g(z) \;=\; \lambda, 
       \qquad\text{so}\quad f(z) \;=\; \lambda \, z
       \quad\text{for all } |z|<1.
    \]
    Hence if the equality $|f(z_0)| = |z_0|$ holds for some nonzero $z_0$, then $f$ must be a rotation
    of the form $f(z) = \lambda z$ with $|\lambda|=1$.
    
    \end{proof}
    \section*{Mean Value Theorem for Analytic Functions}

    \noindent
    \textbf{Statement.} 
    If $f$ is an analytic function on a domain $D \subset \mathbb{C}$ and if
    $z_0 \in D$ is such that the closed disk $\{ z : |z - z_0| \le r \}$ is contained in $D$, 
    then
    \[
      f(z_0)
      \;=\;
      \frac{1}{2\pi}
      \int_{0}^{2\pi}
          f\bigl(z_0 + r e^{i t}\bigr)\,dt,
    \]
    where $0 \le t < 2\pi$.  This says that \emph{the value of $f$ at the center $z_0$ is the
    average of its values on the circle $|z - z_0| = r$}.
    
    \bigskip
    
    \noindent
    \textbf{Step 1. Recall Cauchy's Integral Formula.}\\
    By Cauchy's Integral Formula (Chapter~2, Theorem~4 in many texts), for a circle $\gamma$ of radius $r$ around $z_0$,
    \[
      f(z_0)
      \;=\;
      \frac{1}{2\pi i}
      \int_{\gamma} \frac{f(\zeta)}{\zeta - z_0}\, d\zeta,
    \]
    provided that $f$ is analytic in a simply connected domain containing $\gamma$ and $z_0$.
    
    \medskip
    
    \noindent
    \textbf{Step 2. Parametrize the circle.}\\
    Let $\gamma$ be the circle $|\zeta - z_0| = r$.  A standard parametrization is
    \[
      \zeta = z_0 + r e^{i t},
      \quad t \in [0, 2\pi].
    \]
    Then $d\zeta = i r e^{i t} \,dt$, and $\zeta - z_0 = r e^{i t}$.
    
    \medskip
    
    \noindent
    \textbf{Step 3. Rewrite the integral.}\\
    Substituting $\zeta = z_0 + r e^{i t}$ into Cauchy's formula gives
    \[
      f(z_0)
      = \frac{1}{2\pi i}
        \int_{0}^{2\pi}
          \frac{f\bigl(z_0 + r e^{i t}\bigr)}{r e^{i t}}
          \cdot \bigl(i r e^{i t}\bigr)\, dt.
    \]
    Inside the integral, the factor $\frac{1}{r e^{i t}}$ multiplies $i r e^{i t}$ to yield $i$.  Hence
    \[
      f(z_0)
      = \frac{1}{2\pi}
        \int_{0}^{2\pi}
          f\bigl(z_0 + r e^{i t}\bigr)\, dt.
    \]
    Thus, we obtain the \emph{Mean‐Value Theorem} for $f$:
    \[
      f(z_0)
      = \frac{1}{2\pi}
        \int_{0}^{2\pi}
        f\bigl(z_0 + r e^{i t}\bigr)\, dt.
    \]
    
    \medskip
    
    \noindent
    \textbf{Step 4. Real (and imaginary) parts give the mean‐value property for harmonic functions.}\\
    Let $u = \Re(f)$ be the real part of $f$.  Taking real parts in the above formula yields
    \[
      u(z_0)
      = \Re\bigl(f(z_0)\bigr)
      = \Re\!\biggl(\,\frac{1}{2\pi}\int_{0}^{2\pi} f(z_0 + re^{i t})\, dt\biggr)
      = \frac{1}{2\pi}\int_{0}^{2\pi}
          \Re\bigl(f(z_0 + re^{i t})\bigr)\, dt
      = \frac{1}{2\pi}\int_{0}^{2\pi} u\bigl(z_0 + re^{i t}\bigr)\, dt.
    \]
    Hence $u$ itself satisfies the \emph{harmonic mean‐value property}: 
    the value of $u$ at the center is the \emph{average} of its values on the boundary circle.
    
    \medskip
    
    \noindent
    \textbf{Step 5. Why is this important?}
    \begin{itemize}
    \item \emph{Connection to Harmonic Functions:} 
      Since the real (and imaginary) parts of an analytic function are harmonic, the above formula 
      immediately shows that harmonic functions have the mean‐value property.  This is essential 
      in studying \emph{potential theory} and PDEs (partial differential equations).
    \item \emph{Local Behavior and Uniqueness:} 
      The mean‐value property underlies important uniqueness theorems: for example, if a harmonic 
      function attains its maximum (or minimum) in the interior of a domain, it must be constant 
      (the \emph{Maximum Principle} for harmonic functions).
    \item \emph{Recovering Information on the Inside from the Boundary:} 
      In engineering and physics, boundary data (on a circle or sphere) can be used to reconstruct 
      interior values of a harmonic or analytic function.  Formula~(5)/(6) in the text is a 
      one‐dimensional case of a more general integral representation.
    \item \emph{Bridge to Deeper Results:} 
      The mean‐value property sets the stage for more advanced results, such as Harnack's inequality, 
      the Poisson integral formula, and the general approach to boundary value problems for Laplace's equation.
    \end{itemize}
    
    \bigskip
    
    \begin{remark}
    In summary, the mean‐value theorem for analytic functions (and its harmonic corollary) tells us 
    that \textbf{knowing the boundary values} of an analytic or harmonic function on a circle often 
    \textbf{uniquely determines} the function’s interior values. It is fundamental in both theoretical 
    analysis and numerous physical applications (heat, electrostatics, fluid flow, etc.).
    \end{remark}    
\end{document}
