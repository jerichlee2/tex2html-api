\documentclass[12pt]{article}

% Packages
\usepackage[margin=1in]{geometry}
\usepackage{amsmath,amssymb,amsthm}
\usepackage{enumitem}
\usepackage{hyperref}
\usepackage{xcolor}
\usepackage{import}
\usepackage{xifthen}
\usepackage{pdfpages}
\usepackage{transparent}
\usepackage{listings}
\usepackage{tikz}
\usepackage{physics}
\usepackage{siunitx}
\usepackage{booktabs}
\usepackage{cancel}
  \usetikzlibrary{calc,patterns,arrows.meta,decorations.markings}


\DeclareMathOperator{\Log}{Log}
\DeclareMathOperator{\Arg}{Arg}

\lstset{
    breaklines=true,         % Enable line wrapping
    breakatwhitespace=false, % Wrap lines even if there's no whitespace
    basicstyle=\ttfamily,    % Use monospaced font
    frame=single,            % Add a frame around the code
    columns=fullflexible,    % Better handling of variable-width fonts
}

\newcommand{\incfig}[1]{%
    \def\svgwidth{\columnwidth}
    \import{./Figures/}{#1.pdf_tex}
}
\theoremstyle{definition} % This style uses normal (non-italicized) text
\newtheorem{solution}{Solution}
\newtheorem{proposition}{Proposition}
\newtheorem{problem}{Problem}
\newtheorem{lemma}{Lemma}
\newtheorem{theorem}{Theorem}
\newtheorem{remark}{Remark}
\newtheorem{note}{Note}
\newtheorem{definition}{Definition}
\newtheorem{example}{Example}
\newtheorem{corollary}{Corollary}
\theoremstyle{plain} % Restore the default style for other theorem environments
%

% Theorem-like environments
% Title information
\title{MATH-448 Practice Final Exam 7}
\author{Jerich Lee}
\date{\today}

\begin{document}

\maketitle
\pagebreak

  
  %-----------------------------------------------------------------
  \pagebreak
  \begin{problem}[Values and properties of elementary functions]\mbox{}\\[4pt]
  \begin{enumerate}[label=(\alph*),itemsep=6pt]
    \item Determine \emph{all} complex values of 
          \[
            \bigl(1+i\sqrt{3}\bigr)^{\frac{11}{3}},
          \]
          and write each value in the form $x+iy$ with $x,y\in\mathbb{R}$.
    \item Compute the principal value of the logarithm
          \[
            \Log\!\bigl(-3i\bigr),
          \]
          and state precisely the branch cut you have used.
    \item Prove the identity
          \[
            \arccos z
            \;=\;
            \tfrac{\pi}{2}-\arcsin z
            \;=\;
            \tfrac{\pi}{2}
            -\,\frac{1}{i}\,\Log\!\bigl(z+i\sqrt{1-z^{2}}\bigr),
            \qquad z\in\mathbb{C}\setminus\bigl(\,(-\infty,-1]\cup[1,\infty)\bigr),
          \]
          taking care to justify analyticity on the indicated domain.
  \end{enumerate}
  \end{problem}
  
  %-----------------------------------------------------------------
  \pagebreak
  \begin{problem}[Isolated singularities and residues]\mbox{}\\[4pt]
  Consider
  \[
    f(z)=\frac{e^{\frac{1}{z-\pi}}-\cos z}{(z+2)^{3}}.
  \]
  \begin{enumerate}[label=(\alph*),itemsep=6pt]
    \item Classify the isolated singularities of $f$ at $z=\pi$ and $z=-2$ as removable, poles, or essential.
    \item Compute both residues $\operatorname*{Res}_{z=\pi}f(z)$ and $\operatorname*{Res}_{z=-2}f(z)$.
    \item Write the first \emph{four} non‑zero terms of the Laurent expansion of $f$ about $z=\pi$.
  \end{enumerate}
  \end{problem}
  
  %-----------------------------------------------------------------
  \pagebreak
  \begin{problem}[Laurent expansion in an annulus — multi‑step]\mbox{}\\[4pt]
  Find the Laurent expansion
  \[
    \sum_{n=-\infty}^{\infty}a_{n}\,(z-1)^{n}
  \]
  of the function
  \[
    g(z)=\frac{z^{2}}{(z-3)(z+1)^{2}}
  \]
  in the annulus \(2<\lvert z-1\rvert<4\) as follows:
  \begin{enumerate}[label=(\alph*),itemsep=6pt]
    \item Show that, on the given annulus, one may write
          \[
            \frac{1}{z-3}
            \;=\;
            -\frac{1}{2}\,
            \sum_{k=0}^{\infty}\bigl(\tfrac{z-1}{2}\bigr)^{k}.
          \]
    \item Prove that, again on the annulus, 
          \[
            \frac{1}{(z+1)^{2}}
            \;=\;
            \frac{1}{4}
            \sum_{m=0}^{\infty}(-1)^{m}(m+1)
            \Bigl(\tfrac{z-1}{2}\Bigr)^{m}.
          \]
    \item Multiply the two series from (a) and (b) term‑by‑term, then multiply by $z^{2}$, to obtain the full Laurent expansion for $g$.
    \item Give a closed‑form expression for the general coefficient $a_{n}$ valid for all $n\in\mathbb{Z}$.
  \end{enumerate}
  \end{problem}
  
  %-----------------------------------------------------------------
  \pagebreak
  \begin{problem}[Closed‑contour integral]\mbox{}\\[4pt]
  Using residues, evaluate
  \[
    \oint_{\lvert z\rvert=3}
    \frac{z^{2}}{(z-1)(z+2)^{2}(z-4)}\,dz.
  \]
  (You may assume the positive—oriented—circle is traversed once.)
  \end{problem}
  
  %-----------------------------------------------------------------
  \pagebreak
  \begin{problem}[Improper integral via complex methods]\mbox{}\\[4pt]
  Show that the integral
  \[
    I=\int_{0}^{\infty}\frac{x^{2}\cos(4x)}{(x^{2}+4)^{2}}\,dx
  \]
  converges and compute its exact value.  (Contour‑integration in the upper half‑plane is recommended.)
  \end{problem}
  
  %-----------------------------------------------------------------
  \pagebreak
  \begin{problem}[Counting zeros with Rouché]\mbox{}\\[4pt]
  Let
  \[
    P(z)=z^{6}+4z^{3}-8z+2.
  \]
  Apply Rouché’s Theorem on a suitable circle to determine how many zeros of $P$ lie strictly inside the circle \(\lvert z\rvert=2\).
  \end{problem}
  
  %-----------------------------------------------------------------
  \pagebreak
  \begin{problem}[Conformal mapping]\mbox{}\\[4pt]
  Construct an explicit conformal map \(T\) that sends the right‑angled triangle
  \[
    \Delta=\{z:\;0<\arg z<\tfrac{\pi}{2},\; \lvert z\rvert<1\}
  \]
  onto the upper half‑plane \(\operatorname{Im}w>0\).  
  Indicate clearly the intermediate steps (Schwarz–Christoffel, power maps, linear fractional maps, etc.) and show how the three vertices of $\Delta$ are sent to \(-1,0,1\).
  \end{problem}
  
  %-----------------------------------------------------------------
  \pagebreak
  \begin{problem}[Schwarz–Pick type estimate]\mbox{}\\[4pt]
  Let \(f\) be analytic on the unit disc $\mathbb{D}$ and suppose $f(0)=a$ with $0<\lvert a\rvert<1$.
  \begin{enumerate}[label=(\alph*),itemsep=6pt]
    \item Use the Schwarz–Pick Theorem to prove
          \[
            \lvert f'(0)\rvert
            \;\le\;
            \frac{1-\lvert a\rvert^{2}}{1-0^{2}}
            \;=\;
            1-\lvert a\rvert^{2}.
          \]
    \item Show that equality holds in (a) if and only if $f$ is a Möbius automorphism of $\mathbb{D}$.
  \end{enumerate}
  \end{problem}
\end{document}
