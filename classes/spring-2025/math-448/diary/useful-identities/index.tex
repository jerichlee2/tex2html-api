\documentclass[9pt]{article}

% Packages
\usepackage[margin=.5in]{geometry}
\usepackage{amsmath,amssymb,amsthm}
\usepackage{enumitem}
\usepackage{hyperref}
\usepackage{xcolor}
\usepackage{import}
\usepackage{xifthen}
\usepackage{pdfpages}
\usepackage{transparent}
\usepackage{listings}


\lstset{
    breaklines=true,         % Enable line wrapping
    breakatwhitespace=false, % Wrap lines even if there's no whitespace
    basicstyle=\ttfamily,    % Use monospaced font
    frame=single,            % Add a frame around the code
    columns=fullflexible,    % Better handling of variable-width fonts
}

\newcommand{\incfig}[1]{%
    \def\svgwidth{\columnwidth}
    \import{./Figures/}{#1.pdf_tex}
}
\theoremstyle{definition} % This style uses normal (non-italicized) text
\newtheorem{solution}{Solution}
\newtheorem{proposition}{Proposition}
\newtheorem{problem}{Problem}
\newtheorem{lemma}{Lemma}
\newtheorem{theorem}{Theorem}
\newtheorem{remark}{Remark}
\newtheorem{note}{Note}
\theoremstyle{plain} % Restore the default style for other theorem environments
%

% Theorem-like environments
% Title information
\title{Useful Identities in Complex Variables}
\author{Jerich Lee}
\date{\today}

\begin{document}

\maketitle
\subsubsection*{Complex Conjugate of $z$}
\begin{align}
    \overline{z} = x-iy
\end{align}
\subsubsection*{Division of Complex Numbers}
\begin{align}
    \frac{z}{w}= \frac{(xs+yt)+i(ys-xt)}{s^{2}+t^{2}}, w \neq 0
\end{align}
\subsubsection*{$z\overline{z}$}
\begin{align}
    \left( x+iy \right)\left( x-iy \right)=x^{2}+y^{2}=\left\vert z \right\vert ^{2}  
\end{align}
\subsubsection*{Some useful equivalences}
\noindent
\begin{enumerate}
    \item \begin{align}
    \left\vert zw \right\vert = \left\vert z \right\vert \left\vert w \right\vert 
    \end{align}
    \item \begin{align}
        \overline{zw}= \overline{z} \overline{w}  
    \end{align}
\end{enumerate}
\subsubsection*{Modulus of $z$}
\begin{align}
    \left\vert z \right\vert = \sqrt{x^{2}+y^{2}} 
\end{align}
\subsubsection*{Polar Repr. of $z$}
\begin{align}
    z = \left\vert z \right\vert \left( \cos \theta+ i \sin \theta \right) 
\end{align}
\subsubsection*{DeMoivre's Theorem}
\begin{align}
   \left( \cos\theta + i \sin \theta \right)^{n}=\cos n\theta + i \sin n\theta  
\end{align}
\subsubsection*{Arguments}
\begin{align}
    \text{Arg}(zw)= \text{Arg} z+ \text{Arg} w \left( \text{mod} 2\pi \right) 
\end{align}
\subsubsection*{Closed form of $\sum_{k=1}^{n} kx^{k}$}
\begin{align}
    \sum_{k=1}^n k\,x^k
= x \cdot \frac{d}{dx}\left(\sum_{k=0}^n x^k\right)
= x \cdot \frac{d}{dx}\left(\frac{1 - x^{n+1}}{1 - x}\right)
= \frac{x \bigl[1 - (n+1)x^n + n x^{n+1}\bigr]}{(1-x)^2}.
\end{align}
\subsubsection*{Closed form of $\sum_{k=1}^{n} n$ }
\begin{align}
    n + n + \cdots + n \quad (\text{\(n\) times}) \;=\; n \cdot n \;=\; n^2.
\end{align}
\subsubsection*{Closed form of $\sum_{k=1}^{n} x^{k}$ (Geometric Series)}
\begin{align}
\sum_{k=1}^{n} x^{k}
= \frac{x \bigl(1 - x^n\bigr)}{1 - x},
\quad x \neq 1,
\quad\text{and} \quad
\sum_{k=1}^{n} 1^k = n
\end{align}
\subsubsection*{Appolonius' Circle}
    We want to show that
\[
4\Bigl(\lvert z\rvert^{2} \;-\; 2\,\mathrm{Re}(z\,\overline{i}) \;+\; \lvert i\rvert^{2}\Bigr)
\;=\;
\lvert z\rvert^{2} \;-\; 2\,\mathrm{Re}(z) \;+\; 1
\]
implies
\[
3\,\lvert z\rvert^{2} \;-\;8\,\mathrm{Im}(z) \;+\;2\,\mathrm{Re}(z)
\;=\;
-3.
\]

First, note that $\lvert z\rvert^{2} = z\,\overline{z}$ and $\lvert i\rvert^{2} = 1$.  
Also, $\mathrm{Re}(z) = \frac{z + \overline{z}}{2}$, $\mathrm{Im}(z) = \frac{z - \overline{z}}{2\,i}$, and $\overline{i} = -\,i$.  
Hence 
\[
\mathrm{Re}\bigl(z\,\overline{i}\bigr) 
= 
\mathrm{Re}(-\,i\,z) 
= 
-\,\mathrm{Im}(z).
\]

On the left, 
\[
4\Bigl(\lvert z\rvert^{2} - 2\,\mathrm{Re}[z\,\overline{i}] + \lvert i\rvert^{2}\Bigr)
=
4\Bigl(z\,\overline{z} + 2\,\mathrm{Im}(z) + 1\Bigr)
=
4\,z\,\overline{z} + 8\,\mathrm{Im}(z) + 4.
\]

On the right, 
\[
\lvert z\rvert^{2} - 2\,\mathrm{Re}(z) + 1
=
z\,\overline{z} - 2\,\mathrm{Re}(z) + 1.
\]

Equating and moving all terms to one side:
\[
4\,z\,\overline{z} + 8\,\mathrm{Im}(z) + 4
=
z\,\overline{z} - 2\,\mathrm{Re}(z) + 1
\;\;\Longrightarrow\;\;
3\,z\,\overline{z} + 2\,\mathrm{Re}(z) + 8\,\mathrm{Im}(z) + 3 
= 
0.
\]
Since $z\,\overline{z} = \lvert z\rvert^{2}$,
\[
3\,\lvert z\rvert^{2} + 2\,\mathrm{Re}(z) + 8\,\mathrm{Im}(z) = -3.
\]
Often, one writes
\(
3\,\lvert z\rvert^{2} - 8\,\mathrm{Im}(z) + 2\,\mathrm{Re}(z) = -3
\)
depending on the chosen sign for $\mathrm{Im}(z)$.
\subsubsection*{Projections of Complex Numbers}
\noindent
\begin{enumerate}
    \item \begin{align}
        \text{Re}(z) = \frac{z+\overline{z} }{2}
    \end{align}
    \item \begin{align}
        \text{Im} (z)= \frac{z-\overline{z} }{2i}
    \end{align}
    \item \begin{align}
        \overline{i} = -i
    \end{align}
\end{enumerate}

\subsubsection*{hyperbolic trig}
\begin{align}
    \cosh^{2}x-\sinh^{2}x=1 \\[10pt] 
    \cosh z = \frac{1}{2}(e^{z}+e^{-z}) \\[10pt] 
    \sinh z = \frac{1}{2}(e^{z}-e^{-z})
\end{align}
\end{document}
