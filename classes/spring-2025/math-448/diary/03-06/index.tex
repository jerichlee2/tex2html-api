\documentclass[12pt]{article}

% Packages
\usepackage[margin=1in]{geometry}
\usepackage{amsmath,amssymb,amsthm}
\usepackage{enumitem}
\usepackage{hyperref}
\usepackage{xcolor}
\usepackage{import}
\usepackage{xifthen}
\usepackage{pdfpages}
\usepackage{transparent}
\usepackage{listings}


\lstset{
    breaklines=true,         % Enable line wrapping
    breakatwhitespace=false, % Wrap lines even if there's no whitespace
    basicstyle=\ttfamily,    % Use monospaced font
    frame=single,            % Add a frame around the code
    columns=fullflexible,    % Better handling of variable-width fonts
}

\newcommand{\incfig}[1]{%
    \def\svgwidth{\columnwidth}
    \import{./Figures/}{#1.pdf_tex}
}
\theoremstyle{definition} % This style uses normal (non-italicized) text
\newtheorem{solution}{Solution}
\newtheorem{proposition}{Proposition}
\newtheorem{problem}{Problem}
\newtheorem{lemma}{Lemma}
\newtheorem{theorem}{Theorem}
\newtheorem{remark}{Remark}
\newtheorem{note}{Note}
\newtheorem{definition}{Definition}
\newtheorem{example}{Example}
\theoremstyle{plain} % Restore the default style for other theorem environments
%

% Theorem-like environments
% Title information
\title{}
\author{Jerich Lee}
\date{\today}

\begin{document}

\maketitle

\textbf{Problem Statement:} Suppose that $f$ is an entire function and $\mathrm{Re}\bigl(f(z)\bigr) < c$ for all $z \in \mathbb{C}$, where $c$ is a real constant. Show that $f$ is constant. (\textit{Hint: Consider $\exp(f(z))$.})

\bigskip

\textbf{Solution:}

\begin{enumerate}
\item \textbf{Define a new function:} Let 
\[
g(z) = \exp\bigl(f(z)\bigr).
\]
Since $f(z)$ is entire, the composition $\exp(f(z))$ is also an entire function.

\item \textbf{Bound the magnitude of $g(z)$:} Recall that 
\[
\left| g(z) \right| = \left| \exp\bigl(f(z)\bigr) \right|
= \exp\bigl(\mathrm{Re}\bigl(f(z)\bigr)\bigr).
\]
By hypothesis, $\mathrm{Re}\bigl(f(z)\bigr) < c$ for all $z$. Hence,
\[
|g(z)| = \exp\bigl(\mathrm{Re}(f(z))\bigr) \le \exp(c).
\]
Thus $g(z)$ is bounded in the entire complex plane.

\item \textbf{Apply Liouville's Theorem:} Since $g(z)$ is entire and bounded, Liouville's Theorem implies that $g(z)$ must be constant. In other words, there exists some $K \in \mathbb{C}$ such that
\[
g(z) = \exp\bigl(f(z)\bigr) = K
\]
for all $z \in \mathbb{C}$.

\item \textbf{Conclude $f$ is constant:} From $\exp(f(z)) = K$, we can write
\[
f(z) = \ln(K).
\]
Because $K$ is a constant (does not depend on $z$), $\ln(K)$ is also a constant (assuming we choose a principal branch or any fixed branch of the logarithm). Hence $f(z)$ is constant throughout the complex plane.

\end{enumerate}

\textbf{Conclusion:} The assumption that $\mathrm{Re}(f(z))$ is bounded above by a constant $c$ forces $f$ to be constant, completing the proof.

\textbf{Problem Statement:}  
Suppose that $f$ is an entire function (analytic on all of $\mathbb{C}$) and there are positive constants $A$ and $m$ such that
\[
|f(z)| \le A\,|z|^m 
\quad
\text{for sufficiently large } |z|.
\]
Show that $f$ is a polynomial of degree at most $m$. 

\bigskip

\textbf{Solution:}

\begin{enumerate}
\item \textbf{Recall the Cauchy estimates for derivatives:}  
If $f$ is entire, then for any $r > 0$, the $n$th derivative of $f$ at $0$ is given by
\[
f^{(n)}(0)
= 
\frac{n!}{2\pi i} \int_{|z|=r} \frac{f(z)}{z^{n+1}}\, dz.
\]
A well-known consequence (the Cauchy estimate) is:
\[
|f^{(n)}(0)|
\le 
\frac{n!}{r^n} \,\max_{|z|=r} |f(z)|.
\]

\item \textbf{Apply the growth condition:}  
By hypothesis, for sufficiently large $r$, 
\[
\max_{|z|=r} |f(z)| \le A\,r^m.
\]
Hence, for $r$ large,
\[
|f^{(n)}(0)|
\;\le\; 
\frac{n!}{r^n}\, A\,r^m
\;=\;
A\,n!\,\frac{r^m}{r^n}
\;=\;
A\,n!\,r^{\,m-n}.
\]

\item \textbf{Consider $n > m$:}  
For $n > m$, we have $m - n < 0$. Let $r \to \infty$:
\[
\lim_{r \to \infty} A\,n!\,r^{\,m-n} 
\;=\;
A\,n!\,\lim_{r \to \infty} r^{m-n}
\;=\;
A\,n!\,\cdot 0
\;=\;
0.
\]
Thus,
\[
|f^{(n)}(0)| \le \lim_{r \to \infty} A\,n!\,r^{\,m-n} 
= 
0
\quad
\Longrightarrow
\quad
f^{(n)}(0) = 0
\quad
\text{for all } n > m.
\]

\item \textbf{Conclude $f$ is a polynomial:}  
The fact that $f^{(n)}(0) = 0$ for all $n > m$ implies that in the Taylor series expansion of $f$ around $0$,
\[
f(z) = \sum_{k=0}^{\infty} \frac{f^{(k)}(0)}{k!} \, z^k,
\]
all terms with $k > m$ must vanish. Therefore, $f(z)$ reduces to a finite sum,
\[
f(z) = \sum_{k=0}^{m} \frac{f^{(k)}(0)}{k!} \, z^k,
\]
which is a polynomial of degree at most $m$.

\end{enumerate}

\bigskip

\textbf{Conclusion:}  
Under the given growth condition $|f(z)| \le A\,|z|^m$ for large $|z|$, the function $f$ must be a polynomial of degree $\le m$. This follows directly from the Cauchy estimates and letting the radius of integration $r \to \infty$.
We want to expand $\sin(\pi z)$ about $z = \tfrac12$. Let
\[
  z \;=\; \tfrac12 + w,
\]
then
\[
  \sin(\pi z)
  \;=\;
  \sin\!\Bigl(\pi\bigl(\tfrac12 + w\bigr)\Bigr)
  \;=\;
  \sin\!\Bigl(\tfrac{\pi}{2} + \pi w\Bigr)
  \;=\;
  \cos(\pi w).
\]
Next, recall the Maclaurin series expansion for $\cos(\pi w)$:
\[
  \cos(\pi w)
  \;=\;
  1
  \;-\;
  \frac{(\pi w)^{2}}{2!}
  \;+\;
  \frac{(\pi w)^{4}}{4!}
  \;-\;
  \dots
\]
Substituting $w = z - \tfrac12$ back in, we get the Taylor expansion around $z = \tfrac12$:
\[
  \sin(\pi z)
  \;=\;
  1
  \;-\;
  \frac{\pi^2}{2!}\,\bigl(z - \tfrac12\bigr)^2
  \;+\;
  \frac{\pi^4}{4!}\,\bigl(z - \tfrac12\bigr)^4
  \;-\;
  \cdots
\]
When we wish to expand a function $f(z)$ about a point $z_0$, a standard strategy is to define a new variable $w$ such that
\[
   z \;=\; z_0 + w.
\]
This reformulates the expansion of $f(z)$ around $z=z_0$ as an expansion around $w=0$. 

In this particular case, we want to expand around $z_0 = \tfrac12$, so we set
\[
   z \;=\; \tfrac12 + w.
\]
Then, expanding around $z = \tfrac12$ is the same as expanding in powers of $w$ around $w = 0$.
$\sum_{i=0}^{\infty} (-1)^{n}\frac{(\pi z-\frac{1}{2})^{{2n+1}}}{(2n+1)!}$


We want to expand 
\[
   \sin(\pi z)
   \quad
   \text{around}
   \quad
   z = \tfrac12.
\]
Set
\[
   z 
   \;=\; 
   \tfrac12 + w
   \quad\Longrightarrow\quad
   \sin\bigl(\pi (\tfrac12 + w)\bigr)
   \;=\;
   \sin\Bigl(\tfrac{\pi}{2} + \pi w\Bigr)
   \;=\;
   \cos(\pi w).
\]
Next, we use the Maclaurin series for $\cos(\pi w)$ around $w=0$:
\[
   \cos(\pi w)
   \;=\;
   1
   \;-\;
   \frac{(\pi w)^2}{2!}
   \;+\;
   \frac{(\pi w)^4}{4!}
   \;-\;
   \cdots
\]
which is a series in \emph{even powers} of $w$.

\bigskip

\noindent
\textbf{Why not a series in odd powers?}

\smallskip

One might ask why we cannot write something like
\[
   \sum_{n=0}^{\infty} (-1)^n 
   \,\frac{\bigl(\pi z-\tfrac{1}{2}\bigr)^{2n+1}}{(2n+1)!}
   \quad
   \text{for}
   \quad
   \sin(\pi z).
\]
Once we have made the shift $z = \tfrac12 + w$, the function becomes $\cos(\pi w)$. Its Taylor expansion about $w=0$ is \emph{necessarily} an even-power series in $\pi w$. In other words, $\cos(x)$ always has the form
\[
   \cos(x) 
   \;=\; 
   1 
   \;-\; 
   \frac{x^2}{2!}
   \;+\; 
   \frac{x^4}{4!}
   \;-\; 
   \cdots,
\]
so $\cos(\pi w)$ cannot be captured by a series of purely odd powers.

Therefore, the series with only odd powers in $\bigl(\pi z - \tfrac{1}{2}\bigr)$ does not match the actual expansion of $\sin(\pi z)$ around $z = \tfrac12$. 
\textbf{Why introduce a new variable?}

When expanding a function $f(z)$ near a point $z_0$, you can write
\[
   f(z)
   \;=\;
   \sum_{n=0}^\infty
   \frac{f^{(n)}(z_0)}{n!}
   \,\bigl(z - z_0\bigr)^n.
\]
There is no strict need to introduce a new variable. However, it is often convenient to set
\[
   z
   \;=\;
   z_0 + w
   \quad
   \Longrightarrow
   \quad
   w
   \;=\;
   z - z_0,
\]
so that the expansion in powers of $\bigl(z - z_0\bigr)$ becomes a standard Maclaurin expansion in $w$.

\medskip

\noindent
\textbf{Example: Expanding $\sin(\pi z)$ around $z = \tfrac12$.}

We can write
\[
   z
   \;=\;
   \tfrac12 + w
   \quad\Longrightarrow\quad
   \sin(\pi z)
   \;=\;
   \sin\Bigl(\tfrac{\pi}{2} + \pi w\Bigr)
   \;=\;
   \cos(\pi w).
\]
Since the Maclaurin expansion for $\cos x$ is already known,
\[
   \cos x
   \;=\;
   1
   \;-\;
   \frac{x^2}{2!}
   \;+\;
   \frac{x^4}{4!}
   \;-\;\cdots,
\]
we substitute $x = \pi w$ to get
\[
   \cos(\pi w)
   \;=\;
   1
   \;-\;
   \frac{(\pi w)^2}{2!}
   \;+\;
   \frac{(\pi w)^4}{4!}
   \;-\;\cdots.
\]
Then reverting $w = z - \tfrac12$ yields
\[
   \sin(\pi z)
   \;=\;
   1
   \;-\;
   \frac{\pi^2}{2}
   \,\bigl(z - \tfrac12\bigr)^2
   \;+\;
   \frac{\pi^4}{4!}
   \,\bigl(z - \tfrac12\bigr)^4
   \;-\;\cdots.
\]

If you prefer, you could do everything directly in terms of $(z - \tfrac12)$. The introduction of $w$ just makes it transparent that we are applying a known zero-centered expansion for $\cos(\pi w)$.

\end{document}
