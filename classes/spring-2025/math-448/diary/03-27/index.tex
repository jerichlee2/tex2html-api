\documentclass[12pt]{article}

% Packages
\usepackage[margin=1in]{geometry}
\usepackage{amsmath,amssymb,amsthm}
\usepackage{enumitem}
\usepackage{hyperref}
\usepackage{xcolor}
\usepackage{import}
\usepackage{xifthen}
\usepackage{pdfpages}
\usepackage{transparent}
\usepackage{listings}
\DeclareMathOperator{\Log}{Log}
\DeclareMathOperator{\Arg}{Arg}

\lstset{
    breaklines=true,         % Enable line wrapping
    breakatwhitespace=false, % Wrap lines even if there's no whitespace
    basicstyle=\ttfamily,    % Use monospaced font
    frame=single,            % Add a frame around the code
    columns=fullflexible,    % Better handling of variable-width fonts
}

\newcommand{\incfig}[1]{%
    \def\svgwidth{\columnwidth}
    \import{./Figures/}{#1.pdf_tex}
}
\theoremstyle{definition} % This style uses normal (non-italicized) text
\newtheorem{solution}{Solution}
\newtheorem{proposition}{Proposition}
\newtheorem{problem}{Problem}
\newtheorem{lemma}{Lemma}
\newtheorem{theorem}{Theorem}
\newtheorem{remark}{Remark}
\newtheorem{note}{Note}
\newtheorem{definition}{Definition}
\newtheorem{example}{Example}
\newtheorem{corollary}{Corollary}
\theoremstyle{plain} % Restore the default style for other theorem environments
%

% Theorem-like environments
% Title information
\title{}
\author{Jerich Lee}
\date{\today}

\begin{document}

\maketitle
can you explain why 
\begin{align}
    \left\vert \int_{0}^{\pi}f(Re^{i \theta})Re^{i \theta}  \,\mathrm{d}\theta  \right\vert \leq R \int_{0}^{\pi}\frac{e^{-R\sin \theta}}{R^{2}-\alpha^{2}}  \,\mathrm{d}\theta 
\end{align} 

\section*{Explanation of the Inequality}
\[
\left\lvert \int_{0}^{\pi} f\bigl(Re^{i\theta}\bigr)\,Re^{i\theta}\,d\theta \right\rvert 
\;\;\le\;\;
R \int_{0}^{\pi} \frac{e^{-R\sin(\theta)}}{\,R^{2}-\alpha^{2}\,} \,d\theta.
\]

\subsection*{1. Parametrization and Factor of \(\lvert dz\rvert\)}
When integrating over the semicircle in the upper half-plane of radius $R$, we set 
\[
z(\theta) = Re^{i\theta},\quad \theta\in[0,\pi].
\]
Then 
\[
dz = iRe^{i\theta}\,d\theta,\quad \text{so } |dz| = R\,d\theta.
\]
Hence, inside the integral
\[
|f(z)\,dz| \le \bigl|f(Re^{i\theta})\bigr|\cdot R\,d\theta.
\]
Often we rewrite it as $f(Re^{i\theta})\,Re^{i\theta}\,d\theta$ in the integrand. Taking the absolute value yields a factor $R$ times $|f(Re^{i\theta})|$.

\subsection*{2. Exponential Factor and Decay}
If $f(z)$ involves $e^{iz}$ or $e^{iRe^{i\theta}}$, then
\[
\left\lvert e^{\,iRe^{i\theta}}\right\rvert 
= \exp\!\bigl(\Re\bigl(iRe^{i\theta}\bigr)\bigr)
= \exp\!\bigl(-\,R\,\sin(\theta)\bigr).
\]
This provides a decaying exponential factor $e^{-\,R\sin\theta}$ for $\theta\in(0,\pi)$.

\subsection*{3. Denominator Bound}
If the denominator is something like $(z^2+\alpha^2)$, then on the circle $|z|=R$,
\[
|z^2 + \alpha^2|
\;\ge\;
\bigl||z^2| - |\alpha^2|\bigr|
= \bigl|R^2 - \alpha^2\bigr|.
\]
Hence
\[
\left\lvert \frac{e^{\,iRe^{i\theta}}}{\,z^2 + \alpha^2}\right\rvert
\;\le\;
\frac{e^{-\,R\sin(\theta)}}{\,R^2 - \alpha^2\,}.
\]

\subsection*{4. Putting It All Together}
Combining these points:

\begin{itemize}
\item The factor $R$ from $|dz|$,
\item The exponential decay $e^{-R\sin(\theta)}$,
\item The denominator bound $R^2 - \alpha^2$,
\end{itemize}
we get:
\[
\bigl\lvert f(Re^{i\theta})\,Re^{i\theta}\bigr\rvert 
\;\le\; 
R\,\frac{e^{-R\sin(\theta)}}{\,R^2-\alpha^2\,}.
\]
Thus
\[
\left\lvert \int_{0}^{\pi} f\bigl(Re^{i\theta}\bigr)\,Re^{i\theta}\,d\theta\right\rvert
\;\le\;
\int_{0}^{\pi} \Bigl\lvert f(Re^{i\theta})\,Re^{i\theta}\Bigr\rvert d\theta
\;\le\;
R \int_{0}^{\pi} \frac{e^{-R\sin(\theta)}}{\,R^2 - \alpha^2\,}\,d\theta.
\]

\section*{Derivation of \(\lvert e^{\,i R e^{i\theta}} \rvert = e^{-\,R\sin(\theta)}\)}

\subsection*{Step 1: Express \(i R e^{i\theta}\) in Cartesian Form}

We have:
\[
i R e^{i\theta} 
= i R \bigl(\cos\theta + i\,\sin\theta\bigr)
= i R \cos\theta \;+\; i^2 R \sin\theta 
= i R \cos\theta \;-\; R \sin\theta.
\]
Hence, in terms of real and imaginary parts:
\[
i R e^{i\theta} 
= \underbrace{(- R \sin\theta)}_{\text{Real part}}
\;+\;
\underbrace{\bigl(i R \cos\theta\bigr)}_{\text{Imag.\ part}}.
\]

\subsection*{Step 2: Exponential of a Complex Number}

Recall that for any complex number \(z = x + i y\),
\[
\bigl\lvert e^z \bigr\rvert 
= e^{\Re(z)}.
\]
Here, \(\Re(z)\) is the real part of \(z\). In our problem, let
\[
z = i R e^{i\theta} = - R \sin\theta + i\bigl(R\cos\theta\bigr).
\]
Hence, \(\Re\bigl(i R e^{i\theta}\bigr) = -R \sin\theta\).

\subsection*{Step 3: Combine the Results}

Therefore,
\[
\left\lvert e^{\,i R e^{i\theta}} \right\rvert 
= \exp\!\Bigl(\Re\bigl(i R e^{i\theta}\bigr)\Bigr)
= \exp\bigl(-R \sin\theta\bigr).
\]

\subsection*{Conclusion}

Hence we have shown that
\[
\boxed{
\bigl\lvert e^{\,i R e^{i\theta}}\bigr\rvert 
= e^{-\,R\,\sin(\theta)},
\quad \theta\in(0,\pi).
}
\]

\section*{Computing \(\mathrm{Re}\bigl(i R e^{i\theta}\bigr)\) via \(\frac{z + \overline{z}}{2}\)}

\subsection*{1. Write \(z = i R e^{i\theta}\) in standard form}

First, note that
\[
e^{i\theta} = \cos\theta + i\sin\theta.
\]
Hence,
\[
z = i R e^{i\theta}
= i R (\cos\theta + i\sin\theta)
= iR\cos\theta + i^2 R\sin\theta
= iR\cos\theta \;-\; R\sin\theta.
\]
Therefore, in standard ``\(x + i\,y\)'' form,
\[
z = -\,R\sin\theta \;+\; i\bigl(R\cos\theta\bigr).
\]

\subsection*{2. Find the Complex Conjugate \(\overline{z}\)}

The conjugate is obtained by negating the imaginary part:
\[
\overline{z} 
= -\,R\sin\theta \;-\; i\bigl(R\cos\theta\bigr).
\]

\subsection*{3. Apply the Formula \(\mathrm{Re}(z) = \frac{z + \overline{z}}{2}\)}

\[
z + \overline{z}
= \Bigl[-\,R\sin\theta + i(R\cos\theta)\Bigr]
 + \Bigl[-\,R\sin\theta - i(R\cos\theta)\Bigr]
= -\,2\,R\,\sin\theta.
\]
Divide by 2:
\[
\mathrm{Re}(z)
= \frac{z + \overline{z}}{2}
= \frac{-2R\sin\theta}{2}
= -\,R\,\sin\theta.
\]

\subsection*{4. Conclusion}

Thus, by using the real-part formula,
\[
\boxed{
\mathrm{Re}\bigl(iR e^{i\theta}\bigr)
= -\,R\,\sin(\theta).
}
\]


\section*{Why the Integral is Bounded by \(\frac{\pi R}{R^2-\alpha^2}\) and Tends to 0}

In many contour‐integration arguments (Jordan's lemma–type estimates), one examines
\[
\left\lvert \int_{0}^{\pi} f\bigl(Re^{i\theta}\bigr)\,Re^{i\theta}\,d\theta \right\rvert 
\;\;\le\;\;
R \int_{0}^{\pi} \frac{e^{-R\sin(\theta)}}{\,R^2 - \alpha^2\,}\,d\theta,
\]
and then shows this goes to zero as \(R \to \infty\).  Here is why:

\subsection*{1. Factor \(R\) from the Parametrization}

If \(z=Re^{i\theta}\), then \(dz = iRe^{i\theta} d\theta\).  In magnitude,
\[
|dz| = R\,d\theta.
\]
Hence there is a factor of \(R\) in front when bounding \(\bigl|f(Re^{i\theta})\,dz\bigr|\).

\subsection*{2. Exponential Decay: \(e^{-R\sin\theta}\le 1\)}

On the upper semicircle, $0 \le \theta \le \pi$, we have $\sin\theta \ge 0$, so
\[
e^{-R\sin\theta} \;\le\; 1.
\]
Thus,
\[
\frac{e^{-R\sin\theta}}{R^2 - \alpha^2} 
\;\le\; \frac{1}{R^2 - \alpha^2}.
\]

\subsection*{3. Integral Bound over $0 \le \theta \le \pi$}

Therefore,
\[
R \int_{0}^{\pi} \frac{e^{-R\sin(\theta)}}{\,R^2 - \alpha^2\,}\,d\theta
\;\;\le\;\;
R \int_{0}^{\pi} \frac{1}{\,R^2 - \alpha^2\,}\,d\theta
= \frac{R}{R^2 - \alpha^2} \,\times\, \pi.
\]
Hence
\[
\Bigl\lvert \int_{0}^{\pi} f(Re^{i\theta})\,Re^{i\theta}\,d\theta \Bigr\rvert
\;\le\;
\frac{\pi R}{\,R^2 - \alpha^2\,}.
\]

\subsection*{4. Limit as $R \to \infty$}

As $R\to\infty$, the ratio $\frac{R}{R^2 - \alpha^2}$ behaves like $\frac{R}{R^2}=\frac{1}{R}$, so
\[
\frac{\pi R}{\,R^2 - \alpha^2\,}
\;\longrightarrow\; 0.
\]
Hence the integral goes to zero.

\subsection*{Conclusion}

The steps are:
\begin{itemize}
\item Factor out $R$ from $|dz|$,
\item Bound $e^{-R\sin\theta}$ by 1 (since $\sin\theta\ge 0$),
\item Integrate a constant $\frac{R}{R^2-\alpha^2}$ over $\theta\in[0,\pi]$,
\item Observe it vanishes like $\frac{1}{R}$ as $R\to\infty$.
\end{itemize}
Therefore the entire integral tends to $0$.
how do we know $\sin \theta \geq (\frac{2}{\pi})\theta$ for $0\leq \theta \leq \frac{\pi}{2}$  
\end{document}
