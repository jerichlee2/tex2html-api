\documentclass[12pt]{article}

% Packages
\usepackage[margin=1in]{geometry}
\usepackage{amsmath,amssymb,amsthm}
\usepackage{enumitem}
\usepackage{hyperref}
\usepackage{xcolor}
\usepackage{import}
\usepackage{xifthen}
\usepackage{pdfpages}
\usepackage{transparent}
\usepackage{listings}
\usepackage{tikz}
\usepackage{physics}
\usepackage{siunitx}
  \usetikzlibrary{calc,patterns,arrows.meta,decorations.markings}


\DeclareMathOperator{\Log}{Log}
\DeclareMathOperator{\Arg}{Arg}

\lstset{
    breaklines=true,         % Enable line wrapping
    breakatwhitespace=false, % Wrap lines even if there's no whitespace
    basicstyle=\ttfamily,    % Use monospaced font
    frame=single,            % Add a frame around the code
    columns=fullflexible,    % Better handling of variable-width fonts
}

\newcommand{\incfig}[1]{%
    \def\svgwidth{\columnwidth}
    \import{./Figures/}{#1.pdf_tex}
}
\theoremstyle{definition} % This style uses normal (non-italicized) text
\newtheorem{solution}{Solution}
\newtheorem{proposition}{Proposition}
\newtheorem{problem}{Problem}
\newtheorem{lemma}{Lemma}
\newtheorem{theorem}{Theorem}
\newtheorem{remark}{Remark}
\newtheorem{note}{Note}
\newtheorem{definition}{Definition}
\newtheorem{example}{Example}
\newtheorem{corollary}{Corollary}
\theoremstyle{plain} % Restore the default style for other theorem environments
%

% Theorem-like environments
% Title information
\title{}
\author{Jerich Lee}
\date{\today}

\begin{document}

\maketitle

%------------------------------------------------------------------------------
%  Inversion  \(T(z)=\dfrac{1}{z}\) sends \emph{generalised circles}
%  (i.e.\ Euclidean circles or straight lines) to generalised circles.
%  Below is a concise statement, a coordinate–free proof, and a
%  coordinate proof that makes the geometry completely explicit.
%------------------------------------------------------------------------------

\begin{theorem}[Inversion preserves circles and lines]
  Let
  \[
     T : \widehat{\mathbb C}\longrightarrow\widehat{\mathbb C},
     \qquad T(z)=\frac{1}{z},
  \]
  where \(\widehat{\mathbb C}=\mathbb C\cup\{\infty\}\) is the Riemann
  sphere and \(T(0)=\infty,\; T(\infty)=0\).
  For any generalised circle  
  \(C\subset\widehat{\mathbb C}\) (that is, a circle or a straight line,
  with the understanding that a line “passes through” \(\infty\)),
  its image \(T(C)\) is again a generalised circle.
  \end{theorem}
  
  \subsection*{1. Conceptual proof (cross–ratio)}
  Any Möbius (linear–fractional) map
  \(M(z)=\dfrac{az+b}{cz+d}\;(ad-bc\neq0)\) is completely determined by
  the image of three distinct points and \emph{preserves the cross–ratio}
  \((z_1,z_2,z_3,z_4)\).  A generalised circle is precisely the set of
  points \(z_4\) whose cross–ratio with three fixed, distinct
  points \(z_1,z_2,z_3\) is real.  
  Because invariance of the cross–ratio implies
  \[
     (z_1,z_2,z_3,z_4)\in\mathbb R 
     \;\Longleftrightarrow\;
     (M(z_1),M(z_2),M(z_3),M(z_4))\in\mathbb R,
  \]
  Möbius maps carry generalised circles to generalised circles.
  The inversion \(T(z)=1/z\) is a special Möbius map
  \(\bigl(\begin{smallmatrix}0&1\\ 1&0\end{smallmatrix}\bigr)\), hence
  inherits this property.
  \qed
  
  \subsection*{2. Coordinate proof}
  Write any generalised circle as
  \[
     \alpha\,z\bar z+\beta\,z+\overline{\beta}\,\bar z+\gamma=0,
     \qquad \alpha,\gamma\in\mathbb R,\;
     (\alpha,\beta,\gamma)\neq(0,0,0).
     \tag{$\star$}
  \]
  \begin{itemize}
  \item A \emph{circle} is the case \(\alpha\neq0\);
  \item A \emph{line} is the case \(\alpha=0\).
  \end{itemize}
  
  Under \(w=T(z)=1/z\) we have \(z=\dfrac{1}{w}\) and
  \(z\bar z=\dfrac{1}{w\bar w}\).
  Substituting \(z=1/w\) in \((\star)\) gives
  \[
     \alpha\frac{1}{w\bar w}
     +\beta\frac{1}{w}
     +\overline{\beta}\frac{1}{\bar w}
     +\gamma =0.
  \]
  Multiplying by \(w\bar w\) (which is non–zero except at \(w=0\), where
  we interpret the equation in the extended plane) yields
  \[
     \gamma\,w\bar w
     +\overline{\beta}\,w
     +\beta\,\bar w
     +\alpha =0.
     \tag{$\star\star$}
  \]
  Equation \((\star\star)\) is of the same algebraic form as
  \((\star)\), so the image set \(T(C)\) is a generalised circle.
  
  \medskip
  \noindent
  \textbf{Classification of special cases.}
  Let \(L\) be a straight line and \(S\) a true circle.
  \[
  \begin{array}{c|c}
  \text{Pre–image set}  & \text{Image under }T(z)=\dfrac{1}{z} \\ \hline
  \text{Line through }0 & \text{Same line (fixed)} \\[3pt]
  \text{Line not through }0 & \text{Circle through }0 \\[6pt]
  \text{Circle through }0 & \text{Line not through }0 \\[6pt]
  \text{Circle not through }0 & \text{Circle not through }0
  \end{array}
  \]
  
  \smallskip
  \emph{Example.}
  The vertical line \(\Re z=2\) (does not meet \(0\)) has parametrisation
  \(z=2+it\;(t\in\mathbb R)\).  Then
  \[
     w=T(z)=\frac{1}{2+it}
       =\frac{2-it}{4+t^{2}}
       =\frac{2}{4+t^{2}}-\;i\frac{t}{4+t^{2}},
  \]
  and one checks directly that
  \(
     \bigl|\,w-\tfrac14\,\bigr|=\tfrac14,
  \)
  so the image is the circle centred at \(1/4\) with radius \(1/4\) that
  passes through \(0\).
  
  \subsection*{3. Geometric interpretation}
  Algebraically
  \(
     T(z)=1/z
     =\overline z/|z|^{2}
  \)
  is the composition of
  \begin{enumerate}
  \item inversion in the unit circle \(I(z)=z/|z|^{2}\), and
  \item reflection across the real axis \(R(z)=\bar z\).
  \end{enumerate}
  Both \(I\) and \(R\) individually send circles/lines to
  circles/lines; hence so does \(T=R\circ I\).
  Intuitively, inversion swaps the inside and outside of the unit circle,
  while reflection flips each point across the real axis, preserving the
  circle/line structure throughout.
  
  \bigskip
  \centerline{$\boxed{\;T(z)=1/z\text{ maps generalised circles to generalised circles.}\;}$}
  %------------------------------------------------------------------------------
%  Why does the horizontal line  $\ell_2=\{z\in\Bbb C:\operatorname{Im}z=1\}$
%  go to the circle  $\{w:|w+\tfrac{i}{2}|=\tfrac12\}$ under the inversion
%  $T(z)=\dfrac1z$\,?
%------------------------------------------------------------------------------

\begin{proof}
  Write a point of the line \(\ell_2\) as
  \[
     z=x+i, \qquad x\in\Bbb R.
  \]
  Applying the map \(T(z)=\dfrac1z\) gives  
  \[
     w=T(z)=\frac1{x+i}
          =\frac{x-i}{x^{2}+1}
          =\frac{x}{x^{2}+1}\;-\;i\,\frac1{x^{2}+1}.
  \]
  
  \medskip
  \noindent
  \textbf{Step 1.  Show that \(w\) satisfies the claimed circle equation.}
  
  Add \(\dfrac{i}{2}\) to \(w\):
  \[
     w+\frac{i}{2}
     =\frac{x}{x^{2}+1}
      +i\!\left(\frac12-\frac1{x^{2}+1}\right).
  \]
  Set \(D:=x^{2}+1>0\) for brevity.  Then
  \[
     |\,w+\tfrac{i}{2}\,|^{2}
     =\left(\frac{x}{D}\right)^{2}
      +\left(\frac12-\frac1{D}\right)^{2}.
  \]
  Since \(x^{2}=D-1\),
  \[
     \left(\frac{x}{D}\right)^{2}
     =\frac{D-1}{D^{2}}
     =\frac1{D}-\frac1{D^{2}}.
  \]
  Hence
  \[
     |\,w+\tfrac{i}{2}\,|^{2}
     =\Bigl(\frac1{D}-\frac1{D^{2}}\Bigr)
      +\Bigl(\frac14-\frac1{D}+\frac1{D^{2}}\Bigr)
     =\frac14.
  \]
  Therefore \(|\,w+\tfrac{i}{2}\,|=\tfrac12\), so every image point
  indeed lies on the circle
  \(C:=\{\,w\in\Bbb C:|\,w+\tfrac{i}{2}\,|=\tfrac12\}\).
  
  \medskip
  \noindent
  \textbf{Step 2.  The circle \(C\) contains \(0\).}
  
  Putting \(w=0\) gives \(|\,0+\tfrac{i}{2}\,|=\tfrac12\), so
  \(0\in C\); this fits the general rule that a line not passing through
  \(0\) is sent to a circle \emph{through} \(0\).
  
  \medskip
  \noindent
  \textbf{Step 3.  No other points are missed.}
  
  Conversely, let \(w\in C\setminus\{0\}\).
  Then \(z:=\dfrac1w\) is well-defined and
  \(
     \operatorname{Im}z
     =\operatorname{Im}\!\bigl(\tfrac1w\bigr)
     =\operatorname{Im}\!\bigl(\overline{w}/|w|^{2}\bigr)
     =\frac{-\operatorname{Im}w}{|w|^{2}}.
  \)
  The condition \(|\,w+\tfrac{i}{2}\,|=\tfrac12\) forces
  \(\operatorname{Im}w=-\dfrac12\) and \(|w|^{2}=\dfrac14\),
  so \(\operatorname{Im}z=1\); hence \(z\in\ell_2\).
  Thus \(T(\ell_2)=C\).
  
  \medskip
  \[
     \boxed{%
        T\bigl(\,\operatorname{Im}z=1\,\bigr)
        \;=\;
        \Bigl\{\,w\in\Bbb C : \bigl|\,w+\tfrac{i}{2}\,\bigr|=\tfrac12\Bigr\}.}
  \]
  
  \end{proof}

  find all the order of the zeros on the upper half plane Im(z)>0 of:

  $z^{8}+4z^{5}+iz+i+1$ 
\end{document}
