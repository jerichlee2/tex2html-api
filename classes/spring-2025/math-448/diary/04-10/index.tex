\documentclass[12pt]{article}

% Packages
\usepackage[margin=1in]{geometry}
\usepackage{amsmath,amssymb,amsthm}
\usepackage{enumitem}
\usepackage{hyperref}
\usepackage{xcolor}
\usepackage{import}
\usepackage{xifthen}
\usepackage{pdfpages}
\usepackage{transparent}
\usepackage{listings}
\usepackage{tikz}

\DeclareMathOperator{\Log}{Log}
\DeclareMathOperator{\Arg}{Arg}

\lstset{
    breaklines=true,         % Enable line wrapping
    breakatwhitespace=false, % Wrap lines even if there's no whitespace
    basicstyle=\ttfamily,    % Use monospaced font
    frame=single,            % Add a frame around the code
    columns=fullflexible,    % Better handling of variable-width fonts
}

\newcommand{\incfig}[1]{%
    \def\svgwidth{\columnwidth}
    \import{./Figures/}{#1.pdf_tex}
}
\theoremstyle{definition} % This style uses normal (non-italicized) text
\newtheorem{solution}{Solution}
\newtheorem{proposition}{Proposition}
\newtheorem{problem}{Problem}
\newtheorem{lemma}{Lemma}
\newtheorem{theorem}{Theorem}
\newtheorem{remark}{Remark}
\newtheorem{note}{Note}
\newtheorem{definition}{Definition}
\newtheorem{example}{Example}
\newtheorem{corollary}{Corollary}
\theoremstyle{plain} % Restore the default style for other theorem environments
%

% Theorem-like environments
% Title information
\title{}
\author{Jerich Lee}
\date{\today}

\begin{document}

\maketitle
\begin{problem}
    \begin{align}
        f(z) &= z^{4}-3z^{2}+3
    \end{align}

    \begin{enumerate}

        \item \textbf{Contour and strategy (recap).}  
              Take the quarter‑circle
              $$
                  C \;=\; [0,R]\;\cup\;[iR,0]\;\cup\;C_R,
              $$
              where $C_R=\{Re^{i\theta}\mid0\le\theta\le\pi/2\}$.  
              We pick $R>\sqrt6$ so large that $f$ has no zeros on $C$.  
              By the Argument Principle, the number $N_{Q_1}$ of zeros in the first quadrant equals
              $$
                  N_{Q_1}\;=\;\frac1{2\pi}\,\Delta_C\arg f(z).
              $$
              We therefore compute the change in $\arg f$ along each piece of $C$.

        \item \textbf{(a) Positive real axis $\,[0,R]$.}

              \begin{enumerate}
                  \item[$\triangleright$] \emph{Rewrite $f$ as a real function.}  
                        Put $z=x>0$ and $t=x^{2}$.  Then
                        $$
                            f(x)=x^{4}-3x^{2}+3 \;=\;t^{2}-3t+3.
                        $$
                  \item[$\triangleright$] \emph{Show $f(x)>0$.}  
                        The quadratic $t^{2}-3t+3$ has discriminant
                        $$
                            \Delta=(-3)^{2}-4\cdot1\cdot3 = 9-12 = -3<0,
                        $$
                        so it has no real roots.  Because the leading coefficient is $+1$, the parabola opens upward; hence $t^{2}-3t+3>0$ for all $t\in\mathbb R$, and in particular $f(x)>0$ for $x\in[0,R]$.
                  \item[$\triangleright$] \emph{Consequences for the argument.}  
                        A positive real number has argument $0$.  Since $f(x)$ never crosses the negative real axis or $0$, the argument of $f$ stays constant along $[0,R]$.  
                        $$\Delta_{[0,R]}\arg f = 0.$$
              \end{enumerate}

        \item \textbf{(b) Positive imaginary axis $\,[iR,0]$.}

              \begin{enumerate}
                  \item[$\triangleright$] \emph{Rewrite $f$ as a real function of $y$.}  
                        Put $z=iy$ with $y>0$.  Then
                        $$
                            f(iy) = (iy)^{4}-3(iy)^{2}+3 = y^{4}+3y^{2}+3.
                        $$
                  \item[$\triangleright$] \emph{Show $f(iy)>0$.}  
                        Each term $y^{4},\,3y^{2},\,3$ is non‑negative and at least one is strictly positive, so $f(iy)>0$ for all $y>0$.
                  \item[$\triangleright$] \emph{Consequences for the argument.}  
                        Again $f(iy)$ stays on the positive real axis, so the argument is constant:
                        $$\Delta_{[iR,0]}\arg f = 0.$$
              \end{enumerate}

        \item \textbf{(c) Quarter‑circle arc $C_R$.}

              \begin{enumerate}
                  \item[$\triangleright$] \emph{Factor $f$ to separate a known winding.}  
                        Write
                        $$
                            f(z) = z^{4}\,g(z), \qquad
                            g(z)=1-\frac{3}{z^{2}}+\frac{3}{z^{4}}.
                        $$
                        Along $C_R$ we have $|z|=R$, so
                        $$
                            |g(z)-1|
                            \;\le\;
                            \frac{3}{R^{2}}+\frac{3}{R^{4}}
                            \;<\;
                            \frac12
                            \quad\text{for }R>\sqrt6.
                        $$
                        Thus $g(C_R)$ lies inside the open disc of radius $\tfrac12$ centred at $1$, which does not encircle the origin.
                  \item[$\triangleright$] \emph{Compute the argument change of each factor.}  
                        \begin{align}
                            \Delta_{C_R}\arg z^{4} &= 4\;\Delta_{C_R}\arg z
                                                    = 4\Bigl(\frac{\pi}{2}\Bigr)
                                                    = 2\pi,\\
                            |\Delta_{C_R}\arg g(z)| &< \frac{\pi}{6}
                            \quad\text{(no winding around $0$).}
                        \end{align}
                        Hence
                        $$
                            \Delta_{C_R}\arg f(z)
                            = 2\pi + \varepsilon,
                            \qquad |\varepsilon|<\frac{\pi}{6}.
                        $$
              \end{enumerate}

        \item \textbf{Putting the pieces together.}  
              Summing:
              $$
                  \Delta_C\arg f(z)
                  = 0 + 0 + (2\pi+\varepsilon)
                  = 2\pi+\varepsilon,
                  \qquad |\varepsilon|<\frac{\pi}{6}.
              $$
              Therefore
              $$
                  N_{Q_1}
                  = \frac{1}{2\pi}\,\Delta_C\arg f(z)
                  \in \Bigl(\tfrac56,\tfrac76\Bigr)\cap\mathbb Z
                  = \{1\}.
              $$
              So $f$ possesses \textbf{exactly one zero in the first quadrant}.
    \end{enumerate}
\end{problem}
\begin{problem}
    \begin{align}
        f(z)&=z^{4}-3z^{2}+3
    \end{align}

    \begin{enumerate}
        \item \textbf{Arc parameterisation.}  
              Along $C_{R}$ we write
              \[
                  z = R e^{i\theta},
                  \qquad 0\le\theta\le\frac{\pi}{2}.
              \]
              Hence
              \[
                  \arg z = \theta
                  \quad\Longrightarrow\quad
                  \Delta_{C_R}\arg z
                  = \theta\Big|_{0}^{\pi/2}
                  = \frac{\pi}{2}.
              \]

        \item \textbf{Argument change of the first factor \(z^{4}\).}

              \begin{enumerate}
                  \item[$\triangleright$]  
                        Because \(\arg z=\theta\), we have
                        \(\arg z^{4}=4\theta\).
                  \item[$\triangleright$]  
                        As \(\theta\) runs from \(0\) to \(\pi/2\),
                        \[
                            \Delta_{C_R}\arg z^{4}
                            = 4\theta\Big|_{0}^{\pi/2}
                            = 4\left(\frac{\pi}{2}\right)
                            = 2\pi.
                        \]
                        Intuitively, \(z^{4}\) winds once counter‑clockwise
                        around the origin while we traverse the quarter‑circle.
              \end{enumerate}

        \item \textbf{Argument change of the perturbation factor \(g(z)\).}

              \begin{enumerate}
                  \item[$\triangleright$]  
                        Along \(C_{R}\) we have \(|z|=R\), so
                        \[
                            |g(z)-1|
                            = \left|-\frac{3}{z^{2}}+\frac{3}{z^{4}}\right|
                            \le \frac{3}{R^{2}}+\frac{3}{R^{4}}
                            =: \rho(R).
                        \]
                  \item[$\triangleright$]  
                        Choose \(R>\sqrt6\).  Then
                        \(\rho(R)<\dfrac12\), so the entire image \(g(C_{R})\)
                        is contained in the open disc
                        \[
                            D\Bigl(1,\tfrac12\Bigr)
                            =\{w\in\mathbb C : |w-1|<\tfrac12\},
                        \]
                        which lies strictly to the right of the imaginary axis
                        and does **not** enclose the origin.
                  \item[$\triangleright$]  
                        Because \(g(z)\) never crosses the negative real axis
                        (indeed, \(\Re g(z) > \tfrac12\) on \(C_{R}\)),
                        its argument can vary only within the angle subtended by
                        that disc:
                        \[
                            |\Delta_{C_R}\arg g(z)|
                            < \arcsin\!\Bigl(\tfrac12\Bigr)
                            = \frac{\pi}{6}.
                        \]
                        Any tighter bound (e.g.\ \(\pi/10\)) is also valid; we
                        merely need “\emph{small enough}’’ so that the sum with
                        \(2\pi\) still sits between \((5/6)\pi\) and \((7/6)\pi\).
              \end{enumerate}

        \item \textbf{Combine the two contributions.}
              \[
                  \Delta_{C_R}\arg f(z)
                  = \underbrace{2\pi}_{\text{from }z^{4}}
                    \;+\;
                    \underbrace{\varepsilon}_{|\varepsilon|<\pi/6\text{ from }g}
                  = 2\pi+\varepsilon,
                  \qquad |\varepsilon|<\frac{\pi}{6}.
              \]
              This completes the detailed justification of the arc
              calculation used in the earlier proof.
    \end{enumerate}
\end{problem}

\begin{problem}
    \begin{align}
        g(z) &= z^{3}-3z+1
    \end{align}

    \begin{enumerate}

        \item \textbf{Goal.}  
              Determine how many zeros of $g$ lie in the annulus  
              $$1<|z|<2.$$
              The plan is to use \emph{Rouché’s theorem} on the two circles  
              $|z|=1$ and $|z|=2$ so that we can compare $g$ with simpler
              polynomials whose zeros are already known.

        \item \textbf{Inside the unit disc $|z|<1$.}

              \begin{enumerate}
                  \item[$\triangleright$] \emph{Comparison function.}  
                        Take
                        $$f_{1}(z)=3z.$$
                        On the circle $|z|=1$ we have $|f_{1}(z)|=3$.

                  \item[$\triangleright$] \emph{Estimate the perturbation.}  
                        Write $g$ as $g=f_{1}+h_{1}$ where
                        $$h_{1}(z)=z^{3}+1.$$
                        Then on $|z|=1$
                        $$|h_{1}(z)|\le|z|^{3}+1 = 1+1 = 2.$$

                  \item[$\triangleright$] \emph{Rouché’s inequality.}  
                        Because $|h_{1}(z)|<|f_{1}(z)|$ on $|z|=1$,  
                        Rouché’s theorem says $f_{1}$ and $g$ have the
                        \emph{same number of zeros (counting multiplicity)}
                        inside $|z|<1$.

                  \item[$\triangleright$] \emph{Count the zeros of $f_{1}$.}  
                        $f_{1}(z)=3z$ has exactly \textbf{one} zero, namely $z=0$,
                        of multiplicity 1.

                  \item[$\triangleright$] \emph{Conclusion for $|z|<1$.}  
                        Hence $g$ has precisely  
                        $$N_{g}\bigl(|z|<1\bigr)=1$$
                        zero in the open unit disc.  
                        (The strict inequality $|h_{1}|<|f_{1}|$ also guarantees
                        $g$ has \emph{no} zeros on the circle $|z|=1$.)
              \end{enumerate}

        \item \textbf{Inside the disc $|z|<2$.}

              \begin{enumerate}
                  \item[$\triangleright$] \emph{Comparison function.}  
                        Now set
                        $$f_{2}(z)=z^{3},\qquad |f_{2}(z)|=|z|^{3}=8
                                                     \quad\text{on }|z|=2.$$

                  \item[$\triangleright$] \emph{Estimate the perturbation.}  
                        Express $g$ as $g=f_{2}-h_{2}$ where
                        $$h_{2}(z)=3z-1.$$
                        On $|z|=2$
                        $$|h_{2}(z)|\le 3|z|+1 = 3\cdot2+1 = 7.$$

                  \item[$\triangleright$] \emph{Rouché’s inequality.}  
                        Because $|h_{2}(z)|<|f_{2}(z)|$ on $|z|=2$,  
                        Rouché’s theorem gives that $f_{2}$ and $g$ share the
                        same number of zeros in $|z|<2$.

                  \item[$\triangleright$] \emph{Count the zeros of $f_{2}$.}  
                        $f_{2}(z)=z^{3}$ has \textbf{three} zeros at $z=0$
                        (multiplicity 3).

                  \item[$\triangleright$] \emph{Conclusion for $|z|<2$.}  
                        Thus
                        $$N_{g}\bigl(|z|<2\bigr)=3,$$
                        and again the strict inequality ensures $g$ has no
                        zeros \emph{on} the circle $|z|=2$.
              \end{enumerate}

        \item \textbf{Zeros in the annulus $1<|z|<2$.}  
              Subtracting the two counts,
              \[
                  N_{g}\bigl(1<|z|<2\bigr)
                  = N_{g}\bigl(|z|<2\bigr) - N_{g}\bigl(|z|<1\bigr)
                  = 3-1 = 2.
              \]
              Therefore $g(z)=z^{3}-3z+1$ has
              \[
                  \boxed{\text{exactly two zeros in }1<|z|<2.}
              \]
    \end{enumerate}
\end{problem}

\begin{theorem}[Rouch\'e]
    \begin{align}
        \text{Let }&C\text{ be a positively oriented, simple, closed contour} \\
        &\text{and let }f,g\text{ be analytic on and inside }C. \\
        &\text{If }|g(z)|<|f(z)|\quad\text{for every }z\in C, \\
        &\text{then }f\text{ and }f+g\text{ have the same number of zeros} \\
        &\text{inside }C,\text{ counted with multiplicity.}
    \end{align}

    \begin{enumerate}
        \item Equivalently, writing $h=f+g$, the hypothesis can be phrased as
              $|h(z)-f(z)|<|f(z)|$ on $C$.
        \item The conclusion applies to poles as well if one first multiplies
              by a suitable analytic factor that clears the poles; in most
              practical uses, however, $f$ and $g$ are polynomials or entire
              functions, so no poles are present.
    \end{enumerate}
\end{theorem}

\begin{problem}
    \begin{align}
        g(z) &= z^{3}-3z+1
    \end{align}

    \begin{enumerate}

        \item \textbf{Why compare with $f_{1}(z)=3z$ on $|z|=1$?}

              \begin{enumerate}
                  \item[$\triangleright$] On the unit circle we have
                        $|3z|=3,\;|z^{3}|=1,\;|1|=1$.
                  \item[$\triangleright$] Thus the linear term \emph{dominates}:
                        \[
                            |g(z)-3z|
                            =|z^{3}+1|
                            \le 1+1=2 < 3 = |3z|.
                        \]
                  \item[$\triangleright$] Choosing $f_{1}(z)=3z$ therefore
                        satisfies Rouché’s inequality
                        $|g-f_{1}|<|f_{1}|$ on $|z|=1$, while $f_{1}$ has an
                        easy zero count (exactly one, at $z=0$).
              \end{enumerate}

        \item \textbf{Why compare with $f_{2}(z)=z^{3}$ on $|z|=2$?}

              \begin{enumerate}
                  \item[$\triangleright$] On the circle $|z|=2$ we find
                        $|z^{3}|=8,\;|3z|=6,\;|1|=1$.
                  \item[$\triangleright$] The cubic term now dominates:
                        \[
                            |g(z)-z^{3}|
                            =|{-}\,3z+1|
                            \le 6+1 = 7 < 8 = |z^{3}|.
                        \]
                  \item[$\triangleright$] Hence $f_{2}(z)=z^{3}$ fulfils
                        $|g-f_{2}|<|f_{2}|$ on $|z|=2$, and $f_{2}$’s zero
                        count is also trivial (three at the origin).
              \end{enumerate}

        \item \textbf{General heuristic.}
              For a polynomial $p(z)=\sum_{k=0}^{n}a_{k}z^{k}$ and a fixed
              radius $r$, the term with the largest value of $|a_{k}|\,r^{k}$
              usually controls the size of $p$ on $|z|=r$.  Picking that term
              as $f$ often makes Rouché’s inequality automatic and leaves us
              with a comparison function whose zeros are straightforward to
              enumerate.
    \end{enumerate}
\end{problem}
\begin{problem}
    \begin{align}
        f(z) &= 2z^{4}-2iz^{3}+z^{2}+2iz-1
    \end{align}

    \begin{enumerate}
    \item \textbf{Objective.}  
          Count the zeros of $f$ in the open upper half–plane  
          $$\Im z>0.$$
          We use the Argument Principle with the contour
          \[
              C \;=\;[-R,R]\;\cup\;C_{R},
              \qquad
              C_{R}=\{Re^{i\theta}\mid0\le\theta\le\pi\},
          \]
          oriented counter‑clockwise and with $R\gg1$ so that $f$ has no zeros
          on $C$.

    \item \textbf{Contribution of the semicircle $C_{R}$.}

          \begin{enumerate}
          \item[$\triangleright$] \emph{Dominant term.}  
                Factor
                \[
                    f(z)=2z^{4}\,g(z),
                    \quad
                    g(z)=1-\frac{i}{z}+\frac{1}{2z^{2}}+\frac{i}{z^{3}}
                          -\frac{1}{2z^{4}}.
                \]
          \item[$\triangleright$] \emph{Bounding $g(z)$.}  
                On $C_{R}$ we have $|z|=R$, so
                \(
                    |g(z)-1|\le\frac1R+\frac{1}{2R^{2}}+\frac1{R^{3}}
                             +\frac{1}{2R^{4}}
                           <\frac12
                \)
                when $R>2$.  Thus $g(C_{R})$ stays in the disc
                $D(1,\tfrac12)$, which never encircles $0$, and
                $|\Delta_{C_{R}}\arg g(z)|<\pi/3$.
          \item[$\triangleright$] \emph{Argument of the $z^{4}$ factor.}  
                Along $C_{R}$, $\arg z$ increases from $0$ to $\pi$, so
                \[
                    \Delta_{C_{R}}\arg z^{4}=4\pi.
                \]
          \item[$\triangleright$] \emph{Total for $C_{R}$.}  
                \[
                    \Delta_{C_{R}}\arg f(z)
                    = 4\pi + \varepsilon,
                    \qquad |\varepsilon|<\frac{\pi}{3}.
                \]
          \end{enumerate}

    \item \textbf{Contribution of the real segment $[-R,R]$.}

          Write $z=x\in\mathbb R$.  Then
          \[
              f(x)=\underbrace{\bigl(2x^{4}+x^{2}-1\bigr)}_{\displaystyle
                      \Re f(x)}
                   +\,i\,\underbrace{\bigl(-2x^{3}+2x\bigr)}_{\displaystyle
                      \Im f(x)}.
          \]

          \begin{enumerate}
          \item[$\triangleright$] \emph{Zeros of the real and imaginary parts.}
                \[
                    \Re f(x)=0
                    \;\Longleftrightarrow\;
                    2x^{4}+x^{2}-1=0
                    \;\Longleftrightarrow\;
                    x=\pm\frac{\sqrt2}{2},
                \]
                \[
                    \Im f(x)=0
                    \;\Longleftrightarrow\;
                    2x(1-x^{2})=0
                    \;\Longleftrightarrow\;
                    x\in\{-1,0,1\}.
                \]
                These five points split $[-R,R]$ into six sub‑intervals, but
                only the central three lie between $-\tfrac{\sqrt2}{2}$ and
                $\tfrac{\sqrt2}{2}$, where $\Re f$ changes sign; the outer
                parts behave uniformly.  We therefore track the argument in
                three blocks:

                \[
                    L_{1}=(-R,-\tfrac{\sqrt2}{2}],\quad
                    L_{2}=(-\tfrac{\sqrt2}{2},\tfrac{\sqrt2}{2}),\quad
                    L_{3}=[\tfrac{\sqrt2}{2},R).
                \]

          \item[$\triangleright$] \emph{Endpoints’ arguments.}
                \[
                    f(\pm R)\approx 2R^{4}>0
                    \;\Longrightarrow\;
                    \arg f(\pm R)\approx 0,
                \]
                \[
                    f\!\Bigl(\pm\frac{\sqrt2}{2}\Bigr)
                    = \pm i\,\frac{\sqrt2}{2}
                    \;\Longrightarrow\;
                    \arg f\Bigl(\pm\frac{\sqrt2}{2}\Bigr)
                    = \pm\frac{\pi}{2}.
                \]

          \item[$\triangleright$] \emph{Block $L_{1}$ ($\Re f>0$).}  
                Here $f$ lies in the right half‑plane, crossing from
                $\arg f(-R)\approx0$ down to
                $\arg f(-\tfrac{\sqrt2}{2})=-\pi/2$.  Hence
                \[
                    \Delta_{L_{1}}\arg f=-\frac{\pi}{2}.
                \]

          \item[$\triangleright$] \emph{Block $L_{3}$ (symmetry).}  
                By the same reasoning,
                \(
                    \Delta_{L_{3}}\arg f
                    =\arg f(R)-\arg f(\tfrac{\sqrt2}{2})
                    \approx 0-\frac{\pi}{2}=-\frac{\pi}{2}.
                \)

          \item[$\triangleright$] \emph{Block $L_{2}$ ($\Re f\le0$).}  
                Between the two points where $\Re f=0$ the real part is
                non‑positive while $\Im f$ changes sign twice ($x=-1,0,1$).  
                A direct plot or sign table shows the image of $L_{2}$ lies in
                the left half‑plane and swings from angle $+\pi/2$ down to
                $-\pi/2$ and back to $+\pi/2$, giving
                \[
                    \Delta_{L_{2}}\arg f=-\pi.
                \]

          \item[$\triangleright$] \emph{Total for $[-R,R]$.}
                \[
                    \Delta_{[-R,R]}\arg f
                    = -\frac{\pi}{2}-\pi-\frac{\pi}{2}
                    = -2\pi.
                \]
          \end{enumerate}

    \item \textbf{Putting it all together.}

          \[
              \Delta_{C}\arg f(z)
              = (4\pi+\varepsilon) + (-2\pi)
              = 2\pi+\varepsilon,
              \qquad |\varepsilon|<\frac{\pi}{3}.
          \]
          Hence
          \[
              N_{\Im z>0}
              =\frac{1}{2\pi}\,\Delta_{C}\arg f(z)
              \in\Bigl(\tfrac{2\pi-\pi/3}{2\pi},\tfrac{2\pi+\pi/3}{2\pi}\Bigr)
              \cap\mathbb Z
              =\{1\}.
          \]
          Therefore $f(z)$ has exactly \textbf{one zero in the upper
          half–plane}.
    \end{enumerate}
\end{problem}

\begin{problem}
    \begin{align}
        \text{Claim: }\;
        \nexists\,\text{entire }F\text{ such that }
        F(x)=1-e^{2\pi i/x}\quad\text{for all }x\in[1,2].
    \end{align}

    \begin{enumerate}
        \item \textbf{Suppose, for contradiction, that such an $F$ exists.}
              Define
              \[
                  G(z)=1-e^{2\pi i/z}.
              \]
              The function $G$ is analytic on $\Bbb C\setminus\{0\}$.

        \item \textbf{$F$ and $G$ agree on a set with a limit point in the domain of analyticity.}
              The closed interval $[1,2]\subset\Bbb R$ lies inside
              $\Bbb C\setminus\{0\}$ and contains infinitely many points,
              hence has an accumulation point in that domain.
              By hypothesis $F(x)=G(x)$ for every $x\in[1,2]$.

        \item \textbf{Apply the Identity (Uniqueness) Theorem.}
              Because $F$ and $G$ are analytic on $\Bbb C\setminus\{0\}$
              and coincide on a set with an accumulation point there,
              the Identity Theorem yields
              \[
                  F(z)=G(z)\quad\text{for all }z\in\Bbb C\setminus\{0\}.
              \]

        \item \textbf{Examine the behaviour at $z=0$.}
              \begin{enumerate}
                  \item[$\triangleright$] $F$ is assumed \emph{entire},
                        so it must be analytic at $0$; in particular
                        $\displaystyle\lim_{z\to0}F(z)$ exists and is finite.
                  \item[$\triangleright$] $G$ has an \emph{essential singularity}
                        at $z=0$ because $e^{2\pi i/z}$ does.
                        Consequently $\displaystyle\lim_{z\to0}G(z)$
                        does \emph{not} exist.
              \end{enumerate}
              Since $F(z)=G(z)$ for $z\neq0$, their limits as $z\to0$
              must coincide, which is impossible.

        \item \textbf{Contradiction and conclusion.}
              The contradiction shows the initial assumption was false.
              Therefore
              \[
                  \boxed{\text{no entire function }F\text{ satisfies }
                         F(x)=1-e^{2\pi i/x}\text{ on }[1,2].}
              \]
    \end{enumerate}
\end{problem}
count the zeros of $f$ in the open upper complex half-plane such that $Im>0$ 
\begin{align}
    f(z) = z^{4}-16+i(-z^{3}+z)
\end{align}

\renewcommand{\arraystretch}{1.2}   % set once, outside math‑mode

\begin{problem}
    %---------- definition of the polynomial -------------------------------
    \begin{align}
        f(z) &= z^{4}-16 + i\!\bigl(-z^{3}+z\bigr)
    \end{align}

    \begin{enumerate}
    %================= 1. Contour & strategy =================================
        \item \textbf{Contour and strategy (Argument Principle).}\;
              Use the counter‑clockwise contour
              \[
                  C=[-R,R]\cup C_R,
                  \qquad
                  C_R=\{\,Re^{i\theta}\mid 0\le\theta\le\pi\},
              \]
              and choose $R>2$ so $f$ has no zeros on $C$.
              The number $N_{+}$ of zeros in the upper half–plane satisfies
              \[
                  N_{+}=\frac{1}{2\pi}\,\Delta_{C}\arg f(z).
              \]

    %================= 2. Semicircle contribution ===========================
        \item \textbf{Semicircle $C_{R}$.}\;
              Factor
              \[
                  f(z)=z^{4}\,g(z),
                  \qquad
                  g(z)=1-\frac{16}{z^{4}}+i\!\Bigl(-\frac{1}{z}+\frac{1}{z^{3}}\Bigr).
              \]
              On $C_R$ we have $|g(z)-1|\le 16R^{-4}+R^{-1}+R^{-3}<\tfrac12$,
              so $g(C_R)$ never winds around $0$ and
              \[
                  \Delta_{C_R}\arg f(z)=4\pi+\varepsilon,
                  \qquad |\varepsilon|<\frac{\pi}{3}.
              \]

    %================= 3. Real‑axis contribution ============================
        \item \textbf{Real segment $[-R,R]$.}

              \[
                  f(x)=\underbrace{\bigl(x^{4}-16\bigr)}_{=:R(x)}
                       +\,i\,\underbrace{\bigl(-x^{3}+x\bigr)}_{=:I(x)}
                  \quad (x\in\mathbb R).
              \]

              The zeros of $R$ and $I$ are
              \[
                  R(x)=0\iff x=\pm2,
                  \qquad
                  I(x)=0\iff x\in\{-1,0,1\}.
              \]

              \[
              \begin{array}{c|ccccccc}
                  x & -R & -2 & -1 & 0 & 1 & 2 & R\\\hline
                  R(x) & + & 0 & - & - & - & 0 & +\\
                  I(x) & + & + & 0 & 0 & 0 & - & -
              \end{array}
              \]

              Reading off the quadrant changes one finds
              \[
                  \Delta_{[-R,R]}\arg f(z)=0.
              \]

    %================= 4. Tally & conclusion =================================
        \item \textbf{Total change and zero count.}\;
              \[
                  \Delta_{C}\arg f(z)=\bigl(4\pi+\varepsilon\bigr)+0
                                      =4\pi+\varepsilon,
                  \qquad |\varepsilon|<\frac{\pi}{3}.
              \]
              Hence
              \[
                  N_{+}
                  =\frac{1}{2\pi}\,\Delta_{C}\arg f(z)
                  \in\Bigl(2-\tfrac16,\,2+\tfrac16\Bigr)\cap\mathbb Z
                  =2.
              \]
              Therefore $f$ has exactly \textbf{two zeros in the open upper
              half–plane} $\Im z>0$.
    \end{enumerate}
\end{problem}
\end{document}
