\documentclass[12pt]{article}

% Packages
\usepackage[margin=1in]{geometry}
\usepackage{amsmath,amssymb,amsthm}
\usepackage{enumitem}
\usepackage{hyperref}
\usepackage{xcolor}
\usepackage{import}
\usepackage{xifthen}
\usepackage{pdfpages}
\usepackage{transparent}
\usepackage{listings}
\usepackage{tikz}
\usepackage{physics}
\usepackage{siunitx}
  \usetikzlibrary{calc,patterns,arrows.meta,decorations.markings}


\DeclareMathOperator{\Log}{Log}
\DeclareMathOperator{\Arg}{Arg}

\lstset{
    breaklines=true,         % Enable line wrapping
    breakatwhitespace=false, % Wrap lines even if there's no whitespace
    basicstyle=\ttfamily,    % Use monospaced font
    frame=single,            % Add a frame around the code
    columns=fullflexible,    % Better handling of variable-width fonts
}

\newcommand{\incfig}[1]{%
    \def\svgwidth{\columnwidth}
    \import{./Figures/}{#1.pdf_tex}
}
\theoremstyle{definition} % This style uses normal (non-italicized) text
\newtheorem{solution}{Solution}
\newtheorem{proposition}{Proposition}
\newtheorem{problem}{Problem}
\newtheorem{lemma}{Lemma}
\newtheorem{theorem}{Theorem}
\newtheorem{remark}{Remark}
\newtheorem{note}{Note}
\newtheorem{definition}{Definition}
\newtheorem{example}{Example}
\newtheorem{corollary}{Corollary}
\theoremstyle{plain} % Restore the default style for other theorem environments
%

% Theorem-like environments
% Title information
\title{}
\author{Jerich Lee}
\date{\today}

\begin{document}

\maketitle
\begin{problem}
  Find all one–to–one (i.e.\ biholomorphic) analytic functions that map the
  upper half–plane
  \[
     \mathbb{U}\;:=\;\{\,z\in\mathbb{C}\mid\operatorname{Im}z>0\}
  \]
  onto itself.
  
  \medskip
  \noindent
  \textbf{Hint.}\;
  The map
  \(
     \varphi(w)=\dfrac{i(1+w)}{1-w}
  \)
  sends the open unit disc
  \(
     \mathbb{D}:=\{\,w\in\mathbb{C}\mid |w|<1\}
  \)
  bijectively onto $\mathbb{U}$.
  \end{problem}
  
  \begin{solution}
  We proceed in four elementary steps.
  
  \medskip
  \textbf{1.\;Conjugate the unknown map with $\varphi$.}
  Let \(f:\mathbb{U}\to\mathbb{U}\) be analytic and one–to–one.
  Define
  \[
     g \;:=\; \varphi^{-1}\circ f\circ\varphi
     \;:\;
     \mathbb{D}\longrightarrow\mathbb{D}.
  \]
  Because each factor in the composition is biholomorphic,  
  \(g\) is a biholomorphic self–map of the unit disc.
  
  \medskip
  \textbf{2.\;Use the classification of disc automorphisms.}
  By the Schwarz–Pick theorem (or the
  Riemann mapping theorem in its refined form),
  every biholomorphic map of~$\mathbb{D}$ onto itself has the form  
  \[
     g(w)
     \;=\;
     e^{i\theta}\,\frac{w-\alpha}{1-\overline{\alpha}\,w},
     \qquad
        \theta\in\mathbb{R},\;
        \alpha\in\mathbb{D}.
     \tag{2.1}
  \]
  
  \medskip
  \textbf{3.\;Transform back to the half–plane.}
  Write \(w=\varphi^{-1}(z)=\dfrac{z-i}{\,z+i\,}\).
  Substituting \(w\) from this relation into~(2.1) and applying \(\varphi\)
  again yields
  \[
     f(z)
     \;=\;
     \varphi\!\Bigl(
        e^{i\theta}\,
        \frac{\dfrac{z-i}{z+i}-\alpha}{1-\overline{\alpha}\,\dfrac{z-i}{z+i}}
     \Bigr).
  \]
  A short computation (multiply numerator and denominator by \(z+i\))
  shows that \(f\) is a Möbius transformation,
  \[
     f(z)
     \;=\;
     \frac{az+b}{cz+d},
     \qquad\text{with }a,b,c,d\in\mathbb{R}.
     \tag{3.1}
  \]
  Indeed,
  the coefficients are\footnote{You do \emph{not} have to memorise these
  formulas; only their reality matters.  Setting
  \(
     \alpha=x+iy
  \)
  and expanding verifies \(a,b,c,d\in\mathbb{R}\).}
  \[
  \begin{aligned}
     a &= \cos\theta-\sin\theta\,x, 
  &\qquad
     b &= \sin\theta+\cos\theta\,x+2\sin\theta\,y,\\
     c &= -\sin\theta, 
  &\qquad
     d &= \cos\theta+\sin\theta\,x .
  \end{aligned}
  \]
  
  \medskip
  \textbf{4.\;Characterise the allowable coefficients.}
  Because \(f\) is analytic (hence orientation–preserving)
  and one–to–one,
  its determinant must be positive:
  \[
     \Delta:=ad-bc>0.
  \]
  Conversely, any Möbius map
  \(
     f(z)=\dfrac{az+b}{cz+d}
  \)
  with real coefficients and \(\Delta>0\) sends $\mathbb{U}$ onto itself
  (one checks directly that
  \(\operatorname{Im}f(z)=\Delta\,\operatorname{Im}z/|cz+d|^{2}>0\)).
  Hence every such map is a valid solution.
  
  \medskip
  \textbf{Final description.}
  The full set of one–to–one analytic self–maps of the upper half–plane is
  \[
     \boxed{\,\displaystyle
     f(z)=\frac{az+b}{cz+d},
     \quad a,b,c,d\in\mathbb{R},\;
           ad-bc>0\, }.
  \]
  This group of transformations is precisely
  \( \operatorname{PSL}(2,\mathbb{R})\), the projective special linear group
  over the reals.
  \end{solution}
  
  without additional packages.
  \begin{problem}
    Show that every \emph{one–to–one} entire function $f:\mathbb{C}\to\mathbb{C}$ has the form  
    \[
       f(z)=A\,z+B, \qquad A\neq 0 .
    \]
    \emph{Hint:} study the singularity of $f$ at the point ``$\infty$'' on the Riemann sphere~$\widehat{\mathbb{C}}$.
    \end{problem}
    
    \begin{solution}
    We analyse the extension of $f$ to $\widehat{\mathbb{C}}=\mathbb{C}\cup\{\infty\}$.
    
    \medskip
    \textbf{Step 1.\;The singularity at $\boldsymbol{\infty}$ is \emph{not} essential.}
    
    If $f$ had an essential singularity at $\infty$, then by the
    Casorati–Weierstrass (or Great Picard) theorem the image of every neighbourhood of $\infty$ would be dense in~$\mathbb{C}$.  
    That contradicts injectivity, which forces $f$ to omit every value except one.  
    Hence the singularity at $\infty$ is either removable or a pole.
    
    \medskip
    \textbf{Step 2.\;$\boldsymbol{\infty}$ is \emph{not} a removable singularity.}
    
    Were the singularity removable, $f$ would extend continuously to $\infty$ with a finite value.  
    Then $f$ would be bounded near~$\infty$ and hence (by Liouville’s theorem) bounded on~$\mathbb{C}$, forcing $f$ to be constant.  
    A constant map is never one–to–one, so the removable case is impossible.
    
    \medskip
    \textbf{Step 3.\;Therefore $\boldsymbol{\infty}$ is a pole—$f$ is a polynomial.}
    
    If the order of the pole is $m\ge1$ we obtain
    \[
       f(z)=a_m z^{m}+a_{m-1}z^{m-1}+\dots+a_0 ,
       \qquad a_m\neq0 .
    \]
    
    \medskip
    \textbf{Step 4.\;The degree $\boldsymbol{m}$ must equal $1$.}
    
    \begin{itemize}
       \item[(i)]  \emph{Algebraic argument.}
       Fix any complex number $w$.  
       The equation $f(z)=w$ is a polynomial equation of degree~$m$, so it has
       exactly $m$ (not necessarily distinct) solutions in $\widehat{\mathbb{C}}$.  
       If $m\ge2$, at least one value $w$ is assumed at two distinct points,
       violating injectivity.  
       Hence $m=1$.
       
       \item[(ii)] \emph{Critical–point argument (alternative).}
       For $m\ge2$ we have $f'(z)=m\,a_m z^{m-1}+\dots$,
       a polynomial of degree $m-1\ge1$, hence it has a root $z_0$.  
       The local mapping theorem then shows $f$ is \emph{not} one–to–one near $z_0$,
       again contradicting injectivity.
    \end{itemize}
    
    \medskip
    \textbf{Step 5.\;Conclusion.}
    Thus $m=1$ and
    \[
         f(z)=A z+B ,\qquad A\neq0 .
    \]
    Every affine map with $A\ne0$ is entire and one–to–one, so the above list is exhaustive.
    
    \[
       \boxed{\,f(z)=A z + B,\; A\in\mathbb{C}\setminus\{0\},\; B\in\mathbb{C}\,}
    \]
    \end{solution}
    
    \begin{problem}
      Let \(f\) be a one–to–one analytic function on a domain \(D\subset\mathbb{C}\).
      Assume there exists an analytic function \(h:D\to\mathbb{C}\) such that  
      \[
         h(z)^{2}=f(z)\qquad (z\in D).
      \]
      Show that \(h\) is also one–to–one (injective).
      \end{problem}
      
      \begin{solution}
      We give a short contradiction proof, laid out step by step.
      
      \medskip
      \textbf{Step 1.\;Assume non-injectivity of \(h\).}  
      Suppose there exist two \emph{distinct} points \(z_{1},z_{2}\in D\) with  
      \(h(z_{1})=h(z_{2})\).
      
      \smallskip
      \textbf{Step 2.\;Pass to \(f\) by squaring.}  
      Because \(h^{2}=f\) we obtain  
      \[
         f(z_{1})=h(z_{1})^{2}=h(z_{2})^{2}=f(z_{2}).
      \]
      
      \smallskip
      \textbf{Step 3.\;Injectivity of \(f\) forces equality of inputs.}  
      Since \(f\) is one–to–one on \(D\), the equality \(f(z_{1})=f(z_{2})\) implies  
      \(z_{1}=z_{2}\), contradicting the choice of distinct points.
      
      \smallskip
      \textbf{Step 4.\;Conclusion.}  
      Thus no two different points of \(D\) can share the same \(h\)-value;  
      \(h\) is injective.
      
      \medskip
      \emph{Remark.}  
      Even if a hypothetical counterexample satisfied \(h(z_{1})=-h(z_{2})\)
      instead of \(h(z_{1})=h(z_{2})\),
      squaring would still give \(f(z_{1})=f(z_{2})\),
      and the same argument rules this out.
      Therefore \(h\) is one–to–one in every case.
      \end{solution}
      \begin{problem}
        Let \(F:D_{1}\to D_{2}\) be a \textbf{one–to–one} (injective) analytic map
        \emph{onto} the domain \(D_{2}\).
        Suppose \(\{z_{k}\}_{k\in\mathbb N}\subset D_{1}\) satisfies
        \(z_{k}\longrightarrow p\) where 
        \(p\in\partial D_{1}\) (the boundary of \(D_{1}\)).
        Show that if \(\{F(z_{k})\}\) converges to a point \(q\), then  
        \(q\in\partial D_{2}\).
        
        \emph{Hint.}  Assume \(q\in D_{2}\); then \(q=F(z_{0})\) for some
        \(z_{0}\in D_{1}\).
        Use injectivity plus the open–mapping theorem to get a contradiction.
        \end{problem}
        
        \begin{solution}
        We argue by contradiction.
        
        \medskip
        \textbf{Step 1.\;Assume \(q\in D_{2}\).}
        Because \(F\) is onto, there exists
        \(z_{0}\in D_{1}\) such that
        \(F(z_{0})=q\).
        
        \medskip
        \textbf{Step 2.\;Exploit conformality at \(z_{0}\).}
        Being analytic and one–to–one, \(F\) is
        \emph{conformal} at \(z_{0}\);
        in particular \(F'(z_{0})\neq0\).
        Hence, by the open–mapping theorem (or the inverse–function theorem),
        there is a radius \(r>0\) such that
        \[
           F\bigl(\Delta(z_{0},r)\bigr)\subset\Delta(q,R)
           \quad\text{for some }R>0,
        \]
        where \(\Delta(a,\rho)=\{\,z\in\mathbb{C}:|z-a|<\rho\}\).
        
        \medskip
        \textbf{Step 3.\;Separate the discs from the boundary.}
        Because \(z_{0}\in D_{1}\) and \(D_{1}\) is open,
        we may choose \(r\) so small that
        \[
           \overline{\Delta}(z_{0},r)\subset D_{1}.
        \tag{3.1}
        \]
        Consequently \(\Delta(z_{0},r)\) lies at
        positive distance from the boundary point \(p\).
        
        \medskip
        \textbf{Step 4.\;Compare \(\{z_{k}\}\) with the small disc.}
        Since \(z_{k}\to p\) and \(p\notin\overline{\Delta}(z_{0},r)\),
        there exists an integer \(N\) such that for all \(k\ge N\)
        \[
           z_{k}\notin\overline{\Delta}(z_{0},r).
        \tag{4.1}
        \]
        
        \medskip
        \textbf{Step 5.\;Compare their images with the image disc.}
        By assumption \(F(z_{k})\to q\).
        Therefore there is \(K\ge N\) such that for all \(k\ge K\)
        \[
           F(z_{k})\in\Delta(q,R).
        \tag{5.1}
        \]
        
        \medskip
        \textbf{Step 6.\;Use injectivity to reach a contradiction.}
        From \eqref{5.1} and the inclusion in Step~2 we know
        \(F(z_{k})=F(w_{k})\) with
        \(w_{k}\in\Delta(z_{0},r)\).
        Injectivity of \(F\) forces \(w_{k}=z_{k}\), hence
        \[
           z_{k}\in\Delta(z_{0},r)\qquad(k\ge K),
        \]
        contradicting \eqref{4.1}.
        
        \medskip
        \textbf{Step 7.\;Conclusion.}
        The assumption \(q\in D_{2}\) is impossible;
        hence the limit point \(q\) must lie on the boundary:
        \[
           \boxed{\,q\in\partial D_{2}\,}.
        \]
        
        \end{solution}
        \[
          \Delta(a,r)  \;:=\; \{\,z\in\mathbb{{C}} : |z-a|<r\,\},
          \qquad
          \overline{\Delta}(a,r) \;:=\; \{\,z\in\mathbb{{C}}c : |z-a|\le r\,\}.
       \]
       
       \begin{itemize}
         \item \(\Delta(a,r)\) is the \emph{open disc} (or open ball) of radius \(r\)
               centred at the point \(a\).
         \item The overline in \(\overline{\Delta}(a,r)\) denotes **topological closure** in
               \(\mathbb{{C}}\).
               Thus \(\overline{\Delta}(a,r)\) is the \emph{closed disc}:
               one adds every boundary point \(|z-a|=r\) to the open disc.
       \end{itemize}
       
       In the proof above we needed a neighbourhood that is still entirely
       contained in the domain \(D_{1}\); writing
       \(\overline{\Delta}(z_{0},r)\subset D_{1}\) guarantees that
       \emph{both} the interior points and the boundary circle of that small disc
       stay inside \(D_{1}\). 
       \begin{theorem}[Open Mapping Theorem]
        Let $D\subset\mathbb{{C}}$ be a non-empty domain\footnote{A **domain** is a connected open set.} and
        let $f:D\longrightarrow\mathbb{{C}}$ be a *non-constant* analytic function.
        Then $f$ is an \emph{open map}; that is, for every open set
        $U\subset D$ the image $f(U)$ is an open subset of $\mathbb{{C}}$.
        \end{theorem}
        
        \begin{proof}[Proof sketch]
        Fix $z_{0}\in D$ and set $w_{0}:=f(z_{0})$.
        Because $f$ is analytic and non-constant, its derivative cannot vanish
        identically.  Two cases:
        
        \smallskip
        \emph{(i) $f'(z_{0})\neq0$.}\;
        By the complex inverse–function theorem
        there is a neighbourhood $\Delta(z_{0},r)$ on which $f$ is
        conformal, hence a homeomorphism onto
        $f\bigl(\Delta(z_{0},r)\bigr)$; the latter is therefore open.
        
        \smallskip
        \emph{(ii) $f'(z_{0})=0$.}\;
        Write the local power-series
        $f(z)=w_{0}+a_{m}(z-z_{0})^{m}+\dots$ with $m\ge2$ and $a_{m}\neq0$.
        Choose $\rho>0$ small enough that the series converges in
        $\Delta(z_{0},\rho)$ and $f$ has no other critical points there.
        On $\Delta(z_{0},\rho)\setminus\{z_{0}\}$ we are now in Case (i),
        so $f$ maps each punctured disc
        $\Delta(z_{0},\rho)\setminus\{z_{0}\}$ onto an open set.
        Because $(z-z_{0})^{m}$ maps a small disc \emph{onto} a
        small disc (though $m$‐to‐$1$), the full image
        $f\bigl(\Delta(z_{0},\rho)\bigr)$ is again a neighbourhood of $w_{0}$.
        \end{proof}
        
        \noindent
        Thus every point $w_{0}=f(z_{0})$ is an interior point of $f(U)$ for some
        small open $U\ni z_{0}$, proving that $f(U)$ is open.
        \qed
        \paragraph{The two analytic facts we need}

\begin{enumerate}
   \item[\textbf{(A)}] \textbf{Open Mapping Theorem.}\;
         A non–constant analytic function sends every non–empty \emph{open}
         set to an open set.

   \item[\textbf{(B)}] \textbf{Continuity of analytic maps.}\;
         If $f$ is analytic and $f(z_{0})=q$, then for every
         $\varepsilon>0$ there exists $\delta>0$ such that
         $|z-z_{0}|<\delta\;\Longrightarrow\;|f(z)-q|<\varepsilon$.
\end{enumerate}
\end{document}
