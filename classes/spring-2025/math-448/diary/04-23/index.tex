\documentclass[12pt]{article}

% Packages
\usepackage[margin=1in]{geometry}
\usepackage{amsmath,amssymb,amsthm}
\usepackage{enumitem}
\usepackage{hyperref}
\usepackage{xcolor}
\usepackage{import}
\usepackage{xifthen}
\usepackage{pdfpages}
\usepackage{transparent}
\usepackage{listings}
\usepackage{tikz}
\usepackage{physics}
\usepackage{siunitx}
\usepackage[T1]{fontenc}   % good font encoding
\usepackage[utf8]{inputenc}  % allow UTF-8 characters
  \usetikzlibrary{calc,patterns,arrows.meta,decorations.markings}


\DeclareMathOperator{\Log}{Log}
\DeclareMathOperator{\Arg}{Arg}

\lstset{
    breaklines=true,         % Enable line wrapping
    breakatwhitespace=false, % Wrap lines even if there's no whitespace
    basicstyle=\ttfamily,    % Use monospaced font
    frame=single,            % Add a frame around the code
    columns=fullflexible,    % Better handling of variable-width fonts
}

\newcommand{\incfig}[1]{%
    \def\svgwidth{\columnwidth}
    \import{./Figures/}{#1.pdf_tex}
}
\theoremstyle{definition} % This style uses normal (non-italicized) text
\newtheorem{solution}{Solution}
\newtheorem{proposition}{Proposition}
\newtheorem{problem}{Problem}
\newtheorem{lemma}{Lemma}
\newtheorem{theorem}{Theorem}
\newtheorem{remark}{Remark}
\newtheorem{note}{Note}
\newtheorem{definition}{Definition}
\newtheorem{example}{Example}
\newtheorem{corollary}{Corollary}
\newtheorem{answer}{Answer}
\theoremstyle{plain} % Restore the default style for other theorem environments
%

% Theorem-like environments
% Title information
\title{}
\author{Jerich Lee}
\date{\today}

\begin{document}

\maketitle
%------------------------------------------------------------
%  A careful walk-through of the argument appearing in the image
%  that every Möbius (linear fractional) transformation sends
%  circles and straight–lines to circles and straight–lines.
%------------------------------------------------------------

\bigskip
\textbf{1.  Decomposing the Möbius map}

Let
\[
   T(z)=\frac{az+b}{cz+d},\qquad  a,b,c,d\in\mathbb{{C}},\; c\neq 0.
\]
Rewrite the fraction by separating the constant term:
\[
   T(z)=\frac{1}{c}\!\left(\frac{bc-ad}{\,cz+d\,}+a\right).
\]
Hence \(T\) can be expressed as a composition of three simpler maps
(everywhere defined except where indicated):
\[
   U(z)=cz+d,\qquad
   V(w)=\frac{1}{w}\quad(w\neq 0),\qquad
   W(\zeta)=\frac{1}{c}\bigl[(bc-ad)\,\zeta+a\bigr].
\]
Indeed,
\[
   T = W\circ V\circ U.
\]

\bigskip
\textbf{2.  Two of the three components already work}

\begin{itemize}
  \item \(U(z)=cz+d\) is an \emph{affine} map: it is the composition of
        a rotation–dilation and a translation, so it clearly maps circles
        to circles and lines to lines.
  \item \(W(\zeta)=\dfrac{1}{c}\bigl[(bc-ad)\,\zeta+a\bigr]\) is again
        affine, hence enjoys the same property.
\end{itemize}

Therefore the only non-trivial part is to check that the
\emph{inversion} \(V(w)=1/w\) carries circles and lines to circles and
lines.

\bigskip
\textbf{3.  A real-variable description of circles and lines}

Write \(z=x+iy\;(x,y\in\mathbb{R})\).
Every (generalised) circle or straight line in~\(\mathbb{{C}}\) can be described
by a real equation of the form
\[
   \alpha\,(x^{2}+y^{2})+\beta\,x+\gamma\,y=\delta,
   \tag{1}
\]
where \(\alpha,\beta,\gamma,\delta\in\mathbb{R}\) are \emph{not} all
zero.  Precisely:

\begin{itemize}
  \item if \(\alpha\neq 0\) and \(\beta^{2}+\gamma^{2}+4\alpha\delta>0\),
        (1) is a circle;
  \item if \(\alpha=0\) the equation reduces to
        \(\beta x+\gamma y=\delta\), i.e.\ a straight line.
\end{itemize}

\bigskip
\textbf{4.  Substitute the inversion \(w=\dfrac{1}{z}\)}

Write
\[
   w=u+iv=\frac{1}{z}=\frac{x-iy}{x^{2}+y^{2}}
   \qquad\Longrightarrow\qquad
   u=\frac{x}{x^{2}+y^{2}},\quad
   v=-\frac{y}{x^{2}+y^{2}}.
\]
Multiplying both equalities by \(x^{2}+y^{2}\) gives
\(x= u(x^{2}+y^{2})\) and \(y= -v(x^{2}+y^{2})\).
Plug these into (1):

\[
   \alpha(x^{2}+y^{2})+\beta\,x+\gamma\,y
   =\alpha\frac{1}{u^{2}+v^{2}}
     +\beta\,\frac{u}{u^{2}+v^{2}}
     -\gamma\,\frac{v}{u^{2}+v^{2}}
   =\delta.
\]
Multiplying by the positive denominator \(u^{2}+v^{2}\) we arrive at
\[
   \boxed{\;\delta\,(u^{2}+v^{2})-\beta\,u+\gamma\,v=\alpha.\;}
   \tag{2}
\]

\bigskip
\textbf{5.  (2) is of the same type as (1)}

Equation (2) has precisely the same
shape as (1): the coefficients have simply
\emph{permuted roles}.
Consequently

\[
   \begin{cases}
     \text{if }(1)\text{ is a circle }(\alpha\neq 0),&
       (2)\text{ is also a circle;}\\[4pt]
     \text{if }(1)\text{ is a line }(\alpha=0),&
       (2)\text{ is also a line }(\delta=0).
   \end{cases}
\]

Thus the inversion \(V(w)=1/w\) sends every circle or straight line to
another circle or straight line (possibly sending the line at infinity
to an ordinary circle and vice-versa).

\bigskip
\textbf{6.  Putting the pieces together}

Since
\[
   T = W\circ V\circ U,\qquad
   \underbrace{U,W}_{\text{affine}}\;+\;
   \underbrace{V}_{\text{inversion}}
\]
\[
   \Longrightarrow\;
   T \text{ takes circles and lines to circles and lines.}
\]

\bigskip
\centerline{\(\Box\)}
\pagebreak
% Problem 6
\begin{problem}[]
  Show that a linear fractional transformation $T$ that maps the circle $|z| = 1$ onto itself has the form
  \[
      T(z) = \lambda \frac{z - \gamma}{1 - \overline{\gamma}\,z}, 
      \qquad |\lambda| = 1,\; |\gamma| \ne 1,
  \]
  or
  \[
      Tz = \frac{\lambda}{z}, 
      \qquad |\lambda| = 1.
  \] 
\end{problem}
\begin{solution}
  \[
\begin{aligned}
|a-z|^{2}
      &= (a-z)\,\overline{(a-z)} \\[4pt]
      &= (a-z)\,(\bar a-\bar z) \\[4pt]
      &= a\bar a \;-\; a\bar z \;-\; \bar a z \;+\; z\bar z \\[4pt]
      &= |a|^{2} \;-\; a\bar z \;-\; \bar a z \;+\; |z|^{2} \\[4pt]
      &= |a|^{2} \;-\; \bigl(a\bar z+\bar a z\bigr) \;+\; |z|^{2} \\[4pt]
      &= |a|^{2} \;-\; 2\,\mathbb{R}e\!\bigl(\bar a z\bigr) \;+\; |z|^{2}.
\end{aligned}
\]
\end{solution}
\begin{problem}[]
 % Problem 7(c)
Find a linear fractional transformation $T$ such that
\[
    T(\mathbb{R}) = \mathbb{R},
    \qquad
    T\bigl(i\mathbb{R}\bigr)
        = \Bigl\{\,w\in\mathbb{C} : \bigl|\,w-\tfrac54\,\bigr| = \tfrac34\,\Bigr\}.
\] 
\end{problem}
\begin{solution}
   \begin{solution}
      \paragraph{1.  Geometry of the problem.}
      The two sets we want to map intersect at the points $0$ and $\infty$:
      \[
         \mathbb{R}\cap i\mathbb{R}=\{0,\infty\}.
      \]
      Their images under $T$ must therefore lie on \emph{both}
      $T(\mathbb{R})=\mathbb{R}$ and the target circle
      \[
         C=\Bigl\{w\in\mathbb{C}\;:\;\bigl|\,w-\tfrac54\,\bigr|=\tfrac34\Bigr\}.
      \]
      The real points of $C$ solve 
      $\lvert w-\tfrac54\rvert=\tfrac34$, hence
      \[
         w_1=\tfrac54+\tfrac34=2,
         \qquad
         w_2=\tfrac54-\tfrac34=\tfrac12.
      \]
      We may therefore prescribe
      \[
         T(0)=2,
         \qquad
         T(\infty)=\frac12.
         \tag{1}
      \]
      
      \paragraph{2.  A first normal form.}
      Write the Möbius map with \emph{real} coefficients
      \[
         T(z)=\frac{az+b}{cz+d}, \qquad ad-bc\neq0,\quad a,b,c,d\in\mathbb{R},
      \]
      so that $T(\mathbb{R})=\mathbb{R}$ automatically.
      Condition (1) gives
      \[
         \frac{b}{d}=2,
         \qquad
         \frac{a}{c}=\frac12
         \;\Longrightarrow\;
         b=2d,\;c=2a.
      \]
      Scaling all coefficients by the same non–zero factor allows us to fix $d=1$, leading to the one-parameter family
      \[
         T_a(z)=\frac{az+2}{2az+1},
         \qquad a\in\mathbb{R}\setminus\{0\}.
         \tag{2}
      \]
      
      \paragraph{3.  Fixing the parameter \(a\).}
      It suffices to send a third point of $i\mathbb{R}$ to a known point of $C$.
      Choose $z=i$ and the \emph{topmost} point of $C$,
      \[
         w_0=\frac54+\frac34\,i,
      \]
      and impose $T_a(i)=w_0$:
      \[
         \frac{ai+2}{1+2ai}=w_0.
      \]
      Writing $a=k$ and clearing denominators yields the real system
      \[
         \begin{cases}
            \displaystyle\frac54-2k\cdot\frac34=2,\\[6pt]
            \displaystyle\frac34+2k\cdot\frac54=k,
         \end{cases}
         \qquad\Longrightarrow\qquad
         k=-\dfrac12.
      \]
      
      \paragraph{4.  The desired map.}
      Setting $a=-\tfrac12$ in (2) gives
      \[
         T(z)=\frac{-\tfrac12 z+2}{-\;z+1}
              =\frac{z-4}{2\,(z-1)}.
         \tag{3}
      \]
      
      \paragraph{5.  Verification.}
      \begin{enumerate}[label=\textbf{\alph*)}]
         \item \emph{$T(\mathbb{R})=\mathbb{R}$.}  
               Because the coefficients in \eqref{3} are real, conjugation commutes with $T$:
               $T(\bar z)=\overline{T(z)}$, so the real axis is taken to itself.
         \item \emph{$T(i\mathbb{R})=C$.}  
               Möbius maps send circles/lines to circles/lines.  
               The three distinct points
               \[
                  0,\;\infty,\;i
                  \quad\mapsto\quad
                  2,\;\tfrac12,\;w_0
               \]
               determine $i\mathbb{R}$ and $C$, respectively, so the whole axis $i\mathbb{R}$ is carried to the unique circle through $2,\tfrac12,w_0$, namely $C$.
      \end{enumerate}
      
      \paragraph{6.  Alternative orientation.}
      Swapping the images in (1), i.e.\ taking $T(0)=\tfrac12$ and $T(\infty)=2$, produces the companion map
      \[
         \widetilde{T}(z)=\frac{2z+1}{z+2},
      \]
      which satisfies the same two requirements.
      \end{solution}
\end{solution}
\begin{problem}[]
 % Problem 8(a)
In Exercise 7(c), determine what happens to the first quadrant under the action of $T$. 
\end{problem}
\begin{solution}
   \begin{solution}
      \paragraph{Image of the first quadrant \(Q_1=\{z=x+iy : x>0,\;y>0\}\).}
      
      \begin{enumerate}
        \item \textbf{Describe the boundary of \(Q_1\).}\\
              The quadrant is bounded by the two rays  
              \[
                  R_+ := \{x \in \mathbb{R}_{>0}\}, 
                  \qquad 
                  I_+ := \{\,it : t>0\}.
              \]
      
        \item \textbf{Send the boundary rays through \(T\).}\\
              For the map
              \[
                  T(z)=\frac{z-4}{2\,(z-1)}, 
                  \qquad 
                  c=\frac54,\; r=\frac34,
              \]
              we have:
              \begin{itemize}
                \item Because \(T\) has real coefficients, \(T(\mathbb{R})=\mathbb{R}\), so 
                      \(T(R_+)=\mathbb{R}_{>0}\).
                \item Parameterise \(I_+\) by \(z=it\;(t>0)\):
                      \[
                          T(it)=\frac{it-4}{2\,(it-1)}
                               =c + r \,e^{i\theta(t)}, 
                          \qquad \theta(t)\in(0,\pi).
                      \]
                      Thus
                      \[
                          T(I_+)=\text{upper semicircle of } C
                                 =\{\,w:|w-c|=r,\; \operatorname{Im}w>0\},
                      \]
                      running from \(w=2\) (as \(t\to0^+\)) to \(w=\tfrac12\) (as \(t\to+\infty\)).
              \end{itemize}
      
        \item \textbf{Locate \(T(Q_1)\) relative to the circle \(C\).}\\
              Take the interior test point \(z_0=1+i\):
              \[
                  w_0 := T(1+i)
                        =\frac{1+i-4}{2\,(1+i-1)}
                        =\frac{-3+i}{2i}
                        =\frac12+\frac32\,i.
              \]
              Its distance from \(c\) is
              \[
                  |w_0-c|
                    =\sqrt{(-0.75)^2 + 1.5^2}
                    > r,
              \]
              so \(w_0\) (hence all of \(T(Q_1)\)) lies \emph{outside} the circle \(C\).
      
        \item \textbf{Result.}\\
              \[
                  T(Q_1)=
                  \Bigl\{
                    w\in\mathbb{C} :
                    \operatorname{Im}w>0,\;
                    |\,w-\tfrac54\,|>\tfrac34
                  \Bigr\},
              \]
              the unbounded region in the \emph{upper half-plane} exterior to \(C\).
              Its boundary is the positive real axis together with the upper semicircle of \(C\).
      \end{enumerate}
      \end{solution} 
\end{solution}
\begin{problem}[]
  % Problem 16
Show that if $T$ is a linear fractional transformation with $Tz_j = w_j$ for $j = 1, 2, 3$, then
\[
    (z, z_1, z_2, z_3) = (T(z), w_1, w_2, w_3)
    \quad\text{for all } z.
\]
\end{problem}
\begin{solution}
  
\end{solution}

Show that a linear-frac. trans. T that maps $\left\vert z \right\vert =1$ to itself has the form 
\begin{align}
   T(z)=\lambda \frac{z-\gamma}{1-\overline{\gamma} z}, \left\vert \lambda \right\vert =1, \left\vert \gamma \right\vert \neq 1
\end{align} 
or 
\begin{align}
   T(z) = \frac{\lambda}{z}
\end{align} 
Solution:
$T: \mathbb{{C}}_{\infty} \to \mathbb{{C}}_{\infty} $ is one-to-one. Let $c= T^{-1}(\infty)$.
Case 1:
$c=\infty, T(\infty) = \infty$. Then, $T(z) = \lambda z + b, \lambda \neq  0$, maps $\left\vert z \right\vert =1$ to a circle with center at $b$. Hence $ b=0$. Then $T(z) = \lambda z, \left\vert \lambda \right\vert=1, \gamma = 0$. 
Case 2: $c=0, T(0)=\infty$. Put $g(z) = T(\frac{1}{z})$. $g(\infty) = T(0)= \infty$. $g$ maps $\left\vert z \right\vert =1$ to itself. By case 1, $g(z) = \lambda z, \left\vert \lambda \right\vert = 1. $ Then $T(z) = g(\frac{1}{z}) = \frac{\lambda}{z}$
Case 3:
$c\neq 0, c\neq  \infty$. Obviously $\left\vert c \right\vert \neq 1$. Let $g(z) = T(\phi(z)), \phi(z) = \frac{z+\gamma}{1+\overline{\gamma} z}, \gamma = \frac{1}{\overline{c} }$. Then $g(\infty) = T(\phi(\infty))=T(c) = \infty, \left\vert \gamma \right\vert \neq 1$. Recall $\phi$ maps $\left\vert z \right\vert =1$ to itself, $\phi^{-1}(z)= \frac{z-\gamma}{1-\overline{\gamma} z}$. By case 1, $g(z)=\lambda z, \left\vert \lambda \right\vert =1$. Then $T(z) = g(\phi^{-1}(z))= \lambda \frac{z-\gamma}{1-\overline{\gamma} z}$  

%------------------------------------------------------------
%  Theorem.
%  Every linear–fractional (Möbius) map that sends the unit
%  circle $\lvert z\rvert=1$ onto itself has one of the two forms
%
%     T(z) = \lambda\,\dfrac{z-\gamma}{1-\overline{\gamma}\,z},
%        \qquad |\lambda| = 1,\; |\gamma|\neq 1,
%
%  or
%
%     T(z) = \dfrac{\lambda}{z},\qquad |\lambda| = 1.
%
%  Below is a detailed step-by-step proof, organised by cases
%  according to the point whose image is~$\infty$.
%------------------------------------------------------------
\begin{proof}
   Let
   \[
      T(z)=\frac{az+b}{cz+d},\qquad ad-bc\neq0,
   \]
   be a Möbius transformation that maps the unit circle
   $\lvert z\rvert=1$ bijectively onto itself.
   Because every Möbius map is a bijection of the extended complex
   plane $\mathbb{{C}}_\infty:=\mathbb{{C}}\cup\{\infty\}$, there is a unique point
   \[
      c \;=\; T^{-1}(\infty)\;\in\;\mathbb{{C}}_\infty
   \]
   (the pole of~$T$).
   We analyse three mutually exclusive cases.
   
   \bigskip
   \noindent\textbf{Case 1: $c=\infty$ (so $T(\infty)=\infty$).}
   
   If $c=\infty$ then necessarily $c=0$ in the matrix of $T$, so
   \(
      T(z)=\dfrac{az+b}{d}=\lambda z+b
   \)
   with $\lambda:=a/d\neq0$.
   \textbf{Why does $c=\infty$ force the matrix entry $c$ to be $0$?}

\medskip
\textbf{1.  General form of a Möbius map.}
\[
   T(z)=\frac{az+b}{cz+d},\qquad ad-bc\neq0.
\]
The value of $T$ is \emph{infinite} precisely when the denominator
vanishes:
\[
   cz+d = 0
   \;\;\Longrightarrow\;\;
   z = -\frac{d}{c}.
\]

\medskip
\textbf{2.  What does the symbol $c=\infty$ mean here?}

In the proof we set
\[
   c:=T^{-1}(\infty),
\]
i.e.\ \emph{the point $z$ that is sent to $\infty$ by $T$}.
Saying “$c=\infty$’’ means:

> There is \emph{no finite} complex number $z$ for which $T(z)=\infty$.

\medskip
\textbf{3.  Consequence for the coefficient $c$ in the denominator.}

* If $c\neq 0$, then $z=-d/c$ is a \emph{finite} complex number, so
  $T\bigl(-d/c\bigr)=\infty$.  
  This contradicts $c=\infty$.
* Hence the only way to avoid a finite pole is to have $c=0$ in the
  matrix representation.

\medskip
\textbf{4.  Simplifying the formula when $c=0$.}

With $c=0$ the map reduces to
\[
   T(z)=\frac{az+b}{d}.
\]
Because the matrix $(a\;b;0\;d)$ is projective, we may divide all
entries by $d\neq0$ (scaling does not change the Möbius transformation):
\[
   T(z)=\frac{a}{d}\,z + \frac{b}{d}
       = \lambda z + b',
   \qquad\lambda:=\frac{a}{d}\neq 0,\;
          b':=\frac{b}{d}.
\]

Thus “$c=\infty$ implies $c=0$’’ and yields the affine form
\(T(z)=\lambda z+b'\).
   Geometrically, $T$ is an \emph{affine map} composed of a rotation-dilation
   ($z\mapsto\lambda z$) followed by a translation ($z\mapsto z+b$).
   
   \begin{itemize}
     \item[\(\circ\)]
       The image of the unit circle under $z\mapsto\lambda z$ is still the
       unit circle if and only if $|\lambda|=1$ (a pure rotation).
     \item[\(\circ\)]
       The subsequent translation by $b$ would shift the centre away from
       the origin, hence cannot preserve the unit circle unless $b=0$.
   \end{itemize}
   Thus
   \[
      T(z)=\lambda z,\qquad |\lambda|=1.
   \]
   This is the special case of the first formula with $\gamma=0$.
   
   \bigskip
   \noindent\textbf{Case 2: $c=0$ (so $T(0)=\infty$).}
   
   Define
   \[
      g(z)\;:=\;T\!\bigl(\tfrac1z\bigr)\quad(z\neq0).
   \]
   Because $\tfrac1z$ also sends the unit circle onto itself, the map $g$
   shares the properties:
   \[
      g(\mathbb{{C}}_\infty\setminus\{0\})\subseteq\mathbb{{C}}_\infty,
      \quad g(\infty)=T(0)=\infty,
      \quad g(\lvert z\rvert=1)=\lvert z\rvert=1.
   \]
   Hence $g$ falls under Case 1 and must be a rotation:
   \(
      g(z)=\lambda z,\;|\lambda|=1.
   \)
   Consequently
   \[
      T(z)=g\!\bigl(\tfrac1z\bigr)=\frac{\lambda}{z},\qquad |\lambda|=1,
   \]
   which is exactly the second form claimed.
   %%%%%%%%%%%%%%%%%%%%%%%%%%%%%%%%%%%%%%%%%%%%%%%%%%%%%%%%%%%%%%%%%%%%
%  Expanded explanation of \textbf{Case 2}  
%  (“Pole at the origin”) in the proof that a Möbius map
%  preserving the unit circle has the form
%       T(z)=\lambda(z-\gamma)/(1-\bar\gamma z)   or   T(z)=\lambda/z.
%
%  Assumption for Case 2:
%      c:=T^{-1}(\infty)=0  \quad\Longrightarrow\quad  T(0)=\infty .
%
%  Goal:  Show   T(z)=\lambda/z \; \text{with}\; |\lambda|=1.
%%%%%%%%%%%%%%%%%%%%%%%%%%%%%%%%%%%%%%%%%%%%%%%%%%%%%%%%%%%%%%%%%%%%
\subsubsection*{Case \textup{2}: $c=0$ so $T(0)=\infty$}

\begin{enumerate}
%-------------------------------------------------------------------
\item[\textbf{Step 1.}] \textbf{Pre-compose with the inversion $z\mapsto 1/z$.}

Define a new Möbius transformation
\[
    g(z)\;:=\;T\!\Bigl(\frac{1}{z}\Bigr),
    \qquad z\in\mathbb{{C}}_\infty\setminus\{0\}.
\]
(The point $z=0$ will be handled below.)

\vspace{4pt}
%-------------------------------------------------------------------
\item[\textbf{Step 2.}] \textbf{$g$ shares the crucial properties of $T$.}

\begin{itemize}
  \item The inversion $I(z)=1/z$ satisfies $|I(z)|=1$ whenever
        $|z|=1$.  Thus
        \[
           |z|=1\;\Longrightarrow\;|I(z)|=1
           \;\Longrightarrow\;|T(I(z))|=1
           \;\Longrightarrow\;|g(z)|=1.
        \]
        Hence $g$ maps the unit circle onto itself.

  \item As $z\to\infty$ we have $I(z)\to0$, so
        \[
            g(\infty)=\lim_{z\to\infty}T\!\Bigl(\frac1z\Bigr)=T(0)=\infty .
        \]
\end{itemize}

\vspace{4pt}
%%%%%%%%%%%%%%%%%%%%%%%%%%%%%%%%%%%%%%%%%%%%%%%%%%%%%%%%%%%%
%  Why does  g(\infty)=\infty  when
%       g(z)=T\!\bigl(\frac1z\bigr)  and  T(0)=\infty?
%
%  (Expanded explanation of the displayed limit.)
%%%%%%%%%%%%%%%%%%%%%%%%%%%%%%%%%%%%%%%%%%%%%%%%%%%%%%%%%%%%

\paragraph{Setting.}
For Case~2 we have a Möbius map
\[
   T(z)=\frac{az+b}{cz+d},\qquad c=0,\;ad\neq0,
\]
so that \(\displaystyle T(0)=\frac{b}{d}=\infty\).
We define
\[
   g(z):=T\!\Bigl(\frac1z\Bigr),\qquad z\neq0.
\]

\paragraph{What does \(g(\infty)\) mean?}
On the Riemann sphere \(\widehat{\mathbb{{C}}}=\mathbb{{C}}\cup\{\infty\}\)
the value of a meromorphic function at~\(\infty\) is defined by the
limit along any path tending to~\(\infty\):
\[
   g(\infty)
   \;:=\;\lim_{z\to\infty}g(z)
   \;=\;\lim_{|z|\to\infty}g(z),
\]
provided this limit exists in \(\widehat{\mathbb{{C}}}\).

\paragraph{Computing the limit.}
Because \(g(z)=T\!\bigl(\frac1z\bigr)\), substitute \(w=\frac1z\):

\[
  \boxed{\;
  \displaystyle
   \lim_{z\to\infty}g(z)
     =\lim_{|z|\to\infty}T\!\Bigl(\frac1z\Bigr)
     =\lim_{w\to 0}T(w)
     =T(0)
     =\infty.
  }
\]

\emph{Why does \(w\to0\)?}  
If \(|z|\to\infty\) then \(|1/z|\to0\); conversely every sequence
\((w_n)\) with \(w_n\to0\) arises from \(z_n=1/w_n\) with
\(|z_n|\to\infty\).

\paragraph{Alternate sequence-based justification.}
Choose any sequence \((z_n)\) in \(\mathbb{{C}}\) with \(|z_n|\to\infty\).
Then \(w_n:=1/z_n\to0\).  Since Möbius maps are continuous on
\(\widehat{\mathbb{{C}}}\),
\[
   g(z_n)=T(w_n)\longrightarrow T(0)=\infty.
\]
Because the limit is independent of the sequence, the spherical limit
exists and equals~\(\infty\).  Therefore

\[
   g(\infty)=\infty.
\]

\bigskip
(This completes the expanded justification of the line
\(\displaystyle g(\infty)=\lim_{z\to\infty}T(1/z)=T(0)=\infty\).)
%%%%%%%%%%%%%%%%%%%%%%%%%%%%%%%%%%%%%%%%%%%%%%%%%%%%%%%%%%%%%%%%%%%%
%  Why does $T(0)=\infty$ in \textbf{Case 2}?
%%%%%%%%%%%%%%%%%%%%%%%%%%%%%%%%%%%%%%%%%%%%%%%%%%%%%%%%%%%%%%%%%%%%
\paragraph{Two different symbols called $c$.}
\begin{itemize}
\item[(i)]  In the matrix representation of a Möbius map  
  \[
     T(z)=\frac{az+b}{cz+d},\qquad ad-bc\neq0,
  \]
  the \emph{coefficient} $c$ multiplies $z$ in the denominator.
\item[(ii)]  In the proof we introduced a \emph{point}
  \[
        c:=T^{-1}(\infty)\in\widehat{\mathbb{{C}}},
  \]
  i.e.\ the \textbf{unique point of the domain that is sent to $\infty$}.  
  This “$c$’’ is *not* the same object as the matrix entry $c$—it is a
  complex number (possibly $\infty$).
\end{itemize}

\paragraph{Definition of Case 2.}
Case 2 is defined by the statement  
\[
   \boxed{\,c = T^{-1}(\infty)=0\,}.
\]
In words: the \emph{origin} is the point that $T$ blows up at.
\[
   \Longrightarrow\quad
   T(0)=\infty
   \quad\text{by definition.}
\]
Nothing about the matrix coefficients has been used yet; we have only
unpacked what “$c=0$’’ means.

\paragraph{Consistency with the algebraic formula.}
To see that $T(0)=\infty$ forces a certain pattern in the coefficients,
set $z=0$ in the generic formula:
\[
   T(0)=\frac{b}{d}.
\]
For this quotient to be infinite we must have $d=0$ (while $b\neq0$
because $ad-bc\neq0$).  
Thus a Möbius map with $T(0)=\infty$ can always be written
\[
   T(z)=\frac{az+b}{cz}\quad(c\neq0,\;b\neq0),
\]
and indeed the limit as $z\to0$ is $\infty$.

Hence there is no contradiction:  
* \textit{Case 2 assumes $T^{-1}(\infty)=0$; this is precisely the
  statement $T(0)=\infty$.}
%%%%%%%%%%%%%%%%%%%%%%%%%%%%%%%%%%%%%%%%%%%%%%%%%%%%%%%%%%%%%%%%%%%%
%%%%%%%%%%%%%%%%%%%%%%%%%%%%%%%%%%%%%%%%%%%%%%%%%%%%%%%%%%%%
%-------------------------------------------------------------------
\item[\textbf{Step 3.}] \textbf{Determine the explicit form of $g$.}

Because
\[
   g:\mathbb{{C}}_\infty\longrightarrow\mathbb{{C}}_\infty,\quad
   g(\lvert z\rvert=1)=\lvert z\rvert=1,\quad
   g(\infty)=\infty,
\]
$g$ falls under \emph{Case 1} of the proof (no finite pole).  Therefore
$g$ must be an \emph{affine rotation}:
\[
   g(z)=\lambda z + b,
   \qquad |\lambda|=1,\;b\in\mathbb{{C}} .
\]

\medskip\noindent
\emph{Why is the translation term $b$ forced to vanish?}  
If $b\neq0$, the image of the unit circle under $z\mapsto z+b$
would be a circle \emph{whose centre is $b$}.  That centre has modulus
$|b|>0$, so the image would \emph{no longer} be centred at the origin,
contradicting the fact that $g$ maps the unit circle onto itself.
Hence $b=0$ and
\[
   g(z)=\lambda z,\qquad |\lambda|=1.
\]

\vspace{4pt}
%-------------------------------------------------------------------
\item[\textbf{Step 4.}] \textbf{Translate the information back to $T$.}

For every $z\in\mathbb{{C}}_\infty$,
\[
   T(z)=g\!\Bigl(\frac{1}{z}\Bigr)=\lambda\,\frac{1}{z}.
\]
Because $|\lambda|=1$ the map $T$ indeed sends $|z|=1$ onto itself.

\vspace{2pt}
\[
   \boxed{\;T(z)=\frac{\lambda}{z},\quad |\lambda|=1\;}
\]

This completes the detailed justification of Case 2 and yields exactly
the \emph{second} normal form stated in the theorem.
\end{enumerate}
   \bigskip
   \noindent\textbf{Case 3: $c\in\mathbb{{C}}$ with $|c|\neq1$ (finite pole off the unit circle).}
   
   \medskip
   \emph{Step 3.1 –   Construct a Blaschke map that moves $c$ to $\infty$.}
   
   Set
   \[
      \gamma:=\frac{1}{\overline{c}},\qquad\text{so that }|\gamma|\neq1.
   \]
   Define the Möbius map
   \[
      \phi(z)=\frac{z+\gamma}{1+\overline{\gamma}\,z},
   \]
   a (generalised) \emph{Blaschke factor}.  One checks
   \[
      |\phi(z)|=1\quad\text{whenever }|z|=1
      \quad\Longrightarrow\quad \phi\bigl(\lvert z\rvert=1\bigr)
      =\lvert z\rvert=1,
   \]
   and
   \(
      \phi(c)=\infty,\;\; \phi(\infty)= -\gamma.
   \)
   
   \medskip
   \emph{Step 3.2 –  Reduce to Case 1 by composing with $\phi$.}
   
   Let
   \[
      g(z):=T\!\bigl(\phi(z)\bigr),\qquad z\in\mathbb{{C}}_\infty.
   \]
   Then
   \[
      g(\infty)=T\!\bigl(\phi(\infty)\bigr)=T(-\gamma)\neq\infty,
      \qquad
      g\bigl(\phi^{-1}(\infty)\bigr)=g(c)=\infty.
   \]
   Because $\phi$ and $T$ each send the unit circle onto itself,
   so does their composition \(g\).
   Moreover $g(\infty)=\infty$, hence \(g\) lies in Case 1 and must be a
   rotation:
   \(
      g(z)=\lambda z,\;|\lambda|=1.
   \)
   
   \medskip
   \emph{Step 3.3 –  Solve back for $T$.}
   
   Since $\phi^{-1}(z)=\dfrac{z-\gamma}{1-\overline{\gamma}\,z}$,
   we obtain
   \[
      T(z)=g\!\bigl(\phi^{-1}(z)\bigr)
          =\lambda\,\frac{z-\gamma}{1-\overline{\gamma}\,z},
      \qquad |\lambda|=1,\;|\gamma|\neq1.
   \]
   
   \bigskip
   \noindent\textbf{Conclusion.}
   Exactly one of the three cases applies, and they exhaust all linear
   fractional maps that preserve the unit circle, yielding the two forms
   
   \[
      \boxed{\;
         T(z)=\lambda \frac{z-\gamma}{1-\overline{\gamma}\,z},
         \;|\lambda|=1,\;|\gamma|\neq1
      \;}
      \quad\text{or}\quad
      \boxed{\;
         T(z)=\frac{\lambda}{z},
         \;|\lambda|=1
      \;}.
   \]
   \end{proof}
   \begin{answer}
      \textbf{Why break the proof into three cases?}
      
      Let \(T(z)=\dfrac{az+b}{cz+d}\;(ad-bc\neq0)\) be a Möbius map that
      sends the unit circle \(\Gamma=\{\,|z|=1\}\) onto itself.  
      One structural feature immediately singles out \emph{three and only three}
      qualitatively different situations:
      
      \[
         \boxed{\; \text{the \emph{pole}}
                 \;c:=T^{-1}(\infty)= -\tfrac{d}{c_{\text{matrix}}}\;}
         \qquad(\text{a single point in }\widehat{\mathbb{{C}}}).
      \]
      
      Because \(T\) is a bijection of the Riemann sphere, \(c\) is the
      \emph{unique} point where \(T\) blows up.  Where that point sits
      dictates which elementary conformal “building block’’
      (rotation, inversion, Blaschke factor) is best suited to normalise the
      transformation.
      
      \medskip
      \begin{enumerate}
      %-----------------------------------------------------------------
      \item[\textbf{Case 1:\;}] \(\boldsymbol{c=\infty}\).
            \\
            \textit{Interpretation.}  The denominator of \(T\) never vanishes,
            so \(T\) is an \emph{affine map}.  
            A single rotation \(z\mapsto\lambda z\,(|\lambda|=1)\) is already
            the desired normal form.  No other technique is needed.
      
      %-----------------------------------------------------------------
      \item[\textbf{Case 2:\;}] \(\boldsymbol{c=0}\).
            \\
            \textit{Interpretation.}  The pole lies at the origin, so \(T\) is
            necessarily of the form \(T(z)=\dfrac{\lambda}{z}\) up to a
            rotation.  
            The quickest way to see this is to \emph{pre-compose} with the
            inversion \(I(z)=1/z\), thereby \emph{sending the pole back to
            infinity}.  
            Once infinity is the unique pole, we are reduced to Case~1 again.
      
      %-----------------------------------------------------------------
      \item[\textbf{Case 3:\;}] \(\boldsymbol{c\in\mathbb{{C}},\;|c|\neq1}\).
            \\
            \textit{Interpretation.}  The pole lies in the finite plane
            but \emph{off} the unit circle (it \textbf{must} be off
            \(\Gamma\); otherwise \(T\) could not map \(\Gamma\) onto itself).
            A Blaschke factor
            \[
               \phi_{c}(z)=\frac{z+\gamma}{1+\bar\gamma z},
               \qquad \gamma=\frac1{\bar c},
            \]
            is the canonical automorphism of the unit disc that
            sends \(c\) to \(\infty\).
            Composing \(T\) with \(\phi_{c}\) therefore converts Case 3 to
            Case 1.  Undoing the composition yields the standard Blaschke
            form \(\lambda\,(z-\gamma)/(1-\bar\gamma z)\).
      
      \end{enumerate}
      
      \bigskip
      \textbf{Why these three cases are exhaustive and mutually exclusive}
      
      \vspace{-2mm}
      \[
         c\;=\;T^{-1}(\infty)\;\in\;\widehat{\mathbb{{C}}}
         \;=\;\mathbb{{C}}\cup\{\infty\}
         \quad\Longrightarrow\quad
         c=\infty,\;c=0,\;\text{or }c\in\mathbb{{C}}\setminus\{0\}.
      \]
      \(\Gamma\)-preservation forces \(|c|\neq1\) in the finite situation, so
      no further sub-cases appear.  Hence every Möbius map that keeps the unit
      circle invariant lands in exactly \emph{one} of the three scenarios,
      and each scenario has its own most economical normalising trick:
      
      \[
         \begin{array}{c|c}
           \text{location of the pole} & \text{tool that fixes the pole at }\infty
           \\\hline
             \infty & \text{none (already affine)} \\[4pt]
             0 & I(z)=1/z \\[4pt]
             c,\;|c|\neq1 & \phi_{c}(z)=\dfrac{z+\gamma}{1+\bar\gamma z}
         \end{array}
      \]
      
      By treating these cases separately we keep the algebra
      minimal and make clear \emph{which elementary conformal maps are doing
      the work} in each circumstance.  An alternative one-size-fits-all
      derivation exists (via the Schwarz lemma on the unit disc) but hides
      these geometrically meaningful distinctions.
      \end{answer}
\end{document}
