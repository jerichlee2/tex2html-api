\documentclass[12pt]{article}

% Packages
\usepackage[margin=1in]{geometry}
\usepackage{amsmath,amssymb,amsthm}
\usepackage{enumitem}
\usepackage{hyperref}
\usepackage{xcolor}
\usepackage{import}
\usepackage{xifthen}
\usepackage{pdfpages}
\usepackage{transparent}
\usepackage{listings}


\lstset{
    breaklines=true,         % Enable line wrapping
    breakatwhitespace=false, % Wrap lines even if there's no whitespace
    basicstyle=\ttfamily,    % Use monospaced font
    frame=single,            % Add a frame around the code
    columns=fullflexible,    % Better handling of variable-width fonts
}

\newcommand{\incfig}[1]{%
    \def\svgwidth{\columnwidth}
    \import{./Figures/}{#1.pdf_tex}
}
\theoremstyle{definition} % This style uses normal (non-italicized) text
\newtheorem{solution}{Solution}
\newtheorem{proposition}{Proposition}
\newtheorem{problem}{Problem}
\newtheorem{lemma}{Lemma}
\newtheorem{theorem}{Theorem}
\newtheorem{remark}{Remark}
\newtheorem{note}{Note}
\theoremstyle{plain} % Restore the default style for other theorem environments
%

% Theorem-like environments
% Title information
\title{Quiz 1 Corrections}
\author{Jerich Lee}
\date{\today}

\begin{document}

\maketitle
\begin{align}
    4z^2 - 4z i - 1 &= 4 + 4i z + i^2 z^2 \\
    4z^2 - 4z i - 1 &= 4 + 4i z - z^2 \\
    5z^2 - 8z i - 5 &= 0 \\
    z^2 - \frac{8}{5} z i - 1 &= 0 \\
    \left[z^2 - \frac{8}{5} z i + \left(\frac{8i}{10}\right)^2 \right] - 1 &= \left(\frac{8i}{10}\right)^2 \\
    \left(z - i \frac{4}{5} \right)^2 &= \left(\frac{8i}{10}\right)^2 + 1 \\
    &= \left(\frac{8i}{10}\right)^2 + \frac{25}{25} \\
    &= \frac{-16}{25} + \frac{25}{25} \\
    &= \frac{9}{25}
    \end{align}

    \paragraph{Key Point:} For a complex number $w$, the square of the modulus is 
    \[
      |w|^2 \;=\; w\,\overline{w},
    \]
    while
    \[
      w^2 \;=\; w \cdot w.
    \]
    These are not the same unless $w$ is purely real. In particular,
    \[
      |\,2z - i\,|^2 
      \;\neq\; (2z - i)^2
      \quad\text{for complex }z.
    \]
    
    \subsection*{Detailed Explanation}
    
    We start with the equation
    \[
    \bigl\lvert 2z - i \bigr\rvert 
    \;=\; 
    \bigl\lvert 2 + i\,z \bigr\rvert
    \quad\Longleftrightarrow\quad
    \bigl\lvert 2z - i \bigr\rvert^2 
    \;=\; 
    \bigl\lvert 2 + i\,z \bigr\rvert^2.
    \]
    
    \begin{enumerate}
    \item \textbf{Incorrect Move:}  
      A common mistake is to replace
      \[
        \bigl\lvert 2z - i\bigr\rvert^2
        \quad\text{by}\quad
        (2z - i)^2.
      \]
      In fact,
      \[
        \bigl\lvert 2z - i\bigr\rvert^2
        \;=\;(2z - i)\,\overline{(2z - i)}
        \;=\;(2z - i)\bigl(2\overline{z} - \overline{i}\bigr).
      \]
      Because $\overline{i} = -\,i$, the correct expression is
      \[
        \bigl\lvert 2z - i\bigr\rvert^2 
        \;=\;
        (2z - i)\bigl(2\overline{z} + i\bigr),
      \]
      not $(2z - i)^2$.
    
    \item \textbf{Correct Method via Real and Imag Parts:}
    
      Let $z = x + i\,y$, where $x,y\in\mathbb{R}$. Then:
      \[
        2z - i 
        \;=\; 2\bigl(x + i\,y\bigr) - i
        \;=\; 2x + 2i\,y - i
        \;=\; 2x \;+\; i(2y - 1).
      \]
      Hence
      \[
        \bigl\lvert 2z - i \bigr\rvert^2
        \;=\; \bigl(2x\bigr)^2 + \bigl(2y - 1\bigr)^2.
      \]
      Likewise,
      \[
        i\,z 
        \;=\; i\bigl(x + i\,y\bigr) 
        \;=\; i\,x + i^2\,y 
        \;=\; i\,x - y,
      \]
      so
      \[
        2 + i\,z 
        \;=\; 2 + (i\,x - y)
        \;=\; (2 - y) + i\,x.
      \]
      Hence
      \[
        \bigl\lvert 2 + i\,z \bigr\rvert^2
        \;=\;\bigl(2 - y\bigr)^2 + x^2.
      \]
      The original condition becomes
      \[
        4x^2 + (2y - 1)^2 
        \;=\; (2 - y)^2 + x^2.
      \]
      Expand and simplify:
      \[
        4x^2 + 4y^2 - 4y + 1 
        \;=\; x^2 + y^2 + 4 - 4y
        \;\;\Longrightarrow\;\;
        3x^2 + 3y^2 - 3 = 0
        \;\;\Longrightarrow\;\;
        x^2 + y^2 = 1.
      \]
      Therefore, the solution set in the complex plane is precisely the \emph{unit circle}:
      \[
        \bigl\{\,z : |z| = 1 \bigr\}.
      \]

    \end{enumerate}

    \section*{Solution: Points of Continuity of \texorpdfstring{$f(z)$}{f(z)}}

    We are given a function
    \[
    f(z) \;=\;
    \begin{cases}
    \dfrac{z^3 + i}{\,z - i\,}, & z \neq i,\\[6pt]
    -3, & z = i.
    \end{cases}
    \]
    
    \subsection*{Step 1: Identify the Potential Problem Point}
    
    Since the definition of $f(z)$ changes at $z=i$, the only possible discontinuity could occur there (the quotient is perfectly fine for $z \neq i$).  
    
    \subsection*{Step 2: Compute the Limit as \texorpdfstring{$z \to i$}{z -> i}}
    
    We check whether
    \[
    \lim_{z \to i} \dfrac{z^3 + i}{\,z - i\,}
    \quad
    \text{exists and equals } -3.
    \]
    
    \paragraph{Factorization.}
    Observe that $z=i$ is actually a root of $z^3 + i$, because $i^3 = -\,i$.  Hence
    \[
    i^3 + i = -i + i = 0,
    \]
    so $z-i$ divides $z^3 + i$.  By polynomial (or synthetic) division,
    \[
    z^3 + i \;=\; (z - i)\,\bigl(z^2 + i\,z - 1\bigr).
    \]
    Therefore, for $z \neq i$,
    \[
    \frac{z^3 + i}{z - i} 
    \;=\;
    z^2 + i\,z - 1.
    \]
    Taking $z \to i$ in this simplified form gives
    \[
    \lim_{z \to i} \bigl(z^2 + i\,z - 1\bigr)
    \;=\;
    i^2 + i\,\bigl(i\bigr) - 1
    \;=\;
    (-1) \;+\; (i^2) \;-\;1
    \;=\;
    -1 + (-1) - 1
    \;=\;
    -3.
    \]
    
    \subsection*{Step 3: Compare with the Value at \texorpdfstring{$z=i$}{z=i}}
    
    By definition, $f(i) = -3$.  Since the limit from above also equals $-3$, we conclude
    \[
    \lim_{z \to i} \dfrac{z^3 + i}{\,z - i\,} 
    \;=\;
    -3
    \;=\;
    f(i).
    \]
    Hence $f(z)$ is continuous at $z=i$.
    
    \subsection*{Step 4: Final Conclusion}
    
    \begin{itemize}
      \item For $z \neq i$, $f(z)$ is given by a rational function with denominator $z-i$; that is continuous everywhere except possibly at $z=i$.
      \item At $z=i$, the limit of $\tfrac{z^3+i}{z-i}$ is exactly $-3$, which matches the piecewise definition $f(i)=-3$.  
    
    Therefore, \emph{the function $f$ is continuous for all complex $z$, including $z=i$.}
    \end{itemize}

    To use the formula 

\[
\operatorname{Re}(z) = \frac{z + \bar{z}}{2}
\]

to find \( \operatorname{Re}(z^2) \), follow these steps:

\section*{Step 1: Compute \( z^2 \)}
For \( z = x + iy \), we expand:

\[
z^2 = (x + iy)^2 = x^2 - y^2 + 2ixy
\]

\section*{Step 2: Compute \( \bar{z^2} \)}
The complex conjugate of \( z^2 \) is:

\[
\bar{z^2} = \overline{x^2 - y^2 + 2ixy} = x^2 - y^2 - 2ixy
\]

\section*{Step 3: Apply the Formula}
Using the real part formula:

\[
\operatorname{Re}(z^2) = \frac{z^2 + \bar{z^2}}{2}
\]

we substitute:

\[
\operatorname{Re}(z^2) = \frac{(x^2 - y^2 + 2ixy) + (x^2 - y^2 - 2ixy)}{2}
\]

Simplifying,

\[
\operatorname{Re}(z^2) = \frac{x^2 - y^2 + x^2 - y^2}{2}
\]

\[
= \frac{2x^2 - 2y^2}{2}
\]

\[
= x^2 - y^2
\]

\section*{Conclusion}
Using the formula \( \operatorname{Re}(z) = \frac{z + \bar{z}}{2} \), we have verified that:

\[
\operatorname{Re}(z^2) = x^2 - y^2
\]

which matches our earlier result.

To evaluate

\[
\lim_{z \to i} \frac{x^2 - y^2}{z},
\]

we proceed as follows:

\section*{Step 1: Identify \((x, y)\) at \(z = i\)}
If \(z \to i\), then 
\[
z = x + i y \quad \longrightarrow \quad i = 0 + i \cdot 1,
\]
which corresponds to \(x = 0\) and \(y = 1\).

\section*{Step 2: Substitute into the function}
We compute:
\[
x^2 - y^2 = 0^2 - 1^2 = -1,
\]
and since \(z = i\),
\[
f(i) = \frac{-1}{i}.
\]

\section*{Step 3: Simplify}
Recall that \(\frac{1}{i} = -i\). Thus,
\[
\frac{-1}{i} = -1 \cdot \frac{1}{i} = -1 \cdot (-i) = i.
\]

\section*{Conclusion}
The limit is:
\[
\lim_{z \to i} \frac{x^2 - y^2}{z} = i \neq 0.
\]

A concise way to see that the limit does not exist is to write $z$ in polar form:

\begin{align}
z &= r e^{i\theta}, \quad \text{so} \quad z^2 = r^2 e^{2i\theta}.
\end{align}

Then
\begin{align}
\operatorname{Re}(z^2) &= r^2 \cos(2\theta),
\end{align}
and
\begin{align}
\frac{\operatorname{Re}(z^2)}{z^2} 
&= \frac{r^2 \cos(2\theta)}{r^2 e^{2i\theta}} \\
&= \cos(2\theta)\,e^{-2i\theta}.
\end{align}

Notice that this ratio depends only on the angle $\theta$ and \emph{not} on $r$. As $r \to \infty$, different choices of $\theta$ give different values for 
\begin{align}
\cos(2\theta)\,e^{-2i\theta}.
\end{align}

Hence there is no single value to which the expression converges in all directions in the complex plane, and the limit does not exist.
\[
\textbf{Derivation of }|\cosh(x + i y)|^2 = \sinh^2 x + \cos^2 y
\]
\begin{align*}
\cosh(x + i y) 
&= \frac{e^{x + i y} + e^{-x - i y}}{2} \\[6pt]
&= \frac{1}{2} \Bigl( e^x(\cos y + i\,\sin y) \;+\; e^{-x}(\cos y - i\,\sin y)\Bigr) \\[6pt]
&= \cosh x\,\cos y \;+\; i\,\sinh x\,\sin y.
\end{align*}

Taking the modulus squared:
\[
|\cosh(x + i y)|^2 
= \bigl(\cosh x \,\cos y\bigr)^2 + \bigl(\sinh x \,\sin y\bigr)^2 
= \cosh^2 x\,\cos^2 y + \sinh^2 x\,\sin^2 y.
\]

Using the identity \(\cosh^2 x - \sinh^2 x = 1 \Rightarrow \cosh^2 x = 1 + \sinh^2 x,\) we get
\begin{align*}
\cosh^2 x\,\cos^2 y + \sinh^2 x\,\sin^2 y
&= (1 + \sinh^2 x)\cos^2 y \;+\; \sinh^2 x\,\sin^2 y \\[4pt]
&= \cos^2 y \;+\; \sinh^2 x \bigl(\cos^2 y + \sin^2 y\bigr) \\[4pt]
&= \cos^2 y + \sinh^2 x.
\end{align*}

Hence,
\[
|\cosh(x + i y)|^2 
\;=\; \sinh^2 x \;+\; \cos^2 y,
\]
as required.
\end{document}
