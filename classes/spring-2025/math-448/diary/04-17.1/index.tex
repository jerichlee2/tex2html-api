\documentclass[12pt]{article}

% Packages
\usepackage[margin=1in]{geometry}
\usepackage{amsmath,amssymb,amsthm}
\usepackage{enumitem}
\usepackage{hyperref}
\usepackage{xcolor}
\usepackage{import}
\usepackage{xifthen}
\usepackage{pdfpages}
\usepackage{transparent}
\usepackage{listings}
\usepackage{tikz}
  \usetikzlibrary{calc,patterns,arrows.meta,decorations.markings}


\DeclareMathOperator{\Log}{Log}
\DeclareMathOperator{\Arg}{Arg}

\lstset{
    breaklines=true,         % Enable line wrapping
    breakatwhitespace=false, % Wrap lines even if there's no whitespace
    basicstyle=\ttfamily,    % Use monospaced font
    frame=single,            % Add a frame around the code
    columns=fullflexible,    % Better handling of variable-width fonts
}

\newcommand{\incfig}[1]{%
    \def\svgwidth{\columnwidth}
    \import{./Figures/}{#1.pdf_tex}
}
\theoremstyle{definition} % This style uses normal (non-italicized) text
\newtheorem{solution}{Solution}
\newtheorem{proposition}{Proposition}
\newtheorem{problem}{Problem}
\newtheorem{lemma}{Lemma}
\newtheorem{theorem}{Theorem}
\newtheorem{remark}{Remark}
\newtheorem{note}{Note}
\newtheorem{definition}{Definition}
\newtheorem{example}{Example}
\newtheorem{corollary}{Corollary}
\theoremstyle{plain} % Restore the default style for other theorem environments
%

% Theorem-like environments
% Title information
\title{}
\author{Jerich Lee}
\date{\today}

\begin{document}

\maketitle
\subsubsection*{Expanded Step 3:  Isolating the zero of $g$}

We wish to prove that there exists a radius $r>0$ such that 
\[
   0<|z-z_{0}|\le r \;\Longrightarrow\; g(z)\neq0.
\]
This follows from the \emph{factorisation theorem} (sometimes called the
Weierstrass or local‐structure theorem for analytic functions), which we now
recall and apply carefully.

\paragraph{3.1  Local factorisation near a zero.}
Because $g$ is analytic and vanishes at $z_{0}$, there exists an integer
$m\ge1$ and an analytic function $h$ with $h(z_{0})\neq0$ such that  
\[
   g(z)=(z-z_{0})^{m}\,h(z),\qquad z\in D. \tag{3.1}
\]
(The integer $m$ is the \emph{order} of the zero.)

\paragraph{3.2  Non–vanishing of the auxiliary factor $h$.}
Since $h$ is continuous and $h(z_{0})\neq0$, there exists a number 
$r_{1}>0$ for which  
\[
   |z-z_{0}|<r_{1}\;\Longrightarrow\; h(z)\neq0 . \tag{3.2}
\]

\paragraph{3.3  Choosing the isolation radius.}
Let $r:=r_{1}$.  For any $z$ satisfying $0<|z-z_{0}|<r$ we have  
$h(z)\neq0$ by \eqref{3.2} and $(z-z_{0})^{m}\neq0$ because the factor
$(z-z_{0})$ is non‑zero.  Consequently
\[
   g(z)=(z-z_{0})^{m}h(z)\neq0 . \tag{3.3}
\]
Hence $g$ has \textbf{no} other zeros in the punctured disc
$0<|z-z_{0}|<r$.  We say that the zero at $z_{0}$ is \emph{isolated}.

\paragraph{3.4  Why this matters for openness of $f(D)$.}
The isolation property \eqref{3.3} will allow us, in Step 4 of the proof,
to apply the \emph{maximum–modulus principle} (to $1/g$) or equivalently
the \emph{open mapping theorem} on the disc $|z-z_{0}|<r$ to show that
$g$ (hence $f$) sends a neighbourhood of $z_{0}$ onto a neighbourhood of
$w_{0}=f(z_{0})$.  That produces the open ball contained in $f(D)$, thereby
establishing that $f(D)$ itself is an open set.

\paragraph{Why does $g$ vanish at $z_0$?}

Recall the two choices made in Steps 1–2:

1.  We pick a point \(w_0\in f(D)\).  
2.  Because \(w_0\) lies in the image, there exists (by definition of
    the image set) a point \(z_0\in D\) with  
    \[
       f(z_0)=w_0 .
    \]

We then \emph{define} the auxiliary function

\[
   g(z):=f(z)-w_0 .
\]

Evaluating \(g\) at the specific point \(z_0\) gives

\[
   g(z_0)=f(z_0)-w_0 = w_0-w_0 = 0 .
\]

Hence \(z_0\) is a zero of \(g\); i.e.\ \(g\) \emph{vanishes} at \(z_0\).
The subsequent factorisation
\(
   g(z)=(z-z_0)^{m}h(z)
\)
is simply the standard local representation of an analytic function at an
isolated zero of (finite) order \(m\ge1\).
\end{document}
