\documentclass[12pt]{article}

% Packages
\usepackage[margin=1in]{geometry}
\usepackage{amsmath,amssymb,amsthm}
\usepackage{enumitem}
\usepackage{hyperref}
\usepackage{xcolor}
\usepackage{import}
\usepackage{xifthen}
\usepackage{pdfpages}
\usepackage{transparent}
\usepackage{listings}


\lstset{
    breaklines=true,         % Enable line wrapping
    breakatwhitespace=false, % Wrap lines even if there's no whitespace
    basicstyle=\ttfamily,    % Use monospaced font
    frame=single,            % Add a frame around the code
    columns=fullflexible,    % Better handling of variable-width fonts
}

\newcommand{\incfig}[1]{%
    \def\svgwidth{\columnwidth}
    \import{./Figures/}{#1.pdf_tex}
}
\theoremstyle{definition} % This style uses normal (non-italicized) text
\newtheorem{solution}{Solution}
\newtheorem{proposition}{Proposition}
\newtheorem{problem}{Problem}
\newtheorem{lemma}{Lemma}
\newtheorem{theorem}{Theorem}
\newtheorem{remark}{Remark}
\newtheorem{note}{Note}
\theoremstyle{plain} % Restore the default style for other theorem environments
%

% Theorem-like environments
% Title information
\title{}
\author{Jerich Lee}
\date{\today}

\begin{document}

\maketitle
\[
I \;=\; \int_{-1 - i}^{\,i} \frac{1}{z}\,dz.
\]

\section*{Step-by-Step Solution}

\subsection*{1. Parameterize the line segment}

We integrate along the straight line from 
\[
z_0 = -1 - i 
\quad\text{to}\quad 
z_1 = i.
\]
A convenient parameterization is
\[
z(t) \;=\; z_0 + t\,[\,z_1 - z_0\,], 
\quad 0 \le t \le 1.
\]
Concretely,
\[
z_1 - z_0 \;=\; i - (-1 - i) \;=\; 1 + 2i,
\]
so
\[
z(t) \;=\; (-1 - i) + t\,(1 + 2i).
\]
Then
\[
\frac{dz}{dt} \;=\; 1 + 2i,
\quad
dz \;=\; (1 + 2i)\,dt.
\]

\subsection*{2. Substituting into the integral}

Along this path, 
\[
\frac{1}{z} \;=\; \frac{1}{\,z(t)\,} 
\;=\; \frac{1}{\,(-1 - i) + t\,(1+2i)\,}.
\]
Hence,
\[
I 
\;=\;
\int_{-1-i}^{\,i} \frac{1}{z}\,dz
\;=\;
\int_{0}^{1} 
\frac{1}{z(t)} 
\,\frac{dz}{dt}
\,dt
\;=\;
\int_{0}^{1}
\frac{1 + 2i}{\,(-1 - i) + t\,(1+2i)\,}
\,dt.
\]

\subsection*{3. Simplify the integrand}

Notice that
\[
(-1 - i) + t\,(1+2i)
\;=\;
(1 + 2i)\left(\,
t + \frac{-1 - i}{\,1 + 2i\,}
\right).
\]
Thus,
\[
\frac{1 + 2i}{\,(-1 - i) + t\,(1 + 2i)\,}
\;=\;
\frac{1 + 2i}{\,(1 + 2i)\,\Bigl(\,t + \frac{-1 - i}{1 + 2i}\Bigr)}
\;=\;
\frac{1}{\,t + A\,},
\]
where we set
\[
A 
\;=\;
\frac{-1 - i}{\,1 + 2i\,}.
\]
Hence,
\[
I 
\;=\;
\int_{0}^{1}
\frac{1}{\,t + A\,}\,dt
\;=\;
\left.
\ln\bigl(t + A\bigr)
\right|_{t=0}^{t=1}
\;=\;
\ln(1 + A) \;-\; \ln(A).
\]

\subsection*{4. Compute $A$ explicitly}

First, we simplify $A$:
\[
A 
\;=\;
\frac{-1 - i}{\,1 + 2i\,}
\,\cdot\,
\frac{1 - 2i}{\,1 - 2i\,}
\;=\;
\frac{(-1 - i)\,(1 - 2i)}{\,1 + (2i)(-2i)\,}
\;=\;
\frac{(-1 - i)\,(1 - 2i)}{1 + 4}.
\]
The denominator is $5$, so
\[
(-1 - i)\,(1 - 2i)
\;=\;
-1(1 - 2i)\;-\;i(1 - 2i)
\;=\;
(-1 + 2i)\;+\;(-i + 2i^2).
\]
Recall $i^2 = -1$, so
\[
(-1 + 2i)\;+\;(-i + 2(-1))
\;=\;
-1 + 2i - i - 2
\;=\;
-3 + i.
\]
Hence,
\[
A = \frac{-3 + i}{5} 
= -\frac{3}{5} + \frac{1}{5}i
= -0.6 + 0.2\,i.
\]

\subsection*{5. Polar forms of $A$ and $1+A$}

\paragraph{(a) $A$ in polar form.}

\[
\lvert A\rvert 
\;=\;
\sqrt{(-0.6)^2 + 0.2^2} 
= \sqrt{0.36 + 0.04}
= \sqrt{0.40}
\;=\;
0.6324555\ldots
\]
The argument $\theta_A$ satisfies
\[
\theta_A = \arg(A),
\]
where $\Re(A) < 0$ and $\Im(A) > 0$, placing $A$ in the second quadrant. Numerically:
\[
\alpha = \arctan\!\bigl(\tfrac{0.2}{0.6}\bigr)
= \arctan\!\bigl(\tfrac{1}{3}\bigr) 
\approx 0.32175,
\]
so
\[
\theta_A = \pi - \alpha
\approx 3.14159 - 0.32175 
\approx 2.81984.
\]
Therefore,
\[
\ln(A)
= \ln\!\bigl(\lvert A\rvert\bigr) \;+\; i\,\theta_A.
\]

\paragraph{(b) $1 + A$ in polar form.}

Since
\[
1 + A 
= 1 + \bigl(-0.6 + 0.2\,i\bigr)
= 0.4 + 0.2\,i,
\]
we get
\[
\lvert 1 + A\rvert
= \sqrt{0.4^2 + 0.2^2}
= \sqrt{0.16 + 0.04}
= \sqrt{0.20}
= 0.4472136\ldots
\]
and
\[
\arg\bigl(1 + A\bigr) 
= \arctan\!\bigl(\tfrac{0.2}{0.4}\bigr)
= \arctan(0.5)
\approx 0.4636476.
\]
Hence,
\[
\ln(1 + A)
= \ln\!\bigl(\lvert 1 + A\rvert\bigr)
+ i\,\arg(1 + A).
\]

\subsection*{6. Difference of logarithms}

Putting it all together:
\[
\ln(1 + A) - \ln(A)
\;=\;
\bigl[\ln\lvert 1 + A\rvert - \ln\lvert A\rvert\bigr]
\;+\;
i\,\bigl[\arg(1 + A) - \arg(A)\bigr].
\]

\paragraph{(a) Real part.}

\[
\ln\lvert 1 + A\rvert 
- \ln\lvert A\rvert
\;=\;
\ln\!\Bigl(\frac{0.4472136}{0.6324555}\Bigr).
\]
One can show that this ratio equals 
\(\tfrac{1/\sqrt{5}}{( \sqrt{0.4} )}\) which simplifies to \(1/\sqrt{2}\). Thus
\[
\ln\lvert 1 + A\rvert 
- \ln\lvert A\rvert
= \ln\!\Bigl(\tfrac{1}{\sqrt{2}}\Bigr)
= -\tfrac12\,\ln(2).
\]

\paragraph{(b) Imaginary part.}

\[
\arg(1 + A) - \arg(A)
\;=\;
0.4636476 - 2.819842
\approx -2.3561944
= -\frac{3\pi}{4}.
\]
Thus the imaginary part is 
\(-\,i\,\tfrac{3\pi}{4}\).

\subsection*{7. Final value of the integral}

Combining both parts, we obtain:
\[
I
\;=\;
\ln(1 + A) \;-\; \ln(A)
\;=\;
-\;\tfrac12\,\ln(2)
\;-\;
i \,\frac{3\pi}{4}.
\]
Hence, the value of the integral is
\[
\boxed{
\int_{-1 - i}^{\,i} \frac{dz}{z}
\;=\;
-\;\tfrac12\,\ln(2)
\;-\;
i\,\frac{3\pi}{4}.
}
\]
\end{document}
