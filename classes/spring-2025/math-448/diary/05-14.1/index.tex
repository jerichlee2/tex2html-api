\documentclass[12pt]{article}

% Packages
\usepackage[margin=.5in]{geometry}
\usepackage{amsmath,amssymb,amsthm}
\usepackage{enumitem}
\usepackage{hyperref}
\usepackage{xcolor}
\usepackage{import}
\usepackage{xifthen}
\usepackage{pdfpages}
\usepackage{transparent}
\usepackage{listings}
\usepackage{tikz}
\usepackage{physics}
\usepackage{siunitx}
\usepackage{booktabs}
\usepackage{cancel}
\usepackage{mathtools} % <-- THIS defines \xmapsto

  \usetikzlibrary{calc,patterns,arrows.meta,decorations.markings}


\DeclareMathOperator{\Log}{Log}
\DeclareMathOperator{\Arg}{Arg}

\lstset{
    breaklines=true,         % Enable line wrapping
    breakatwhitespace=false, % Wrap lines even if there's no whitespace
    basicstyle=\ttfamily,    % Use monospaced font
    frame=single,            % Add a frame around the code
    columns=fullflexible,    % Better handling of variable-width fonts
}

\newcommand{\incfig}[1]{%
    \def\svgwidth{\columnwidth}
    \import{./Figures/}{#1.pdf_tex}
}
\theoremstyle{definition} % This style uses normal (non-italicized) text
\newtheorem{solution}{Solution}
\newtheorem{proposition}{Proposition}
\newtheorem{problem}{Problem}
\newtheorem{lemma}{Lemma}
\newtheorem{theorem}{Theorem}
\newtheorem{remark}{Remark}
\newtheorem{note}{Note}
\newtheorem{definition}{Definition}
\newtheorem{example}{Example}
\newtheorem{corollary}{Corollary}
\theoremstyle{plain} % Restore the default style for other theorem environments
%

% Theorem-like environments
% Title information
\title{Conformal Map Cheat Sheet}
\author{Jerich Lee}
\date{\today}

\begin{document}

\maketitle
%--------------------------------------------------------------------
%  Mapping the upper semicircle $\{|z|<1,\; \Im z>0\}$ to $\mathbb{H}$
%--------------------------------------------------------------------
\begin{solution}
  We decompose the map
  \[
     w \;=\; -\Bigl[\,(z+1)^{-1}-\tfrac12\Bigr]^{2}
  \tag{$\ast$}\label{eq:fullmap}
  \]
  into four elementary conformal steps.  At each stage we describe the
  image of the \emph{interior} domain
  \[
     \Omega := \{\,z\in\mathbb{C} : |z|<1,\;\Im z>0\,\}
     \qquad
     \bigl(\text{upper semicircle from $-1$ to $1$}\bigr)
  \]
  and the image of its \emph{boundary} (the diameter $(-1,1)$ and the arc
  $\{|z|=1,\Im z=0\}$).
  
  \medskip
  \begin{enumerate}[label=\textbf{Step~\arabic*.},wide,labelwidth=!,labelindent=0pt]
  
  %----------------------------------------------------------------
  \item \textbf{Translate left by $1$:}
        \[
           \zeta := z+1 .
        \]
        \emph{Effect.}
        \begin{itemize}
           \item Centre of the semicircle moves from $0$ to $-1$.
           \item Diameter $(-1,1)$ becomes $(0,2)$ on the real axis.
           \item The semicircle $\Omega$ becomes
                 \[
                   \Omega_1
                   = \bigl\{\zeta : |\zeta-1|<1,\;\Im\zeta>0\bigr\},
                 \]
                 still a semicircle but now centred at $\zeta=1$.
        \end{itemize}
  
  %----------------------------------------------------------------
  \item \textbf{Invert in the unit circle:}
        \[
           \eta := \frac{1}{\zeta}\;=\;\frac{1}{\,z+1\,}.
        \]
        \emph{Effect.}
        \begin{itemize}
           \item The circle $|\zeta|=1$ (passing through $0$ and $2$)
                 maps to itself.
           \item Points inside that circle map outside and vice‑versa.
                 Consequently the semicircle $\Omega_1$ maps to the
                 \emph{right half–plane}
                 \[
                    \Omega_2 = \{\,\eta : \mathbb{R}e\eta> \tfrac12\,\}.
                 \]
           \item The diameter $(0,2)$ becomes the vertical line
                 $\mathbb{R}e\eta=\tfrac12$.
        \end{itemize}
  
  %----------------------------------------------------------------
  \item \textbf{Translate so the boundary line becomes $\mathbb{R}e\xi=0$:}
        \[
           \xi := \eta - \tfrac12.
        \]
        \emph{Effect.}
        \[
           \Omega_3
           = \{\,\xi : \mathbb{R}e\xi>0\,\}
           \quad\text{(open right half–plane).}
        \]
        The boundary $\mathbb{R}e\eta=\tfrac12$ slides to $\mathbb{R}e\xi=0$.
  
  %----------------------------------------------------------------
  \item \textbf{Square and rotate:}
        \[
           w := -\xi^{\,2}.
        \]
        \emph{Effect.}
        \begin{itemize}
          \item Squaring maps the right half–plane conformally onto
                $\mathbb{C}\setminus(-\infty,0]$ (a slit plane whose boundary is
                the non‑positive real axis).
          \item Multiplying by $-1$ rotates this slit plane by $\pi$,
                sending it to the \emph{upper half–plane}
                \[
                   \Omega_4
                   = \mathbb{H}
                   = \{\,w\in\mathbb{C} : \Im w>0\,\}.
                \]
          \item The boundary line $\mathbb{R}e\xi=0$ (imaginary axis) maps onto
                the real axis under $w=-\xi^{2}$.
        \end{itemize}
  \end{enumerate}
  
  \bigskip
  \textbf{Conclusion.}
  The composition of these steps is exactly the map
  \eqref{eq:fullmap}, and it carries the upper semicircle $\Omega$
  \emph{bijectively and conformally} onto the upper half–plane
  $\mathbb{H}$, while the semicircle’s boundary goes to $\mathbb{R}$.
  \end{solution}
  %--------------------------------------------------------------------
%  Mapping the vertical strip  $\{-1<\mathbb{R}e z<1,\; \Im z>0\}$
%  onto the upper semicircle  $\{|w|<1,\; \Im w>0\}$
%--------------------------------------------------------------------
\begin{solution}
  We factor the map
  \[
     w \;=\; i\,e^{\frac{\pi i z}{2}}
  \tag{$\ast$}\label{eq:stripmap}
  \]
  into two elementary steps and track the image of the
  \emph{half–strip}
  \[
     S \;:=\;\{\,z\in\mathbb{C} : -1<\mathbb{R}e z<1,\; \Im z>0\,\}.
  \]
  
  \begin{enumerate}[label=\textbf{Step~\arabic*.},wide,labelwidth=!,labelindent=0pt]
  
  %----------------------------------------------------------------
  \item \textbf{Exponential map with scaling:}
        \[
           \zeta := e^{\frac{\pi i z}{2}}.
        \]
  
        Write $z=x+iy$ with $-1<x<1$ and $y>0$.
        Then
        \[
           \zeta
           = e^{\frac{\pi i}{2}(x+iy)}
           = e^{-\frac{\pi y}{2}}\,
             e^{\,i\frac{\pi x}{2}}.
        \]
  
        \emph{Geometry.}
        \begin{itemize}
          \item \emph{Magnitude:} $|\zeta| = e^{-\pi y/2}<1$  
                (since $y>0$).  Hence the image lies \emph{inside} the
                unit circle.
          \item \emph{Argument:} $\arg\zeta = (\pi x/2)\in(-\tfrac{\pi}{2},
                \tfrac{\pi}{2})$.  
                Thus the image lies in the \emph{right half} of the unit disc.
        \end{itemize}
  
        Consequently
        \[
          \zeta(S)
          = \bigl\{\zeta : |\zeta|<1,\; \mathbb{R}e\zeta>0\bigr\}
          \quad
          \text{(open right half of the unit disc).}
        \]
  
  %----------------------------------------------------------------
  \item \textbf{Rotation by \boldmath$+\tfrac{\pi}{2}$:}
        \[
           w := i\,\zeta.
        \]
  
        Multiplication by $i$ rotates every point
        counter‑clockwise by $90^{\circ}$:
        \[
           \mathbb{R}e\zeta>0
           \quad\Longrightarrow\quad
           \Im(i\zeta)>0.
        \]
        The radius $|\zeta|$ is unchanged, so
        \[
           w(S)
           = \bigl\{\,w : |w|<1,\; \Im w>0\,\bigr\},
        \]
        exactly the \emph{upper semicircle} of the unit disc—the domain
        used as the starting point in \textbf{Case 1}.
  
  \end{enumerate}
  
  \bigskip
  \textbf{Conclusion.}
  The composition \eqref{eq:stripmap} sends the vertical half–strip $S$
  \emph{conformally and bijectively} onto the upper semicircle
  $\{|w|<1,\Im w>0\}$.
  From Case 1, that semicircle is mapped further to the upper half–plane
  $\mathbb{H}$, completing the reduction.
  
  \end{solution}
  %--------------------------------------------------------------------
%  Mapping  $\displaystyle U:=\bigl\{z\in\mathbb{C} : \Im z>0\bigr\}\setminus[0,i]$
%          onto the upper half–plane  $\mathbb{H}=\{\,w:\Im w>0\,\}$
%--------------------------------------------------------------------
\begin{solution}
  We factor the map
  \[
     w \;=\; \bigl(z^{2}+1\bigr)^{1/2}
  \tag{$\ast$}\label{eq:full}
  \]
  into three elementary conformal steps and track the image of both the
  \emph{interior} and the \emph{boundary}.
  
  \medskip
  \begin{enumerate}[label=\textbf{Step~\arabic*.},wide,labelwidth=!,labelindent=0pt]
  
  %----------------------------------------------------------------
  \item \textbf{Square:}
        \[
           \zeta := z^{2}.
        \]
  
        \emph{Effect on the domain.}
        \begin{itemize}
          \item The upper half–plane $\Im z>0$ folds across the imaginary
                axis; the image is still the upper half–plane
                \[
                   \Im\zeta>0.
                \]
          \item The removed segment $[0,i]$ (imaginary axis) maps to the
                \emph{negative real interval} $[-1,0]$ via
                $z=it\;(0\le t\le1)\;\mapsto\;\zeta=-t^{2}$.
          \item Hence
                \[
                   U_1
                   :=\zeta(U)
                   = \bigl\{\Im\zeta>0\bigr\}\setminus[-1,0].
                \]
                Geometrically: upper half–plane minus a horizontal slit on
                the \emph{negative} real axis.
        \end{itemize}
  
  %----------------------------------------------------------------
  \item \textbf{Translate right by \(1\):}
        \[
           \xi := \zeta + 1.
        \]
  
        \emph{Effect.}
        \begin{itemize}
          \item The slit $[-1,0]$ is shifted to the \emph{positive}
                interval $[0,1]$.
          \item The half–plane $\Im\zeta>0$ remains $\Im\xi>0$.
          \item Thus
                \[
                   U_2
                   = \bigl\{\Im\xi>0\bigr\}\setminus[0,1].
                \]
                We now have an upper half–plane with a slit on the
                positive real axis.
        \end{itemize}
  
  %----------------------------------------------------------------
  \item \textbf{Principal square root (branch cut along \([0,\infty)\)):}
        \[
           w := \sqrt{\xi}.
        \]
  
        \emph{Effect.}
        \begin{itemize}
          \item The function $\sqrt{\bullet}$ (principal branch) is
                conformal on the slit plane
                $\mathbb{C}\setminus[0,\infty)$ and maps it onto
                the upper half–plane $\mathbb{H}$.
          \item Because $U_2$ lies entirely in that slit plane, its image
                is
                \[
                   w(U)
                   = \mathbb{H}.
                \]
          \item The boundary slit \([0,1]\) maps to the segment
                \([0,1]\) on the \emph{real} axis, which forms part of
                the boundary of $\mathbb{H}$.
        \end{itemize}
  
  \end{enumerate}
  
  \bigskip
  \textbf{Conclusion.}
  The composition \eqref{eq:full}—square, translate by \(1\),
  then take the principal square root—carries the upper half–plane
  with the vertical slit \([0,i]\) removed \emph{bijectively and
  conformally} onto the pure upper half–plane~\(\mathbb{H}\).
  
  \end{solution}
  %--------------------------------------------------------------------
%  Mapping  $\displaystyle 
%            D\;=\;\mathbb{C}\setminus\bigl((-\infty,-1]\cup[1,\infty)\bigr)$
%  onto the upper half–plane  $\mathbb{H}=\{\,w\in\mathbb{C}:\Im w>0\,\}$
%--------------------------------------------------------------------
\begin{solution}
  The composite map
  \[
     w
     \;=\;
     \sqrt{2}\,i\,
     \Bigl[(z+1)^{-1}-\tfrac12\Bigr]^{1/2}
  \tag{$\ast$}\label{eq:full}
  \]
  is dissected into four elementary conformal steps.
  At every stage we describe the image of the domain and the behaviour of
  the two removed rays.
  
  \bigskip
  \begin{enumerate}[label=\textbf{Step~\arabic*.},wide,labelwidth=!,labelindent=0pt]
  
  %----------------------------------------------------------------
  \item \textbf{Translate right by $1$:}
        \[
           \zeta:=z+1.
        \]
  
        \emph{Effect on the deleted rays.}
        \[
           (-\infty,-1]\longmapsto(-\infty,0], 
           \qquad
           [1,\infty)\longmapsto[2,\infty).
        \]
        So
        \[
           D_1
           \;=\;
           \mathbb{C}\setminus\bigl((-\infty,0]\cup[2,\infty)\bigr).
        \]
  
  %----------------------------------------------------------------
  \item \textbf{Invert in the unit circle:}
        \[
           \eta:=\frac{1}{\zeta}.
        \]
  
        \emph{Effect.}
        \begin{itemize}
          \item For $\zeta\le 0$ (negative real axis) we get
                $\eta\le 0$; thus $(-\infty,0]$ becomes
                $(-\infty,0]$ again, but with \emph{finite} endpoint
                (the point $\zeta=0$ goes to $\eta=-\infty$).
          \item For $\zeta\ge 2$ we have $0<\eta\le\tfrac12$; hence
                $[2,\infty)$ is mapped to $\bigl(0,\tfrac12\bigr]$.
        \end{itemize}
        The two images now lie \emph{on the same real axis}:
        \[
           D_2
           =\mathbb{C}\setminus\bigl((-\infty,0]\cup[0,\tfrac12]\bigr)
           =\mathbb{C}\setminus(-\infty,\tfrac12].
        \]
        Geometrically this is the plane slit along the real axis from
        $-\infty$ to $\tfrac12$.
  
  %----------------------------------------------------------------
  \item \textbf{Translate left by $\tfrac12$:}
        \[
           \xi:=\eta-\tfrac12.
        \]
  
        \emph{Effect.}
        The slit moves to the \emph{negative} real axis:
        \[
           (-\infty,\tfrac12]\;\longmapsto\;(-\infty,0],
           \qquad
           D_3=\mathbb{C}\setminus(-\infty,0].
        \]
        This is the standard branch–cut plane for the principal square
        root.
  
  %----------------------------------------------------------------
  \item \textbf{Principal square root and rotation:}
        \[
           w:=\sqrt{2}\,i\,\sqrt{\xi}.
        \]
  
        \emph{Effect.}
        \begin{itemize}
          \item $\sqrt{\xi}$ (principal branch) maps 
                $\mathbb{C}\setminus(-\infty,0]$ \emph{conformally} onto the
                \emph{right half–plane} $\{\,u:\Re u>0\,\}$, because the
                argument range $(-\pi,\pi)$ of $\xi$ is halved.
          \item Multiplication by $i$ rotates the right half–plane
                counter‑clockwise by $90^{\circ}$, giving the upper
                half–plane $\mathbb{H}$.
          \item The additional real factor $\sqrt{2}$ merely scales
                distances and has no effect on the image domain.
        \end{itemize}
        Hence
        \[
           w(D)=\mathbb{H}.
        \]
  
  \end{enumerate}
  
  \bigskip
  \textbf{Conclusion.}
  The composition \eqref{eq:full} sends the doubly–slit plane
  $D=\mathbb{C}\setminus\bigl((-\infty,-1]\cup[1,\infty)\bigr)$ 
  \emph{bijectively and conformally} onto the upper half–plane
  $\mathbb{H}$, while the two original rays are taken to the real axis,
  the boundary of~$\mathbb{H}$.
  
  \end{solution}
  \begin{solution}
    Below we build, \emph{piece-by-piece}, a conformal map that sends  
    
    \[
    \boxed{\; 
       \mathcal D
         = \bigl\{\,z\in\mathbb{C} : |z|<1,\; \Im z>0 \bigr\}
           \setminus
           \bigl\{\,it : 0\le t\le \tfrac12 \bigr\}
       \;}
    \]
    
    (the upper half of the unit disc with the vertical slit
    $[0,i/2]$ removed) onto the \emph{standard upper half-plane}
    $\mathcal H=\{w\in\mathbb{C}:\Im w>0\}$.
    The composite map consists of four elementary steps:
    
    \[
    F(z)
      \;=\;
      \underbrace{\sqrt{\,
            e^{i\vartheta}\,
            \frac{\displaystyle\Psi(z)-\alpha}
                 {\displaystyle\Psi(z)-\overline{\alpha}}
          }}_{\text{Steps \,2--4}}
      ,\qquad
      \Psi(z)
         = i\,\frac{1+z}{1-z}\, .
    \]
    
    %-------------------------------
    \begin{enumerate}[label=\textbf{Step \arabic*:}, itemsep=1.4ex]
    
    %---------------------------------
    \item \textbf{Cayley transform to the upper half-plane.}\;
    Set
    \[
    w_1 \;=\; \Psi(z) \;=\; i\,\frac{1+z}{1-z}\, .
    \]
    The classical Cayley map carries the \emph{entire} unit disc
    conformally onto $\mathcal H$.
    Because $\partial\mathcal D$ is a subset of $\partial\Bbb D$,
    the upper semicircular arc $|z|=1$ ($0<\arg z<\pi$) goes to the
    \emph{real} axis,
    whereas the interval $[-1,1]$ on the real axis maps to the
    positive imaginary axis.
    Hence $w_1$ already lands in $\mathcal H$, but the image of the slit
    \[
    S=\{it:0\le t\le\tfrac12\}
    \]
    becomes the circular arc
    \(
    \Gamma_1 := \Psi(S)
    \)
    joining the two interior points
    \[
    w_A = \Psi(0)=i, 
    \qquad
    w_B = \Psi\!\bigl(\tfrac{i}{2}\bigr)
          = i\,\frac{1+\tfrac{i}{2}}{1-\tfrac{i}{2}}
          =: \alpha
          \;.
    \]
    
    %---------------------------------
    \item \textbf{Straighten the arc into a ray.}\;
    Because Möbius transformations map (generalised) circles to
    circles or lines, we can send the arc $\Gamma_1$ to a straight ray.
    Define
    \[
    w_2
       \;=\;
       \frac{\,w_1-\alpha\,}{\,w_1-\overline{\alpha}\,},
    \]
    where
    $\displaystyle
       \alpha = i\,\frac{1+\tfrac{i}{2}}{1-\tfrac{i}{2}}
              \approx -0.471 + 0.882\,i
    $.
    The coefficients are complex conjugates, so
    $w_2$ maps $\mathcal H$ to itself.
    Moreover
    \[
    w_2(\alpha)=0,
    \quad
    w_2(\overline{\alpha})=\infty,
    \]
    so the circular arc $\Gamma_1$ is now the \emph{positive real} ray
    $[0,\infty)$.
    
    %---------------------------------
    \item \textbf{Rotate so the slit lies on the imaginary axis.}\;
    At this point the slit is the real ray;
    we want it on the \emph{imaginary} axis so that the next square-root
    will act as a two-to-one “unwinding’’ map.
    Set
    \[
    w_3
       \;=\;
       e^{i\vartheta}\,w_2,
       \quad
       \vartheta
         := \frac{\pi}{2}-\arg\bigl(w_2(i)\bigr),
    \]
    so that $w_3(i)$ is pure imaginary and
    \(
       w_3\bigl([0,\infty)\bigr) = [0,\,\infty)\,i
    \)
    (the positive imaginary axis).
    Again $\vartheta$ is a real constant, hence $w_3$
    keeps $\mathcal H$ invariant.
    
    %---------------------------------
    \item \textbf{Square-root to the full half-plane.}\;
    Finally apply the principal square root
    \[
    w \;=\; w_4
       = \sqrt{\,w_3\,},
       \quad 0<\arg w_3<\pi,
    \]
    whose branch cut is the \emph{non-positive} imaginary axis.
    Because the slit of $w_3$ sits on the \emph{positive} imaginary axis,
    taking the square root maps
    $\mathcal H\setminus[0,i\infty)$ onto the \emph{whole}
    upper half-plane~$\mathcal H$.
    
    \end{enumerate}
    
    %---------------------------------
    \noindent
    \textbf{Result.}\;
    The composite map
    \[
    \boxed{
       F(z)
         = \sqrt{\,e^{i\vartheta}\,
                  \frac{\Psi(z)-\alpha}{\Psi(z)-\overline{\alpha}}
                },
       \qquad
       \Psi(z)=i\,\frac{1+z}{1-z}
     }
    \]
    is conformal on $\mathcal D$ and furnishes a bijection
    \(
       F:\mathcal D \to \mathcal H.
    \)
    
    \medskip
    \noindent
    \textbf{Why this helps.}\;
    Under $F$ the original integrand
    \[
    f(z)
       = i\,\frac65\!
         \left(\frac{1}{z^{2}+1}-\frac12\right)
    \]
    can be rewritten in the $w$-variable and integrated over a
    \emph{straight} contour in~$\mathcal H$
    (the ``Case~3’’ integral in your notes), avoiding the delicate slit
    geometry in the $z$-plane.
    \end{solution}
    \begin{solution}
      We build the map  
      \[
      F(z)\;=\;i\bigl(e^{\pi z}-1\bigr)^{\tfrac12}
      \]
      as the composition of four elementary conformal maps.  Throughout,
      the \emph{principal branch} of every multi–valued function is used.
      
      \medskip
      %-------------------------------
      \textbf{Domain to be mapped.}
      \[
      \boxed{
         \mathcal S
         =\bigl\{\,z\in\mathbb{C} : -1<\Im z<1\bigr\}
          \setminus (-\infty,0)
       }
      \]
      is the horizontal strip of height~$2$, with the \emph{negative real
      axis} removed.
      
      \medskip
      %-------------------------------
      \begin{enumerate}[label=\textbf{Step \arabic*:}, itemsep=1.4ex]
      
      %---------------------------------
      \item  \textbf{Exponentiation to a slit plane.}  
      \[
      u \;=\; e^{\pi z}.
      \]
      Write $z=x+iy$; then  
      $u=e^{\pi x}\,e^{i\pi y}$ with
      \(
      -\pi<\arg u<\pi.
      \)
      Hence
      \[
      \mathcal S
      \;\xrightarrow{\,e^{\pi z}\,}\;
      \mathcal D_1
        = \mathbb{C}\setminus(-\infty,0],
      \]
      the plane cut along the \emph{non–positive} real axis.  
      The strip’s excluded set $(-\infty,0)$ (real, $y=0$) is mapped to the
      interval $(0,1)$ on the \emph{positive} real axis, so that
      \(
      \mathcal D_1
      =\mathbb{C}\setminus(-\infty,1].
      \)
      
      %---------------------------------
      \item \textbf{Translate the cut so it starts at the origin.}  
      \[
      v \;=\; u-1.
      \]
      This sends the point $u=1$ to the origin, converting the slit
      $(-\infty,1]$ into
      \[
      (-\infty,0].
      \]
      Thus
      \[
      \mathcal D_2
        = \mathbb{C}\setminus(-\infty,0],
      \]
      a plane cut exactly along the \emph{negative} real axis.
      
      %---------------------------------
      \item \textbf{Principal square root to the right half–plane.}  
      \[
      w_0 \;=\; \sqrt{v},
      \qquad
      -\pi<\arg v<\pi.
      \]
      Because $\sqrt{\;}$ halves arguments,
      \(
      -\tfrac{\pi}{2}<\arg w_0<\tfrac{\pi}{2},
      \)
      so
      \[
      \mathcal D_2
      \;\xrightarrow{\,\sqrt{\;}\,}\;
      \mathcal D_3
        = \{w_0\in\mathbb{C} : \Re w_0>0\},
      \]
      the \emph{right half–plane}.  
      (The negative real cut becomes the imaginary axis,
      which is the branch cut of the square root.)
      
      %---------------------------------
      \item \textbf{Quarter-turn into the upper half–plane.}  
      \[
      w \;=\; i\,w_0.
      \]
      Multiplying by $i$ rotates the right half–plane
      counter-clockwise by~$90^{\circ}$:
      \[
      \mathcal D_3
      \;\xrightarrow{\,i\cdot\,}\;
      \boxed{
         \mathcal H
         = \{w\in\mathbb{C} : \Im w>0\}
       }.
      \]
      
      \end{enumerate}
      
      \medskip
      %-------------------------------
      \textbf{Summary.}  
      The composite map
      \[
      \boxed{\;
         F(z)=i\bigl(e^{\pi z}-1\bigr)^{\tfrac12}
       \;}
      \]
      is analytic and one–to–one on the strip
      $\mathcal S$, and carries $\mathcal S$ \emph{onto} the standard
      upper half–plane~$\mathcal H$.
      \end{solution}
      \begin{solution}
        We construct, \emph{piece–by–piece}, a conformal map that sends the
        upper part of the ellipse passing through $-1$ and $1$
        (with boundary tangents meeting the real axis at the common angle
        $\alpha>0$) onto the \emph{upper half–plane}
        \[
        \mathcal H \;=\;\{\,w\in\mathbb{C} : \Im w>0\}.
        \]
        The composite map required in your notes is  
        
        \[
        \boxed{\;
           w \;=\;-\Bigl(\tfrac{2}{z+1}-1\Bigr)^{\pi/2}
         \;},
        \]
        and we now explain why it works.
        
        \bigskip
        \noindent
        \textbf{Starting domain.}
        \[
        \mathcal E
          \;=\;
          \Bigl\{\,z\in\mathbb{C} :
             \text{$z$ lies \emph{inside} the ellipse
             meeting the real axis at $-1$ and $1$,}
             \;
             \arg\bigl(\tfrac{dz}{dx}\bigr)\big|_{x=\pm1}=\pm\alpha,
             \;
             \Im z>0
          \Bigr\}.
        \]
        At each endpoint $\pm1$ the interior of $\mathcal E$
        meets the real axis in a wedge of opening
        angle~$\alpha$ (see the figure below).
        Consequently, \emph{after} we remove the two interior points,
        the boundary touching the real axis forms a total interior angle
        of~$2\alpha$ at~$z=1$ and a straight angle ($\pi$) at $z=-1$.
        
        \medskip
        \begin{enumerate}[label=\textbf{Step \arabic*:}, itemsep=1.4ex]
        
        %------------------------------------------------------------
        \item  \textbf{Translate the left endpoint to the origin.}  
        \[
        \zeta
           \;=\;z+1.
        \]
        Now $\zeta=0$ corresponds to $z=-1$, while the right endpoint
        moves to $\zeta=2$.
        The strip $\bigl\{\Im z=0,\, -1<x<1\bigr\}$ is carried to the segment
        $\bigl\{\Im\zeta=0,\, 0<\Re\zeta<2\bigr\}$,
        and the interior angle at $\zeta=2$ is still $2\alpha$.
        
        %------------------------------------------------------------
        \item  \textbf{Invert and scale so that the \emph{finite} endpoint
              becomes the origin.}  
        \[
        \eta
           \;=\;
           \frac{2}{\zeta}.
        \]
        This Möbius map sends
        \(
           \zeta=0\longmapsto\eta=\infty,
        \quad
           \zeta=2\longmapsto\eta=1,
        \)
        so the finite endpoint lands at~$1$.
        Because inversion reverses orientation, the arc of $\mathcal E$
        is taken to a smooth curve meeting the unit
        circle at $\eta=1$ with the \emph{same} interior angle~$2\alpha$.
        
        %------------------------------------------------------------
        \item  \textbf{Shift the unit point to the origin.}  
        \[
        \mu
           \;=\;
           \eta-1
           \;=\;\frac{2}{z+1}-1.
        \]
        Now the endpoint that had angle~$2\alpha$ lies at $\mu=0$,
        while the other endpoint is pushed to~$\mu=-1$.
        In a neighbourhood of $\mu=0$
        the domain is a wedge of opening angle~$2\alpha$.
        
        %------------------------------------------------------------
        \item  \textbf{Raise to a power to open the wedge to a half–plane.}  
        Let
        \[
        \nu
           \;=\;
           \mu^{\,\pi/(2\alpha)}.
        \]
        In general, if $\mu$ subtends an angle $2\alpha$ at the origin,
        then $\nu$ subtends
        \(
           \bigl(\tfrac{\pi}{2\alpha}\bigr)\!\cdot 2\alpha=\pi,
        \)
        i.e.\ a straight angle.
        Hence $\nu$ maps the wedge \emph{conformally} onto the right
        half–plane $\Re\nu>0$.
        In your particular formula the power has already been
        \emph{specialised} to $\tfrac{\pi}{2}$,
        so it is implicit that the ellipse meets the real axis at
        \[
        \alpha
           \;=\;1\;{\rm rad}\approx57.3^{\circ},
        \]
        for which $2\alpha=2$ and
        \(
           (2)\!\cdot(\tfrac{\pi}{2})=\pi.
        \)
        
        %------------------------------------------------------------
        \item  \textbf{Quarter–turn into the \emph{upper} half–plane.}  
        \[
        w
           \;=\;
           -\,\nu
           \;=\;
           -\Bigl(\tfrac{2}{z+1}-1\Bigr)^{\pi/2}.
        \]
        Multiplying by $-1$ (a $180^{\circ}$ rotation) sends the right
        half–plane onto the upper half–plane
        $\mathcal H=\{w:\Im w>0\}$.
        \end{enumerate}
        
        \bigskip
        \noindent
        \textbf{Conclusion.}\;
        The composite map
        \[
        \boxed{\;
          w \;=\;
          -\bigl[\tfrac{2}{z+1}-1\bigr]^{\,\pi/2}
         \;}
        \]
        is analytic and one–to–one on the upper half of the ellipse
        segment~$\mathcal E$, and it
        carries $\mathcal E$ conformally onto the standard
        upper half–plane~$\mathcal H$.
        \end{solution}
        \begin{solution}
          We explain why the composite map  
          
          \[
          \boxed{\;
             w
             \;=\;
             i\Bigl(
                  \frac{3}{\,4\bigl(\dfrac{2}{z+1}-1\bigr)^{2}-1}-1
                \Bigr)^{\tfrac12}
           \;}
          \]
          
          sends the domain  
          
          \[
          \mathcal D_0
            \;=\;
            \{z\in\mathbb{C}:\;|z|<1\}
            \;\setminus\;
            \Bigl(
               (-\infty,0]\cup [\tfrac13,1)
            \Bigr)
          \]
          
          (the open unit disc with two slits: the
          \emph{negative} real axis and the segment $(\tfrac13,1)$ on the real
          axis) conformally onto the \emph{upper half–plane}
          $\displaystyle\mathcal H=\{w\in\mathbb{C}:\Im w>0\}$.
          
          Throughout we use the \emph{principal branch}
          of every multi-valued function (argument in $(-\pi,\pi)$).
          
          \medskip
          \begin{enumerate}[label=\textbf{Step \arabic*:}, itemsep=1.4ex]
          
          %------------------------------------------------------------
          \item  \textbf{Translate the disc so that $-1$ moves to the origin.}
          \[
          \zeta = z+1.
          \]
          Now $\zeta( -1)=0$ and the slits become
          \[
          \zeta\bigl((-\infty,0]\bigr)=(-1,1],
          \qquad
          \zeta\bigl([\tfrac13,1)\bigr)=[\tfrac43,2).
          \]
          Hence
          \[
          \mathcal D_1
            = \bigl\{\zeta:|\zeta-1|<1\bigr\}
              \setminus\bigl(( -1,1]\cup [\tfrac43,2)\bigr).
          \]
          
          %------------------------------------------------------------
          \item  \textbf{Invert and scale so the finite slit–endpoint $\,\zeta=2$ goes to $0$.}
          \[
          \eta=\frac{2}{\zeta}.
          \]
          This Möbius map interchanges $0\leftrightarrow\infty$ and sends
          \[
          \zeta=2\;\longmapsto\;\eta=1,
          \qquad
          \zeta=-1\;\longmapsto\;\eta=-2.
          \]
          The image
          \(
             \mathcal D_2
             =\mathbb{C}\setminus\bigl(( -\infty,-2]\cup [1,\infty)\bigr)
          \)
          is the plane with two symmetric \emph{horizontal} slits.
          
          %------------------------------------------------------------
          \item  \textbf{Translate so the right-hand slit starts at the origin.}
          \[
          \mu=\eta-1
            =\frac{2}{z+1}-1.
          \]
          Now the slits are
          \(
             (-\infty,-3] \cup [0,\infty).
          \)
          Thus $\mu=0$ corresponds to $z=\tfrac13$; the interval $[0,\infty)$
          is the former $(\tfrac13,1)$-slit.
          
          %------------------------------------------------------------
          \item  \textbf{Open the symmetric slits into a single line.}
          Square, then scale and shift:
          \[
          \sigma
             =4\mu^{2}-1.
          \]
          Because $\mu\mapsto\mu^{2}$ doubles arguments,
          the two rays $(-\infty,-3]$ and $[0,\infty)$ are glued into the
          \emph{negative} real axis $(-\infty,-1]$.
          Multiplying by $4$ and subtracting $1$ merely rescales the cut,
          so
          \[
          \mathcal D_3
            = \mathbb{C}\setminus(-\infty,-1].
          \]
          
          %------------------------------------------------------------
          \item  \textbf{Invert and translate so the cut lies on $(0,\infty)$.}
          \[
          \rho=\frac{3}{\sigma}-1
              =\frac{3}{4\mu^{2}-1}-1.
          \]
          The transformation $u\mapsto\frac{3}{u}$ swaps $0\leftrightarrow\infty$
          and keeps the ray $(-\infty,-1]$ on the real axis; the shift $-1$
          moves it to the \emph{positive} ray $(0,\infty)$.
          Hence
          \[
          \mathcal D_4
            = \mathbb{C}\setminus(0,\infty).
          \]
          
          %------------------------------------------------------------
          \item \textbf{Take the principal square root.}
          \[
          \eta=\sqrt{\rho},
          \qquad -\pi<\arg\rho<\pi.
          \]
          Since the branch cut of $\sqrt{\;}$ is \emph{the negative real axis},
          the slit $(0,\infty)$ is carried onto the
          \emph{imaginary} axis, giving the right half–plane
          \[
          \mathcal D_5=\{\eta\in\mathbb{C}:\Re\eta>0\}.
          \]
          
          %------------------------------------------------------------
          \item  \textbf{Quarter-turn into the upper half-plane.}
          \[
          w=i\eta.
          \]
          Multiplication by $i$ rotates the right half–plane
          counter-clockwise through $90^{\circ}$, producing
          \[
          \boxed{\;
             w\in\mathcal H=\{w:\Im w>0\}
           }.
          \]
          
          \end{enumerate}
          
          \medskip
          \noindent
          \textbf{Conclusion.}\;
          The chain of conformal maps  
          
          \[
          z\xmapsto{\;1\;}\zeta
          \xmapsto{\;2\;}\eta
          \xmapsto{\;3\;}\mu
          \xmapsto{\;4\;}\sigma
          \xmapsto{\;5\;}\rho
          \xmapsto{\;6\;}\eta
          \xmapsto{\;7\;}w
          \]
          
          is precisely the single formula  
          
          \[
          w
            = i\Bigl(
                  \tfrac{3}{4\bigl(\tfrac{2}{z+1}-1\bigr)^{2}-1}-1
                \Bigr)^{\tfrac12},
          \]
          which is analytic and one-to-one on $\mathcal D_0$
          and maps $\mathcal D_0$ conformally onto the
          standard upper half-plane~$\mathcal H$.
          \end{solution}
\end{document}
