\documentclass[12pt]{article}

% Packages
\usepackage[margin=1in]{geometry}
\usepackage{amsmath,amssymb,amsthm}
\usepackage{enumitem}
\usepackage{hyperref}
\usepackage{xcolor}
\usepackage{import}
\usepackage{xifthen}
\usepackage{pdfpages}
\usepackage{transparent}
\usepackage{listings}
\usepackage{tikz}
\usepackage{physics}
\usepackage{siunitx}
  \usetikzlibrary{calc,patterns,arrows.meta,decorations.markings}


\DeclareMathOperator{\Log}{Log}
\DeclareMathOperator{\Arg}{Arg}

\lstset{
    breaklines=true,         % Enable line wrapping
    breakatwhitespace=false, % Wrap lines even if there's no whitespace
    basicstyle=\ttfamily,    % Use monospaced font
    frame=single,            % Add a frame around the code
    columns=fullflexible,    % Better handling of variable-width fonts
}

\newcommand{\incfig}[1]{%
    \def\svgwidth{\columnwidth}
    \import{./Figures/}{#1.pdf_tex}
}
\theoremstyle{definition} % This style uses normal (non-italicized) text
\newtheorem{solution}{Solution}
\newtheorem{proposition}{Proposition}
\newtheorem{problem}{Problem}
\newtheorem{lemma}{Lemma}
\newtheorem{theorem}{Theorem}
\newtheorem{remark}{Remark}
\newtheorem{note}{Note}
\newtheorem{definition}{Definition}
\newtheorem{example}{Example}
\newtheorem{corollary}{Corollary}
\theoremstyle{plain} % Restore the default style for other theorem environments
%

% Theorem-like environments
% Title information
\title{}
\author{Jerich Lee}
\date{\today}

\begin{document}

\maketitle
\section*{1.  Preliminary facts about $f$}

\paragraph{1.1\;No rational roots.}
By the Rational–Root Theorem any rational zero must divide $1$, i.e. it could only be
$\pm1$.  But $f(\pm1)=1+4+1=6\neq0$, hence there are \emph{no} rational roots.

\paragraph{1.2\;Irreducibility over $\mathbb{Q}$.}
Shift $x\mapsto y+1$:
\[
f(y+1)=y^{8}+8y^{7}+28y^{6}+60y^{5}+90y^{4}+96y^{3}+68y^{2}+28y+6.
\]
All coefficients except the leading one are divisible by $2$, while the constant
term $6$ is divisible by $2$ but not by $4$.  
Hence, by Eisenstein’s criterion with the prime $p=2$, $f(y+1)$ (and thus $f$)
is irreducible in $\mathbb{Q}[x]$.

\paragraph{1.3\;Real zeros.}
\begin{itemize}
  \item $f(x)=x^{8}+4x^{5}+1>0$ for every $x>0$, so there are no positive real zeros.
  \item Define $g(x):=f(-x)=x^{8}-4x^{5}+1$.
        Descartes’ rule of signs gives two or zero sign changes,
        hence either $0$ or $2$ negative real zeros.
        Because $g(1)=-2<0$ and $g(0)=1>0$, one
        zero lies in $(-1,0)$.  
        Also $g(2)=256-128+1=129>0$ so the second zero lies in $(-2,-1)$.
        Therefore $f$ has \emph{exactly two} simple negative real zeros and six
        complex (non-real) ones that occur in conjugate pairs.
\end{itemize}
\section*{2.  Converting the degree 8 equation to a quartic}

Divide by $x^{4}$:
\[
x^{4}+x^{-4}+4x=0.
\tag{$\star$}
\]

Introduce the classical reciprocal–variable substitution
\[
y:=x+\frac1x \qquad\Longrightarrow\qquad
\begin{cases}
x^{2}+x^{-2}=y^{2}-2,\\[6pt]
x^{4}+x^{-4}=(y^{2}-2)^{2}-2=y^{4}-4y^{2}+2.
\end{cases}
\]

Equation $(\star)$ becomes  
\[
y^{4}-4y^{2}+2+4x=0.
\tag{$\ast$}
\]

But $x$ itself is determined by $y$ via the quadratic
\[
x^{2}-yx+1=0
\quad\Longrightarrow\quad
x=\dfrac{y\pm\sqrt{\,y^{2}-4\,}}{2}.
\]

Substituting this expression for $x$ in $(\ast)$ and squaring once removes the
radical and produces a \emph{quartic in $y$}.  Solving that quartic with
Ferrari’s method gives up to four values $y_{1},\dots,y_{4}$, and each $y_{k}$
feeds the simple quadratic above to deliver the two corresponding $x$–values.
Altogether one collects the expected $4\times2=8$ roots.
The explicit radicals are lengthy (several pages) but algorithmically
straight-forward.

*Why bother?*  
Because every step (reciprocal substitution $\to$ quartic $\to$ quadratic)
is solvable by radicals, this shows \(f\) is solvable in radicals even though it
is degree 8 and irreducible.

solve 
\begin{align}
  \int_{-\infty}^{\infty} \frac{1-\cos(3x)}{z^{2}} \,\mathrm{d}x 
\end{align}
\begin{proof}[Contour–integration method]
  \begin{enumerate}[label=\textup{(\roman*)},nosep]
  
  %-----------------------------------------------
  \item \textbf{Replace $\cos(3x)$ by an exponential.}
  
  Write
  \[
      \cos(3x)=\tfrac12\!\bigl(e^{3ix}+e^{-3ix}\bigr),
  \]
  so that it is enough to compute the real part of
  \begin{align}
      J
      =\operatorname*{PV}\!\int_{-\infty}^{\infty}
        \frac{1-e^{3ix}}{x^{2}}\,dx .
  \end{align}
  (The imaginary part, proportional to $\sin(3x)/x^{2}$, is odd and
  hence integrates to $0$.)
  
  %-----------------------------------------------
  \item \textbf{Choose a keyhole contour.}
  
  For $R>\varepsilon>0$ let $\mathcal C_{R,\varepsilon}$ be the positively
  oriented path consisting of  
  \[
      \Gamma_1 : [-R,-\varepsilon],\quad
      C_\varepsilon : \varepsilon e^{i\theta},\; \theta\in[0,\pi],\quad
      \Gamma_2 : [\varepsilon,R],\quad
      \Gamma_R : R e^{i\theta},\; \theta\in[0,\pi].
  \]
  
  %-----------------------------------------------
  \item \textbf{Identify the singularity at $z=0$.}
  
  Set
  \[
      f(z)=\frac{1-e^{3iz}}{z^{2}},\qquad z\neq0.
  \]
  A Taylor expansion gives  
  \(
      1-e^{3iz}=-3iz+\tfrac92 z^{2}+O(z^{3}),
  \)
  hence
  \[
      f(z)=-\frac{3i}{z}+\tfrac92+O(z)
  \quad\Longrightarrow\quad
      \operatorname*{Res}_{z=0}f(z)=-3i .
  \]
  
  %-----------------------------------------------
  \item \textbf{The big arc $\Gamma_R$ vanishes.}
  
  On $\Gamma_R$ we have $|z|=R$ and $\Re(iz)\le0$, so
  $|1-e^{3iz}|\le2$.  Thus
  \[
      |f(z)|\le \frac{2}{R^{2}},\qquad
      \bigl|\!\int_{\Gamma_R}f(z)\,dz\bigr|\le \frac{2\pi}{R}\xrightarrow{R\to\infty}0 .
  \]
  
  %-----------------------------------------------
  \item \textbf{The small arc $C_\varepsilon$ contributes $3\pi$.}
  
  Parametrise $z=\varepsilon e^{i\theta}$ ($0\le\theta\le\pi$):
  \begin{align}
      \int_{C_\varepsilon}\!\!f(z)\,dz
      &=\int_{0}^{\pi}\!\!
        \Bigl(-\frac{3i}{\varepsilon e^{i\theta}}+O(1)\Bigr)
        \bigl(i\varepsilon e^{i\theta}\bigr)\,d\theta
        \;=\;
        \int_{0}^{\pi} 3\,d\theta + O(\varepsilon)
        \xrightarrow{\varepsilon\to0^{+}} 3\pi .
  \end{align}
  
  %-----------------------------------------------
  \item \textbf{Apply the residue theorem.}
  
  Because $\mathcal C_{R,\varepsilon}$ encloses only $z=0$,
  \[
      \int_{\mathcal C_{R,\varepsilon}}\!\!f(z)\,dz
      = 2\pi i\,\bigl(-3i\bigr)=6\pi .
  \]
  Letting $R\to\infty$ and then $\varepsilon\to0^{+}$ gives
  \begin{align}
      \operatorname*{PV}\!\int_{-\infty}^{\infty}
      \frac{1-e^{3ix}}{x^{2}}\,dx
      \;+\; 0
      \;+\; 3\pi
      \;=\;
      6\pi
      \quad\Longrightarrow\quad
      \operatorname*{PV}\!\int_{-\infty}^{\infty}
      \frac{1-e^{3ix}}{x^{2}}\,dx = 3\pi .
  \end{align}
  
  %-----------------------------------------------
  \item \textbf{Take the real part.}
  
  Finally,
  \[
      \int_{-\infty}^{\infty}\frac{1-\cos(3x)}{x^{2}}\,dx
      =\Re(J)=\boxed{\,3\pi\,}.
  \]
  \end{enumerate}
  \end{proof}

prove that if $\left\vert f(z) \right\vert \leq M, \left\vert z \right\vert =1$, then 
$\left\vert f(z) \right\vert \leq \left\vert z \right\vert ^{n} M $ when $z\geq 1$. Hint: apply the maximum principle to $w^{n}f(\frac{1}{w}), w \leq 1$.

\begin{problem}
  Let \(f\) be analytic in the exterior of the unit disk together with the point
  at \(\infty\); assume that \(\displaystyle\lim_{z\to\infty}z^{-n}f(z)=0\)
  (i.e.\ \(f\) has a pole of order at most \(n\) at~\(\infty\)).
  If
  \[
        |f(z)|\;\le M ,\qquad |z|=1 ,
  \]
  prove that
  \[
        |f(z)|\;\le\;|z|^{\,n}\,M ,\qquad |z|\ge1 .
  \]
  (Hint: apply the maximum‐modulus principle to
  \(g(w)=w^{n}f(1/w)\) on the unit disk.)
  \end{problem}
  
  \begin{proof}
  \textbf{1.  Define an interior function.}
  Put
  \[
      g(w)\;=\;w^{n}\,f\!\bigl(\tfrac1w\bigr),\qquad |w|\le1 .
  \]
  Because \(f\) is analytic for \(|z|\ge1\) and \(z=\tfrac1w\),
  the only potential singularity of \(g\) is at \(w=0\).
  But the hypothesis
  \(\lim_{z\to\infty}z^{-n}f(z)=0\) is exactly the statement
  \(\lim_{w\to0}g(w)=0\); hence \(w=0\) is a \emph{removable}
  singularity and \(g\) extends to an analytic function on the
  \emph{closed} unit disk \(\overline{\mathbb D}\).
  
  \medskip
  \textbf{2.  Estimate \(g\) on the boundary.}
  For \(|w|=1\) we have \(z=\tfrac1w\) with \(|z|=1\), so
  \[
        |g(w)|
        \;=\;
        |w|^{n}\,|f(1/w)|
        \;=\;
        |f(z)|
        \;\le\;
        M .
  \]
  Thus \(\lvert g\rvert\le M\) on the boundary circle
  \(|w|=1\).
  
  \medskip
  \textbf{3.  Apply the maximum‐modulus principle.}
  Since \(g\) is analytic on \(\overline{\mathbb D}\),
  the maximum–modulus principle gives, for every \(|w|\le1\),
  \[
        |g(w)|\;\le\;\max_{|\,\zeta|=1}|g(\zeta)|
        \;\le\;M .
  \]
  
  \medskip
  \textbf{4.  Translate the bound back to \(f\).}
  Let \(z\) satisfy \(|z|\ge1\) and set \(w=\tfrac1z\); then
  \(|w|\le1\) and
  \[
        |f(z)|
        \;=\;
        |f(1/w)|
        \;=\;
        |w|^{-n}\,|g(w)|
        \;\le\;
        |w|^{-n}\,M
        \;=\;
        |z|^{\,n}\,M .
  \]
  Hence the desired estimate \( |f(z)| \le |z|^{n}M \) holds for
  every \(|z|\ge1\).
  \end{proof}

prove that if $f_n$ converges uniformly to $f$, then there exist an zero of $f_n$ 

\begin{problem}
  Let \(D\subset\mathbb C\) be a domain and
  \(\{f_n\}_{n\ge1}\subset\mathcal O(D)\) a sequence of analytic
  functions that converges \emph{locally uniformly} (i.e.\ uniformly on every
  compact \(K\Subset D\)) to an analytic limit function \(f\not\equiv0\).
  Show that if \(f\) has a zero at some point \(z_0\in D\), then every
  sufficiently large \(n\) possesses at least one zero \(z_n\in D\);
  indeed \(z_n\to z_0\) as \(n\to\infty\).
  \end{problem}
  
  \begin{proof}
  \textbf{1.  Reduce to a small closed disk.}
  Because \(f\) is analytic and non-constant, its zeros are isolated.
  Pick
  \[
        0<r<\operatorname{dist}\bigl(z_0,\partial D\bigr)
  \]
  so small that the closed disk
  \(\overline{\Delta}:=\{z:|z-z_0|\le r\}\) contains \emph{no other}
  zeros of \(f\).  
  Set
  \[
        m \;=\; \min_{|z-z_0|=r}|f(z)|  \;>\;0
  \]
  (the minimum exists by continuity on the compact circle).
  
  \medskip
  \textbf{2.  Uniform convergence on the circle.}
  Because the convergence \(f_n\to f\) is uniform on
  \(\partial\Delta=\{\,|z-z_0|=r\,\}\), there is an index
  \(N\) such that
  \[
        |f_n(z)-f(z)| \;<\; m ,\qquad
        \forall z\in\partial\Delta,\; n\ge N.
  \]
  
  \medskip
  \textbf{3.  Apply Rouché’s theorem (or Hurwitz’s theorem).}
  For \(n\ge N\) and \(z\in\partial\Delta\) we have
  \[
        |f_n(z)| \;\ge\;
        |f(z)|-|f_n(z)-f(z)|
        \;>\;
        m-m=0 ,
  \]
  so neither \(f\) nor \(f_n\) vanishes on the circle
  \(\partial\Delta\).
  Moreover
  \[
        |f(z)-f_n(z)| \;<\; |f(z)|
        \quad\text{on }\partial\Delta ,
  \]
  hence \(f\) and \(f_n\) have the same number of zeros
  (counting multiplicities) inside \(\Delta\) by Rouché’s theorem.
  But \(f\) has exactly one zero there (namely \(z_0\)),
  so \(f_n\) must have \emph{at least one} zero \(z_n\in\Delta\).
  
  \medskip
  \textbf{4.  Convergence of the zeros.}
  Because \(|z_n-z_0|\le r\) for all \(n\ge N\),
  the sequence \(\{z_n\}\) is bounded.  If \(z_{n_k}\) is a subsequence,
  take a further convergent subsequence (by compactness of
  \(\overline\Delta\)) with limit \(z_\infty\in\overline\Delta\).
  The continuity of \(f\) and the fact that
  \(f_{n_k}(z_{n_k})=0\) imply
  \(f(z_\infty)=\lim_{k\to\infty}f_{n_k}(z_{n_k})=0\).
  Since \(z_0\) is the \emph{unique} zero of \(f\) in \(\Delta\),
  we must have \(z_\infty=z_0\).  Hence \(\,z_n\to z_0\).
  
  \medskip
  \textbf{5.  Conclusion.}
  Thus for every sufficiently large \(n\) the function \(f_n\)
  possesses a zero, and these zeros converge to the zero \(z_0\)
  of the limit function \(f\).
  \end{proof}
  
  \begin{remark}
  The argument above is a standard corollary of
  \textbf{Hurwitz’s theorem}:  
  if analytic functions \(g_n\) converge locally uniformly to
  a non-constant analytic function \(g\) and
  \(g_n\) never vanish on \(D\), then either \(g\) is
  identically zero or \(g\) never vanishes.
  Applying the contrapositive to \(g_n=f_n\) and \(g=f\) shows that
  the presence of a genuine zero of \(f\) forces every
  sufficiently large \(f_n\) to vanish as well.
  \end{remark}

%------------------------------------------------------------------------------
%  How to decide \emph{without numerics} how many (and what kind of) zeros
%  the polynomial 
%
%        \(f(z)=z^{8}+4z^{5}+iz+i+1\)            \hfill\((\ast)\)
%
%  possesses in the open half–plane  \(\Im z>0\), and what their orders are.
%------------------------------------------------------------------------------
\section*{1.  Counting the zeros: Argument Principle}

\medskip
\textbf{Step 1.  Choose a contour.}
Fix \(R>1\) and set  
\[
   \Gamma=\bigl[-R,R\bigr]\;\cup\;
           \bigl\{\,Re^{it}:0\le t\le\pi\bigr\},
\]
i.e.\ the real segment \([-R,R]\) followed by the upper semicircle
\(\{Re^{it}\}\).  
Orient \(\Gamma\) counter-clockwise.

\medskip
\textbf{Step 2.  The change of argument on the semicircle.}
When \(z=Re^{it}\) with \(0\le t\le\pi\) and \(R\gg1\),
\(|z^{8}|\gg|4z^{5}+iz+i+1|\); hence
\(f(z)=z^{8}\bigl(1+o(1)\bigr)\) and
\[
   \Delta_{\text{semi}}\!\arg f
   =8\Delta\arg z
   =8\pi .
\]

\medskip
\textbf{Step 3.  The change of argument on the real axis.}
Write \(x\in[-R,R]\subset\Bbb R\).  Along that segment
\[
   f(x)=P(x)+iQ(x),\quad
   P(x)=x^{8}+4x^{5}+1,\;
   Q(x)=x+1 .
\]
\emph{Only one} crossing of the real axis occurs, namely at
\(x=-1\) where \(Q(-1)=0\) and \(P(-1)=-2<0\).
Tracing the orientation of the curve
\((P(x),Q(x))\) shows that the argument of \(f(x)\)
\emph{decreases} by \(2\pi\) while $x$ runs from $-R$ to $R$:
\[
   \Delta_{\text{real}}\!\arg f=-2\pi .
\]

\medskip
\textbf{Step 4.  Argument Principle.}
The total change of argument around \(\Gamma\) is therefore
\[
   \Delta_{\Gamma}\!\arg f
   \;=\;\Delta_{\text{real}}+\Delta_{\text{semi}}
   =-2\pi+8\pi
   =6\pi .
\]
By the Argument Principle this equals \(2\pi N\), where
\(N\) is the number of zeros (counted with multiplicity) of \(f\)
inside the contour, i.e.\ in the half–plane \(\Im z>0\).
Hence
\[
   N=\frac{6\pi}{2\pi}=3 .
\]

\paragraph{Conclusion.}
\[
   \boxed{\text{$f$ possesses exactly \emph{three} zeros in }\Im z>0.}
\]

%------------------------------------------------------------------------------
\section*{2.  Their multiplicity is \(1\)}

A zero \(z_{0}\) of a polynomial is multiple
iff it is a common zero of the polynomial and its derivative.
Compute
\[
   f'(z)=8z^{7}+20z^{4}+i.
\]

\medskip
\textbf{Suppose} \(f(z_{0})=0=f'(z_{0})\).
The second equation gives
\(\displaystyle i=-8z_{0}^{7}-20z_{0}^{4}\).
Eliminate \(i\) from \(f(z_{0})=0\):
\[
   -7z_{0}^{8}-8z_{0}^{7}-16z_{0}^{5}-20z_{0}^{4}+1=0.
   \tag{$\dagger$}
\]
But \((\dagger)\) is a \emph{real-coefficient} polynomial,
so any non-real root would appear with its conjugate.
Yet substituting \(z_{0}\) back into
\(i=-8z_{0}^{7}-20z_{0}^{4}\) forces the right-hand side
to be \emph{exactly} \(i\), which is impossible for the conjugate
\(\overline{z_{0}}\).
Therefore the system has \emph{no solution}, hence
\(f\) and \(f'\) share \emph{no common zero}.  
All zeros of \(f\) are consequently \textbf{simple}.

\[
   \boxed{\text{Each of those three zeros has order }1.}
\]

%------------------------------------------------------------------------------
\section*{3.  What we have shown}

* There are precisely \(3\) (simple) zeros of \(f\) in the half-plane
  \(\Im z>0\).
* No numerical approximations were needed—only the  
  Argument Principle and the fact that
  \(f\) and \(f'\) are coprime.
  \subsection*{Step 3. Change of argument on the real axis}

On the interval \(x\in[-R,R]\) write
\[
  f(x)=P(x)+i\,Q(x),
  \qquad
  P(x)=x^8+4x^5+1,
  \quad
  Q(x)=x+1.
\]
Then
\[
  Q(x)=x+1
  =\begin{cases}
     <0,& x<-1,\\
     =0,& x=-1,\\
     >0,& x>-1,
   \end{cases}
\]
so the path \(\gamma(x)=f(x)\) travels in the lower half–plane for \(x<-1\),
crosses the real axis at
\(\gamma(-1)=f(-1)=-2\), and then travels in the upper half–plane for \(x>-1\).

\medskip
As \(x\to -R\) (with \(R\gg1\)), 
\[
  f(-R)\approx(-R)^8 - iR
  \quad\text{lies in the 4th quadrant,}
  \quad
  \arg f(-R)\approx 0^-.
\]
As \(x\to R\),
\[
  f(R)\approx R^8 + iR
  \quad\text{lies in the 1st quadrant,}
  \quad
  \arg f(R)\approx 0^+.
\]
Because the path crosses the negative real axis once and winds around
the origin in the \emph{clockwise} sense, it accumulates a total change
of argument
\[
  \Delta_{\mathrm{real}} \!\arg f
  \;=\;\arg f(R)\;-\;\arg f(-R)
     \;-\;2\pi
  \;=\;-2\pi.
\]
\end{document}
