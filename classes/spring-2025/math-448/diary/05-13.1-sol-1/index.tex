\documentclass[12pt]{article}

% Packages
\usepackage[margin=1in]{geometry}
\usepackage{amsmath,amssymb,amsthm}
\usepackage{enumitem}
\usepackage{hyperref}
\usepackage{xcolor}
\usepackage{import}
\usepackage{xifthen}
\usepackage{pdfpages}
\usepackage{transparent}
\usepackage{listings}
\usepackage{tikz}
\usepackage{physics}
\usepackage{siunitx}
\usepackage{booktabs}
\usepackage{cancel}
  \usetikzlibrary{calc,patterns,arrows.meta,decorations.markings}


\DeclareMathOperator{\Log}{Log}
\DeclareMathOperator{\Arg}{Arg}

\lstset{
    breaklines=true,         % Enable line wrapping
    breakatwhitespace=false, % Wrap lines even if there's no whitespace
    basicstyle=\ttfamily,    % Use monospaced font
    frame=single,            % Add a frame around the code
    columns=fullflexible,    % Better handling of variable-width fonts
}

\newcommand{\incfig}[1]{%
    \def\svgwidth{\columnwidth}
    \import{./Figures/}{#1.pdf_tex}
}
\theoremstyle{definition} % This style uses normal (non-italicized) text
\newtheorem{solution}{Solution}
\newtheorem{proposition}{Proposition}
\newtheorem{problem}{Problem}
\newtheorem{lemma}{Lemma}
\newtheorem{theorem}{Theorem}
\newtheorem{remark}{Remark}
\newtheorem{note}{Note}
\newtheorem{definition}{Definition}
\newtheorem{example}{Example}
\newtheorem{corollary}{Corollary}
\theoremstyle{plain} % Restore the default style for other theorem environments
%

% Theorem-like environments
% Title information
\title{MATH-448 Practice Final Exam 1 Solution}
\author{Jerich Lee}
\date{\today}

\begin{document}

\maketitle
\begin{solution}
  Let 
  \[
    z = \sqrt{3}+i .
  \]
  We wish to find \emph{all} values of \(z^{15/2}\).
  
  \subsection*{1.  Polar form of \(z\)}
  \[
    |z| = \sqrt{(\sqrt{3})^{2}+1^{2}} = \sqrt{3+1}=2,
    \qquad
    \arg z = \arctan\!\Bigl(\tfrac{1}{\sqrt{3}}\Bigr)=\frac{\pi}{6}.
  \]
  Hence
  \[
    z = 2\,e^{\,i(\pi/6+2\pi k)}, 
    \qquad k\in\mathbb{Z}.
  \]
  
  \subsection*{2.  Principal logarithm (multi‑valued)}
  \[
    \Log z=\ln 2 + i\bigl(\tfrac{\pi}{6}+2\pi k\bigr), 
    \qquad k\in\mathbb{Z}.
  \]
  
  \subsection*{3.  Raising to the power \(15/2\)}
  \[
    z^{15/2} 
    = \exp\!\Bigl(\tfrac{15}{2}\,\Log z\Bigr)
    = 2^{15/2}\,
      \exp\!\Bigl(i\tfrac{15}{2}\bigl(\tfrac{\pi}{6}+2\pi k\bigr)\Bigr),
    \qquad k\in\mathbb{Z}.
  \]
  
  
  \subsection*{4.  Distinct values}
  Because the exponent \(\tfrac{15}{2}\) is in lowest terms, there are exactly
  \(2\) distinct values, corresponding to \(k=0,1\):
  
  \[
  \begin{aligned}
  k &= 0: &
  \phi_0 &= \tfrac{15}{2}\cdot\tfrac{\pi}{6}= \frac{5\pi}{4},
  \\[4pt]
  k &= 1: &
  \phi_1 &= \tfrac{15}{2}\!\Bigl(\tfrac{\pi}{6}+2\pi\Bigr)
         =\tfrac{15}{2}\cdot\tfrac{13\pi}{6}
         =\frac{65\pi}{4}
         \equiv \frac{\pi}{4}\pmod{2\pi}.
  \end{aligned}
  \]
  \subsection*{Why “lowest terms’’ \(\displaystyle\frac{15}{2}\) $\;\Longrightarrow\;$ exactly \(2\) values}

Let \(z=re^{i\theta}\) with \(r>0\) and \(\theta\in\mathbb{R}\).
For any rational exponent \(\dfrac{p}{q}\) written in \emph{lowest terms}
(\(\gcd(p,q)=1\)),
the multi–valued power is  
\[
   z^{p/q}=r^{\,p/q}\;
            \exp\!\Bigl(i\tfrac{p}{q}(\theta+2\pi k)\Bigr),
   \qquad k\in\mathbb{Z}.
\]

\paragraph{Step 1:  When do two choices of \(k\) give the same value?}
Pick \(k_1,k_2\in\mathbb{Z}\).  
The two values coincide iff their arguments differ by an integer multiple of
\(2\pi\):
\[
  \tfrac{p}{q}\bigl(\theta+2\pi k_1\bigr)
  \;-\;
  \tfrac{p}{q}\bigl(\theta+2\pi k_2\bigr)
  \;=\;
  2\pi m
  \quad(m\in\mathbb{Z}).
\]
After cancelling \(\theta\) this becomes
\[
   \frac{2\pi p}{q}\,(k_1-k_2)=2\pi m
   \;\;\Longrightarrow\;\;
   p\,(k_1-k_2)=q\,m.
\]

\paragraph{Step 2:  Use \(\gcd(p,q)=1\).}
Because \(p\) and \(q\) are coprime, the only way for
\(p\,(k_1-k_2)\) to be divisible by \(q\) is that
\(\boxed{k_1-k_2\equiv 0\pmod{q}}\).
Hence increasing \(k\) by \(q\) (or any multiple of \(q\)) returns to the
same value, while \(k,k+1,\dots,k+q-1\) generate \emph{different} values.

\paragraph{Step 3:  Apply to \(\dfrac{15}{2}\).}
Here \(p=15,\;q=2,\;\gcd(15,2)=1\).
Therefore  
\[
   k\mapsto k+2
\]
is the first (smallest) shift in \(k\) that reproduces the same value, so the
set \(\{k=0,1\}\) already exhausts all distinct values.

\[
\boxed{\text{Exactly \(q=2\) distinct values.}}
\]

\paragraph{Mnemonic.}
For an exponent \(\dfrac{p}{q}\) in lowest terms,  
\emph{the denominator \(q\) counts the number of branches} of the
multi‑valued function \(z^{p/q}\).
  
  \subsection*{5.  Numerical evaluation}
  \[
    2^{15/2}=2^{7+\tfrac12}=128\sqrt{2}.
  \]
  
  \[
  \begin{aligned}
  z_0 &= 128\sqrt{2}\,
        \bigl(\cos\tfrac{5\pi}{4}+i\sin\tfrac{5\pi}{4}\bigr)
       =128\sqrt{2}\, \Bigl(-\tfrac{\sqrt{2}}{2}-i\tfrac{\sqrt{2}}{2}\Bigr)
       =-128-128\,i,
  \\[6pt]
  z_1 &= 128\sqrt{2}\,
        \bigl(\cos\tfrac{\pi}{4}+i\sin\tfrac{\pi}{4}\bigr)
       =128\sqrt{2}\, \Bigl(\tfrac{\sqrt{2}}{2}+i\tfrac{\sqrt{2}}{2}\Bigr)
       =128+128\,i.
  \end{aligned}
  \]
  
  \[
  \boxed{(\sqrt{3}+i)^{15/2}\in\{-128-128\,i,\;128+128\,i\}.}
  \]
  \end{solution}
  \begin{solution}
    \textbf{Goal.}  Find every value of the fourth root
    \[
          w^4=-8-8i\sqrt3,
          \qquad w\in\mathbb{C},
    \]
    and write the answers in the form $a+ib$ with real $a,b$.
    
    \bigskip
    %%%%%%%%%%%%%%%%%%%%%%%%%%%%%%%%%%%%%%%%%%%%%%%%%%%%%%%%%%%%%%%%%%%%%%%%%%%%%%%%
    \textbf{Step 1.  Express the radicand in polar form.}
    
    \vspace{-6pt}
    \begin{align*}
    z&=-8-8i\sqrt3\\
    |z|&=\sqrt{(-8)^2+\bigl(-8\sqrt3\bigr)^2}=\sqrt{64+192}=16,\\
    \arg(z)&=\pi+\arctan\!\Bigl(\tfrac{\sqrt3}{1}\Bigr)=\pi+\frac{\pi}{3}=\frac{4\pi}{3}
              \quad\bigl(\text{in the 3rd quadrant}\bigr).
    \end{align*}
    
    Hence
    \[
       z=16\,e^{\,i(4\pi/3+2\pi k)},\qquad k\in\mathbb{Z}.
    \]
    
    \bigskip
    %%%%%%%%%%%%%%%%%%%%%%%%%%%%%%%%%%%%%%%%%%%%%%%%%%%%%%%%%%%%%%%%%%%%%%%%%%%%%%%%
    \textbf{Step 2.  Take the exponent $1/4$ (with $p=1,\;q=4$ in lowest terms).}
    
    Using the multi‑valued power formula,
    \[
       z^{1/4}=16^{\,1/4}\;
                \exp\!\Bigl(i\tfrac14\bigl(\tfrac{4\pi}{3}+2\pi k\bigr)\Bigr),
                \qquad k\in\mathbb{Z}.
    \]
    
    \[
       16^{1/4}=2,\qquad
       \tfrac14\!\bigl(\tfrac{4\pi}{3}\bigr)=\tfrac{\pi}{3},\qquad
       \tfrac14(2\pi k)=\tfrac{\pi}{2}k.
    \]
    
    Thus
    \[
       w_k
       =2\,e^{\,i(\pi/3+\pi k/2)},\qquad k=0,1,2,3.
    \]
    
    \emph{Why only $k=0,1,2,3$?}  
    Because $q=4$ (the denominator of $1/4$ in lowest terms), increasing $k$
    by $4$ returns to the same value: $k\mapsto k+4$ is the first repeat.
    
    \bigskip
    %%%%%%%%%%%%%%%%%%%%%%%%%%%%%%%%%%%%%%%%%%%%%%%%%%%%%%%%%%%%%%%%%%%%%%%%%%%%%%%%
    \textbf{Step 3.  Convert each $w_k$ to $a+ib$.}
    
    \[
    \begin{aligned}
    k=0:\;w_0&=2\!\left(\cos\frac{\pi}{3}+i\sin\frac{\pi}{3}\right)
             =2\!\left(\frac12+i\frac{\sqrt3}{2}\right)
             =\boxed{1+i\sqrt3},\\[6pt]
    k=1:\;w_1&=2\!\left(\cos\frac{5\pi}{6}+i\sin\frac{5\pi}{6}\right)
             =2\!\left(-\frac{\sqrt3}{2}+i\frac12\right)
             =\boxed{-\sqrt3+i},\\[6pt]
    k=2:\;w_2&=2\!\left(\cos\frac{4\pi}{3}+i\sin\frac{4\pi}{3}\right)
             =2\!\left(-\frac12-i\frac{\sqrt3}{2}\right)
             =\boxed{-1-i\sqrt3},\\[6pt]
    k=3:\;w_3&=2\!\left(\cos\frac{11\pi}{6}+i\sin\frac{11\pi}{6}\right)
             =2\!\left(\frac{\sqrt3}{2}-i\frac12\right)
             =\boxed{\sqrt3-i}.
    \end{aligned}
    \]
    
    \bigskip
    %%%%%%%%%%%%%%%%%%%%%%%%%%%%%%%%%%%%%%%%%%%%%%%%%%%%%%%%%%%%%%%%%%%%%%%%%%%%%%%%
    \textbf{Answer.}  All fourth‑roots of $-8-8i\sqrt3$ are
    \[
    \boxed{\;
       1+i\sqrt3,\;
       -\sqrt3+i,\;
       -1-i\sqrt3,\;
       \sqrt3-i
    \;}.
    \]
    
    \bigskip
    \textbf{Branch‑count check.}  
    Because the exponent is $1/4$ with denominator $q=4$ in lowest terms,
    the theory guarantees \emph{exactly $4$ distinct values}, matching the list above.
    \end{solution}
    \begin{solution}
      \textbf{Principal branch of the complex logarithm.}
      For any non‑zero \(z\in\mathbb{C}\),
      \[
         \Log z \;=\; \ln|z| \;+\; i\,\Arg(z),
      \]
      where
      \[
         \Arg(z)\in(-\pi,\pi]
      \]
      is the principal argument.
      
      \bigskip
      %%%%%%%%%%%%%%%%%%%%%%%%%%%%%%%%%%%%%%%%%%%%%%%%%%%%%%%%%%%%%%%%%%%%%%%%%%%%%%%%
      \textbf{Step 1.  Polar form of \(1-i\).}
      
      \[
         1-i \;=\; \sqrt2\,e^{\,i(-\pi/4)},
      \]
      because
      \[
         |\,1-i\,|=\sqrt{1^2+(-1)^2}=\sqrt2,
         \qquad
         \Arg(1-i)=-\frac{\pi}{4}\in(-\pi,\pi].
      \]
      
      \bigskip
      %%%%%%%%%%%%%%%%%%%%%%%%%%%%%%%%%%%%%%%%%%%%%%%%%%%%%%%%%%%%%%%%%%%%%%%%%%%%%%%%
      \textbf{Step 2.  Apply the formula.}
      
      \[
         \Log(1-i)
         \;=\;
         \ln\!\bigl(\sqrt2\bigr)
         + i\!\left(-\frac{\pi}{4}\right)
         \;=\;
         \frac{1}{2}\ln 2 \;-\; i\,\frac{\pi}{4}.
      \]
      
      \[
      \boxed{\displaystyle
         \Log(1-i)=\frac{\ln 2}{2}-\frac{\pi i}{4}}
      \]
      
      \medskip
      \textit{Only this value lies on the principal branch; all other logarithmic
      values differ by integer multiples of \(2\pi i\).}
      \end{solution}
      \begin{solution}
        \textbf{Claim.}\; For every \(z\in\mathcal{C}\setminus\{\pm i\}\) one has
        \[
           \arctan z \;=\;\frac{1}{2i}\,
           \Log\!\Bigl(\frac{1+iz}{\,1-iz\,}\Bigr),
        \]
        where \(\Log\) denotes the principal branch of the logarithm
        (\(\Arg\in(-\pi,\pi]\)).
        
        \bigskip
        %%%%%%%%%%%%%%%%%%%%%%%%%%%%%%%%%%%%%%%%%%%%%%%%%%%%%%%%%%%%%%%%%%%%%%%%%%%%%%%%
        \textbf{1.  Exponential representation of \(\tan w\).}
        
        For any \(w\in\mathcal{C}\) the tangent can be written in terms of exponentials:
        \[
           \tan w
           \;=\;
           \frac{\sin w}{\cos w}
           \;=\;
           \frac{e^{iw}-e^{-iw}}{i\bigl(e^{iw}+e^{-iw}\bigr)}
           \;=\;
           \frac{e^{2iw}-1}{\,i\bigl(e^{2iw}+1\bigr)}.
        \tag{1}
        \]
        
        \bigskip
        %%%%%%%%%%%%%%%%%%%%%%%%%%%%%%%%%%%%%%%%%%%%%%%%%%%%%%%%%%%%%%%%%%%%%%%%%%%%%%%%
        \textbf{2.  Solve \(\tan w=z\) for \(e^{2iw}\).}
        
        Set \(w=\arctan z\), so \(\tan w=z\).  
        Using (1),
        \[
           z
           =\frac{e^{2iw}-1}{\,i\bigl(e^{2iw}+1\bigr)}
           \;\Longrightarrow\;
           i z\bigl(e^{2iw}+1\bigr)=e^{2iw}-1.
        \]
        Rearrange:
        \[
           e^{2iw}(1-iz)=1+iz
           \;\Longrightarrow\;
           e^{2iw}
             =\frac{1+iz}{1-iz},
             \qquad z\neq\pm i.
        \tag{2}
        \]
        
        \bigskip
        %%%%%%%%%%%%%%%%%%%%%%%%%%%%%%%%%%%%%%%%%%%%%%%%%%%%%%%%%%%%%%%%%%%%%%%%%%%%%%%%
        \textbf{3.  Take the principal logarithm.}
        
        Applying the principal logarithm to (2):
        \[
           2iw
           =\Log\!\Bigl(\tfrac{1+iz}{\,1-iz\,}\Bigr),
        \]
        hence
        \[
           w
           =\frac{1}{2i}\,
             \Log\!\Bigl(\tfrac{1+iz}{\,1-iz\,}\Bigr).
        \tag{3}
        \]
        
        \bigskip
        %%%%%%%%%%%%%%%%%%%%%%%%%%%%%%%%%%%%%%%%%%%%%%%%%%%%%%%%%%%%%%%%%%%%%%%%%%%%%%%%
        \textbf{4.  Identify \(w\) with \(\arctan z\).}
        
        Equation (3) expresses the same \(w\) which, by definition, satisfies
        \(\tan w=z\) and lies on the principal branch of \(\arctan\).  
        Therefore
        \[
           \boxed{\displaystyle
              \arctan z
              \;=\;
              \frac{1}{2i}\,
              \Log\!\Bigl(\frac{1+iz}{\,1-iz\,}\Bigr)}
              \qquad(z\in\mathcal{C}\setminus\{\pm i\}).
        \]
        
        \medskip
        \textit{Branch discussion.}  
        The points \(z=\pm i\) are excluded because \(1\mp iz=0\) makes the
        logarithm singular.  
        Away from those points the formula is analytic, and the choice of the
        principal branch of \(\Log\) ensures \(\arctan z\) coincides with its usual
        real‑valued restriction on \((-\,\tfrac{\pi}{2},\tfrac{\pi}{2})\).
        \end{solution}
        \begin{solution}
          Start with
          \[
             \frac{e^{iw}-e^{-iw}}{\,i\bigl(e^{iw}+e^{-iw}\bigr)}.
          \]
          
          \bigskip
          %%%%%%%%%%%%%%%%%%%%%%%%%%%%%%%%%%%%%%%%%%%%%%%%%%%%%%%%%%%%%%%%%%%%%%%%%%%%%%%%
          \textbf{Step 1.  Multiply \emph{numerator and denominator} by $e^{iw}$.}
          
          (The goal is to clear the negative exponent so everything involves
          $e^{2iw}$.)
          
          \[
          \frac{e^{iw}-e^{-iw}}{\,i\bigl(e^{iw}+e^{-iw}\bigr)}
          \;
          \overset{\times\,e^{iw}}{=}\;
          \frac{e^{iw}\bigl(e^{iw}-e^{-iw}\bigr)}
               {\,i\,e^{iw}\bigl(e^{iw}+e^{-iw}\bigr)}.
          \]
          
          \bigskip
          %%%%%%%%%%%%%%%%%%%%%%%%%%%%%%%%%%%%%%%%%%%%%%%%%%%%%%%%%%%%%%%%%%%%%%%%%%%%%%%%
          \textbf{Step 2.  Distribute the factor $e^{iw}$ in the numerator and
          denominator.}
          
          \[
          \begin{aligned}
          \text{Numerator:}\quad
          e^{iw}\bigl(e^{iw}-e^{-iw}\bigr)
                &=e^{2iw}-1,\\[4pt]
          \text{Denominator:}\quad
          e^{iw}\bigl(e^{iw}+e^{-iw}\bigr)
                &=e^{2iw}+1.
          \end{aligned}
          \]
          
          \bigskip
          %%%%%%%%%%%%%%%%%%%%%%%%%%%%%%%%%%%%%%%%%%%%%%%%%%%%%%%%%%%%%%%%%%%%%%%%%%%%%%%%
          \textbf{Step 3.  Substitute these back.}
          
          \[
             \frac{e^{2iw}-1}{\,i\bigl(e^{2iw}+1\bigr)}.
          \]
          
          \bigskip
          %%%%%%%%%%%%%%%%%%%%%%%%%%%%%%%%%%%%%%%%%%%%%%%%%%%%%%%%%%%%%%%%%%%%%%%%%%%%%%%%
          \textbf{Conclusion.}\;
          \[
             \boxed{
               \displaystyle
               \frac{e^{iw}-e^{-iw}}{\,i\bigl(e^{iw}+e^{-iw}\bigr)}
               \;=\;
               \frac{e^{2iw}-1}{\,i\bigl(e^{2iw}+1\bigr)}
             }.
          \]
          
          \medskip
          \emph{Why this matters.}  
          The left‑hand ratio is exactly \(\tan w\) written in exponential form, so the
          right‑hand version—depending only on \(e^{2iw}\)—is convenient for solving
          \(\tan w=z\) and leads directly to the logarithmic identity for \(\arctan z\).
          \end{solution}
          \begin{solution}
            Let
            \[
               f(z)=\frac{\sin z}{z^{3}(1-z)},\qquad z\in\mathcal{C}\,\setminus\{0,1\}.
            \]
            
            %%%%%%%%%%%%%%%%%%%%%%%%%%%%%%%%%%%%%%%%%%%%%%%%%%%%%%%%%%%%%%%%%%%%%%%%%%%%%%%%
            \subsection*{(a)  Singularity at \(z=0\)}
            
            \textbf{Laurent expansion.}  
            Recall the Maclaurin series
            \(
               \sin z = z-\dfrac{z^{3}}{6}+O(z^{5}).
            \)
            Hence
            \[
               f(z)=\frac{z-\dfrac{z^{3}}{6}+O(z^{5})}{z^{3}(1-z)}
                   =\frac{1-\dfrac{z^{2}}{6}+O(z^{4})}{z^{2}(1-z)}.
            \]
            Since \(\dfrac1{1-z}=1+z+z^{2}+O(z^{3})\),
            \[
               f(z)=\frac{1+z+\dfrac{5}{6}z^{2}+O(z^{3})}{z^{2}}
                    =\frac{1}{z^{2}}+\frac{1}{z}+O(1).
            \]
            
            \textbf{Classification.}  
            The highest negative power is \(z^{-2}\); therefore \(z=0\) is a
            \emph{pole of order \(2\)} (and not removable or essential).
            
            %%%%%%%%%%%%%%%%%%%%%%%%%%%%%%%%%%%%%%%%%%%%%%%%%%%%%%%%%%%%%%%%%%%%%%%%%%%%%%%%
            \subsection*{(b)  Singularity at \(z=1\)}
            
            Set \(t=z-1\) so that \(t\to0\) means \(z\to1\).
            Then
            \[
               \sin z
                  =\sin(1+t)
                  =\sin1+\cos1\,t-\frac{\sin1}{2}t^{2}+O(t^{3}),
            \]
            and
            \[
               z^{3}(1-z)=(1+t)^{3}(-t)=-t\bigl(1+3t+3t^{2}+t^{3}\bigr).
            \]
            Thus
            \[
               f(z)
               =-\frac{\sin1+\cos1\,t+O(t^{2})}{t\bigl(1+3t+3t^{2}+t^{3}\bigr)}
               =-\frac{\sin1}{t}+O(1).
            \]
            
            \textbf{Classification.}  
            The principal part contains only the term \(\dfrac{-\sin1}{t}\);
            hence \(z=1\) is a \emph{simple pole} (pole of order \(1\)).
            
            \bigskip
            \[
            \boxed{\;
               z=0:\ \text{pole of order }2,\qquad
               z=1:\ \text{pole of order }1
            \;}
            \]
            \end{solution}
            \subsection*{(b)  Residues at the isolated singularities}

Recall 
\[
   f(z)=\frac{\sin z}{z^{3}(1-z)}, 
   \qquad f\colon\mathbb{C}\setminus\{0,1\}\to\mathbb{C} .
\]

%%%%%%%%%%%%%%%%%%%%%%%%%%%%%%%%%%%%%%%%%%%%%%%%%%%%%%%%%%%%%%%%%%%%%%%%%%%%%%%%
\paragraph{Residue at \(z=0\) (pole of order 2).}

For a pole of order \(m\) at \(z_0\) the residue is
\[
   \Res_{z=z_0}f \;=\;
   \frac{1}{(m-1)!}\,
   \lim_{z\to z_0}
   \frac{\mathrm{d}^{\,m-1}}{\mathrm{d}z^{\,m-1}}
   \Bigl[(z-z_0)^{m}f(z)\Bigr].
\]

Here \(m=2,\;z_0=0\):
\[
\Res_{z=0}f
   =\lim_{z\to0}\frac{\mathrm{d}}{\mathrm{d}z}
     \Bigl[z^{2}\,f(z)\Bigr]
   =\lim_{z\to0}\frac{\mathrm{d}}{\mathrm{d}z}
     \Bigl[\frac{\sin z}{z(1-z)}\Bigr].
\]

Differentiate and then take the limit:

\[
\frac{\mathrm{d}}{\mathrm{d}z}
   \Bigl(\frac{\sin z}{z(1-z)}\Bigr)
   =\frac{z(1-z)\cos z-\sin z\bigl[(1-z)-z\bigr]}
          {z^{2}(1-z)^{2}}
   =\frac{z(1-z)\cos z-\sin z\,(1-2z)}{z^{2}(1-z)^{2}}.
\]

As \(z\to0\):
\[
   z(1-z)\cos z \;\longrightarrow\; 0,
   \qquad
   \sin z\,(1-2z) \;\sim\; z.
\]

Therefore
\[
   \Res_{z=0}f
   =\lim_{z\to0}\frac{-z}{z^{2}}=-1\cdot(-1)=\boxed{1}.
\]
(A quicker route is the Laurent series expansion
\(f(z)=z^{-2}+z^{-1}+O(1)\), whose \(z^{-1}\)-coefficient is also \(1\).)

%%%%%%%%%%%%%%%%%%%%%%%%%%%%%%%%%%%%%%%%%%%%%%%%%%%%%%%%%%%%%%%%%%%%%%%%%%%%%%%%
\paragraph{Residue at \(z=1\) (simple pole).}

Because the pole at \(z=1\) is simple,
\[
   \Res_{z=1}f
   =\lim_{z\to1}(z-1)\,f(z)
   =\lim_{z\to1}
     \frac{(z-1)\sin z}{z^{3}(1-z)}.
\]

Note that \((z-1)/(1-z)=-1\).  Hence
\[
   \Res_{z=1}f
   =-\;\frac{\sin 1}{1^{3}}
   =\boxed{-\sin 1}.
\]

\bigskip
\[
   \boxed{\Res_{z=0}f=1,\qquad \Res_{z=1}f=-\sin 1.}
\]
\begin{solution}
  We already have
  \[
     f(z)=\frac{\sin z}{z^{3}(1-z)},\qquad z\neq 0 .
  \]
  
  %%%%%%%%%%%%%%%%%%%%%%%%%%%%%%%%%%%%%%%%%%%%%%%%%%%%%%%%%%%%%%%%%%%%%%%%%%%%%%%%
  \textbf{Step 1.  Series needed.}
  
  \[
     \sin z = z-\frac{z^{3}}{6}+\frac{z^{5}}{120}-\dotsb,
     \qquad
     \frac{1}{1-z}=1+z+z^{2}+z^{3}+\dotsb ,
     \qquad |z|<1.
  \]
  
  \bigskip
  %%%%%%%%%%%%%%%%%%%%%%%%%%%%%%%%%%%%%%%%%%%%%%%%%%%%%%%%%%%%%%%%%%%%%%%%%%%%%%%%
  \textbf{Step 2.  Divide \(\sin z\) by \(z^{3}\).}
  
  \[
     \frac{\sin z}{z^{3}}
     =\frac{z}{z^{3}}-\frac{z^{3}}{6z^{3}}+\frac{z^{5}}{120z^{3}}-\dotsb
     =\frac{1}{z^{2}}-\frac16+\frac{z^{2}}{120}-\dotsb .
  \]
  
  \bigskip
  %%%%%%%%%%%%%%%%%%%%%%%%%%%%%%%%%%%%%%%%%%%%%%%%%%%%%%%%%%%%%%%%%%%%%%%%%%%%%%%%
  \textbf{Step 3.  Multiply by \(1/(1-z)\).}
  
  Keep only the terms capable of producing the first three
  non‑zero powers (\,\(z^{-2},z^{-1},z^{0}\)\,):
  
  \[
  \begin{aligned}
  f(z)
  &=\bigl(z^{-2}-\tfrac16+\dotsb\bigr)\bigl(1+z+z^{2}+\dotsb\bigr)\\[4pt]
  &=z^{-2}\bigl(1+z+z^{2}\bigr)
     -\tfrac16(1)+\dotsb\\[4pt]
  &=\frac{1}{z^{2}}+\frac{1}{z}
     +\Bigl(1-\frac16\Bigr)+O(z)\\[4pt]
  &=\frac{1}{z^{2}}+\frac{1}{z}+\frac56+O(z).
  \end{aligned}
  \]
  
  \bigskip
  %%%%%%%%%%%%%%%%%%%%%%%%%%%%%%%%%%%%%%%%%%%%%%%%%%%%%%%%%%%%%%%%%%%%%%%%%%%%%%%%
  \textbf{Answer (first three non‑zero terms).}
  \[
     \boxed{\displaystyle
        f(z)=\frac{1}{z^{2}}+\frac{1}{z}+\frac56+O(z)\quad\text{as }z\to0 }.
  \]
  \end{solution}
  \begin{solution}
    We need the Laurent expansion of
    \[
       g(z)=\frac{1}{z^{2}\sin z}
    \]
    about \(z=0\), listing all terms up to (and including) the \(z^{2}\) term.
    
    \bigskip
    %%%%%%%%%%%%%%%%%%%%%%%%%%%%%%%%%%%%%%%%%%%%%%%%%%%%%%%%%%%%%%%%%%%%%%%%%%%%%%%%
    \textbf{1.  Series for \(\sin z\) and its reciprocal.}
    
    \[
       \sin z
       = z-\frac{z^{3}}{6}+\frac{z^{5}}{120}-\frac{z^{7}}{5040}+\dotsb
       \qquad(|z|<\infty).
    \]
    
    Factor out the leading \(z\):
    \[
       \sin z
       =z\Bigl(1-\frac{z^{2}}{6}+\frac{z^{4}}{120}-\dotsb\Bigr).
    \]
    
    Hence
    \[
    \frac{1}{\sin z}
       =\frac{1}{z}\;
        \frac{1}{1-\dfrac{z^{2}}{6}+\dfrac{z^{4}}{120}-\dotsb}.
    \]
    
    Use the geometric‑series inversion
    \(
       \dfrac{1}{1-w}=1+w+w^{2}+w^{3}+\dotsb
    \)
    with
    \(w=\dfrac{z^{2}}{6}-\dfrac{z^{4}}{120}+\dotsb\):
    \[
    \begin{aligned}
    \frac{1}{\sin z}
     &=\frac{1}{z}
       \Bigl[
          1+\frac{z^{2}}{6}
            +\Bigl(\frac{z^{2}}{6}\Bigr)^{2}
            -\frac{z^{4}}{120}
            +O\!\bigl(z^{6}\bigr)
       \Bigr]\\[4pt]
     &=\frac{1}{z}
       \Bigl(
          1+\frac{z^{2}}{6}+\frac{z^{4}}{36}-\frac{z^{4}}{120}
          +O\!\bigl(z^{6}\bigr)
       \Bigr)\\[4pt]
     &=\frac{1}{z}
       \Bigl(
          1+\frac{z^{2}}{6}+\frac{7z^{4}}{360}+O\!\bigl(z^{6}\bigr)
       \Bigr)\\[4pt]
     &=\frac{1}{z}+\frac{z}{6}+\frac{7z^{3}}{360}+O\!\bigl(z^{5}\bigr).
    \end{aligned}
    \]
    
    \bigskip
    %%%%%%%%%%%%%%%%%%%%%%%%%%%%%%%%%%%%%%%%%%%%%%%%%%%%%%%%%%%%%%%%%%%%%%%%%%%%%%%%
    \textbf{2.  Multiply by \(1/z^{2}\).}
    
    \[
       g(z)=\frac{1}{z^{2}}\cdot\frac{1}{\sin z}
            =\frac{1}{z^{2}}
             \biggl(
               \frac{1}{z}+\frac{z}{6}+\frac{7z^{3}}{360}+O(z^{5})
             \biggr)
            =\frac{1}{z^{3}}+\frac{1}{6z}+\frac{7z}{360}+O(z^{3}).
    \]
    
    \bigskip
    %%%%%%%%%%%%%%%%%%%%%%%%%%%%%%%%%%%%%%%%%%%%%%%%%%%%%%%%%%%%%%%%%%%%%%%%%%%%%%%%
    \textbf{3.  Laurent series up to the \(z^{2}\) term.}
    
    \[
    \boxed{\displaystyle
       g(z)=\frac{1}{z^{3}}+\frac{1}{6z}+\frac{7}{360}\,z
            +0\cdot z^{2}+O\!\bigl(z^{3}\bigr)
       \quad\text{as }z\to0 }.
    \]
    
    (The coefficient of \(z^{2}\) happens to be zero.)
    \end{solution}
    \begin{solution}
      %%%%%%%%%%%%%%%%%%%%%%%%%%%%%%%%%%%%%%%%%%%%%%%%%%%%%%%%%%%%%%%%%%%%%%%%%%%%%%%%
      \textbf{Setup.}
      The integrand
      \[
         F(z)\;=\;\frac{e^{z}}{(z-1)(z-2)^{2}}
      \]
      has isolated singularities at
      \[
         z=1 \quad(\text{simple pole}), 
         \qquad
         z=2 \quad(\text{double pole}).
      \]
      The contour is the circle
      \(\displaystyle C:\ |z-2|=1\).
      It encloses \(z=2\) (which lies at the centre) but \emph{does not enclose}
      \(z=1\) (which sits exactly one radius away on~$C$ and is therefore outside
      the interior taken by the residue theorem).  
      Hence
      \[
         \oint_{C} F(z)\,dz
         \;=\;2\pi i\;\Res_{z=2}F.
      \]
      
      \bigskip
      %%%%%%%%%%%%%%%%%%%%%%%%%%%%%%%%%%%%%%%%%%%%%%%%%%%%%%%%%%%%%%%%%%%%%%%%%%%%%%%%
      \textbf{Residue at the double pole \(z=2\).}
      For a pole of order~\(2\) at \(z_{0}\),
      \[
         \Res_{z=z_{0}}F
         \;=\;
         \lim_{z\to z_{0}}
         \frac{d}{dz}\Bigl[(z-z_{0})^{2}F(z)\Bigr].
      \]
      Here \(z_{0}=2\), so
      \[
         \Res_{z=2}F
         =\left.\frac{d}{dz}
           \left[
              \frac{e^{z}}{z-1}
           \right]
           \right|_{z=2}.
      \]
      Differentiate:
      \[
         \frac{d}{dz}\!\left(\frac{e^{z}}{z-1}\right)
         =\frac{e^{z}}{z-1}-\frac{e^{z}}{(z-1)^{2}}.
      \]
      Evaluate at \(z=2\):
      \[
         \Res_{z=2}F
         =\frac{e^{2}}{1}-\frac{e^{2}}{1^{2}}
         =0.
      \]
      
      \bigskip
      %%%%%%%%%%%%%%%%%%%%%%%%%%%%%%%%%%%%%%%%%%%%%%%%%%%%%%%%%%%%%%%%%%%%%%%%%%%%%%%%
      \textbf{Integral value.}
      \[
         \oint_{|z-2|=1}\frac{e^{z}}{(z-1)(z-2)^{2}}\;dz
         \;=\;
         2\pi i\;(0)
         \;=\;
         \boxed{0}.
      \]
      
      \medskip
      \textit{Remark.}  
      Because the lone enclosed pole contributes zero residue,
      the entire contour integral vanishes.
      \end{solution}
      \begin{solution}
        \textbf{Taylor (Maclaurin) series of $\sin(z+1)$ about $z=0$.}
        
        Start from the Maclaurin series of $\sin w$:
        \[
           \sin w \;=\;\sum_{n=0}^{\infty}
                        (-1)^{n}\,\frac{w^{2n+1}}{(2n+1)!},
           \qquad w\in\mathbb{C}.
        \]
        Set $w=z+1$:
        \[
           \sin(z+1)
           \;=\;
           \sum_{n=0}^{\infty}
               (-1)^{n}\,\frac{(z+1)^{2n+1}}{(2n+1)!}.
        \]
        
        \bigskip
        \textbf{Expand in powers of $z$.}  
        A convenient closed form—already expanded—is obtained by using
        elementary trig identities:
        \[
           \boxed{\;
             \sin(z+1)=\sin 1\,\cos z+\cos 1\,\sin z
           \;}
        \]
        and then substituting the Maclaurin series for $\sin z$ and $\cos z$:
        \[
        \begin{aligned}
        \sin(z+1)
        &=\sin 1\left(1-\frac{z^{2}}{2!}+\frac{z^{4}}{4!}-\cdots\right)
         +\cos 1\left(z-\frac{z^{3}}{3!}+\frac{z^{5}}{5!}-\cdots\right)\\[4pt]
        &=\sin 1
          +z\cos 1
          -\frac{z^{2}}{2!}\sin 1
          -\frac{z^{3}}{3!}\cos 1
          +\frac{z^{4}}{4!}\sin 1
          +\frac{z^{5}}{5!}\cos 1
          -\cdots.
        \end{aligned}
        \]
        
        \bigskip
        \textbf{Coefficient form.}
        Writing
        \[
           \sin(z+1)=\sum_{k=0}^{\infty}a_k\,z^{k},
        \]
        the coefficients are
        \[
           a_{2m}=(-1)^{m}\frac{\sin 1}{(2m)!},
           \qquad
           a_{2m+1}=(-1)^{m}\frac{\cos 1}{(2m+1)!},
           \qquad m=0,1,2,\dots.
        \]
        
        \bigskip
        \textbf{First few terms.}
        \[
           \boxed{\displaystyle
             \sin(z+1)=
               \sin 1
               +(\cos 1)\,z
               -\frac{\sin 1}{2}\,z^{2}
               -\frac{\cos 1}{6}\,z^{3}
               +\frac{\sin 1}{24}\,z^{4}
               +\frac{\cos 1}{120}\,z^{5}
               +\cdots
           } 
        \]
        valid for all $z\in\mathbb{C}$ because $\sin$ is entire.
        \end{solution}
        \begin{solution}
          \textbf{Goal.}  Construct a biholomorphic map
          \[
             T:\;
             \Omega=\{\,z\in\mathbb{C}:0<\Re z<\pi,\;\Im z>0\,\}
             \;\longrightarrow\;
             \mathbb{H}^{+}:=\{\,w\in\mathbb{C}:\Im w>0\,\}.
          \]
          
          The construction is achieved by composing two elementary conformal maps.
          
          \bigskip
          %%%%%%%%%%%%%%%%%%%%%%%%%%%%%%%%%%%%%%%%%%%%%%%%%%%%%%%%%%%%%%%%%%%%%%%%%%%%
          \textbf{Step 1 – Exponential map to the upper semicircle of the unit disc.}
          
          Define
          \[
             w_1 \;=\; e^{\,i z}.
          \]
          
          Write $z=x+iy$ with $0<x<\pi$ and $y>0$.  Then
          \[
             w_1 \;=\; e^{\,i(x+iy)}
                  \;=\; e^{-y}\,e^{\,ix},
             \qquad
             0<e^{-y}<1,\; 0<x<\pi.
          \]
          Hence
          \[
             |w_1| = e^{-y} \in (0,1), 
             \quad
             \arg w_1 = x \in (0,\pi).
          \]
          Therefore $w_1$ ranges over the \emph{open upper semicircle}
          \[
             D^{+} \;=\;\{\,\zeta\in\mathbb{C}:|\zeta|<1,\;\Im\zeta>0\,\},
          \]
          and the map $z\mapsto w_1=e^{iz}$ is conformal on~$\Omega$ because
          $e^{iz}$ is entire and $e^{iz}\neq0$ in~$\Omega$.
          
          \bigskip
          %%%%%%%%%%%%%%%%%%%%%%%%%%%%%%%%%%%%%%%%%%%%%%%%%%%%%%%%%%%%%%%%%%%%%%%%%%%%
          \textbf{Step 2 – Cayley transform to the upper half–plane.}
          
          The classical Cayley transform
          \[
             C(\zeta) \;=\;
             i\,\frac{1+\zeta}{1-\zeta},
             \qquad |\zeta|<1,
          \]
          is a Möbius map satisfying $C(D)=\mathbb{H}^{+}$ and
          $C(\partial D)=\mathbb{R}$.  Because $D^{+}\subset D$, the restriction
          of~$C$ to $D^{+}$ is still conformal, and its image is the whole upper
          half–plane (the complement $D\setminus D^{+}$ merely overlaps onto
          $\mathbb{R}$ under~$C$).
          
          \bigskip
          %%%%%%%%%%%%%%%%%%%%%%%%%%%%%%%%%%%%%%%%%%%%%%%%%%%%%%%%%%%%%%%%%%%%%%%%%%%%
          \textbf{Step 3 – Composition.}
          
          Set
          \[
             T(z) \;=\;
             C\!\bigl(w_1\bigr)
             \;=\;
             i\,
             \frac{1+e^{\,i z}}{1-e^{\,i z}},
             \qquad z\in\Omega.
          \]
          Since each constituent map is holomorphic with non–vanishing derivative
          in its domain, the composition~$T$ is holomorphic and one–to–one on
          $\Omega$.  Moreover, by construction $T(\Omega)=\mathbb{H}^{+}$.
          
          \bigskip
          %%%%%%%%%%%%%%%%%%%%%%%%%%%%%%%%%%%%%%%%%%%%%%%%%%%%%%%%%%%%%%%%%%%%%%%%%%%%
          \textbf{Boundary behaviour check.}
          
          \begin{itemize}
            \item \emph{Bottom edge} ($y=0$, $0<x<\pi$):  
                  $e^{\,ix}$ lies on the unit circle,
                  so $T(z)\in\mathbb{R}$.
          
            \item \emph{Left edge} ($x=0$, $y>0$):  
                  $w_1=e^{-y}\in(0,1)$ is real.  
                  Then $T(z)=i\frac{1+w_1}{1-w_1}\in i(1,\infty)$, i.e.\ the
                  positive imaginary axis.
          
            \item \emph{Right edge} ($x=\pi$, $y>0$):  
                  $w_1=-e^{-y}\in(-1,0)$ is real.  
                  Hence $T(z)=i\frac{1-w_1}{1+w_1}\in i(0,1)$, i.e.\ the same
                  positive imaginary axis but between $0$ and~$i$.
          
            \item \emph{Vertex at infinity} ($y\to\infty$):  
                  $w_1\to0$ and $T(z)\to i$.
          \end{itemize}
          
          These images meet precisely on the real line~$\mathbb{R}$, confirming
          that $T$ indeed carries the interior of~$\Omega$ bijectively onto
          $\mathbb{H}^{+}$ and the boundary of~$\Omega$ onto~$\partial\mathbb{H}^{+}$.
          
          \bigskip
          %%%%%%%%%%%%%%%%%%%%%%%%%%%%%%%%%%%%%%%%%%%%%%%%%%%%%%%%%%%%%%%%%%%%%%%%%%%%
          \textbf{Final explicit conformal map.}
          \[
             \boxed{\;
               T(z)
               \;=\;
               i\,\frac{1+e^{\,i z}}{1-e^{\,i z}},
               \qquad z\in\Omega
             \;}
          \]
          \end{solution}
          \begin{solution}
            We must classify each singular point of 
            \[
               f(z)\;=\;\frac{e^{\pi/z}+1}{z^{2}\,(z-i)\,(z+2i)^{2}}.
            \]
            The denominator vanishes (and hence $f$ can fail to be analytic) at
            \[
               z=0,\quad z=i,\quad z=-2i.
            \]
            We analyse the nature of the singularity at each point in turn.
            
            \bigskip
            %%%%%%%%%%%%%%%%%%%%%%%%%%%%%%%%%%%%%%%%%%%%%%%%%%%%%%%%%%%%%%%%%%%%%%%%%%%%
            \textbf{1.\ The point $z=0$ is an \emph{essential} singularity.}
            
            The exponential $e^{\pi/z}$ has the Laurent expansion
            \[
               e^{\pi/z}
               \;=\;
               \sum_{n=0}^{\infty}\frac{\pi^{n}}{n!}\,z^{-n},
               \qquad z\neq0,
            \]
            containing \emph{infinitely many} negative powers; therefore
            $e^{\pi/z}$ is essential at $z=0$.
            Multiplying by the additional factor
            $1/(z^{2}(z-i)(z+2i)^{2})$ cannot cancel this infinite principal part,
            so $f$ itself also has an \textbf{essential singularity at $z=0$}.
            
            \bigskip
            %%%%%%%%%%%%%%%%%%%%%%%%%%%%%%%%%%%%%%%%%%%%%%%%%%%%%%%%%%%%%%%%%%%%%%%%%%%%
            \textbf{2.\ The point $z=i$ is \emph{removable}.}
            
            Compute the numerator at $z=i$:
            \[
               e^{\pi/i}+1
               \;=\;
               e^{-\mathrm{i}\pi}+1
               =
               -1+1
               =0.
            \]
            Thus $e^{\pi/z}+1$ possesses a \emph{simple} zero at $z=i$.
            To see this, differentiate:
            \[
               \left.\frac{d}{dz}e^{\pi/z}\right|_{z=i}
               = -\frac{\pi}{i^{2}}\,e^{\pi/i}
               = \pi\bigl(-1\bigr)
               \neq 0,
            \]
            so the zero is of order~$1$.
            
            Because the denominator also vanishes to first order in $(z-i)$,
            the factors cancel.  Define
            \[
               g(z)
               \;:=\;
               \frac{e^{\pi/z}+1}{z^{2}\,(z+2i)^{2}},
               \qquad\text{analytic near }z=i,
            \]
            then $f(z)=g(z)/(z-i)$ and $g(i)=0$.
            Hence
            \[
               \lim_{z\to i}f(z)
               =\lim_{z\to i}\frac{g(z)-g(i)}{z-i}
               =g'(i)
               \;\in\;\mathbb{C},
            \]
            showing that $f$ extends holomorphically to $z=i$.
            Hence the singularity is \textbf{removable at $z=i$}.
            
            \bigskip
            %%%%%%%%%%%%%%%%%%%%%%%%%%%%%%%%%%%%%%%%%%%%%%%%%%%%%%%%%%%%%%%%%%%%%%%%%%%%
            \textbf{3.\ The point $z=-2i$ is a \emph{pole of order $2$}.}
            
            At $z=-2i$ the denominator has a factor $(z+2i)^{2}$.
            Since
            \[
               e^{\pi/(-2i)}+1
               =e^{-\pi i/2}+1
               =-\,\mathrm{i}+1
               \;\neq\;0,
            \]
            the numerator \emph{does not vanish} there.
            Moreover the other denominator factors $z^{2}$ and $(z-i)$ are non‑zero
            at $z=-2i$.
            Consequently
            \[
               (z+2i)^{2}f(z)
               =\frac{e^{\pi/z}+1}{z^{2}(z-i)}
            \]
            is analytic and non‑zero at $z=-2i$, so the singularity is a
            \textbf{pole of order $2$ at $z=-2i$}.
            
            \bigskip
            %%%%%%%%%%%%%%%%%%%%%%%%%%%%%%%%%%%%%%%%%%%%%%%%%%%%%%%%%%%%%%%%%%%%%%%%%%%%
            \textbf{Summary.}\;  
            \[
               \boxed{\;
                  \begin{array}{lcl}
                     z=0     &\longrightarrow& \text{essential singularity},\\[2pt]
                     z=i     &\longrightarrow& \text{removable singularity},\\[2pt]
                     z=-2i   &\longrightarrow& \text{pole of order }2.
                  \end{array}
               \;}
            \]
            \end{solution}
\end{document}
