\documentclass[12pt]{article}

% Packages
\usepackage[margin=1in]{geometry}
\usepackage{amsmath,amssymb,amsthm}
\usepackage{enumitem}
\usepackage{hyperref}
\usepackage{xcolor}
\usepackage{import}
\usepackage{xifthen}
\usepackage{pdfpages}
\usepackage{transparent}
\usepackage{listings}
\usepackage{tikz}
\usepackage{physics}
\usepackage{siunitx}
\usepackage{booktabs}
\usepackage{cancel}
  \usetikzlibrary{calc,patterns,arrows.meta,decorations.markings}

\DeclareMathOperator{\Log}{Log}
\DeclareMathOperator{\Arg}{Arg}

\lstset{
    breaklines=true,
    breakatwhitespace=false,
    basicstyle=\ttfamily,
    frame=single,
    columns=fullflexible,
}

\newcommand{\incfig}[1]{%
    \def\svgwidth{\columnwidth}
    \import{./Figures/}{#1.pdf_tex}
}

% Theorem‑like environments
\theoremstyle{definition}
\newtheorem{solution}{Solution}
\newtheorem{proposition}{Proposition}
\newtheorem{problem}{Problem}
\newtheorem{lemma}{Lemma}
\newtheorem{theorem}{Theorem}
\newtheorem{remark}{Remark}
\newtheorem{note}{Note}
\newtheorem{definition}{Definition}
\newtheorem{example}{Example}
\newtheorem{corollary}{Corollary}

%----------------------------------------------------
\title{MATH‑448 Practice Final Exam 1}
\author{Jerich Lee}
\date{\today}
\begin{document}

\maketitle
\pagebreak

%-----------------------------------------------------------
\begin{problem}[Values and properties of elementary functions]
\begin{enumerate}[label=(\alph*),itemsep=6pt]
  \item Find \emph{all} values of \((\sqrt{3}+i)^{15/2}\).  Express each value in the form \(x+iy\) with \(x,y\in\mathbb{R}\).
  \item Compute the principal value of \(\displaystyle \log(1-i)\).
  \item Prove the identity
        \[
          \arctan z \;=\; \frac{1}{2i}\,
          \log\!\left(\frac{1+iz}{1-iz}\right),
          \qquad z\in\mathbb{C}\setminus\{\pm i\},
        \]
        where the principal branch of \(\log\) is used.
\end{enumerate}
\end{problem}
\pagebreak
%-----------------------------------------------------------
\begin{problem}[Isolated singularities]
Let
\[
  f(z)\;=\;\frac{\sin z}{z^{3}(1-z)}.
\]
\begin{enumerate}[label=(\alph*),itemsep=6pt]
  \item Classify the isolated singularities of \(f\) at \(z=0\) and \(z=1\) as removable, poles, or essential.
  \item Compute the residues \(\operatorname*{Res}_{z=0}f(z)\) and \(\operatorname*{Res}_{z=1}f(z)\).
  \item Write the first three non‑zero terms of the Laurent series of \(f\) about \(z=0\).
\end{enumerate}
\end{problem}
\pagebreak
%-----------------------------------------------------------
\begin{problem}[Find Taylor/Laurent series]
Obtain the Laurent series of
\[
  g(z)\;=\;\frac{1}{z^{2}\sin z}
\]
about \(z=0\), listing terms up to and including the \(z^{2}\) term.
\end{problem}
\pagebreak
%-----------------------------------------------------------
\begin{problem}[Evaluate an integral over a closed contour]
Using the residue theorem, evaluate
\[
  \int_{\lvert z-2\rvert = 1}
  \frac{e^{z}}{(z-1)(z-2)^{2}}\,dz.
\]
\end{problem}
\pagebreak
%-----------------------------------------------------------
\begin{problem}[Evaluate an improper integral]
Evaluate the real integral
\[
  \int_{0}^{\infty} \frac{x\cos(2x)}{x^{2}+9}\,dx
\]
by extending the integrand to the complex plane and applying contour‑integration techniques.
\end{problem}
\pagebreak
%-----------------------------------------------------------
\begin{problem}[Find the number of zeros of a function]
Determine how many zeros of the polynomial
\[
  f(z)\;=\;z^{4}+2z^{2}+2
\]
lie in the first quadrant \(\{\operatorname{Re}z>0,\ \operatorname{Im}z>0\}\).
Justify your answer using an appropriate theorem (e.g.\ Rouché or Argument Principle).
\end{problem}
\pagebreak
%-----------------------------------------------------------
\begin{problem}[Find a conformal map]
Construct an explicit conformal map \(T\) that carries the half‑strip
\[
  \Omega=\bigl\{z : 0<\operatorname{Re}z<\pi,\ \operatorname{Im}z>0\bigr\}
\]
onto the upper half‑plane \(\operatorname{Im}w>0\).  
Clearly indicate the intermediate mappings used in your construction.
\end{problem}
\pagebreak
%-----------------------------------------------------------
\begin{problem}[Theoretical question]
Prove that if \(f\) is an entire function whose real part is bounded above on \(\mathbb{C}\), then \(f\) is constant.  
\emph{Hint:} consider the function \(e^{f(z)}\).
\end{problem}

\end{document}