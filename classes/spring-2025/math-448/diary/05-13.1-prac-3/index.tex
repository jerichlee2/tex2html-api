\documentclass[12pt]{article}

% Packages
\usepackage[margin=1in]{geometry}
\usepackage{amsmath,amssymb,amsthm}
\usepackage{enumitem}
\usepackage{hyperref}
\usepackage{xcolor}
\usepackage{import}
\usepackage{xifthen}
\usepackage{pdfpages}
\usepackage{transparent}
\usepackage{listings}
\usepackage{tikz}
\usepackage{physics}
\usepackage{siunitx}
\usepackage{booktabs}
\usepackage{cancel}
  \usetikzlibrary{calc,patterns,arrows.meta,decorations.markings}


\DeclareMathOperator{\Log}{Log}
\DeclareMathOperator{\Arg}{Arg}

\lstset{
    breaklines=true,         % Enable line wrapping
    breakatwhitespace=false, % Wrap lines even if there's no whitespace
    basicstyle=\ttfamily,    % Use monospaced font
    frame=single,            % Add a frame around the code
    columns=fullflexible,    % Better handling of variable-width fonts
}

\newcommand{\incfig}[1]{%
    \def\svgwidth{\columnwidth}
    \import{./Figures/}{#1.pdf_tex}
}
\theoremstyle{definition} % This style uses normal (non-italicized) text
\newtheorem{solution}{Solution}
\newtheorem{proposition}{Proposition}
\newtheorem{problem}{Problem}
\newtheorem{lemma}{Lemma}
\newtheorem{theorem}{Theorem}
\newtheorem{remark}{Remark}
\newtheorem{note}{Note}
\newtheorem{definition}{Definition}
\newtheorem{example}{Example}
\newtheorem{corollary}{Corollary}
\theoremstyle{plain} % Restore the default style for other theorem environments
%

% Theorem-like environments
% Title information
\title{MATH-448 Practice Final Exam 3}
\author{Jerich Lee}
\date{\today}

\begin{document}

\maketitle
\pagebreak

  
  \begin{problem}
  \textbf{Values and properties of elementary functions}  
  
  Compute the \emph{principal value} of each expression, writing every answer in the form \(a+ib\) with \(a,b\in\mathbb{R}\).
  \begin{enumerate}\itemsep6pt
    \item[(a)] \(\displaystyle\bigl(1+i\sqrt{3}\bigr)^{40}\).
    \item[(b)] \(\displaystyle e^{\frac12\bigl(\log(2+2i)-\log(2-2i)\bigr)}\).
    \item[(c)] \(\displaystyle \log\!\bigl(\cosh(1+i)\bigr)\).
  \end{enumerate}
  \end{problem}
  
  \pagebreak
  
  \begin{problem}
  \textbf{Isolated singularities}  
  
  Let  
  \[
    f(z)\;=\;\frac{e^{1/z}}{z^{2}(z-1)}.
  \]
  \begin{enumerate}\itemsep6pt
    \item[(a)] Classify the isolated singularities of \(f\) at \(z=0\) and \(z=1\) (removable, pole of which order, or essential).
    \item[(b)] Compute \(\displaystyle \operatorname*{Res}_{z=0}f(z)\).
    \item[(c)] Write the principal part of the Laurent expansion of \(f\) about \(z=0\).
  \end{enumerate}
  \end{problem}
  
  \pagebreak
  
  \begin{problem}
  \textbf{Taylor/Laurent series}  
  
  Find the Laurent series of  
  \[
    g(z)\;=\;\frac{\log(1+z)}{z^{2}}
  \]
  about \(z=0\) and list all terms up to and including the coefficient of \(z^{2}\).
  \end{problem}
  
  \pagebreak
  
  \begin{problem}
  \textbf{Integral over a closed contour}  
  
  Evaluate, using residues,
  \[
    \oint_{\lvert z-2\rvert = 2}
    \frac{z^{2}}{(z^{2}+1)\,(z-3)}\,dz.
  \]
  \end{problem}
  
  \pagebreak
  
  \begin{problem}
  \textbf{Improper integral}  
  
  Show that
  \[
    I \;=\;\int_{0}^{\infty}\frac{x\sin x}{x^{2}+1}\,dx
  \]
  converges and compute its exact value.
  \end{problem}
  
  \pagebreak
  
  \begin{problem}
  \textbf{Number of zeros of a function}  
  
  Let  
  \[
    h(z)\;=\;z^{5}+2z^{2}+10.
  \]
  Use Rouché’s Theorem (or the Argument Principle) to determine the number of zeros of \(h\) inside the circle \(\lvert z\rvert = 2\).
  \end{problem}
  
  \pagebreak
  
  \begin{problem}
  \textbf{Conformal mapping}  
  
  Construct an explicit conformal map \(T:\Omega\to\mathbb{H}^{+}\), where  
  \[
    \Omega=\bigl\{\,z : 0<\operatorname{Im}z<\pi,\; \operatorname{Re}z>0\bigr\},
    \qquad
    \mathbb{H}^{+}=\bigl\{\,w : \operatorname{Im}w>0\bigr\}.
  \]
  Describe the intermediate mappings you use.
  \end{problem}
  
  \pagebreak
  
  \begin{problem}
  \textbf{Theoretical question}  
  
  Prove that if \(f\) is an entire function satisfying
  \[
    \bigl|f(z)\bigr|\;\ge\;\bigl|z\bigr|
    \quad\text{for all }z\in\mathbb{C},
  \]
  then \(f(z)=az\) for some constant \(a\) with \(\lvert a\rvert\ge 1\).  
  \emph{Hint:} Apply the Maximum‑Modulus Principle to \(f(z)/z\) and analyse the behaviour at infinity.
  \end{problem}
\end{document}
