\documentclass[12pt]{article}

% Packages
\usepackage[margin=1in]{geometry}
\usepackage{amsmath,amssymb,amsthm}
\usepackage{enumitem}
\usepackage{hyperref}
\usepackage{xcolor}
\usepackage{import}
\usepackage{xifthen}
\usepackage{pdfpages}
\usepackage{transparent}
\usepackage{listings}
\usepackage{tikz}
  \usetikzlibrary{calc,patterns,arrows.meta,decorations.markings}


\DeclareMathOperator{\Log}{Log}
\DeclareMathOperator{\Arg}{Arg}

\lstset{
    breaklines=true,         % Enable line wrapping
    breakatwhitespace=false, % Wrap lines even if there's no whitespace
    basicstyle=\ttfamily,    % Use monospaced font
    frame=single,            % Add a frame around the code
    columns=fullflexible,    % Better handling of variable-width fonts
}

\newcommand{\incfig}[1]{%
    \def\svgwidth{\columnwidth}
    \import{./Figures/}{#1.pdf_tex}
}
\theoremstyle{definition} % This style uses normal (non-italicized) text
\newtheorem{solution}{Solution}
\newtheorem{proposition}{Proposition}
\newtheorem{problem}{Problem}
\newtheorem{lemma}{Lemma}
\newtheorem{theorem}{Theorem}
\newtheorem{remark}{Remark}
\newtheorem{note}{Note}
\newtheorem{definition}{Definition}
\newtheorem{example}{Example}
\newtheorem{corollary}{Corollary}
\theoremstyle{plain} % Restore the default style for other theorem environments
%

% Theorem-like environments
% Title information
\title{}
\author{Jerich Lee}
\date{\today}

\begin{document}

\maketitle
\section*{Linear Fractional (Möbius) Transformations:  A Step--by--Step Walkthrough}

\subsection*{1. Definition}
A \emph{linear fractional transformation} (LFT) is a rational function
\[
   T(z)=\frac{az+b}{cz+d},
   \qquad a,b,c,d\in\mathbb C,
   \qquad ad-bc\neq0.
\]
The non–vanishing of the determinant $ad-bc$ is essential; it guarantees that
$T$ is \emph{non‑constant} and \emph{invertible}.

\subsection*{2. Why \boldmath$ad-bc\neq0$ is required}
\begin{itemize}
  \item Compute the derivative:
        \[
           T'(z)=\frac{ad-bc}{(cz+d)^{2}}.
        \]
  \item If $ad-bc=0$, then $T'(z)\equiv0$ and $T$ is constant
        (its numerator and denominator share a linear factor).
  \item Hence $ad-bc\neq0$ is necessary to avoid the trivial case.
\end{itemize}

\subsection*{3. Injectivity (One‑to‑One Property)}
Assume $z_{1},z_{2}$ satisfy $T(z_{1})=T(z_{2})$:
\[
   \frac{az_{1}+b}{cz_{1}+d} = \frac{az_{2}+b}{cz_{2}+d}.
\]
Cross‑multiplying,
\[
   (az_{1}+b)(cz_{2}+d)=(az_{2}+b)(cz_{1}+d).
\]
Expanding and cancelling like terms yields
\[
   bc z_{2}+ad z_{1}=ad z_{2}+bc z_{1}
   \;\Longrightarrow\;
   (ad-bc)(z_{1}-z_{2})=0.
   \]
Because $ad-bc\neq0$, we conclude $z_{1}=z_{2}$.
Thus $T$ maps distinct points to distinct images.

\subsection*{4. Behaviour at Special Points}
\begin{itemize}
  \item \textbf{Pole.}  $T$ has a simple pole at $z=-\dfrac{d}{c}$ (provided $c\neq0$).
  \item \textbf{Value at infinity.}  Viewing $T$ on the Riemann sphere,
        \[
            \lim_{|z|\to\infty}T(z)=\frac{a}{c}.
        \]
\end{itemize}

\subsection*{5. Existence and Form of the Inverse}
Because $T$ is one‑to‑one on $\widehat{\mathbb C}=\mathbb C\cup\{\infty\}$,
there exists an analytic inverse $T^{-1}$ satisfying
\[
   T^{-1}\!\bigl(T(z)\bigr)=z
   \quad\text{for every }z\in\widehat{\mathbb C}.
\]
\begin{enumerate}
  \item Set $w=T(z)=\dfrac{az+b}{cz+d}$.
  \item Solve for $z$:
        \[
          wz(cz+d)=az+b
          \;\Longrightarrow\;
          czw-az=b-dw
          \;\Longrightarrow\;
          z=\frac{-dw+b}{cw-a}.
        \]
\end{enumerate}
Hence
\[
   T^{-1}(w)=\frac{-dw+b}{cw-a},
\]
which is itself a linear fractional transformation with determinant
$(-d)(-a)-bc=ad-bc\neq0$.

\subsection*{6. Mapping Property}
Therefore an LFT is a bijective analytic map of the extended complex plane
onto itself.  Conversely (see the referenced Exercise), any analytic
one‑to‑one self‑map of $\widehat{\mathbb C}$ must be a linear fractional
transformation.

\section*{Example 1:  An Automorphism of the Unit Disc}

\paragraph{Claim.}
Let
\[
   T(z)=\lambda\,\frac{a-z}{1-\overline{a}\,z},
   \qquad |a|<1,\;|\lambda|=1 .
\]
Then $T$ maps the open unit disc
\[
   \Delta=\{\,z\in\mathbb C : |z|<1\,\}
\]
\emph{onto} itself.

\subsection*{Step‑by‑Step Proof}

\begin{enumerate}
\item \textbf{Reduction to the case $\lambda=1$.}  \\
      Multiplication by a unimodular constant is an isometry of
      $\Delta$, so it suffices to show that
      \[
          \varphi_{a}(z):=\frac{a-z}{1-\overline{a}\,z}
      \]
      maps $\Delta$ into itself.  The result for $T$ then follows by the
      composition $T=\lambda\varphi_{a}$.

\item \textbf{Compute the modulus of $\varphi_{a}$.}  For $|z|<1$,
      \[
          |\varphi_{a}(z)|^{2}
          =\frac{|a-z|^{2}}{|1-\overline{a}z|^{2}}
          =\frac{|a|^{2}-2\Re(\overline{a}z)+|z|^{2}}
                 {1-2\Re(\overline{a}z)+|a|^{2}|z|^{2}}.
      \]

\item \textbf{Show $|\varphi_{a}(z)|<1$.}\\
      The inequality $|\varphi_{a}(z)|<1$ is equivalent to
      \[
         |a|^{2}-2\Re(\overline{a}z)+|z|^{2}
         \;<\;
         1-2\Re(\overline{a}z)+|a|^{2}|z|^{2},
      \]
      which simplifies to
      \[
         (1-|a|^{2})(1-|z|^{2})>0.
      \]
      Because $|a|<1$ and $|z|<1$, both factors are positive, hence
      $|\varphi_{a}(z)|<1$.  Thus $\varphi_{a}(\Delta)\subset \Delta$,
      and therefore $T(\Delta)\subset\Delta$.

\item \textbf{Construct the inverse map and verify surjectivity.}\\
      Solve $w=\lambda\varphi_{a}(z)$ for $z$:
      \[
          z
          =\frac{\overline{\lambda}\,a-\overline{\lambda}\,w}
                 {1-\overline{a}\,\overline{\lambda}\,w}
          \;=:\;T^{-1}(w).
      \]
      Observe that $|\,\overline{\lambda}\,a|=|a|<1$, so $T^{-1}$ has the
      same form as $T$ (with $a$ replaced by $\overline{\lambda}a$ and
      $\lambda$ by $\overline{\lambda}$) and hence maps $\Delta$
      into itself.  Because $T^{-1}\circ T=\mathrm{id}_{\Delta}$,  
      $T$ is surjective onto~$\Delta$.

\end{enumerate}

\paragraph{Conclusion.}  
$T$ is a bijective analytic self‑map of the unit disc~$\Delta$; i.e.\ an
\emph{automorphism} of~$\Delta$. \qed
\section*{Fixed Points and Triples for Linear Fractional Transformations}

Throughout, a \emph{linear fractional transformation} (LFT) is a map
\[
   T(z)=\frac{az+b}{cz+d},
   \qquad a,b,c,d\in\mathbb C,\;ad-bc\ne0 .
\]

%------------------------------------------------
\subsection*{1.\; An LFT has at most two fixed points}

\begin{itemize}
   \item A fixed point satisfies $T(z)=z$, i.e.
         \[
            z=\frac{az+b}{cz+d}
            \;\Longleftrightarrow\;
            cz^{2}+(d-a)z-b=0.
         \]
   \item The right–hand side is a quadratic polynomial.
         Hence there are \emph{at most two} distinct solutions, so an
         LFT that is not the identity has at most two fixed points.
\end{itemize}

%------------------------------------------------
\subsection*{2.\; Equality of two LFTs at three points forces equality everywhere}

\begin{theorem}\label{unique}
If $T$ and $S$ are LFTs and
\(
   T(z_{j})=S(z_{j}) \;(j=1,2,3)
\)
for three \emph{distinct} points $z_{1},z_{2},z_{3}$,
then $T\equiv S$ on $\widehat{\mathbb C}$.
\end{theorem}

\begin{proof}[Step–by–step proof]
\begin{enumerate}
   \item Consider the composition
         \(
            R:=S^{-1}\!\circ\,T.
         \)
         It is an LFT because the class of LFTs is closed under
         composition and inversion.
   \item For each $j=1,2,3$,
         \(
            R(z_{j})=S^{-1}\!\bigl(T(z_{j})\bigr)
                   =S^{-1}\!\bigl(S(z_{j})\bigr)=z_{j},
         \)
         so $z_{1},z_{2},z_{3}$ are fixed points of $R$.
   \item By Section~1, a non‑identity LFT has at most two fixed points.
         Therefore $R$ must be the identity map, i.e.\ $R(z)=z$ for all
         $z$.
   \item Hence $T=S$ on all of $\widehat{\mathbb C}$.
\end{enumerate}
\end{proof}

%------------------------------------------------
\subsection*{3.\; The unique LFT sending one triple to another}

\paragraph{Problem.}
Given three distinct points $z_{1},z_{2},z_{3}\in\widehat{\mathbb C}$
and another triple $w_{1},w_{2},w_{3}\in\widehat{\mathbb C}$, show that
there is a \emph{unique} LFT $L$ with
\(
   L(z_{j})=w_{j}\;(j=1,2,3).
\)

\subsubsection*{Construction}

\begin{enumerate}
   \item\label{Tdef}
         Define
         \[
             T(z):=\frac{z-z_{1}}{z-z_{3}}\,
                   \frac{\,z_{2}-z_{3}\,}{\,z_{2}-z_{1}\,}.
         \]
         A direct check gives $T(z_{1})=0$, $T(z_{2})=1$, $T(z_{3})=\infty$.
   \item\label{Sdef}
         Analogously, set
         \[
             S(w):=\frac{w-w_{1}}{w-w_{3}}\,
                   \frac{\,w_{2}-w_{3}\,}{\,w_{2}-w_{1}\,},
         \]
         so $S(w_{1})=0$, $S(w_{2})=1$, $S(w_{3})=\infty$.
   \item\label{Ldef}
         Define the desired map as the conjugate
         \[
             L(z):=S^{-1}\!\bigl(T(z)\bigr).
         \]
         Then
         \(
            L(z_{j})=S^{-1}\!\bigl(T(z_{j})\bigr)=S^{-1}(j\text{-th special value})
                    =w_{j}\;(j=1,2,3).
         \)
\end{enumerate}

\subsubsection*{Uniqueness}

Suppose $\tilde{L}$ is another LFT with $\tilde{L}(z_{j})=w_{j}$.
By Theorem~\ref{unique}, $L$ and $\tilde{L}$ coincide at the three
distinct points $z_{1},z_{2},z_{3}$, hence $L\equiv\tilde{L}$.

\paragraph{Conclusion}
There exists exactly one linear fractional transformation sending a
given ordered triple $(z_{1},z_{2},z_{3})$ to another ordered triple
$(w_{1},w_{2},w_{3})$, and its explicit form is $L=S^{-1}\circ T$ as
constructed above.
\qed

\section*{Example 2.  A Linear Fractional Transformation Sending $0,1,2$ to $-1,0,4$}

\subsection*{Goal}
Construct the unique linear fractional transformation \(L\) such that
\[
    L(0)=-1,\qquad
    L(1)=0,\qquad
    L(2)=4 .
\]

\subsection*{Step 1.  Build an auxiliary map \(T\)}
Choose the \emph{domain} triple
\(\;(z_{1},z_{2},z_{3})=(0,1,2)\) and map it to \((0,1,\infty)\):  
\[
   T(z)=
      \frac{z-z_{1}}{z-z_{3}}\,
      \frac{z_{2}-z_{3}}{\,z_{2}-z_{1}}
      \;=\;
      \frac{z-0}{z-2}\,
      \frac{1-2}{\,1-0}
      =
      \frac{z}{\,2-z}.
\]
Thus
\(
    T(0)=0,\;
    T(1)=1,\;
    T(2)=\infty.
\)

\subsection*{Step 2.  Build an auxiliary map \(S\)}
Choose the \emph{target} triple
\((w_{1},w_{2},w_{3})=(-1,0,4)\) and map it to \((0,1,\infty)\):  
\[
   S(w)=
      \frac{w-w_{1}}{w-w_{3}}\,
      \frac{w_{2}-w_{3}}{\,w_{2}-w_{1}}
      =
      \frac{w-(-1)}{w-4}\,
      \frac{0-4}{\,0-(-1)}
      =
      -4\,\frac{w+1}{w-4}.
\]
Hence
\(
    S(-1)=0,\;
    S(0)=1,\;
    S(4)=\infty.
\)

\subsection*{Step 3.  Invert \(S\)}
Solve \(S(w)=z\) for \(w\):
\[
   z=-4\,\frac{w+1}{w-4}
   \;\Longrightarrow\;
   z(w-4)=-4(w+1)
   \;\Longrightarrow\;
   zw-4z=-4w-4
   \;\Longrightarrow\;
   (z+4)w=4(z-1).
\]
Therefore
\[
   S^{-1}(z)=\frac{4(z-1)}{\,z+4}.
\]

\subsection*{Step 4.  Compose to obtain \(L\)}
Define
\[
   L(z):=S^{-1}\!\bigl(T(z)\bigr)
        =\frac{4\bigl(T(z)-1\bigr)}{T(z)+4}.
\]
Insert \(T(z)=\dfrac{z}{2-z}\):
\[
   T(z)-1=\frac{z}{2-z}-1=\frac{z-(2-z)}{2-z}=\frac{2z-2}{2-z},\qquad
   T(z)+4=\frac{z}{2-z}+4=\frac{z+4(2-z)}{2-z}=\frac{8-3z}{2-z}.
\]
Hence
\[
   L(z)=
   \frac{4\bigl(2z-2\bigr)}{2-z}\;
   \Big/\;
   \frac{8-3z}{2-z}
   =\frac{4(2z-2)}{8-3z}
   =\frac{8z-8}{-3z+8}.
\]

\subsection*{Step 5.  Verification}
\[
\begin{aligned}
   L(0)&=\frac{-8}{8}=-1,\\
   L(1)&=\frac{0}{5}=0,\\
   L(2)&=\frac{8\cdot2-8}{-3\cdot2+8}
        =\frac{8}{2}=4,
\end{aligned}
\qquad\checkmark
\]

\paragraph{Result}
\[
   \boxed{\,L(z)=\dfrac{8z-8}{\, -3z+8}\,},
\]
the desired linear fractional transformation.
\section*{Lines and Circles under Linear Fractional Transformations}

Let 
\[
   T(z)=\frac{az+b}{cz+d},\qquad a,b,c,d\in\mathbb C,\;ad-bc\ne0.
\]
We show \emph{step by step} that $T$ maps every straight line or circle in
the extended complex plane $\widehat{\Bbb C}$ onto either a line or a
circle.

\subsection*{1.  Two trivial cases}

\begin{enumerate}
\item[\textbf{(i)}] \textbf{Translation/rotation/dilation.}  
      If $c=0$ the map reduces to 
      \(
         T(z)=az+b\;(a\ne0).
      \)
      The image of a circle 
      \(
         C=\{\,|z-z_{0}|=r\,\}
      \)
      is the circle
      \(
         C'=\{\,|w-(az_{0}+b)|=|a|r\,\};
      \)
      the image of a line
      \(
         L=\{\Re(Az+B)=0\}
      \)
      is the line
      \(
         L'=\{\Re(\bar{a}^{-1}A\,w+B-b\bar{a}^{-1}A)=0\}.
      \)
\item[\textbf{(ii)}] \textbf{Constant map.}  
      Excluded by $ad-bc\ne0$, hence ignored.
\end{enumerate}
From now on assume $c\ne0$.

\subsection*{2.  Factor \(T\) as a composition of simpler maps}

Write
\[
   T(z)=\frac{az+b}{cz+d}
       =\frac1c\Bigl(\frac{bc-ad}{cz+d}+a\Bigr).
\]
Define three auxiliary maps
\[
   U(z)=cz+d,\qquad
   V(w)=\frac1w,\qquad
   W(t)=\frac{1}{c}\bigl[(bc-ad)t+a\bigr],
\]
so that
\[
   T(z)=W\!\bigl(V(U(z))\bigr)=W\circ V\circ U.
\]

\paragraph{Observation.}
\begin{itemize}
\item $U$ is affine $\implies$ sends circles to circles, lines to lines.
\item $W$ is affine $\implies$ same property.
\item $V$ is the \emph{inversion} $w\mapsto 1/w$,
      the only non‑affine step; we therefore verify its effect separately.
\end{itemize}

\subsection*{3.  Effect of inversion \(V(w)=1/w\)}

Write $w=u+iv$ so that
\(
   1/w=(u-iv)/(u^{2}+v^{2})=x+iy
\)
with
\[
   x=\frac{u}{u^{2}+v^{2}},\qquad
   y=-\frac{v}{u^{2}+v^{2}}.
\]
Hence
\(
   x^{2}+y^{2}=\dfrac1{u^{2}+v^{2}}.
\)

\medskip
\noindent\textbf{(a) A circle/line \emph{not} passing through the origin.}\\
Such a curve is determined by
\[
   \alpha(u^{2}+v^{2})+\beta u+\gamma v=\delta,\qquad
   \alpha,\beta,\gamma,\delta\in\Bbb R,\ 
   (\alpha,\beta,\gamma)\ne(0,0,0).
\]
Multiply by $x^{2}+y^{2}=1/(u^{2}+v^{2})$ to obtain
\[
   \delta(x^{2}+y^{2})-\beta x-\gamma y=\alpha,
\]
still the general equation of a circle or a line.  
Therefore \(V\) sends a line/circle \emph{not} through \(0\) to a
circle/line.

\medskip
\noindent\textbf{(b) A line or circle \emph{through} the origin.}\\
Set, for instance, $\beta u+\gamma v=\delta$ with $\delta=0$.  
After inversion this becomes
\(
   -\beta x-\gamma y=\alpha(x^{2}+y^{2}),
\)
a quadratic with non‑zero quadratic term, i.e.\ a \emph{circle}
passing through the origin.  
Thus a line through the origin is sent to a circle through the origin
(and vice‑versa).

\subsection*{4.  Composition argument}

Since \(U\) and \(W\) individually preserve the \emph{type} (circle/line),
and \(V\) sends circles and lines to circles and lines,
their composition \(T=W\circ V\circ U\) has the same property.

\subsection*{5.  Summary}

Every linear fractional transformation with \(ad-bc\ne0\) carries:

\begin{itemize}
\item each straight line that does \emph{not} pass through the inversion
      centre into a straight line;
\item each line through the inversion centre into a circle through that
      centre;
\item each circle not through the inversion centre into a circle;
\item each circle through the inversion centre into a line.
\end{itemize}
Hence \textbf{every} line or circle is mapped to a line or a circle.
\qed
\section*{Conformal Mapping:  Step--by--Step Walkthrough}

\subsection*{1.\;  Setup}

Let $\gamma:[a,b]\to\mathbb C,\;t\mapsto z(t)$ be a smooth (i.e.\ $C^{1}$) curve
with non--vanishing tangent vector $z'(t_{0})\neq0$ at $t_{0}\in(a,b)$.
Denote $z_{0}=z(t_{0})$.
For an analytic function $f$, define the \emph{image curve}
\[
   \Gamma:\qquad w(t)=f(z(t)),\qquad a\le t\le b .
\]

\subsection*{2.\;  Transformation of the tangent vector}

\paragraph{Chain rule.}
Since $w(t)=f(z(t))$ and $f$ is differentiable,
\[
   w'(t)=f'(z(t))\,z'(t),\qquad
   \text{hence}\quad
   w'(t_{0}) = f'(z_{0})\,z'(t_{0}).      \tag{1}
\]

\paragraph{Scaling of length.}
Taking moduli in \eqref{1},
\[
   |w'(t_{0})| = |f'(z_{0})|\,|z'(t_{0})| .       \tag{2}
\]
Thus the tangent vector is \emph{magnified} by the factor $|f'(z_{0})|$.

\paragraph{Rotation of direction.}
For arguments (angles) we have
\[
   \arg w'(t_{0}) = \arg f'(z_{0}) + \arg z'(t_{0})\pmod{2\pi}. \tag{3}
\]
So the tangent is \emph{rotated} through an angle
$\theta_{0}:=\arg f'(z_{0})$.

\subsection*{3.\;  Angle between two curves}

Let $\gamma_{1},\gamma_{2}$ be two smooth curves intersecting at $z_{0}$, with
non--zero tangent vectors $z_{1}'(t_{0}),\,z_{2}'(s_{0})$.
Define their oriented angle at $z_{0}$ by
\[
   \angle(\gamma_{1},\gamma_{2}) := 
   \arg z_{2}'(s_{0}) - \arg z_{1}'(t_{0}) \pmod{2\pi}. \tag{4}
\]

\subsection*{4.\;  Preservation of angle and orientation}

\paragraph{Image tangents.}
By \eqref{1},
\(
   w_{1}' = f'(z_{0})\,z_{1}',\;
   w_{2}' = f'(z_{0})\,z_{2}' .
\)
Hence, using \eqref{3},
\[
   \arg w_{2}' - \arg w_{1}'
   = \bigl[\arg f'(z_{0})+\arg z_{2}'\bigr]
     -\bigl[\arg f'(z_{0})+\arg z_{1}'\bigr]
   = \arg z_{2}'-\arg z_{1}'. \tag{5}
\]
Comparing \eqref{4} and \eqref{5} shows both the \emph{magnitude} and the
\emph{orientation} (sense) of the angle are preserved.

\subsection*{5.\;  Local conformality criterion}

\begin{theorem}
\label{thm:local}
If $f$ is analytic in a disc $|z-z_{0}|<r$ and $f'(z_{0})\neq0$, then
$f$ is \emph{conformal} at $z_{0}$: it preserves oriented angles between any
two smooth curves meeting at $z_{0}$.
\end{theorem}

\begin{proof}
Given two such curves $\gamma_{1},\gamma_{2}$, formula \eqref{5} shows that
the angle from $\gamma_{1}$ to $\gamma_{2}$ at $z_{0}$ equals the angle from
their images at $w_{0}=f(z_{0})$; thus conformality holds.
\end{proof}

\subsection*{6.\;  Remarks on critical points}

\begin{itemize}
\item If $f$ has a zero of order $m\ge2$ at $z_{0}$, then $|f'(z_{0})|=0$ and
      the argument calculation generalises to show the angle between two
      curves is \emph{multiplied} by $m$; see the exercises following the
      text.
\item Because $f'(z)\neq0$ for a one--to--one analytic map (Section~3.2),
      we obtain:

      \begin{theorem}
      If $f$ is analytic and injective on a domain $D$, then $f$ is conformal
      at every point of $D$.
      \end{theorem}
\end{itemize}

\end{document}
