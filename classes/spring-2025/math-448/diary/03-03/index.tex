\documentclass[12pt]{article}

% Packages
\usepackage[margin=1in]{geometry}
\usepackage{amsmath,amssymb,amsthm}
\usepackage{enumitem}
\usepackage{hyperref}
\usepackage{xcolor}
\usepackage{import}
\usepackage{xifthen}
\usepackage{pdfpages}
\usepackage{transparent}
\usepackage{listings}


\lstset{
    breaklines=true,         % Enable line wrapping
    breakatwhitespace=false, % Wrap lines even if there's no whitespace
    basicstyle=\ttfamily,    % Use monospaced font
    frame=single,            % Add a frame around the code
    columns=fullflexible,    % Better handling of variable-width fonts
}

\newcommand{\incfig}[1]{%
    \def\svgwidth{\columnwidth}
    \import{./Figures/}{#1.pdf_tex}
}
\theoremstyle{definition} % This style uses normal (non-italicized) text
\newtheorem{solution}{Solution}
\newtheorem{proposition}{Proposition}
\newtheorem{problem}{Problem}
\newtheorem{lemma}{Lemma}
\newtheorem{theorem}{Theorem}
\newtheorem{remark}{Remark}
\newtheorem{note}{Note}
\newtheorem{definition}{Definition}
\newtheorem{example}{Example}
\theoremstyle{plain} % Restore the default style for other theorem environments
%

% Theorem-like environments
% Title information
\title{}
\author{Jerich Lee}
\date{\today}

\begin{document}

\maketitle
\textbf{Derivation of the residue formula:}

\bigskip

Let $f$ be a function with an isolated singularity at $z_0$. Then in a punctured neighborhood of $z_0$, $f$ can be expanded in a Laurent series:
\[
f(z) \;=\; \sum_{n=-\infty}^{\infty} a_{n} \,(z - z_0)^n.
\]

By definition, the residue of $f$ at $z_0$ is the coefficient of $(z - z_0)^{-1}$ in this expansion:
\[
\operatorname{Res}(f; z_0) \;=\; a_{-1}.
\]

Now consider the contour integral of $f(z)$ around a small circle $|z - z_0| = s$. Substitute the Laurent series into the integral:
\[
\int_{|z - z_0|=s} f(z)\, dz
\;=\;
\int_{|z - z_0|=s} \sum_{n=-\infty}^\infty a_n (z - z_0)^n \, dz.
\]
By uniform convergence on the circle, we can interchange the sum and the integral:
\[
\int_{|z - z_0|=s} \sum_{n=-\infty}^\infty a_n (z - z_0)^n \, dz
\;=\;
\sum_{n=-\infty}^\infty
a_n \int_{|z - z_0|=s} (z - z_0)^n \, dz.
\]
Parametrize the contour by $z - z_0 = s e^{i\theta}$ with $\theta$ going from $0$ to $2\pi$. One finds that
\[
\int_{|z - z_0|=s} (z - z_0)^n \, dz = 
\begin{cases}
0, & n \neq -1, \\
2\pi i, & n = -1.
\end{cases}
\]
Hence, the only term that survives is the one for $n = -1$:
\[
\int_{|z - z_0|=s} f(z)\, dz
\;=\;
a_{-1} \cdot 2\pi i
\;=\;
2\pi i \,\operatorname{Res}(f; z_0).
\]
Therefore,
\[
\operatorname{Res}(f; z_0)
\;=\;
\frac{1}{2\pi i}
\int_{|z - z_0| = s} f(z)\, dz.
\]
\textbf{Relationship between the Residue Theorem and Cauchy's Integral Formula:}

\medskip

\noindent
\textbf{Cauchy's Integral Formula:}  
If $f$ is holomorphic in a simply connected domain containing $z_0$, then Cauchy's integral formula states
\[
f(z_0)
\;=\;
\frac{1}{2\pi i}
\int_{\Gamma}
\frac{f(z)}{z - z_0}\, dz,
\]
where $\Gamma$ is a simple closed contour enclosing $z_0$.

\medskip

\noindent
\textbf{Residue Theorem:}  
If $g(z)$ is meromorphic in a domain bounded by $\Gamma$ and on $\Gamma$, with isolated singularities $z_1, z_2, \dots, z_n$ inside, then the residue theorem says
\[
\int_{\Gamma} g(z)\, dz
\;=\;
2\pi i
\sum_{k=1}^n
\operatorname{Res}\bigl(g; z_k\bigr).
\]

\medskip

\noindent
\textbf{Connecting Them:}  
To see how Cauchy's integral formula follows from the residue theorem, let
\[
g(z)
\;=\;
\frac{f(z)}{z - z_0}.
\]
The only singularity of $g(z)$ inside $\Gamma$ is a simple pole at $z = z_0$, and its residue there is $f(z_0)$. By the residue theorem:
\[
\int_{\Gamma}
\frac{f(z)}{z - z_0}
\,dz
\;=\;
2\pi i
\,\operatorname{Res}\!\Bigl(\tfrac{f(z)}{z - z_0};\, z_0\Bigr)
\;=\;
2\pi i
\,f(z_0).
\]
Dividing both sides by $2\pi i$ gives the statement of Cauchy's integral formula.

\end{document}
