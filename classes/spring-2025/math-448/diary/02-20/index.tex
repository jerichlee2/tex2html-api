\documentclass[12pt]{article}

% Packages
\usepackage[margin=.5in]{geometry}
\usepackage{amsmath,amssymb,amsthm}
\usepackage{enumitem}
\usepackage{hyperref}
\usepackage{xcolor}
\usepackage{import}
\usepackage{xifthen}
\usepackage{pdfpages}
\usepackage{transparent}
\usepackage{listings}


\lstset{
    breaklines=true,         % Enable line wrapping
    breakatwhitespace=false, % Wrap lines even if there's no whitespace
    basicstyle=\ttfamily,    % Use monospaced font
    frame=single,            % Add a frame around the code
    columns=fullflexible,    % Better handling of variable-width fonts
}

\newcommand{\incfig}[1]{%
    \def\svgwidth{\columnwidth}
    \import{./Figures/}{#1.pdf_tex}
}
\theoremstyle{definition} % This style uses normal (non-italicized) text
\newtheorem{solution}{Solution}
\newtheorem{proposition}{Proposition}
\newtheorem{problem}{Problem}
\newtheorem{lemma}{Lemma}
\newtheorem{theorem}{Theorem}
\newtheorem{remark}{Remark}
\newtheorem{note}{Note}
\theoremstyle{plain} % Restore the default style for other theorem environments
%

% Theorem-like environments
% Title information
\title{}
\author{Jerich Lee}
\date{\today}

\begin{document}

\maketitle


\textbf{Problem 1. Find values of given functions (e.g. $\exp, \log, \sqrt{\cdot}, \sin, \cos$, etc.)}

\[
\textbf{Example: Compute } \log(1+i) \text{ (principal branch) and } \sqrt{1+i}.
\]

\textbf{Solution:}

\textbf{Step 1: Write } $1+i$ \text{ in polar form.}

We have 
\[
1+i = \sqrt{2}\, e^{i\pi/4}.
\]
Indeed, $|1+i| = \sqrt{1^2 + 1^2} = \sqrt{2}$ and 
$\arg(1+i) = \pi/4$ \text{ (since $1+i$ lies on the line $y=x$ in the first quadrant).}

\textbf{Step 2: Compute the principal logarithm.}

By definition, on the principal branch (i.e.\ $-\pi < \theta \le \pi$),
\[
\log(1+i) \;=\; \log\bigl(\sqrt{2} \, e^{i\pi/4}\bigr) \;=\; \log(\sqrt{2}) \;+\; i\left(\frac{\pi}{4}\right).
\]
Hence,
\[
\log(1+i) = \frac{1}{2}\ln(2) \;+\; i \frac{\pi}{4}.
\]

\textbf{Step 3: Compute} $\sqrt{1+i}$.

By definition, a square root of $1 + i$ satisfies
\[
(\sqrt{1+i})^2 = 1 + i.
\]
When $z = r e^{i\theta}$, one standard choice for the square root is
\[
\sqrt{z} = \sqrt{r}\, e^{i\theta/2},
\]
where $-\pi < \theta \le \pi$. Here we have $r = \sqrt{2}$ and $\theta = \pi/4$, so
\[
\sqrt{1+i} = \sqrt{\sqrt{2}}\, e^{i(\pi/4)/2} 
= 2^{1/4}\, e^{i\pi/8} 
= 2^{1/4}\Bigl[\cos\bigl(\tfrac{\pi}{8}\bigr) + i\,\sin\bigl(\tfrac{\pi}{8}\bigr)\Bigr].
\]

Thus, one principal value is
\[
\sqrt{1+i} = \sqrt[4]{2}\, e^{i\pi/8}.
\]

\textbf{Problem 2. Derive and verify identities/inequalities involving functions (e.g.\ $\cos^2 z + \sin^2 z = 1$, etc.)}

\[
\textbf{Example: Prove that } \cos^2 z + \sin^2 z = 1 \text{ for all complex } z.
\]

\textbf{Solution:}

We use the exponential definitions of sine and cosine for complex $z$:
\[
\sin z = \frac{e^{iz} - e^{-iz}}{2i}, 
\quad
\cos z = \frac{e^{iz} + e^{-iz}}{2}.
\]

\textbf{Step 1: Compute } $\sin^2 z$ \text{ and } $\cos^2 z$.

\[
\sin^2 z = \left(\frac{e^{iz} - e^{-iz}}{2i}\right)^2 
= -\frac{1}{4}\bigl(e^{2iz} - 2 + e^{-2iz}\bigr),
\]
\[
\cos^2 z = \left(\frac{e^{iz} + e^{-iz}}{2}\right)^2 
= \frac{1}{4}\bigl(e^{2iz} + 2 + e^{-2iz}\bigr).
\]

\textbf{Step 2: Sum the two expressions.}

\[
\cos^2 z + \sin^2 z 
= \frac{1}{4}\bigl(e^{2iz} + 2 + e^{-2iz}\bigr)
-\frac{1}{4}\bigl(e^{2iz} - 2 + e^{-2iz}\bigr).
\]
Combine like terms:
\[
= \frac{1}{4}\Bigl(e^{2iz} + 2 + e^{-2iz} - e^{2iz} + 2 - e^{-2iz}\Bigr)
= \frac{1}{4}\cdot(4) = 1.
\]

Hence,
\[
\cos^2 z + \sin^2 z = 1,
\]
as desired.

\textbf{Problem 3. Find the image of a set under a given function (e.g.\ $1/z, e^z, \log z, \sin z$, etc.)}

\[
\textbf{Example: Find the image of } \{x + i y : x > 0\} \text{ under } f(z) = \frac{1}{z}.
\]

\textbf{Solution:}

\textbf{Step 1: Let } z = x + i y \text{ with } x > 0.

Then
\[
f(z) = \frac{1}{z} = \frac{1}{x + i y}.
\]
We can rewrite this by multiplying numerator and denominator by the complex conjugate of the denominator:
\[
\frac{1}{x + i y} = \frac{x - i y}{(x + i y)(x - i y)} = \frac{x - i y}{x^2 + y^2}.
\]
Hence,
\[
f(z) = \frac{x}{x^2 + y^2} - i\,\frac{y}{x^2 + y^2}.
\]

\textbf{Step 2: Identify the set of all such points.}

If $u + i v = f(z)$, then
\[
u = \frac{x}{x^2 + y^2}, 
\quad 
v = -\frac{y}{x^2 + y^2}.
\]

\textbf{Step 3: Determine the condition on $(u, v)$ that comes from $x > 0$.}

We have
\[
x = \frac{u}{u^2 + v^2} 
\quad \text{and} \quad
y = -\frac{v}{u^2 + v^2}
\]
because $(u^2 + v^2)(x^2 + y^2) = 1$. The condition $x > 0$ translates to 
\[
\frac{u}{u^2 + v^2} > 0.
\]
Since $u^2 + v^2 > 0$ unless $u = v = 0$, the sign of $x$ is just the sign of $u$. Therefore, $x>0$ if and only if $u>0$.

\textbf{Step 4: Conclude the image.}

Thus, the image of $\{z : \Re(z) > 0\}$ under $f(z) = 1/z$ is 
\[
\{ w = u + i v : u < 0 \} \cup \{\infty\}.
\]
But let's be precise: if $x > 0$, then $u = x/(x^2+y^2)$ is also $> 0$ or $< 0$? Let's check the sign:

- $u = \frac{x}{x^2+y^2}$ is \emph{positive} for $x>0$ because $x^2+y^2>0$. 

Actually, this means $u > 0$ if $x>0$. 

Hence, the image is the \emph{right half-plane}
\[
\{ w = u + iv : u > 0\} 
\]
in the $w$-plane, together (sometimes) with the point at infinity in the extended complex plane. 

\textbf{Answer:} The right half-plane $x>0$ is mapped onto the right half-plane $u>0$.

\textbf{Problem 4. Find where $f'(z)$ exists and compute $f'(z)$ for a given function.}

\[
\textbf{Example: Let } f(z) = \overline{z}. \text{ Determine where $f$ is complex differentiable and find $f'(z)$.}
\]

\textbf{Solution:}

\textbf{Step 1: Express } $z = x + i y$, \text{ so } $\overline{z} = x - i y.$

We want to check the Cauchy-Riemann equations. Write $f(x+iy) = u(x,y) + i\,v(x,y)$ where
\[
u(x,y) = x, \quad v(x,y) = -y.
\]

\textbf{Step 2: Apply the Cauchy-Riemann equations.}

The Cauchy-Riemann equations for $u$ and $v$ are
\[
\frac{\partial u}{\partial x} = \frac{\partial v}{\partial y},
\quad
\frac{\partial u}{\partial y} = -\,\frac{\partial v}{\partial x}.
\]

Compute these partial derivatives:
\[
\frac{\partial u}{\partial x} = 1, \quad \frac{\partial u}{\partial y} = 0,
\quad \frac{\partial v}{\partial x} = 0, \quad \frac{\partial v}{\partial y} = -1.
\]

Hence the Cauchy-Riemann equations become
\[
1 = -1, 
\quad 
0 = 0.
\]
Clearly, $1 = -1$ is false. 

\textbf{Step 3: Conclude differentiability.}

Because the Cauchy-Riemann equations cannot be satisfied for any $(x,y)$, $f(z) = \overline{z}$ is \emph{not complex differentiable} at any point $z \in \mathbb{C}$ (in the sense of holomorphic functions).

\textbf{Answer:} $f'(z)$ does not exist anywhere. Hence, there is no point in $\mathbb{C}$ where $f$ is holomorphic, and we cannot assign a nonzero complex derivative.


\textbf{Problem 5. Find power series expansions for a given function (e.g.\ $1/(z+a)$, $1/(z+a)^n$, $e^z$, $\log z$, etc.)}

\[
\textbf{Example: Expand } f(z) = \frac{1}{z+1} \text{ in a power series around } z=0 \text{ and find its radius of convergence.}
\]

\textbf{Solution:}

\textbf{Step 1: Rewrite the function in a convenient form.}

We have
\[
\frac{1}{z+1} = \frac{1}{1 + z}.
\]
If $|z| < 1$, we can use the geometric series expansion:
\[
\frac{1}{1 + z} = \frac{1}{1 - (-z)} = \sum_{n=0}^\infty (-z)^n = \sum_{n=0}^\infty (-1)^n z^n.
\]

\textbf{Step 2: Write out the series explicitly.}

Hence, for $|z| < 1$,
\[
\frac{1}{z+1} = 1 - z + z^2 - z^3 + z^4 - \cdots = \sum_{n=0}^\infty (-1)^n \, z^n.
\]

\textbf{Step 3: Determine the radius of convergence.}

The geometric series $\sum_{n=0}^\infty (-z)^n$ converges if and only if $|-z| < 1$, i.e.\ $|z| < 1$. Thus, the radius of convergence of this expansion is $1$.

\textbf{Answer:} 
\[
\frac{1}{z+1} = \sum_{n=0}^\infty (-1)^n \, z^n, 
\quad \text{for } |z|<1,
\]
\[
\text{Radius of convergence} = 1.
\]
\section*{Problem 1}
\textbf{Find all values of } $\sqrt[3]{-8}$\textbf{.  Write each value in the form $a + i\,b$ with real $a,b$, and do not use trigonometric or inverse‐trigonometric functions in the final answer.}

\subsubsection*{Solution}

\paragraph{Step 1: Express $-8$ in polar form (for reference only).}
We know $-8$ can be viewed as $8\,e^{i\pi}$ in the complex plane, but we will ultimately give our final answers in rectangular form $a + i\,b$.

\paragraph{Step 2: General formula for the cube roots.}
In general, the $n$th roots of a nonzero complex number $re^{i\theta}$ are
\[
\sqrt[n]{r}\; e^{\,i\,\frac{\theta + 2\pi k}{n}},
\quad k = 0,1,\dots,n-1.
\]
For $r=8$, $\theta = \pi$, and $n=3$, we have
\[
\sqrt[3]{-8}
\;=\;
\sqrt[3]{8}\; e^{\,i\,\frac{\pi + 2\pi k}{3}}
\;=\;
2\, e^{\,i\,\frac{\pi + 2\pi k}{3}},
\quad k=0,1,2.
\]
Although we derived them via exponentials, we will rewrite each root as $a + i\,b$.

\paragraph{Step 3: Convert each root to rectangular form.}

\begin{itemize}
\item For $k=0$:
\[
2 \, e^{\,i\,\frac{\pi}{3}}
\;=\;
2\Bigl(\cos\!\tfrac{\pi}{3} + i\,\sin\!\tfrac{\pi}{3}\Bigr)
\;=\;
2\Bigl(\tfrac12 + i\,\tfrac{\sqrt{3}}{2}\Bigr)
\;=\;
1 \;+\; i\,\sqrt{3}.
\]

\item For $k=1$:
\[
2 \, e^{\,i\,\frac{\pi + 2\pi}{3}}
\;=\;
2 \, e^{\,i\,\pi}
\;=\;
2\,(-1 + 0\,i)
\;=\;
-2.
\]

\item For $k=2$:
\[
2 \, e^{\,i\,\frac{\pi + 4\pi}{3}}
\;=\;
2 \, e^{\,i\,\frac{5\pi}{3}}
\;=\;
2\Bigl(\cos\!\tfrac{5\pi}{3} + i\,\sin\!\tfrac{5\pi}{3}\Bigr)
\;=\;
2\Bigl(\tfrac12 - i\,\tfrac{\sqrt{3}}{2}\Bigr)
\;=\;
1 \;-\; i\,\sqrt{3}.
\]
\end{itemize}

Hence, the three cube roots of $-8$ in rectangular form are
\[
\boxed{\,1 + i\sqrt{3}, \quad -2, \quad 1 - i\sqrt{3}\,}.
\]

\vspace{1em}

\section*{Problem 2}
\textbf{Describe in geometric terms the set of all points $z$ in the complex plane such that}
\[
\lvert 2z - i \rvert \;=\; \lvert 2 + i\,z \rvert.
\]

\subsubsection*{Solution}

\paragraph{Step 1: Square both sides.}
We have
\[
|2z - i|^2 \;=\; |2 + i z|^2.
\]
Recall that for any complex $w$, $|w|^2 = w\,\overline{w}.$  Hence:
\[
(2z - i)\,\bigl(2\overline{z} - \overline{i}\bigr)
\;=\;
(2 + i z)\,\bigl(2 + i\,\overline{z}\bigr)^*.
\]
But carefully writing the conjugates, we get
\[
(2z - i)\,(2\overline{z} + i)
\;=\;
(2 + i\,z)\,(2 - i\,\overline{z}).
\]

\paragraph{Step 2: Expand both sides.}

\textbf{Left side:}
\[
(2z - i)(2\overline{z} + i)
= 4\,z\overline{z} \;+\; 2i\,z \;-\; 2i\,\overline{z} \;-\; i^2
= 4\,|z|^2 \;+\; 2i\,z \;-\; 2i\,\overline{z} \;+\; 1.
\]

\textbf{Right side:}
\[
(2 + i z)(2 - i\,\overline{z})
= 4 \;+\; 2\,i z \;-\; 2\,i\,\overline{z} \;-\; i^2\,z\,\overline{z}
= 4 \;+\; 2i\,z \;-\; 2i\,\overline{z} \;+\; |z|^2.
\]

\paragraph{Step 3: Equate and simplify.}
Equating the expanded forms,
\[
4\,|z|^2 \;+\; 2i\,z \;-\; 2i\,\overline{z} \;+\; 1
\;=\;
4 \;+\; |z|^2 \;+\; 2i\,z \;-\; 2i\,\overline{z}.
\]
Cancelling the common terms $2i\,z - 2i\,\overline{z}$ on both sides, we get
\[
4\,|z|^2 + 1 = 4 + |z|^2
\quad\Longrightarrow\quad
3\,|z|^2 = 3
\quad\Longrightarrow\quad
|z|^2 = 1.
\]
Hence,
\[
|z| = 1.
\]

\paragraph{Conclusion.}
The condition $|z|=1$ describes the \textbf{circle of radius 1 centered at the origin} in the complex plane.  Therefore, the set of all points $z$ satisfying $|2z - i| = |2 + i z|$ is precisely the unit circle,
\[
\boxed{\{\,z : |z| = 1\}.}
\]
\begin{center}
    {\Large \textbf{Solutions}}
    \end{center}
    
    \section*{Problem 1}
    \[
    \text{Evaluate the limit or show it does not exist:}
    \quad
    \lim_{z \to 1} \frac{\,z\,\overline{z}\;-\;1\,}{\,z - 1\,}.
    \]
    
    \subsection*{Solution}
    Let $z = x + i\,y$.  Then $\overline{z} = x - i\,y$, so
    \[
    z \,\overline{z} = (x + i\,y)\,(x - i\,y) = x^2 + y^2.
    \]
    Hence the expression becomes
    \[
    \frac{\,z\overline{z} - 1\,}{\,z-1\,}
    \;=\;
    \frac{x^2 + y^2 - 1}{\,x + i\,y - 1\,}.
    \]
    We examine the limit as $z \to 1$, i.e.\ $(x,y) \to (1,0)$, along different paths:
    
    \medskip
    \noindent
    \textbf{Path 1: Along the real axis.}  
    Set $y=0$ and let $x \to 1$.  Then $z\overline{z} - 1 = x^2 - 1 = (x-1)(x+1)$, and $z-1 = (x-1)$.  Thus
    \[
    \lim_{\substack{x \to 1 \\ y = 0}}
    \frac{(x-1)(x+1)}{x-1}
    =
    \lim_{x \to 1}
    (x+1)
    = 2.
    \]
    
    \medskip
    \noindent
    \textbf{Path 2: Along the line } $z = 1 + i\,y$.
    Here $x=1$, so $z-1 = i\,y$.  Also
    \[
    z\overline{z} - 1 = (1 + i\,y)(1 - i\,y) - 1 = (1 + y^2) - 1 = y^2.
    \]
    Thus
    \[
    \lim_{\substack{y \to 0}}
    \frac{\,y^2\,}{i\,y}
    =
    \lim_{y \to 0}
    \frac{y}{i}
    =
    0.
    \]
    
    Since the two path‐based limits are different (\(2\) along the real axis versus \(0\) along the vertical line), 
    the original limit \emph{does not exist}.  In other words,
    \[
    \boxed{\text{The limit does not exist.}}
    \]
    
    \bigskip
    
    \section*{Problem 2}
    \[
    \text{Derive a formula for } \arctan(z) \text{ (without using inverse trigonometric functions).}
    \]
    
    \subsection*{Solution}
    We want a closed‐form expression for $w = \arctan(z)$, i.e.\ a $w$ such that
    \[
    z = \tan(w) \;=\; \frac{\sin(w)}{\cos(w)}.
    \]
    Using the exponential form of sine and cosine,
    \[
    \sin(w) = \frac{\,e^{i w} - e^{-i w}\,}{2\,i},
    \quad
    \cos(w) = \frac{\,e^{i w} + e^{-i w}\,}{2}.
    \]
    Hence
    \[
    \tan(w)
    = \frac{\sin(w)}{\cos(w)}
    = \frac{\,e^{i w} - e^{-i w}\,}{\,i\,(\,e^{i w} + e^{-i w}\,)}.
    \]
    Thus
    \[
    z = \frac{\,e^{i w} - e^{-i w}\,}{\,i\,(e^{i w} + e^{-i w})}.
    \]
    Cross‐multiplying,
    \[
    z\,i \,\bigl(e^{i w} + e^{-i w}\bigr)
    = e^{i w} - e^{-i w}.
    \]
    Rewrite $e^{-i w}$ in each term to collect factors of $e^{2 i w}$:
    \[
    z\,i \,\Bigl(e^{i w} + e^{-i w}\Bigr)
    = e^{i w} - e^{-i w}
    \;\;\Longrightarrow\;\;
    z\,i \,\Bigl(e^{2 i w} + 1\Bigr)
    = e^{2 i w} - 1,
    \]
    where we multiplied everything by $e^{i w}$.  Rearrange to solve for $e^{2 i w}$:
    \[
    e^{2 i w} \;-\; z\,i \,e^{2 i w}
    = -1 \;-\; z\,i
    \;\;\Longrightarrow\;\;
    e^{2 i w}\,\bigl(1 \;-\; z\,i\bigr)
    =  -\bigl(1 + z\,i\bigr).
    \]
    Hence,
    \[
    e^{2 i w} 
    = \frac{-(1 + i\,z)}{\,1 - i\,z\,}
    = \frac{\,1 + i\,z\,}{\,1 - i\,z\,} \times (-1)
    = \frac{\,1 + i\,z\,}{\,1 - i\,z\,} \;(\text{the factor $-1$ can be absorbed if we wish}).
    \]
    A more standard form (dropping the extra \(-1\) factor, which only changes $w$ by $\pi$, consistent with the multi‐valued nature of $\tan$) is
    \[
    e^{2 i w} 
    = \frac{1 + i\,z}{\,1 - i\,z\,}.
    \]
    Take the natural logarithm on both sides:
    \[
    2 i\,w = \ln\!\Bigl(\tfrac{1 + i\,z}{1 - i\,z}\Bigr),
    \]
    so
    \[
    w = \arctan(z)
    = \frac{1}{2\,i}\,\ln\!\Bigl(\tfrac{1 + i\,z}{\,1 - i\,z}\Bigr).
    \]
    Sometimes one writes
    \[
    \frac{1}{2\,i} = -\frac{i}{2},
    \]
    and so equivalently:
    \[
    \arctan(z) 
    = -\frac{i}{2}\,\ln\!\Bigl(\tfrac{1 + i\,z}{\,1 - i\,z}\Bigr)
    = \frac{i}{2}\,\ln\!\Bigl(\tfrac{1 - i\,z}{\,1 + i\,z}\Bigr).
    \]
    In any case, the principal formula is
    \[
    \boxed{\displaystyle
    \arctan(z)
    = \frac{1}{2\,i}\,\ln\!\Bigl(\frac{\,1 + i\,z\,}{\,1 - i\,z\,}\Bigr).
    }
    \]
    This is the standard complex‐logarithm definition of $\arctan(z)$, avoiding any direct use of inverse trigonometric functions.
    \begin{center}
        {\Large \textbf{Solutions}}
      \end{center}
      
      \section*{Problem 1.}
      \[
      \text{Find the power series (about $z=0$) of } 
      f(z) \;=\; \frac{1}{\,z^{2} \;-\; 3z \;+\; 2\,}.
      \]
      \subsection*{Solution}
      
      \paragraph{Step 1: Factor the denominator and use partial fractions.}
      Notice that
      \[
      z^2 - 3z + 2 
      = (z-1)(z-2).
      \]
      Hence
      \[
      f(z) \;=\; \frac{1}{(z-1)(z-2)} 
      =\; \frac{A}{\,z-2\,} \;+\; \frac{B}{\,z-1\,}.
      \]
      We find $A$ and $B$ by the usual cover‐up or by matching coefficients:
      \[
      1 = A\,(z-1) \;+\; B\,(z-2).
      \]
      Setting $z=2$ gives
      \[
      1 = A\,(2-1) + B\,(0) \quad \Longrightarrow \quad A = 1.
      \]
      Setting $z=1$ gives
      \[
      1 = A\,(0) + B\,(1-2) \quad \Longrightarrow \quad B = -1.
      \]
      Thus
      \[
      f(z) \;=\; \frac{1}{\,z-2\,} \;-\; \frac{1}{\,z-1\,}.
      \]
      
      \paragraph{Step 2: Expand each partial‐fraction term about $z=0$.}
      
      We want geometric‐series expansions valid for sufficiently small $|z|$.  Observe:
      \[
      \frac{1}{z-2}
      = \frac{1}{-2\bigl(1 - \tfrac{z}{2}\bigr)}
      = -\frac{1}{2}\cdot \frac{1}{\,1 - (z/2)\,}
      = -\frac{1}{2}\,\sum_{n=0}^\infty \left(\frac{z}{2}\right)^{n}
      \;=\;
      \sum_{n=0}^\infty \bigl(-\tfrac{1}{2^{\,n+1}}\bigr)\,z^n.
      \]
      This series converges for $\left|\frac{z}{2}\right|<1$, i.e.\ $|z|<2$.
      
      Meanwhile,
      \[
      \frac{1}{z-1}
      = \frac{1}{-1\bigl(1 - z\bigr)}
      = -\,\frac{1}{\,1 - z\,}
      = -\,\sum_{n=0}^\infty z^n
      \quad
      (\text{for }|z|<1).
      \]
      
      Hence
      \[
      -\frac{1}{\,z-1\,} 
      = \sum_{n=0}^\infty z^n,
      \]
      but remember our $f(z)$ has a minus sign in front of $1/(z-1)$, so
      \[
      -\;\frac{1}{\,z-1\,} 
      = \sum_{n=0}^\infty z^n.
      \]
      
      \paragraph{Step 3: Combine the two expansions.}
      
      Thus for $|z|<1$ (so that both expansions are valid), we have
      \[
      f(z)
      = \frac{1}{\,z-2\,} - \frac{1}{\,z-1\,}
      = \left(\sum_{n=0}^\infty -\frac{1}{\,2^{n+1}}\,z^n\right)
      \;-\;
      \left(-\,\sum_{n=0}^\infty z^n\right)
      =
      \sum_{n=0}^\infty 
      \Bigl(-\tfrac{1}{2^{n+1}} + 1\Bigr) \,z^n.
      \]
      It is often more natural to factor out the minus sign in the first sum.  Equivalently, one can write
      \[
      f(z)
      = \sum_{n=0}^\infty \Bigl(1 - \frac{1}{2^{n+1}}\Bigr)\,z^n.
      \]
      Either way, we see a power‐series expansion about $z=0$.  
      
      \paragraph{Step 4: Domain of convergence.}
      The second term $-1/(z-1)$ requires $|z|<1$ for its geometric series, while the first term $1/(z-2)$ requires $|z|<2$.  The stricter condition is $|z|<1$.  Therefore the final power series is
      \[
      \boxed{
      f(z) 
      = \sum_{n=0}^\infty \Bigl(1 \;-\; 2^{-(n+1)}\Bigr)\,z^n
      \quad\text{for } |z|<1.
      }
      \]
      
      \bigskip
      
      \section*{Problem 2.}
      \[
      \text{Let $f = u + i\,v$ be analytic on a domain, and suppose }2u + 3v = 4.
      \;\text{Prove $f$ is constant.}
      \]
      
      \subsection*{Solution}
      
      Because $f$ is analytic, $u$ and $v$ satisfy the Cauchy‐Riemann (CR) equations:
      \[
      u_x = v_y, 
      \quad
      u_y = -\,v_x.
      \]
      The condition $2u + 3v = 4$ holds throughout the domain.  Differentiate it with respect to $x$ and $y$:
      
      \[
      \frac{\partial}{\partial x}(2u + 3v) = 0 
      \quad\Longrightarrow\quad
      2\,u_x + 3\,v_x = 0,
      \]
      \[
      \frac{\partial}{\partial y}(2u + 3v) = 0 
      \quad\Longrightarrow\quad
      2\,u_y + 3\,v_y = 0.
      \]
      
      Using the CR equations to rewrite $u_x$ and $u_y$:
      \[
      u_x = v_y, 
      \quad 
      u_y = -v_x.
      \]
      Hence the above equations become
      \[
      2v_y + 3v_x = 0,
      \quad
      2(-v_x) + 3v_y = 0.
      \]
      Or more neatly,
      \[
      \begin{cases}
      2\,v_y + 3\,v_x = 0,\\
      -2\,v_x + 3\,v_y = 0.
      \end{cases}
      \]
      This is a homogeneous linear system in $v_x,\,v_y$.  In matrix form:
      \[
      \begin{pmatrix}2 & 3\\[4pt]-2 & 3\end{pmatrix}
      \begin{pmatrix}v_y\\[2pt]v_x\end{pmatrix}
      = \begin{pmatrix}0\\0\end{pmatrix}.
      \]
      The determinant of the coefficient matrix is $2\cdot 3 - 3\cdot(-2) = 6+6=12\neq0$, so the only solution is $v_x=0$ and $v_y=0$.  Thus $v$ is constant in the domain; by $u_x=v_y=0$ we get $u_x=0$, and by $u_y=-v_x=0$ we get $u_y=0$.  Therefore $u$ is also constant.
      
      Since $u$ and $v$ are both constant, $f = u + i\,v$ is constant throughout the (connected) domain.  Hence,
      \[
      \boxed{f \text{ is constant.}}
      \]

      \section*{Problem 1}
\[
\text{Show that } \cos x + \cos(3x) + \cdots + \cos\!\bigl((2n+1)x\bigr)
\;=\;
\frac{\sin\!\bigl(2(n+1)x\bigr)}{2\,\sin x}.
\]

\subsection*{Solution}

\paragraph{Step 1: Rewrite each cosine in exponential form.}
Recall that $\cos \theta = \Re\!\bigl(e^{i\theta}\bigr)$.  Hence,
\[
\cos x \;+\; \cos(3x) \;+\;\cdots\;+\; \cos\!\bigl((2n+1)x\bigr)
\;=\;
\Re\!\Bigl(e^{i x} + e^{i (3x)} + \cdots + e^{i((2n+1)x)}\Bigr).
\]
Factor out $e^{i x}$:
\[
=\;
\Re\!\Bigl(\,e^{i x} \bigl(1 + e^{2i x} + e^{4 i x} + \cdots + e^{2n i x}\bigr)\Bigr).
\]

\paragraph{Step 2: Sum the finite geometric series.}
Inside the parentheses is a geometric sum with ratio $e^{2i x}$:
\[
1 + e^{2i x} + e^{4 i x} + \cdots + e^{2n i x}
\;=\;
\frac{\,e^{2(n+1)i x} - 1\,}{\,e^{2 i x} - 1\,}
\quad
(\text{valid provided } e^{2i x}\neq 1).
\]
Hence the sum becomes
\[
\Re\!\Bigl(
e^{i x}
\;\frac{\,e^{2(n+1)i x} - 1\,}{\,e^{2 i x} - 1\,}
\Bigr).
\]

\paragraph{Step 3: Simplify the denominator.}
We know
\[
e^{2 i x} - 1 = e^{i x}(e^{i x} - e^{-i x})
= e^{i x}\,\bigl(2i \sin x\bigr)
= 2i\,e^{i x}\,\sin x.
\]
Thus
\[
\frac{1}{\,e^{2 i x} - 1\,}
= \frac{1}{\,2i\, e^{i x}\,\sin x\,}
= \frac{1}{\,2i\,\sin x\,}\,\cdot \frac{1}{\,e^{i x}\,}.
\]
So
\[
e^{i x}\,\frac{\,e^{2(n+1)i x} - 1\,}{\,e^{2 i x} - 1\,}
= \frac{\,e^{i x}\,(\,e^{2(n+1)i x} - 1\,)}{\,e^{2 i x} - 1\,}
= \frac{e^{2(n+1)i x} - 1}{\,2i\,\sin x\,}.
\]
Therefore our sum is
\[
\Re\!\Bigl(
\frac{\,e^{2(n+1)i x} - 1\,}{\,2i\,\sin x\,}
\Bigr).
\]

\paragraph{Step 4: Take the real part.}
Write
\[
e^{2(n+1)i x} - 1
= e^{i\bigl(2(n+1)x\bigr)} - 1.
\]
The real part of $\frac{z}{i}$ is the same as $-\text{Im}(z)$, but a standard approach is to recall
\[
e^{i\alpha} - 1 
= e^{i\alpha/2}\Bigl(e^{i\alpha/2} - e^{-i\alpha/2}\Bigr)
= e^{i\alpha/2}(2i \sin(\alpha/2)).
\]
Then
\[
\Re\!\bigl(e^{i\alpha} - 1\bigr)
= \Re\!\bigl(2i\,e^{i\alpha/2}\,\sin(\alpha/2)\bigr).
\]
However, in this classic identity for summing cosines, one sees directly that
\[
\frac{\,e^{2(n+1)i x} - 1\,}{2i}
= \frac{1}{2i}\bigl(e^{2(n+1)i x} - 1\bigr)
= \frac{1}{2}\Bigl(\frac{e^{2(n+1)i x} - 1}{i}\Bigr)
= \frac{1}{2}\,\Bigl(e^{(n+1)i x}\,\frac{\,e^{(n+1)i x} - e^{-(n+1)i x}\,}{i}\Bigr),
\]
and its real part is precisely $\sin\!\bigl(2(n+1)x\bigr)/2$.  Dividing by $\sin x$ then yields
\[
\cos x + \cos(3x) + \cdots + \cos\bigl((2n+1)x\bigr)
= \frac{\sin\!\bigl(2(n+1)x\bigr)}{2\,\sin x}.
\]
Hence
\[
\boxed{
\cos x + \cos(3x) + \cdots + \cos\bigl((2n+1)x\bigr)
= \frac{\sin\!\bigl(2(n+1)x\bigr)}{2\,\sin x}
}.
\]

\bigskip

\section*{Problem 2}
\[
\text{Compute } \bigl(\sqrt{3} + i\bigr)^{100}\text{ in rectangular (}\;a + i\,b\text{) form}.
\]

\subsection*{Solution}

\paragraph{Step 1: Express $\sqrt{3}+i$ in polar form.}
We observe
\[
|\sqrt{3} + i|
= \sqrt{(\sqrt{3})^2 + 1^2}
= \sqrt{3 + 1}
= 2.
\]
Also the argument $\theta = \arg(\sqrt{3}+i)$ satisfies
\[
\cos\theta = \frac{\sqrt{3}}{2}, 
\quad
\sin\theta = \frac{1}{2},
\]
which corresponds to $\theta = \frac{\pi}{6}$ (30 degrees).  Thus
\[
\sqrt{3} + i = 2\,e^{\,i\pi/6}.
\]

\paragraph{Step 2: Raise to the 100th power.}
Hence
\[
(\sqrt{3} + i)^{100}
= \bigl(2\,e^{\,i\pi/6}\bigr)^{100}
= 2^{100}\,e^{\,i\,\frac{100\pi}{6}}
= 2^{100}\,e^{\,i\,\frac{50\pi}{3}}
= 2^{100}\,e^{\,i\,\bigl(16\pi + \tfrac{2\pi}{3}\bigr)}.
\]
Since $e^{\,i\,16\pi} = 1$, we get
\[
(\sqrt{3} + i)^{100}
= 2^{100}\,e^{\,i\,\frac{2\pi}{3}}
= 2^{100}\Bigl(\cos\tfrac{2\pi}{3} + i\,\sin\tfrac{2\pi}{3}\Bigr).
\]

\paragraph{Step 3: Convert back to rectangular form.}
Recall
\[
\cos\!\bigl(\tfrac{2\pi}{3}\bigr) = -\tfrac12,
\quad
\sin\!\bigl(\tfrac{2\pi}{3}\bigr) = \tfrac{\sqrt{3}}{2}.
\]
Thus
\[
2^{100}\,e^{\,i\,\tfrac{2\pi}{3}}
= 2^{100}\Bigl(-\tfrac12 + i\,\tfrac{\sqrt{3}}{2}\Bigr)
= 2^{100}\,\bigl(-\tfrac12\bigr) \;+\; i\,2^{100}\,\bigl(\tfrac{\sqrt{3}}{2}\bigr)
= -\,2^{99} \;+\; i\,2^{99}\,\sqrt{3}.
\]
Hence
\[
\boxed{(\sqrt{3} + i)^{100}
= -\,2^{99} \;+\; i\,2^{99}\,\sqrt{3}.}
\]

\section*{Problem 1}
\[
\text{Solve the locus: }|z - 4| \;=\; 4\,|z|\quad\text{in the complex plane}.
\]
\subsection*{Solution}

Let $z = x + i\,y$, so $|z| = \sqrt{x^2 + y^2}.$ The equation becomes
\[
|\,x + i\,y - 4\,| \;=\; 4\,\sqrt{x^2 + y^2}.
\]
Square both sides:
\[
|\,z - 4\,|^2 \;=\; 16\,|z|^2
\;\;\Longrightarrow\;\;
(z-4)\,\overline{(z-4)} \;=\; 16\,z\,\overline{z}.
\]
That is
\[
|z|^2 \;-\; 4\,z \;-\; 4\,\overline{z} \;+\;16 
\;=\; 16\,|z|^2,
\]
since $|z|^2 = z\,\overline{z}.$  Rearrange:
\[
-4\,(z + \overline{z}) \;+\;16 
\;=\; 16\,|z|^2 - |z|^2
\;=\;
15\,|z|^2.
\]
But $z + \overline{z} = 2\,x$, so
\[
15\,(x^2 + y^2) \;-\; 8\,x \;+\;16 = 0
\;\;\Longrightarrow\;\;
15\,x^2 + 15\,y^2 - 8\,x + 16 = 0.
\]
Rearrange to complete the square in $x$:
\[
15\,x^2 - 8\,x + 16 + 15\,y^2 = 0
\;\;\Longrightarrow\;\;
15\Bigl(x^2 - \tfrac{8}{15}x\Bigr) + 15\,y^2 + 16 = 0.
\]
Write
\[
x^2 - \tfrac{8}{15}x 
\;=\;
x^2 - \tfrac{8}{15}x + \Bigl(\tfrac{4}{15}\Bigr)^2 
\;-\;
\Bigl(\tfrac{4}{15}\Bigr)^2.
\]
Hence
\[
15\Bigl[\Bigl(x + \tfrac{4}{15}\Bigr)^2 - \Bigl(\tfrac{4}{15}\Bigr)^2\Bigr] 
+ 15\,y^2 + 16 = 0.
\]
Distribute:
\[
15\Bigl(x + \tfrac{4}{15}\Bigr)^2 - 15\Bigl(\tfrac{4}{15}\Bigr)^2 + 15\,y^2 + 16 = 0.
\]
But $\bigl(\tfrac{4}{15}\bigr)^2 = \tfrac{16}{225},$ so
\[
-\,15\,\tfrac{16}{225} = -\,\tfrac{16}{15}, 
\quad
\text{and then}
\quad
16 + \Bigl(-\,\tfrac{16}{15}\Bigr) = \tfrac{240}{15} - \tfrac{16}{15} = \tfrac{224}{15}.
\]
So the equation becomes
\[
15\Bigl(x + \tfrac{4}{15}\Bigr)^2 + 15\,y^2
= \tfrac{16}{15}\,\cdot 15 + \tfrac{224}{15}
\;-\; \tfrac{224}{15},
\]
(you can reorganize the constants in one step).  In fact, a simpler direct approach is:

\[
15\Bigl(x + \tfrac{4}{15}\Bigr)^2 + 15\,y^2 
= 16 + \tfrac{16}{15},
\]
or
\[
\Bigl(x + \tfrac{4}{15}\Bigr)^2 + y^2
= \frac{16 + \tfrac{16}{15}}{15}
= \frac{\tfrac{240}{15} + \tfrac{16}{15}}{15}
= \frac{256}{15}\,\frac{1}{15}
= \Bigl(\tfrac{16}{15}\Bigr)^2.
\]
Hence the locus is the circle
\[
\boxed{
\Bigl|\,z + \tfrac{4}{15}\Bigr| = \tfrac{16}{15}
,\;}
\]
i.e.\ a circle of radius $\tfrac{16}{15}$ centered at $\bigl(-\tfrac{4}{15},0\bigr)$ in the $(x,y)$‐plane.

\bigskip

\section*{Problem 2}
\[
\text{Find }\cos 72^\circ\;\text{ and }\;\sin 72^\circ.
\]
(Recall $72^\circ = \frac{2\pi}{5}$ radians.)

\subsection*{Solution}

One standard method uses the fact that $e^{i(2\pi/5)}$ is a primitive 5th root of unity on the unit circle.  Equivalently, let 
\[
\theta = 72^\circ = \frac{2\pi}{5}, 
\quad
\text{then }
z = \cos \theta + i\,\sin\theta
\text{ satisfies } z^5 = 1.
\]
We want $x = \cos\theta$ and $y = \sin\theta$.  Since $x^2 + y^2=1$, solving $z^5=1$ leads to a classic result:

\[
\cos(72^\circ) 
= \frac{\sqrt{5} - 1}{4},
\quad
\sin(72^\circ)
= \sqrt{\,1 - \Bigl(\frac{\sqrt{5} - 1}{4}\Bigr)^2}
= \frac{\sqrt{\,10 + 2\,\sqrt{5}\,}}{4}.
\]

\paragraph{Sketch of a derivation:}
We know $z^5=1$ and $|z|=1$, so $z$ is one of the points $e^{\,i\,2\pi k/5}$, $k=0,1,2,3,4$.  The one with argument $2\pi/5$ has real part $x>0$.  A polynomial approach (or standard golden‐ratio identities) yields
\[
\cos\!\bigl(\tfrac{2\pi}{5}\bigr) 
= \frac{\sqrt{5} - 1}{4}.
\]
Then $\sin\bigl(\tfrac{2\pi}{5}\bigr)$ is positive and found by $\sqrt{\,1 - x^2}$.  

Hence
\[
\boxed{
\cos 72^\circ = \frac{\sqrt{5} - 1}{4},
\quad
\sin 72^\circ = \frac{\sqrt{10 + 2\,\sqrt{5}}}{4}.
}
\]

\noindent
\textbf{Problem Statement.}
Define the function
\[
f(z) \;=\;
\begin{cases}
\dfrac{z^{3} + i}{\,z - i\,}, & z \neq i, \\[6pt]
-3, & z = i.
\end{cases}
\]
Find all points of continuity of \(f\) in the complex plane.

\medskip

\noindent
\textbf{Solution.}
\begin{enumerate}
\item \emph{Continuity for \(z \neq i\).}

When \(z \neq i\), the formula 
\[
f(z) \;=\; \frac{z^3 + i}{\,z - i\,}
\]
is a quotient of polynomials.  As long as the denominator \(z - i \neq 0,\) this quotient is continuous.  Therefore \(f\) is continuous for all \(z \neq i\).

\item \emph{Continuity at \(z = i\).}

We must check whether
\[
\lim_{z \to i} \frac{z^3 + i}{\,z - i\,} \;=\; f(i) \;=\; -3.
\]
A convenient way is to perform polynomial division or use the factorization 
\(\,z^3 - i^3 = (z - i)(z^2 + i\,z + i^2)\) and observe that
\[
z^3 + i \;=\; z^3 - \bigl(i^3\bigr)
\quad\text{since } i^3 = -i \implies -\,i^3 = i.
\]
Hence
\[
\frac{z^3 + i}{\,z - i\,}
\;=\;
\frac{\,z^3 - i^3\,}{\,z - i\,}
\;=\;
z^2 \;+\; i\,z \;+\; i^2
\;=\;
z^2 \;+\; i\,z \;-\; 1.
\]
Now taking the limit as \(z \to i\),
\[
\lim_{z \to i} \bigl(z^2 + i\,z - 1\bigr)
= i^2 \;+\; i\cdot i \;-\; 1
= -1 \;-\;1 \;-\;1
= -3.
\]
This matches exactly the value \(f(i) = -3.\)

Hence the limit exists and equals the assigned value at \(z = i\).  Therefore \(f\) is continuous at \(z = i\) as well.

\end{enumerate}

\noindent
\textbf{Conclusion.}
The function \(f\) is continuous at every point \(z \in \mathbb{C}\).  In particular, there is no point of discontinuity.

\section*{1) \; $\displaystyle \lim_{z \to 0}\,\frac{\Re\bigl(z^2\bigr)}{z^2}$}

\paragraph{Claim:} This limit does \emph{not} exist.

\medskip
\noindent
\textbf{Reason (Path Dependence).}
Let $z = x \in \mathbb{R}$ tend to $0$.  
Then $z^2 = x^2$ is real, hence $\Re(z^2) = x^2$.  So
\[
\frac{\Re(z^2)}{z^2} 
= \frac{x^2}{x^2} 
= 1,
\quad \text{as }x \to 0 \ (\text{along the real axis}).
\]
By contrast, let $z = (1+i)\,t$ with $t \in \mathbb{R}$ tend to $0$.  
Then
\[
z^2 = \bigl((1+i)\,t\bigr)^2 
= (1 + 2i + i^2)\,t^2
= 2i\,t^2.
\]
Thus $\Re(z^2) = \Re(2i\,t^2)=0$, while $z^2 = 2i\,t^2 \neq 0$.  Hence 
\[
\frac{\Re(z^2)}{z^2}
= \frac{0}{2i\,t^2} 
= 0.
\]
So along this line, the ratio goes to $0$.  
Because we get $1$ by one path and $0$ by another, the limit does not exist.

\bigskip

\section*{2) \; $\displaystyle \lim_{z \to i}\,\frac{\Re\bigl(z^2\bigr)}{z^2}$}

\paragraph{Claim:} This limit \emph{does} exist, and equals $1$.

\medskip
\noindent
\textbf{Reason (Direct Substitution).}
Both $\Re(z^2)$ and $z^2$ are continuous functions of $z$, and $z^2 \neq 0$ at $z=i$ (since $i^2=-1\neq 0$).  
Hence
\[
\lim_{z\to i} \frac{\Re(z^2)}{z^2}
= \frac{\Re(i^2)}{\,i^2\,}
= \frac{\Re(-1)}{-1}
= \frac{-1}{-1}
= 1.
\]

\bigskip

\section*{3) \; $\displaystyle \lim_{|z|\to \infty}\,\frac{\Re\bigl(z^2\bigr)}{z^2}$}

\paragraph{Claim:} This limit \emph{does not} exist (again path‐dependence).

\medskip
\noindent
\textbf{Reason (Two Different Directions to Infinity).}
Let $z = x \in \mathbb{R}$ with $x \to +\infty$. Then $z^2 = x^2$, so 
\(\Re(z^2) = x^2\), and 
\[
\frac{\Re(z^2)}{z^2} = \frac{x^2}{x^2} = 1.
\]
On the other hand, let $z = (1+i)\,x$ with $x \to +\infty$. Then
\[
z^2 = (1+i)^2\,x^2 = (1 + 2i + i^2)\,x^2 = 2i\,x^2,
\]
and so $\Re(z^2)=0$, while $z^2=2i\,x^2$. Hence
\[
\frac{\Re(z^2)}{z^2} = \frac{0}{2i\,x^2} = 0.
\]
Therefore the value of the ratio tends to $1$ along the real axis, but tends to $0$ along the line $z=(1+i)\,x$.  No single limit exists as $|z|\to\infty$.

\section*{Verification of $\bigl|\cosh z\bigr|^2 = \sinh^2 x + \cos^2 y$}

\paragraph{Setup.} Let $z = x + i\,y$, where $x,y \in \mathbb{R}$. Recall
\[
\cosh z \;=\; \frac{\,e^z + e^{-z}\,}{2}.
\]
We wish to show
\[
\bigl|\cosh z\bigr|^2
\;=\;
\sinh^2 x
\;+\;
\cos^2 y.
\]

\paragraph{Step 1: Compute $|\cosh z|^2$.}
By definition,
\[
|\cosh z|^2
\;=\;
\cosh z \,\overline{\cosh z}
\;=\;
\frac{1}{4}\,\bigl(e^z + e^{-z}\bigr)\,\bigl(e^{\overline{z}} + e^{-\overline{z}}\bigr).
\]
Since $z = x + i\,y$, its complex conjugate is $\overline{z} = x - i\,y$. Thus
\[
z + \overline{z} = 2x,
\quad
z - \overline{z} = 2i\,y.
\]
Hence
\[
\bigl(e^z + e^{-z}\bigr)\,\bigl(e^{\overline{z}} + e^{-\overline{z}}\bigr)
=
e^{\,z + \overline{z}} \;+\; e^{\,z - \overline{z}} \;+\; e^{-\,z + \overline{z}} \;+\; e^{-\,z - \overline{z}}
=
e^{\,2x} \;+\; e^{\,2i\,y} \;+\; e^{-\,2i\,y} \;+\; e^{-\,2x}.
\]
Noting that $e^{\,2i\,y} + e^{-\,2i\,y} = 2\cos(2y)$ and $e^{\,2x} + e^{-\,2x} = 2\cosh(2x)$, we get
\[
|\cosh z|^2 
= \frac{1}{4}\,\Bigl(2\,\cosh(2x) + 2\,\cos(2y)\Bigr)
= \frac{1}{2}\,\bigl(\cosh(2x) + \cos(2y)\bigr).
\]

\paragraph{Step 2: Rewrite in terms of $\sinh^2 x$ and $\cos^2 y$.}
Recall the standard identities:
\[
\cosh(2x) 
= 1 + 2\,\sinh^2 x,
\quad
\cos(2y) 
= 1 - 2\,\sin^2 y 
= 2\,\cos^2 y - 1.
\]
From these, one checks directly that
\[
\sinh^2 x + \cos^2 y
= \frac{\cosh(2x) - 1}{2} + \frac{1 + \cos(2y)}{2}
= \frac{\cosh(2x) + \cos(2y)}{2}.
\]
Hence
\[
|\cosh z|^2 
= \frac{1}{2}\,\bigl(\cosh(2x) + \cos(2y)\bigr)
= \sinh^2 x + \cos^2 y.
\]

\paragraph{Conclusion.} 
Thus for $z = x + i\,y$,
\[
\boxed{\,|\cosh z|^2 \;=\; \sinh^2 x \;+\; \cos^2 y\,}.
\]
\section*{Chain Rule in Complex Analysis}

\paragraph{Statement.}
Let $f$ be complex differentiable at $z_0$ and let $g$ be complex differentiable at $w_0 = f(z_0)$. Then $g \circ f$ is complex differentiable at $z_0$ and
\[
(g \circ f)'(z_0) \;=\; g'\bigl(f(z_0)\bigr)\,\cdot\,f'(z_0).
\]

\paragraph{Proof (via Limit Definition).}

\begin{enumerate}
\item \textbf{Set‐up the difference quotient.}
By definition of the derivative of $g\circ f$ at $z_0$, we must show
\[
\lim_{z \to z_0} \frac{g\bigl(f(z)\bigr) \;-\; g\bigl(f(z_0)\bigr)}{\,z - z_0\,}
\;=\;
g'\bigl(f(z_0)\bigr)\,f'(z_0).
\]

\item \textbf{Use the differentiability of $f$ at $z_0$.}
Because $f$ is differentiable at $z_0$, we have
\[
f(z) - f(z_0) 
\;=\;
f'(z_0)\,(z - z_0) \;+\; \varepsilon(z),
\]
where $\varepsilon(z)$ satisfies
\[
\lim_{z \to z_0} \frac{\varepsilon(z)}{\,z - z_0\,} = 0.
\]
Equivalently,
\[
f(z) = f(z_0) + f'(z_0)\,(z - z_0) + \varepsilon(z).
\]

\item \textbf{Apply the differentiability of $g$ at $w_0 = f(z_0)$.}
Since $g$ is differentiable at $w_0$, for $w$ near $w_0$ we can write
\[
g(w) - g(w_0) 
\;=\;
g'(w_0)\,\bigl(w - w_0\bigr) \;+\; \delta(w),
\]
where $\delta(w)$ satisfies
\[
\lim_{w \to w_0} \frac{\delta(w)}{\,w - w_0\,} = 0.
\]

\item \textbf{Substitute $w = f(z)$.}
In our problem, $w_0 = f(z_0)$ and $w = f(z)$.  So
\[
g\bigl(f(z)\bigr) - g\bigl(f(z_0)\bigr)
\;=\;
g'\bigl(f(z_0)\bigr)\,\bigl(f(z) - f(z_0)\bigr) 
\;+\;
\delta\bigl(f(z)\bigr).
\]
Dividing through by $(z - z_0)$,
\[
\frac{g\bigl(f(z)\bigr) - g\bigl(f(z_0)\bigr)}{\,z - z_0\,}
=
g'\bigl(f(z_0)\bigr)\,\frac{f(z) - f(z_0)}{\,z - z_0\,}
\;+\;
\frac{\delta\bigl(f(z)\bigr)}{\,z - z_0\,}.
\]

\item \textbf{Analyze each term in the limit as $z \to z_0$.}
\begin{itemize}
\item By definition of $f'(z_0)$,
\[
\lim_{z \to z_0} \frac{f(z) - f(z_0)}{\,z - z_0\,} 
= f'(z_0).
\]
\item For the term $\delta\bigl(f(z)\bigr)$, note that $f(z) \to f(z_0)$ as $z \to z_0$, so
\[
\lim_{z \to z_0} \frac{\delta\bigl(f(z)\bigr)}{\,f(z) - f(z_0)\,} 
= 0.
\]
But also
\[
\frac{\delta\bigl(f(z)\bigr)}{\,z - z_0\,}
=
\frac{\delta\bigl(f(z)\bigr)}{\,f(z) - f(z_0)\,} \;\cdot\; \frac{f(z) - f(z_0)}{\,z - z_0\,}.
\]
As $z \to z_0$, the first factor $\to 0$ and the second factor $\to f'(z_0)$ (which is finite).  
Hence
\[
\lim_{z \to z_0} \frac{\delta\bigl(f(z)\bigr)}{\,z - z_0\,} 
= 0 \times f'(z_0) 
= 0.
\]
\end{itemize}

\item \textbf{Combine the results.}
Putting it all together,
\[
\lim_{z \to z_0}
\frac{g\bigl(f(z)\bigr) - g\bigl(f(z_0)\bigr)}{\,z - z_0\,}
=
g'\bigl(f(z_0)\bigr)\,\lim_{z \to z_0}\frac{f(z) - f(z_0)}{\,z - z_0\,}
\;+\;
\lim_{z \to z_0} \frac{\delta\bigl(f(z)\bigr)}{\,z - z_0\,}
=
g'\bigl(f(z_0)\bigr)\,f'(z_0) + 0.
\]
Hence
\[
(g \circ f)'(z_0) 
= g'\bigl(f(z_0)\bigr)\,f'(z_0),
\]
as desired.
\end{enumerate}

\paragraph{Why a ``naive'' proof can fail.}
A common incorrect attempt tries to write 
\[
\lim_{z \to z_0}
\frac{g\bigl(f(z)\bigr) - g\bigl(f(z_0)\bigr)}{\,z - z_0\,}
=
\lim_{z \to z_0} 
\biggl[
\frac{g\bigl(f(z)\bigr) - g\bigl(f(z_0)\bigr)}{\,f(z) - f(z_0)\,}
\;\cdot\;
\frac{f(z) - f(z_0)}{\,z - z_0\,}
\biggr],
\]
and then ``plug in'' $z=z_0$ to claim
\(
g'(f(z_0)) \cdot f'(z_0).
\)
Strictly speaking, one must justify that $f(z) - f(z_0)\neq 0$ near $z_0$ and also handle the limit of the product carefully.  In particular, we need to ensure $f(z)\to f(z_0)$ \emph{continuously} so that the difference‐quotient for $g$ is well‐defined and tends to $g'(f(z_0))$.  The proof above takes care of these details by explicitly controlling the error terms $\varepsilon(z)$ and $\delta(w)$, ensuring no hidden assumptions are made.

\section*{Finding the Radius of Convergence}

Consider the power series
\[
\sum_{k=0}^{\infty} a_k \,(z-2)^k
\quad\text{where}\quad
a_k \;=\; \frac{(k!)^2}{(2k)!}.
\]
We want to determine the radius of convergence $R$.  A convenient way is to use the ratio test (or equivalently the Cauchy--Hadamard formula).  First compute the ratio
\[
\frac{a_{k+1}}{a_k}
=
\frac{\tfrac{((k+1)!)^2}{(2(k+1))!}}{\tfrac{(k!)^2}{(2k)!}}
=
\frac{(k+1)^2 (k!)^2 / \bigl((2k+2)!\bigr)}
     {(k!)^2 / (2k)!}
=
\frac{(k+1)^2}{(2k+2)(2k+1)}.
\]
Simplify and take the limit as $k\to\infty$.  Notice
\[
(2k+2)(2k+1) = 4k^2 + 6k + 2,
\quad
(k+1)^2 = k^2 + 2k + 1.
\]
Hence
\[
\lim_{k\to\infty}
\frac{(k+1)^2}{(2k+2)(2k+1)}
=
\lim_{k\to\infty}
\frac{k^2 + 2k + 1}{\,4k^2 + 6k + 2\,}
= \frac{1}{4}.
\]
By the standard ratio‐test reasoning, the radius of convergence is
\[
R
= \frac{1}{\,\lim_{k\to\infty} \bigl|a_{k+1}/a_k\bigr|}
= \frac{1}{\tfrac{1}{4}}
= 4.
\]
Therefore, the given series converges for
\[
|z-2| < 4,
\]
and the \textbf{radius of convergence} is $\boxed{4}$.

\section*{Problem}
Define 
\[
f(z) \;=\; 2\,x\,y^{3}\;+\; i\bigl(3\,y \;-\; 3\,x^{2}\,y^{2}\bigr),
\quad
\text{where }z = x + i\,y.
\]
\begin{enumerate}
\item Find all points $z$ in the plane for which $f'(z)$ (the complex derivative) exists.
\item Compute $f'(z)$ at those points.
\end{enumerate}

\section*{Solution}

\paragraph{Step 1: Express $f$ in terms of $u$ and $v$.}
We write $f(z) = u(x,y) + i\,v(x,y)$, where
\[
u(x,y) = 2\,x\,y^{3},
\quad
v(x,y) = 3\,y \;-\; 3\,x^{2}\,y^{2}.
\]

\paragraph{Step 2: Compute the partial derivatives.}
\[
u_{x} = 2\,y^{3},
\quad
u_{y} = 6\,x\,y^{2},
\]
\[
v_{x} = \frac{\partial}{\partial x}\bigl(3\,y - 3\,x^{2}y^{2}\bigr) 
= -6\,x\,y^{2},
\quad
v_{y} = \frac{\partial}{\partial y}\bigl(3\,y - 3\,x^{2}y^{2}\bigr) 
= 3 \;-\;6\,x^{2}\,y.
\]

\paragraph{Step 3: Impose the Cauchy--Riemann equations.}
For $f$ to be holomorphic at $(x,y)$, we need
\[
u_{x} = v_{y}
\quad\text{and}\quad
u_{y} = -\,v_{x}.
\]

\subparagraph{(a) The second equation:}
\[
u_{y} = 6\,x\,y^{2}, 
\quad
-\,v_{x} = -\bigl(-6\,x\,y^{2}\bigr) = 6\,x\,y^{2}.
\]
Hence 
\[
u_{y} = -\,v_{x}
\quad\Longrightarrow\quad
6\,x\,y^{2} = 6\,x\,y^{2}
\]
which is automatically true for all $(x,y)$.  No restriction arises here.

\subparagraph{(b) The first equation:}
\[
u_{x} = 2\,y^{3},
\quad
v_{y} = 3 \;-\; 6\,x^{2}\,y.
\]
Hence
\[
u_{x} = v_{y}
\quad\Longrightarrow\quad
2\,y^{3} \;=\; 3 \;-\;6\,x^{2}\,y.
\]
Rearrange:
\[
2\,y^{3} \;+\; 6\,x^{2}\,y \;=\; 3
\quad\Longrightarrow\quad
2\,y\bigl(y^{2} + 3\,x^{2}\bigr) \;=\; 3.
\]
Thus a point $(x,y)$ satisfies the Cauchy--Riemann equations if and only if
\[
\boxed{\,2\,y\bigl(y^{2} + 3\,x^{2}\bigr) = 3.}
\]
Equivalently, $y \neq 0$ and
\[
y^{2} + 3\,x^{2} = \frac{3}{2\,y}.
\]

\paragraph{Conclusion for part (1).}
\emph{All points $z=x+ i\,y$ at which $f'(z)$ exists} are precisely those $(x,y)$ satisfying
\[
2\,y\bigl(y^{2} + 3\,x^{2}\bigr) = 3.
\]
Geometrically, this is a level curve in the plane (not a standard circle or ellipse in the usual $x^2 + y^2$ sense, but a real‐analytic curve defined by that equation).

\paragraph{Step 4: Compute $f'(z)$ at those points.}
When the Cauchy--Riemann equations hold, one standard formula for the complex derivative is
\[
f'(z) \;=\; u_{x} \;+\; i\,v_{x}.
\]
(Equivalently, $f'(z) = v_{y} - i\,u_{y}$, but $u_x + i v_x$ is often more direct.)

\[
u_{x} = 2\,y^{3},
\quad
v_{x} = -6\,x\,y^{2}.
\]
Hence
\[
f'(z) 
= u_{x} + i\,v_{x}
= 2\,y^{3} \;+\; i\bigl(-6\,x\,y^{2}\bigr)
= 2\,y^{3} \;-\; 6\,i\,x\,y^{2}.
\]
One may also substitute $2\,y^{3} = 3 - 6\,x^{2}\,y$ from the CR condition if desired, but typically it is fine to leave it in the form
\[
\boxed{\,f'(z) = 2\,y^{3} \;-\; 6\,i\,x\,y^{2},}
\]
valid exactly on the locus $2\,y(y^{2} + 3\,x^{2})=3$.

\bigskip

\noindent
\textbf{Answer:}
\begin{itemize}
\item 
\emph{Points of differentiability}: 
\[
\{(x,y)\in \mathbb{R}^2 : 2\,y\bigl(y^{2} + 3\,x^{2}\bigr) = 3\}.
\]
\item 
\emph{Derivative there}:
\[
f'(z) = 2\,y^{3} \;-\; 6\,i\,x\,y^{2}.
\]
\end{itemize}

\section*{Finding the Fourth Roots of \(-8-8\sqrt{3}\,i\)}

We wish to find the fourth roots of 
\[
z = -8-8\sqrt{3}\,i,
\]
i.e., determine
\[
w = z^{1/4} = (-8-8\sqrt{3}\,i)^{1/4},
\]
in the form \(a+bi\).

\subsection*{Step 1: Express \(z\) in Polar Form}
\textbf{Magnitude:}
\[
r = |z| = \sqrt{(-8)^2+(-8\sqrt{3})^2} = \sqrt{64+192} = \sqrt{256} = 16.
\]

\textbf{Argument:}  
Since
\[
\cos\theta = \frac{-8}{16} = -\frac{1}{2}, \quad \sin\theta = \frac{-8\sqrt{3}}{16} = -\frac{\sqrt{3}}{2},
\]
the angle \(\theta\) lies in the third quadrant. Recognizing that
\[
\cos\frac{4\pi}{3} = -\frac{1}{2} \quad \text{and} \quad \sin\frac{4\pi}{3} = -\frac{\sqrt{3}}{2},
\]
we have
\[
\theta = \frac{4\pi}{3}.
\]

Thus, the polar form of \(z\) is:
\[
z = 16\left(\cos\frac{4\pi}{3} + i\sin\frac{4\pi}{3}\right).
\]

\subsection*{Step 2: Apply De Moivre's Theorem}
The fourth roots of \(z\) are given by:
\[
z^{1/4} = r^{1/4}\,\text{cis}\left(\frac{\theta+2\pi k}{4}\right),\quad k=0,1,2,3.
\]
Since \(r^{1/4}=16^{1/4}=2\), we obtain:
\[
z^{1/4} = 2\,\text{cis}\left(\frac{\frac{4\pi}{3}+2\pi k}{4}\right)
= 2\,\text{cis}\left(\frac{\pi}{3}+\frac{\pi k}{2}\right),\quad k=0,1,2,3.
\]

\subsection*{Step 3: Express Each Root in Rectangular Form \(a+bi\)}

\textbf{For \(k=0\):}
\[
z^{1/4} = 2\,\text{cis}\left(\frac{\pi}{3}\right)
= 2\left(\cos\frac{\pi}{3}+i\sin\frac{\pi}{3}\right)
= 2\left(\frac{1}{2}+i\frac{\sqrt{3}}{2}\right)
= 1+i\sqrt{3}.
\]

\textbf{For \(k=1\):}
\[
z^{1/4} = 2\,\text{cis}\left(\frac{\pi}{3}+\frac{\pi}{2}\right)
= 2\,\text{cis}\left(\frac{5\pi}{6}\right)
= 2\left(\cos\frac{5\pi}{6}+i\sin\frac{5\pi}{6}\right).
\]
Since
\[
\cos\frac{5\pi}{6} = -\frac{\sqrt{3}}{2},\quad \sin\frac{5\pi}{6} = \frac{1}{2},
\]
it follows that:
\[
z^{1/4} = 2\left(-\frac{\sqrt{3}}{2}+i\frac{1}{2}\right)
= -\sqrt{3}+i.
\]

\textbf{For \(k=2\):}
\[
z^{1/4} = 2\,\text{cis}\left(\frac{\pi}{3}+\pi\right)
= 2\,\text{cis}\left(\frac{4\pi}{3}\right)
= 2\left(\cos\frac{4\pi}{3}+i\sin\frac{4\pi}{3}\right).
\]
Using
\[
\cos\frac{4\pi}{3} = -\frac{1}{2},\quad \sin\frac{4\pi}{3} = -\frac{\sqrt{3}}{2},
\]
we have:
\[
z^{1/4} = 2\left(-\frac{1}{2}-i\frac{\sqrt{3}}{2}\right)
= -1-i\sqrt{3}.
\]

\textbf{For \(k=3\):}
\[
z^{1/4} = 2\,\text{cis}\left(\frac{\pi}{3}+\frac{3\pi}{2}\right)
= 2\,\text{cis}\left(\frac{11\pi}{6}\right)
= 2\left(\cos\frac{11\pi}{6}+i\sin\frac{11\pi}{6}\right).
\]
Noting that
\[
\cos\frac{11\pi}{6} = \frac{\sqrt{3}}{2},\quad \sin\frac{11\pi}{6} = -\frac{1}{2},
\]
we obtain:
\[
z^{1/4} = 2\left(\frac{\sqrt{3}}{2}-i\frac{1}{2}\right)
= \sqrt{3}-i.
\]

\subsection*{Final Answer}
The four fourth roots of \(-8-8\sqrt{3}\,i\) are:
\[
\boxed{
\begin{aligned}
1.\quad &1+i\sqrt{3},\\[1mm]
2.\quad &-\sqrt{3}+i,\\[1mm]
3.\quad &-1-i\sqrt{3},\\[1mm]
4.\quad &\sqrt{3}-i.
\end{aligned}
}
\]
Often, the principal value is taken as:
\[
\boxed{1+i\sqrt{3}}.
\]t
\end{document}
