\documentclass[12pt]{article}

% Packages
\usepackage[margin=1in]{geometry}
\usepackage{amsmath,amssymb,amsthm}
\usepackage{enumitem}
\usepackage{hyperref}
\usepackage{xcolor}
\usepackage{import}
\usepackage{xifthen}
\usepackage{pdfpages}
\usepackage{transparent}
\usepackage{listings}


\lstset{
    breaklines=true,         % Enable line wrapping
    breakatwhitespace=false, % Wrap lines even if there's no whitespace
    basicstyle=\ttfamily,    % Use monospaced font
    frame=single,            % Add a frame around the code
    columns=fullflexible,    % Better handling of variable-width fonts
}

\newcommand{\incfig}[1]{%
    \def\svgwidth{\columnwidth}
    \import{./Figures/}{#1.pdf_tex}
}
\theoremstyle{definition} % This style uses normal (non-italicized) text
\newtheorem{solution}{Solution}
\newtheorem{proposition}{Proposition}
\newtheorem{problem}{Problem}
\newtheorem{lemma}{Lemma}
\newtheorem{theorem}{Theorem}
\newtheorem{remark}{Remark}
\newtheorem{note}{Note}
\newtheorem{definition}{Definition}
\newtheorem{example}{Example}
\theoremstyle{plain} % Restore the default style for other theorem environments
%

% Theorem-like environments
% Title information
\title{}
\author{Jerich Lee}
\date{\today}

\begin{document}

\maketitle
We want to show that
\[
\lvert e^{iz}\rvert \;=\; e^{\operatorname{Re}(iz)} \;=\; e^{-R\sin\theta}.
\]

\textbf{Step 1: Express $z$ in polar form.}  
Let
\[
z = R e^{i\theta},
\]
where \(R = \sqrt{x^2 + y^2}\) and \(\theta = \arg(z)\).

\textbf{Step 2: Compute $iz$.}  
Since \(z = x + i\,y\), we have
\[
iz \;=\; i(x + i\,y) 
\;=\; i\,x + i^2\,y 
\;=\; i\,x - y 
\;=\; -y \;+\; i\,x.
\]
Equivalently, using polar form,
\[
iz 
\;=\;
i \bigl(R e^{i\theta}\bigr)
\;=\;
R\,\bigl(i\,e^{i\theta}\bigr)
\;=\;
R\,e^{i(\theta + \pi/2)}.
\]
In either case, we see that
\[
\operatorname{Re}(iz) 
\;=\; -\,R\sin\theta.
\]

\textbf{Step 3: The magnitude of $e^{iz}$.}  
By the general property of complex exponentials,
\[
\lvert e^{w}\rvert 
\;=\;
e^{\operatorname{Re}(w)},
\]
so
\[
\lvert e^{iz}\rvert 
\;=\;
e^{\operatorname{Re}(iz)} 
\;=\;
e^{-\,R\sin\theta}.
\]


\end{document}
