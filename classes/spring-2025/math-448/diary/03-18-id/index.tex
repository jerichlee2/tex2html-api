\documentclass[12pt]{article}

% Packages
\usepackage[margin=.5in]{geometry}
\usepackage{amsmath,amssymb,amsthm}
\usepackage{enumitem}
\usepackage{hyperref}
\usepackage{xcolor}
\usepackage{import}
\usepackage{xifthen}
\usepackage{pdfpages}
\usepackage{transparent}
\usepackage{listings}
\DeclareMathOperator{\Log}{Log}
\DeclareMathOperator{\Arg}{Arg}


\lstset{
    breaklines=true,         % Enable line wrapping
    breakatwhitespace=false, % Wrap lines even if there's no whitespace
    basicstyle=\ttfamily,    % Use monospaced font
    frame=single,            % Add a frame around the code
    columns=fullflexible,    % Better handling of variable-width fonts
}

\newcommand{\incfig}[1]{%
    \def\svgwidth{\columnwidth}
    \import{./Figures/}{#1.pdf_tex}
}
\theoremstyle{definition} % This style uses normal (non-italicized) text
\newtheorem{solution}{Solution}
\newtheorem{proposition}{Proposition}
\newtheorem{problem}{Problem}
\newtheorem{lemma}{Lemma}
\newtheorem{theorem}{Theorem}
\newtheorem{remark}{Remark}
\newtheorem{note}{Note}
\newtheorem{definition}{Definition}
\newtheorem{example}{Example}
\newtheorem{corollary}{Corollary}
\theoremstyle{plain} % Restore the default style for other theorem environments
%

% Theorem-like environments
% Title information
\title{MATH 448}
\author{Jerich Lee}
\date{\today}

\begin{document}

\maketitle

\section{Definitions}

\begin{definition}[Complex Numbers]
The set of \textbf{complex numbers} is defined as
\[
\mathbb{C} = \{\, a + bi \mid a,b \in \mathbb{R}, \, i^2 = -1 \,\}.
\]
\end{definition}

\begin{definition}[Euler's Formula]
    $$    
    R\cos\theta + Ri\sin\theta = Re^{i\theta} \quad (\text{Euler's formula})  
    $$
\end{definition}
\subsubsection*{Random Identity}
We wish to show that
\[
\frac{1}{i} = -i.
\]

\textbf{Derivation:}

Start with the left-hand side:
\[
\frac{1}{i}.
\]
Multiply the numerator and the denominator by \(-i\) to rationalize the denominator:
\[
\frac{1}{i} = \frac{1}{i} \cdot \frac{-i}{-i} = \frac{-i}{-i^2}.
\]
Since \(i^2 = -1\), we have
\[
-i^2 = -(-1) = 1.
\]
Thus,
\[
\frac{-i}{-i^2} = \frac{-i}{1} = -i.
\]

Therefore, we have verified that
\[
\frac{1}{i} = -i.
\]


\subsubsection*{Identities (Complex Numbers):}

\begin{align}
|z| &= \sqrt{x^2 + y^2}, \quad z = x + iy \\
\overline{z} &= x - iy \\
z\overline{z} &= x^2 + y^2 = |z|^2 = |\overline{z}|^2 \\
\operatorname{Re}(z) &= \frac{z + \overline{z}}{2}, \quad \operatorname{Im}(z) = \frac{z - \overline{z}}{2i} \\
\cos(\theta) &= \frac{e^{i\theta} + e^{-i\theta}}{2}, \quad \sin(\theta) = \frac{e^{i\theta} - e^{-i\theta}}{2i} \\
\left\vert e^{iz} \right\vert = e^{\text{Re}(iz) }= e^{-R\sin \theta}
\end{align}

\subsubsection*{De Moivre's Theorem:}
\[
(\cos \theta + i \sin \theta)^n = \cos(n\theta) + i\sin(n\theta), \quad n \in \mathbb{Z}.
\]

\subsubsection*{Common Taylor Series Expansions}
\begin{align}
    e^z = \sum_{n=0}^{\infty} \frac{(z - z_0)^n}{n!} e^{z_0}\\
&= 
e^{z_0} 
\;+\; e^{z_0}\,(z - z_0) 
\;+\; \frac{e^{z_0}\,(z - z_0)^2}{2!}
\;+\; \frac{e^{z_0}\,(z - z_0)^3}{3!}
\;+\; \cdots
\\
    \sin x &= \sum_{n=0}^{\infty} (-1)^n \frac{x^{2n+1}}{(2n+1)!} \\
    = x - \frac{x^3}{3!} + \frac{x^5}{5!} - \frac{x^7}{7!} + \dots \\[8pt]
    %
    \cos x &= \sum_{n=0}^{\infty} (-1)^n \frac{x^{2n}}{(2n)!} \\
    = 1 - \frac{x^2}{2!} + \frac{x^4}{4!} - \frac{x^6}{6!} + \dots \\[8pt]
    %
    \tan x &= \sum_{n=1}^{\infty} \frac{B_{2n}(-4)^n(1-4^n)}{(2n)!} x^{2n-1}\\
    = x + \frac{x^3}{3} + \frac{2x^5}{15} + \frac{17x^7}{315} + \dots\\
    \textit{Note: \(B_n\) represents Bernoulli numbers.}
\end{align}
\section*{Taylor Expansions of \(\cos\) and \(\sin\) Around \(z_0\)}

Let \(z = z_0 + w\), with \(|w|\ll 1\). Then:

\[
\cos(z_0 + w)
= \cos(z_0) 
- \sin(z_0)\,w
- \frac{\cos(z_0)}{2!}\,w^2
+ O(w^3).
\]

\[
\sin(z_0 + w)
= \sin(z_0)
+ \cos(z_0)\,w
- \frac{\sin(z_0)}{2!}\,w^2
+ O(w^3).
\]

These follow from the usual Taylor series expansions and the derivatives:
\[
\frac{d}{dz}\cos z = -\sin z,\quad
\frac{d}{dz}\sin z = \cos z,\quad \text{etc.}
\]

\subsubsection*{Trig Identities}

\begin{enumerate}
\item \textbf{Algebraic Identities:}
\[
    (a + b)(a^2 - ab + b^2) = a^3 + b^3
    \]

\item \textbf{Pythagorean Identities:}
\[
\sin^2 x + \cos^2 x = 1,\quad 1+\tan^2 x = \sec^2 x,\quad 1+\cot^2 x = \csc^2 x
\]

\item \textbf{Angle Addition and Subtraction:}
\[
\sin(x \pm y) = \sin x \cos y \pm \cos x \sin y
\]
\[
\cos(x \pm y) = \cos x \cos y \mp \sin x \sin y
\]
\[
\tan(x \pm y) = \frac{\tan x \pm \tan y}{1 \mp \tan x \tan y}
\]

\item \textbf{Double-Angle Identities:}
\[
\sin(2x) = 2 \sin x \cos x
\]
\[
\cos(2x) = \cos^2 x - \sin^2 x = 2\cos^2 x - 1 = 1 - 2\sin^2 x
\]
\[
\tan(2x) = \frac{2\tan x}{1 - \tan^2 x}
\]
\pagebreak
\item \textbf{Half-Angle Identities:}
\[
\sin\frac{x}{2} = \pm\sqrt{\frac{1 - \cos x}{2}}, \quad 
\cos\frac{x}{2} = \pm\sqrt{\frac{1 + \cos x}{2}}
\]
\[
\tan\frac{x}{2} = \frac{\sin x}{1 + \cos x} = \frac{1 - \cos x}{\sin x}
\]

\item \textbf{Product-to-Sum Identities:}
\[
\sin x \sin y = \frac{1}{2}\left[\cos(x - y) - \cos(x + y)\right]
\]
\[
\cos x \cos y = \frac{1}{2}\left[\cos(x - y) + \cos(x + y)\right]
\]
\[
\sin x \cos y = \frac{1}{2}\left[\sin(x + y) + \sin(x - y)\right]
\]

\item \textbf{Sum-to-Product Identities:}
\[
\sin x + \sin y = 2\sin\frac{x + y}{2}\cos\frac{x - y}{2}
\]
\[
\sin x - \sin y = 2\cos\frac{x + y}{2}\sin\frac{x - y}{2}
\]
\[
\cos x + \cos y = 2\cos\frac{x + y}{2}\cos\frac{x - y}{2}
\]
\[
\cos x - \cos y = -2\sin\frac{x + y}{2}\sin\frac{x - y}{2}
\]

\end{enumerate}

\begin{definition}[Branch of a Multi-Valued Function]

    In complex analysis, a \emph{branch} of a multi-valued function (e.g.\ the logarithm, square root, or any power function) is a single-valued “slice” obtained by:
    
    \begin{itemize}
      \item \textbf{Choosing a branch cut:} A curve or line in the complex plane along which we remove points to prevent “looping around” the singularity.
      \item \textbf{Restricting the argument:} We impose a condition on $\arg(z)$ so that each $z$ has exactly one value for the function, rather than infinitely many.
    \end{itemize}
    
    Concretely, for the complex logarithm,
    \[
    \log z = \ln|z| + i(\arg z + 2\pi k), \quad k \in \mathbb{Z},
    \]
    selecting a branch means fixing $k$ (or fixing an interval for $\arg z$) so that $\log z$ becomes single-valued and analytic on $\mathbb{C}$ minus the chosen branch cut.
      
\end{definition}
\subsection*{Important Identities for the Complex Logarithm}

\subsubsection*{1. Definition of the Multi-valued Logarithm}
For any nonzero complex number \(z\),
\[
\log z = \ln |z| + i\Bigl(\arg z + 2\pi k\Bigr), \quad k\in\mathbb{Z},
\]
where \(\ln|z|\) is the usual real logarithm and \(\arg z\) denotes any argument of \(z\).

\subsubsection*{2. The Principal Branch of the Logarithm}
The principal value of the logarithm, denoted by \(\Log z\), is defined by
\[
\Log z = \ln |z| + i\,\Arg z,\quad \Arg z\in (-\pi,\pi],
\]
so that \(\Log z\) is single-valued on its domain.

\subsubsection*{3. Exponential and Logarithm Inversion}
A fundamental property is that exponentiation inverts the logarithm:
\[
e^{\log z} = z,
\]
but more precisely, for the principal branch we have
\[
e^{\Log z} = z.
\]
\textbf{Derivation:}  
Start with the definition of the principal logarithm:
\[
\Log z = \ln|z| + i\,\Arg z.
\]
Then exponentiate:
\[
e^{\Log z} = e^{\ln|z| + i\,\Arg z} 
= e^{\ln|z|}\,e^{i\,\Arg z} 
= |z|\Bigl(\cos(\Arg z) + i\sin(\Arg z)\Bigr).
\]
Since \(z\) can be written in polar form as \(z = |z|e^{i\,\Arg z}\), it follows that
\[
e^{\Log z} = z.
\]

\subsubsection*{4. Logarithm of a Product}
For nonzero complex numbers \(z\) and \(w\),
\[
\log(zw) = \log z + \log w \quad \text{(modulo }2\pi i\text{)}.
\]
In terms of the principal branch,
\[
\Log(zw) = \Log z + \Log w + 2\pi i k, \quad k\in\mathbb{Z},
\]
and if \(\Log z + \Log w\) lies in \((-\pi, \pi]\) (or is adjusted accordingly), then \(k=0\).

\subsubsection*{5. Logarithm of a Quotient}
Similarly, for nonzero \(z\) and \(w\),
\[
\log\left(\frac{z}{w}\right) = \log z - \log w \quad \text{(modulo }2\pi i\text{)}.
\]

\subsubsection*{6. Derivative of the Logarithm}
The derivative of the principal branch \(\Log z\) is given by
\[
\frac{d}{dz}\Log z = \frac{1}{z}, \quad z\neq 0.
\]

\subsubsection*{7. Inverse Property for the Exponential Function}
For any complex number \(z\),
\[
\ln\Bigl(e^z\Bigr) = z + 2\pi i k, \quad k\in\mathbb{Z}.
\]
If we restrict to the principal branch (assuming \(\Im(z) \in (-\pi,\pi]\)), then
\[
\Log\Bigl(e^z\Bigr) = z.
\]
\subsubsection*{Generalized Binomial Expansion:}

For any real or complex exponent \(\alpha\), the binomial series is given by:
\[
(1 + x)^\alpha = \sum_{n=0}^{\infty} \binom{\alpha}{n} x^n, \quad |x| < 1,
\]

where the generalized binomial coefficient is defined as:
\[
\binom{\alpha}{n} = \frac{\alpha(\alpha - 1)(\alpha - 2)\dots(\alpha - n + 1)}{n!}.
\]

\subsubsection*{Geometric Series Formulas:}

\begin{enumerate}

\item \textbf{Finite Geometric Series:}
\[
\sum_{k=0}^{n} ar^k = a\frac{1 - r^{n+1}}{1 - r}, \quad r \neq 1
\]

\item \textbf{Infinite Geometric Series:}
\[
\sum_{k=0}^{\infty} ar^k = \frac{a}{1 - r}, \quad |r| < 1
\]
\[
\frac{1}{1+u} 
\;=\; \sum_{n=0}^{\infty} (-1)^n\,u^n
\quad\text{for }|u| < 1.
\]


\end{enumerate}
\subsubsection*{Derivatives of \(\displaystyle \frac{1}{z}\) in Terms of \(\frac{1}{z^n}\)}

\[
f(z) = \frac{1}{z}.
\]
We compute successive derivatives:
\[
f'(z) = -\frac{1}{z^2},\quad
f''(z) = 2!\,\frac{1}{z^3} \quad(\text{with a sign }(-1)^2=+1),
\quad
f^{(3)}(z) = -3!\,\frac{1}{z^4}, 
\quad \dots
\]
By induction or direct pattern recognition, the general formula for the \(n\)-th derivative (\(n\ge 1\)) is
\[
f^{(n)}(z) 
= \frac{d^n}{dz^n}\bigl(\tfrac{1}{z}\bigr)
= (-1)^n\,n!\,\frac{1}{z^{\,n+1}}.
\]
\subsubsection*{Parameterizations:}

\begin{itemize}
    \item Line segment from \( z_0 \) to \( z_1 \):
    \[
    \gamma(t) = t z_1 + (1 - t) z_0, \quad 0 \leq t \leq 1
    \]

    \item Circle of radius \( R \) centered at \( p \):
    \[
    \gamma(t) = p + R e^{i t}, \quad 0 \leq t \leq 2\pi
    \]
\end{itemize}


\begin{definition}[Complex Differentiability and Holomorphic Functions]
Let \(U \subset \mathbb{C}\) be an open set. A function \(f: U \to \mathbb{C}\) is said to be \textbf{complex differentiable} at \(z_0 \in U\) if the limit
\[
f'(z_0) = \lim_{z \to z_0} \frac{f(z)-f(z_0)}{z-z_0}
\]
exists. If \(f\) is differentiable at every point in \(U\), then \(f\) is called \textbf{holomorphic} (or \textbf{analytic}) on \(U\).
\end{definition}

\begin{definition}[Cauchy--Riemann Equations]
Let \(f(z)=u(x,y)+iv(x,y)\) be defined on an open subset of \(\mathbb{C}\) with \(z=x+iy\). If \(f\) is differentiable at a point, then the real functions \(u\) and \(v\) satisfy the \textbf{Cauchy--Riemann equations}:
\[
u_x = v_y \quad \text{and} \quad u_y = -v_x.
\]
\end{definition}

\begin{definition}[Power Series and Radius of Convergence]
A \textbf{power series} centered at \(z_0\) is an expression of the form
\[
\sum_{n=0}^{\infty} a_n (z-z_0)^n, \quad a_n\in \mathbb{C}.
\]
There exists a \textbf{radius of convergence} \(R \ge 0\) such that the series converges absolutely for \(|z-z_0| < R\) and diverges for \(|z-z_0| > R\).
\end{definition}

\subsubsection*{Power Series:}

\begin{align*}
f(z) &= \sum_{n=0}^{\infty} a_n (z - z_0)^n \\[6pt]
f'(z) &= \sum_{n=1}^{\infty} n a_n (z - z_0)^{n-1} \\[6pt]
(fg)(z) &= \sum_{n=0}^{\infty} c_n (z - z_0)^n, \quad c_n = \sum_{k=0}^{n} a_k b_{n-k} \\[6pt]
a_n &= \frac{1}{2\pi i} \oint_\gamma \frac{f(z)}{(z - z_0)^{n+1}}\,dz
\end{align*}

\subsubsection*{Logarithmic Series Expansions:}

\begin{align*}
\log(1 + w) &= \sum_{n=1}^{\infty} (-1)^{n+1}\frac{w^n}{n}, \quad |w|<1 \\[6pt]
\log(1 - z) &= -\sum_{n=1}^{\infty} \frac{z^n}{n}, \quad |z|<1
\end{align*}

\begin{definition}[Taylor Series]
If \(f\) is analytic in a disk \(|z-z_0| < R\), then it can be represented as a \textbf{Taylor series}:
\[
f(z) = \sum_{n=0}^{\infty} \frac{f^{(n)}(z_0)}{n!}(z-z_0)^n.
\]
\end{definition}

\begin{definition}[Laurent Series]
Let \(f\) be analytic in an annulus \(A = \{\,z \in \mathbb{C} : r < |z-z_0| < R\,\}\). Then \(f\) has a \textbf{Laurent series expansion}
\[
f(z) = \sum_{n=-\infty}^{\infty} c_n (z-z_0)^n,
\]
which converges for \(z \in A\).
\end{definition}

\begin{definition}[Isolated Singularities]
Let \(z_0\) be an isolated singular point of \(f\). Then:
\begin{enumerate}[label=(\roman*)]
    \item \(z_0\) is a \textbf{removable singularity} if \(f\) can be redefined at \(z_0\) to become analytic.
    \item \(z_0\) is a \textbf{pole} of order \(m\) if \((z-z_0)^m f(z)\) is analytic and nonzero at \(z_0\).
    \item \(z_0\) is an \textbf{essential singularity} if it is neither removable nor a pole.
\end{enumerate}
\end{definition}



\begin{definition}[Residue]
If \(f\) has an isolated singularity at \(z_0\) with Laurent expansion
\[
f(z) = \sum_{n=-\infty}^{\infty} c_n (z-z_0)^n,
\]
the coefficient \(c_{-1}\) is called the \textbf{residue} of \(f\) at \(z_0\), denoted \(\operatorname{Res}(f,z_0)\).
\end{definition}

\subsubsection*{Residue Formulas:}

\begin{itemize}
    \item Residue of a function at a pole of order $m$ at $z_0$:
    \[
    \text{Res}(f; z_0) = c_{-1} = \frac{H^{(m-1)}(z_0)}{(m-1)!}
    \]
    \textit{Note: $c_{-1}$ is the coefficient of the $(z - z_0)^{-1}$ term in the Laurent expansion around $z_0$.}

    \item Residue of a rational function $\frac{F}{G}$ at a simple pole $z_0$ (where $G(z_0) = 0$ and $G'(z_0) \neq 0$):
    \[
    \text{Res}\left(\frac{F}{G}; z_0\right) = \frac{F(z_0)}{G'(z_0)}
    \]
    \textit{Note: This formula is applicable only for simple poles.}
\end{itemize}

\begin{definition}[Contour and Contour Integral]
A \textbf{contour} is a piecewise smooth curve \(\gamma: [a,b] \to \mathbb{C}\). The \textbf{contour integral} of a function \(f\) along \(\gamma\) is defined by
\[
\int_\gamma f(z) \,dz = \int_a^b f(\gamma(t)) \, \gamma'(t) \, dt.
\]
\end{definition}

\begin{definition}[Branches and Branch Cuts]
For a multi-valued function (such as \(\log z\) or \(z^{1/n}\)), a \textbf{branch} is a single-valued, continuous selection of values. A \textbf{branch cut} is a curve (or set) in \(\mathbb{C}\) along which the function is discontinuous to maintain single-valuedness on the complement.
\end{definition}

\section{Major Theorems, Propositions, and Lemmas}
\subsubsection*{Fundamental Theorem of Calculus in the Complex Plane.}  
In complex analysis, the result you are using is often called the \emph{Fundamental Theorem of Calculus for Line Integrals} (in the complex plane) or, equivalently, “if a function has an antiderivative, then the integral is path‐independent.”

\begin{theorem}[Fundamental Theorem of Calculus in the Complex Plane]
Let $f$ be a function that is analytic on a simply connected domain $D$, and suppose $F$ is an antiderivative of $f$ on $D$ (that is, $F'(z) = f(z)$ for all $z \in D$). Then for any path $\gamma$ in $D$ from $z_0$ to $z_1$,
\[
\int_\gamma f(z)\,dz \;=\; F(z_1) \;-\; F(z_0).
\]
In particular, the integral of $f$ over any closed loop in $D$ is zero, so $\int_\gamma f(z)\,dz$ is independent of the path chosen.
\end{theorem}

\textbf{Example.}  
In your example,
\[
f(z) \;=\;\frac{1}{z^2},
\]
you observed that
\[
\frac{d}{dz}\bigl(-\tfrac{1}{z}\bigr) \;=\;\frac{1}{z^2}.
\]
Hence $-\tfrac{1}{z}$ is an antiderivative of $\tfrac{1}{z^2}$. Therefore, on any domain in $\mathbb{C}\setminus\{0\}$ (which is simply connected once you exclude the origin), the line integral of $\tfrac{1}{z^2}$ depends only on the endpoints and not on the specific path.

\begin{theorem}[Cauchy's Integral Theorem]
Let \(U\) be a simply connected open subset of \(\mathbb{C}\) and let \(f: U \to \mathbb{C}\) be analytic. Then for every closed contour \(\gamma\) in \(U\),
\[
\int_\gamma f(z)\,dz = 0.
\]
\end{theorem}

\begin{theorem}[Cauchy's Integral Formula]
Let \(f\) be analytic in a simply connected domain \(U\) and let \(\gamma\) be a positively oriented, simple closed contour in \(U\). If \(z_0\) is interior to \(\gamma\), then
\[
f(z_0) = \frac{1}{2\pi i} \int_\gamma \frac{f(z)}{z-z_0}\,dz.
\]
Moreover, for any \(n \geq 1\),
\[
f^{(n)}(z_0) = \frac{n!}{2\pi i} \int_\gamma \frac{f(z)}{(z-z_0)^{n+1}}\,dz.
\]
\end{theorem}

\begin{theorem}[Morera's Theorem]
Let \(f\) be continuous on an open set \(U\). If
\[
\int_\gamma f(z)\,dz = 0
\]
for every closed contour \(\gamma\) in \(U\), then \(f\) is analytic in \(U\).
\end{theorem}

\begin{theorem}[Taylor's Theorem for Analytic Functions]
If \(f\) is analytic in a disk \(|z-z_0| < R\), then
\[
f(z) = \sum_{n=0}^{\infty} \frac{f^{(n)}(z_0)}{n!}(z-z_0)^n,
\]
with the series converging absolutely for \(|z-z_0| < R\).
\end{theorem}

\begin{theorem}[Laurent Series Theorem]
If \(f\) is analytic in an annulus \(r < |z-z_0| < R\), then \(f\) can be represented by a Laurent series
\[
f(z) = \sum_{n=-\infty}^{\infty} c_n (z-z_0)^n,
\]
which converges for \(z\) in the annulus.
\end{theorem}

\subsubsection*{Laurent Series (Inside an Annulus):}

If \( f(z) \) is analytic on the annulus 
\[
r < |z - z_0| < R,
\]
then \( f(z) \) can be decomposed into:
\[
f(z) = f_1(z) + f_2(z),
\]
where
\[
f_1(z) = \sum_{n=0}^{\infty} a_n (z - z_0)^n \quad\text{(analytic part)},
\]
and
\[
f_2(z) = \sum_{n=1}^{\infty} a_{-n} (z - z_0)^{-n} \quad\text{(principal part)}.
\]

Here, \(f_1(z)\) is analytic at \(z_0\), while \(f_2(z)\) contains the singular behavior of \(f(z)\) at \(z_0\).

\begin{theorem}[Residue Theorem]
Let \(f\) be analytic in a region \(U\) except for finitely many isolated singularities \(z_1, z_2, \dots, z_n\). If \(\gamma\) is a positively oriented, simple closed contour in \(U\) that encloses these singularities, then
\[
\int_\gamma f(z)\,dz = 2\pi i \sum_{k=1}^{n} \operatorname{Res}(f,z_k).
\]
\end{theorem}

\begin{theorem}[Classification of Isolated Singularities]
Let \(z_0\) be an isolated singularity of \(f\). Then \(z_0\) is either:
\begin{enumerate}[label=(\roman*)]
    \item a \textbf{removable singularity},
    \item a \textbf{pole} of order \(m\) (if there exists a minimal positive integer \(m\) such that \((z-z_0)^m f(z)\) is analytic and nonzero at \(z_0\)), or
    \item an \textbf{essential singularity}.
\end{enumerate}
\end{theorem}

\begin{theorem}[Casorati–Weierstrass]
If \(f(z)\) is analytic in a neighborhood of a point \(z_0\) except possibly at \(z_0\), where \(f(z)\) has an essential singularity, then the set of values taken by \(f(z)\) in any punctured neighborhood of \(z_0\) is dense in the complex plane. 

Formally, for every \(\varepsilon > 0\) and every complex number \(w\), there exists a \(z\) with \(0 < |z - z_0| < \varepsilon\) such that
\[
|f(z) - w| < \varepsilon.
\]
\end{theorem}

\begin{theorem}[Riemann's Removable Singularity]
Suppose a function \( f(z) \) is analytic on a punctured neighborhood \(0 < |z - z_0| < r \) and that
\[
\lim_{z \to z_0}(z - z_0)f(z) = 0.
\]

Then the singularity at \( z_0 \) is removable. In other words, \( f(z) \) can be analytically extended to \( z_0 \) by defining
\[
f(z_0) = \lim_{z \to z_0} f(z).
\]
\end{theorem}

\begin{theorem}[Maximum Modulus Principle]
If \(f\) is a nonconstant analytic function on a domain \(U\), then \(|f(z)|\) cannot attain a local maximum in \(U\).
\end{theorem}

\begin{theorem}[Liouville's Theorem]
Every bounded entire function (analytic on \(\mathbb{C}\)) is constant.
\end{theorem}

\begin{theorem}[Fundamental Theorem of Algebra]
Every nonconstant polynomial \(p(z)\) with complex coefficients has at least one zero in \(\mathbb{C}\).
\end{theorem}

\begin{theorem}[Argument Principle]
Let \(f\) be meromorphic inside and on a closed contour \(\gamma\) (with no zeros or poles on \(\gamma\)). Then
\[
\frac{1}{2\pi i}\int_\gamma \frac{f'(z)}{f(z)}\,dz = N - P,
\]
where \(N\) (respectively, \(P\)) is the number of zeros (respectively, poles) of \(f\) inside \(\gamma\), counted with multiplicity.
\end{theorem}

\begin{theorem}[Rouché's Theorem]
Let \(f\) and \(g\) be analytic inside and on a simple closed contour \(\gamma\). If \(|g(z)| < |f(z)|\) on \(\gamma\), then \(f\) and \(f+g\) have the same number of zeros inside \(\gamma\), counting multiplicity.
\end{theorem}

\begin{theorem}[Schwarz Reflection Principle]
Suppose \(f\) is analytic in a domain \(U\) that is symmetric with respect to the real axis, and \(f\) maps the real axis into \(\mathbb{R}\). Then \(f\) can be extended to an analytic function on the reflection of \(U\) across the real axis by defining
\[
f(\overline{z}) = \overline{f(z)}.
\]
\end{theorem}

\begin{theorem}[Open Mapping Theorem]
If \(f\) is a nonconstant analytic function on a domain \(U\), then \(f\) maps open subsets of \(U\) onto open subsets of \(\mathbb{C}\).
\end{theorem}

\begin{theorem}[Identity Theorem]
Let \(f\) and \(g\) be analytic on a connected domain \(U\). If the set
\[
\{ z \in U : f(z)=g(z) \}
\]
has an accumulation point in \(U\), then \(f\equiv g\) on \(U\).
\end{theorem}

\section{Lemmas and Propositions}

\begin{lemma}[Cauchy's Inequality]
Let \(f\) be analytic in the closed disk \(\overline{D}(z_0,R)=\{z: |z-z_0| \le R\}\). Then, for every \(n \ge 0\),
\[
\left| f^{(n)}(z_0) \right| \le \frac{n!}{R^n} \max_{|z-z_0|=R} |f(z)|.
\]
\end{lemma}

\begin{proposition}[Properties of Contour Integrals]
If \(f\) is continuous on a contour \(\gamma\), then the value of the contour integral
\[
\int_\gamma f(z)\,dz
\]
is independent of the parameterization of \(\gamma\).
\end{proposition}

\begin{proposition}[Existence of Primitives]
If \(f\) is analytic in a simply connected domain \(U\), then \(f\) has an antiderivative (or primitive) \(F\) in \(U\) such that
\[
F'(z) = f(z) \quad \text{for all } z \in U.
\]
\end{proposition}
\textbf{PROPOSITION} \quad Suppose $P$ and $Q$ are polynomials that are real-valued on the real axis and for which the degree of $Q$ exceeds the degree of $P$ by 2 or more. If $Q(x) \neq 0$ for all real $x$, then
\begin{align}
    \int_{-\infty}^{\infty} \frac{P(x)}{Q(x)} \, dx = 2\pi i \sum_U \operatorname{Res}\left(\frac{P}{Q}; z_j\right),
\end{align}
where the sum is taken over all poles of $P/Q$ that lie in the upper half-plane $U = \{z : \operatorname{Im}\, z > 0\}$.

\end{document}
