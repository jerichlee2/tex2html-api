\documentclass[12pt]{article}

% Packages
\usepackage[margin=1in]{geometry}
\usepackage{amsmath,amssymb,amsthm}
\usepackage{enumitem}
\usepackage{hyperref}
\usepackage{xcolor}
\usepackage{import}
\usepackage{xifthen}
\usepackage{pdfpages}
\usepackage{transparent}
\usepackage{listings}
\usepackage{tikz}
\usepackage{physics}
\usepackage{siunitx}
\usepackage{booktabs}
\usepackage{cancel}
  \usetikzlibrary{calc,patterns,arrows.meta,decorations.markings}


\DeclareMathOperator{\Log}{Log}
\DeclareMathOperator{\Arg}{Arg}

\lstset{
    breaklines=true,         % Enable line wrapping
    breakatwhitespace=false, % Wrap lines even if there's no whitespace
    basicstyle=\ttfamily,    % Use monospaced font
    frame=single,            % Add a frame around the code
    columns=fullflexible,    % Better handling of variable-width fonts
}

\newcommand{\incfig}[1]{%
    \def\svgwidth{\columnwidth}
    \import{./Figures/}{#1.pdf_tex}
}
\theoremstyle{definition} % This style uses normal (non-italicized) text
\newtheorem{solution}{Solution}
\newtheorem{proposition}{Proposition}
\newtheorem{problem}{Problem}
\newtheorem{lemma}{Lemma}
\newtheorem{theorem}{Theorem}
\newtheorem{remark}{Remark}
\newtheorem{note}{Note}
\newtheorem{definition}{Definition}
\newtheorem{example}{Example}
\newtheorem{corollary}{Corollary}
\theoremstyle{plain} % Restore the default style for other theorem environments
%

% Theorem-like environments
% Title information
\title{}
\author{Jerich Lee}
\date{\today}

\begin{document}

\maketitle
\begin{solution}
  Let $f$ be analytic in the punctured plane $\mathbb{C}\setminus\{0\}$ and
  assume the global estimate
  \[
     |f(z)| \;\le\; |z|^{-10.5},
     \qquad z\neq 0.
  \]
  
  \bigskip
  %%%%%%%%%%%%%%%%%%%%%%%%%%%%%%%%%%%%%%%%%%%%%%%%%%%%%%%%%%%%%%%%%%%%%%%%%%%%
  \textbf{1.\ Laurent expansion about the origin.}
  
  Because $f$ is analytic on an annulus centred at~$0$ of \emph{any} positive
  inner and outer radius, it admits a Laurent series that converges
  throughout $\mathbb{C}\setminus\{0\}$:
  \[
     f(z)
     \;=\;
     \sum_{n=-\infty}^{+\infty} a_n z^{\,n},
     \qquad z\neq 0.
  \]
  
  \bigskip
  %%%%%%%%%%%%%%%%%%%%%%%%%%%%%%%%%%%%%%%%%%%%%%%%%%%%%%%%%%%%%%%%%%%%%%%%%%%%
  \textbf{2.\ Cauchy estimates for Laurent coefficients.}
  
  For any radius $R>0$ let
  \[
     M_R \;:=\; \max_{|z|=R} |f(z)|.
  \]
  By hypothesis, $M_R\le R^{-10.5}$.  
  Cauchy’s integral formula for Laurent coefficients gives
  \[
     a_n
     \;=\;
     \frac{1}{2\pi i} \oint_{|z|=R} \frac{f(z)}{z^{\,n+1}}\;dz,
     \qquad\text{so that}\qquad
     |a_n|
     \;\le\;
     \frac{M_R}{R^{\,n}}
     \;=\;
     R^{-10.5-n}.
  \]
  This inequality is valid for \emph{every} $R>0$.
  
  \bigskip
  %%%%%%%%%%%%%%%%%%%%%%%%%%%%%%%%%%%%%%%%%%%%%%%%%%%%%%%%%%%%%%%%%%%%%%%%%%%%
  \textbf{3.\ Vanishing of all coefficients.}
  
  \begin{itemize}
    \item[(i)] \emph{Coefficients with $n\ge -10$.}\;
          Fix such an integer $n$ and let $R\to\infty$ in the estimate
          $|a_n|\le R^{-10.5-n}$.  Because $-10.5-n<0$, the right‑hand side
          tends to $0$, hence $a_n=0$.
  
    \item[(ii)] \emph{Coefficients with $n\le -11$.}\;
          Now fix $n\le-11$ and let $R\to 0^{+}$.  Since
          $-10.5-n>0$, we still have $R^{-10.5-n}\to 0$, giving $a_n=0$.
  \end{itemize}
  
  Thus $a_n=0$ for \emph{all} $n\in\mathbb{Z}$.
  
  \bigskip
  %%%%%%%%%%%%%%%%%%%%%%%%%%%%%%%%%%%%%%%%%%%%%%%%%%%%%%%%%%%%%%%%%%%%%%%%%%%%
  \textbf{4.\ Conclusion.}
  
  Every Laurent coefficient of $f$ vanishes, whence $f$ itself is
  identically zero on $\mathbb{C}\setminus\{0\}$:
  \[
     \boxed{\,f(z)=0\quad\text{for all }z\neq 0.}
  \]
  \end{solution}
  %%%%%%%%%%%%%%%%%%%%%%%%%%%%%%%%%%%%%%%%%%%%%%%%%%%%%%%%%%%%%%%%%%%%%%%%%%%%
\subsection*{Cauchy estimates for Laurent coefficients (fully expanded)}

\paragraph{ Laurent series of $f$.}
Because $f$ is analytic on the punctured plane $\mathbb{C}\setminus\{0\}$,
it admits a \emph{globally convergent} Laurent expansion  
\[
   f(z)=\sum_{n=-\infty}^{+\infty} a_n z^{\,n},
   \qquad z\neq 0.
\]

\paragraph{Coefficient formula.}
For any fixed radius $R>0$ the circle $\Gamma_R:=\{\,|z|=R\,\}$
lies entirely inside the domain of analyticity,
so Cauchy’s integral representation for Laurent coefficients gives  
\begin{equation}\label{eq:coeff}
   a_n
   \;=\;
   \frac{1}{2\pi i}
   \oint_{\Gamma_R}
      \frac{f(z)}{z^{\,n+1}}\,
   dz,
   \qquad n\in\mathbb{Z} .
\end{equation}

\paragraph{Bounding $|a_n|$.}
Write $M_R:=\max_{|z|=R} |f(z)|$.
Taking moduli in~\eqref{eq:coeff} and using
$|dz|=R\,d\theta$ (arc‑length element) and $|z|=R$ on $\Gamma_R$,
\[
   |a_n|
   \;\le\;
   \frac{1}{2\pi}
   \int_{0}^{2\pi}
      \frac{M_R}{R^{\,n+1}}\,
      R\,d\theta
   \;=\;
   \frac{M_R}{R^{\,n}},
   \qquad\text{for every }R>0.
\]

\paragraph{Inserting the growth hypothesis.}
The problem assumes $|f(z)|\le |z|^{-10.5}$ for all $z\neq 0$,
hence $M_R\le R^{-10.5}$.
Substituting,
\[
   \boxed{\;
      |a_n|
      \;\le\;
      R^{-10.5-n},
      \qquad\text{for \emph{every} }R>0\text{ and all }n\in\mathbb{Z}.
   \;}
\]

\medskip
\noindent
\textsl{Why this single inequality is so powerful.}
\begin{itemize}
   \item If $n\ge -10$, choose $R\to\infty$.
         Then the exponent $-10.5-n<0$, so $R^{-10.5-n}\to 0$,
         forcing $a_n=0$.
   \item If $n\le -11$, choose $R\to 0^{+}$.
         Now $-10.5-n>0$ and again $R^{-10.5-n}\to 0$,
         giving $a_n=0$.
\end{itemize}
Because \emph{every} integer satisfies one of these two cases, all
coefficients vanish and thus $f\equiv 0$ in $\mathbb{C}\setminus\{0\}$.
%%%%%%%%%%%%%%%%%%%%%%%%%%%%%%%%%%%%%%%%%%%%%%%%%%%%%%%%%%%%%%%%%%%%%%%%%%%%
\pagebreak
\begin{problem}
   Find the Laurent expansion 
   \[
      f(z)\;=\;2\,z^{-3}+\frac1{z+2}
      \;=\;
      \sum_{n=-\infty}^{\infty} a_n\,(z-1)^{\,n},
      \qquad 
      1<|z-1|<3 .
   \]
   \end{problem}

   \begin{solution}
   Throughout we set 
   \[
      w:=z-1,
      \qquad\qquad
      z=1+w .
   \]
   The annulus $1<|w|<3$ is centred at $w=0$.
   
   %%%%%%%%%%%%%%%%%%%%%%%%%%%%%%%%%%%%%%%%%%%%%%%%%%%%%%%%%%%%%%%%%%%%%%%%%%%%
   \subsection*{(a) A series for \boldmath$z^{-1}$ valid when $|w|>1$}
   
   \[
      \frac1{z}
      =\frac1{1+w}
      =\frac1{w}\,\frac1{1+\dfrac1{w}}
      =\frac1{w}\sum_{k=0}^{\infty}(-1)^{k}\,w^{-k}
      =\sum_{k=0}^{\infty}(-1)^{k}\,w^{-(k+1)} .
   \]
   
   Renaming the index $n:=-(k+1)$ ($n=-1,-2,\dots$) we obtain
   \[
      \boxed{\;
         \displaystyle
         z^{-1}
         =\sum_{n=-\infty}^{-1} (-1)^{\,n+1}\,w^{\,n},
         \quad |w|>1
      \;}
   \]
   
   %%%%%%%%%%%%%%%%%%%%%%%%%%%%%%%%%%%%%%%%%%%%%%%%%%%%%%%%%%%%%%%%%%%%%%%%%%%%
   \subsection*{(b) A series for \boldmath$2\,z^{-3}$}
   
   Because $\dfrac{d^{\,2}}{dz^{2}}(z^{-1})=2\,z^{-3}$, we may differentiate
   the series in (a) termwise:
   
   \[
   \begin{aligned}
      \frac{d}{dz}\bigl(w^{\,n}\bigr)
         &= n\,w^{\,n-1},\\[2pt]
      \frac{d^{\,2}}{dz^{2}}\bigl(w^{\,n}\bigr)
         &= n(n-1)\,w^{\,n-2}.
   \end{aligned}
   \]
   
   Hence
   \[
      2\,z^{-3}
      =\sum_{n=-\infty}^{-1} (-1)^{\,n+1}\,
        n(n-1)\,w^{\,n-2}
      =\sum_{m=-\infty}^{-3}
         (-1)^{\,m+1}\,(m+2)(m+1)\,w^{\,m},
   \]
   where we have set $m:=n-2$.
   Thus
   \[
      \boxed{\;
         2\,z^{-3}
         =\sum_{n=-\infty}^{-3}
             (-1)^{\,n+1}\,(n+2)(n+1)\,w^{\,n},
         \quad |w|>1
      \;}
   \]
   
   %%%%%%%%%%%%%%%%%%%%%%%%%%%%%%%%%%%%%%%%%%%%%%%%%%%%%%%%%%%%%%%%%%%%%%%%%%%%
   \subsection*{(c) A series for \boldmath$(z+2)^{-1}$ valid when $|w|<3$}
   
   \[
      \frac1{z+2}
      =\frac1{3+w}
      =\frac1{3}\,\frac1{1+\dfrac{w}{3}}
      =\frac1{3}\sum_{k=0}^{\infty}(-1)^{k}\Bigl(\frac{w}{3}\Bigr)^{k}
      =\sum_{k=0}^{\infty}
           (-1)^{k}\,3^{-(k+1)}\,w^{\,k}.
   \]
   
   Relabelling $n:=k$ ($n=0,1,2,\dots$) we have
   \[
      \boxed{\;
         \frac1{z+2}
         =\sum_{n=0}^{\infty}
            (-1)^{\,n}\,3^{-(n+1)}\,w^{\,n},
         \quad |w|<3
      \;}
   \]
   
   %%%%%%%%%%%%%%%%%%%%%%%%%%%%%%%%%%%%%%%%%%%%%%%%%%%%%%%%%%%%%%%%%%%%%%%%%%%%
   \subsection*{(d) Combine the series}
   
   Both series in (b) and (c) converge in the overlap $1<|w|<3$, so their
   sum gives the desired Laurent expansion of $f$:
   
   \[
      f(z)
      =\sum_{n=-\infty}^{-3}
           (-1)^{\,n+1}\,(n+2)(n+1)\,w^{\,n}
        \;+\;
        \sum_{n=0}^{\infty}
           (-1)^{\,n}\,3^{-(n+1)}\,w^{\,n},
      \qquad 1<|w|<3 .
   \]
   
   \bigskip
   \noindent
   \textbf{Coefficients $a_n$.}
   Reading off the coefficients of $w^{\,n}=(z-1)^{\,n}$:
   
   \[
      a_n =
      \begin{cases}
         (-1)^{\,n+1}\,(n+2)(n+1), & n\le -3,\\[6pt]
         0,                        & n=-2,-1,\\[6pt]
         (-1)^{\,n}\,3^{-(n+1)},   & n\ge 0.
      \end{cases}
   \]
   
   \[
      \boxed{\displaystyle
         f(z)=\sum_{n=-\infty}^{\infty} a_n\,(z-1)^{\,n},
         \quad
         a_n\text{ as above},
         \quad
         1<|z-1|<3
      }
   \]
   \end{solution}
   \pagebreak
   \begin{theorem}[Schwarz’s Lemma]
      Let $f$ be analytic on the open unit disc
      \[
         \mathbb{D} := \{\,z\in\mathbb{C}:|z|<1\,\},
         \qquad
         f(0)=0,
         \qquad
         |f(z)|\le 1 \;\text{ for all } z\in\mathbb{D}.
      \]
      Then
      \[
         |f(z)|\le |z|,
         \qquad |z|<1.
      \]
      Moreover, if equality holds at some point $z_0\ne 0$, then
      $f(z)=\lambda z$ for all $z\in\mathbb{D}$ with $|\lambda|=1$.
      \end{theorem}
      
      \begin{proof}
      \textbf{1.  Define an auxiliary function.}
      Because $f(0)=0$, the quotient
      \[
         g(z):=\frac{f(z)}{z},
         \qquad z\in\mathbb{D}\setminus\{0\},
      \]
      extends analytically to $z=0$ by setting $g(0):=f'(0)$ (removable
      singularity).  Hence $g$ is analytic on the whole disc $\mathbb{D}$.
      
      \medskip
      \textbf{2.  Boundary estimate on a circle of radius $r<1$.}
      Fix $0<r<1$ and consider the circle $\Gamma_r=\{\,|z|=r\,\}$.
      For $z\in\Gamma_r$ we have
      \[
         |g(z)|
         =\frac{|f(z)|}{|z|}
         \le\frac{1}{r},
      \]
      because $|f(z)|\le 1$ by hypothesis and $|z|=r$.
      
      \medskip
      \textbf{3.  Maximum‑modulus principle.}
      Since $g$ is analytic on $|z|\le r$ and $|g(z)|\le r^{-1}$ on the
      boundary, the maximum‑modulus principle gives
      \[
         |g(z)|\le \frac{1}{r},
         \qquad |z|<r.
      \]
      Because $r$ can be chosen arbitrarily close to $1$, letting $r\to 1^{-}$
      yields
      \[
         |g(z)|\le 1,
         \qquad |z|<1.
      \]
      
      \medskip
      %%%%%%%%%%%%%%%%%%%%%%%%%%%%%%%%%%%%%%%%%%%%%%%%%%%%%%%%%%%%%%%%%%%%%%%%%%%%
\paragraph{Step 3 (expanded):  Application of the Maximum–Modulus Principle}

\textbf{Set–up.}  
Fix a number $r$ with $0<r<1$ and denote by
\[
   \overline{D_r}\;:=\;\{\,z\in\mathbb{C} : |z|\le r\,\}, 
   \qquad 
   \partial D_r=\{\,|z|=r\,\}
\]
the closed disc of radius $r$ and its boundary circle, respectively.
The auxiliary function
\[
   g(z):=\frac{f(z)}{z}
\]
is analytic on the open unit disc $\mathbb{D}$ and therefore analytic on the
smaller open disc $D_r=\{\,|z|<r\,\}$ as well; in addition, $g$ is
\emph{continuous} on the closure $\overline{D_r}$.

\medskip
\textbf{Boundary estimate.}  
By hypothesis $|f(z)|\le 1$ for every $z\in\mathbb{D}$, and on $\partial D_r$
we have $|z|=r$; hence for $z\in\partial D_r$
\[
   |g(z)|
   \;=\;\frac{|f(z)|}{|z|}
   \;\le\;\frac{1}{r}.
\]

\medskip
\textbf{Maximum–Modulus Principle (MMP).}  
Because $g$ is analytic on $D_r$ and continuous on $\overline{D_r}$,
the MMP states that
\[
   \max_{|z|\le r}|g(z)|
   \;=\;
   \max_{|z|=r}|g(z)|.
\]
The right–hand side is bounded by $1/r$, so
\[
   |g(z)|\;\le\;\frac{1}{r},
   \qquad
   \text{for every }z\text{ with }|z|<r.
\]

\medskip
\textbf{Passing to the unit disc.}  
The radius $r$ was \emph{arbitrary} in $(0,1)$.
Letting $r\to 1^{-}$ (i.e.\ choosing radii closer and closer to~$1$),
the bound $1/r$ decreases to $1$:
\[
   \forall\,z\in\mathbb{D}\; (|z|<1)\quad
   |g(z)|
   \;\le\;
   \lim_{r\to 1^-} \frac{1}{r}
   = 1.
\]

\[
   \boxed{\;|g(z)|\le 1,\quad |z|<1\;}
\]

\noindent
This inequality is the crucial bridge between the given constraint
$|f(z)|\le 1$ and the desired estimate $|f(z)|\le |z|$, because
$f(z)=z\,g(z)$.
%%%%%%%%%%%%%%%%%%%%%%%%%%%%%%%%%%%%%%%%%%%%%%%%%%%%%%%%%%%%%%%%%%%%%%%%%%%%
      \textbf{4.  Inequality for $f$.}
      Recalling that $g(z)=f(z)/z$ we obtain
      \[
         |f(z)|
         =|z|\,|g(z)|
         \le |z|\cdot 1
         = |z|,
         \qquad |z|<1.
      \]
      
      \medskip
      \textbf{5.  Characterisation of the equality case.}
      Assume there exists $z_0\in\mathbb{D}\setminus\{0\}$ with
      $|f(z_0)|=|z_0|$.  Then $|g(z_0)|=1$, so $|g|$ attains its maximum
      value inside $\mathbb{D}$.  By the maximum‑modulus principle, $g$ must
      be constant:
      \[
         g(z)\equiv\lambda,
         \qquad |\lambda|=1.
      \]
      Consequently $f(z)=\lambda z$ for all $z\in\mathbb{D}$.
      
      \end{proof}
      %---------------------------------------------------------
%  Intuitive meaning of Schwarz’s Lemma
%---------------------------------------------------------
\pagebreak
\begin{solution}

   \begin{center}
   \begin{tabular}{|l|p{8cm}|}
   \hline
   \textbf{Formal hypothesis} & \textbf{Geometric meaning} \\ \hline\hline
   $f$ analytic on the unit disc $\mathbb{D}:=\{\,|z|<1\}$ and $f(0)=0$
     & The map is holomorphic everywhere inside the disc and \emph{anchors} the centre to itself. \\ \hline
   $|f(z)|\le 1$ for all $z\in\mathbb{D}$
     & $f$ never leaves the unit disc—it is “confined’’ by the same circular boundary. \\ \hline
   \end{tabular}
   \end{center}
   
   \bigskip
   \subsection*{1.\ A “speed‑limit’’ inside the disc}
   Visualise the unit disc as a sheet of perfectly elastic rubber whose centre is nailed to the table.
   Schwarz’s Lemma states that
   \[
      |f(z)|\le |z| \quad (|z|<1),
   \]
   which means: \emph{when you tug on any interior point, you can never move it farther from the centre than you moved your own finger}.  
   The map cannot \emph{expand} radial distances; it can only shrink them.
   
   \bigskip
   \subsection*{2.\ Local‐to‐global control}
   The inequality is global—it holds for \emph{every} $z$—yet it is forced purely by local analyticity plus the boundary bound.
   In complex analysis, boundary information “leaks’’ into the interior.
   
   \bigskip
   \subsection*{3.\ The rigidity phenomenon}
   If equality occurs at some interior point $z_0\neq0$, i.e.\ $|f(z_0)|=|z_0|$, the rubber sheet is stretched to its limit there.
   Analyticity forbids any further distortion, so the entire map must be a rigid rotation:
   \[
      f(z)=\lambda z, \qquad |\lambda|=1.
   \]
   
   \bigskip
   \subsection*{4.\ Derivative bound at the centre}
   Zooming in at $z=0$,
   \[
      |f'(0)|\le 1.
   \]
   The initial “speed’’ with which $f$ moves points away from $0$ cannot exceed the speed limit set by Schwarz’s Lemma.  
   If $|f'(0)|=1$, then $f(z)=\lambda z$ globally.
   
   \bigskip
   \subsection*{5.\ Practical takeaway}
   Schwarz’s Lemma is the prototype of many \emph{distortion theorems}:
   \[
      \text{Boundary control} \;+\; \text{analyticity} \;\Longrightarrow\; \text{strong interior control}.
   \]
   Holomorphic maps are dramatically less flexible than real‐variable functions; once pinned down at a point and bounded on the boundary, their stretching behaviour—and, in extremal cases, their entire form—is essentially dictated.
   
   \end{solution}
   %---------------------------------------------------------
%  How the Maximum–Modulus Principle (MMP) drives Schwarz’s Lemma
%---------------------------------------------------------
\begin{solution}

   \subsection*{1.\  The two ingredients of Schwarz’s Lemma}
   
   \begin{enumerate}[label=\textbf{\arabic*.},wide,labelwidth=!, labelindent=0pt]
     \item \textbf{Boundary control}  
           We are told that the analytic map $f$ never leaves the unit disc:
           \[
              |f(z)| \;\le\; 1
              \quad\text{for all } z\text{ with }|z|<1.
           \]
     \item \textbf{Anchoring condition}  
           The origin is fixed: $f(0)=0$.
   \end{enumerate}
   
   \medskip
   \subsection*{2.\  Rephrasing the problem in terms of \boldmath$g(z)=\frac{f(z)}{z}$}
   
   Since $f(0)=0$, the quotient
   \[
      g(z):=\frac{f(z)}{z}
   \]
   is analytic on the whole disc (removable singularity at $z=0$).  
   If we can show that $|g(z)|\le1$ inside $\mathbb{D}$, then multiplying by $|z|$
   immediately gives the desired inequality
   $|f(z)|\le|z|$.
   
   \medskip
   \subsection*{3.\  What the Maximum–Modulus Principle (MMP) says intuitively}
   
   > *“An analytic function cannot achieve a \emph{strict} maximum of its
   >  modulus in the interior of a domain—unless it is constant.”*
   
   Picture the modulus $|g|$ as a flexible membrane stretched over the
   disc: the MMP forces this membrane to attain all its high points on the
   boundary and to “sag’’ inward everywhere else.
   
   \medskip
   \subsection*{4.\  Applying the MMP on concentric circles}
   
   \begin{enumerate}[label=\textbf{\alph*.},wide,labelwidth=!, labelindent=0pt]
     \item Fix a radius $r$ with $0<r<1$ and look at the \emph{smaller} disc
           $D_r:=\{\,|z|<r\,\}$.\par
           \vskip4pt
           Along its boundary $|z|=r$ we already know
           \[
              |g(z)| = \frac{|f(z)|}{|z|}
                      \;\le\; \frac{1}{r}.
           \]
     \item By the MMP this boundary bound automatically pushes into the interior:
           \[
              |g(z)|\;\le\;\frac{1}{r},
              \qquad |z|<r.
           \]
   \end{enumerate}
   
   \medskip
   \subsection*{5.\  Expanding the control to the \emph{whole} unit disc}
   
   The crucial observation: the radius $r$ was arbitrary.  
   As we let $r\uparrow1$, the factor $1/r$ slides down to~$1$.  
   Therefore
   \[
      |g(z)|\;\le\;1
      \quad\text{for every } z\text{ with }|z|<1,
   \]
   which instantly yields
   \[
      |f(z)| \;=\; |z|\cdot|g(z)|
                 \;\le\; |z|.
   \]
   
   \medskip
   \subsection*{6.\  Where does the \emph{rigidity} come from?}
   
   Suppose equality happens at a single interior point $z_0\neq0$:
   $|f(z_0)|=|z_0|$.  
   Then $|g(z_0)|=1$, so the modulus $|g|$ attains its \emph{maximum} value
   inside the domain, contradicting the MMP \emph{unless $g$ is constant}.
   Hence $g(z)\equiv\lambda$ with $|\lambda|=1$, and
   \[
      f(z)=\lambda z
   \]
   is forced globally.  
   One point of equality rigidly determines the entire map.
   
   \bigskip
   \noindent
   \textbf{Take‑home intuition.}\;
   The MMP is the engine that turns a \emph{weak}, one–line boundary
   condition ($|f|\le1$) into a \emph{strong}, everywhere‑inside constraint
   ($|f(z)|\le|z|$) and, in the extremal case, pins the whole function down
   to a rigid rotation.  Holomorphy couples boundary behaviour to interior
   behaviour so tightly that a single inequality on the rim controls every
   point in the disc.
   
   \end{solution}
   \noindent
   \begin{enumerate}
      \item From the upper semi-circle from [-1,1] to the upper half plane: $w = -\left[ (z+1)^{-1}-\frac{1}{2} \right]^{2} $
      \item reducing from the upper half-plane strip from [-1,1] to case 1: $i(e^{\frac{\pi i z}{2}}$ 
      \item from the upper half-plane, excluding all points from [0,i] on the imaginary axis: $w=(z^{2}+1)^{\frac{1}{2}}$
      \item from the entire plane excluding $\left[ -\infty, -1 \right] \cup\left[ 1,\infty  \right] $: $w=\sqrt{2}i \left[ (z+1)^{-1}- \frac{1}{2} \right]^{\frac{1}{2}}  $
      \item reducing from the upper semi-circle from $\left[ -1,1 \right] $ excluding all points from $\left[ 0, \frac{i}{2} \right] $  on the imaginary axis: $i(\frac{6}{5}((\frac{1}{(z^{2}+1)}-\frac{1}{2})))$ to case 3
      \item from the horizontal strip from negative infinity to positive infinity and from -i to i, excluding the negative real axis: $w=i(e^{\pi z}-1)^{\frac{1}{2}}$ 
      \item the upper ellipse segment from -1 to 1, with tangent angles of alpha at the end points of -1 and 1: $w = -(\frac{2}{z+1}-1)^{\frac{\pi}{2}}$ 
      \item the circle centered at zero excluding the negative real axis and all real values greater than 1/3: $w= i(\frac{3}{4(\frac{2}{z+1}-1)^{2}-1}-1)^{\frac{1}{2}}$ 
   \end{enumerate}
   \begin{solution}
      We build the map  
      \[
      F(z)\;=\;i\bigl(e^{\pi z}-1\bigr)^{\tfrac12}
      \]
      as the composition of four elementary conformal maps.  Throughout,
      the \emph{principal branch} of every multi–valued function is used.
      
      \medskip
      %-------------------------------
      \textbf{Domain to be mapped.}
      \[
      \boxed{
         \mathcal S
         =\bigl\{\,z\in\mathbb{C} : -1<\Im z<1\bigr\}
          \setminus (-\infty,0)
       }
      \]
      is the horizontal strip of height~$2$, with the \emph{negative real
      axis} removed.
      
      \medskip
      %-------------------------------
      \begin{enumerate}[label=\textbf{Step \arabic*:}, itemsep=1.4ex]
      
      %---------------------------------
      \item  \textbf{Exponentiation to a slit plane.}  
      \[
      u \;=\; e^{\pi z}.
      \]
      Write $z=x+iy$; then  
      $u=e^{\pi x}\,e^{i\pi y}$ with
      \(
      -\pi<\arg u<\pi.
      \)
      Hence
      \[
      \mathcal S
      \;\xrightarrow{\,e^{\pi z}\,}\;
      \mathcal D_1
        = \mathbb{C}\setminus(-\infty,0],
      \]
      the plane cut along the \emph{non–positive} real axis.  
      The strip’s excluded set $(-\infty,0)$ (real, $y=0$) is mapped to the
      interval $(0,1)$ on the \emph{positive} real axis, so that
      \(
      \mathcal D_1
      =\mathbb{C}\setminus(-\infty,1].
      \)
      
      %---------------------------------
      \item \textbf{Translate the cut so it starts at the origin.}  
      \[
      v \;=\; u-1.
      \]
      This sends the point $u=1$ to the origin, converting the slit
      $(-\infty,1]$ into
      \[
      (-\infty,0].
      \]
      Thus
      \[
      \mathcal D_2
        = \mathbb{C}\setminus(-\infty,0],
      \]
      a plane cut exactly along the \emph{negative} real axis.
      
      %---------------------------------
      \item \textbf{Principal square root to the right half–plane.}  
      \[
      w_0 \;=\; \sqrt{v},
      \qquad
      -\pi<\arg v<\pi.
      \]
      Because $\sqrt{\;}$ halves arguments,
      \(
      -\tfrac{\pi}{2}<\arg w_0<\tfrac{\pi}{2},
      \)
      so
      \[
      \mathcal D_2
      \;\xrightarrow{\,\sqrt{\;}\,}\;
      \mathcal D_3
        = \{w_0\in\mathbb{C} : \Re w_0>0\},
      \]
      the \emph{right half–plane}.  
      (The negative real cut becomes the imaginary axis,
      which is the branch cut of the square root.)
      
      %---------------------------------
      \item \textbf{Quarter-turn into the upper half–plane.}  
      \[
      w \;=\; i\,w_0.
      \]
      Multiplying by $i$ rotates the right half–plane
      counter-clockwise by~$90^{\circ}$:
      \[
      \mathcal D_3
      \;\xrightarrow{\,i\cdot\,}\;
      \boxed{
         \mathcal H
         = \{w\in\mathbb{C} : \Im w>0\}
       }.
      \]
      
      \end{enumerate}
      
      \medskip
      %-------------------------------
      \textbf{Summary.}  
      The composite map
      \[
      \boxed{\;
         F(z)=i\bigl(e^{\pi z}-1\bigr)^{\tfrac12}
       \;}
      \]
      is analytic and one–to–one on the strip
      $\mathcal S$, and carries $\mathcal S$ \emph{onto} the standard
      upper half–plane~$\mathcal H$.
      \end{solution} 
      %==========================================================
%           Conformal Mapping ― Quick-Reference
%==========================================================
% Minimal preamble additions you might need:
%
% \usepackage{amsmath,amssymb}   % basic math
% \usepackage{mathtools}         % \coloneqq etc. (optional)
%----------------------------------------------------------

\begin{center}
   \Large\bfseries Conformal Mapping Cheat Sheet
 \end{center}
 \medskip
 
 \begin{itemize}
 %-----------------------------------------------------------------
 \item[\textbf{(1)}] \textbf{Definition.}
       A function \(f\colon\Omega\!\to\!\Bbb C\) on a domain
       \(\Omega\subset\Bbb C\) is \emph{conformal} at
       \(z_{0}\in\Omega\) iff it is analytic there and
       \(f'(z_{0})\neq 0\).
       (Equivalent: it preserves oriented angles and infinitesimal
       shapes.)
 
 %-----------------------------------------------------------------
 \item[\textbf{(2)}] \textbf{Cauchy–Riemann $\Longrightarrow$ conformal.}
       If \(u,v\) satisfy the C–R system
       \(u_{x}=v_{y},\;u_{y}=-v_{x}\) and
       \(\nabla u\neq(0,0)\) at a point, then
       \(f=u+iv\) is conformal there.
 
 %-----------------------------------------------------------------
 \item[\textbf{(3)}] \textbf{Angle preservation.}
       For curves \(\gamma_{1},\gamma_{2}\) meeting at \(z_{0}\),
       the oriented angle between \(f\!\circ\!\gamma_{1}\) and
       \(f\!\circ\!\gamma_{2}\) equals the angle between
       \(\gamma_{1},\gamma_{2}\).
 
 %-----------------------------------------------------------------
 \item[\textbf{(4)}] \textbf{Local similarity.}
       Near \(z_{0}\):
       \[
         f(z) \;=\; f(z_{0}) + f'(z_{0})\,(z-z_{0})
                   + \mathcal O(|z-z_{0}|^{2}),
       \]
       so up to first order \(f\) is a rotation–dilation by
       \(f'(z_{0})\).
 
 %-----------------------------------------------------------------
 \item[\textbf{(5)}] \textbf{Möbius (linear‐fractional) maps.}
       \[
         T(z)=\frac{az+b}{cz+d},
         \quad ad-bc\neq0,
       \]
       send circles/lines to circles/lines and are conformal on
       \(\widehat{\Bbb C}\) minus the pole \(z=-d/c\).
 
 %-----------------------------------------------------------------
 \item[\textbf{(6)}] \textbf{Automorphisms of classic domains.}
       \[
         \varphi_{\alpha}(z)=e^{i\theta}\,
           \frac{z-\alpha}{1-\overline{\alpha}\,z},
         \quad |\alpha|<1,
       \qquad
         \psi_{a}(z)=\frac{az+b}{\overline{b}z+\overline{a}},
         \quad a\overline{a}-b\overline{b}>0,
       \]
       give all conformal self-maps of the unit disc
       \(\Bbb D\) and the upper half-plane \(\mathcal H\)
       respectively.
 
 %-----------------------------------------------------------------
 \item[\textbf{(7)}] \textbf{Riemann Mapping Theorem.}
       Any simply connected, proper open set
       \(\Omega\subsetneq\Bbb C\) is conformally equivalent to
       \(\Bbb D\).  (Uniqueness up to a disc automorphism.)
 
 %-----------------------------------------------------------------
 \item[\textbf{(8)}] \textbf{Maximum Modulus \& Open Mapping.}
       Non-constant analytic $\to $  modulus attains no interior maxima and
       image of an open set is open.
 
 %-----------------------------------------------------------------
 \item[\textbf{(9)}] \textbf{Schwarz Lemma (Disc $\to$ Disc).}
       \(f(0)=0\Rightarrow |f(z)|\le|z|\) and
       \(|f'(0)|\le 1\); equality $\to $  \(f\) is a rotation.
 
 %-----------------------------------------------------------------
 \item[\textbf{(10)}] \textbf{Schwarz Reflection Principle.}
       If \(f\) is analytic in a domain symmetric about a line/arc and
       real-valued on the symmetric subset, then it extends
       analytically by reflection.
 
 %-----------------------------------------------------------------
 \item[\textbf{(11)}] \textbf{Schwarz–Christoffel (polygon maps).}
       The integral
       \(
         F(z)=A + C\int^{z}\prod_{k}(t-z_{k})^{\alpha_{k}-1}dt
       \)
       sends \(\mathcal H\) (or \(\Bbb D\)) onto a polygon whose
       interior angles are \(\alpha_{k}\pi\).
 
 %-----------------------------------------------------------------
 \item[\textbf{(12)}] \textbf{Koebe $1/4$ Theorem.}
       If \(f\colon\Bbb D\to\Bbb C\) is univalent with \(f(0)=0\) and
       \(f'(0)=1\), then \(f(\Bbb D)\) contains the disc
       \(|w|<\tfrac14\).
 
 %-----------------------------------------------------------------
 \item[\textbf{(13)}] \textbf{Conformal invariants.}
       Cross-ratio
       \(
         (z_{1},z_{2};z_{3},z_{4})
         =
         \dfrac{(z_{1}-z_{3})(z_{2}-z_{4})}
               {(z_{1}-z_{4})(z_{2}-z_{3})}
       \)
       is preserved by Möbius maps.  
       Harmonic measure, Green’s function, and Laplace’s equation are
       invariant under conformal maps.
 
 %-----------------------------------------------------------------
 \item[\textbf{(14)}] \textbf{Normal Families (Montel).}
       A locally bounded family of analytic maps is
       a normal family; Vitali + Arzelà–Ascoli yield subsequence
       convergence.
 
 %-----------------------------------------------------------------
 \item[\textbf{(15)}] \textbf{Removal, Pole, Essential.}
       Isolated singularities of analytic functions fall into:
       removable, pole (order $m$), or essential; behaviour detected
       by Laurent series coefficients.
 
 %-----------------------------------------------------------------
 \item[\textbf{(16)}] \textbf{Principle of Conformal Mapping.}
       Analytic solutions of boundary–value PDEs (Laplace, Poisson)
       can be transplanted via conformal maps:
       \(u\circ f^{-1}\) is harmonic if \(u\) is harmonic.
 
 \end{itemize}
 
 \vspace{1em}
 \hrule
 \centerline{\small * Memorise these \(16\) facts and most exam‐style
 conformal‐mapping problems collapse to a line or two of work!}
 \pagebreak
 %==========================================================
 \begin{solution}
   Let
   \[
     D \;=\;
     \bigl\{\,z\in\mathbb{C} : \Im z>0,\; |z|>1,\;\Re z<1\bigr\},
     \qquad
     f(z)=\frac1z .
   \]
   
   %%%%%%%%%%%%%%%%%%%%%%%%%%%%%%%%%%%%%%%%%%%%%%%%%%%%%%%%%%%%
   \subsection*{1.  Decompose the boundary of \(D\)}
   Write \(D=D_{1}\cap D_{2}\cap D_{3}\), where
   \[
     \boxed{D_{1}}=\{\Im z>0\},\qquad
     \boxed{D_{2}}=\{|z|>1\},\qquad
     \boxed{D_{3}}=\{\Re z<1\}.
   \]
   The boundary is the union of three oriented curves
   \[
     \ell_{1}: \Im z=0,\quad
     \ell_{2}: |z|=1\ (\text{upper semicircle}),\quad
     \ell_{3}: \Re z=1,\; \Im z>0.
   \]
   
   %%%%%%%%%%%%%%%%%%%%%%%%%%%%%%%%%%%%%%%%%%%%%%%%%%%%%%%%%%%%
   \subsection*{2.  Map each piece under \(f(z)=1/z\)}
   \begin{itemize}
     \item \(\displaystyle f(\ell_{1})=\Im w=0.\)
           (Inversion takes the real axis to itself.)
     \item \(\displaystyle f(\ell_{2})=\{|w|=1\}.\)
           (The unit circle is invariant, with inside/outside interchanged.)
     \item \(\displaystyle f(\ell_{3})=\bigl\{\,w:|w-\tfrac12|=\tfrac12\bigr\}.\)
           Indeed, for \(z=1+iy\) we have
           \(
             w=\dfrac{1-iy}{1+y^{2}},\;
             |w-\tfrac12|=\tfrac12
           \).
   \end{itemize}
   
   %%%%%%%%%%%%%%%%%%%%%%%%%%%%%%%%%%%%%%%%%%%%%%%%%%%%%%%%%%%%
   \subsection*{3.  Map the three half/open sets}
   \[
     \begin{aligned}
       f(D_{1})&=\{\Im w<0\}
         &&\text{(upper $\to$ lower half--plane)},\\
       f(D_{2})&=\{|w|<1\}
         &&\text{(exterior $\to$ interior of unit circle)},\\
       f(D_{3})&=\{\,w:|w-\tfrac12|>\tfrac12\}
         &&\text{(left of \(x=1\) $\to$ \emph{outside} the circle)}.
     \end{aligned}
   \]
   
   %%%%%%%%%%%%%%%%%%%%%%%%%%%%%%%%%%%%%%%%%%%%%%%%%%%%%%%%%%%%
   \subsection*{4.  Intersection gives the image domain}
   \[
     f(D)
     =f(D_{1})\cap f(D_{2})\cap f(D_{3})
     \;=\;
     \boxed{\;
       \bigl\{\,w\in\mathbb{C} :
          \Im w<0,\;
          |w|<1,\;
          |\,w-\frac12\,|>\frac12
       \bigr\}
     \;}.
   \]
   
   \vspace{0.5em}
   Thus \(f\) carries \(D\) onto the shaded lens--shaped region inside the 
   unit disc, \emph{below} the real axis, but \emph{outside} the little circle 
   of radius \(\tfrac12\) centred at \(w=\tfrac12\).
   \end{solution}
   %==========================================================
%   Key Properties of the Möbius Transformation f(z)=1/z
%==========================================================
% Needs only amsmath/amsfonts/amssymb if not already loaded
%----------------------------------------------------------

\begin{itemize}
   %----------------------------------------------------------
   \item[\textbf{Analyticity}] 
         \(f(z)=1/z\) is meromorphic on the extended plane  
         \(\widehat{\Bbb C}=\Bbb C\cup\{\infty\}\)
         with a single simple pole at \(z=0\) and a removable
         singularity at \(\infty\) (where \(f(\infty)=0\)).
   
   %----------------------------------------------------------
   \item[\textbf{Derivative}] 
         \[
           f'(z)=-\frac{1}{z^{2}},\qquad z\neq0,
         \]
         so \(f'(z)\neq0\) everywhere on its domain of holomorphy.
         Hence \(f\) is conformal (angle‐preserving) at every finite
         point \(\,z\neq0\) and at \(\infty\).
   
   %----------------------------------------------------------
   \item[\textbf{Involution}] 
         \(f\) is its own inverse:
         \[
             f\bigl(f(z)\bigr)=\frac{1}{1/z}=z
             \quad
             (\forall z\in\widehat{\Bbb C}\setminus\{0,\infty\}).
         \]
         Thus \(f\) is a \emph{Möbius involution} of order~\(2\).
   
   %----------------------------------------------------------
   \item[\textbf{Effect on Modulus}] 
         \[
           |f(z)|=\frac{1}{|z|}.
         \]
         Hence the unit circle \(\{|z|=1\}\) is \emph{invariant},
         the exterior \(\{|z|>1\}\) is sent to the interior
         \(\{|w|<1\}\), and vice versa.
   
   %----------------------------------------------------------
   \item[\textbf{Effect on Argument}] 
         \[
           \arg f(z)= -\arg z \pmod{2\pi}.
         \]
         Rays through the origin are preserved
         but their orientation is reversed.
   
   %----------------------------------------------------------
   \item[\textbf{Upper/Lower Half–Planes}] 
         \[
           \Im z>0 \;\Longrightarrow\; \Im\!\bigl(1/z\bigr)<0,
         \qquad
           \Im z<0 \;\Longrightarrow\; \Im\!\bigl(1/z\bigr)>0.
         \]
         Thus \(f\) swaps the upper and lower half–planes.
   
   %----------------------------------------------------------
   \item[\textbf{Circles and Lines}] 
         As a Möbius map, \(f\) sends generalized circles
         (lines and circles) to generalized circles.  
         In particular:
         \begin{itemize}
           \item Lines through \(0\) map to themselves.
           \item A circle not passing through \(0\) maps to another circle.
           \item A circle through \(0\) maps to a line not passing through \(0\).
         \end{itemize}
   
   %----------------------------------------------------------
   \item[\textbf{Orientation}] 
         Because \(f\) is holomorphic (not antiholomorphic),
         it is orientation–preserving on the Riemann sphere.
   
   %----------------------------------------------------------
   \item[\textbf{Cross–Ratio Preservation}] 
         Being Möbius, \(f\) preserves the cross–ratio:
         \[
           (z_{1},z_{2};z_{3},z_{4})
             =(f(z_{1}),f(z_{2});f(z_{3}),f(z_{4})).
         \]
   
   %----------------------------------------------------------
   \item[\textbf{Critical Points}] 
         \(f'(z)\) never vanishes, so \(f\) has \emph{no} critical
         points; it is univalent (injective) on
         \(\widehat{\Bbb C}\setminus\{0,\infty\}\).
   
   \end{itemize}
   
   %==========================================================
   \begin{solution}
      Let \(f(z)=1/z\) and write \(z=x+iy\) with \(x=\Re z\), \(y=\Im z\).  
      We want to prove that the half–plane
      \[
           D_{3}=\{z\in\mathbb{C}:\;x<1\}
      \]
      is carried by \(f\) onto the exterior of the little circle
      \(\displaystyle C=\bigl\{\,w:|w-\tfrac12|=\tfrac12\bigr\}\).
      
      %-----------------------------------------------------------------
      \subsection*{1.  The image of the \(\boldsymbol{\,\Re z=1\,}\) boundary}
      
      Set \(x=1\).  Then
      \[
         z=1+iy
         \quad\Longrightarrow\quad
         w=f(z)=\frac{1-iy}{1+y^{2}}
              =\frac{1}{1+y^{2}}-i\frac{y}{1+y^{2}} .
      \]
      Compute its distance from the centre \(w_{0}=\tfrac12\):
      \[
        \bigl|\,w-w_{0}\bigr|^{2}
          =\Bigl(\frac{1}{1+y^{2}}-\frac12\Bigr)^{2}
           +\Bigl(\frac{y}{1+y^{2}}\Bigr)^{2}
          =\frac14 .
      \]
      Hence every point of the vertical line \(\Re z=1\) lands on the
      circle \(C\); conversely every point of \(C\) has a pre-image with
      \(x=1\).  Thus
      \[
         f\bigl(\Re z=1\bigr)=C.
      \]
      
      %-----------------------------------------------------------------
      \subsection*{2.  Inequality characterisation of the image domain}
      
      For general \(z=x+iy\) (\emph{with} \(z\neq0\)) put
      \(r=x^{2}+y^{2}>0\).  Then
      \[
           w=\frac{x-iy}{r},
           \qquad
           \bigl|\,w-\tfrac12\bigr|^{2}
             =\Bigl(\frac{x}{r}-\frac12\Bigr)^{2}
              +\Bigl(\frac{y}{r}\Bigr)^{2}.
      \]
      We need to decide when the inequality
      \(\bigl|w-\tfrac12\bigr|>\tfrac12\) holds.  Square both sides and
      multiply by \(4r^{2}\):
      \[
         4\Bigl(x-\frac{r}{2}\Bigr)^{2} + 4y^{2} > r^{2}.
      \]
      Expand and simplify:
      \[
         4\bigl(x^{2}+y^{2}\bigr) - 4xr > 0
         \;\;\Longleftrightarrow\;\;
         4r(1-x) > 0.
      \]
      Because \(r>0\), the strict inequality is equivalent to
      \(\boxed{x<1}\).
      
      \medskip
      \noindent
      \emph{Conclusion:}
      \[
            x<1
            \quad\Longleftrightarrow\quad
            \bigl|w-\tfrac12\bigr|>\tfrac12 .
      \]
      Therefore
      \[
         f(D_{3})
           =\{\,w:|w-\tfrac12|>\tfrac12\},
      \]
      exactly the shaded region \emph{outside} the circle of radius
      \(\tfrac12\) centred at \(w=\tfrac12\).
      
      %-----------------------------------------------------------------
      \subsection*{3.  Orientation check (optional)}
      
      Pick a test point to see which side of the boundary goes where:  
      \(z=i\) lies in \(D_{3}\) (because \(\Re i=0<1\)), and
      \(w=f(i)=-i\) satisfies
      \(\bigl|-i-\tfrac12\bigr|=\sqrt{\tfrac14+1}>\tfrac12\),
      hence indeed lands \emph{outside} \(C\).
      
      \end{solution}
      \pagebreak
      \begin{remark}
         Recall that \(f(z)=1/z\) behaves like an \emph{inversion} with respect
         to the unit circle together with a reflection across the real axis:
         
         \[
             z \;\longmapsto\; \underbrace{\frac{\overline{z}}{|z|^{2}}}_{\text{reflect + rescale}}
                              =\frac{1}{z}.
         \]
         
         \textbf{1.  What happens to a straight line that \emph{misses} the
         origin?}\\
         Any line or circle that does \emph{not} pass through the origin is
         sent to a \emph{circle}.  
         Why?  Because the cross–ratio is preserved by Möbius maps; lines are
         “circles through $\infty$,” and inversion swaps
         finite points with \(\infty\) unless the line already passes through
         the origin.
         
         Here the boundary line is \(\Re z = 1\).  
         It is one unit to the right of the origin, so it certainly misses~\(0\);
         hence its image must be a circle.  
         A quick calculation shows that every point on \(\Re z=1\) lands at
         distance \(1/2\) from the point \(w_{0}=1/2\), so the entire line
         wraps onto that little circle.
         
         \textbf{2.  Which \emph{side} of the line becomes the \emph{outside}
         of the circle?}\\
         Inversion flips “near” and “far”:
         
         \[
            |z|\;\text{small} \;\;\Longrightarrow\;\;
            |1/z|\;\text{large},
            \qquad
            |z|\;\text{large} \;\;\Longrightarrow\;\;
            |1/z|\;\text{small}.
         \]
         
         The region \(\Re z<1\) is the half–plane that \emph{contains} the
         origin (points there can be arbitrarily close to \(0\)).
         Those near–zero points are hurled far away by \(1/z\);
         they cannot fit inside the little circle, whose radius is only \(1/2\).
         Therefore the “origin side’’ of the line must map to the
         \emph{exterior} of the circle.
         
         \textbf{3.  Visual check with two sample points.}
         
         \begin{center}
         \begin{tabular}{c|c|c}
         \(z\) & in \(\Re z<1\)? & \(w=1/z\) \\
         \hline
         \( z = \tfrac12\)      & yes & \( w = 2     \quad (\text{outside})\) \\
         \( z = 2             \) & no  & \( w = \tfrac12\quad(\text{inside})\)
         \end{tabular}
         \end{center}
         
         The first point lands \emph{outside} the little circle, the second
         lands \emph{inside}.  
         That empirical test matches the geometric reasoning.
         
         \medskip
         \noindent
         \emph{Hence:} the half–plane \(\Re z<1\) (shaded to the left of the
         vertical line) must go to the region \emph{outside} the circle centred
         at \(1/2\) with radius \(1/2\).
         \end{remark}
\end{document}
