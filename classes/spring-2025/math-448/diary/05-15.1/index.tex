\documentclass[12pt]{article}

% Packages
\usepackage[margin=1in]{geometry}
\usepackage{amsmath,amssymb,amsthm}
\usepackage{enumitem}
\usepackage{hyperref}
\usepackage{xcolor}
\usepackage{import}
\usepackage{xifthen}
\usepackage{pdfpages}
\usepackage{transparent}
\usepackage{listings}
\usepackage{tikz}
\usepackage{physics}
\usepackage{siunitx}
\usepackage{booktabs}
\usepackage{cancel}
  \usetikzlibrary{calc,patterns,arrows.meta,decorations.markings}


\DeclareMathOperator{\Log}{Log}
\DeclareMathOperator{\Arg}{Arg}

\lstset{
    breaklines=true,         % Enable line wrapping
    breakatwhitespace=false, % Wrap lines even if there's no whitespace
    basicstyle=\ttfamily,    % Use monospaced font
    frame=single,            % Add a frame around the code
    columns=fullflexible,    % Better handling of variable-width fonts
}

\newcommand{\incfig}[1]{%
    \def\svgwidth{\columnwidth}
    \import{./Figures/}{#1.pdf_tex}
}
\theoremstyle{definition} % This style uses normal (non-italicized) text
\newtheorem{solution}{Solution}
\newtheorem{proposition}{Proposition}
\newtheorem{problem}{Problem}
\newtheorem{lemma}{Lemma}
\newtheorem{theorem}{Theorem}
\newtheorem{remark}{Remark}
\newtheorem{note}{Note}
\newtheorem{definition}{Definition}
\newtheorem{example}{Example}
\newtheorem{corollary}{Corollary}
\theoremstyle{plain} % Restore the default style for other theorem environments
%

% Theorem-like environments
% Title information
\title{}
\author{Jerich Lee}
\date{\today}

\begin{document}

\maketitle
%------------------------------------------------------------
%  Laurent (actually Maclaurin) expansion of  h(z)=z\cot z
%  about z = 0 – first four non-zero terms
%------------------------------------------------------------

\[
\cot z \;=\; \frac{\cos z}{\sin z}
           \;=\; \frac{1-\frac{z^{2}}{2!}+\frac{z^{4}}{4!}-\dots}
                     {z-\frac{z^{3}}{3!}+\frac{z^{5}}{5!}-\dots}
           \;=\; \frac{1}{z}
                 \Bigl(1-\tfrac{z^{2}}{3}-\tfrac{z^{4}}{45}
                       -\tfrac{2z^{6}}{945}-\dots\Bigr).
\]

Multiplying by \(z\) gives

\[
\boxed{\;
   z\cot z
     \;=\;
     1 \;-\;\frac{z^{2}}{3}
        \;-\;\frac{z^{4}}{45}
        \;-\;\frac{2z^{6}}{945}
        \;+\;\cdots
   \;}
\]

Hence the first four non-zero terms of the Laurent
(Maclaurin) series about \(z=0\) are
\[
1,\; -\dfrac{z^{2}}{3},\; -\dfrac{z^{4}}{45},
   \; -\dfrac{2z^{6}}{945}.
\]
\[
f(z)\;=\;\frac{z-\sin z}{(z-1)^{2}}
\]

\textbf{Which point are we talking about?}  
The only place where the \emph{numerator} vanishes is \(z=0\) (because
\(\sin 0=0\)).  Around \(z=1\) the denominator blows up, so
\(z=1\) is \emph{not} a zero but a pole of order \(2\).
Hence the “order of the zero’’ question concerns \(z=0\).

\bigskip
\textbf{Definition.}  
A holomorphic function \(f\) has a \emph{zero of order \(m\)} at
\(z=z_{0}\) if
\[
f(z)= (z-z_{0})^{m}\,g(z),
\qquad g \text{ analytic and } g(z_{0})\neq0 .
\]

\bigskip
\textbf{Step 1 – series of the numerator.}
Use the Maclaurin expansion
\[
\sin z = z-\frac{z^{3}}{3!}+\frac{z^{5}}{5!}-\cdots .
\]
Subtracting from \(z\) gives
\[
z-\sin z
   = z-\Bigl(z-\frac{z^{3}}{6}+\frac{z^{5}}{120}-\cdots\Bigr)
   = \frac{z^{3}}{6}-\frac{z^{5}}{120}+\cdots
   = z^{3}\Bigl(\frac16-\frac{z^{2}}{120}+\cdots\Bigr).
\]

\textbf{Step 2 – behaviour of the denominator at \(z=0\).}
\((z-1)^{2}=1-2z+z^{2}\), which is non-zero at \(z=0\).

\textbf{Step 3 – factor out the lowest power of \(z\).}
Near \(z=0\)
\[
f(z)=
\frac{z^{3}\bigl(\tfrac16-\tfrac{z^{2}}{120}+\cdots\bigr)}
     {1-2z+z^{2}}
     = z^{3}\,\underbrace{\frac{\tfrac16-\tfrac{z^{2}}{120}+\cdots}
                               {1-2z+z^{2}}}_{g(z)},
\]
and the bracketed quotient \(g(z)\) is analytic with
\(g(0)=\tfrac16\neq0\).

\bigskip
\[
\boxed{\;
\text{Therefore } f \text{ has a zero of order } 3 \text{ at } z=0.
\;}
\]

(Meanwhile, \(z=1\) is a pole of order \(2\) because the numerator
is non-zero there while the denominator behaves like \((z-1)^{2}\).)
\[
\textbf{Why isolate (“factor out’’) the \emph{smallest} power of }
      (z-z_{0})?
\]

\begin{enumerate}
\item \textbf{Definition recap.}  
      A holomorphic function \(f\) has a zero of \emph{order \(m\)} at
      \(z_{0}\) iff one (hence both) of the following equivalent
      conditions holds:
      \[
      \boxed{\;
        f(z) = (z-z_{0})^{m}\,g(z),
        \quad g(z_{0})\neq0
      \;}
      \quad\Longleftrightarrow\quad
      \boxed{\;
        \displaystyle
        \lim_{z\to z_{0}}\frac{f(z)}{(z-z_{0})^{m}} = g(z_{0})\neq0
      \;}
      \]
      That is, \emph{exactly} \(m\) factors of \((z-z_{0})\) vanish and
      the remainder \(g\) stays non-zero.

\item \textbf{Smallest power \(\Rightarrow\) largest non-vanishing term.}  
      In the Taylor/Laurent series
      \[
      f(z)=\sum_{n\ge0} a_{n}\,(z-z_{0})^{n},
      \]
      the first non-zero coefficient appears at some index \(m\):
      \(a_{0}=a_{1}=\dots=a_{m-1}=0,\;a_{m}\neq0\).
      Factoring out \((z-z_{0})^{m}\) extracts this earliest
      non-vanishing term, leaving a series whose constant term
      \(a_{m}\) is \(\neq0\).  Any higher power would make the
      remaining series start with \(a_{m}\,(z-z_{0})^{k}\) (\(k>0\)),
      which \emph{does} vanish at \(z_{0}\)—so \(g(z_{0})\) would be
      \(0\), contradicting the definition.

\item \textbf{Geometric meaning.}  
      A zero of order
      \(m\) is a point where the graph of \(f\) “touches’’ the \(0\)
      level with multiplicity \(m\); algebraically it is where
      the first \(m-1\) derivatives vanish:
      \[
        f(z_{0})=f'(z_{0})=\dots=f^{(m-1)}(z_{0})=0,
        \qquad f^{(m)}(z_{0})\neq0.
      \]
      Factoring out the lowest power makes these derivative conditions
      manifest.

\item \textbf{Analytic convenience.}  
      Once we isolate the smallest exponent, the quotient \(g\) is
      analytic and non-zero at \(z_{0}\).  Therefore
      \(\log g\) is analytic there, residues are easy to
      compute, limits are finite, etc.—all the standard tools of
      complex analysis apply cleanly.

\end{enumerate}

\[
\boxed{\;
\text{Hence we factor the \emph{lowest} power so that the leftover
      factor is \underline{non-vanishing}; this power is, by definition,
      the order (multiplicity) of the zero.}
\;}\]
\pagebreak
%-----------------------------------------------------------------
%  Determine the type of each singular point of
%      f(z)=\dfrac{e^{\pi/z}+1}{z^{2}(z-i)(z+2i)^{2}}
%  (principal branches assumed)
%-----------------------------------------------------------------

\begin{enumerate}[label=\textbf{\arabic*.},leftmargin=*]

  %-----------------------------------------------------------------
  \item \textbf{$z=0$ : essential singularity}
  
  \begin{enumerate}[label=(\alph*)]
  \item The factor $e^{\pi/z}$ is \emph{essential} at $z=0$ because its
        Laurent series
        \(
            e^{\pi/z}=\displaystyle\sum_{n=0}^{\infty}
                       \dfrac{\pi^{\,n}}{n!}\,z^{-n}
        \)
        contains \emph{infinitely many negative powers}.
  \item Multiplying by the rational factor
        $\displaystyle\dfrac{1}{z^{2}(z-i)(z+2i)^{2}}$ 
        does not remove any of those infinitely many terms; it only adds a
        finite number of additional negative powers.
  \end{enumerate}
  
  \[
  \boxed{\;
     z=0 \text{ is an \emph{essential} singularity of } f
   \;}
  \]
  
  %-----------------------------------------------------------------
  \item \textbf{$z=i$ : removable singularity}
  
  \begin{enumerate}[label=(\alph*)]
  \item Evaluate the numerator:
        \[
        e^{\pi/i}+1
        =e^{-\pi i}+1
        =(-1)+1
        =0 .
        \]
        Hence the numerator vanishes at $z=i$.
  \item To see \emph{how} it vanishes, expand
        $g(z):=e^{\pi/z}$ near $z=i$:
        \[
        g'(z)= -\frac{\pi\,e^{\pi/z}}{z^{2}}
        \quad\Longrightarrow\quad
        g'(i)= -\frac{\pi(-1)}{i^{2}}=-\pi\neq0.
        \]
        Thus
        \(
           g(z)+1 = (z-i)\bigl(-\pi+\mathcal O(z-i)\bigr),
        \)
        i.e.\ the zero is \emph{simple}.
  \item The denominator contains exactly the same simple factor
        $(z-i)$, so it cancels completely and leaves a finite,
        holomorphic limit.
  \end{enumerate}
  
  \[
  \boxed{\;
     z=i \text{ is a \emph{removable} singularity of } f
   \;}
  \]
  
  %-----------------------------------------------------------------
  \item \textbf{$z=-2i$ : pole of order $2$}
  
  \begin{enumerate}[label=(\alph*)]
  \item The denominator has $(z+2i)^{2}$, so at worst $z=-2i$ is a pole
        of order $2$.
  \item Check the numerator:
        \[
        e^{\pi/(-2i)}+1
        =e^{\,\frac{i\pi}{2}}+1
        =i+1\neq0,
        \]
        i.e.\ it \emph{does not} vanish at $z=-2i$.
  \end{enumerate}
  
  \[
  \boxed{\;
     z=-2i \text{ is a \emph{pole of order }2 \text{ for } f}
   \;}
  \]
  
  \end{enumerate}
  
  \bigskip
  \hrule
  \bigskip
  \noindent
  \textbf{Summary}
  
  \[
  \boxed{
  \begin{array}{c|c}
  \text{Point} & \text{Type of singularity}\\\hline
  z=0      & \text{essential}\\
  z=i      & \text{removable (defines }f(i)\text{)}\\
  z=-2i    & \text{pole of order }2
  \end{array}}
  \]
  \pagebreak
  %-----------------------------------------------------------------
%  Let $f$ be a polynomial of degree $n$ satisfying
%         $\displaystyle |f(z)|\le M \quad\text{for }|z|=1.$
%  Prove that
%         $\displaystyle |f(z)|\le M\,|z|^{n}\quad\text{for }|z|\ge 1.$
%
%  (Hint given: apply the Maximum Modulus Principle to
%               $w^{\,n}f(1/w)$ in the disk $|w|\le 1$.)
%-----------------------------------------------------------------

\begin{proof}
  \textbf{1.\; Set--up.}  Write the polynomial explicitly:
  \[
  f(z)=a_{n}z^{n}+a_{n-1}z^{\,n-1}+\dots+a_{1}z+a_{0},
  \qquad a_{n}\neq0,\;n\in\Bbb N.
  \]
  
  \bigskip
  \textbf{2.\; Define an auxiliary function on the punctured unit disk.}
  For $w\ne0$ put
  \[
  g(w):=w^{\,n}\,f\!\left(\frac1w\right).
  \tag{2.1}
  \]
  \textit{Observation.}  Since $f$ is polynomial, $f(1/w)$ is
  holomorphic on $0<|w|\le1$ and the prefactor $w^{\,n}$ cancels the pole
  of order $n$ at $w=0$.  Hence $g$ is holomorphic on the entire disk
  $|w|\le1$ (Riemann’s removable–singularity theorem).
  
  Indeed, expanding (2.1) gives
  \[
  g(w)
    =w^{\,n}\Bigl(a_{n}w^{-n}+a_{n-1}w^{-(n-1)}+\dots+a_{1}w^{-1}+a_{0}\Bigr)
    =a_{n}+a_{n-1}w+\dots+a_{0}w^{\,n},
  \]
  a polynomial in $w$.
  
  \bigskip
  \textbf{3.\; Boundary estimate for $g$.}
  If $|w|=1$ then $|1/w|=1$, so by the hypothesis
  $|f(1/w)|\le M$.  Therefore
  \[
  |g(w)|
    = |w^{\,n}|\,\bigl|f(1/w)\bigr|
    = 1^{n}\,|f(1/w)|
    \le M,
    \qquad |w|=1.
  \tag{3.1}
  \]
  
  \bigskip
  \textbf{4.\; Apply the Maximum Modulus Principle.}
  Because $g$ is holomorphic on the closed unit disk and satisfies
  $|g(w)|\le M$ on the boundary circle, the Maximum Modulus Principle
  implies
  \[
  |g(w)|\le M
  \quad\text{for every }|w|\le1.
  \tag{4.1}
  \]
  
  \bigskip
  \textbf{5.\; Translate the bound back to $f$.}
  Take an arbitrary $z$ with $|z|\ge1$ and set $w:=1/z$.  Then
  $0<|w|\le1$ and (4.1) gives
  \[
  |g(w)|=|w^{\,n}f(1/w)|
        =|w|^{\,n}\,|f(z)|
        =|z|^{-n}\,|f(z)|
        \;\le\; M.
  \]
  Multiply both sides by $|z|^{n}$:
  \[
  |f(z)|\;\le\; M\,|z|^{\,n},
  \qquad |z|\ge1.
  \tag{5.1}
  \]
  
  \bigskip
  \textbf{6.\; Conclusion.}  Inequality \emph{(5.1)} establishes the desired
  estimate for all $|z|\ge1$.
  
  \[
  \boxed{\;
          |f(z)|\;\le\; M\,|z|^{\,n},
          \quad\forall\,|z|\ge1
        \;}
  \]
  
  \end{proof}
  \pagebreak
  %------------------------------------------------------------
%  Liouville’s Theorem (Complex Analysis)
%------------------------------------------------------------

\begin{theorem}[Liouville]
  If a function \(f:\,\mathbb{C} \to \mathbb{C}\) is entire\footnote{%
  Holomorphic (analytic) on the whole complex plane.}
  and \emph{bounded}, i.e.\ there exists \(M>0\) such that
  \[
  |f(z)|\;\le\;M\qquad\text{for all }z\in\mathbb{C},
  \]
  then \(f\) is \emph{constant}.
  \end{theorem}
  
  \begin{proof}
  Fix an arbitrary \(R>0\).  
  For every \(z\) with \(|z|<\tfrac{R}{2}\) Cauchy’s estimates give
  \[
  |f'(z)|
     \;\le\;
     \frac{\max_{|w|=R}|f(w)|}{R-|z|}
     \;\le\;
     \frac{M}{R-|z|}
     \;\le\;
     \frac{M}{R/2}
     \;=\;\frac{2M}{R}.
  \]
  
  Letting \(R\to\infty\) forces the bound \(|f'(z)|\le\frac{2M}{R}\to 0\),
  hence \(f'(z)=0\) for \emph{all} \(z\in\mathbb{C}\).  
  Therefore \(f\) is constant.
  \end{proof}
  
  \bigskip
  \textbf{Consequences.}
  \begin{enumerate}
    \item \emph{Fundamental Theorem of Algebra.}  
          If \(p(z)\) is a non-constant complex polynomial, then \(p(z)\)
          must attain every complex value—including \(0\); hence \(p\)
          has at least one root in \(\mathbb{C}\).
    \item \emph{Entire functions with growth conditions.}  
          If an entire function satisfies
          \(\displaystyle |f(z)|\le C(1+|z|)^{k}\) for some \(k<\infty\),
          then successive applications of Cauchy’s estimates show that
          \(f\) must be a polynomial of degree at most \(k\).
  \end{enumerate}
  \pagebreak
  %-----------------------------------------------------------------
%  Exercise 2.4 #21                (Entire functions of polynomial growth)
%-----------------------------------------------------------------
\begin{problem}
  Let \(f\in\mathcal A(\mathbb{C})\) (entire) and assume there exist constants
  \(A>0\), \(R_{0}>0\), \(m>0\) such that
  \[
     |f(z)| \;\le\; A\,|z|^{m}
     \qquad\text{whenever } |z|\;\ge\;R_{0}.
  \]
  Prove that \(f\) must be a polynomial of degree \(\le m\).
  \end{problem}
  
  \begin{proof}[Step-by-step solution]
  \textbf{1.\; Expand \(f\) in a Maclaurin series.}  
  Because \(f\) is entire,
  \[
     f(z) \;=\; \sum_{n=0}^{\infty} a_{n}\,z^{n},
     \qquad z\in\mathbb{C},
  \]
  and the series converges everywhere.
  
  \bigskip
  \textbf{2.\; Cauchy’s coefficient estimate.}  
  For any radius \(R>0\) let
  \[
     M_{R} \;:=\;
     \max_{|z|=R} |f(z)|.
  \]
  Cauchy’s estimates give
  \[
     |a_{n}|
     \;\le\;
     \frac{M_{R}}{R^{n}},
     \qquad n=0,1,2,\dots
  \tag{2.1}
  \]
  
  \bigskip
  \textbf{3.\; Use the growth condition to bound \(M_{R}\).}  
  If \(R>R_{0}\) then every point on the circle \(|z|=R\) satisfies the
  hypothesis, hence
  \[
     M_{R} \;=\; \max_{|z|=R}|f(z)|
              \;\le\;
              A\,R^{m}.
  \tag{3.1}
  \]
  
  \bigskip
  \textbf{4.\; Combine (2.1) and (3.1).}  
  For \(R>R_{0}\) we have
  \[
     |a_{n}|
     \;\le\;
     \frac{A\,R^{m}}{R^{n}}
     \;=\;
     A\,R^{\,m-n}.
  \tag{4.1}
  \]
  
  \bigskip
  \textbf{5.\; Show that coefficients with \(n>m\) vanish.}  
  Fix any index \(n>m\).
  Letting \(R\to\infty\) in (4.1) forces
  \(A\,R^{\,m-n}\to0\) (because \(m-n<0\)), so
  \(|a_{n}|\le 0\).  Hence
  \[
     a_{n}=0
     \qquad\text{for every } n>m.
  \]
  
  \bigskip
  \textbf{6.\; Conclude the form of \(f\).}  
  Since all coefficients beyond degree \(m\) are zero,
  \[
     f(z)
     \;=\;
     \sum_{n=0}^{m} a_{n}\,z^{n},
  \]
  a polynomial of degree at most \(m\).
  \end{proof}
  %------------------------------------------------------------
%  Rouché’s Theorem
%------------------------------------------------------------
\begin{theorem}[Rouché]
  Let \(f\) and \(g\) be holomorphic on an open set containing a
  simple closed positively oriented contour \(\Gamma\)
  and its interior \(D\).
  Suppose that on \(\Gamma\) one has the strict inequality
  \[
    \boxed{\;|g(z)| < |f(z)|\quad\text{for every }z\in\Gamma\;.}
  \]
  Then \(f\) and \(f+g\) have the \emph{same number of zeros
  (counted with multiplicities)} in the interior \(D\).
  \end{theorem}
  
  \bigskip
  \noindent
  \textbf{Sketch of proof.}
  Because \(|g(z)|<|f(z)|\) on \(\Gamma\), the function
  \[
    h_t(z):=f(z)+t\,g(z),
    \qquad 0\le t\le1,
  \]
  never vanishes on \(\Gamma\); indeed
  \(|t\,g(z)|\le|g(z)|<|f(z)|\) so \(h_t(z)\neq0\) there.
  Hence for each \(t\) the argument principle applies:
  \[
    N(t)=\frac1{2\pi i}\oint_{\Gamma}\!
          \frac{h_t'(z)}{h_t(z)}\,dz
  \]
  
  is an integer counting the zeros of \(h_t\) in \(D\).
  The integrand depends holomorphically on \((t,z)\), so \(N(t)\)
  is continuous in \(t\) but integer-valued—therefore constant.
  Taking \(t=0\) and \(t=1\) we conclude \(N(0)=N(1)\):
  \(f\) and \(f+g\) possess the same number of interior zeros. \(\qed\)
  
  \bigskip
  \textbf{Common corollary (polynomial version).}
  If \(p(z)=a_n z^n+\dots+a_0\) is a polynomial and
  \[
    |a_n z^n| > \bigl|a_{n-1}z^{n-1}+\dots+a_0\bigr|
    \quad\text{on }\Gamma:\;|z|=R,
  \]
  then \(p\) has exactly \(n\) zeros inside \(|z|<R\).  
  Taking \(R\) large makes the inequality hold automatically, giving the
  classical fact that a degree-\(n\) polynomial has \(n\) zeros in \(\mathbb{C}\).
\end{document}
