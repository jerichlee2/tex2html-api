\documentclass[12pt]{article}

% Packages
\usepackage[margin=1in]{geometry}
\usepackage{amsmath,amssymb,amsthm}
\usepackage{enumitem}
\usepackage{hyperref}
\usepackage{xcolor}
\usepackage{import}
\usepackage{xifthen}
\usepackage{pdfpages}
\usepackage{transparent}
\usepackage{listings}
\newcommand{\Log}{\operatorname{Log}}


\lstset{
    breaklines=true,         % Enable line wrapping
    breakatwhitespace=false, % Wrap lines even if there's no whitespace
    basicstyle=\ttfamily,    % Use monospaced font
    frame=single,            % Add a frame around the code
    columns=fullflexible,    % Better handling of variable-width fonts
}

\newcommand{\incfig}[1]{%
    \def\svgwidth{\columnwidth}
    \import{./Figures/}{#1.pdf_tex}
}
\theoremstyle{definition} % This style uses normal (non-italicized) text
\newtheorem{solution}{Solution}
\newtheorem{proposition}{Proposition}
\newtheorem{problem}{Problem}
\newtheorem{lemma}{Lemma}
\newtheorem{theorem}{Theorem}
\newtheorem{remark}{Remark}
\newtheorem{note}{Note}
\theoremstyle{plain} % Restore the default style for other theorem environments
%

% Theorem-like environments
% Title information
\title{HW3}
\author{Jerich Lee}
\date{\today}

\begin{document}

\maketitle
\section*{Step-by-step derivations}

\subsection*{(a) $\displaystyle \frac{d}{dz} \sin z = \cos z$}

\begin{align}
\frac{d}{dz}\bigl(\sin z\bigr)
  \;=\;
  \lim_{h\to 0}\frac{\sin(z + h) - \sin(z)}{h}.
\end{align}
Using the sine addition formula,
\begin{align}
\sin(z + h) = \sin z \cos h + \cos z \sin h,
\end{align}
we rewrite:
\begin{align}
\sin(z + h) - \sin(z) 
  = \sin z \cos h + \cos z \sin h - \sin z 
  = \sin z(\cos h - 1) + \cos z\,\sin h.
\end{align}
Hence,
\begin{align}
\frac{\sin(z + h) - \sin(z)}{h}
  = \sin z \,\frac{\cos h - 1}{h} \;+\; \cos z \,\frac{\sin h}{h}.
\end{align}
Taking the limit as $h \to 0$ and knowing that 
$\displaystyle \lim_{h\to 0}\frac{\sin h}{h} = 1$ 
and 
$\displaystyle \lim_{h\to 0}\frac{\cos h - 1}{h} = 0,$ 
we get
\begin{align}
\frac{d}{dz}\bigl(\sin z\bigr) = \cos z.
\end{align}

\subsection*{(b) $\displaystyle \frac{d}{dz} \cos z = - \sin z$}

Using the same limit definition,
\begin{align}
\frac{d}{dz}\bigl(\cos z\bigr)
  = \lim_{h \to 0} \frac{\cos(z + h) - \cos z}{h}.
\end{align}
Using the addition formula 
$\cos(z + h) = \cos z \cos h - \sin z \sin h,$
\begin{align}
\cos(z + h) - \cos z 
  = \cos z(\cos h - 1) \;-\; \sin z \,\sin h.
\end{align}
Hence,
\begin{align}
\frac{\cos(z + h) - \cos z}{h}
  = \cos z \,\frac{\cos h - 1}{h}
    \;-\; \sin z \,\frac{\sin h}{h}.
\end{align}
Taking $h \to 0$ gives $\cos h - 1 \to 0$ at order $h^2,$ so the first term goes to $0$, while $\sin h / h \to 1$.  Therefore,
\begin{align}
\frac{d}{dz}\bigl(\cos z\bigr) 
  = -\,\sin z.
\end{align}

\subsection*{(c) $\displaystyle \frac{d}{dz} \sinh z = \cosh z$}

Recall 
\begin{align}
\sinh z \;=\; \frac{e^z - e^{-z}}{2}.
\end{align}
Thus,
\begin{align}
\frac{d}{dz}\bigl(\sinh z\bigr)
  = \frac{d}{dz}\Bigl(\tfrac{e^z - e^{-z}}{2}\Bigr)
  = \frac{1}{2}\bigl(e^z + e^{-z}\bigr)
  = \cosh z.
\end{align}

\subsection*{(d) $\displaystyle \frac{d}{dz} \cosh z = \sinh z$}

Similarly, 
\begin{align}
\cosh z \;=\; \frac{e^z + e^{-z}}{2},
\end{align}
so
\begin{align}
\frac{d}{dz}\bigl(\cosh z\bigr)
  = \frac{d}{dz}\Bigl(\tfrac{e^z + e^{-z}}{2}\Bigr)
  = \frac{1}{2}\bigl(e^z - e^{-z}\bigr)
  = \sinh z.
\end{align}

\subsection*{(e) $\displaystyle \frac{d}{dz} \tan z = \sec^2 z$}

Using the quotient rule with 
$$\tan z = \frac{\sin z}{\cos z}$$:
\begin{align}
\frac{d}{dz}\bigl(\tan z\bigr)
  = \frac{(\cos z)\,\frac{d}{dz}(\sin z) - (\sin z)\,\frac{d}{dz}(\cos z)}{\cos^2 z}
  = \frac{\cos z \cos z - \sin z(-\,\sin z)}{\cos^2 z}
  = \frac{\cos^2 z + \sin^2 z}{\cos^2 z}
  = \frac{1}{\cos^2 z}
  = \sec^2 z.
\end{align}

\subsection*{(f) $\displaystyle \frac{d}{dz}\bigl(\arcsin z\bigr) 
             = (1 - z^2)^{-\frac{1}{2}}$}

Let $$y = \arcsin z$$.  Then $$\sin y = z$$.  
Differentiate both sides with respect to $$z$$:
\begin{align}
\cos y \,\frac{dy}{dz} = 1 \quad\Longrightarrow\quad
\frac{dy}{dz} = \frac{1}{\cos y}.
\end{align}
But from $$\sin^2 y + \cos^2 y = 1,$$ and $$\sin y = z,$$ we have $$\cos y = \sqrt{1 - z^2}.$$
Hence,
\begin{align}
\frac{d}{dz}\bigl(\arcsin z\bigr) 
  = \frac{1}{\sqrt{1 - z^2}} 
  = (1 - z^2)^{-\tfrac{1}{2}}.
\end{align}

\subsection*{(g) $\displaystyle \frac{d}{dz}\bigl(\arctan z\bigr) 
             = (1 + z^2)^{-1}$}

Let $$y = \arctan z$$.  Then $$\tan y = z$$.  
Differentiate implicitly:
\begin{align}
\sec^2 y \,\frac{dy}{dz} = 1
\quad \Longrightarrow\quad
\frac{dy}{dz} = \frac{1}{\sec^2 y} = \cos^2 y.
\end{align}
But 
$$\cos^2 y = \frac{1}{\sec^2 y} = \frac{1}{1 + \tan^2 y} = \frac{1}{1 + z^2}.$$
Hence,
\begin{align}
\frac{d}{dz}\bigl(\arctan z\bigr)
  = \frac{1}{1 + z^2} 
  = (1 + z^2)^{-1}.
\end{align}

\section*{Solutions to Exercises 5, 6, and 7}

\subsection*{5.\quad $\displaystyle \frac{d}{dz}\bigl[(z^3 + 100)^{-4}\bigr]$}

\begin{align}
f(z) \;=\; (z^3 + 100)^{-4}.
\end{align}
Use the chain rule.  Let $u(z) = z^3 + 100$ and $g(u) = u^{-4}.$  
Then
\begin{align}
\frac{d}{dz}\bigl(f(z)\bigr)
  \;=\;
  \frac{d}{dz}\bigl(g(u(z))\bigr)
  \;=\;
  g'(u)\,u'(z).
\end{align}
First compute
\begin{align}
g'(u) 
  = -4\,u^{-5}, 
\quad
u'(z) 
  = 3z^2.
\end{align}
Hence,
\begin{align}
\frac{d}{dz}\bigl[(z^3 + 100)^{-4}\bigr]
  \;=\;
  -4\,(z^3 + 100)^{-5}\;\cdot\;3z^2
  \;=\;
  -12\,z^2\,(z^3 + 100)^{-5}.
\end{align}

\subsection*{6.\quad $\displaystyle \frac{d}{dz}\bigl[\bigl(\Log z\bigr)^3\bigr]$}

We assume $\Log z$ denotes the principal branch of the complex logarithm (defined on $\mathbb{C}\setminus(-\infty,0]$).  
Let
\begin{align}
f(z) = (\Log z)^3.
\end{align}
By the chain rule, let $u(z) = \Log z$ and $g(u) = u^3$.  
Then
\begin{align}
g'(u) = 3\,u^2,
\quad
u'(z) = \frac{d}{dz}\bigl(\Log z\bigr) = \frac{1}{z}.
\end{align}
Thus
\begin{align}
\frac{d}{dz}\bigl[\bigl(\Log z\bigr)^3\bigr]
  \;=\;
  3\,(\Log z)^2 \;\cdot\;\frac{1}{z}
  \;=\;
  \frac{3\,(\Log z)^2}{z}.
\end{align}

\subsection*{7.\quad $\displaystyle \frac{d}{dz}\bigl[\sinh(e^z)\bigr]$}

Let $f(z) = \sinh(e^z)$.  
Again apply the chain rule:  
If $u(z) = e^z$ and $g(u) = \sinh(u)$, then 
\begin{align}
g'(u) = \cosh(u),
\quad
u'(z) = \frac{d}{dz}\bigl(e^z\bigr) = e^z.
\end{align}
Hence
\begin{align}
\frac{d}{dz}\sinh\bigl(e^z\bigr)
  = 
  \cosh\bigl(e^z\bigr)\,\cdot\,e^z.
\end{align}

\textbf{Solution for Exercise 9:}\\

We have
\begin{align}
f(z) \;=\; \frac{z^4 + 1}{z^2}.
\end{align}
Rewrite this as 
\begin{align}
f(z) \;=\; z^2 + \frac{1}{z^2}.
\end{align}
To find an analytic function $$F$$ with $$F'(z) = f(z)$$, we integrate term by term:
\begin{align}
F(z)
\;=\; \int \left( z^2 + \frac{1}{z^2} \right) \, dz
\;=\; \int z^2 \, dz \;+\; \int \frac{1}{z^2} \, dz.
\end{align}
Compute each integral:
\begin{align}
\int z^2 \, dz \;=\; \frac{z^3}{3},
\qquad
\int \frac{1}{z^2} \, dz \;=\; \int z^{-2} \, dz \;=\; -\,\frac{1}{z}.
\end{align}
Hence,
\begin{align}
F(z)
\;=\;
\frac{z^3}{3}
-\frac{1}{z}
+ C,
\end{align}
where $$C$$ is an arbitrary constant. \\

\textbf{Solution for Exercise 10:}\\

We want to find an analytic function $$F$$ such that
\begin{align}
F'(z) \;=\; f(z) \;=\; \sin z \, \cos z.
\end{align}
Observe the trigonometric identity:
\begin{align}
\sin z \, \cos z \;=\; \tfrac12\,\sin(2z).
\end{align}
Hence,
\begin{align}
f(z) \;=\; \tfrac12 \sin(2z).
\end{align}
We integrate:
\begin{align}
F(z) 
\;=\;
\int \tfrac12 \sin(2z) \, dz.
\end{align}
Recall that 
$$\displaystyle \frac{d}{dz}\bigl[-\tfrac12 \cos(2z)\bigr] \;=\; \sin(2z)$$.
Thus, multiplying by $$\tfrac12$$,
\begin{align}
F(z)
\;=\;
\tfrac12 \left(-\tfrac12 \cos(2z)\right)
\;+\; C
\;=\;
-\tfrac14 \cos(2z)
\;+\; C.
\end{align}
Therefore, the required antiderivative is
\begin{align}
F(z) \;=\; -\tfrac14 \cos(2z) \;+\; C.
\end{align}

\section*{Problem 12}

\textbf{Statement.} 
Let $f$ and $g$ be analytic on a domain $D$. Show that:
\begin{enumerate}
\item[(a)] $f+g$ is analytic on $D$, and $(f+g)' = f' + g'$.
\item[(b)] $fg$ is analytic on $D$, and $(fg)' = f'g + fg'$.
\item[(c)] $f/g$ is analytic on $D$ wherever $g(z)\neq 0$, and 
\begin{align}
\left(\frac{f}{g}\right)' \;=\; \frac{f'g - fg'}{g^2}.
\end{align}
\end{enumerate}

\subsection*{(a) Sum Rule}

\textbf{Proof.} 
Since $f$ and $g$ are analytic on $D$, they are differentiable at each point $z \in D$. Consider
\begin{align}
(f + g)(z + h) \;-\; (f + g)(z)
\;=\;
\bigl(f(z + h) - f(z)\bigr)
\;+\;
\bigl(g(z + h) - g(z)\bigr).
\end{align}
Divide by $h$ and take the limit as $h \to 0$:
\begin{align}
\lim_{h \to 0} \frac{(f + g)(z + h) - (f + g)(z)}{h}
\;=\;
\lim_{h \to 0} \frac{f(z + h) - f(z)}{h}
\;+\;
\lim_{h \to 0} \frac{g(z + h) - g(z)}{h}.
\end{align}
By the definition of the derivative (and the fact that $f$ and $g$ are differentiable),
\begin{align}
(f + g)'(z) = f'(z) + g'(z).
\end{align}
Thus $f+g$ is analytic (being the sum of analytic functions) and its derivative is as given. 
\quad $\square$

\subsection*{(b) Product Rule}

\textbf{Proof.}
Consider
\begin{align}
(fg)(z + h) - (fg)(z) 
\;=\;
f(z + h)\,g(z + h) - f(z)\,g(z).
\end{align}
Add and subtract $f(z+h)\,g(z)$ to rearrange:
\begin{align}
f(z + h)\,g(z + h)
- f(z)\,g(z)
\;=\;
f(z + h)\,\bigl[g(z + h) - g(z)\bigr]
\;+\;
g(z)\,\bigl[f(z + h) - f(z)\bigr].
\end{align}
Divide by $h$ and take the limit as $h \to 0$. Since $f$ and $g$ are continuous and differentiable,
\begin{align}
\lim_{h \to 0} \frac{f(z + h)\,\bigl[g(z + h) - g(z)\bigr]}{h}
=
f(z)\,\lim_{h \to 0}\frac{g(z + h) - g(z)}{h}
=
f(z)\,g'(z),
\end{align}
and similarly,
\begin{align}
\lim_{h \to 0} \frac{g(z)\,\bigl[f(z + h) - f(z)\bigr]}{h}
=
g(z)\,\lim_{h \to 0}\frac{f(z + h) - f(z)}{h}
=
g(z)\,f'(z).
\end{align}
Hence,
\begin{align}
(fg)'(z) = f'(z)\,g(z) \;+\; f(z)\,g'(z).
\end{align}
This proves that $fg$ is analytic and satisfies the product rule.
\quad $\square$

\subsection*{(c) Quotient Rule}

\textbf{Proof.}
Assume $g(z) \neq 0$ on $D$ so that $f/g$ is well-defined there. Then
\begin{align}
\left(\frac{f}{g}\right)(z + h) - \left(\frac{f}{g}\right)(z)
\;=\;
\frac{f(z + h)}{g(z + h)} - \frac{f(z)}{g(z)}
\;=\;
\frac{f(z + h)\,g(z) - f(z)\,g(z + h)}{g(z + h)\,g(z)}.
\end{align}
Rewrite the numerator:
\begin{align}
f(z + h)\,g(z) - f(z)\,g(z + h)
\;=\;
\bigl[f(z + h) - f(z)\bigr]\,g(z)
\;-\;
f(z)\,\bigl[g(z + h) - g(z)\bigr].
\end{align}
Divide by $h$ and take the limit as $h \to 0$. Again using continuity and differentiability of $f$ and $g$,
\begin{align}
\lim_{h \to 0} 
\frac{\bigl[f(z + h) - f(z)\bigr]\,g(z)}{h}
=
f'(z)\,g(z),
\quad
\lim_{h \to 0}
\frac{f(z)\,\bigl[g(z + h) - g(z)\bigr]}{h}
=
f(z)\,g'(z).
\end{align}
Meanwhile,
\begin{align}
\lim_{h \to 0} \frac{1}{g(z + h)\,g(z)} 
\;=\;
\frac{1}{g(z)^2},
\end{align}
because $g(z)\neq 0$ and $g$ is continuous. Putting it all together:
\begin{align}
\left(\frac{f}{g}\right)'(z)
=
\frac{f'(z)\,g(z) - f(z)\,g'(z)}{g(z)^2}.
\end{align}
This proves that $f/g$ is analytic wherever $g \neq 0$, with the stated derivative.
\quad $\square$

\section*{Problem 13}

\textbf{Statement.} 
Suppose $f$ is analytic on a domain $D$ and $g$ is analytic on a domain $\Omega$ containing the range of $f$. Show that $g\bigl(f(z)\bigr)$ is analytic on $D$ and that 
\begin{align}
\bigl(g(f(z))\bigr)' = g'\bigl(f(z)\bigr)\,f'(z).
\end{align}

\subsection*{Chain Rule for Complex Functions}

\textbf{Proof.}
Fix $z \in D$. By assumption, $f(z) \in \Omega$ where $g$ is analytic. We use the definition of the derivative:
\begin{align}
\bigl(g(f(z))\bigr)'
\;=\;
\lim_{h \to 0}
\frac{g\bigl(f(z + h)\bigr) - g\bigl(f(z)\bigr)}{h}.
\end{align}
Since $f$ is continuous and differentiable at $z$, for $h$ sufficiently small, $f(z + h)$ stays in $\Omega$. We can write
\begin{align}
\frac{g\bigl(f(z + h)\bigr) - g\bigl(f(z)\bigr)}{h}
=
\frac{g\bigl(f(z + h)\bigr) - g\bigl(f(z)\bigr)}{f(z + h) - f(z)}
\,\cdot\,
\frac{f(z + h) - f(z)}{h}.
\end{align}
Taking the limit as $h \to 0$, we recognize the first fraction as the derivative $g'\bigl(f(z)\bigr)$ (by the definition of $g'$ at $f(z)$), and the second fraction as $f'(z)$. Thus
\begin{align}
\lim_{h \to 0} 
\left[
\frac{g\bigl(f(z + h)\bigr) - g\bigl(f(z)\bigr)}{f(z + h) - f(z)}
\,\cdot\,
\frac{f(z + h) - f(z)}{h}
\right]
=
g'\bigl(f(z)\bigr)\,f'(z).
\end{align}
Hence 
\begin{align}
\bigl(g(f(z))\bigr)' \;=\; g'\bigl(f(z)\bigr)\,f'(z).
\end{align}
This completes the proof that $g \circ f$ is analytic on $D$ and satisfies the chain rule. 
\quad $\square$

\section*{Problem 15}
\textbf{Statement:} Let $f$ be analytic on a domain $D$ and suppose $f'(z)=0$ for all $z\in D$. Show that $f$ is constant on $D$.

\subsection*{Solution}
Since $f$ is analytic on $D$, it has a power series expansion about each point in $D$.  In particular, fix any $z_0 \in D$.  Then in some neighborhood of $z_0$, $f$ can be expressed as
\begin{align}
f(z) \;=\; a_0 + a_1(z - z_0) + a_2(z - z_0)^2 + \cdots,
\end{align}
where $a_n \in \mathbb{C}$.  The derivative $f'(z)$ is then
\begin{align}
f'(z)
\;=\;
a_1 + 2\,a_2\, (z-z_0) + 3\,a_3\, (z - z_0)^2 + \cdots.
\end{align}
By hypothesis, $f'(z) = 0$ for all $z \in D$.  This implies each coefficient of $(z - z_0)^n$ in $f'(z)$ must be zero. Hence
\begin{align}
a_1 = 0, \quad a_2 = 0, \quad a_3=0,\quad \dots
\end{align}
and so on.  Therefore,
\begin{align}
f(z) \;=\; a_0,
\end{align}
a constant in any neighborhood of $z_0$.  Since $D$ is a domain (connected open set), a function that is locally constant is actually constant throughout $D$.  Therefore $f$ is constant on $D$.

\quad $\square$

\section*{Problem 18}
\textbf{Statement:} Show that $h(z) = \overline{z}$ is not analytic on any domain. (Hint: Check the Cauchy--Riemann equations.)

\subsection*{Solution}
Write $z = x + i\,y$, where $x,y \in \mathbb{R}$.  Then $\overline{z} = x - i\,y$.  In terms of real and imaginary parts,
\begin{align}
u(x,y) = x, 
\qquad
v(x,y) = -\,y.
\end{align}
If $h(z)$ were analytic, it would satisfy the Cauchy--Riemann equations:
\begin{align}
\frac{\partial u}{\partial x} 
\;=\;
\frac{\partial v}{\partial y},
\quad\text{and}\quad
\frac{\partial u}{\partial y}
\;=\;
-\,
\frac{\partial v}{\partial x}.
\end{align}
Compute these partial derivatives:
\begin{align}
\frac{\partial u}{\partial x} = 1,
\quad
\frac{\partial v}{\partial y} = -1,
\quad
\frac{\partial u}{\partial y} = 0,
\quad
\frac{\partial v}{\partial x} = 0.
\end{align}
The first CR equation would require $1 = -1$, which is false.  Therefore the Cauchy--Riemann equations fail to hold except possibly at isolated points (e.g.\ $x$ and $y$ going to special values does not fix this systematically).

Since the Cauchy--Riemann equations fail on any open set in $\mathbb{C}$, $h(z)=\overline{z}$ cannot be analytic on any domain. 

\quad $\square$

\section*{Radius of Convergence Examples}

\subsection*{1) The series \(\displaystyle \sum_{k=1}^{\infty} k\,(z-1)^k\)}

Here, the general term is \(a_k = k\) in front of \((z-1)^k\).  We can use the root test or the standard formula for the radius of convergence of a power series
\begin{align}
\sum_{k=0}^{\infty} a_k (z-z_0)^k,
\end{align}
which says 
\begin{align}
R 
\;=\; 
\frac{1}{\displaystyle \limsup_{k \to \infty} \sqrt[k]{|a_k|}}.
\end{align}
In our case, \(a_k = k\). Hence
\begin{align}
\lim_{k \to \infty} \sqrt[k]{k} = 1.
\end{align}
Therefore 
\begin{align}
R = \frac{1}{1} = 1.
\end{align}
Thus, the radius of convergence of \(\sum_{k=1}^\infty k\,(z-1)^k\) is \(1\).

\subsection*{3) The series \(\displaystyle \sum_{j=0}^{\infty}\frac{z^{3j}}{2^j}\)}

We can rewrite each term as
\begin{align}
\frac{z^{3j}}{2^j}
\;=\;
\left(\frac{z^3}{2}\right)^j.
\end{align}
This is a geometric series in the variable \(\tfrac{z^3}{2}\).  A geometric series 
\(\sum_{j=0}^\infty r^j\) converges if and only if \(|r| < 1\).  Hence in our case:
\begin{align}
\left|\frac{z^3}{2}\right| < 1
\quad\Longleftrightarrow\quad
|z^3| < 2
\quad\Longleftrightarrow\quad
|z|^3 < 2
\quad\Longleftrightarrow\quad
|z| < \sqrt[3]{2}.
\end{align}
Therefore, the radius of convergence is 
\begin{align}
R = \sqrt[3]{2}.
\end{align}

\section*{Power Series about the Origin}

Below are the Maclaurin (power) series for each given function, valid in the specified regions.

\medskip
\noindent
\textbf{7) $e^{-z}$:}
\begin{align}
e^{-z}
\;=\;
\sum_{n=0}^\infty \frac{(-z)^n}{n!}
\;=\;
\sum_{n=0}^\infty (-1)^n \,\frac{z^n}{n!}.
\end{align}

\medskip
\noindent
\textbf{8) $z^2 \cos(z)$:}
Recall 
\(\displaystyle
\cos(z)
=\sum_{n=0}^\infty (-1)^n \frac{z^{2n}}{(2n)!}.
\)
Hence,
\begin{align}
z^2 \cos(z)
\;=\;
z^2
\sum_{n=0}^\infty 
(-1)^n 
\frac{z^{2n}}{(2n)!}
\;=\;
\sum_{n=0}^\infty
(-1)^n
\frac{z^{2n+2}}{(2n)!}.
\end{align}

\medskip
\noindent
\textbf{9) $\displaystyle \frac{z^3}{1 - z^3}$, \quad $|z|<1$:}
Write
\(\tfrac{1}{1 - z^3}=\sum_{n=0}^\infty (z^3)^n\)
for \(|z|<1\).  Then
\begin{align}
\frac{z^3}{1 - z^3}
\;=\;
z^3 \sum_{n=0}^\infty (z^3)^n 
\;=\;
\sum_{n=1}^\infty z^{3n}.
\end{align}

\medskip
\noindent
\textbf{10) $\displaystyle \frac{1 + z}{1 - z}$, \quad $|z|<1$:}
We can split the fraction as 
\begin{align}
\frac{1+z}{1-z}
\;=\;
\frac{1}{1-z} \;+\; \frac{z}{1-z}
\;=\;
\sum_{n=0}^\infty z^n
\;+\;
z \sum_{n=0}^\infty z^n.
\end{align}
Hence,
\begin{align}
\frac{1 + z}{1 - z}
\;=\;
1
\;+\;
2\,z
\;+\;
2\,z^2
\;+\;
2\,z^3
\;+\;\cdots
\;=\;
1 \;+\; 2\sum_{n=1}^\infty z^n.
\end{align}

\medskip
\noindent
\textbf{11) $\displaystyle \frac{z^2}{(4-z)^2}$, \quad $|z|<4$:}
Factor out $4$ in the denominator:
\begin{align}
\frac{1}{(4-z)^2}
\;=\;
\frac{1}{16}\,\frac{1}{\bigl(1 - \tfrac{z}{4}\bigr)^2}.
\end{align}
Recall the identity
\(\displaystyle
\frac{1}{(1-w)^2}
\;=\;
\sum_{n=0}^\infty (n+1)\,w^n,
\)
valid for \(|w| < 1\).  Taking $w = \frac{z}{4}$ gives
\begin{align}
\frac{1}{(4 - z)^2}
\;=\;
\frac{1}{16}
\sum_{n=0}^\infty (n+1)\,\bigl(\tfrac{z}{4}\bigr)^n
\;=\;
\sum_{n=0}^\infty 
(n+1)\,\frac{z^n}{4^{n+2}}.
\end{align}
Multiplying by $z^2$ yields
\begin{align}
\frac{z^2}{(4 - z)^2}
\;=\;
z^2
\sum_{n=0}^\infty 
(n+1)\,\frac{z^n}{4^{n+2}}
\;=\;
\sum_{n=0}^\infty 
(n+1)\,\frac{z^{n+2}}{4^{n+2}}.
\end{align}
Thus the series converges for $\bigl|\tfrac{z}{4}\bigr|<1$, i.e.\ $|z|<4$.

\section*{Closed--Form Expressions}

\noindent
\textbf{14)} 
\begin{align}
\sum_{n=0}^{\infty} \frac{z^{2n}}{n!}.
\end{align}
Observe that 
\begin{align}
\sum_{n=0}^\infty \frac{(z^2)^n}{n!} 
\;=\;
e^{z^2}.
\end{align}
Hence the series sums to 
\begin{align}
e^{z^2}.
\end{align}

\medskip
\noindent
\textbf{15)} 
\begin{align}
\sum_{n=1}^{\infty} z^{3n}.
\end{align}
This is a geometric series in $z^3$ (starting at $n=1$):
\begin{align}
z^3 + z^6 + z^9 + \cdots 
\;=\;
\sum_{n=1}^\infty (z^3)^n 
\;=\;
\frac{z^3}{1 - z^3}
\quad
\bigl(\text{for }|z^3|<1, \text{i.e.\ }|z|<1\bigr).
\end{align}

\medskip
\noindent
\textbf{16)} 
\begin{align}
\sum_{n=1}^{\infty} n\,\bigl(z - 1\bigr)^{\,n-1}.
\end{align}
Recognize this as the derivative of the geometric series.  In general,
\begin{align}
\sum_{n=1}^\infty n\,x^{n-1}
\;=\;
\frac{d}{dx}\!\Bigl(\sum_{n=1}^\infty x^n\Bigr)
=
\frac{d}{dx}\!\biggl(\frac{x}{1 - x}\biggr)
=
\frac{1}{(1 - x)^2}.
\end{align}
Set $x = z-1$.  Then, for $|z-1|<1$,
\begin{align}
\sum_{n=1}^{\infty} n\,(z - 1)^{n-1}
\;=\;
\frac{1}{\bigl(1 - (z-1)\bigr)^2}
\;=\;
\frac{1}{(2 - z)^2}.
\end{align}

\section*{Problem Statement}
\[
\text{Let } f(z) \;=\; 2\,x\,y^{3} \;+\; i\bigl(3\,y^{2} \;-\; 3\,x^{2}\,y^{2}\bigr),
\quad \text{where } z = x + i\,y.
\]
\textit{Find all points $z$ at which $f'(z)$ exists, and compute $f'(z)$ at those points.}

\bigskip

\noindent
\textbf{1.\ Rewrite $f(z)$ in terms of its real and imaginary parts.}

Since $z = x + i\,y$, we identify
\[
u(x,y) \;=\; 2\,x\,y^{3}, 
\qquad
v(x,y) \;=\; 3\,y^{2} \;-\; 3\,x^{2}\,y^{2}.
\]
Hence
\[
f(z) \;=\; u(x,y) \;+\; i\,v(x,y).
\]

\bigskip

\noindent
\textbf{2.\ Compute the partial derivatives.}

\[
u_x 
\;=\; \frac{\partial}{\partial x}\bigl(2\,x\,y^{3}\bigr)
\;=\; 2\,y^{3},
\qquad
u_y 
\;=\; \frac{\partial}{\partial y}\bigl(2\,x\,y^{3}\bigr)
\;=\; 6\,x\,y^{2}.
\]
\[
v_x 
\;=\; \frac{\partial}{\partial x}\bigl(3\,y^{2} - 3\,x^{2}\,y^{2}\bigr)
\;=\; -\,6\,x\,y^{2},
\qquad
v_y 
\;=\; \frac{\partial}{\partial y}\bigl(3\,y^{2} - 3\,x^{2}\,y^{2}\bigr)
\;=\; 6\,y\,\bigl(1 - x^{2}\bigr).
\]

\bigskip

\noindent
\textbf{3.\ Apply the Cauchy--Riemann equations.}

For $f$ to be complex differentiable at $(x,y)$, the Cauchy--Riemann (CR) system must hold:
\[
u_x \;=\; v_y,
\quad
u_y \;=\; -\,v_x.
\]

\smallskip
\noindent
\textbf{(a) First CR equation: } $u_x = v_y.$

\[
2\,y^{3}
\;=\;
6\,y\,\bigl(1 - x^{2}\bigr).
\]
If $y=0$, then both sides vanish and the condition is satisfied for any $x$.  If $y \neq 0$, we divide by $2\,y$:
\[
y^{2}
\;=\;
3\,\bigl(1 - x^{2}\bigr).
\]
Equivalently,
\[
y^{2} + 3\,x^{2} 
\;=\;
3.
\]
This describes an ellipse in the $(x,y)$-plane (with $|x|\le 1$).

\smallskip
\noindent
\textbf{(b) Second CR equation: } $u_y = -\,v_x.$

\[
6\,x\,y^{2}
\;=\;
-\bigl(-6\,x\,y^{2}\bigr)
\;=\;
6\,x\,y^{2}.
\]
This is automatically true for all $x,y$.  

\smallskip
\noindent
\textbf{Summary of CR solutions:}

The points where $f'(z)$ exists are:
\[
\bigl\{ (x,y) \,:\, y=0 \bigr\}
\quad\text{or}\quad
\bigl\{ (x,y) \,:\, y^{2} = 3\,(1 - x^{2}) \bigr\}.
\]
In terms of $z = x+ i\,y$, these are \textit{all real $z$} ($y=0$) and \textit{all $z$ on the ellipse} $y^{2} + 3\,x^{2} = 3$.

\bigskip

\noindent
\textbf{4.\ Compute $f'(z)$ at those points.}

A standard formula for the complex derivative, when CR hold, is:
\[
f'(z) 
\;=\; u_x + i\,v_x.
\]
Thus
\[
f'(z)
\;=\;
u_x(x,y) 
\;+\;
i\,v_x(x,y)
\;=\;
2\,y^{3}
\;+\;
i\,\bigl(-\,6\,x\,y^{2}\bigr)
\;=\;
2\,y^{3}
\;-\;
6\,i\,x\,y^{2}.
\]

\smallskip
\noindent
\underline{\textit{Case 1:}} $y=0$.

If $y=0$, then
\[
f'(z)
\;=\;
2\,(0)^{3} - 6\,i\,x\,(0)^{2}
\;=\;
0.
\]
So on the real axis, $f'(z)=0$ for all $x$.

\smallskip
\noindent
\underline{\textit{Case 2:}} $y^{2} = 3\,(1 - x^{2})$ with $y \neq 0$.

Then
\[
f'(z)
\;=\;
2\,y^{3}
\;-\;
6\,i\,x\,y^{2}.
\]
One could substitute $y^{2} = 3(1 - x^{2})$ if desired, for example:
\[
f'(z)
\;=\;
2\,y \,y^{2}
\;-\;
6\,i\,x\,y^{2}
\;=\;
2\,y\,\bigl(3(1 - x^{2})\bigr)
\;-\;
6\,i\,x\,\bigl(3(1 - x^{2})\bigr),
\]
\[
=\;
6\,(1 - x^{2})\,\Bigl[y \;-\; i\,3\,x\Bigr].
\]
In any case, this fully describes $f'(z)$ at those ellipse points.

\bigskip

\noindent
\textbf{Answer:}
\begin{itemize}
\item $f'(z)$ exists exactly at all points where $y=0$ or $y^{2}=3(1-x^{2}).$
\item At those points,
\[
f'(z) \;=\; 2\,y^{3} \;-\; 6\,i\,x\,y^{2}.
\]
In particular, on the real axis $(y=0)$, $f'(z) = 0.$
\end{itemize}

\section*{Power Series of $\displaystyle \frac{1}{z - 1}$ about $z=-2$}

\noindent
\textbf{Step 1: Rewrite the denominator in terms of $(z+2)$.}

Observe that
\[
z - 1 
\;=\; (z + 2) \;-\; 3.
\]
Hence
\[
\frac{1}{z - 1}
\;=\;
\frac{1}{(z+2) - 3}.
\]

\medskip
\noindent
\textbf{Step 2: Factor out $-3$ in the denominator.}

\[
\frac{1}{(z+2) - 3}
\;=\;
\frac{1}{-\,3\bigl[1 - \tfrac{z+2}{3}\bigr]}
\;=\;
-\,\frac{1}{3}\,
\frac{1}{1 \;-\; \frac{z+2}{3}}.
\]

\medskip
\noindent
\textbf{Step 3: Use the geometric--series expansion.}

Recall that for $|w|<1,$
\(\displaystyle \frac{1}{1-w} = \sum_{n=0}^\infty w^n.\)
Set $w = \tfrac{z+2}{3}$. Then as long as
\[
\left|\frac{z+2}{3}\right| < 1
\quad\Longleftrightarrow\quad
|z+2| < 3,
\]
we get
\[
\frac{1}{1 \;-\; \frac{z+2}{3}}
\;=\;
\sum_{n=0}^\infty
\left(\frac{z+2}{3}\right)^n.
\]
Hence
\[
\frac{1}{z - 1}
\;=\;
-\,\frac{1}{3}
\sum_{n=0}^{\infty}
\left(\frac{z+2}{3}\right)^n
\;=\;
-\sum_{n=0}^\infty
\frac{(z+2)^n}{3^{\,n+1}}.
\]

\medskip
\noindent
\textbf{Step 4: Final form of the power series.}

Therefore, for $|z+2|<3,$
\[
\frac{1}{z-1}
\;=\;
-\,\sum_{n=0}^\infty 
\frac{(z+2)^n}{3^{\,n+1}}
\;=\;
-\,\frac{1}{3}
\;-\;\frac{z+2}{3^2}
\;-\;\frac{(z+2)^2}{3^3}
\;-\;\cdots.
\]

\end{document}
