\documentclass[12pt]{article}
\usepackage[margin=.5in]{geometry}
\usepackage{amsmath,amssymb,amsthm}
\usepackage{hyperref}
\usepackage{xcolor}
\newcommand{\Log}{\operatorname{Log}}

\theoremstyle{plain}
\newtheorem{theorem}{Theorem}
\newtheorem{proposition}{Proposition}
\newtheorem{lemma}{Lemma}

\theoremstyle{definition}
\newtheorem{definition}{Definition}
\newtheorem{example}{Example}
\newtheorem{remark}{Remark}

\begin{document}

\begin{center}
    {\Large \textbf{Important Theorems and Key Equations}}
\end{center}
\vspace{1em}

\section*{I. Elementary Derivatives}

\begin{theorem}[Derivatives of Sine and Cosine]
\label{thm:sin-cos-derivatives}
\[
\frac{d}{dz}\bigl(\sin z\bigr) = \cos z,
\quad
\frac{d}{dz}\bigl(\cos z\bigr) = -\sin z.
\]
\end{theorem}

\begin{theorem}[Derivatives of Hyperbolic Sine and Cosine]
\[
\frac{d}{dz}\bigl(\sinh z\bigr) = \cosh z,
\quad
\frac{d}{dz}\bigl(\cosh z\bigr) = \sinh z.
\]
\end{theorem}

\begin{theorem}[Derivative of the Tangent]
\[
\frac{d}{dz}\bigl(\tan z\bigr) \;=\; \sec^2 z.
\]
\end{theorem}

\begin{theorem}[Derivatives of the Inverse Trigonometric Functions]
\[
\frac{d}{dz}\bigl(\arcsin z\bigr) = \frac{1}{\sqrt{1 - z^2}},
\quad
\frac{d}{dz}\bigl(\arctan z\bigr) = \frac{1}{1 + z^2}.
\]
\end{theorem}


\section*{II. Chain Rule and Related Examples}

\begin{theorem}[Chain Rule in the Complex Plane]
If \(f\) is analytic on a domain \(D\) and \(g\) is analytic on a domain \(\Omega\) containing the range of \(f\), then \(g\circ f\) is analytic on \(D\).  Moreover,
\[
\bigl(g(f(z))\bigr)' 
\;=\;
g'\bigl(f(z)\bigr)\;f'(z).
\]
\end{theorem}

\begin{theorem}[Examples of the Chain Rule]
\ 
\begin{itemize}
\item \(\displaystyle \frac{d}{dz}\bigl[(z^3 + 100)^{-4}\bigr]
  = -12\,z^2\,(z^3 + 100)^{-5}.\)
\smallskip
\item \(\displaystyle \frac{d}{dz}\bigl[(\Log z)^3\bigr]
  = \frac{3\,(\Log z)^2}{z}.\)
\smallskip
\item \(\displaystyle \frac{d}{dz}\bigl[\sinh(e^z)\bigr]
  = \cosh(e^z)\,\cdot\,e^z.\)
\end{itemize}
\end{theorem}


\section*{III. Sum, Product, and Quotient Rules}

\begin{theorem}[Sum Rule]
\label{thm:sum-rule}
If \(f\) and \(g\) are analytic on a domain \(D\), then
\[
(f + g)'(z) \;=\; f'(z) + g'(z).
\]
\end{theorem}

\begin{theorem}[Product Rule]
\label{thm:product-rule}
If \(f\) and \(g\) are analytic on a domain \(D\), then
\[
(f\,g)'(z) \;=\; f'(z)\,g(z) \;+\; f(z)\,g'(z).
\]
\end{theorem}

\begin{theorem}[Quotient Rule]
\label{thm:quotient-rule}
If \(f\) and \(g\) are analytic on a domain \(D\) and \(g(z)\neq 0\), then
\[
\biggl(\frac{f}{g}\biggr)'(z)
\;=\;
\frac{f'(z)\,g(z) - f(z)\,g'(z)}{g(z)^2}.
\]
\end{theorem}

\section*{IV. Zeros of the Derivative \(\implies\) Constant Function}

\begin{theorem}[Identity Theorem for the Derivative Zero Case]
\label{thm:constant-if-deriv-zero}
Let \(f\) be analytic on a domain \(D\).  If \(f'(z)=0\) for all \(z \in D\), then \(f\) is constant on \(D\).
\end{theorem}

\section*{V. Non-Analytic Example}

\begin{theorem}[\(\overline{z}\) is Not Analytic]
\label{thm:zbar-non-analytic}
The function \(h(z) = \overline{z}\) is \emph{not} analytic on any domain in \(\mathbb{C}\).  Indeed, its partial derivatives fail the Cauchy--Riemann equations everywhere except possibly at isolated points.
\end{theorem}

\section*{VI. Power-Series Radius of Convergence: Key Formulas}

\begin{theorem}[Radius of Convergence via Root Test]
\label{thm:radius-of-convergence}
For a power series 
\(\displaystyle \sum_{k=0}^{\infty} a_k (z - z_0)^k,\)
the radius of convergence \(R\) is given by
\[
R 
\;=\;
\frac{1}{\limsup_{k \to \infty} \sqrt[k]{|a_k|}}
\quad
\Bigl(\text{provided the limit supremum is }>0\Bigr).
\]
\end{theorem}

\begin{example}[Illustrations]
\ 
\begin{itemize}
\item \(\displaystyle \sum_{k=1}^{\infty} k\,(z-1)^k\):
\quad
Here \(a_k = k\).  Since \(\lim_{k\to \infty}\sqrt[k]{k}=1,\) the radius of convergence is \(R=1.\)

\item \(\displaystyle \sum_{j=0}^{\infty}\frac{z^{3j}}{2^j}\):
\quad
Geometric in \(z^3/2\).  Converges if \(\bigl|\tfrac{z^3}{2}\bigr|<1,\) i.e.\ \(|z|<\sqrt[3]{2}.\)
\end{itemize}
\end{example}

\section*{VII. Selected Maclaurin Series}

\begin{theorem}[Exponential and Cosine Expansions]
\ 
\begin{align*}
& e^{-z} 
  \;=\;
  \sum_{n=0}^\infty \frac{(-z)^n}{n!}
  \;=\;
  \sum_{n=0}^\infty 
  (-1)^n \,\frac{z^n}{n!}.
\\[6pt]
& \cos z
  \;=\;
  \sum_{n=0}^\infty (-1)^n \,\frac{z^{2n}}{(2n)!}
  \quad\Longrightarrow\quad
  z^2\,\cos z
  \;=\;
  \sum_{n=0}^\infty (-1)^n \,\frac{z^{2n+2}}{(2n)!}.
\end{align*}
\end{theorem}

\begin{theorem}[Geometric-Type Expansions]
For \(|z|<1,\)
\[
\frac{1}{1-z} \;=\; \sum_{n=0}^{\infty} z^n,
\quad
\frac{1+z}{1-z}
\;=\;
1 \;+\; 2\sum_{n=1}^\infty z^n,
\quad
\frac{z^3}{1 - z^3}
\;=\;
\sum_{n=1}^\infty z^{3n}.
\]
\end{theorem}

\begin{theorem}[Example with \((4 - z)^{-2}\)]
\[
\frac{1}{(4 - z)^2}
\;=\;
\frac{1}{16}\,\frac{1}{\bigl(1 - \tfrac{z}{4}\bigr)^2}
\;=\;
\sum_{n=0}^\infty 
(n+1)\,\frac{z^n}{4^{n+2}}
\quad (\text{for }|z|<4).
\]
\end{theorem}


\section*{VIII. Examples of Summation to Closed Form}

\begin{theorem}[Examples of Closed-Form Sums]
\ 
\begin{itemize}
\item \(\displaystyle \sum_{n=0}^{\infty} \frac{z^{2n}}{n!} 
    \;=\;
    e^{z^2}.\)

\item \(\displaystyle \sum_{n=1}^{\infty} z^{3n}
    \;=\;
    \frac{z^3}{1 - z^3}
    \quad (|z^3|<1).\)

\item \(\displaystyle \sum_{n=1}^{\infty} n\,(z-1)^{n-1}
    \;=\;
    \frac{1}{(2 - z)^2}
    \quad (|z-1|<1).\)
\end{itemize}
\end{theorem}

\end{document}