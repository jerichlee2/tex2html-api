\documentclass[12pt]{article}

% Packages
\usepackage[margin=1in]{geometry}
\usepackage{amsmath,amssymb,amsthm}
\usepackage{enumitem}
\usepackage{hyperref}
\usepackage{xcolor}
\usepackage{import}
\usepackage{xifthen}
\usepackage{pdfpages}
\usepackage{transparent}
\usepackage{listings}


\lstset{
    breaklines=true,         % Enable line wrapping
    breakatwhitespace=false, % Wrap lines even if there's no whitespace
    basicstyle=\ttfamily,    % Use monospaced font
    frame=single,            % Add a frame around the code
    columns=fullflexible,    % Better handling of variable-width fonts
}

\newcommand{\incfig}[1]{%
    \def\svgwidth{\columnwidth}
    \import{./Figures/}{#1.pdf_tex}
}
\theoremstyle{definition} % This style uses normal (non-italicized) text
\newtheorem{solution}{Solution}
\newtheorem{proposition}{Proposition}
\newtheorem{problem}{Problem}
\newtheorem{lemma}{Lemma}
\newtheorem{theorem}{Theorem}
\newtheorem{remark}{Remark}
\newtheorem{note}{Note}
\newtheorem{definition}{Definition}
\newtheorem{example}{Example}
\theoremstyle{plain} % Restore the default style for other theorem environments
%

% Theorem-like environments
% Title information
\title{}
\author{Jerich Lee}
\date{\today}

\begin{document}

\maketitle
We wish to evaluate the integral
\[
\oint_C \frac{dz}{\left(z^3+1\right)^2},
\]
where \(C\) is the circle centered at \(-2\) with radius \(2\). We will solve this using the generalized Cauchy integral formula for derivatives.

\section*{Step 1. Express the Integrand}

Factor the denominator:
\[
z^3+1 = (z+1)(z^2-z+1).
\]
Thus, we have
\[
\frac{1}{\left(z^3+1\right)^2} = \frac{1}{(z+1)^2 (z^2-z+1)^2}.
\]
Define
\[
f(z) = \frac{1}{\left(z^2-z+1\right)^2}.
\]
Then the integrand becomes
\[
\frac{1}{\left(z^3+1\right)^2} = \frac{f(z)}{(z+1)^2}.
\]

\section*{Step 2. Locate the Singularities}

The singularities of the integrand occur when
\[
z^3+1=0 \quad \Longrightarrow \quad z^3=-1.
\]
The cube roots of \(-1\) are:
\[
z=-1,\quad z=\frac{1}{2}+\frac{\sqrt{3}}{2}i,\quad z=\frac{1}{2}-\frac{\sqrt{3}}{2}i.
\]
Since the integrand has a squared factor in the denominator, each singularity is a double pole. The circle \( |z+2|=2 \) contains only the singularity at \( z=-1 \) (since \( |-1+2| = 1 < 2 \)).

\section*{Step 3. Apply the Cauchy Integral Formula for Derivatives}

The generalized Cauchy integral formula for derivatives states that if \( f \) is analytic inside and on a closed contour \( C \) and \( z_0 \) is inside \( C \), then
\[
\frac{1}{2\pi i}\oint_C \frac{f(z)}{(z-z_0)^{n+1}} \, dz = \frac{f^{(n)}(z_0)}{n!}.
\]
Here, with \( z_0=-1 \) and \( n=1 \) (since we have a double pole), we have
\[
\oint_C \frac{f(z)}{(z+1)^2} \, dz = 2\pi i\, f'(-1).
\]

\section*{Step 4. Compute \( f'(z) \)}

Recall that
\[
f(z) = \frac{1}{\left(z^2-z+1\right)^2}.
\]
Let
\[
g(z)=z^2-z+1,
\]
so that
\[
f(z) = \frac{1}{[g(z)]^2}.
\]
Differentiating using the chain rule,
\[
f'(z) = -2 \frac{g'(z)}{[g(z)]^3}.
\]
Since
\[
g'(z)=\frac{d}{dz}(z^2-z+1)=2z-1,
\]
we evaluate at \( z=-1 \):
\[
g(-1)=(-1)^2-(-1)+1=1+1+1=3,
\]
\[
g'(-1)=2(-1)-1=-2-1=-3.
\]
Thus,
\[
f'(-1) = -2\frac{-3}{3^3} = \frac{6}{27} = \frac{2}{9}.
\]

\section*{Step 5. Final Answer}

Substituting back into the formula,
\[
\oint_C \frac{dz}{\left(z^3+1\right)^2} = 2\pi i\, f'(-1) = 2\pi i \left(\frac{2}{9}\right) = \frac{4\pi i}{9}.
\]

\[
\boxed{\frac{4\pi i}{9}}
\]

\end{document}
