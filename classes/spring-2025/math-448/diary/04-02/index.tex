\documentclass[12pt]{article}

% Packages
\usepackage[margin=1in]{geometry}
\usepackage{amsmath,amssymb,amsthm}
\usepackage{enumitem}
\usepackage{hyperref}
\usepackage{xcolor}
\usepackage{import}
\usepackage{xifthen}
\usepackage{pdfpages}
\usepackage{transparent}
\usepackage{listings}
\DeclareMathOperator{\Log}{Log}
\DeclareMathOperator{\Arg}{Arg}

\lstset{
    breaklines=true,         % Enable line wrapping
    breakatwhitespace=false, % Wrap lines even if there's no whitespace
    basicstyle=\ttfamily,    % Use monospaced font
    frame=single,            % Add a frame around the code
    columns=fullflexible,    % Better handling of variable-width fonts
}

\newcommand{\incfig}[1]{%
    \def\svgwidth{\columnwidth}
    \import{./Figures/}{#1.pdf_tex}
}
\theoremstyle{definition} % This style uses normal (non-italicized) text
\newtheorem{solution}{Solution}
\newtheorem{proposition}{Proposition}
\newtheorem{problem}{Problem}
\newtheorem{lemma}{Lemma}
\newtheorem{theorem}{Theorem}
\newtheorem{remark}{Remark}
\newtheorem{note}{Note}
\newtheorem{definition}{Definition}
\newtheorem{example}{Example}
\newtheorem{corollary}{Corollary}
\theoremstyle{plain} % Restore the default style for other theorem environments
%

% Theorem-like environments
% Title information
\title{}
\author{Jerich Lee}
\date{\today}

\begin{document}

\maketitle
\textbf{Claim.} 
\[
\frac{1}{2\pi i}\oint_{|z - z_0| = s} (z - z_0)^j \, dz 
= 
\begin{cases}
1 & \text{if } j = -1,\\
0 & \text{if } j \text{ is any other integer}.
\end{cases}
\]

\begin{proof}
\textbf{Step 1: Parametrize the contour.}\\
Let the contour be the circle of radius $s$ centered at $z_0$, i.e.\ $|z - z_0| = s$. We may parametrize this contour by
\[
z = z_0 + s e^{i\theta},
\quad\quad
\theta \in [0,2\pi].
\]
Then the differential $dz$ is given by
\[
dz = \frac{d}{d\theta}(z_0 + s e^{i\theta}) \, d\theta 
= i s e^{i\theta} \, d\theta.
\]

\noindent
\textbf{Step 2: Rewrite the integrand in the chosen parameter.}\\
On this contour,
\[
(z - z_0)^j = (s e^{i\theta})^j = s^j e^{i j \theta}.
\]
Hence the integrand becomes
\[
(z - z_0)^j \, dz 
= s^j e^{i j \theta} \,\bigl(i s e^{i\theta}\bigr) \, d\theta 
= i s^{j+1} e^{i (j+1)\theta} \, d\theta.
\]

\noindent
\textbf{Step 3: Evaluate the integral.}\\
Therefore,
\[
\oint_{|z - z_0|=s} (z - z_0)^j \, dz 
= 
\int_{0}^{2\pi} i s^{j+1} e^{i (j+1)\theta} \, d\theta 
= 
i s^{j+1} \int_{0}^{2\pi} e^{i (j+1)\theta} \, d\theta.
\]
We now incorporate the factor $\frac{1}{2 \pi i}$:
\[
\frac{1}{2\pi i}
\oint_{|z - z_0| = s} (z - z_0)^j \, dz 
= 
\frac{1}{2\pi i} 
\Bigl(
i s^{j+1} \int_{0}^{2\pi} e^{i (j+1)\theta} \, d\theta
\Bigr).
\]
Simplify:
\[
= 
\frac{s^{j+1}}{2\pi} 
\int_{0}^{2\pi} e^{i (j+1)\theta} \, d\theta.
\]

\noindent
\textbf{Step 4: Evaluate the remaining integral.}\\
We recognize that
\[
\int_{0}^{2\pi} e^{i (j+1)\theta} \, d\theta 
=
\begin{cases}
2\pi, & \text{if } j+1 = 0 \\
0, & \text{if } j+1 \neq 0
\end{cases}
\quad
\text{i.e. }
\begin{cases}
2\pi, & \text{if } j = -1, \\
0, & \text{if } j \neq -1.
\end{cases}
\]
Thus:
\[
\frac{s^{j+1}}{2\pi} \int_{0}^{2\pi} e^{i (j+1)\theta} \, d\theta 
= 
\begin{cases}
\frac{s^{-1+1}}{2\pi} \cdot 2\pi = 1, & \text{if } j = -1, \\
0, & \text{otherwise}.
\end{cases}
\]
Hence,
\[
\frac{1}{2\pi i}\oint_{|z - z_0| = s} (z - z_0)^j \, dz 
= 
\begin{cases}
1 & \text{if } j = -1,\\
0 & \text{otherwise}.
\end{cases}
\]

\end{proof}

\noindent
\textbf{Claim.} For large \(R\), the integral
\[
\int_{0}^{\pi} f\bigl(Re^{i\theta}\bigr)\,R\,e^{i\theta}\,d\theta
\]
goes to \(0\) as \(R \to \infty\), assuming 
\[
\left|f\bigl(Re^{i\theta}\bigr)\right| \le \frac{e^{-R\sin\theta}}{R^2 - \alpha^2}.
\]

\begin{proof}

\textbf{Step 1: Set up the integral and take absolute values.}\\
We want to estimate 
\[
\int_{0}^{\pi} f\bigl(Re^{i\theta}\bigr)\,R\,e^{i\theta}\,d\theta.
\]
First, take the absolute value inside an integral:
\[
\left|
\int_{0}^{\pi} f\bigl(Re^{i\theta}\bigr)\,R\,e^{i\theta}\,d\theta
\right|
\;\le\;
\int_{0}^{\pi}
\Bigl| f\bigl(Re^{i\theta}\bigr)\,R\,e^{i\theta}\Bigr|
\,d\theta.
\]

\textbf{Step 2: Use the given bound on \(f\).}\\
By assumption, for each \(\theta \in [0,\pi]\),
\[
\bigl|f\bigl(Re^{i\theta}\bigr)\bigr|
\;\le\;
\frac{e^{-\,R\sin\theta}}{R^{2}-\alpha^{2}}.
\]
Also, \(\bigl|R\,e^{i\theta}\bigr| = R\). Hence
\[
\bigl|
f\bigl(Re^{i\theta}\bigr)\,R\,e^{i\theta}
\bigr|
\;\le\;
R \,\frac{e^{-\,R\sin\theta}}{R^{2}-\alpha^{2}}
\;=\;
\frac{R\, e^{-\,R\sin\theta}}{R^{2}-\alpha^{2}}.
\]
Therefore,
\[
\int_{0}^{\pi}
\Bigl| f\bigl(Re^{i\theta}\bigr)\,R\,e^{i\theta}\Bigr|
\,d\theta
\;\le\;
\int_{0}^{\pi}
\frac{R\, e^{-\,R\sin\theta}}{R^{2}-\alpha^{2}}
\,d\theta
\;=\;
\frac{R}{R^{2}-\alpha^{2}}
\;\int_{0}^{\pi}
e^{-\,R\sin\theta}\,d\theta.
\]

\textbf{Step 3: Bound the exponential integral.}\\
Note that \(e^{-\,R\sin\theta} \le 1\) for all \(\theta\). Hence
\[
\int_{0}^{\pi} e^{-\,R\sin\theta}\,d\theta
\;\le\;
\int_{0}^{\pi} 1\,d\theta 
\;=\; \pi.
\]
Thus,
\[
\frac{R}{R^{2}-\alpha^{2}}
\;\int_{0}^{\pi}
e^{-\,R\sin\theta}\,d\theta
\;\le\;
\frac{R}{R^{2}-\alpha^{2}}
\;\pi
\;=\;
\frac{\pi\,R}{R^{2}-\alpha^{2}}.
\]

\textbf{Step 4: Take the limit as \(R \to \infty\).}\\
Finally, as \(R \to \infty\),
\[
\frac{\pi\,R}{R^{2}-\alpha^{2}}
\;=\;
\frac{\pi}{R - \frac{\alpha^{2}}{R}}
\;\longrightarrow\; 0.
\]
Hence
\[
\left|
\int_{0}^{\pi} f\bigl(Re^{i\theta}\bigr)\,R\,e^{i\theta}\,d\theta
\right|
\;\le\;
\frac{\pi\,R}{R^{2}-\alpha^{2}}
\;\longrightarrow\;
0,
\]
which completes the proof.
\end{proof}

\noindent
\textbf{Claim.} For $\alpha>0$,
\[
\int_{-\infty}^{\infty} \frac{\cos x}{x^2 + \alpha^2}\,dx
\;=\;\frac{\pi}{\alpha}\,e^{-\alpha}.
\]

\begin{proof}
\textbf{Step 1: Set up the complex integral.}\\
Consider the complex function
\[
f(z)
\;=\;
\frac{e^{\,i\,z}}{z^2 + \alpha^2}.
\]
We seek to relate
\(\displaystyle \int_{-\infty}^{\infty} \frac{\cos x}{x^2+\alpha^2}\,dx\)
to
\(\displaystyle \int_{-\infty}^{\infty} \frac{e^{\,i\,x}}{x^2+\alpha^2}\,dx.\)

Note that
\[
\cos x \;=\;\mathrm{Re}\bigl(e^{\,i\,x}\bigr).
\]
Thus
\[
\int_{-\infty}^\infty \frac{\cos x}{x^2 + \alpha^2}\,dx
\;=\;
\mathrm{Re}\!
\Bigl(
\int_{-\infty}^\infty \frac{e^{\,i\,x}}{x^2 + \alpha^2}\,dx
\Bigr).
\]
Hence, if we can compute 
\(\int_{-\infty}^\infty \frac{e^{\,i\,x}}{x^2 + \alpha^2}\,dx,\)
then taking the real part will give us the desired integral.

\bigskip
\noindent
\textbf{Step 2: Choose a semicircular contour in the upper half‐plane.}\\
Let $\Gamma_R$ be the contour consisting of:
\begin{itemize}
\item the real line segment from $-R$ to $R$, and
\item the semicircle $C_R$ in the upper half‐plane centered at the origin with radius $R$, parametrized by $z = Re^{i\theta}$ for $\theta\in [0,\pi]$.
\end{itemize}
We will let $R \to \infty$ eventually. The contour encloses the pole at $z=i\alpha$ but \emph{not} the pole at $z=-\,i\alpha$ (which is in the lower half‐plane).

\bigskip
\noindent
\textbf{Step 3: Apply the Residue Theorem.}\\
By the residue theorem,
\[
\oint_{\Gamma_R} f(z)\,dz
\;=\;
2\pi i\,\bigl(\text{sum of residues of }f\text{ inside }\Gamma_R\bigr).
\]
Since the only pole of $f(z)=\dfrac{e^{iz}}{z^2+\alpha^2}$ in the upper half‐plane is at $z=i\alpha$, we need to compute $\mathrm{Res}\bigl(f(z), z=i\alpha\bigr).$

\bigskip
\noindent
\textbf{Step 4: Compute the residue at $z=i\alpha$.}\\
We have
\[
z^2 + \alpha^2
\;=\;
(z - i\alpha)\,(z + i\alpha),
\]
so a simple pole occurs at $z=i\alpha$. Hence
\[
\mathrm{Res}\bigl(f(z), z=i\alpha\bigr)
\;=\;
\lim_{z\to i\alpha}\,(z - i\alpha)\,\frac{e^{i z}}{(z - i\alpha)\,(z + i\alpha)}
\;=\;
\frac{e^{\,i\,(i\alpha)}}{2\,i\alpha}.
\]
Since $e^{i(i\alpha)}=e^{-\alpha}$,
\[
\mathrm{Res}\bigl(f(z), z=i\alpha\bigr)
\;=\;
\frac{e^{-\alpha}}{2\,i\,\alpha}.
\]
Thus, the residue theorem gives
\[
\oint_{\Gamma_R} \frac{e^{\,i\,z}}{z^2 + \alpha^2}\,dz
\;=\;
2\pi i 
\;\times\;
\frac{\,e^{-\alpha}\,}{2\,i\,\alpha}
\;=\;
\frac{\pi}{\alpha}\,e^{-\alpha}.
\]

\bigskip
\noindent
\textbf{Step 5: Show that the integral over the semicircle $C_R$ vanishes as $R\to\infty$.}\\
Parametrize $C_R$ by $z=Re^{i\theta}$, $\theta\in [0,\pi]$.  On $C_R$,
\[
\left| e^{\,i\,z}\right|
\;=\;
\bigl|e^{\,i\,R e^{i\theta}}\bigr|
\;=\;
\bigl|e^{\,iR(\cos\theta + i\sin\theta)}\bigr|
\;=\;
e^{-\,R\sin\theta}.
\]
Because $\sin\theta\ge0$ for $\theta\in [0,\pi]$, we have exponential decay in $\mathrm{Im}(z)>0$.  This implies that 
\[
\left|\int_{C_R} \frac{e^{\,i\,z}}{z^2+\alpha^2}\,dz\right|
\;\to\;
0
\quad\text{as }R\to\infty,
\]
often shown by Jordan's lemma or a direct comparison estimate.

\bigskip
\noindent
\textbf{Step 6: Take the limit as $R\to \infty$.}\\
Hence, in the limit,
\[
\int_{-R}^R \frac{e^{\,i\,x}}{x^2+\alpha^2}\,dx
\;+\;
\int_{C_R} \frac{e^{\,i\,z}}{z^2+\alpha^2}\,dz
\;=\;
\oint_{\Gamma_R} f(z)\,dz
\;=\;
\frac{\pi}{\alpha}\,e^{-\alpha}.
\]
As $R\to \infty$, the integral over $C_R$ vanishes, so
\[
\int_{-\infty}^{\infty} \frac{e^{\,i\,x}}{x^2 + \alpha^2}\,dx 
\;=\;
\frac{\pi}{\alpha}\,e^{-\alpha}.
\]

\bigskip
\noindent
\textbf{Step 7: Extract the real part to get the original integral.}\\
Recall that $\cos x = \mathrm{Re}\bigl(e^{\,i\,x}\bigr)$.  Therefore,
\[
\int_{-\infty}^{\infty} \frac{\cos x}{x^2 + \alpha^2}\,dx
\;=\;
\mathrm{Re}
\!\Bigl(
\int_{-\infty}^{\infty} \frac{e^{\,i\,x}}{x^2 + \alpha^2}\,dx
\Bigr)
\;=\;
\mathrm{Re}\Bigl(\frac{\pi}{\alpha}\,e^{-\alpha}\Bigr)
\;=\;
\frac{\pi}{\alpha}\,e^{-\alpha}.
\]
Since $e^{-\alpha}$ is real and positive, taking the real part has no effect on its value.  

\bigskip
\noindent
Hence we conclude
\[
\boxed{
\int_{-\infty}^{\infty} \frac{\cos x}{x^2 + \alpha^2}\,dx
\;=\;
\frac{\pi}{\alpha}\,e^{-\alpha}.
}
\]

\end{proof}
\noindent
\textbf{Jordan's Lemma (Upper Half-Plane).}\\[6pt]
\emph{Statement.} 
Let $a>0$. Suppose $g$ is a continuous function on the semicircle 
$C_R = \{R e^{i\theta} : \theta\in[0,\pi]\}$, 
and that $g(z)$ does not grow too rapidly as $|z|\to\infty$ (for instance, if $|g(z)| \le M$ for some constant $M$, or decays polynomially, etc.). 
Then
\[
\left|
\int_{C_R} e^{\,i\,a\,z} \, g(z)\,dz
\right|
\;\longrightarrow\; 0
\quad
\text{as }R\to \infty.
\]

\begin{proof}
\textbf{Step 1: Parametrize the semicircle.}\\
Let $C_R$ be the arc of the circle of radius $R$ in the upper half-plane.  A convenient parametrization is
\[
z(\theta) \;=\; R\,e^{i\theta},
\quad
\theta \in [0,\pi].
\]
Then $dz = i\,R\,e^{i\theta}\,d\theta$, so
\[
\int_{C_R} e^{\,i\,a\,z} \, g(z)\,dz
\;=\;
\int_{0}^{\pi} e^{\,i\,a\,\bigl(R e^{i\theta}\bigr)} 
\,g\bigl(R e^{i\theta}\bigr)\,\bigl(i\,R\,e^{i\theta}\bigr)\,d\theta.
\]

\noindent
\textbf{Step 2: Take absolute values and simplify.}\\
By the triangle inequality,
\[
\left|
\int_{0}^{\pi} e^{\,i\,a\,\bigl(R e^{i\theta}\bigr)} 
\,g\bigl(R e^{i\theta}\bigr)\,\bigl(i\,R\,e^{i\theta}\bigr)\,d\theta
\right|
\;\le\;
\int_{0}^{\pi} 
\left|
e^{\,i\,a\,\bigl(R e^{i\theta}\bigr)}
\right|
\cdot
\bigl|g\bigl(R e^{i\theta}\bigr)\bigr|
\cdot
\bigl|i\,R\,e^{i\theta}\bigr|
\,d\theta.
\]
Notice:
\[
\bigl|i\,R\,e^{i\theta}\bigr|
\;=\;
R,
\]
and
\[
\left|
e^{\,i\,a\,\bigl(R e^{i\theta}\bigr)}
\right|
\;=\;
\left|
e^{\,i\,a\,R(\cos\theta + i\,\sin\theta)}
\right|
\;=\;
\left|
e^{\,i\,a\,R\cos\theta}
\right|
\cdot
\left|
e^{-\,a\,R\,\sin\theta}
\right|
\;=\;
e^{-\,a\,R\,\sin\theta}.
\]
Hence,
\[
\left|
\int_{C_R} e^{\,i\,a\,z} \, g(z)\,dz
\right|
\;\le\;
\int_{0}^{\pi} 
e^{-\,a\,R\,\sin\theta}
\;\bigl|g\bigl(R e^{i\theta}\bigr)\bigr|
\;R
\,d\theta.
\]

\noindent
\textbf{Step 3: Use the exponential decay in the upper half-plane.}\\
For $\theta\in(0,\pi)$, we have $\sin\theta>0$, so $e^{-\,a\,R\,\sin\theta}$ provides strong decay as $R\to\infty$.  If $g$ is not too large, e.g.\ if there is a constant $M$ such that 
$\bigl|g\bigl(Re^{i\theta}\bigr)\bigr|\le M$ for all large $R$, 
then
\[
\bigl|g\bigl(R e^{i\theta}\bigr)\bigr|
\;\le\; M
\quad
\text{and we get}
\quad
\left|
\int_{C_R} e^{\,i\,a\,z} \, g(z)\,dz
\right|
\;\le\;
M \,R
\int_{0}^{\pi} 
e^{-\,a\,R\,\sin\theta}\,d\theta.
\]
By splitting the integral near $\theta=0,\pi$ and using the fact that $\sin\theta$ is bounded away from 0 in the interior of $(0,\pi)$, one can show 
\(\displaystyle \int_{0}^{\pi} e^{-\,a\,R\,\sin\theta}\,d\theta \to 0\) 
as $R\to\infty$.  

More precisely, a simple upper bound is:
\[
\int_{0}^{\pi} e^{-\,a\,R\,\sin\theta}\,d\theta 
\;\le\;
\int_{0}^{\pi} e^{-\,a\,R\left(\frac{2\theta}{\pi}\right)}\,d\theta
\quad (\text{using a geometric bound on }\sin\theta)
\;=\;
\frac{\pi}{2aR}\,\bigl(1 - e^{-\,2\,a\,R}\bigr)
\;\longrightarrow\;0.
\]
(One can use other bounds or compare with known integrals. The main idea is that $e^{-\,a\,R\,\sin\theta}$ goes to zero rapidly except possibly very close to $0$ or $\pi$, but there it can still be shown to vanish in the limit.)

\noindent
Hence the entire integral over $C_R$ goes to $0$ as $R\to\infty$:
\[
\left|\int_{C_R} e^{\,i\,a\,z}\,g(z)\,dz\right|\;\longrightarrow\;0.
\]
\end{proof}

how do we do 
\begin{align}
    f(z)= \frac{e^{iz}}{z^{2}+\alpha^{2}} \\[10pt] 
    \left\vert \int_{0}^{\pi}f(Re^{i\theta})Re^{i\theta}  \,\mathrm{d}\theta  \right\vert \leq R \int_{0}^{\pi} \frac{e^{-R\sin \theta}}{R^{2}-\alpha^{2}} \,\mathrm{d}\theta \\[10pt] 
    \leq  \frac{\pi R}{R^{2}-\alpha^{2}}\to 0, R \to \infty 
\end{align}

\noindent
\textbf{Claim.} 
\[
\left|
\int_{0}^{\pi} f\bigl(Re^{i\theta}\bigr)\,R\,e^{i\theta}\,\mathrm{d}\theta
\right|
\;\le\;
\frac{\pi\,R}{\,R^2 - \alpha^2\,}
\;\longrightarrow\;0 
\quad
\text{as } R \to \infty,
\]
where
\[
f(z) \;=\; \frac{e^{\,i\,z}}{z^{2}+\alpha^{2}}, 
\quad
z = Re^{i\theta}.
\]

\begin{proof}
\textbf{Step 1: Parametrize the contour.}\\
We consider the semicircle in the upper half-plane of radius $R$, parametrized by 
\[
z(\theta) \;=\; R e^{i\theta},
\quad
\theta \in [0,\pi].
\]
Then the integrand in question is $f\bigl(Re^{i\theta}\bigr)\,R\,e^{i\theta}$.

\bigskip
\noindent
\textbf{Step 2: Take absolute values and use the exponential decay.}\\
Write 
\[
f\bigl(Re^{i\theta}\bigr)\,R\,e^{i\theta}
\;=\;
\frac{e^{\,i\,(R e^{i\theta})}}{\bigl(Re^{i\theta}\bigr)^2 + \alpha^2}
\;\times\;(R e^{i\theta}).
\]
Hence,
\[
\Bigl| f\bigl(Re^{i\theta}\bigr)\,R\,e^{i\theta}\Bigr|
\;=\;
\frac{\bigl| e^{\,i\,R e^{i\theta}}\bigr|}
{\bigl|\bigl(Re^{i\theta}\bigr)^2 + \alpha^2\bigr|}
\;\times\; \bigl| R e^{i\theta}\bigr|.
\]
Notice that
\[
\bigl| e^{\,i\,R e^{i\theta}}\bigr|
\;=\;
e^{-\,R\sin\theta},
\quad
\bigl| R e^{i\theta}\bigr|
\;=\;
R,
\]
and by the reverse triangle inequality, for large $R$,
\[
\Bigl|\bigl(Re^{i\theta}\bigr)^2 + \alpha^2\Bigr|
\;=\;
\bigl|R^2 e^{2i\theta} + \alpha^2\bigr|
\;\ge\;
R^2 - \alpha^2.
\]
Hence,
\[
\Bigl| f\bigl(Re^{i\theta}\bigr)\,R\,e^{i\theta}\Bigr|
\;\le\;
\frac{R\, e^{-\,R\sin\theta}}{R^2 - \alpha^2}.
\]

\bigskip
\noindent
\textbf{Step 3: Integrate from $0$ to $\pi$.}\\
Thus,
\[
\left|
\int_{0}^{\pi} f\bigl(Re^{i\theta}\bigr)\,R\,e^{i\theta}\,\mathrm{d}\theta
\right|
\;\le\;
\int_{0}^{\pi}
\Bigl| f\bigl(Re^{i\theta}\bigr)\,R\,e^{i\theta}\Bigr|
\,\mathrm{d}\theta
\;\le\;
\int_{0}^{\pi} \frac{R\, e^{-\,R\sin\theta}}{R^2 - \alpha^2}
\,\mathrm{d}\theta.
\]
Factor out $\displaystyle \frac{R}{R^2 - \alpha^2}$:
\[
=\;
\frac{R}{R^2 - \alpha^2}
\int_{0}^{\pi}
e^{-\,R\sin\theta}
\,\mathrm{d}\theta.
\]
Since $e^{-\,R\sin\theta} \le 1$ for all $\theta \in [0,\pi]$,
\[
\int_{0}^{\pi} e^{-\,R\sin\theta}
\,\mathrm{d}\theta
\;\le\;
\int_{0}^{\pi} 1 \,\mathrm{d}\theta
\;=\;
\pi.
\]
Hence,
\[
\frac{R}{R^2 - \alpha^2}
\int_{0}^{\pi}
e^{-\,R\sin\theta}
\,\mathrm{d}\theta
\;\le\;
\frac{\pi\,R}{R^2 - \alpha^2}.
\]

\bigskip
\noindent
\textbf{Step 4: Take the limit as $R\to\infty$.}\\
Finally,
\[
\frac{\pi\,R}{\,R^2 - \alpha^2\,}
\;=\;
\frac{\pi}{\,R - \tfrac{\alpha^2}{R}\,}
\;\xrightarrow{R\to\infty}\;
0.
\]
Therefore,
\[
\left|
\int_{0}^{\pi} f\bigl(Re^{i\theta}\bigr)\,R\,e^{i\theta}
\,\mathrm{d}\theta
\right|
\;\xrightarrow{R\to\infty}\;0,
\]
as desired.
\end{proof}


\begin{theorem}[Reverse Triangle Inequality]
    \label{thm:reverse_triangle_inequality}
    For any real or complex numbers $z$ and $w$, 
    \[
    \bigl| \lvert z\rvert - \lvert w\rvert \bigr| 
    \;\le\; 
    \lvert z - w\rvert.
    \]
    \end{theorem}
    
    \begin{proof}
    We prove this using the standard (forward) triangle inequality 
    \[
    \lvert a + b\rvert \;\le\; \lvert a\rvert + \lvert b\rvert.
    \]
    
    \noindent
    \textbf{Step 1.} Consider 
    \[
    \lvert z\rvert 
    \;=\; 
    \bigl\lvert (z - w) + w\bigr\rvert
    \;\le\; 
    \lvert z - w\rvert + \lvert w\rvert.
    \]
    Rearranging, 
    \[
    \lvert z\rvert - \lvert w\rvert 
    \;\le\; 
    \lvert z - w\rvert.
    \tag{1}
    \]
    
    \noindent
    \textbf{Step 2.} Swap $z$ and $w$. Then
    \[
    \lvert w\rvert 
    \;=\; 
    \bigl\lvert (w - z) + z\bigr\rvert
    \;\le\;
    \lvert w - z\rvert + \lvert z\rvert.
    \]
    Since $\lvert w - z\rvert = \lvert z - w\rvert$, we get
    \[
    \lvert w\rvert - \lvert z\rvert 
    \;\le\; 
    \lvert z - w\rvert.
    \tag{2}
    \]
    
    \noindent
    \textbf{Step 3.} Inequalities \eqref{(1)} and \eqref{(2)} together imply
    \[
    -\lvert z - w\rvert 
    \;\le\;
    \lvert z\rvert - \lvert w\rvert 
    \;\le\;
    \lvert z - w\rvert,
    \]
    which can be rewritten as
    \[
    \bigl|\lvert z\rvert - \lvert w\rvert\bigr|
    \;\le\;
    \lvert z - w\rvert.
    \]
    This completes the proof.
    \end{proof}
    \begin{align}
        f(z)=\frac{e^{iz}}{z^{2}+\alpha^{2}}
    \end{align}

    $\int_{-\infty}^{\infty } \frac{\sin z}{z^{2}+\alpha^{2}} \,\mathrm{d}z =0$  
    $\text{Im}\left\{ 2\pi i \left( \frac{1}{4}\exp\left( \sqrt{2} (-1+i) \right) + \exp\left( \sqrt{2} (-1-i)  \right) \right) \right\} = \pi e^{-\sqrt{2} }\cos(\sqrt{2}) $ 

    $f(z)=e^{iz}\frac{z^{3}}{z^{4}+16}$
    $\text{Res} (f; \sqrt{2} (-1+i))$ 
    
    We consider the function 
\[
f(z) = \frac{e^{iz}\,z^3}{z^4 + 16},
\]
and let
\[
\text{Res}_1 = \mathrm{Res}\bigl(f; \sqrt{2}\,(1 + i)\bigr)
\quad \text{and} \quad
\text{Res}_2 = \mathrm{Res}\bigl(f; \sqrt{2}\,(-1 + i)\bigr).
\]
Each residue can be computed using the standard formula for a simple pole:
\[
\mathrm{Res}\bigl(f; z_0\bigr)
= \frac{h(z_0)}{g'(z_0)}
\quad \text{where} \quad
f(z) = \frac{h(z)}{g(z)}, \; g(z_0) = 0, \; g'(z_0) \neq 0.
\]
Here,
\[
h(z) = e^{iz}\,z^3, \quad g(z) = z^4 + 16, \quad g'(z) = 4z^3.
\]
Thus for each root $z_0$ of $z^4 + 16 = 0$, the residue is
\[
\mathrm{Res}\bigl(f; z_0\bigr)
= \frac{e^{i z_0}\,z_0^3}{4\,z_0^3}
= \frac{e^{i z_0}}{4}.
\]
Hence
\[
\text{Res}_1 + \text{Res}_2 
= \frac{1}{4}\Bigl(e^{i z_1} + e^{i z_2}\Bigr),
\]
where $z_1 = \sqrt{2}\,(1 + i)$ and $z_2 = \sqrt{2}\,(-1 + i)$. We then look at
\[
2\pi i \,\bigl[\text{Res}_1 + \text{Res}_2 \bigr]
= 2\pi i \cdot \frac{1}{4}\Bigl(e^{i z_1} + e^{i z_2}\Bigr)
= \frac{\pi i}{2}\,\Bigl(e^{i z_1} + e^{i z_2}\Bigr).
\]

\paragraph{Step 1: Simplify $e^{i z_1} + e^{i z_2}$.}

We have
\[
z_1 = \sqrt{2}\,(1 + i), \qquad z_2 = \sqrt{2}\,(-1 + i).
\]
Then
\[
i z_1 = i \sqrt{2}\,(1 + i)
= \sqrt{2}\,\bigl(i + i^2\bigr)
= \sqrt{2}\,(i - 1)
= -\sqrt{2} + i\,\sqrt{2},
\]
\[
i z_2 = i \sqrt{2}\,(-1 + i)
= \sqrt{2}\,\bigl(-i + i^2\bigr)
= \sqrt{2}\,\bigl(-i - 1\bigr)
= -\sqrt{2} - i\,\sqrt{2}.
\]
Hence
\[
e^{i z_1}
= e^{-\sqrt{2} + i\sqrt{2}}
= e^{-\sqrt{2}}\Bigl(\cos(\sqrt{2}) + i\sin(\sqrt{2})\Bigr),
\]
\[
e^{i z_2}
= e^{-\sqrt{2} - i\sqrt{2}}
= e^{-\sqrt{2}}\Bigl(\cos(\sqrt{2}) - i\sin(\sqrt{2})\Bigr).
\]
Adding these,
\[
e^{i z_1} + e^{i z_2}
= e^{-\sqrt{2}} \Bigl[ 
  \cos(\sqrt{2}) + i\sin(\sqrt{2}) 
  + \cos(\sqrt{2}) - i\sin(\sqrt{2})
\Bigr]
= 2\,e^{-\sqrt{2}}\,\cos(\sqrt{2}).
\]

\paragraph{Step 2: Multiply by $\frac{\pi i}{2}$ and extract the imaginary part.}

Substitute into
\[
2\pi i \,\bigl[\text{Res}_1 + \text{Res}_2 \bigr]
= \frac{\pi i}{2}\,\Bigl(e^{i z_1} + e^{i z_2}\Bigr)
= \frac{\pi i}{2} \cdot 2\,e^{-\sqrt{2}}\cos(\sqrt{2})
= \pi\,i\,e^{-\sqrt{2}}\cos(\sqrt{2}).
\]
Since $\pi\,i\,e^{-\sqrt{2}}\cos(\sqrt{2})$ is purely imaginary, its imaginary part is
\[
\pi\,e^{-\sqrt{2}}\cos(\sqrt{2}).
\]
Hence
\[
\Im\Bigl\{2\pi i\,[\text{Res}_1+\text{Res}_2]\Bigr\}
= \boxed{\pi\,e^{-\sqrt{2}}\cos\bigl(\sqrt{2}\bigr)}.
\]

$f(z)=\frac{\log(z)}{(1+z^{2})^{2}}$
$\left\vert f(z) \right\vert \leq \frac{2\log(R)}{(R^{2}-1)^{2}}, \left\vert z \right\vert = R$ 


\section*{Bounding \(\left| f(z)\right|\) on \(\lvert z\rvert = R\)}

We have the function
\[
f(z) \;=\; \frac{\log(z)}{\bigl(1 + z^2\bigr)^2},
\]
and we want to show that on the (upper) semicircle \(\{\,z : |z| = R\}\) with \(R>1\),
\[
\bigl|f(z)\bigr|
\;\le\;
\frac{2\,\log(R)}{(R^2 - 1)^2}.
\]

\subsection*{1. Bound the denominator \(\lvert (1 + z^2)^2\rvert\)}
On the circle \(\lvert z\rvert = R\), we have \(\lvert z^2\rvert = R^2\). By the reverse triangle inequality,
\[
\lvert 1 + z^2 \rvert
\;\ge\;
\bigl|\lvert z^2\rvert - \lvert 1\rvert\bigr|
\;=\;
\lvert R^2 - 1\rvert
\;=\;
R^2 - 1
\quad (\text{because }R>1).
\]
Hence
\[
\lvert (1 + z^2)^2\rvert
\;=\;
\bigl\lvert 1 + z^2\bigr\rvert^2
\;\ge\;
(R^2 - 1)^2,
\]
so
\[
\frac{1}{\lvert (1 + z^2)^2\rvert}
\;\le\;
\frac{1}{(R^2 - 1)^2}.
\]

\subsection*{2. Bound \(\lvert \log(z)\rvert\)}
For \(z\) on the circle \(\lvert z\rvert = R\) (\(R>1\)), in the principal branch we write
\[
\log(z)
\;=\;
\ln\!\bigl|z\bigr| + i\,\arg(z)
\;=\;
\ln(R) + i\,\theta,
\quad
\theta \in [0, \pi]
\ (\text{upper semicircle}).
\]
Thus
\[
\lvert \log(z)\rvert
\;=\;
\sqrt{\bigl(\ln(R)\bigr)^2 + \theta^2}.
\]
Because \(0 \le \theta \le \pi\) and \(\ln(R)\ge 0\) for \(R>1\), we can use the inequality
\[
\sqrt{\bigl(\ln(R)\bigr)^2 + \theta^2}
\;\le\;
\ln(R) + \lvert \theta\rvert
\;\le\;
\ln(R) + \pi.
\]
For sufficiently large \(R\), we typically have \(\ln(R) \ge \pi\), which allows
\[
\ln(R) + \pi \;\le\; 2\,\ln(R).
\]
Hence, a common simplified bound is
\[
\lvert \log(z)\rvert 
\;\le\;
2\,\ln(R)
\quad (\text{for }R>1).
\]

\subsection*{3. Combine both bounds}
Putting it all together:
\[
\bigl|f(z)\bigr|
\;=\;
\left|\frac{\log(z)}{\bigl(1 + z^2\bigr)^2}\right|
\;\le\;
\frac{\lvert \log(z)\rvert}{\lvert (1 + z^2)^2\rvert}
\;\le\;
\frac{2\,\ln(R)}{(R^2 - 1)^2}.
\]
Therefore, on the contour \(\lvert z\rvert = R\) (with \(R>1\)),
\[
\boxed{
\lvert f(z)\rvert
\;\le\;
\frac{2\,\ln(R)}{(R^2 - 1)^2}.
}
\]


We begin with the equation
\[
-\frac{\pi}{2} \;+\; i\,\frac{\pi^2}{4}
\;=\;
\int_{\varepsilon}^R
\frac{\log x}{(1 + x^2)^2}\,dx
\;+\;
\int_{-R}^{-\varepsilon}
\frac{\log(-x) \;+\; i\,\pi}{(1 + x^2)^2}\,dx
\;+\; E,
\]
where \(\,E \to 0\,\) as \(\,\varepsilon \to 0\,\) and \(\,R \to \infty.\)

\paragraph{1. Take the real part.}

On the left-hand side, the real part is simply \(\,-\tfrac{\pi}{2}\).  
On the right-hand side, note that
\(\,\log(-x)\) is real (for \(x>0\)) and \(i\pi\) is purely imaginary, so the only real contributions in the integrals come from \(\,\log x\) and \(\,\log(-x)\).

\paragraph{2. Rewrite the second integral by substituting \(\,x = -t\).}

For the second integral
\[
\int_{-R}^{-\varepsilon} 
\frac{\log(-x) \;+\; i\,\pi}{(1 + x^2)^2}\,dx,
\]
let \(\,x = -t\).  Then \(\,dx = -\,dt\) and when \(x=-R\), \(t=R\); when \(x=-\varepsilon\), \(t=\varepsilon\). Hence
\[
\int_{-R}^{-\varepsilon} 
\frac{\log(-x) \;+\; i\,\pi}{(1 + x^2)^2}\,dx
\;=\;
\int_{\,R}^{\,\varepsilon} 
\frac{\log(t) \;+\; i\,\pi}{(1 + t^2)^2} 
\bigl(-\,dt\bigr)
\;=\;
-\int_{\varepsilon}^{R} 
\frac{\log(t) \;+\; i\,\pi}{(1 + t^2)^2}\,dt.
\]
When we take only the \emph{real part}, the term \(i\,\pi\) drops out.  Hence
\[
\mathrm{Re}\!\Bigl[
\int_{-R}^{-\varepsilon} 
\frac{\log(-x) \;+\; i\,\pi}{(1 + x^2)^2}\,dx
\Bigr]
\;=\;
-\int_{\varepsilon}^{R} 
\frac{\log(t)}{(1 + t^2)^2}\,dt.
\]

\paragraph{3. Combine with the first integral.}

Thus the sum of the real parts of the two integrals becomes
\[
\int_{\varepsilon}^{R} \frac{\log x}{(1 + x^2)^2}\,dx
\;+\;
\Bigl[\,-\int_{\varepsilon}^{R} \frac{\log(t)}{(1 + t^2)^2}\,dt\,\Bigr]
\;=\;
0.
\]
So, collecting real parts on the right-hand side, we are left with 
\[
\mathrm{Re}[\,E\,] 
\;\;\;\text{and nothing else from the integrals.}
\]
But by hypothesis, \(E \to 0\) as \(\varepsilon \to 0\) and \(R \to \infty\), so its real part also goes to \(0\).  

\paragraph{4. Equate real parts to find the integral from \(0\) to \(\infty\).}

In more careful detail, we notice that after taking the real part and doing the substitution, the two integrals \emph{together} can be recast as
\[
\int_{\varepsilon}^{R} \frac{\log x}{(1 + x^2)^2}\,dx 
\;+\;
\int_{\varepsilon}^{R} \frac{\log x}{(1 + x^2)^2}\,dx
\;+\;\text{(vanishing terms)}
\;=\;
2 \int_{\varepsilon}^{R} \frac{\log x}{(1 + x^2)^2}\,dx.
\]
Hence comparing with the real part on the left side,
\[
-\frac{\pi}{2} 
\;=\;
2 \int_{0}^{\infty} \frac{\log x}{(1 + x^2)^2}\,dx.
\]
Divide both sides by \(2\) to conclude
\[
\int_{0}^{\infty} \frac{\log x}{(1 + x^2)^2}\,dx
\;=\;
-\frac{\pi}{4},
\]
which is precisely the final statement.

\begin{align}
    \frac{1}{(2-\frac{1}{2i}(z-\frac{1}{z}))^{2}} \\[10pt] 
    \frac{1}{(\frac{1}{2iz})^{2}(4iz-z^{2}+1)} \\[10pt] 
    \frac{z}{(z^{2}-4iz-1)^{2}}
\end{align}
\end{document}
