\documentclass[12pt]{article}

% Packages
\usepackage[margin=1in]{geometry}
\usepackage{amsmath,amssymb,amsthm}
\usepackage{enumitem}
\usepackage{hyperref}
\usepackage{xcolor}
\usepackage{import}
\usepackage{xifthen}
\usepackage{pdfpages}
\usepackage{transparent}
\usepackage{listings}
\usepackage{tikz}
\usepackage{physics}
\usepackage{siunitx}
\usepackage{booktabs}
\usepackage{cancel}
  \usetikzlibrary{calc,patterns,arrows.meta,decorations.markings}


\DeclareMathOperator{\Log}{Log}
\DeclareMathOperator{\Arg}{Arg}

\lstset{
    breaklines=true,         % Enable line wrapping
    breakatwhitespace=false, % Wrap lines even if there's no whitespace
    basicstyle=\ttfamily,    % Use monospaced font
    frame=single,            % Add a frame around the code
    columns=fullflexible,    % Better handling of variable-width fonts
}

\newcommand{\incfig}[1]{%
    \def\svgwidth{\columnwidth}
    \import{./Figures/}{#1.pdf_tex}
}
\theoremstyle{definition} % This style uses normal (non-italicized) text
\newtheorem{solution}{Solution}
\newtheorem{proposition}{Proposition}
\newtheorem{problem}{Problem}
\newtheorem{lemma}{Lemma}
\newtheorem{theorem}{Theorem}
\newtheorem{remark}{Remark}
\newtheorem{note}{Note}
\newtheorem{definition}{Definition}
\newtheorem{example}{Example}
\newtheorem{corollary}{Corollary}
\theoremstyle{plain} % Restore the default style for other theorem environments
%

% Theorem-like environments
% Title information
\title{MATH-448 Practice Final Exam 2}
\author{Jerich Lee}
\date{\today}

\begin{document}

\maketitle
\pagebreak
\begin{problem}[Values and properties of elementary functions]
  \begin{enumerate}
    \item[(a)] Compute \emph{all} values of 
      \[
        (1-i\sqrt{3})^{\,\tfrac52}
      \]
      and express each value in the form $re^{i\theta}$ with
      $-\pi<\theta\le\pi$.
    \item[(b)] Find every branch value of 
      \[
        \Log\!\bigl(-\sqrt{2}+i\sqrt{2}\bigr)
      \]
      (principal value included).  List them explicitly as
      $\,\ln r + i\Theta_k\,$, $k\in\mathbb{Z}$.
  \end{enumerate}
  \end{problem}
  
  % ---------------------------------------------------------
  % 2. Isolated singularities
  % ---------------------------------------------------------
  \pagebreak
  \begin{problem}[Isolated singularities]
  For 
  \[
    f(z)=\frac{z^{2}}{(z+1)^{2}\sin z},
  \]
  \begin{enumerate}
    \item[(i)] locate and classify \emph{all} isolated singularities
          in $\mathbb{C}$ as removable, poles (state the order),
          or essential;
    \item[(ii)] compute the residue $\operatorname{Res}\!\bigl(f,0\bigr)$.
  \end{enumerate}
  \end{problem}
  
  % ---------------------------------------------------------
  % 3. Taylor / Laurent series
  % ---------------------------------------------------------
  \pagebreak
  \begin{problem}[Taylor / Laurent series]
  Find the Laurent expansion of
  \[
    \frac{\cos z-1}{z^{3}}
  \]
  about $z=0$ and write out the first four non‑zero terms explicitly.
  \end{problem}
  
  % ---------------------------------------------------------
  % 4. Closed‑contour integral
  % ---------------------------------------------------------
  \pagebreak
  \begin{problem}[Closed‑contour integral]
  Evaluate
  \[
    \oint_{\lvert z-2\rvert = 1}
          \frac{e^{z}}{(z-1)(z-3)}\,dz .
  \]
  \end{problem}
  
  % ---------------------------------------------------------
  % 5. Improper integral
  % ---------------------------------------------------------
  \pagebreak
  \begin{problem}[Improper integral]
  Show that
  \[
    I=\int_{0}^{\infty}\frac{x^{2}}{x^{4}+1}\,dx
  \]
  converges and determine its exact value.
  \end{problem}
  
  % ---------------------------------------------------------
  % 6. Number of zeros
  % ---------------------------------------------------------
  \pagebreak
  \begin{problem}[Counting zeros with Rouché / Argument Principle]
  Using an appropriate theorem, determine how many zeros of
  \[
    f(z)=z^{4}+4z+3
  \]
  lie in the upper half‑plane $\operatorname{Im}z>0$.
  \end{problem}
  
  % ---------------------------------------------------------
  % 7. Conformal map
  % ---------------------------------------------------------
  \pagebreak
  \begin{problem}[Conformal mapping]
  Construct an explicit conformal map $T$ that sends the vertical strip
  \[
    S=\bigl\{\,z\in\mathbb{C}:0<\operatorname{Re}z<1\,\bigr\}
  \]
  onto the open unit disc $\mathbb{D}=\{w:|w|<1\}$.
  (You may describe $T$ as a composition of elementary maps.)
  \end{problem}
  
  % ---------------------------------------------------------
  % 8. Theoretical question
  % ---------------------------------------------------------
  \pagebreak
  \begin{problem}[Entire functions of quadratic growth]
  Let $f$ be entire and suppose there exist constants $A,B>0$ such that
  \[
    |f(z)|\le A|z|^{2}+B
    \qquad\text{for all } z\in\mathbb{C}.
  \]
  Prove that $f$ is a polynomial of degree at most $2$.
  \end{problem}
\end{document}
