\documentclass[12pt]{article}

% Packages
\usepackage[margin=1in]{geometry}
\usepackage{amsmath,amssymb,amsthm}
\usepackage{enumitem}
\usepackage{hyperref}
\usepackage{xcolor}
\usepackage{import}
\usepackage{xifthen}
\usepackage{pdfpages}
\usepackage{transparent}
\usepackage{listings}
\usepackage{tikz}
\usepackage{physics}
\usepackage{siunitx}
\usepackage{booktabs}
\usepackage{cancel}
  \usetikzlibrary{calc,patterns,arrows.meta,decorations.markings}


\DeclareMathOperator{\Log}{Log}
\DeclareMathOperator{\Arg}{Arg}
\DeclareMathOperator{\arctanh}{arctanh}


\lstset{
    breaklines=true,         % Enable line wrapping
    breakatwhitespace=false, % Wrap lines even if there's no whitespace
    basicstyle=\ttfamily,    % Use monospaced font
    frame=single,            % Add a frame around the code
    columns=fullflexible,    % Better handling of variable-width fonts
}

\newcommand{\incfig}[1]{%
    \def\svgwidth{\columnwidth}
    \import{./Figures/}{#1.pdf_tex}
}
\theoremstyle{definition} % This style uses normal (non-italicized) text
\newtheorem{solution}{Solution}
\newtheorem{proposition}{Proposition}
\newtheorem{problem}{Problem}
\newtheorem{lemma}{Lemma}
\newtheorem{theorem}{Theorem}
\newtheorem{remark}{Remark}
\newtheorem{note}{Note}
\newtheorem{definition}{Definition}
\newtheorem{example}{Example}
\newtheorem{corollary}{Corollary}
\theoremstyle{plain} % Restore the default style for other theorem environments
%

% Theorem-like environments
% Title information
\title{MATH-448 Practice Final Exam 6}
\author{Jerich Lee}
\date{\today}

\begin{document}

\maketitle
\pagebreak
  
  %-----------------------------------------------------------
  \begin{problem}[Values and properties of elementary functions]\mbox{}\\[4pt]
  \begin{enumerate}[label=(\alph*),itemsep=6pt]
    \item Find \emph{all} values of $\bigl(\sqrt{2}+i\sqrt{2}\bigr)^{9/2}$ and write each value in the form $x+iy$ with $x,y\in\mathbb{R}$.
    \item Show that, for every $z\in\mathbb{C}\setminus\{1\}$,
          \[
            \arctanh z
            \;=\;
            \tfrac12\,
            \Log\!\Bigl(\tfrac{1+z}{1-z}\Bigr),
          \]
          where $\Log$ is the principal logarithm, and state the maximal simply‑connected domain on which the identity holds.
  \end{enumerate}
  \end{problem}
  
  \pagebreak
  %-----------------------------------------------------------
  \begin{problem}[Isolated singularities]\mbox{}\\[4pt]
  Consider
  \[
    f(z)=\frac{e^{\frac{1}{z-2}}}{(z+1)^{4}}.
  \]
  \begin{enumerate}[label=(\alph*),itemsep=6pt]
    \item Classify the isolated singularities of $f$ at $z=2$ and $z=-1$.
    \item Compute the residues $\operatorname*{Res}_{z=2}f(z)$ and $\operatorname*{Res}_{z=-1}f(z)$.
  \end{enumerate}
  \end{problem}
  
  \pagebreak
  %-----------------------------------------------------------
  \begin{problem}[Find Laurent series in an annulus]\mbox{}\\[4pt]
  Find the Laurent expansion
  \[
    \sum_{n=-\infty}^{\infty} a_{n}\,(z-1)^{n}
  \]
  of the function
  \[
    f(z)=2z^{-3}+\frac{1}{z+2},
  \]
  valid in the annulus \(1<\lvert z-1\rvert<3\), as follows:
  \begin{enumerate}[label=(\alph*),itemsep=6pt]
    \item Find a suitable series for \(z^{-1}\) that is valid when \(\lvert z-1\rvert>1\).
    \item Differentiate your result from part (a) term‑by‑term (twice) to obtain a series for \(2z^{-3}\).
    \item Find a suitable series for \((z+2)^{-1}\) that is valid when \(\lvert z-1\rvert<3\).
    \item Combine the results of parts (b) and (c) to write the Laurent expansion of \(f\) in the stated annulus and express the general coefficient \(a_{n}\) explicitly in terms of \(n\).
  \end{enumerate}
  \end{problem}
  
  \pagebreak
  %-----------------------------------------------------------
  \begin{problem}[Evaluate an integral over a closed contour]\mbox{}\\[4pt]
  Evaluate
  \[
    \oint_{\lvert z-2\rvert=3}\frac{z^{2}}{(z^{2}+1)(z-1)^{2}}\,dz
  \]
  using the residue theorem.
  \end{problem}
  
  \pagebreak
  %-----------------------------------------------------------
  \begin{problem}[Evaluate an improper integral]\mbox{}\\[4pt]
  Show that
  \[
    I=\int_{0}^{\infty}\frac{x\sin(2x)}{(x^{2}+1)^{2}}\,dx
  \]
  converges and compute its exact value.
  \end{problem}
  
  \pagebreak
  %-----------------------------------------------------------
  \begin{problem}[Find the number of zeros of a function]\mbox{}\\[4pt]
  Let
  \[
    F(z)=z^{7}+3z^{3}+5.
  \]
  Use Rouché’s Theorem to determine how many zeros of $F$ lie strictly inside the unit circle \(\lvert z\rvert=1\).
  \end{problem}
  
  \pagebreak
  %-----------------------------------------------------------
  \begin{problem}[Find a conformal map]\mbox{}\\[4pt]
  Construct an explicit conformal map \(T\) that sends the horizontal strip
  \[
    S=\{z:0<\operatorname{Im}z<\pi\}
  \]
  onto the open unit disc \(\mathbb{D}=\{w:\lvert w\rvert<1\}\).  
  Describe each intermediate mapping you use.
  \end{problem}
  
  \pagebreak
  %-----------------------------------------------------------
  \begin{problem}[Schwarz‑type estimate]\mbox{}\\[4pt]
  Let \(f\) be analytic on the unit disc \(\mathbb{D}=\{z:\lvert z\rvert<1\}\) and suppose
  \[
    f(0)=0,\qquad f'(0)=0,\qquad 
    \lvert f(z)\rvert\;\le\;\lvert z\rvert^{2}\quad\text{for all }z\in\mathbb{D}.
  \]
  \begin{enumerate}[label=(\alph*),itemsep=6pt]
    \item Prove that \(\lvert f(z)\rvert\le\lvert z\rvert^{2}\) indeed holds (state the classical result you invoke).
    \item Show that \(\lvert f''(0)\rvert\le 2\).
    \item If there exists \(z_{0}\neq0\) with \(\lvert f(z_{0})\rvert=\lvert z_{0}\rvert^{2}\), prove that \(f(z)=c z^{2}\) for some constant \(c\) with \(\lvert c\rvert=1\).
  \end{enumerate}
  (Hint: Apply the Schwarz Lemma to \(g(z)=\tfrac{f(z)}{z^{2}}\).)
  \end{problem}
\end{document}
