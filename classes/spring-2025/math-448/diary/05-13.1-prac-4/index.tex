\documentclass[12pt]{article}

% Packages
\usepackage[margin=1in]{geometry}
\usepackage{amsmath,amssymb,amsthm}
\usepackage{enumitem}
\usepackage{hyperref}
\usepackage{xcolor}
\usepackage{import}
\usepackage{xifthen}
\usepackage{pdfpages}
\usepackage{transparent}
\usepackage{listings}
\usepackage{tikz}
\usepackage{physics}
\usepackage{siunitx}
\usepackage{booktabs}
\usepackage{cancel}
  \usetikzlibrary{calc,patterns,arrows.meta,decorations.markings}


\DeclareMathOperator{\Log}{Log}
\DeclareMathOperator{\Arg}{Arg}

\lstset{
    breaklines=true,         % Enable line wrapping
    breakatwhitespace=false, % Wrap lines even if there's no whitespace
    basicstyle=\ttfamily,    % Use monospaced font
    frame=single,            % Add a frame around the code
    columns=fullflexible,    % Better handling of variable-width fonts
}

\newcommand{\incfig}[1]{%
    \def\svgwidth{\columnwidth}
    \import{./Figures/}{#1.pdf_tex}
}
\theoremstyle{definition} % This style uses normal (non-italicized) text
\newtheorem{solution}{Solution}
\newtheorem{proposition}{Proposition}
\newtheorem{problem}{Problem}
\newtheorem{lemma}{Lemma}
\newtheorem{theorem}{Theorem}
\newtheorem{remark}{Remark}
\newtheorem{note}{Note}
\newtheorem{definition}{Definition}
\newtheorem{example}{Example}
\newtheorem{corollary}{Corollary}
\theoremstyle{plain} % Restore the default style for other theorem environments
%

% Theorem-like environments
% Title information
\title{MATH-448 Practice Final Exam 4}
\author{Jerich Lee}
\date{\today}

\begin{document}

\maketitle
\pagebreak
  
  \begin{problem}[Values and properties of elementary functions]
  \begin{enumerate}[label=(\alph*),itemsep=6pt]
    \item Compute \emph{all} values of \((\sqrt{2}-i)^{20}\) and list them in the form \(x+iy\) with \(x,y\in\mathbb{R}\).
    \item Determine the principal value of \(\displaystyle\exp\!\bigl[\tfrac12\Log\!\bigl(3+4i\bigr)\bigr]\).
    \item Show that, for every \(z\in\mathbb{C}\setminus\{\pm i\}\),
          \[
            \arccos z \;=\; \frac{\pi}{2}-\arcsin z
            \quad\text{where}\quad
            \arcsin z \;=\; -\,i\Log\!\bigl(iz+\sqrt{1-z^{2}}\bigr),
          \]
          and \(\Log\) denotes the principal branch of the complex logarithm.
  \end{enumerate}
  \end{problem}
  
  \pagebreak
  
  \begin{problem}[Isolated singularities]
  Define
  \[
    f(z)\;=\;\frac{(z^{2}+1)\,e^{\tfrac{1}{z-1}}}{(z+2)^{3}}.
  \]
  \begin{enumerate}[label=(\alph*),itemsep=6pt]
    \item Classify the isolated singularities of \(f\) at \(z=1\) and \(z=-2\).
    \item Compute \(\operatorname*{Res}_{z=1}f(z)\) and \(\operatorname*{Res}_{z=-2}f(z)\).
  \end{enumerate}
  \end{problem}
  
  \pagebreak
  
  \begin{problem}[Find Taylor/Laurent series]
  Expand
  \[
    h(z)\;=\;z\cot z
  \]
  in a Laurent series about \(z=0\) and write the first four non‑zero terms.
  \end{problem}
  
  \pagebreak
  
  \begin{problem}[Evaluate an integral over a closed contour]
  Evaluate, using residues,
  \[
    \oint_{\lvert z\rvert = 3}\frac{z^{2}}{(z^{2}+4)(z-1)}\,dz.
  \]
  \end{problem}
  
  \pagebreak
  
  \begin{problem}[Evaluate an improper integral]
  Show that
  \[
    I \;=\;\int_{0}^{\infty}\frac{x}{x^{4}+5x^{2}+6}\,dx
  \]
  converges and compute its exact value.
  \end{problem}
  
  \pagebreak
  
  \begin{problem}[Find the number of zeros of a function]
  Let
  \[
    F(z)\;=\;z^{6}+z^{3}+1.
  \]
  Using the Argument Principle or Rouché’s Theorem, determine how many zeros of \(F\) lie in the annulus \(1<\lvert z\rvert<2\).
  \end{problem}
  
  \pagebreak
  
  \begin{problem}[Find a conformal map]
  Construct an explicit conformal map that sends the sector
  \[
    S=\bigl\{\,z : 0<\Arg z<\tfrac{\pi}{3}\bigr\}
  \]
  onto the upper half‑plane \(\operatorname{Im}w>0\).  
  Describe each intermediate mapping used in your construction.
  \end{problem}
  
  \pagebreak
  
  \begin{problem}[Cauchy estimates for a Laurent series]
  Suppose \(f\) is analytic on the punctured disc \(0<\lvert z\rvert<R\) and has Laurent expansion
  \[
    f(z)=\sum_{n=-\infty}^{\infty}a_{n}z^{\,n}, \qquad 0<\lvert z\rvert<R.
  \]
  Assume there are constants \(M>0\) and \(\alpha>0\) such that
  \[
    \lvert f(z)\rvert \;\le\; \frac{M}{\lvert z\rvert^{\alpha}}
    \quad\text{whenever }0<\lvert z\rvert<\frac{R}{2}.
  \]
  \begin{enumerate}[label=(\alph*),itemsep=6pt]
    \item Show, using the Cauchy estimates for Laurent coefficients, that \(a_{n}=0\) whenever \(n<-\alpha\).
    \item Conclude that \(f\) extends to a meromorphic function on \(\lvert z\rvert<R\) with at most a pole of order \(\lceil\alpha\rceil\) at \(z=0\).
  \end{enumerate}
  \end{problem}
\end{document}
