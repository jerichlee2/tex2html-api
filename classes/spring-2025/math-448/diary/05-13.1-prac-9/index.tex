\documentclass[12pt]{article}

% Packages
\usepackage[margin=1in]{geometry}
\usepackage{amsmath,amssymb,amsthm}
\usepackage{enumitem}
\usepackage{hyperref}
\usepackage{xcolor}
\usepackage{import}
\usepackage{xifthen}
\usepackage{pdfpages}
\usepackage{transparent}
\usepackage{listings}
\usepackage{tikz}
\usepackage{physics}
\usepackage{siunitx}
\usepackage{booktabs}
\usepackage{cancel}
  \usetikzlibrary{calc,patterns,arrows.meta,decorations.markings}


\DeclareMathOperator{\Log}{Log}
\DeclareMathOperator{\Arg}{Arg}


\lstset{
    breaklines=true,         % Enable line wrapping
    breakatwhitespace=false, % Wrap lines even if there's no whitespace
    basicstyle=\ttfamily,    % Use monospaced font
    frame=single,            % Add a frame around the code
    columns=fullflexible,    % Better handling of variable-width fonts
}

\newcommand{\incfig}[1]{%
    \def\svgwidth{\columnwidth}
    \import{./Figures/}{#1.pdf_tex}
}
\theoremstyle{definition} % This style uses normal (non-italicized) text
\newtheorem{solution}{Solution}
\newtheorem{proposition}{Proposition}
\newtheorem{problem}{Problem}
\newtheorem{lemma}{Lemma}
\newtheorem{theorem}{Theorem}
\newtheorem{remark}{Remark}
\newtheorem{note}{Note}
\newtheorem{definition}{Definition}
\newtheorem{example}{Example}
\newtheorem{corollary}{Corollary}
\theoremstyle{plain} % Restore the default style for other theorem environments
%

% Theorem-like environments
% Title information
\title{MATH 448 Practice Final Exam 9}
\author{Jerich Lee}
\date{\today}

\begin{document}

\maketitle
\pagebreak
%%%%%%%%%%%%%%%%%%%%%%%%%%%%%%%%%%%%%%%%%%%%%%%%%%%%%%%%%%%%%%%%%%%%%%%%
%  Complex Analysis – PRACTICE FINAL (Brand‑new version)
%  Eight problems, one per required topic
%%%%%%%%%%%%%%%%%%%%%%%%%%%%%%%%%%%%%%%%%%%%%%%%%%%%%%%%%%%%%%%%%%%%%%%%

  
  \bigskip
  %-----------------------------------------------------------------------
  % 1. Values / elementary functions
  %-----------------------------------------------------------------------
  \begin{problem}[Elementary functions]
  Compute
  \[
  \bigl(\sqrt{2}-i\bigr)^{75}
  \quad\text{and}\quad
  \Log\!\bigl(-2+i\sqrt{3}\bigr),
  \]
  where $\Log$ denotes the principal branch of the complex logarithm.
  Express both answers in the form $x+iy$ with real $x,y$.
  \end{problem}
  \pagebreak
  %-----------------------------------------------------------------------
  % 2. Isolated singularities
  %-----------------------------------------------------------------------
  \begin{problem}[Isolated singularities]
  Consider
  \[
  f(z)=\frac{e^{1/z}}{z^{2}(z^{2}+1)}.
  \]
  \begin{enumerate}[label=(\alph*)]
    \item Classify \emph{every} isolated singularity of $f$ in~$\mathbb{C}$
          as removable, a pole (specify order) or essential.
    \item Compute $\Res\bigl(f,0\bigr)$ and
          $\displaystyle \Res\!\bigl(f,i\bigr)$.
    \item Give the first three non‑zero terms of the Laurent series of
          $f$ about $z=0$.
  \end{enumerate}
  \end{problem}
  
  \pagebreak
  %-----------------------------------------------------------------------
  % 3. Taylor / Laurent series
  %-----------------------------------------------------------------------
  \begin{problem}[Power‑series expansion]        % harder than before
  Find the Taylor series of
  \[
  g(z)=\frac{\log(1+z^{\,3})}{(1-z)^{4}}
  \]
  about $z=0$, and write a closed‑form expression for the general
  coefficient $a_{n}$ in
  $g(z)=\displaystyle\sum_{n=0}^{\infty}a_{n}z^{n}$.
  State the radius of convergence.
  (\emph{Hint: use the Cauchy product of the series for $\log(1+z^{3})$
  and $(1-z)^{-4}$.})
  \end{problem}
  
  \pagebreak
  %-----------------------------------------------------------------------
  % 4. Closed contour integral
  %-----------------------------------------------------------------------
  \begin{problem}[Contour integral]
  Evaluate
  \[
  \oint_{\lvert z\rvert = 2}\frac{z^{2}}{(z^{2}+1)^{3}}\;dz
  \]
  by residue calculus.  (\emph{Note: all singularities lie inside the
  circle $\lvert z\rvert=2$.})
  \end{problem}
  
  \pagebreak
  %-----------------------------------------------------------------------
  % 5. Improper integral
  %-----------------------------------------------------------------------
  \begin{problem}[Improper integral]
  Using a suitable contour in the complex plane, show that
  \[
  \int_{-\infty}^{+\infty}\frac{x^{2}\,dx}{x^{4}+1}\;=\;\frac{\pi}{\sqrt{2}}.
  \]
  (\emph{You may quote without proof the standard residues of
  $\displaystyle \frac{1}{x^{4}+1}$ or compute them directly.})
  \end{problem}
  
  \pagebreak
  %-----------------------------------------------------------------------
  % 6. Zeros and poles
  %-----------------------------------------------------------------------
  \begin{problem}[Counting zeros]
  Determine the number of zeros (counting multiplicities) of the
  polynomial
  \[
  p(z)=z^{6}+3z^{3}+6
  \]
  inside the annulus $1<\lvert z\rvert<2$.
  (\emph{Hint: apply Rouché’s theorem on the two circles
  $\lvert z\rvert=1$ and $\lvert z\rvert=2$.})
  \end{problem}
  
  \pagebreak

  %-----------------------------------------------------------------------
  % 7. Conformal mapping
  %-----------------------------------------------------------------------
  \begin{problem}[Conformal map]
  Find an explicit conformal mapping that sends the infinite strip
  \[
  S=\{\,z\in\mathbb{C} : 0<\operatorname{Im}z<\pi\}
  \]
  onto the upper half‑plane
  $\{w\in\mathbb{C}:\operatorname{Im}w>0\}$ and carries the point $z=i$ to
  $w=i$.  Describe what happens to the horizontal lines
  $\operatorname{Im}z=\text{constant}$ under your map.
  \end{problem}
  \pagebreak 
  %-----------------------------------------------------------------------
  % 8. Theoretical question
  %-----------------------------------------------------------------------
  \begin{problem}[Schwarz–Pick lemma (theoretical)]
  Let $f$ be analytic on the open unit disc $\mathcal{D}=\{\,z:\lvert z\rvert<1\}$
  with $f(\mathcal{D})\subset\mathcal{D}$.  Show that for all $z_{1},z_{2}\in\mathcal{D}$
  \[
  \left|\frac{f(z_{1})-f(z_{2})}{1-\overline{f(z_{2})}\,f(z_{1})}\right|
  \;\le\;
  \left|\frac{z_{1}-z_{2}}{1-\overline{z_{2}}\,z_{1}}\right|.
  \]
  Deduce that
  $\displaystyle\lvert f'(0)\rvert\le 1-\lvert f(0)\rvert^{2}$.
  State the condition for equality and prove your assertions.
  \end{problem}
  \pagebreak
\end{document}
