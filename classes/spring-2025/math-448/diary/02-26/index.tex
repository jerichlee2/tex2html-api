\documentclass[12pt]{article}

% Packages
\usepackage[margin=1in]{geometry}
\usepackage{amsmath,amssymb,amsthm}
\usepackage{enumitem}
\usepackage{hyperref}
\usepackage{xcolor}
\usepackage{import}
\usepackage{xifthen}
\usepackage{pdfpages}
\usepackage{transparent}
\usepackage{listings}


\lstset{
    breaklines=true,         % Enable line wrapping
    breakatwhitespace=false, % Wrap lines even if there's no whitespace
    basicstyle=\ttfamily,    % Use monospaced font
    frame=single,            % Add a frame around the code
    columns=fullflexible,    % Better handling of variable-width fonts
}

\newcommand{\incfig}[1]{%
    \def\svgwidth{\columnwidth}
    \import{./Figures/}{#1.pdf_tex}
}
\theoremstyle{definition} % This style uses normal (non-italicized) text
\newtheorem{solution}{Solution}
\newtheorem{proposition}{Proposition}
\newtheorem{problem}{Problem}
\newtheorem{lemma}{Lemma}
\newtheorem{theorem}{Theorem}
\newtheorem{remark}{Remark}
\newtheorem{note}{Note}
\newtheorem{definition}{Definition}
\newtheorem{example}{Example}
\theoremstyle{plain} % Restore the default style for other theorem environments
%

% Theorem-like environments
% Title information
\title{\today}
\author{Jerich Lee}
\date{\today}

\begin{document}

\maketitle
\textbf{Problem Statement:} 
Describe and sketch the image of the line 
\[
L = \{z \in \mathbb{C} : \Re(z) = 1\}
\]
under the map 
\[
f: \mathbb{C}\setminus\{0\} \to \mathbb{C}, \quad f(z) = \frac{1}{z}.
\]

\textbf{Solution (Step-by-step):}

\textbf{1. Parametrize the line \(\Re(z) = 1\).}

Any point \(z\) on the line \(L\) can be written as 
\[
z = 1 + i\,y, \quad \text{where } y \in \mathbb{R}.
\]
Hence,
\[
L = \{1 + i y : y \in \mathbb{R}\}.
\]

\textbf{2. Apply the map \(f(z) = 1/z\) to this parametrization.}

For \(z = 1 + i\,y\), we compute
\[
f(1 + i\,y) \;=\; \frac{1}{1 + i\,y}.
\]

\textbf{3. Rationalize the denominator.}

To simplify, multiply numerator and denominator by the complex conjugate of the denominator:
\[
\frac{1}{1 + i\,y} 
\;=\; 
\frac{1}{1 + i\,y} \cdot \frac{1 - i\,y}{1 - i\,y}
\;=\;
\frac{\,1 - i\,y\,}{(1 + i\,y)(1 - i\,y)}.
\]
Note that
\[
(1 + i\,y)(1 - i\,y) \;=\; 1 + y^2.
\]
Thus,
\[
f(1 + i\,y) 
\;=\; 
\frac{\,1 - i\,y\,}{1 + y^2}.
\]

\textbf{4. Separate into real and imaginary parts.}

Write \(f(1 + i\,y) = x + i\,u\). Then
\[
x \;=\; \Re\!\Bigl(\frac{1 - i\,y}{1 + y^2}\Bigr) 
= \frac{1}{1 + y^2},
\quad
u \;=\; \Im\!\Bigl(\frac{1 - i\,y}{1 + y^2}\Bigr)
= \frac{-y}{1 + y^2}.
\]
Hence, the image point \(w = f(z)\) in the complex plane is
\[
w = x + i\,u = \frac{1}{1 + y^2} \;-\; i \,\frac{y}{1 + y^2}.
\]

\textbf{5. Find the geometric locus of these points \((x,u)\).}

We want to eliminate the parameter \(y\). From
\[
x = \frac{1}{1 + y^2},
\]
we see 
\[
1 + y^2 = \frac{1}{x} 
\quad \Longrightarrow \quad
y^2 = \frac{1}{x} - 1.
\]
Also,
\[
u = -\frac{y}{1 + y^2} = -y \, x
\]
because multiplying top and bottom of \(\frac{-y}{1+y^2}\) by \(x = \frac{1}{1+y^2}\) gives
\[
u = -\,y \, \Bigl(\frac{1}{1 + y^2}\Bigr) = -\,y \,x.
\]
Squaring \(u\):
\[
u^2 = y^2 \, x^2.
\]
But \(y^2 = \tfrac{1}{x} - 1\). Thus,
\[
u^2 = \bigl(\tfrac{1}{x} - 1\bigr)\,x^2 = x \,(1 - x),
\]
since
\[
\bigl(\tfrac{1}{x} - 1\bigr)\,x^2 
= x^2 \cdot \tfrac{1}{x} - x^2
= x - x^2.
\]
Therefore,
\[
u^2 = x - x^2 
\quad \Longrightarrow \quad
x^2 + u^2 = x.
\]
Rewriting,
\[
x^2 - x + u^2 = 0 
\quad \Longrightarrow \quad
x^2 - x + \tfrac{1}{4} + u^2 = \tfrac{1}{4},
\]
which gives
\[
\bigl(x - \tfrac{1}{2}\bigr)^2 + u^2 = \bigl(\tfrac{1}{2}\bigr)^2.
\]

\textbf{6. Interpret this locus: it is a circle.}

The equation
\[
\bigl(x - \tfrac{1}{2}\bigr)^2 + u^2 = \bigl(\tfrac{1}{2}\bigr)^2
\]
represents a circle in the \((x,u)\)-plane (i.e.\ the image plane) with center 
\(\bigl(\tfrac{1}{2},\,0\bigr)\) and radius \(\tfrac{1}{2}\).

\textbf{7. Determine which points on the circle are attained.}

As \(y\) varies over all real numbers:
\[
x = \frac{1}{1 + y^2} \quad \text{takes values in } (0,1].
\]
- When \(y = 0\), we get \(x = 1\) and \(u = 0\). This corresponds to the point \((1,0)\) on the circle, i.e.\ \(w = 1\).
- As \(|y|\to \infty\), we have \(x \to 0\) and \(u \to 0\). So the image approaches \((0,0)\) on the circle. 

\emph{However}, there is no finite \(y\) for which \(x = 0\). Hence the point \((0,0)\) (the origin in the \(w\)-plane) is a limit point that is \emph{not actually attained} by \(f(1 + i\,y)\). Thus:

\[
\text{Image of the line } \{ \Re(z) = 1 \}
\;=\;
\Bigl\{
w : (x - \tfrac{1}{2})^2 + u^2 = \bigl(\tfrac{1}{2}\bigr)^2
\Bigr\}
\setminus
\{0\}.
\]
In words, it is the circle of radius \(1/2\) centered at \(\bigl(\tfrac{1}{2},0\bigr)\), \emph{minus} the point \(w = 0\). 

\textbf{8. Sketch.}

\begin{itemize}
\item Draw the circle centered at \(\bigl(\tfrac{1}{2},\,0\bigr)\) with radius \(\tfrac{1}{2}\). 
\item This circle passes through the points \((0,0)\) and \((1,0)\) on the real axis.
\item The image is the entire circle \emph{except} that \((0,0)\) (the origin) is not attained. It is only approached as \(|y|\to\infty\).
\end{itemize}

\[
\boxed{
\text{The image is the circle } (x-\tfrac{1}{2})^2 + u^2 = \bigl(\tfrac{1}{2}\bigr)^2 \text{ in the } w\text{-plane, minus the point } w=0.
}
\]
\textbf{Problem (b).} Find the power series expansion
\[
\frac{1}{(z - 5)^3}
\;=\;
\sum_{n=0}^{\infty} a_n \,(z+2)^n
\quad
\text{and determine for which } z \text{ it converges.}
\]

\textbf{Solution:}

\textbf{Step 1: Rewrite the function in terms of } \(\,w = z + 2.\)

Set \(w = z + 2\). Then
\[
z - 5 \;=\; (z + 2) - 7 \;=\; w - 7.
\]
Thus,
\[
\frac{1}{(z - 5)^3}
\;=\;
\frac{1}{(w - 7)^3}.
\]

\textbf{Step 2: Factor out } \(7\) \textbf{ to prepare for a binomial expansion.}

\[
\frac{1}{(w - 7)^3}
\;=\;
\frac{1}{7^3}\,\frac{1}{\Bigl(1 - \tfrac{w}{7}\Bigr)^3}.
\]
Hence,
\[
\frac{1}{(z - 5)^3}
\;=\;
\frac{1}{7^3}\,\frac{1}{\Bigl(1 - \tfrac{w}{7}\Bigr)^3}.
\]

\textbf{Step 3: Use the generalized binomial series (or known formula) for } 
\(\,(1 - x)^{-m}.\)

Recall the standard series expansion for \((1 - x)^{-m}\), valid for \(\lvert x\rvert < 1\):
\[
\frac{1}{(1 - x)^m}
\;=\;
\sum_{n=0}^{\infty} \binom{n+m-1}{m-1}\,x^n,
\]
where \(\displaystyle \binom{n+m-1}{m-1}\) is a binomial coefficient. 

In our case, \(m = 3\). Thus
\[
\frac{1}{(1 - x)^3}
\;=\;
\sum_{n=0}^{\infty} \binom{n+2}{2}\, x^n
\;=\;
\sum_{n=0}^{\infty} \frac{(n+1)(n+2)}{2}\, x^n.
\]

\textbf{Step 4: Apply this to } \(x = \tfrac{w}{7}.\)

So
\[
\frac{1}{(1 - \tfrac{w}{7})^3}
\;=\;
\sum_{n=0}^{\infty} \binom{n+2}{2} 
\biggl(\tfrac{w}{7}\biggr)^n.
\]
Multiplying by \(\tfrac{1}{7^3}\):
\[
\frac{1}{(w-7)^3}
\;=\;
\frac{1}{7^3}\;
\sum_{n=0}^{\infty}
\binom{n+2}{2}\,
\biggl(\tfrac{w}{7}\biggr)^n
\;=\;
\sum_{n=0}^{\infty}
\binom{n+2}{2}\,
\frac{w^n}{7^{n+3}}.
\]

\textbf{Step 5: Substitute back } \(w = z + 2.\)

Therefore,
\[
\frac{1}{(z - 5)^3}
\;=\;
\sum_{n=0}^{\infty}
\binom{n+2}{2}
\frac{(z+2)^n}{7^{n+3}}
\;=\;
\sum_{n=0}^{\infty}
\frac{(n+1)(n+2)}{2}\;
\frac{(z+2)^n}{7^{n+3}}.
\]
Hence the coefficients \(a_n\) in the power series expansion
\[
\sum_{n=0}^{\infty} a_n\,(z+2)^n
\]
are
\[
a_n
\;=\;
\frac{(n+1)(n+2)}{2}\,\frac{1}{7^{n+3}},
\quad n=0,1,2,\dots
\]

\textbf{Step 6: Determine the region of convergence.}

Since we have expanded around the point \(z=-2\), the radius of convergence is determined by the distance from \(-2\) to the nearest singularity of the function \(\frac{1}{(z-5)^3}\). The only singularity is at \(z=5\). The distance between \(-2\) and \(5\) is \(\lvert -2 - 5\rvert = 7\). Therefore, the series converges for
\[
\lvert z + 2\rvert < 7.
\]

\[
\boxed{
\frac{1}{(z-5)^3}
\;=\;
\sum_{n=0}^{\infty}
\frac{(n+1)(n+2)}{2\,7^{\,n+3}}\,(z+2)^n
\quad
\text{for}
\quad
\lvert z+2\rvert < 7.
}
\]

\textbf{Geometric interpretation:} The largest circle (in the complex plane) centered at \(z=-2\) that does not contain the pole \(z=5\) has radius \(7\). Thus the series expansion about \(z=-2\) is valid exactly within that circle, i.e.\ for \(\lvert z+2\rvert < 7\).
$\sum_{i=0}^{\infty} a_n(z-z_0)^{n}$ 


\textbf{Derivative of a Power Series.}

Let 
\[
f(z) \;=\; \sum_{n=0}^{\infty} a_n \,\bigl(z - z_0\bigr)^n.
\]
We can differentiate term-by-term (within the radius of convergence) to obtain
\[
f'(z)
\;=\;
\sum_{n=0}^{\infty} \frac{d}{dz} 
\Bigl[ a_n \,\bigl(z - z_0\bigr)^n \Bigr].
\]
Since
\[
\frac{d}{dz}\Bigl[(z - z_0)^n\Bigr]
=
n\,(z - z_0)^{\,n-1},
\]
each term with \(n=0\) vanishes (the constant term). For \(n \ge 1\),
\[
\frac{d}{dz}\Bigl[a_n (z - z_0)^n\Bigr]
=
a_n \cdot n\, (z - z_0)^{n-1}.
\]
Hence
\[
f'(z)
\;=\;
\sum_{n=1}^{\infty}
n\,a_n\,\bigl(z - z_0\bigr)^{\,n-1}.
\]

\[
\boxed{
\text{If } f(z) = \sum_{n=0}^{\infty} a_n (z - z_0)^n,
\quad\text{then}\quad
f'(z) = \sum_{n=1}^{\infty} n\,a_n \,(z - z_0)^{n-1}.
}
\]

\textbf{Problem:} Evaluate
\[
\int_{\lvert z\rvert = 1} \frac{z}{z^2 - 2}\,dz.
\]

\textbf{Solution:}

\textbf{Step 1: Identify the singularities of the integrand.}

The integrand is 
\[
f(z) \;=\; \frac{z}{z^2 - 2}.
\]
The denominator \(z^2 - 2 = 0\) gives \(z = \pm \sqrt{2}\). Thus the singularities (poles) are at \(z = \sqrt{2}\) and \(z = -\sqrt{2}\).

\textbf{Step 2: Check whether these singularities lie inside the contour \(\lvert z\rvert=1\).}

Both \(\pm \sqrt{2}\) have magnitude \(\sqrt{2}\approx 1.414\). Hence 
\(\lvert \pm \sqrt{2}\rvert = \sqrt{2} > 1.\)
So neither pole lies within the unit circle \(\lvert z\rvert=1\).

\textbf{Step 3: Apply Cauchy's Theorem.}

Since \(f(z)\) is analytic on and inside the circle \(\lvert z\rvert=1\) (the only singularities are outside), Cauchy's Theorem states that the contour integral of an analytic function around a closed curve is \(0\). Hence,
\[
\int_{\lvert z\rvert=1} \frac{z}{z^2 - 2}\,dz
\;=\;
0.
\]

\[
\boxed{0}
\]
\textbf{Problem.} Evaluate
\[
\int_{\lvert z\rvert = 1} \frac{z}{z^2 - 2}\,dz.
\]

\textbf{Step-by-Step Solution using Cauchy’s Integral Formula:}

\textbf{1. Identify the singularities.}

The integrand
\[
f(z) \;=\; \frac{z}{z^2 - 2}
\]
has singularities where the denominator vanishes: \(z^2 - 2 = 0\). Hence the poles are
\[
z = \sqrt{2}
\quad\text{and}\quad
z = -\sqrt{2}.
\]
Since \(\sqrt{2} \approx 1.414\), both poles lie \emph{outside} the unit circle \(\lvert z\rvert=1\).

\textbf{2. Recall the relevant form of Cauchy’s Integral Formula / Residue Theorem.}

- \textbf{Cauchy's Integral Formula} (basic form) says that if \(g\) is analytic inside and on a simple closed contour \(\Gamma\), then
\[
\int_{\Gamma} \frac{g(z)}{z-z_0}\,dz 
\;=\;
2\pi i \,g(z_0),
\quad
\text{provided } z_0 \text{ is inside } \Gamma.
\]
- \textbf{Residue Theorem} (a direct extension) says the integral of a meromorphic function around a closed contour is \(2\pi i\) times the sum of its residues at poles lying inside the contour.

\textbf{3. Rewrite or decompose the integrand to expose possible residues.}

We can do a partial-fraction decomposition (though here it is very quick to see there will be no poles inside):

\[
\frac{z}{z^2 - 2}
\;=\;
\frac{z}{(z-\sqrt{2})(z+\sqrt{2})}
\;=\;
\frac{A}{z-\sqrt{2}} \;+\; \frac{B}{z+\sqrt{2}}
\]
for some constants \(A\) and \(B\). Regardless of \(A\) and \(B\), both denominators \(\bigl(z-\sqrt{2}\bigr)\) and \(\bigl(z+\sqrt{2}\bigr)\) vanish only at \(z=\pm\sqrt{2}\). These are both outside \(\lvert z\rvert=1\).

\textbf{4. Conclude using the Residue Theorem (or Cauchy’s Integral Formula).}

Since neither \(\sqrt{2}\) nor \(-\sqrt{2}\) lies within the unit circle, there are \emph{no poles} of \(f(z)\) inside \(\lvert z\rvert=1\). Thus, by the Residue Theorem,
\[
\int_{\lvert z\rvert=1} \frac{z}{z^2 - 2}\,dz
\;=\;
2\pi i \times \bigl(\text{sum of residues of }f \text{ inside }|z|=1\bigr).
\]
But there are \emph{no} poles inside, so the sum of residues inside is \(0\). Hence
\[
\int_{\lvert z\rvert=1} \frac{z}{z^2 - 2}\,dz 
\;=\;
0.
\]

\textbf{5. Alternatively, use Cauchy’s Integral Formula directly for ``analytic inside'' argument.}

Observe that \(\displaystyle \frac{z}{z^2 - 2}\) is analytic for \(\lvert z\rvert < 1\) (no singularities there). By Cauchy’s Theorem (which itself can be seen as a special case of the integral formula with no poles inside),
\[
\int_{\lvert z\rvert=1} \frac{z}{z^2 - 2}\,dz \;=\; 0.
\]

\[
\boxed{ \text{The value of the integral is } 0. }
\]

\textbf{Problem:} Evaluate
\[
\int_{\lvert z\rvert=2} \frac{e^z}{z-3}\,dz.
\]

\textbf{Solution:}

\textbf{Step 1: Identify the singularity of the integrand.}

The integrand is
\[
f(z) = \frac{e^z}{z-3}.
\]
Since \(e^z\) is entire (no singularities), the only possible singularity arises from the factor \((z-3)\) in the denominator, giving a simple pole at \(z = 3\).

\textbf{Step 2: Determine whether the singularity is inside the contour \(\lvert z\rvert=2\).}

The circle \(\lvert z\rvert = 2\) is centered at \(0\) with radius \(2\). The point \(z=3\) clearly satisfies \(\lvert 3\rvert=3\), which is \emph{greater} than \(2\). Hence \(z=3\) lies \emph{outside} the contour \(\lvert z\rvert=2\).

\textbf{Step 3: Apply Cauchy's Theorem or the Residue Theorem.}

Because \(f(z)\) is analytic on and inside the circle \(\lvert z\rvert=2\) (its only pole at \(z=3\) is outside), it follows from Cauchy's Theorem that
\[
\int_{\lvert z\rvert=2} \frac{e^z}{z-3}\,dz 
\;=\; 0.
\]

\[
\boxed{0}
\]

\textbf{Problem.} Evaluate
\[
\int_{\lvert z\rvert=2} \frac{e^z}{z-3}\,dz.
\]

\textbf{Step-by-Step Solution using Cauchy’s Integral Formula:}

\textbf{1. Identify the singularity.}

The only singularity of
\[
f(z) \;=\; \frac{e^z}{z-3}
\]
occurs at \(z=3\), where the denominator vanishes.

\textbf{2. Check if \(z=3\) lies inside the contour \(\lvert z\rvert=2\).}

The contour \(\lvert z\rvert=2\) is the circle of radius \(2\) centered at the origin. Clearly,
\[
|3| = 3 \;>\; 2,
\]
so \(z=3\) is \emph{outside} the circle \(\lvert z\rvert=2\).

\textbf{3. Recall Cauchy’s Integral Formula.}

If \(g\) is analytic inside and on a simple closed contour \(\Gamma\) and \(z_0\) is a point \emph{inside} \(\Gamma\), then
\[
\int_{\Gamma} \frac{g(z)}{z - z_0}\,dz 
\;=\;
2\pi i \,g(z_0).
\]
However, if \(z_0\) is \emph{outside} \(\Gamma\), then the integral is \(0\) because there is no contribution from any singularity inside the contour.

\textbf{4. Apply Cauchy’s Integral Formula to our integrand.}

Set \(g(z) = e^z\). Then \(f(z) = \dfrac{g(z)}{z-3}\). The point \(z_0 = 3\) (the pole) is \emph{not} inside the circle \(\lvert z\rvert=2\). Therefore, by Cauchy’s Integral Formula,
\[
\int_{\lvert z\rvert=2} \frac{e^z}{z-3}\,dz
\;=\; 0
\]
since there is no singularity enclosed by \(\lvert z\rvert=2\).

\[
\boxed{0}
\]

\textbf{Geometric/Residue Interpretation:} 
The function \(e^z\) is entire (no singularities), and \(\frac{1}{z-3}\) has a simple pole at \(z=3\). Because that pole lies \emph{outside} the given contour, there are no residues to collect \emph{inside} \(\lvert z\rvert=2\). Hence, the integral vanishes.
$\int_{}^{\left\vert z \right\vert =2}  \,\frac{e^{z}}{z}\mathrm{d}z $

\textbf{Problem.} Evaluate
\[
\int_{\lvert z\rvert = 2} \frac{e^z}{z}\,dz.
\]

\textbf{Solution:}

\textbf{1. Identify the singularity.}

The integrand
\[
f(z) = \frac{e^z}{z}
\]
has a simple pole at \(z = 0\). 

\textbf{2. Check whether the singularity is inside the contour \(\lvert z\rvert=2\).}

The contour \(\lvert z\rvert=2\) is the circle of radius \(2\) centered at the origin. Since \(\lvert 0\rvert=0 < 2\), the pole at \(z=0\) \emph{is} inside the contour.

\textbf{3. Apply Cauchy’s Integral Formula.}

Recall Cauchy’s Integral Formula: if \(g\) is analytic inside and on a simple closed contour \(\Gamma\) and \(z_0\) is a point inside \(\Gamma\), then
\[
\int_{\Gamma} \frac{g(z)}{z - z_0}\,dz \;=\; 2\pi i \,g(z_0).
\]
Here, set \(g(z) = e^z\). Then \(g\) is entire (analytic everywhere), and we have
\[
f(z) = \frac{g(z)}{z - 0} = \frac{e^z}{z}.
\]
Since \(z_0 = 0\) lies inside \(\lvert z\rvert=2\), Cauchy’s Integral Formula directly gives:
\[
\int_{\lvert z\rvert = 2} \frac{e^z}{z}\,dz 
\;=\;
2\pi i\,g(0)
\;=\;
2\pi i\,e^0
\;=\;
2\pi i.
\]

\textbf{4. Conclusion.}

Therefore,
\[
\boxed{
\int_{\lvert z\rvert = 2} \frac{e^z}{z}\,dz = 2\pi i.
}
\]
\textbf{Problem.} Evaluate
\[
\int_{\lvert z\rvert=2} \frac{z^2}{4 - z^2}\,dz.
\]

\textbf{Solution:}

\textbf{1. Locate the singularities.}

We have
\[
f(z) \;=\;\frac{z^2}{4 - z^2}.
\]
The denominator \(4 - z^2 = 0\) implies \(z^2 = 4\), so
\[
z = \pm 2.
\]
Hence \(f\) has simple poles at \(z=2\) and \(z=-2\).

\textbf{2. Observe that these poles lie on the contour \(\lvert z\rvert=2\).}

The circle \(\lvert z\rvert=2\) is centered at \(0\) with radius \(2\).  Both points \(z=2\) and \(z=-2\) lie \emph{exactly} on this contour.

\textbf{3. Why the usual Cauchy’s Theorem does not directly apply.}

Cauchy’s Theorem requires \(f(z)\) to be analytic on and \emph{inside} the contour.  Here, \(f\) fails to be analytic on the boundary itself, since \(z=\pm 2\) are singularities.  Strictly speaking, the integral
\(\displaystyle \int_{\lvert z\rvert=2} \frac{z^2}{4-z^2}\,dz\)
is not defined in the standard sense if the integrand blows up on the path.

\textbf{4. Use a limiting argument with circles of radius \(2 \pm \varepsilon\).}

One way to interpret the integral is to consider two families of contours:
\[
\Gamma_{\text{in}}(\varepsilon): \quad \lvert z\rvert = 2 - \varepsilon,
\qquad
\Gamma_{\text{out}}(\varepsilon): \quad \lvert z\rvert = 2 + \varepsilon,
\]
for small \(\varepsilon>0\).  On each of these circles, the function \(\frac{z^2}{4-z^2}\) is analytic (no poles lie strictly inside \(\Gamma_{\text{in}}(\varepsilon)\), and both poles lie inside \(\Gamma_{\text{out}}(\varepsilon)\)).

\begin{itemize}
\item
\(\displaystyle \lvert z\rvert = 2 - \varepsilon\).  
No poles are enclosed, so by Cauchy’s Theorem,
\[
\int_{\lvert z\rvert=2-\varepsilon} \frac{z^2}{4 - z^2}\,dz = 0.
\]

\item
\(\displaystyle \lvert z\rvert = 2 + \varepsilon\).  
Both poles \(z=2\) and \(z=-2\) are enclosed.  We can compute the sum of their residues to see if the integral is nonzero:

\textbf{Residue at \(\boldsymbol{z=2}\).}  
\[
\operatorname{Res}\Bigl(\frac{z^2}{4-z^2},\,z=2\Bigr)
=
\lim_{z\to 2} (z-2)\,\frac{z^2}{4-z^2}.
\]
Since \(4 - z^2 = -(z-2)(z+2)\), we get
\[
(z-2)\,\frac{z^2}{4-z^2}
= 
(z-2)\,\frac{z^2}{-(z-2)(z+2)}
= 
-\,\frac{z^2}{z+2}.
\]
Evaluating at \(z=2\):
\[
\operatorname{Res}\Bigl(\frac{z^2}{4-z^2},\,z=2\Bigr)
= 
-\,\frac{2^2}{2+2}
= 
-\frac{4}{4}
= 
-1.
\]

\textbf{Residue at \(\boldsymbol{z=-2}\).}  
Similarly,
\[
\operatorname{Res}\Bigl(\frac{z^2}{4-z^2},\,z=-2\Bigr)
= 
\lim_{z\to -2} (z+2)\,\frac{z^2}{4-z^2}
= 
-\,\frac{z^2}{z-2}\Bigg|_{z=-2}
= 
-\,\frac{(-2)^2}{-2-2}
= 
-\,\frac{4}{-4}
= 
1.
\]
Hence the sum of residues inside \(\lvert z\rvert=2+\varepsilon\) is \((-1) + (1) = 0\).  By the Residue Theorem,
\[
\int_{\lvert z\rvert=2+\varepsilon} \frac{z^2}{4-z^2}\,dz
=
2\pi i\,\bigl[\text{sum of residues inside}\bigr]
= 
2\pi i \cdot 0
= 
0.
\]
\end{itemize}

Thus, for any \(\varepsilon>0\),
\[
\int_{\lvert z\rvert=2-\varepsilon} \frac{z^2}{4-z^2}\,dz = 0
\quad\text{and}\quad
\int_{\lvert z\rvert=2+\varepsilon} \frac{z^2}{4-z^2}\,dz = 0.
\]

\textbf{5. Conclude the value on the boundary.}

Since both integrals are \(0\) for all \(\varepsilon>0\), it is natural to define
\[
\int_{\lvert z\rvert=2} \frac{z^2}{4-z^2}\,dz
\]
as the limiting value from either side.  In that sense,
\[
\lim_{\varepsilon\to 0^+} 
\int_{\lvert z\rvert=2-\varepsilon} \dots
= 0
\quad\text{and}\quad
\lim_{\varepsilon\to 0^+}
\int_{\lvert z\rvert=2+\varepsilon} \dots
= 0.
\]
Hence one consistently assigns
\[
\int_{\lvert z\rvert=2} \frac{z^2}{4 - z^2}\,dz = 0
\]
(understood as an appropriate limit or principal-value interpretation).

\[
\boxed{0}
\]

\textbf{Remark on “half residues.”}  
When a simple pole lies exactly on a contour, one sometimes encounters a “half‐residue” contribution in real‐axis integrals or in certain deformations of the contour.  In this problem, however, the two boundary poles \(z=2\) and \(z=-2\) happen to have residues \(-1\) and \(+1\) that sum to zero.  Thus even if each pole contributed “half” (in opposite signs), the net effect would be zero.  The limiting‐circle argument above is often the cleanest justification: circles just inside radius 2 enclose no poles; circles just outside radius 2 enclose both poles whose residues cancel.  Either way, the integral is \(0\).
$\int_{\left\vert z+1 \right\vert=2 }^{\infty} \frac{z^{2}}{4-z^{2}} \,\mathrm{d}x $ 

\textbf{Problem.} Evaluate
\[
\int_{\lvert z+1\rvert=2} \frac{z^2}{4 - z^2}\,dz.
\]

\textbf{Solution:}

\textbf{1. Identify the singularities of the integrand.}

The integrand is
\[
f(z) = \frac{z^2}{4 - z^2}.
\]
The denominator \(4 - z^2 = 0\) implies \(z^2 = 4\), so the singularities (simple poles) are
\[
z = 2
\quad\text{and}\quad
z = -2.
\]

\textbf{2. Determine which singularities lie inside the contour \(\lvert z+1\rvert=2\).}

Our contour is the circle of radius 2 centered at \(-1\).  A point \(z\) is inside this circle if \(\lvert z+1\rvert < 2\).  We check each singularity:

\begin{itemize}
\item For \(z=2\):
\[
\lvert 2 + 1\rvert = \lvert 3\rvert = 3 \;>\; 2,
\]
so \(z=2\) is \emph{outside} the circle.

\item For \(z=-2\):
\[
\lvert -2 + 1\rvert = \lvert -1\rvert = 1 \;<\; 2,
\]
so \(z=-2\) \emph{is} inside the circle.
\end{itemize}

Hence the only pole inside \(\lvert z+1\rvert=2\) is \(z=-2\).

\textbf{3. Compute the residue of \(f(z)\) at the pole \(z=-2\).}

We have 
\[
f(z) = \frac{z^2}{4 - z^2}.
\]
At \(z=-2\), factor the denominator:
\[
4 - z^2 = -(z^2 - 4) = -(z-2)(z+2).
\]
A standard way to compute the residue at a simple pole \(z=-2\) is:
\[
\operatorname{Res}\Bigl(f,\,z=-2\Bigr)
= \lim_{z\to -2} (z+2)\,\frac{z^2}{4 - z^2}.
\]
Notice
\[
(z+2)\,\frac{z^2}{4 - z^2}
= (z+2)\,\frac{z^2}{-(z-2)(z+2)}
= -\,\frac{z^2}{z-2}.
\]
Now evaluate at \(z=-2\):
\[
-\,\frac{(-2)^2}{-2 - 2}
= -\,\frac{4}{-4}
= 1.
\]
Hence
\[
\operatorname{Res}\Bigl(\tfrac{z^2}{4 - z^2},\,z=-2\Bigr) = 1.
\]

\textbf{4. Apply the Residue Theorem (or Cauchy’s Integral Formula).}

Since \(z=-2\) is the only pole inside the contour, the integral is
\[
\int_{\lvert z+1\rvert=2} \frac{z^2}{4 - z^2}\,dz
\;=\;
2\pi i \,\times\, \bigl[\text{sum of residues inside}\bigr]
\;=\;
2\pi i \times 1
\;=\;
2\pi i.
\]

\[
\boxed{ \int_{\lvert z+1\rvert=2} \frac{z^2}{4 - z^2}\,dz = 2\pi i. }
\]

\textbf{Summary:}
- Poles are at \(z=\pm2\).
- The circle \(\lvert z+1\rvert=2\) encloses \(z=-2\) but not \(z=2\).
- The residue at \(z=-2\) is \(1\).
- By the Residue Theorem, the integral is \(2\pi i\cdot1 = 2\pi i\).


\textbf{Problem.} Evaluate
\[
\int_{\lvert z+1\rvert=2} \frac{z^2}{4-z^2}\,dz.
\]

\textbf{Solution using Cauchy's Integral Formula:}

\textbf{Step 1: Identify the singularities and the contour.} 

The integrand is
\[
f(z)=\frac{z^2}{4-z^2}.
\]
We factor the denominator as
\[
4-z^2 = -(z-2)(z+2).
\]
Thus,
\[
f(z) = \frac{z^2}{4-z^2} = -\frac{z^2}{(z-2)(z+2)}.
\]
The singularities occur where \(z^2=4\); that is, at \(z=2\) and \(z=-2\).  
Our contour is the circle
\[
\lvert z+1\rvert=2,
\]
which is centered at \(-1\) with radius \(2\). It can be verified that
\[
\lvert -2+1\rvert = \lvert -1\rvert = 1 < 2 \quad \text{(so \(z=-2\) is inside)},
\]
and
\[
\lvert 2+1\rvert = 3 > 2 \quad \text{(so \(z=2\) is outside)}.
\]
Thus, the only singularity enclosed is \(z=-2\).

\textbf{Step 2: Rewrite the integrand in the form required by Cauchy's Integral Formula.}

We want to write \(f(z)\) in the form
\[
\frac{g(z)}{z-(-2)} = \frac{g(z)}{z+2},
\]
where \(g(z)\) is analytic at \(z=-2\).

Starting with
\[
f(z) = -\frac{z^2}{(z-2)(z+2)},
\]
we factor out the \(z+2\) term:
\[
f(z) = \frac{1}{z+2}\left[-\frac{z^2}{z-2}\right].
\]
Thus, define
\[
g(z) = -\frac{z^2}{z-2}.
\]
Note that \(g(z)\) is analytic at \(z=-2\) since the only potential singularity of \(g(z)\) is at \(z=2\), which is not equal to \(-2\).

\textbf{Step 3: Evaluate \(g(z)\) at the singularity \(z=-2\).}

We compute:
\[
g(-2) = -\frac{(-2)^2}{-2-2} 
=\; -\frac{4}{-4} 
=\; 1.
\]

\textbf{Step 4: Apply Cauchy's Integral Formula.}

Cauchy's Integral Formula states that if \(g\) is analytic inside and on a closed contour \(\Gamma\) and \(z_0\) is a point inside \(\Gamma\), then
\[
\int_{\Gamma} \frac{g(z)}{z-z_0}\,dz = 2\pi i\, g(z_0).
\]
Here, \(z_0=-2\) and \(g(-2)=1\). Therefore,
\[
\int_{\lvert z+1\rvert=2} \frac{z^2}{4-z^2}\,dz 
=\; \int_{\lvert z+1\rvert=2} \frac{g(z)}{z+2}\,dz 
=\; 2\pi i \,g(-2)
=\; 2\pi i.
\]

\[
\boxed{2\pi i}
\]

\textbf{Problem:} Evaluate
\[
\int_{\lvert z\rvert=1} \frac{\sin z}{z}\,dz.
\]

\textbf{Solution:}

\textbf{1. Check for singularities of the integrand \(\frac{\sin z}{z}\).}

The only potential issue is at \(z=0\), because of the factor \(\frac{1}{z}\). However, recall that
\[
\sin z = z - \frac{z^3}{3!} + \frac{z^5}{5!} - \cdots
\]
Thus
\[
\frac{\sin z}{z}
= 1 - \frac{z^2}{3!} + \frac{z^4}{5!} - \cdots
\]
This power‐series expansion is perfectly well‐defined at \(z=0\) (the limit is \(1\) there). Hence \(z=0\) is a \emph{removable} singularity. We can define 
\(\frac{\sin z}{z}\bigl|_{z=0} = 1\)
to make it analytic everywhere inside \(\lvert z\rvert=1\).

\textbf{2. Apply Cauchy’s Theorem (or the Residue Theorem).}

Since \(\frac{\sin z}{z}\) is analytic on and inside the circle \(\lvert z\rvert=1\) (after handling the removable singularity at \(z=0\)), it has no poles in that region. By Cauchy’s Theorem (or the Residue Theorem with zero residues inside),
\[
\int_{\lvert z\rvert=1} \frac{\sin z}{z}\,dz 
\;=\; 0.
\]

\textbf{3. Alternative approach: Power‐series integration.}

We can also expand \(\frac{\sin z}{z}\) in a Maclaurin series and integrate term by term:
\[
\frac{\sin z}{z}
= \sum_{n=0}^{\infty} (-1)^n \,\frac{z^{2n}}{(2n+1)!}.
\]
Then
\[
\int_{\lvert z\rvert=1} \frac{\sin z}{z}\,dz
= \int_{\lvert z\rvert=1} 
\sum_{n=0}^{\infty} (-1)^n \,\frac{z^{2n}}{(2n+1)!}\,dz.
\]
Term‐by‐term, each integral
\(\displaystyle \int_{\lvert z\rvert=1} z^{2n}\,dz\)
is zero for \(2n \neq -1\). Since \(2n\) is never \(-1\), each term vanishes. Hence the entire sum integrates to \(0\).

\textbf{Conclusion:}
\[
\boxed{
\int_{\lvert z\rvert=1} \frac{\sin z}{z}\,dz = 0.
}
\]

\textbf{Question:} 
Why does the identity
\[
\int_{0}^{\infty}\cos(t^2)\,dt \;-\; i \int_{0}^{\infty}\sin(t^2)\,dt
\;=\;
\frac{1}{2}\sqrt{\frac{\pi}{2}}\;e^{-\,i\pi/4}
\]
imply
\[
\int_{0}^{\infty}\cos(t^2)\,dt
\;=\;
\int_{0}^{\infty}\sin(t^2)\,dt
\;=\;
\sqrt{\frac{\pi}{8}}\,?
\]

\textbf{Short Answer:} 

Because 
\[
\cos(t^2) \;-\; i\,\sin(t^2) \;=\; e^{-\,i\,t^2},
\]
the left‐hand side is the real part minus \(i\) times the imaginary part of \(e^{-i\,t^2}\).  Evaluating 
\(\displaystyle \int_0^\infty e^{-\,i\,t^2}\,dt\)
via contour integration yields
\(\displaystyle \frac12 \sqrt{\frac{\pi}{2}}\,e^{-\,i\pi/4}.\)
Comparing real and imaginary parts on both sides shows that
\(\displaystyle \int_{0}^{\infty}\cos(t^2)\,dt = \int_{0}^{\infty}\sin(t^2)\,dt = \sqrt{\frac{\pi}{8}}.\)

\bigskip

\textbf{Detailed Explanation:}

\begin{enumerate}
\item \textbf{Rewrite the integrals in terms of exponentials.}

   Note that
   \[
   \cos(t^2) \;-\; i\,\sin(t^2)
   \;=\;
   e^{-\,i\,t^2}.
   \]
   Therefore,
   \[
   \int_{0}^{\infty}\cos(t^2)\,dt
   \;-\;
   i \int_{0}^{\infty}\sin(t^2)\,dt
   \;=\;
   \int_{0}^{\infty} e^{-\,i\,t^2}\,dt.
   \]

\item \textbf{Known result from contour integration.}

   By using a standard contour‐integration argument (similar to the one used for 
   \(\displaystyle \int_{0}^{\infty} e^{\,i\,t^2}\,dt\), 
   but with a slightly different orientation), one shows:
   \[
   \int_{0}^{\infty} e^{-\,i\,t^2}\,dt
   \;=\;
   \frac12 \,\sqrt{\frac{\pi}{2}}\;e^{-\,i\,\pi/4}.
   \]
   (The factor \(e^{-i\pi/4}\) is \(\cos(-\pi/4) + i\,\sin(-\pi/4) = \frac{1}{\sqrt{2}} - i\,\frac{1}{\sqrt{2}}\).)

\item \textbf{Match real and imaginary parts.}

   The left side of
   \[
   \int_{0}^{\infty}\cos(t^2)\,dt \;-\; i \int_{0}^{\infty}\sin(t^2)\,dt
   \;=\;
   \frac12 \,\sqrt{\frac{\pi}{2}}\;e^{-\,i\,\pi/4}
   \]
   can be viewed as:
   \[
   \bigl(\text{Real Part}\bigr)
   \;-\;
   i\;\bigl(\text{Imag Part}\bigr).
   \]
   Meanwhile, on the right side,
   \[
   \frac12 \,\sqrt{\frac{\pi}{2}}\;e^{-\,i\,\pi/4}
   \;=\;
   \frac12 \,\sqrt{\frac{\pi}{2}}\;
   \Bigl(\cos(-\pi/4) + i\,\sin(-\pi/4)\Bigr)
   =
   \frac12 \,\sqrt{\frac{\pi}{2}}\;
   \Bigl(\tfrac{1}{\sqrt{2}} \;-\; i\,\tfrac{1}{\sqrt{2}}\Bigr).
   \]
   Hence the real part on the right is 
   \(\displaystyle \frac12 \sqrt{\tfrac{\pi}{2}} \cdot \tfrac{1}{\sqrt{2}} = \sqrt{\tfrac{\pi}{8}},\)
   and the imaginary part on the right is
   \(-\,\sqrt{\tfrac{\pi}{8}}.\)

\item \textbf{Equate real parts and imaginary parts.}

   - \(\displaystyle \text{Real part:}\)
     \(\displaystyle \int_{0}^{\infty}\cos(t^2)\,dt \;=\;\sqrt{\tfrac{\pi}{8}}.\)
   - \(\displaystyle \text{Imag part:}\)
     \(-\,\int_{0}^{\infty}\sin(t^2)\,dt = -\,\sqrt{\tfrac{\pi}{8}}.\)
     Hence
     \(\displaystyle \int_{0}^{\infty}\sin(t^2)\,dt = \sqrt{\tfrac{\pi}{8}}.\)

\item \textbf{Conclusion.}

   We see that
   \[
   \int_{0}^{\infty}\cos(t^2)\,dt 
   \;=\;
   \int_{0}^{\infty}\sin(t^2)\,dt 
   \;=\;
   \sqrt{\frac{\pi}{8}}.
   \]
   That is exactly the desired Fresnel‐type result.

\end{enumerate}

\[
\textbf{Problem: Evaluate } \int_{\lvert z+i\rvert=2}\,\frac{dz}{\bigl(z^2+4\bigr)^3}.
\]

\textbf{Step 1: Identify the singularities of the integrand.}

The integrand is
\[
f(z) \;=\; \frac{1}{(z^2 + 4)^3}.
\]
The singularities occur where the denominator \((z^2+4)^3 = 0\), i.e.\ when \(z^2+4=0\).  This gives
\[
z^2 = -4 
\quad\Longrightarrow\quad
z = \pm\,2i.
\]

\textbf{Step 2: Determine which singularities lie inside the contour.}

The contour is the circle
\[
\lvert z + i\rvert \;=\; 2.
\]
We check the distance from \(-i\) to the points \(\pm\,2i\):
\[
\lvert -2i + i\rvert \;=\;\lvert -i\rvert \;=\;1,
\quad
\lvert 2i + i\rvert \;=\;\lvert 3i\rvert \;=\;3.
\]
Thus, \(z=-2i\) (distance 1 from \(-i\)) is \emph{inside} the circle, while \(z=2i\) (distance 3) is \emph{outside}.

\textbf{Step 3: Classify the pole and set up the residue calculation.}

At \(z=-2i\), the factor \(z^2+4\) vanishes to first order, but since the integrand has \((z^2+4)^3\) in the denominator, the pole is of order \(3\).  Writing
\[
(z^2+4)
\;=\;
(z-2i)\,(z+2i),
\]
we have
\[
(z^2+4)^3 
\;=\; (z-2i)^3 \,(z+2i)^3,
\]
so
\[
f(z) 
\;=\;
\frac{1}{(z-2i)^3 (z+2i)^3}.
\]

\textbf{Step 4: Use the residue theorem for a pole of order 3.}

For an \(n\)th-order pole at \(z=z_0\), the residue of \(f\) at \(z_0\) is given by
\[
\operatorname{Res}(f, z_0)
\;=\;
\frac{1}{(n-1)!}
\lim_{z \to z_0}
\frac{d^{\,n-1}}{dz^{\,n-1}}
\Bigl[(z-z_0)^n \, f(z)\Bigr].
\]
Here, \(n=3\) and \(z_0=-2i\).  We compute:
\[
(z+2i)^3 \, f(z)
\;=\;
\frac{(z+2i)^3}{(z-2i)^3 (z+2i)^3}
\;=\;
\frac{1}{(z-2i)^3}.
\]
Hence
\[
\operatorname{Res}\bigl(f,-2i\bigr)
\;=\;
\frac{1}{2!}
\lim_{z \to -2i}
\frac{d^2}{dz^2}
\Bigl[\,(z+2i)^3 f(z)\Bigr]
\;=\;
\frac{1}{2}
\lim_{z \to -2i}
\frac{d^2}{dz^2}
\Bigl[\,(z-2i)^{-3}\Bigr].
\]

\textbf{Step 5: Differentiate and evaluate the limit.}

First derivative:
\[
\frac{d}{dz}\,\bigl((z-2i)^{-3}\bigr)
\;=\;
-3\,(z-2i)^{-4}.
\]
Second derivative:
\[
\frac{d^2}{dz^2}\,\bigl((z-2i)^{-3}\bigr)
\;=\;
-3\,\frac{d}{dz}\,\bigl((z-2i)^{-4}\bigr)
\;=\;
-3 \times (-4)\,(z-2i)^{-5}
\;=\;
12\,(z-2i)^{-5}.
\]
Evaluate at \(z=-2i\):
\[
(z-2i)\Bigl|_{z=-2i}
\;=\;
-2i - 2i
\;=\;
-4i,
\]
so
\[
12\,(-4i)^{-5}.
\]
Since \((-4i)^5 = (-4i)^4 \cdot (-4i) = \bigl(256 i^4\bigr)\,(-4i) = 256 \cdot 1 \cdot (-4i) = -1024\,i,\)
we get
\[
12\,(-4i)^{-5}
\;=\;
12 \,\frac{1}{-1024\,i}
\;=\;
-\frac{12}{1024}\,\frac{1}{i}
\;=\;
-\frac{12}{1024}\,(-i)
\;=\;
\frac{12\,i}{1024}
\;=\;
\frac{3\,i}{256}.
\]
Thus
\[
\operatorname{Res}\bigl(f,-2i\bigr)
\;=\;
\frac{1}{2} \times \frac{3\,i}{256}
\;=\;
\frac{3\,i}{512}.
\]

\textbf{Step 6: Apply the residue theorem.}

The contour integral around \(\lvert z+i\rvert=2\) is
\[
\int_{\lvert z+i\rvert=2} 
\frac{dz}{(z^2+4)^3}
\;=\;
2\pi i \times \Bigl[\text{sum of residues inside}\Bigr]
\;=\;
2\pi i \times \operatorname{Res}\bigl(f,-2i\bigr)
\;=\;
2\pi i \times \frac{3\,i}{512}.
\]
Since \(i \times i = i^2 = -1\),
\[
2\pi i \;\cdot\; \frac{3\,i}{512}
\;=\;
\frac{6\,\pi\, (i^2)}{512}
\;=\;
-\,\frac{6\,\pi}{512}
\;=\;
-\,\frac{3\,\pi}{256}.
\]

\[
\boxed{
\int_{\lvert z+i\rvert=2} \frac{dz}{(z^2+4)^3}
\;=\;
-\,\frac{3\,\pi}{256}.
}
\]```

\[
\textbf{Problem: Evaluate } 
\int_{\lvert z + i\rvert=2}\,\frac{dz}{(z^2 + 4)^3}.
\]

\textbf{Step 1: Rewrite the integrand in a suitable form for Cauchy's formula.}

Observe that
\[
\frac{1}{(z^2 + 4)^3}
\;=\;
\frac{1}{\bigl(z - 2i\bigr)^3 \,\bigl(z + 2i\bigr)^3}.
\]
We focus on the pole at \(z = -2i\), which lies inside the circle \(\lvert z + i\rvert = 2\).  
To use Cauchy's formula around \(z_0 = -2i\), note that
\[
z + 2i \;=\; z - (-2i),
\]
so we want the denominator to look like \(\bigl(z - (-2i)\bigr)^3\).  Hence, write
\[
\frac{1}{(z^2 + 4)^3}
\;=\;
\underbrace{\frac{1}{(z - 2i)^3}}_{=:f(z)} 
\;\times\;
\frac{1}{(z + 2i)^3}.
\]
Define
\[
f(z) \;=\; \frac{1}{(z - 2i)^3}.
\]
Then the integrand becomes
\[
\frac{f(z)}{(z + 2i)^3},
\]
which is precisely of the form needed for Cauchy's Integral Formula for the second derivative (because the power is 3, corresponding to an \((n+1)\)th power with \(n=2\)).

\textbf{Step 2: State the generalized Cauchy Integral Formula (for the \(n\)th derivative).}

If \(g\) is analytic in and on a simple closed contour \(C\), then for any \(z_0\) inside \(C\),  
\[
\frac{1}{2\pi i}\,\int_C 
\frac{g(z)}{\bigl(z - z_0\bigr)^{n+1}}
\,dz
\;=\;
\frac{1}{n!}\,g^{(n)}(z_0).
\]
In our situation:
\[
g(z) = f(z) = \frac{1}{(z - 2i)^3},
\quad
z_0 = -2i,
\quad
n = 2
\ (\text{because we have a cube in the denominator}).
\]
Hence, the integral
\[
\int_C \frac{f(z)}{\bigl(z + 2i\bigr)^3}\,dz
\;=\;
\int_C \frac{f(z)}{\bigl(z - (-2i)\bigr)^3}\,dz
\]
will be \(2\pi i\) times the corresponding expression involving the second derivative of \(f\) at \(z=-2i\).

\textbf{Step 3: Compute the required derivative of \(f(z)\).}

We have
\[
f(z) \;=\; (z - 2i)^{-3}.
\]
Differentiate:
\[
f'(z) 
\;=\;
-3\,(z - 2i)^{-4},
\]
\[
f''(z)
\;=\;
-3\,\frac{d}{dz}\bigl[(z - 2i)^{-4}\bigr]
\;=\;
-3\bigl(-4\,(z - 2i)^{-5}\bigr)
\;=\;
12\,(z - 2i)^{-5}.
\]
Next, evaluate at \(z = -2i\):
\[
f''(-2i)
\;=\;
12\bigl((-2i) - 2i\bigr)^{-5}
\;=\;
12\,(-4i)^{-5}.
\]
Since
\[
(-4i)^5 
\;=\; 
(-4i)^4 \times (-4i) 
\;=\;
\bigl(256\,i^4\bigr)\,(-4i)
\;=\;
256\times 1 \times (-4i)
\;=\;
-\,1024\,i,
\]
we get
\[
(-4i)^{-5}
\;=\;
\frac{1}{-\,1024\,i}
\;=\;
-\,\frac{1}{1024}\,\frac{1}{i}
\;=\;
-\,\frac{1}{1024}\,\bigl(-i\bigr)
\;=\;
\frac{i}{1024}.
\]
Hence
\[
f''(-2i)
\;=\;
12 \cdot \frac{i}{1024}
\;=\;
\frac{12\,i}{1024}
\;=\;
\frac{3\,i}{256}.
\]

\textbf{Step 4: Apply the generalized Cauchy Integral Formula.}

Since \(n=2\),
\[
\int_C \frac{f(z)}{(z + 2i)^3}\,dz
\;=\;
2\pi i \times \frac{f''(-2i)}{2!}
\;=\;
2\pi i \times \frac{\tfrac{3\,i}{256}}{2}
\;=\;
2\pi i \times \frac{3\,i}{512}
\;=\;
\frac{6\pi\,i^2}{512}
\;=\;
-\frac{6\pi}{512}
\;=\;
-\frac{3\pi}{256}.
\]

\textbf{Step 5: Conclude.}

Therefore,
\[
\boxed{
\int_{\lvert z + i\rvert=2}\,\frac{dz}{(z^2 + 4)^3}
\;=\;
-\;\frac{3\pi}{256}.
}
\]
$\frac{\pi(1-(\ln(2)+\frac{\pi}{2}))}{32}$ 

\[
\textbf{Problem: Evaluate } \int_{\lvert z + 3i\rvert=2}
   \frac{\log(z)\,dz}{\bigl(z^{2}+4\bigr)^{2}}.
\]

\textbf{Step 1: Identify the singularities and which are inside the contour.}

Factor
\[
z^{2} + 4 \;=\;(z - 2i)\,(z + 2i).
\]
Thus
\[
\frac{\log(z)}{(z^{2}+4)^{2}}
\;=\;
\frac{\log(z)}{\bigl(z - 2i\bigr)^{2}\,\bigl(z + 2i\bigr)^{2}}.
\]
The poles are at \(z = \pm 2i\).  Our contour is the circle
\[
\lvert z + 3i\rvert \;=\; 2,
\]
centered at \(-3i\) with radius \(2\).  Note that
\[
\lvert -2i + 3i\rvert 
\;=\;\lvert i\rvert 
\;=\;1 
\quad(< 2),
\qquad
\lvert 2i + 3i\rvert 
\;=\;\lvert 5i\rvert 
\;=\;5 
\quad(> 2).
\]
Hence \(z = -2i\) is \emph{inside} the circle and \(z = 2i\) is \emph{outside}.

\textbf{Step 2: Express the integrand in the form for the ``second‐order'' Cauchy formula.}

Since the only relevant pole is at \(z_0 = -2i\), notice
\[
(z+2i) \;=\;(z - (-2i)).
\]
Hence
\[
\frac{\log(z)}{(z^{2}+4)^{2}}
\;=\;
\frac{\log(z)}{\bigl(z - 2i\bigr)^{2}}\,\cdot\,\frac{1}{\bigl(z - (-2i)\bigr)^{2}}.
\]
Define
\[
g(z)
\;=\;
\frac{\log(z)}{\bigl(z - 2i\bigr)^{2}},
\]
which is analytic at \(z=-2i\) (the factor \((z-2i)\) does \emph{not} vanish at \(z=-2i\)).  Then the integrand is
\[
\frac{g(z)}{\bigl(z - (-2i)\bigr)^{2}}.
\]
By the Cauchy Integral Formula for derivatives (in the version for a second‐order pole),
\[
\int_{C}
  \frac{g(z)}{\bigl(z - z_0\bigr)^{2}}\;dz
\;=\;
2\pi i \, g'(z_0)
\quad
(\text{where }z_0 \text{ is inside }C).
\]
Here \(z_0 = -2i\).

\textbf{Step 3: Differentiate } \(g(z)\) \textbf{ and evaluate at } \(-2i\).

We have
\[
g(z)
\;=\;
\log(z)\,\cdot\,\bigl(z - 2i\bigr)^{-2}.
\]
Use the product rule:
\[
g'(z)
\;=\;
\bigl(z - 2i\bigr)^{-2}\cdot \frac{d}{dz}\bigl[\log(z)\bigr]
\;+\;
\log(z)\,\cdot
\frac{d}{dz}\bigl[(z - 2i)^{-2}\bigr].
\]
Since \(\frac{d}{dz}\log(z)=\frac{1}{z}\) and
\(\frac{d}{dz}(z-2i)^{-2}=-2(z-2i)^{-3}\),
\[
g'(z)
\;=\;
\frac{1}{z\,(z - 2i)^{2}}
\;-\;
2\,\log(z)\,(z - 2i)^{-3}.
\]
Evaluate at \(z=-2i\).  Note that \((-2i)-2i=-4i\).  Then
\[
g'(-2i)
\;=\;
\frac{1}{\,(-2i)\,(-4i)^{2}\!}
\;-\;
2\,\log(-2i)\,\frac{1}{(-4i)^{3}}.
\]
Since \((-4i)^{2}=-16\) and \((-4i)^{3}=64\,i,\)
\[
\frac{1}{(-2i)\,(-4i)^{2}}
\;=\;
\frac{1}{(-2i)\,(-16)}
\;=\;
\frac{1}{32\,i}
\;=\;
-\;\frac{i}{32},
\]
and
\[
-\,2\,\log(-2i)\,\frac{1}{(-4i)^{3}}
\;=\;
-\,2\,\log(-2i)\,\frac{1}{64\,i}
\;=\;
-\,\frac{1}{32\,i}\,\log(-2i)
\;=\;
\frac{i}{32}\,\log(-2i),
\]
where we used \(1/i = -\,i\).  Combining,
\[
g'(-2i)
\;=\;
-\;\frac{i}{32}
\;+\;
\frac{i}{32}\,\log(-2i)
\;=\;
\frac{i}{32}\,\bigl[\log(-2i)-1\bigr].
\]

\textbf{Step 4: Apply the ``second‐order'' Cauchy Integral Formula.}

Hence
\[
\int_{\lvert z+3i\rvert=2}
\frac{\log(z)\,dz}{\bigl(z^{2}+4\bigr)^{2}}
\;=\;
2\pi i \;\times\; g'(-2i)
\;=\;
2\pi i\;\times\;
\frac{i}{32}\,\bigl[\log(-2i)-1\bigr].
\]
Since \(i\times i = i^{2}=-1,\)
\[
2\pi i \cdot \frac{i}{32}
\;=\;
\frac{2\pi\,i^{2}}{32}
\;=\;
-\,\frac{\pi}{16}.
\]
Therefore,
\[
\int_{\lvert z+3i\rvert=2}
\frac{\log(z)\,dz}{\bigl(z^{2}+4\bigr)^{2}}
\;=\;
-\,\frac{\pi}{16}\,\bigl[\log(-2i)-1\bigr]
\;=\;
\frac{\pi}{16}\,\bigl[\,1 - \log(-2i)\bigr].
\]

\textbf{Step 5 (Optional): Express \(\log(-2i)\) in terms of real/imag parts.}

On the principal branch (argument in \((-\pi,\pi]\)), 
\(\log(-2i)\;=\;\log(2)\,-\,\tfrac{\pi}{2}\,i.\)
Hence
\[
1 - \log(-2i) 
\;=\;
1\;-\;\log(2)\;+\;\tfrac{\pi}{2}\,i,
\]
and
\[
\frac{\pi}{16}\,\bigl[\,1 - \log(-2i)\bigr]
\;=\;
\frac{\pi}{16}\,\bigl[\,1 - \log(2)\bigr]
\;+\;
i\,\frac{\pi^{2}}{32}.
\]
Thus a perfectly nice final form is
\[
\boxed{
\int_{\lvert z+3i\rvert=2}
\frac{\log(z)\,dz}{(z^{2}+4)^{2}}
\;=\;
\frac{\pi}{16}\,\bigl[\,1 - \log(2)\bigr]
\;+\;
i\,\frac{\pi^{2}}{32}.
}
\]

$\frac{1}{32i}+\frac{-2(\ln(2)=i(\frac{\pi}{2}))}{64i}$ 
\end{document}
